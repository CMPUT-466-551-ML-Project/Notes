\documentclass[12pt, dvipsnames]{report}

\usepackage{amsmath}
\usepackage{algorithm}
%\usepackage{algorithmic}
\usepackage[noend]{algpseudocode}

\usepackage{amsmath}
\usepackage{amssymb}
\usepackage{amsthm}
\usepackage{amsopn}

\usepackage{kpfonts}

\usepackage{graphicx}

% Probably don't need this on notes anymore
%\usepackage{kbordermatrix}

% Standard tool for drawing diagrams.
\usepackage{tikz}
\usepackage{tkz-berge}
\usepackage{tikz-cd}
\usepackage{tkz-graph}

\usepackage{comment}

%
\usepackage{multicol}

%
\usepackage{framed}

%
\usepackage{mathtools}

%
\usepackage{float}

%
\usepackage{subfig}

%
\usepackage{wrapfig}

%
\let\savewideparen\wideparen
\let\wideparen\relax
\usepackage{mathabx}
\let\wideparen\savewideparen

% Used for generating `enlightening quotes'
\usepackage{epigraph}

% Forget what this is used for :P
\usepackage[utf8]{inputenc}

% Used for generating quotes.
\usepackage{csquotes}

% Allows what to generate links inside
% generated pdf files
\usepackage{hyperref}

% Allows one to customize theorem
% environments in mathematical proofs.
\usepackage{thmtools}

% Gives access to a proof
\usepackage{lplfitch}

% I forget what this is for.
\usepackage{accents}

% A package for drawing simple trees,
% as a substitute for unnesacary TIKZ code
\usepackage{qtree}

% Enables sequent calculus proofs
\usepackage{ebproof}

% For braket notation
\usepackage{braket}

% To change line spacing when using mathematical notations which require some height!
\usepackage{setspace}

%\usepackage[dvipsnames]{xcolor}

\usepackage{float}

% For block commenting
\usepackage{comment}




\setlength\epigraphwidth{8cm}

\usetikzlibrary{arrows, petri, topaths, decorations.markings}

% So you can do calculations in coordinate specifications
\usetikzlibrary{calc}
\usetikzlibrary{angles}

\theoremstyle{plain}
\newtheorem{theorem}{Theorem}[chapter]
\newtheorem{axiom}{Axiom}
\newtheorem{lemma}[theorem]{Lemma}
\newtheorem{corollary}[theorem]{Corollary}
\newtheorem{prop}[theorem]{Proposition}
\newtheorem{exercise}{Exercise}[chapter]
\newtheorem{fact}{Fact}[chapter]

\newtheorem*{example}{Example}
\newtheorem*{proof*}{Proof}

\theoremstyle{remark}
\newtheorem*{exposition}{Exposition}
\newtheorem*{remark}{Remark}
\newtheorem*{remarks}{Remarks}

\theoremstyle{definition}
\newtheorem*{defi}{Definition}

\usepackage{hyperref}
\hypersetup{
    colorlinks = true,
    linkcolor = black,
}

\usepackage{textgreek}

\makeatletter
\renewcommand*\env@matrix[1][*\c@MaxMatrixCols c]{%
  \hskip -\arraycolsep
  \let\@ifnextchar\new@ifnextchar
  \array{#1}}
\makeatother

\renewcommand*\contentsname{\hfill Table Of Contents \hfill}

\newcommand{\optionalsection}[1]{\section[* #1]{(Important) #1}}
\newcommand{\deriv}[3]{\left. \frac{\partial #1}{\partial #2} \right|_{#3}} % partial derivative involving numerator and denominator.
\newcommand{\lcm}{\operatorname{lcm}}
\newcommand{\im}{\operatorname{im}}
\newcommand{\bint}{\mathbf{Z}}
\newcommand{\gen}[1]{\langle #1 \rangle}

\newcommand{\End}{\operatorname{End}}
\newcommand{\Mor}{\operatorname{Mor}}
\newcommand{\Id}{\operatorname{id}}
\newcommand{\visspace}{\text{\textvisiblespace}}
\newcommand{\Gal}{\text{Gal}}

\newcommand{\xor}{\oplus}
\newcommand{\ft}{\wedge}
\newcommand{\ift}{\vee}

\newcommand{\prob}{\mathbf{P}}
\newcommand{\expect}{\mathbf{E}}
\DeclareMathOperator{\Var}{\mathbf{V}}
\newcommand{\Ber}{\text{Ber}}
\newcommand{\Bin}{\text{Bin}}

%\newcommand{\widecheck}[1]{{#1}^{\ft}}

\DeclareMathOperator{\diam}{\text{diam}}

\DeclareMathOperator{\QQ}{\mathbf{Q}}
\DeclareMathOperator{\ZZ}{\mathbf{Z}}
\DeclareMathOperator{\RR}{\mathbf{R}}
\DeclareMathOperator{\HH}{\mathbf{H}}
\DeclareMathOperator{\CC}{\mathbf{C}}
\DeclareMathOperator{\AB}{\mathbf{A}}
\DeclareMathOperator{\PP}{\mathbf{P}}
\DeclareMathOperator{\MM}{\mathbf{M}}
\DeclareMathOperator{\VV}{\mathbf{V}}
\DeclareMathOperator{\TT}{\mathbf{T}}
\DeclareMathOperator{\LL}{\mathcal{L}}
\DeclareMathOperator{\EE}{\mathbf{E}}
\DeclareMathOperator{\NN}{\mathbf{N}}
\DeclareMathOperator{\DQ}{\mathcal{Q}}
\DeclareMathOperator{\IA}{\mathfrak{a}}
\DeclareMathOperator{\IB}{\mathfrak{b}}
\DeclareMathOperator{\IC}{\mathfrak{c}}
\DeclareMathOperator{\IP}{\mathfrak{p}}
\DeclareMathOperator{\IQ}{\mathfrak{q}}
\DeclareMathOperator{\IM}{\mathfrak{m}}
\DeclareMathOperator{\IN}{\mathfrak{n}}
\DeclareMathOperator{\IK}{\mathfrak{k}}
\DeclareMathOperator{\ord}{\text{ord}}
\DeclareMathOperator{\Ker}{\textsf{Ker}}
\DeclareMathOperator{\Coker}{\textsf{Coker}}
\DeclareMathOperator{\emphcoker}{\emph{coker}}
\DeclareMathOperator{\pp}{\partial}
\DeclareMathOperator{\tr}{\text{tr}}

\DeclareMathOperator{\supp}{\text{supp}}

\DeclareMathOperator{\codim}{\text{codim}}

\DeclareMathOperator{\minkdim}{\dim_{\mathbf{M}}}
\DeclareMathOperator{\hausdim}{\dim_{\mathbf{H}}}
\DeclareMathOperator{\lowminkdim}{\underline{\dim}_{\mathbf{M}}}
\DeclareMathOperator{\upminkdim}{\overline{\dim}_{\mathbf{M}}}
\DeclareMathOperator{\lhdim}{\underline{\dim}_{\mathbf{M}}}
\DeclareMathOperator{\lmbdim}{\underline{\dim}_{\mathbf{MB}}}
\DeclareMathOperator{\packdim}{\text{dim}_{\mathbf{P}}}
\DeclareMathOperator{\fordim}{\dim_{\mathbf{F}}}

\DeclareMathOperator*{\argmax}{arg\,max}
\DeclareMathOperator*{\argmin}{arg\,min}

\DeclareMathOperator{\ssm}{\smallsetminus}

\DeclareMathOperator{\Dom}{Dom}
\DeclareMathOperator{\delin}{\delta_{\text{in}}}
\DeclareMathOperator{\delout}{\delta_{\text{out}}}

\def\@fnsymbol#1{\ensuremath{\ifcase#1\or *\or \dagger\or \ddagger\or
   \mathsection\or \mathparagraph\or \|\or **\or \dagger\dagger
   \or \ddagger\ddagger \else\@ctrerr\fi}}
\renewcommand{\thefootnote}{\fnsymbol{footnote}}

\title{Combinatorial Optimization}
\author{Jacob Denson}

\begin{document}

\pagenumbering{gobble}

\maketitle

\tableofcontents

\pagenumbering{arabic}

\chapter{Matchings and Flows}

\section{Bipartite Matching}

Let $G = (V,E)$ be a graph. A {\bf matching} is $M \subset E$ such that no vertex in $V$ is the endpoint of more than one edge in $M$. $M$ is maximal if $|M| \geq |M'|$ for any other matching $M'$. The maximal matching problem asks us to find a fast algorithm to find a maximum matching in any graph.

There is a polynomial time algorithm which can find matchings on any graph, but the problem is much more simple if $G$ is bipartite -- that is, if we may partition $V$ into the disjoint union $W \amalg U$ of two sets of vertices, such that every edge in $E$ contains a point in $W$ and a point in $U$. The bipartite matching asks us to find a maximal matching in a bipartite graph.

\begin{figure}
\begin{center}
\begin{tikzpicture}[scale=0.5,transform shape]
    \GraphInit[vstyle=Classic]

    \Vertex[NoLabel,x=1,y=0]{A1}
    \Vertex[NoLabel,x=1,y=-1.5]{B1}
    \Vertex[NoLabel,x=1,y=-3]{C1}
    \Vertex[NoLabel,x=1,y=-4.5]{D1}
    \Vertex[NoLabel,x=1,y=-6]{E1}

    \Vertex[NoLabel,x=6,y=-0.75]{A2}
    \Vertex[NoLabel,x=6,y=-2.25]{B2}
    \Vertex[NoLabel,x=6,y=-3.75]{C2}
    \Vertex[NoLabel,x=6,y=-5.25]{D2}

    \Edge[style={dashed}](A1)(A2)
    \Edge(B1)(A2)
    \Edge[style={dashed}](B1)(D2)
    \Edge(C1)(B2)
    \Edge[style={dashed}](D1)(C2)
    \Edge(E1)(C2)

    \Vertex[x=9,y=-3]{S}
    \Vertex[x=18,y=-3]{T}

    \Vertex[NoLabel,x=11,y=0]{A1'}
    \Vertex[NoLabel,x=11,y=-1.5]{B1'}
    \Vertex[NoLabel,x=11,y=-3]{C1'}
    \Vertex[NoLabel,x=11,y=-4.5]{D1'}
    \Vertex[NoLabel,x=11,y=-6]{E1'}

    \Vertex[NoLabel,x=16,y=-0.75]{A2'}
    \Vertex[NoLabel,x=16,y=-2.25]{B2'}
    \Vertex[NoLabel,x=16,y=-3.75]{C2'}
    \Vertex[NoLabel,x=16,y=-5.25]{D2'}

    \draw[->, thick](S)--(A1');
    \draw[->, dashed, thick](S)--(D1');
    \draw[->, thick](D2')--(T);
    \draw[->, thick](A1')--(A2');
    \draw[->, thick](B1')--(D2');
    \draw[->, dashed, thick](D1')--(C2');
    \draw[->, thick](A2')--(B1');
    \draw[->, dashed, thick](B2')--(C1');
    \draw[->, dashed, thick](C2')--(E1');
\end{tikzpicture}
\end{center}
\caption{The left graph shows a matching in a graph of cardinality 3. The solid lines correspond to the edges chosen. On the right is the augmenting path graph corresponding to the matching. It contains a path from $s$ to $t$, hence there is a way to improve the matching to a match of cardinality 4.}
\end{figure}

Suppose we are given a particular matching in a bipartite graph. Is there a reliable procedure to improve the matching? We could proceed by a guess and check method -- we remove an edge in our matching, then try and add an edge using one of the vertices which has been freed up. If this edge cannot be taken because the end point is attached to an edge still in the matching, we remove that edge, freeing up more vertices. If we ever end up adding more edges than we started with (which occurs when we add an edge not attached to any points in the current matching), then we find a matching with an extra edge than before. This process is formalized by the `augmenting paths' construction.

Given a particular matching $M$, construct a directed graph $G_M = (V \cup \{ s, t \},E_M)$, where $s$ and $t$ are new vertices. Let $w \in W$, $u \in U$. Construct the edges $E_M$ such that
%
\begin{itemize}
    \item $wu \in E_M$, if $wu \in M^c$.
    \item $uw \in E_M$, if $uw \in M$.
    \item $sw \in E_M$, if $w$ is not the endpoint of any edge in $M$.
    \item $ut \in E_M$, if $u$ is not the endpoint of any edge in $M$.
\end{itemize}
%
Let $(s,w_1,u_1,\dots,w_n,u_n,t)$ be a path in $G_M$ from $s$ to $t$, with $w_i \in W$, $u_i \in U$. Then $w_i u_i \in M^c$ and $u_iw_{i+1} \in M$, and $w$ and $u$ are both unused in $M$. Let $M'$ be obtained from $M$ by removing all edges of the form $u_iw_{i+1}$, and adding all edges of the form $w_i u_i$. It is easy to convince yourself that $M'$ is a matching, with one more edge than $M$.

Conversely, suppose that there is no path from $s$ to $t$ in $G_M$. We claim $M$ is then a maximal matching on $G$. Otherwise, we would have a matching $M^*$ with $|M^*| > |M|$. Consider the multigraph $H = (V, M \amalg M^*)$. Every vertex in $V$ has at most degree two (for it can only be the endpoint of a single edge in $M$, and a single endpoint in $M^*$), and thus $V$ breaks into cycles and paths. For any cycle $C$, we find $|M \cap C| = |M^* \cap C|$, for the edges in the cycle must alternate being being in $M$ and $M^*$. Since $|M^*| > |M|$, there must be a path $P$ in $H$ containing more edges in $M^*$ than in $M$. $P$ must therefore begin and end with a path in $M^*$, and we see that this path is exactly an augmenting path for $M$.

This argument justifies the correctness of the following matching algorithm, a variant of the `Ford Fulkerson Algorithm'. We take any initial matching $M$. Then, we construct the augmenting path graph $G_M$, and use any of the standard algorithms to find a path from $s$ to $t$. If we have $2n$ nodes, and $m$ edges, then we can find a path in $O(m)$ time, and we can bound the number of augmentations we require to $O(n)$, so the algorithm runs in $O(nm)$ time.

\section{Vertex Covers}

A vertex cover $C \subset V$ is a set of points containing at least one endpoint of every edge in $E$. Clearly if $M$ is a maximal matching in $G$, then $|M| \leq |C|$, because there is an injective function $f:M \to C$, obtained by mapping an edge to one of its endpoints which lies in $C$.

\begin{theorem}[K\"{o}nig - Egerv\'{a}ry]
    Let $G = (W \amalg U, E)$ be bipartite, and let $M$ be a maximal matching in $G$. Let $Z \subset V \cup \{ s, t \}$ be the nodes reachable from $s$ in the augmenting path graph $G_M$. Then $C = (W - Z) \cup (U \cap Z)$ is a cover with $|C| = |M|$.
\end{theorem}
\begin{proof}
    First we prove $C$ is a cover. Let $wu \in E$ be arbitrary.
    %
    \begin{itemize}
        \item Suppose that $wu \in M$. If $u \in Z$, then $w \in Z$, because $uw$ is an edge in $G_M$, so that if $w \not \in C$, $u \in C$. \item Suppose that $wu \not \in M$. Then $wu$ is an edge in $G_M$, so that if $w \in Z$, $u \in Z$, and therfore if $w \not \in C$, $u \in C$.
    \end{itemize}
    %
    Now we show $|U| = |M|$. First, note that $C$ contains {\it exactly one} of each of the endpoints of $M$, so $|U| \geq |M|$. Conversely, let $wu \in E - M$ be arbitrary. If $u \in C$, then $u$ is reachable from $s$, and hence $u$ is the endpoint of an edge in $M$, for otherwise $t$ is reachable from $s$ through $u$. Conversely, if $w \in C$, then $w$ is not reachable from $s$, hence $w$ is on an edge in $M$ (for otherwise there is an edge directly to $w$ from $s$ in $G_M$). This shows that every vertex in $C$ is on one of the edge in $M$, and hence $|U| = |M|$.
\end{proof}

This gives us a method of finding a min cover in a graph. One equivalent form of this theorem is very useful. Given an undirected graph $G = (V,E)$, let $N$ be the neighbour function $N(S) = \{ v \in V : sv \in E, s \in S \}$.

\begin{theorem}[Hall]
    Let $G = (W \amalg U, E)$ be a bipartite graph. Then $G$ has a matching $M$ which covers $W$ if and only if $|N(S)| \geq |S|$ for all $S \subset W$.
\end{theorem}
\begin{proof}
    Certainly if a matching $M$ covers $W$, then it induces an injective function from $S$ to $N(S)$ for each $S \subset W$. Conversely, assume no matching of $M$ covers $W$. Then K\"{o}nig's theorem tells us there is a cover $C$ of $W \amalg U$ containing less than $|W|$ nodes. Then write $A = C \cap W$, $B = C \cap U$. We find $N(W - A) \subset B$, so $|N(W-A)| \leq |B| < |W| - |A|$.
\end{proof}

\section{The Hopcroft-Karp Matching Algorithm}

The augmenting path approach to finding a maximal matching in a Bipartite graph improves the length of a candidate matching by one for each cycle of the algorithm. If we were able to consistently find augmentations which improved matchings by more than one edge, we could find a faster algorithm for maximal matching.

If $M$ is a matching, and $N_M$ be the length of a shortest alternating path in $G_M$, we shall let a collection of vertex disjoint alternating paths $P_1, \dots, P_n$ be called {\bf $\mathbf{M}$-blocking} if $|P_i| = N_M$, and every alternating path of length $N_M$ shares a vertex with one of the $P_i$. The next lemma is a simple generalization of the augmenting paths approach.

\begin{theorem}
    If $P_1, \dots, P_n$ are vertex disjoint $M$-alternating paths, then
    %
    \[ M' = (M - \cup P_i) \cup (\cup P_i - M) \]
    %
    is a matching with $|M'| = |M| + n$.
\end{theorem}

The Hopcroft-Karp algorithm proceeds as in Ford-Fulkerson, albeit improving matchings by taking $M$-blocking paths as augmentations, improving matchings in chunks. We shall discover a method to find an $M$-blocking in linear time, and also show the chunk method yields an asymptotic speedup in the number of cycles to find a maximal matching, so that Hopcroft-Karp is objectively faster than Ford-Fulkerson.

\begin{lemma}
    If $M$ is a matching, and $P_1, \dots, P_n$ an $M$-blocking set of paths. Then $M' = (M - \cup P_i) \cup (\cup P_i - M)$, and $N_{M'} \geq N_M + 1$.
\end{lemma}
\begin{proof}
    Let $Q = (v_1, \dots, v_k)$ be an alternating path in $G_{M'}$ with $|Q| \leq |N_M|$, where $v_1$ and $v_k$ are $M'$ exposed. Then $v_1$ and $v_k$ are also $M$ exposed, because switching from $M$ to $M'$ only removes exposed vertices, not adds them.

    Let $H$ be the directed graph with edges taken from all the edge lying on the $P_i$, and the directed edges of $Q$, except that we remove both copies of duplicate edges. The reverse $vu$ of any edge $uv$ in $Q$ that is not an edge on some $P_i$ must be on some $P_j$
\end{proof}

\section{Network Flow}

Let $G = (V,E)$ be a directed graph with two identified vertices $s \neq t$.  Let $\mu: E \to \mathbf{R}^+$ be a function measuring the `capacity' of each edge in the graph (Thinking of the edges as if they were `pipes' which can only carry a certain throughput). Define, for $U \subset V$,
%
\[ \delout(U) = \{ w \in V - U : vw \in E \}\ \ \ \ \ \ \ \ \ \ \delin(v) = \{ w \in V - U : wv \in E \} \]
%
A flow is a mapping $f: E \to \mathbf{R}^+$ such that $0 \leq f \leq \mu$, and for any $v \neq s,t$,
%
\[ f(\delout(v)) = f(\delin(v))\ \ \footnote{We extend functions $g: X \to \mathbf{R}$ to $g: 2^X \to \mathbf{R}$ by defining $g(A) = \sum_{x \in A} g(x)$} \]
%
a relation known as the flow conservation law. The aim of the maximum flow problem is to find $f$ such that the value function
%
\[ \text{val}(f) = f(\delout(s)) - f(\delin(s)) \]
%
is maximized. This is essentially the amount of flow which is created at $f$. It is also the amount of flow which is `absorbed' at $t$, because
%
\begin{align*}
    f(\delout(t)) - f(\delin(t)) &= f(\delout(t)) - f(\delin(t)) + \sum_{v \neq s,t} f(\delout(v)) - f(\delin(v))\\
    &= \sum_{v \neq s} \left( \sum_{vw \in E} f(vw) - \sum_{wv \in E} f(wv) \right)\\
    &= \left( \sum_{\substack{v \neq s\\vs \in E}} f(vs) + f(ss) \right) - \left( \sum_{\substack{v \neq s\\sv \in E}} f(sv) + f(ss) \right)\\
    &= -[\delout(f,s) - \delin(f,s)]
\end{align*}
%
Thus the maximum flow problem tells us a method of maximizing the amount of flow which gets to $t$.

There is an interesting relation between flows on graphs, and another structure on these graphs known as a cut, which we will take full advantage of in finding solutions to the max flow problem. Define a $(s,t)$ cut on a directed graph $G = (V,E)$ with vertices $s$ and $t$, to be a partition of $V$ into two sets, one containing $s$, and one containing $t$. It is however more simple to consider a cut to be a subset $C$ of $V$ containing $s$, but not $t$. If we have a capacity function $\mu$, then we define the value of the cut to be $\text{val}(C) = \mu(\delout(C))$. The min cut problem is to find a cut of smallest value.

\begin{lemma}
    If $f$ is a flow, and $C$ is an $(s,t)$ cut, then $\text{val}(f) \leq \mu(\delout(C))$.
\end{lemma}
\begin{proof}
    Since $t \not \in C$, similar manipulations to the ones above show that
    %
    \begin{align*}
        \text{val}(f) &= \sum_{v \in C} f(\delout(v)) - f(\delin(v))\\
        &= f(\delout(C)) - f(\delin(C))\\
        &\leq \mu(\delout(C))
    \end{align*}
    %
    Note that we obtain equality if $f(\delout(C)) = \mu(\delout(C))$ and $f(\delin(C)) = 0$, in which case $f$ must be a maximal flow, and $C$ a min cut.
\end{proof}

As with maximal matchings, we attempt to find a maximal flow by finding ways of augmenting suboptimal flows to a maximal solution. Given a flow $f$, we construct the residual graph $G_f$, which has the same vertices as $G$, but whose edges are defined to be the set
%
\[ E_f = \{ uv \in E : f(uv) < \mu(uv) \} \coprod \{ vu : uv \in E, f(uv) > 0 \} \]
%
We shall denote the element $vu$ in the second set which makes up $E_f$ by $\overleftarrow{uv}$, since we obtained in from an edge $uv$ in $G$. Also define a capacity function
%
\[ \mu_f(uv) = \mu(uv) - f(uv)\ \ \ \ \ \ \ \ \ \mu_f(\overleftarrow{uv}) = f(uv) \]
%
Suppose we can find a simple path $P$ from $s$ to $t$ in $G_f$, and define $\alpha = \min_{e \in P} \mu_f(e)$. Consider a new flow
%
\[ f'(uv) = \begin{cases} f(uv) + \alpha : uv \in P \\ f(uv) - \alpha : \overleftarrow{uv} \in P \end{cases} \]
%
First we show that $f'$ is a flow. By the choice of $\alpha$, $0 \leq f' \leq \mu$. For each vertex $v \neq s,t$, we must show that flow conservation still holds. Let $e_1, e_1', \dots, e_n, e_n'$ be the edges in $G$ containing $v$ as an endpoint obtained from $P$, by reversing edges of the form $\overleftarrow{uv}$, considered in the order they appear in $P$. We may write pair up the edges in this manner because, for every edge that enters $v$ in $P$, there must be an edge that leaves $v$ in $P$. Let
%
\[ S_i = \begin{cases} f(e_i) - f'(e_i) & e_i = uv \\ f'(e_i) - f(e_i) & e_i = vu \end{cases} \]
%
Define $S_i'$ similarily for $e_i'$. Then, since no edges are repeated,
%
\begin{align*}
    f'(\delout(v)) - f'(\delin(v)) &= f(\delout(v)) - f(\delin(v))\\
    &\ \ \ \ \ + \sum_{i = 1}^n (S_i + S_i')
\end{align*}
%
We now show $S_i + S_i' = 0$ for each $i$. This breaks into several cases.
%
\begin{itemize}
    \item If $e_i = uv$, $e_i' = vw$, in which case $f'(e_i) = f(e_i) + \alpha$, $f'(e_i') = f(e_i') + \alpha$, and $S_i + S_i' = \alpha - \alpha = 0$.
    \item If $e_i = uv$, $e_i' = wv$, then $e_i'$ was obtained from an edge of the form $\overleftarrow{vw}$ in $P$, in which case $f'(e_i') = f(e_i') - \alpha$, and so $S_i + S_i' = -\alpha -(-\alpha) = 0$.
    \item If $e_i = vu$, $e_i' = wv$, then $e_i$ and $e_i'$ were obtained from reversed edges in $P$, and so $S_i + S_i' = (-\alpha) - (-\alpha) = 0$.
    \item If $e_i = vu$, $e_i' = vw$, then $e_i$ was obtained from reversing edges, and so $S_i + S_i' = (-\alpha) + \alpha = 0$.
\end{itemize}
%
And it follows, because flow conservation holds at $v$ for $f$, that it holds at $v$ for $f'$ as well. Finally, we find that $\text{val}(f') = \text{val}(f) + \alpha$, which can be shown by performing an analysis, similar to $v$ above, for $s$, noting that $s$ will have an extra edge at the beginning of the path, which causes the extra $\alpha$. All that remains is to show this augmentation method yields a maximum flow in all cases, after enough iterations.

Let $f$ be a flow for which $G_f$ has no augmenting path. Let $U$ be the nodes reachable from $s$ in $G_f$, easily verified to be a cut.
%
\[ \text{val}(f) = \mu(\delout(U)) \]
%
proving maximality of the flow, and minimality of the cut $U$. To see this, let $uv \in \delout(U)$. Then $f(uv) = \mu(uv)$, for otherwise $v$ would be reachable from $s$ in $G_f$. Similarily, we must have $f(\delin(U)) = 0$, for if $vu \in \delin(U)$, and $f(vu) > 0$, then $\overleftarrow{vu} \in G_f$, and so $v$ is reachable from $s$, a contradiction. We conclude that
%
\[ \text{val}(f) = f(\delout(U)) - f(\delin(U)) = \mu(\delout(U)) \]
%
which shows that our augmenting paths approach works. Notice that if all edge weights are integral, then the augmenting flows are always integral, and thus there exists a maximal flow with interal weights.

The Ford Fulkerson algorithm solves the max flow problem by repeatedly augmenting an initial flow. In principle, this approach is correct, but if you do not find a residual path in a smart way, this algorithm does not terminate in polynomial time for all inputs. If all edge weights are integers $n_1, \dots, n_m$, then each augmentation increases the value of the flow by at least the greatest common factor of the $n_i$. Since there is a bound on the size of a maximal flow, obtained by taking the trivial cut $\{ s \}$, the algorithm will eventually terminate in time proportional to
%
\[ O \left(\frac{n_1 + \dots + n_m}{\lcm(n_1, \dots, n_m)} \right) \]
%
which is exponential in the bit complexity of the representation. We cannot do any better than this, as the graph in the figure above provides an example. If the edge weights are rational, similar results can be obtained by multiplying out denominators.

\begin{figure}
\begin{center}
\begin{tikzpicture}[transform shape]
    \GraphInit[vstyle=Classic]

    \Vertex[x=-3,y=0]{S}
    \Vertex[NoLabel,x=0,y=1.5]{A1}
    \Vertex[NoLabel,x=0,y=-1.5]{A2}
    \Vertex[x=3,y=0]{T}

    \draw[label={$2^n$}, ->, thick](S)--(A1) node[draw=none,fill=none,midway,above] {$2^n$};
    \draw[label={$1$}, ->, thick](A1)--(A2) node[draw=none,fill=none,midway,left] {1};
    \draw[label={$2^n$}, ->, thick](S)--(A2) node[draw=none,fill=none,midway,below] {$2^n$};
    \draw[label={$2^n$}, ->, thick](A1)--(T) node[draw=none,fill=none,midway,above] {$2^n$};
    \draw[->, thick](A2)--(T) node[draw=none,fill=none,midway,below]  {$2^n$};
\end{tikzpicture}
\end{center}
\caption{If our algorithm continuously switches between adding weights from left to right via the central edge, then we will need to compute $O(2^n)$ residual graphs before termination, even though the graph can be represented in $O(n)$ bits.}
\end{figure}

We achieve much better estimates if we use breadth first search to find our residual path -- that is, we always take the shortest path (in length, not in weight) from $s$ to $t$. But this, of course, requires a careful analysis.

\begin{theorem}[Mader]
    Let $G$ be an undirected graph, and $s,t$ vertices not directly connected by an edge. Then the maximal number of vertex-disjoint $(s,t)$ paths is equal to the minimum cardinality of a set $B \subset V - \{ s, t \}$ which contains a vertex in any path from $s$ to $t$.
\end{theorem}
\begin{proof}
    Form a directed bipartite graph $W$, which consists of two copies of the vertices in $G$. If $v$ is a vertex, then we shall separately denote its copies by $v^1$ and $v^2$. If $uv$ is an edge in $G$, attach edges $u^2v^1$ and $v^2u^1$ to $W$, and attach an edge $v^1v^2$ for each vertex $v$. Define a capacity function on $w$ by
    %
    \[ \mu(u^2v^1) = \mu(v^2u^1) = \infty\ \ \ \ \ \mu(v^1v^2) = 1 \]
    %
    Note that any $(s,t)$ path in $G$ leads to an alternating path in $W$ from $s^2$ to $t^1$, which must cross at least one 1-capacity edge because there is no edge from $s$ to $t$.

    Let $f$ be an integer-valued flow in $W$. Then $f \leq 1$, because an infinite capacity edge $u^2 v^1$ satisfies $f(u^2 v^1) \leq f(\delin(u^2)), f(\delout(v^1))$
\end{proof}

\section{Goldberg-Tarjan Push Relabel Algorithm}

Our previous algorithm found max flows in a graph by starting with a flow, then slowly augmenting it, improving the flow value until it is maximum. An alternative idea would be to take a flow-like function, which would be a maximum flow if it obeyed flow conservation, and slowly correcting the flow until we have a maximum flow. Define a {\bf preflow} on a graph with capacity function $\mu$ to be a positive real-valued function $f$, defined on the edges, such that $f \leq \mu$, and $f(\delout(v)) - f(\delin(v)) \geq 0$ for all $v \neq s$. A preflow defines a residual graph $G_f$ in the same way that a flow does. A non-negative integer valued function $\psi$ on vertices is a {\bf distance label} for a particular preflow $f$ if $\psi(s) = |V|$, $\psi(t) = 0$, and $\psi(v) \leq \psi(w) + 1$ for $vw \in G_f$. Notice that if there is a path from $s$ to $t$ in $G_f$, then a distance label cannot exist, because if it has length $k$, then the inequality property gives
%
\[ \psi(s) \leq (k-1) + \psi(t) \leq (|V|-1) + \psi(t) < |V| \]
%
Thus if we continuously keep track of a preflow and a distance label, and keep adjusting the preflow and distance label to obtain a flow with a distance label, we have obtained a maximal flow. In order to describe a method achieving this, we call a vertex $v \neq s,t$ {\bf active} if $f(\delout(v)) - f(\delin(v)) > 0$, and an edge $vw \in G_f$ {\bf admissable} if $\psi(v) = \psi(w) + 1$. The goal is to find a preflow with no active vertices, in which case we have a flow.

We now describe the push-relabel algorithm for finding the maximal flow in a graph. Begin by setting
%
\[ f(e) = \begin{cases} \mu(e) & e = sv \\ 0 & \text{otherwise} \end{cases}\ \ \ \ \ \ \ \ \ \ \psi(v) = \begin{cases} |V| & v = s \\ 0 & v \neq s \end{cases} \]
%
Then $\psi$ is a distance label for the preflow $f$, since $G_f$ has no edges leaving $s$.

Suppose there is an active vertex $v$ relative to $f$. If there are no admissable edges leaving $v$, we may introduce an admissable edge by setting $\psi(v) = \min_{vw \in G_f} \psi(w) + 1$. This leaves the $f$ a preflow, and $\psi$ a distance label. If there is an admissable edge $vw$ leaving $v$, we set $f(vw)$ to be
%
\[ \min[ \mu_f(vw), f(\delout(v)) - f(\delin(v))] \]
%
Then $f \leq \mu$ still holds, and $f(\delout(v)) - f(\delin(v)) \geq 0$, since we can only have let at most $f(\delout(v)) - f(\delin(v))$ flow out. Also $f(\delout(w)) - f(\delin(w)) \geq 0$, since we have only let more flow in. The distance labeling is still a distance labelling, because we introduce an edge $wv$ in $G_f$, which requires that $\psi(w) \leq \psi(v) + 1 = \psi(w) + 2$, and an edge from $vw$, so that $\psi(v) = \psi(w) + 1 \leq \psi(w) + 1$. Thus the fact that $\psi$ is a distance labelling of the preflow $f$ is an invariant of the algorithm.

\begin{lemma}
    If $f$ is a preflow and $\psi$ a distance label for $f$, then if $f(\delout(v)) - f(\delin(v)) > 0$, there is a $(v,s)$ path in $G_f$.
\end{lemma}
\begin{proof}
    s
\end{proof}

\section{Cuts in Undirected Graphs}

Let $G$ be an undirected graph with capacities $\mu \geq 0$. For a subset of vertices $U$ let $\delta(U)$ denote all edges with a single endpoint in $U$. A {\bf cut} in such a graph is $\emptyset \subsetneq U \subsetneq V$, with value $\mu(\delta(U))$. The global minimum cut problem is to find the cut with $\mu(\delta(U))$ minimized, without regard to where any particular pair $(s,t)$ resides. A naive way to calculate the global minimum cut is to calculate a minimum $(s,t)$ cut, for all pairs $(s,t)$. This requires $n^2$ calls to the maximum flow problem.

And while we discuss this problem, we briefly sketch out how to compute a max flow in an undirected graph. Given such a graph $G$ with capacity $\mu$, we define $H$ to be the directed graph with the same vertices as $G$, and for each edge $uv \in G$, two edges $uv$ and $vu$ in $H$, with a capacity function $\psi$ such that $\psi(uv) = \psi(vu) = \mu(uv)$. Given any flow $f$ in the directed graph, we obtain a flow $f'$ in the undirected graph of the same value by subtracting $\min(f(uv), f(vu))$ from each edge. Thus we can compute a max flow in $O(V^2E)$ time in an undirected graph as well. The value of min cuts in the directed graph is the same as the value of min cuts in the undirected graph, so the max flow min cut theorem holds here as well.

More smartly, we notice that if $s$ is fixed, $s$ lies on either side of the cut, and since $\mu(\delta(U)) = \mu(\delta(U^c))$, we need only try each $(s,t)$ cut as $t$ ranges over all $t \neq s$, which only requires $n-1$ calls to the max flow function. In a directed call, we need try all $(s,t)$ cuts and $(t,s)$ cuts, which requires $2(n-1)$ calls to max flow.

We can do much better than just computing a min-cut in $O(n)$ times. It turns out that there is a way to compactly represent all minimum cuts via $O(n)$ calls to a min-cut algorithm. The first step to the representation is to find a useful relationship between the values of the numbers $\lambda_{u,v}$, the capacities of a min $(u,v)$ cut.

\begin{lemma}
    Let $v_1, \dots, v_k$ be vertices. Then $\lambda_{v_1, v_k} \geq \min(\lambda_{v_1,v_2}, \dots, \lambda_{v_{k-1},v_k})$.
\end{lemma}
\begin{proof}
    Assume that $k \geq 3$, since the proof is trivial for $k = 2$. Let $C$ be a $(v_1,v_k)$ cut with $\mu(\delta(C)) < \min(\lambda_{v_1,v_2}, \dots, \lambda_{v_{k-1},v_k})$. Inductively, we conclude that $v_2, \dots, v_k \in C$. But then by definition $C$ cannot be a $(v_1, v_k)$ cut.
\end{proof}

Let $T = (V,E)$ be a tree, and $e \in E$ an edge in the tree. Then the set of nodes in some connected components of the graph $T' = (V, E - \{ e \})$ is known as the {\bf fundamental cut}. There are only two connected components, one for each vertex on the ends of $e$. A {\bf Gomory-Hu Tree} for an undirected graph $G$ is a tree $T$ with the same vertices as $G$, such that the fundamental cut for every edge $uv \in T$ is a minimal $uv$ cut. It turns out that Gomory-Hu trees always exist, can be computed in $O(n)$ calls to a min $(s,t)$ cut function, and compactly represent every min cut in a graph, for if $(v_1, \dots, v_n)$ is the unique simple path in $T$, then
%
\[ \lambda_{v_1,v_n} = \min(\lambda_{v_1,v_2}, \dots, \lambda_{v_{n-1},v_n}) \]
%
Certainly the left side is $\geq$ to the right side. On the other hand, by construction of the tree, the fundamental cut of $(v_i,v_{i+1})$ is also a $(v_1,v_n)$, hence the left side is $\leq$ to the right side.

\begin{theorem}[Cut Submodularity]
    For $A,B \subset V$, we have
    %
    \[ \mu(\delta(A)) + \mu(\delta(B)) \geq \mu(\delta(A \cap B)) + \mu(\delta(A \cup B)) \]
\end{theorem}
\begin{proof}
    Fix an edge $uv$. We shall show that every edge counted on the right side occurs at least as many times on the left side.
    %
    \begin{itemize}
        \item Suppose $uv \in \delta(A \cap B)$, $uv \not \in \delta(A \cup B)$. Then we may assume $u \in A \cap B$, $v \not \in A \cap B$, $v \in A \cup B$. Then $\mu(uv)$ is counted once on the right hand side, and once on the left hand side.
        \item Suppose $uv \not \in \delta(A \cap B)$, $uv \in \delta(A \cup B)$. Thus we may assume $u \in A \cup B$, $v \not \in A \cup B$, $u \not \in A \cap B$. Then $\mu(uv)$ is counted once on the right hand side, and once on the right hand side as well.
        \item Suppose $uv \in \delta(A \cap B)$, $uv \in \delta(A \cup B)$. Then we may assume $u \in A \cup B$, $v \not \in A \cup B$, and so $u \in A \cap B$. Then $\mu(uv)$ is counted once on the right hand side, and twice on the left hand side.
    \end{itemize}
    %
    Equality holds when $\delta(A \cap B)$ is disjoint from $\delta(A \cup B)$.
\end{proof}

\begin{lemma}
    Let $s \neq t$ be two vertices in a graph $G$, and let $C$ be a min $(s,t)$ cut. If $u \neq v$ are also distinct vertices, then there is a min $(u,v)$ cut $D$ with $C \subset D$ or $D \cap C = \emptyset$.
\end{lemma}
\begin{proof}
    Note that the statement is really redundant, because $C \subset D$ occurs exactly when $D^c \cap C = \emptyset$. Let $B$ be any minimum $(u,v)$ cut. We will modify $B$ to another min cut with the desired properties. Note that
    %
    \[ \mu(\delta(C)) + \mu(\delta(B)) \geq \mu(\delta(C \cap B)) + \mu(\delta(C \cup B)) \]
    %
    We may assume $s,u \in B$. Then $t \not \in B$, $C \cap B$ is an $(s,t)$ cut, hence $\mu(\delta(C \cap B)) \geq \mu(\delta(C))$, and so
    %
    \[ \mu(\delta(B)) \geq \mu(\delta(C \cup B)) \]
    %
    Now $u \in C \cup B$, and $v \not \in C \cup B$ (WHY?), so $D = C \cup B$ satisfies the properties of the lemma.
\end{proof}

\section{Potentials}

Fix a directed graph $G$ with edge costs $c$. A {\bf potential} for $G$ is a mapping $\phi: V \to \mathbf{R}$ such that
%
\[ c_\phi(uv) = c(uv) - (\phi(v) - \phi(u)) \geq 0 \]
%
for al edges $uv$ in $G$. The costs $c_\phi$ are known as the potential costs.

\begin{lemma}
    For any $s-t$ path $P$ in $G$,
    %
    \[ c_\phi(P) = c(P) - (\phi(t) - \phi(s)) \]
\end{lemma}
\begin{proof}
    If $P = (v_1, \dots, v_n)$, then we have an alternating sum
    %
    \[ c_\phi(P) = \sum_{k = 1}^{n-1} c(v_k v_{k+1}) + \phi(v_k) - \phi(v_{k+1}) = \sum_{k = 1}^{n-1} c(v_k v_{k+1}) + \phi(v_1) - \phi(v_n) \]
    %
    and the right side is exactly the form considered in the statement of the theorem.
\end{proof}

This is a discrete version of the fact that the energy change of a particle under a conservative force is invariant of the path of the particle, but only the change of the particles position in space. A simple corollary to the lemma is that the minimal cost paths relative to the metric $c$ are the same as the minimal cost paths relative to $c_\phi$.

If a graph has negative edge weights, but possesses some potential $\phi$, then we can compute minimal paths under the cost function $c_\phi$ (which is now has positive edge weights everywhere) using Dijkstra's algorithm. There is a simple method to finding a potential, which we now detail.

If $P$ is a cycle in a graph $G$ with potential $\phi$, we calculate
%
\[ c_\phi(P) = c(P) \geq 0 \]
%
so that $G$ has no negative weight cycles. Conversely, if $G$ has no negative weight cycles, then it turns out that $G$ has a potential, and this potential can be computed in polyomial time.

\begin{theorem}
    A potential can be computed for a graph $G$ can be computed in $O(nm)$ time.
\end{theorem}
\begin{proof}
    Given $G$, compute a graph $H$ by adding a new vertex $r$ to $G$, and adding edges from $r$ to every vertex in $G$. Set $c(rv) = 0$ for all vertices $v \in V$, and let $\phi(v)$ be the cost of the cheapest walk from $r$ to $v$. This is well defined since $G$ has no negative weight cycles (there is a cheapest walk with no repeated vertices, for if a path has a repeated vertex, it contains a cycle, which must have non-negative weight, and this cycle can be removed with no penalty to the length of the walk -- we then need only consider the finite set of paths with no repeated vertices to obtain an answer). Note that $c(uv) + \phi(u) \geq \phi(v)$, so $\phi$ is a potential for $G$. The Bellman-Ford can compute $\phi$ in $O(nm)$ time.
\end{proof}

Let $\omega$ be a differential 1-form on a manifold $M$ such that
%
\[ \int_\gamma \omega \geq 0 \]
%
for any cycle $\gamma$. Then if $\gamma$ and $\delta$ are two paths from a point $x$ to a point $y$, we find
%
\[ \int_\gamma \omega - \int_\delta \omega, \int_\delta \omega - \int_\gamma \omega \geq 0 \]
%
so $\int_\gamma \omega = \int_\delta \omega$, and given a fixed $x$, we can define
%
\[ F(y) = \int_\gamma \omega \]
%
where $\gamma$ is a path from $x$ to $y$, and one verifies that $dF = \omega$, so that
%
\[ \int_\gamma \omega - (F(y) - F(x)) = 0 \]
%
a stronger fact which is analogous to the result above, although our result is weaker because in directed graphs we cannot perform the swapping trick to obtain equality of all paths.

\section{Minimum Cost Flows}

Throughout this section, we fix a directed graph with edge costs $c$, edge capacities $\mu$, fixed vertices $s,t$, and a target flow value $\gamma$. The minimum cost flow problem tries to find a flow $f$ with value $\text{val}(f) \geq \gamma$, which minimizes the cost $\text{cost}(f) = \sum c(e) f(e)$, or to determine if no such flow exists.

We can certainly compute in polynomial time if a flow exists satisfying the constraints, because we can find a maximal flow in the graph. It is not so clear that we can find such a flow with minimal cost. It shall always be the case that the minimal flow satisfies $\text{val}(f) = \gamma$, because given any flow $f$, the value of the flow $g = \text{val}(f)/\gamma f$ satisfies $\text{val}(g) = \gamma$, and $\text{cost}(g) = \text{val}(f)/\gamma \sum c(e) f(e)$, which will be smaller than $\text{cost}(f)$ if $\text{val}(f) \neq \gamma$.

\section{Minimal Cost Bipartite Matching}

We now take a brief aside to discuss how minimal cost flows can be used to solve the analogous problem in bipartite graphs. Given the name, the minimal cost bipartite matching problem should be fairly self explanatory. We take a particular Bipartite graph $G = (L \cup R, E)$ with edge costs $c$, and we must find a minimum cost perfect matching on $G$ in polynomial time. That is, our goal is a perfect matching $M$ which minimizes
%
\[ \text{cost}(M) = \sum_{e \in M} c(e) \]
%
The similarity to the minimal cost flow problem immediately suggests a connection between the two problems.

We can formulate the perfect matching problem as a minimal cost max flow problem, by modifying slightly the standard reformulation of the maximal matching problem as a max flow problem. The original process is to construct a directed bipartite graph $H$, whose vertices are obtained from $G$ by adding two new vertices $s$ and $t$, orienting all edges in $E$ so that they point from $L$ to $R$, and adding new edges $sv$, for $v \in L$, and $wt$, for $w \in R$. If we define a capacity function $\mu = 1$, then there is a one-to-one correspondence between integral valued $s-t$ flows and matchings in the graph. To see this, note that an integral valued flow $f$ gives rise to a characteristic function $\chi_{A_f}$ on the edges of $G$, and the flow conservation law shows that if $e$ and $e'$ are adjoint, then $\chi_{A_f}(e) + \chi_{A_f}(e') \leq 1$, so that $A_f$ forms a matching on $G$. Correspondingly, the reverse process takes a matching to a characteristic function on the edges of $G$, which can easily be completed into a flow on $H$.

Now we define a cost function $c'$ on $H$ in a way which mirrors the cost function $c$ on $G$. We let
%
\[ c'(sv) = c'(wt) = 0\ \ \ \ \ c'(vw) = c(vw) \]
%
Then we see that the correspondence described above preserves the cost function. That is, if a matching $M$ corresponds to a flow $f$, then $\text{cost}(M) = \text{cost}(f)$, because $f$ is obtained from $M$ by adding edges of the form $sv$ and $wt$, which add no value to the cost of the flow. Thus if we find a minimal cost maximal flow in $H$, we can find a minimal cost perfect matching in $G$, and we know we can find such a maximal flow using the successive shortest paths algorithm, and we need only compute $|L| = |R|$ successive paths, since each iteration of the algorith improves the length of the candidate path, hence now have an algorithm for the minimal cost perfect matching problem which runs in $O(n(m + n \log n))$ time.

\section{Minimum Mean Cycle Cancelling}

We already know that a flow $f$ has minimal cost if $G_f$ contains no negative-cost cycles. This suggests that a strategy for finding minimal cost flows is obtained from successively pruning negative-cost cycles from $G_f$ until there are none left. As with Ford Fulkerson, we must be careful which cycles we prune in order to guarantee a good asymptotic speed to our algorithm. It turns out that fast asymptotic results are available if we prune negative cycles with `minimum mean'.

\begin{algorithm}
\caption{The Minimum Mean Cycle Canceling Algorithm}
\begin{algorithmic}[1]
\State $f \gets$ any $(s,t)$ flow of value $\gamma$ (found in $O(n^3 \sqrt{m})$ time)
\While {$G_f$ contains negative weight cycles}
\State $C \gets$ a cycle in $G$ which minimizes its mean cost $c(C)/|C|$.
\State Augment $f$ along $C$.
\EndWhile
\State \Return $f$
\end{algorithmic}
\end{algorithm}

The algorithm certainly terminates, since the cost of $f$ decreases for every iteration of the algorithm, at a rate bounded by the rational values of the edge costs. It remains to be seen that the algorithm terminates in a polynomial amount of time (and this is not immediately obvious, and is not true if we do not take cycles which minimize the mean cost).

First, we argue why we can find a minimum ratio cycle $C$ in polynomial time.

\begin{lemma}
    Any Eulerian multigraph $H = (W,F)$ with edge costs $c$ contains a cycle $C$ such that
    %
    \[ \frac{c(C)}{|C|} \leq \frac{c(F)}{|F|} \]
\end{lemma}
\begin{proof}
    Perform a cycle decomposition on $H$, writing $F = C_1 \amalg \dots \amalg C_n$, we find
    %
    \[ \frac{c(F)}{|F|} = \frac{c(C_1) + \dots + c(C_n)}{|C_1| + \dots + |C_n|} \]
    %
    If $c(C_i)/|C_i| > c(F)/|F|$ for each $i$, then $c(C_i) |F| > c(F) |C_i|$ also holds for each $i$, and so, summing up we find
    %
    \[ c(F) |F| = \sum_i c(C_i) |F| > \sum_i c(F) |C_i| = |F| c(F) \]
    %
    a clear contradiction which proves the existence of a $C_i$ satisfying the constraints of the lemma.
\end{proof}

Given any cycle $C$, the corresponding multigraph (which contains duplicates of repeated edges in the cycle) is Eulerian, and can thus be broken up into disjoint cycles. The lemma above shows that if $C$ is not a simple cycle, we can always find a strictly smaller cycle $C'$ without increasing the mean value of the cycle, so that we need only search simple cycles to find a minimum mean.

For $k \leq n$, define $\delta_k(v)$ to be the cheapest closed walk from $v$ to itself using exactly $k$ (not necessarily distinct) edges. If we let $\delta_k(v,w)$ be the cheapest path from $v$ to $w$ using $k$ edges, then we find
%
\[ \delta_k(v,w) = \min_{u \in V}\ c(v,u) + \delta_{k-1}(u,w) \]
%
Each value of $\delta_k(v,w)$ can be computed in $O(m)$ time, and so the set of all $\delta_k(v,w)$ can be computed in $O(mn^3)$ time. Let $v^*$ and $k^*$ minimize $\delta_{k^*}(v^*)/k^*$, and pick $k^*$ to be smallest to break ties. Then the walk from $v^*$ in $k^*$ stops is verified to be a cycle, from the above remark, and thus is a minimum ratio cycle, since it is a minimum ratio walk in general.

\begin{theorem}
    The number of iterations of the minimum mean cycle cancelling algorithm is $O(m^2 n \log n)$.
\end{theorem}
\begin{proof}
    Consider the sequence of flows
    %
    \[ f_1, f_2, \dots, f_N \]
    %
    which are found in the MMCC algorithm, together with the cycles
    %
    \[ C_1, \dots, C_{N-1} \]
    %
    which are augmented to produce the sequence of flows. For $i < j$, define
    %
    \[ \alpha(i,j) = \{ e, \overleftarrow{e} : e, \overleftarrow{e} \in C_i \cup C_j \} \]
    %

    \begin{lemma}
        If $\alpha(i,k) = \emptyset$ for $i < k < j$, then we find
        %
        \[ \frac{c(C_i)}{|C_i|} \leq \frac{n}{n-|\alpha(i,j)|}\ \frac{c(C_j)}{|C_j|} \]
    \end{lemma}
    \begin{proof}
    %
    Let $H = C_i \coprod C_j - \alpha(i,j)$ (essentially, $H$ is obtained from the cycles $C_i$ and $C_j$ by keeping duplicate edges, and `cancelling out' edges with their reverse). Then $H$ is Eulerian, for $C_i \coprod C_j$ is Eulerian, and removing edges along with their reverse changes both the in degree and out degree of a vertex by the same amount. Furthermore, each edge in $H$ is in $E_{f_i}$ because $\alpha(i,k) = \emptyset$, so
    %
    \begin{align*}
        \frac{c(C_i)}{|C_i|} (|C_i| + |C_j|) &= \frac{c(C_i)}{|C_i|} (|H| + |\alpha(i,j)|)\\
        &\leq c(H) + |\alpha(i,j)| \frac{c(C_i)}{|C_i|}\\
        &= c(C_i) + c(C_j) + |\alpha(i,j)| \frac{c(C_i)}{|C_i|}
    \end{align*}
    %
    where we have used the fact that $c(H) = c(C_i) + c(C_j)$ since the cost of reverse edges cancels out. This can be rephrased as
    %
    \[ \frac{c(C_i)}{|C_i|} \left( |C_j| - |\alpha(i,j)| \right) \leq \frac{c(C_j)}{|C_j|} |C_j| \]
    %
    And dividing out, we find
    %
    \[ \frac{c(C_i)}{|C_i|} \leq \frac{|C_j|}{|C_j| - |\alpha(i,j)|} \frac{c(C_j)}{|C_j|} \]
    %
    and this inequality can be weakened to the form above, because $x/(x-y)$ is a decreasing function of $x$ for $y > 0$, and $n/(n-|\alpha(i,j)|) \geq 0$.
    \end{proof}

    Returning to our original proof, we see that $j = i+1$ satisfies the theorem, and proves that the minimum mean ratios are non-decreasing. If $i < j$ is the smallest value such that $\alpha(k,i) \neq \emptyset$, for some $i \leq k < j$, then we find
    %
    \[ \frac{c(C_i)}{|C_i|} \leq \frac{n}{n - 2} \frac{c(C_j)}{|C_j|} = \left(1 - \frac{n}{2} \right) \frac{c(C_j)}{|C_j|} \]
    %
    and we have a bound $j \leq i + m$. To see this, note that at least one edge is removed in the residual graph from each of the cycles $C_i, C_{i+1}, \dots, C_{i+m}$ during the algorithm. Thus we conclude
    %
    \[ \frac{c(C_i)}{|C_i|} \leq \left( 1 - \frac{n}{2} \right)^n \frac{c(C_{i+nm})}{|C_{i+nm}|} \leq e^{-1/2} \frac{c(C_{i+nm})}{|C_{i+nm}|} \leq 2 \frac{c(C_{i+nm})}{|C_{i+nm}|} \]
    %
    If the edge weights are integral (and all rational cost problems can be reduced to this form), this already gives us enough information to prove a polynomial time bound. Note that since the minimum mean cycles are simple, we must have
    %
    \[ \frac{|c(C_i)|}{|C_i|} \geq \frac{1}{n} \]
    %
    and
    %
    \[ \frac{|c(C_1)|}{|C_1|} \leq \min_{e \in E} |c(e)| \]
    %
    so if $N$ is large enough that $2^{-N} \min_{e \in E} |c(e)| \leq n^{-1}$, then
    %
    \[ \frac{c(C_{i+Nnm})}{|C_{i+Nnm}|} \geq \frac{1}{2^N} \frac{c(C_1)}{|C_1|} \geq - \frac{\min_{e \in E} |c(e)|}{2^N} \geq -\frac{1}{n} \]
    %
    which forces the mean cycle ratio to be positive, and hence the algorithm will terminate. We can choose any $\lg(n \min_{e \in E} |c(E)|) < N$, so the algorithm terminates in at most
    %
    \[ nm  \lceil \lg n \min_{e \in E} |c(E)| \rceil \]
    %
    iterations, which is polynomial in the bit complexity of the algorithm, since if the costs are given with $l$ bits, then the total bit complexity is about $\Theta(n + ml)$, and $|c(E)| \leq 2^l$, so
    %
    \[ nm \lceil \lg n \min_{e \in E} |c(E)| \rceil \leq nm \lg(n) + lnm \]
    %
    This finishes our initial analysis of the mean cycle canceling algorithm.
\end{proof}

The polynomial bound for this algorithm is absolutely awful, and we can do much much better. In the next lecture we will begin the arguments that give us a much better bound on the runtime on this algorithm.

\section{Minimum Spanning Trees and Arboresences}

Let $G$ be an undirected graph, together with a (possibly-negative) edge cost function $c$. A {\bf spanning tree} $T$ on $G$ is a connected subset of edges containing no cycles (a tree) such that every vertex is covered by an edge in the tree. Note that this definition is equivalent to $T$ having no cycles, and $n-1$ edges (a fact proven most easily by induction, by removing vertexes and edges from $T$ which form the leaves of $T$). There is another equivalent definition which is more important to the spanning tree problems we'll discuss. $T$ is a spanning tree if it has $n-1$ edges, and $|E(S)| \leq |S| - 1$ for all vertex sets $S$ (where $E(S) = \{ uv : u,v \in S, uv \in T \}$).

\[ TODO: Draw a picture of S bounding the number of edges in the set S \]

The minimum spanning tree problem is self explanatory -- given $G$, find a spanning tree containing the fewest edges. We can formulate the problem as an $LP$ in exponentially many constraints in functions $x: E \to \mathbf{R}$,
%
\begin{gather*}
\min \quad \sum_e c(e) x(e) \\
\begin{aligned}
\textup{s.t.}\quad x(E(S)) &\geq |S| - 1\ \ \ \forall S \subsetneq V\\
                   x(E(V)) &= n - 1 \\
                   0 &\leq x
\end{aligned}
\end{gather*}
%
Note that the constraint $x \leq 1$ is already encoded in the problem by taking $S$ to be a pair of vertices corresponding to an edge. Our discussion above entails that any integral solution to this algorithm corresponds to a spanning tree in $G$. It can be shown that all extreme points are integral, but we shall only show that we can choose an optimal extreme point which is integral.

In it's current form, the minimal spanning tree linear program is unfeasible, having far too many constraints. We obtain a much better optimization problem if we switch to the dual of this program. This can be motivated by the fact that $x$ behaves more like a linear functional than a vector in this linear program (the constraints of the program are expressed as $x$'s operations on the vector space $\mathbf{R} \cdot E$ generated by the edges $E$), so that the program is the dual of some other, more easily understood linear program. The dual program finds functions $y: 2^V - \{ \emptyset \} \to \mathbf{R}$ which satisfy
%
\begin{gather*}
\max \quad\ \  -\sum (1 - |S|) y(S) \\
\begin{aligned}
\ \ \ \ \ \ \ \ \ \ \ \ \ \ \ \ \ \ \ \ \textup{s.t.} \quad \sum_{e \in E(S)} y(S) &\geq -c(e)\ \ \ \ \forall e \in E(S) \\
                   y(S) &\geq 0\ \ \ \ \ \ \ \ \ \ \ \ \forall S \neq V
\end{aligned}
\end{gather*}
%
Kruskal's algorithm, a method for finding the minimal spanning tree, can be viewed as a combinatorial method to solving the dual LP of the spanning tree problem. We first recall the simple, greedy method to form a spanning tree.
%
\begin{algorithm}
\caption{Kruskal's Algorithm}
\begin{algorithmic}[1]
\State Set $T = \emptyset$, $K = E$.
\While {$T$ is not a spanning tree}
\State Remove $e \in K$ with minimal weight.
\State Append $e$ to $T$ if it connects two connected components of $T$.
\EndWhile
\State \Return $T$
\end{algorithmic}
\end{algorithm}
%
We can verify this algorithm's correctness using the duality of linear programming. Suppose that $T$ contains the edges $e_1, \dots, e_n$, which are placed in the order they were added to $T$. Let $S_k$ be the component containing $e_k$ in the graph consisting only of the edges $e_1, \dots, e_k$ (it is the component that was freshly merged together in the $k$'th iteration of the algorithm). Let $x: E \to \{ 0, 1 \}$ be the indicator function of $T$, and $y: 2^V - \{ \emptyset \} \to \mathbf{R}$ be defined by letting $y(S_i) = c(e_j) - c(e_i)$, where $j$ is the smallest index greater than $i$ such that one of the endpoints of $e_j$ contains points in $X_i$, and define $y(S_n) = y(V) = - c(e_n)$. Define $y(S) = 0$ otherwise. For any edge $e$, by the telescoping sum property of our definition we have
%
\[ \sum_{e \subset S} y(S) = -c(e_i) \]
%
where $i$ is the smallest index such that $X_i$ contains both endpoints of $e$. The way we selected edges guarantees that $c(e_i) \leq c(e)$, so that our constraints are satisfied, and our tight for $e_1, \dots, e_n$.

We now verify complementary slackness, so that $x$ is verified optimal. If $x(e) > 0$, then our calculation above shows
%
\[ - \sum_{e \subset S} y(S) = c(e) \]
%
If $y(S) > 0$, then $S = S_i$ for some $i$, hence $S_i$ is a connected tree, and contains $n-1$ vertices, so $x(E(S)) = |S| - 1$. Complementary slackness guarantees that $x$ and $y$ are optimal solutions to their corresponding algorithms, verifying correctness of Kruskal's algorithm.

\section{Arboresences}

An (out) {\bf arboresence} in a directed graph with root $s$ is a subset $T$ of $n-1$ edges, such that there is a unique directed path from $s$ to any other vertex. Given a cost function $c \geq 0$ and vertex $s$, we want to try and find the min-cost arboresence at $s$. Like the minimal cost spanning tree problem, we can express this problem as an LP,
%
\begin{gather*}
\min \quad\ \ \sum_e c(e) x(e) \\
\begin{aligned}
\ \ \ \ \ \ \ \ \ \ \ \ \ \ \ \ \ \ \ \ \textup{s.t.} \quad x(\delta^{\text{in}}(U)) &\geq 1\ \ \ \ \forall U \subset V - \{ s \} \\
                  x &\geq 0
\end{aligned}
\end{gather*}
%
If $T$ is an arbitrary arboresence at $v$, then the corresponding characteristic function certainly is a feasible solution to the problem. The problem with this LP is that there are solutions (possibly even optimal ones) which don't look like arboresences. However, we shall not use the LP to solve linear programs, but instead use properties of linear programs to guarantee optimality. We can form a simple argument to show the existence of optimal solutions which are arboresence. First, note that we only every have upper bounds to the algorithm, and there is no need to have $x(e) > 1$ to satisfy any of the constraints, due to the positivity of $x$, so we may always assume that an optimal integral solution takes value in $\{ 0, 1 \}$, and thus correspond to characteristic functions of edges. Second, we see that the subgraph formed contains paths from $v$ to any other vertex. Since $x(\delta^{\text{in}}(V - s) \geq 1$, there is an edge connecting $s$ to some over vertex $v_1$. Then $x(\delta^{\text{in}}(V - \{ s, v_1 \}) \geq 1$, so there is some edge from $s$ to $v_1$, or $v_1$ to $v_2$. In either case, there is an edge to $v_2$. Continuing this process gives you a path to any vertex in the graph. We may now assume our subgraph is a tree, because we can always remove edges to obtain an algorithm that is at least as optimal. Thus we have argued that there is an optimal solution which is an arboresence, like we are looking for.

Let us form the dual LP, so we can use complementary slackness to verify optimality.
%
\begin{gather*}
\max \quad\ \ \sum_S y(S) \\
\begin{aligned}
\ \ \ \ \ \ \ \ \ \ \ \ \ \ \ \ \ \ \ \ \textup{s.t.} \quad \sum_{e \in \delta^{\text{in}}(S)} y(S) &\leq c(e)\\
                  y &\geq 0
\end{aligned}
\end{gather*}
%
We now describe the algorithm which gives us minimal cost arboresences. Note that $y$ is only included here for purposes of algorithmic correctness, and needn't be coded up in real algorithms.

\begin{algorithm}
\caption{Edmond's Algorithm}
\begin{algorithmic}[1]
\State $y_{\{ v \}} = \min c(uv)$, for $v \neq r$.
\State Set $F = \{ e_w : w \in V - \{ r \} \}$, where $e_w$ is the cheapest edge entering $w \neq r$.
\While {$F$ has a cycle}
\State Let $G'$ be the graph obtained by contracting each cycle in $(G,F)$.
\State Define $c'(uv) = c(uv) - y(\{v\})$ if $v$ is on a cycle, else $c'(uv) = c(uv)$.
\State Let $(T',y')$ be a min-cost arboresence in $G'$.
\State Define $T = (F \cup T') - \{ uv : u\ \text{on a cycle} wv \in T' \}$.
\State Set $y(S) = 0$ if $S$ cuts a cycle, and $y'(S)$ if $S = [S]$ after contraction.
\EndWhile
\State \Return $F$
\end{algorithmic}
\end{algorithm}

* DRAW A PICTURE OF CONTRACTING CYCLES

\chapter{Matroids}

If you have a problem, and a matroid is involved, you can probably use a greedy algorithm to solve the problem. This is very useful, because greedy solutions are normally far faster than conservative methods.  Formally, a matroid is a collection of finite subsets of some set $X$ containing the empty set, closed under the subset relation, and satisfying the {\bf exchange property}:
%
\begin{itemize}
    \item If two sets $A$ and $B$ are in the matroid with $|A| < |B|$, there is $z \in B - A$ such that $A \cup \{ z \}$ is in the matroid.
\end{itemize}
%
A matroid is thus a set of sets which can be `greedily expanded' to obtain an optimal set. We shall call sets in a matroid {\bf independent sets}, to distinguish them from other subsets of $X$.

\begin{example}
    If $G$ is a graph with vertices $V$, the set of edge subsets which contain no cycle forms a matroid. Certainly it contains the empty set, and a subset of edges with no cycle contains no cycle. If $A$ and $B$ are subsets of edges with no cycle, and $|A| < |B|$, then $(V,A)$ has more connected components than $(V,B)$, so there must be an edge bridging two of the connected components of $(V,A)$, and this allows us to expand $A$ while making sure there are no cycles in the resulting graph. This verifies the exchange property.
\end{example}

\begin{example}
    If $V$ is a finite dimensional vector space, the collection of linearly independent vector sets forms a matroid. Essentially, you can greedily expand vectors to form a basis.
\end{example}

\begin{example}
    If $X$ is a finite set, and $L$ is a laminar family over $X$, together with a function $f \geq 0$ defined on the Laminar family, then the same of subsets $A$ such that $|A \cap B| \leq f(B)$ for each $B \in L$, forms a matroid known as a laminar family.
\end{example}

\begin{example}
    If $G$ is a directed graph, and $A, B \subset V$ are given, the set of subsets of $A$ which contains vertex disjoint paths into $B$ is a matroid, known as a Gammoid.
\end{example}

\begin{example}
    Given a connected graph $G$, the set of bond matroid.
\end{example}

\begin{example}
    Given a field extension $F \subset K$, the set of finite subsets $Y$ of $F$ with trancendence degree $|Y|$ forms a matroid, known as the algebraic matroid.
\end{example}

An independence oracle is a polynomial time computable algorithm to decide whether an arbitrary subset is an element of a given matroid. This is key to having efficient algorithms for solving matroids, because we will often form matroids over sets with exponentially many elements in order to solve a problem.

A {\bf base} of a matroid is a maximal set. Two bases in a matroid have the same cardinality, and this is essentially the defining feature of a matroid. Indeed, if $B$ and $B'$ are two bases with $|B| < |B'|$, there is $z \in B' - B$ such that $B + z$ is in the matroid, which cannot occur since this is an expansion of $B$. In the matroid of no cycle subsets of edges, a base is a spanning tree. In the matroid of linearly independent vectors over some vector space, a base is a basis for that vector space. It is a basis property of spanning trees that they all have $|V| - 1$ edges, and that every basis of a vector space has the same cardinality.

\section{Minimum Weight Bases}

The first algorithm we consider in the theory of matroids is to efficiently find a minimum weight base for a matroid over a set $X$, given some cost function $c: X \to \mathbf{R}$. This is a generalization of the minimum spanning tree problem, and like the greedy solution to the minimum spanning tree problem, the algorithm to find a minimum weight base is to just greedily add the minimum element of $X$ to a set until we cannot anymore. The set $B$ we end up with is a base by construction, because the order we add elements is immaterial. To prove that $B$ is minimal, consider a minimal base $B^*$, and order the elements of both sets
%
\[ B = \{ x_1, \dots, x_n \}\ \ \ \ \ B^* = \{ x_1^*, \dots, x_n^* \} \]
%
in decreasing order of weight. We prove by induction that $w(x_i) \leq w(x_i^*)$. Indeed, consider the partial set $B_k = \{ x_1, \dots, x_k \}$, and $B_k^* = \{ x_1^*, \dots, x_k^* \}$. If $c(B) > c(B^*)$, there must be a smallest $k$ with $c(x_{i_k}) > c(x_{i_k}^*)$. Then $B_{k-1}$ contains fewer elements than $B_k^*$, so there is $x \in B_k^* - B_{k-1}$, where $B_{k-1} \cup \{ x \}$ is an independant set. But then $c(x_{i_k}^*) \geq c(x) \geq c(x_{i_k})$, which is a contradiction, proving the correctness of the algorithm.

\section{Rank Functions}

The {\bf rank} $r(A)$ of a subset $A$ of $X$, with respect to a matroid $M$ is the cardinality of the largest independent subset of $A$. Then $r$ is order preserving, and $r(A) \leq |A|$. More interestingly,
%
\[ r(A) + r(B) \geq r(A \cup B) + r(A \cap B) \]
%
To prove this, let $J$ be a maximal independent subset of $A \cap B$, with $K \subset A - B$ a complementary subset of $J$, so that $K \cup J$ is a maximal independent subset of $A$. Do the same for $J$ in $B$, forming the independet set $L$. Then
%
\[ r(A) = |K| + |J|\ \ \ \ r(B) = |K| + |L| \]
\[ r(A \cap B) = |K|\ \ \ \ r(A \cup B) \leq |K| + |L| + |J| \]
%
where the last inequality follows because if $C$ is a maximal independent subset of $A \cup B$, then
%
\[ |C \cap (A-B)| \leq |K|\ \ \ |C \cap (B-A)| \leq |L|\ \ \ |C \cap (A \cap B)| \leq |J| \]
%
It turns out that the rank function of a matroid uniquely determines the matroid, but we won't need this property. This inequality is the {\bf submodularity} property of the rank function. This is equivalent to the statement if $B$ is a subset of $A$, and $x$ is not an element of $A$, then $r(B \cup \{ x \}) - r(B) \geq r(A \cup \{ x \}) - r(A)$, which is a kind of `dimiminishing retursn' property for the function $r$.

A subset $C$ of $X$ is a {\bf circuit} if it is not independent, but removing any particular element of $C$ makes it independent. The circuits in a graph are exactly the circuits in the matroid of no-cycle edge sets. If $Y$ is any independent set, and $Y \cup \{ x \}$ is {\it not}   independent, then there is a unique circuit contained in $Y \cup \{ x \}$, and we denote this circuit by $C(Y,x)$. To find this circuit, consider a minimal element of $Y \cup \{ x \}$ which is not independent. It is easy to see that this set is a circuit. The hard part is showing this is the only circuit. Let $C_1$ and $C_2$ be two circuits. If $C_1 \neq C_2$, let $y \in C_1 - C_2$ (we know one circuit cannot be a proper subset of the other). What's more, $x \in C_1 \cap C_2$, because $Y$ is an independent set, and therefore contains no circuits. We will show there is a circuit contained in $(C_1 \cup C_2) - \{ x \} \subset Y$, contradicting the fact that $Y$ is an independent set. Indeed, using submodularity we find
%
\begin{align*}
    [|C_1| - 1]& + [|C_2| - 1] + r(C_1 \cup C_2 - \{ x, y \})\\
    &= r(C_1) + r(C_2) + r(C_1 \cup C_2 - \{ x, y \})\\
    &\geq r(C_1) + r(C_2 - \{ x \}) + r((C_1 \cup C_2) - \{ y \} )\\
    &\geq r(C_1 - \{ y \}) + r(C_2 - \{ y \}) + r(C_1 \cup C_2)\\
    &= [|C_1| - 1] + [|C_2| - 1] + r(C_1 \cup C_2)
\end{align*}
%
The point of this calculation is that now we know $r(C_1 \cup C_2) = r(C_1 \cup C_2 - \{ x, y \})$, so that every base in $C_1 \cup C_2 - \{ x, y \}$ is a base in $C_1 \cup C_2$. In particular, this means that $C_1 \cup C_2 - \{ x \}$ is not independent, so there must be a circuit contained in the set.

That there is always a unique circuit `screwing us over' is very useful in analyzing the problem of matroid intersection. Given two matroids $M$ and $N$ over a set $X$, the intersection problem aks us to find the maximal set which is contained in both matroids. Similarily, we may have a subset $Y$ independent in both $M$ and $N$, and we wish to know whether we can extend $Y$ while preserving independence in both sets. Consider the following algorithm: given $Y$, we construct a directed graph with nodes in $Y$ and $X - Y$, and directed edges $ab$ (for $a \in Y$, $b \in X - Y$) if $Y - a + b \in M$, and $ba$ if $Y - a + b \in N$. We find a shortest path from $Y_1$ to $Y_2$ where
%
\[ Y_1 = \{ x \in X - Y : Y + x \in M \}\ \ \ \ \ \ Y_2 = \{ x \in X - Y: Y + x \in N \} \]
%
and we then augment along that path to find a better matroid. If there is no path from $Y_1$ to $Y_2$, we have a maximal matroid.

\section{Linear Programs for Matroids}

Let $M$ be a matroid over a set $X$, together with a weight function $w: M \to \mathbf{R}$. We can formulate the maximum weight independent set problem over $X$ as a linear program. Specifically, we consider the space of functions in $\mathbf{R}^X$, and take the polytope consisting of functions $x: X \to \mathbf{R}^+$ such that $x(A) \leq r(A)$ for each subset $A$ of $X$. Note that these conditions guarantee that $0 \leq x \leq 1$. The indicator function of every independent set is contained in this polytope, and correspondingly, 0-1 valued points correspond to independent sets in the matroid. If $x = \chi_B$, then $B = x(B) \leq r(B) \leq B$, so $r(B) = B$, and so $B$ is independent. We call this the matroid polytope.

The maximum weight independent set asks us to maximize $\sum w(e) x(e)$ over integral points of the matroid polytope. It turns out that all extreme points of the matroid polytope are integral, so we can solve the problem in polynomial time. Our proof will use an interesting technique known as the uncrossing method, which `thins' the constraints of the LP so they correspond to a Laminar family (If we say $A$ and $B$ are crossed if $A \cap B \neq \emptyset$, then by breaking them into disjoint chains we have `uncrossed them'). First, we embed $2^X$ in $\mathbf{R}$ by mapping $A \subset X$ to the $0/1$ indicator function $\chi_A$. Then we let $\mathcal{C}$ to be the largest linearly ordered subset of $2^X$ (a {\it chain} in $2^X$) only containing subsets $A$ for which we have a tight constraint, $x(A) = r(A)$, and such that the $\chi_A$ are independent. We claim that the $\chi_A$ span $\mathbf{R}^X$.

Next, note that the functionals $x \mapsto x(A)$ are essentially the duals of the characteristic function $\chi_A$, and since $2^X$ spans $\mathbf{R}^X$, the rows of the constraint matrix which $x$ satifies span $\mathbf{R}^X$. In particular, since $x$ is extreme, the set of tight constraints spans $\mathbf{R}^X$, and it suffices for us to prove that $\text{span}(\mathcal{C})$ contains all $\chi_A$ such that $x(A) = r(A)$. Suppose that $\chi_B$ was not in $\text{span}(\mathcal{C})$, where $r(B) = x(B)$. If such a $B$ exists, we may choose $B$ such that the number of elements is minimized
%
\[ \tau(B) = \{ A \in \mathcal{C} : A \not \subset B\ \text{and}\ B \not \subset A \} \]
%
is minimized. Note that we must have $\tau(B)$ must contain at least one element, or else $\mathcal{C} \cup \{ B \}$ would be a chain of linearly independent vectors which is larger than $\mathcal{C}$. So let $T \in \mathcal{C}$ be such that $B \not \in T$ and $T \not \in B$. Note that by using the inclusion/exclusion principle, optimality, and the submodularity property,
%
\begin{align*}
    r(B) + r(T) &= x(B) + x(T)\\
    &= x(B \cup T) + x(B \cap T)\\
    &\leq r(B \cup T) + r(B \cap T)\\
    &\leq r(B) + r(T)
\end{align*}
%
So that $x(B \cup T) + x(B \cap T) = r(B \cup T) + r(B \cap T)$, and since $x(B \cup T) \leq r(B \cup T)$, $x(B \cap T) \leq r(B \cap T)$, we can actually split this inequality into the equalities $x(B \cup T) = r(B \cup T)$, $x(B \cap T) = r(B \cap T)$.

It turns out that $\tau(B \cup T), \tau(B \cap T) \subsetneq \tau(B)$, which by optimality implies $\chi_{B \cup T}, \chi_{B \cap T} \in \text{span}(\mathcal{C})$. This is clearly in contradiction to our assumption, since then
%
\[ \chi_B = \chi_{B \cup T} + \chi_{B \cap T} - \chi_T \in \text{span}(\mathcal{C}) \]
%
so all that remains is to prove properties of the $\tau$ function. This is most easily shown by drawing out Venn diagrams, which is often useful in matroid theory, but we'll also provide a textual description (and drawing your own diagram will probably be much more revealing than looking at someone elses).

Consider $S$ in $\mathcal{C}$. Then either $S \subset T$, or $T \subset S$. If $S \subset T$, then $S \subset B \cup T$ and so $S \not \in \tau(B \cup T)$. If $T \subset S$, then $B \cap T \subset S$, so $S \not \in \tau(B \cap T)$. If $S \in \tau(B \cup T)$ but $S \not \in \tau(B)$, then we would have to have $T \subset S$. This also shows $S \subset B$, because if $B \subset S$, then $B \cup T \subset S$. But then $T \subset B$, which is a contradiction. Similarily, if $S \in \tau(B \cap T)$ but $S \not \in \tau(B)$, then $S \subset T$ and $B \subset S$, so $B \subset T$, another contradiction. We finish this conclusion by noticing that $T \not \in \tau(B \cap T), \tau(B \cup T)$ by obvious relations.

Thus we conclude that $\mathcal{C}$ forms a basis of $\mathbf{R}^X$ (in fact, $\mathcal{C}$ forms a basis for $\mathbf{Z}^X$ -- the same proof follows through since we've never used real numbers, but we won't need this in our proof. Now our homework comes in handy. Consider the laminar constrainted matching problem, over the Bipartite graph whose left vertices are subsets of $X$, and with only one right vertex, with an edge between every left vertex. We consider $L_L = \mathcal{C}$ as a laminar family over the left vertices, and $L_R = \emptyset$, with $b_A = r(A)$ for $A \in \mathcal{C}$. Every solution to this problem can be identified with a characteristic function over the vertices, and by the identification $x$ can be seen as a feasible solution to the Laminar constraints. Since $x$ is tight for {\it all} constraints, it must be a feasible solution to this problem, and in our homework we proved that therefore $x$ has integral coordinates.

The matroid intersection problem can be formulated as an LP with integral extreme points in a very similar manner. We consider two matroids $M$ and $N$, take the same solution space as the maximal independent set LP, and find solutions such that
%
\begin{gather*}
\max \quad\ \ \sum_{e \in X} w(e) x(e) \\
\begin{aligned}
\ \ \ \ \ \ \ \ \ \ \ \ \ \ \ \ \ \ \ \ \textup{s.t.} \quad x(A) &\leq \min(r_M(A), r_N(A))\ \ \ \ \forall A \subset X \\
                  x &\geq 0
\end{aligned}
\end{gather*}
%
The extreme points of this set are integral, which can be verified from the uncrossing technique of the last problem. Here we will end up with a laminar family of tight constraints $\mathcal{C}_M$ and $\mathcal{C}_N$ over each matroid, rather than a chain,prove that $\text{span}(\mathcal{C}_M \cup \mathcal{C}_N)$ is full, and then use the laminar family $L_L = \mathcal{C}_M$, $L_R = \mathcal{C}_N$ to prove integrality. Our homework problem does not generalize to intersections of three or more matroids, and for good reason! The matroid intersection problem for more than 2 matroids is NP complete.

\chapter{Submodular Functions}

A {\bf submodular function} is a function $f:2^\Omega \to \mathbf{R}$, such that
%
\[ f(S) + f(T) \geq f(S \cup T) + f(S \cap T) \]
%
One algorithm that occurs across computing science is to find a maximum set for a submodular function. Consider the greedy approximation which constructs a set by taking the point which increases the current set by the largest amount. If $Z$ is the set constructed, and $Z^*$ is an optimum set, we claim that $f(Z) \geq (1 - 1/e) f(Z^*)$ -- this is the best approximation one can do in polynomial time, unless $P = NP$.

\begin{lemma}
    \[ f(Z^*) \leq k (f(z) - f(z)) \]
\end{lemma}
\begin{proof}
    \[ f(Z^*) \leq f(Z^* \cup Z_{i-1}) \]
\end{proof}

\chapter{Perfect Graphs}

There are many numbers of interest associated to graphs. For instance, we have $\alpha(G)$, the size of the maximum independent set, $\omega(G)$, the maximum clique size, $\chi(G)$, the chromatic number, and $k(G)$, the clique cover number (the minimum number of cliques required to cover the vertices of $G$). A graph is {\bf perfect} if $\chi(H) = \omega(H)$ for every subgraph of $G$. For each graph, we associate the polytope
%
\[ P_{\text{clique}}(G) = \{ x \in \mathbf{R} \langle V \rangle : x \geq 0, x(C) \leq 1,\ \forall\ \text{cliques}\ C: x(C) \leq 1 \} \]
%
We claim this polytope is integral if and only if $G$ is perfect.

Perfect graphs include the families of bipartite graphs and chordal graphs. Given a graph $G$, define the complement of $G$ to be the graph $\overline{G}$ obtained by `flipping edges'. Then
%
\[ \alpha(G) = \omega(\overline{G})\ \ \ \chi(G) = k(\overline{G}) \]
%
It is an important theorem of Lovatz, proved in 1972, that a graph is perfect if and only if its complement is perfect. Beyond the scope of this course is the proof that $G$ is perfect if and only if $G$ and $\overline{G}$ contain no cyclic graphs of size $\geq 5$ (the strong perfect graph theorem).

To prove Lovatz's theorem, consider the graph $G^{+v}$, for $v \in V$, obtained by adding a new vertex connected to ever vertex but $v$.

\begin{lemma}[Replication Lemma]
    $G$ is perfect if and only if $G^{+v}$ is perfect.
\end{lemma}
\begin{proof}
    The lemma is trivial if $G$ has one vertex. For $|V| \geq 2$, it suffices to check $\chi(G^{+v}) = \omega(G^{+v})$. It is always true that $\alpha(G^{+v}) = \alpha(G) + 0/1$. If $\alpha(G^{+v}) = \alpha(G)$
\end{proof}

So suppose $G$ is perfect. Let $x \in P_{\text{clique}}(G)$ be an extreme point. From what we know in class, we know $x \in \mathbf{Q}$. Thus there is an integer $\eta \neq 0$ such that $\eta x \in \mathbf{Z}$. So consider $H$, obtained by copying $G$ $\eta$ times, and then connecting all edges $(u_i,v_i)$, if $u = v$ or $(u,v)$ is an edge in the original graph. The lemma above implies $H$ is perfect. Let $C_H$ be a max clique in $H$, with $|C_H| = \omega(H)$.

\chapter{Polynomials}

In this chapter, we discuss various computational problems which occur in the theory of polynomials. We will begin with the algebraic operations of addition and multiplication of polynomials. The naive addition operation is asymptotically optimal, but nontrivial techniques in the harmonic analysis of abelian groups allow us to compute multiplication in a much faster way than the naive multiplication algorithm.

First, we discuss the various ways we can represent polynomials computationally. The standard way to represent a degree $n$ polynomial $\sum a_k X^k$ is as a collection of $n+1$ numbers $a_0,a_1,\dots,a_n$

As we are working over finite dimensional spaces, the Fourier transform can be given a matrix representation. First, note that there is a one to one correspondence between functions $a: \mathbf{Z}_n \to \mathbf{C}$, and polynomials $f(X) = (1/n) \sum_{k = 0}^{n-1} a(k) X^k \in \mathbf{C}[X]$ of degree $n-1$. If $z$ is a primitive $n$'th root of unity, then $f(z^{-l}) = (1/n) \sum_{k = 0}^{n-1} a(k) z^{-kl}$ gives the $l$'th Fourier coefficient of the function $a$. Thus the Fourier transform is just the values of the polynomial at the $n$'th roots of unity, and if we consider the Fourier transform with respect to the basis $1, X, X^2, \dots, X^{n-1}$, then the matrix representation of the transform is given by the Vandermonde matrix
%
\[ \begin{pmatrix} 1 & 1 & 1 & \dots & 1 \\ 1 & z^{n-1} & z^{n-2} & \dots & z \\ 1 & z^{n-2} & z^{n-4} & \dots & z^2 \\ \vdots & \vdots & \vdots & \ddots & \vdots \\ 1 & z & z^2 & \dots & z^{n-1} \end{pmatrix} \]
%
which is the standard trick to evaluate the polynomial as a matrix. The Fourier inversion formula tells us that the inverse of the Vandermonde matrix, which represents the matrix of the inverse Fourier transform, is
%
\[ (1/n) \begin{pmatrix} 1 & 1 & 1 & \dots & 1 \\ 1 & z & z^2 & \dots & z^{n-1} \\ 1 & z^2 & z^4 & \dots & z^{n-2} \\ \vdots & \vdots & \vdots & \ddots & \vdots \\ 1 & z^{n-1} & z^{n-2} & \dots & z \end{pmatrix} \]
%
Essentially, this implies that a slight modification of the fast Fourier transform obtained by swapping the root $\nu$ used to calculate the transform with $\nu^{-1}$ gives us the inverse Fourier transform, provided we multiply the result by $1/n$ at the end.

Continuing with the computational aspect of discrete Fourier analysis, we now use the fast Fourier transform to multiply two polynomials $f(X) = \sum a_k X^k$ and $g(X) = \sum b_k X^k$, where $f(X)g(X) = \sum c_k X^k$. We can write $c_k = \sum a_i b_{k-i}$, and computing these values for all coefficients $c_k$ takes $O(n^3)$ operations, where $n$ is the degree of the polynomial. However, the definition of $c_k$ involves convolution, which tells us Fourier analysis is probably applicable. However, to employ the fast Fourier transform, we have to compute convolution in $\mathbf{Z}_m$, which is done via modulo arithmetic, the values of the polynomial are done via convolution over $\mathbf{N}$. The trick is that these two convolutions will be equal provided we set $m$ large enough, and pad the values of the polynomials with zeroes, so that $a_i = b_i = 0$ for $m > i > n$. Consider the convolution $\sum a_i b_{k-i}$, where $i$ and $k-i$ are taken as elements of $\mathbf{Z}_m$. Certainly every term defining $c_k$ occurs in this sum, so we need only choose $m$ such that we can add no additional terms, which will occur when $i > k$. These terms can be written as
%
\[ \sum_{i = k+1}^{m-1} a_i b_{k-i} = \sum_{i = 1}^{m-k-1} a_{k+i} b_{m-i} \]
%
where the right side sum now can be computed with considering any modulo tricks. If $a_{k + i} b_{m-i} \neq 0$, then $a_{k+i} \neq 0$, hence $0 \leq k+i \leq n$, and also $b_{m-i} \neq 0$, so $0 \leq m-i \leq n$. Adding together both these inequalities gives $0 \leq m+k \leq 2n$, hence $0 \leq m \leq 2n$. If we choose $m > 2n$, then this problem is avoided, and we can compute $c_k$ as convolution modulo $m$. Thus $c = a * b$, and since the Fourier transform of $a*b$ is the pointwise product $\hat{a} \hat{b}$, we can compute the fast Fourier transform of $a$ and $b$ in $O(m \log m)$ operations, multiply them together with $m$ operations, and then compute the inverse Fourier transform in $O(m \log m)$ operations, giving us the values of $c$ in $O(m \log m) = O(n \log n)$ operations.

Another important use of the fast Fourier transform is to compute multiplication of two numbers. The naive algorithm runs in $O(\log^2 n)$ time, for numbers of size $\leq n$. However, we can use the fast Fourier transform to reduce the computation to $O(\log (n) \log \log (n) \log \log \log (n))$ (and all the logs are important, those the third iteration of the logarithm has almost no effect on the speed of the algorithm for the numbers we deal with (even $x = e^{1000000}$ has $\log \log \log x \leq 1$). Given two $n$ digit numbers $x$ and $y$, write
%
\[ x = \sum_{k = 0}^n a_k 2^k\ \ \ \ \ y = \sum_{k = 0}^n b_k 2^k \]
%
Then
%
\[ xy = \sum_{k = 0}^n \left( \sum_{i = 0}^k a_i b_{k-i} \right) 2^k \]
%
s

\part{Approximation Algorithms}

\chapter{Set Cover}

In the set cover problem, we are given a particular set $X$, and a family of subsets whose union is $X$, so that the sets {\it cover} $X$. Each set $S$ is associated a particular weight $w_S \geq 0$, and the goal is to find the cheapest family of subsets which cover $S$. The case where $w_S = 1$ for all sets $S$ is known as the unweighted version of the set cover problem. The problem is easily seen to be NP hard, because it generalizes the vertex cover problem, which asks, given a graph, to find a subset of vertices which contain one end to every edge. In this chapter, we consider the problem of finding good polynomial time computable heuristics to the set cover problem which give hard bounds on the approximation ratio of returned solutions.

\section{Approximations Via Linear Programming}

Unless we doubt that $\mathbf{P} \neq \mathbf{NP}$, we should not hope to be able to express the set cover problem as a linear program in polynomially many dimensions, because then we could solve the set cover problem in polynomial time (using some linear programming algorithm, like the ellipsoid method). However, we can express the set cover problem as a 0-1 {\it integer} linear program. Theoretically, this is guaranteed, because the problem is $\mathbf{NP}$ complete, but we shall find a {\it natural} integer linear program for set cover which provides inspiration for approximations. Our choice of including $S$ in some solution to the set cover problem can be taken as some binary variable $x_S \in \{ 0, 1 \}$. The constraint that the union of the $S$ chosen contains every element in $S$ can be expressed that for each $a \in X$, $\sum_{a \in S} x_S \geq 1$. The value of such a cover will be $\sum x_S w_S$, and we aim to minimize this value. Thus we have reformulated the set cover problem as a particular instance of an integer linear program. This is, of course, impossible to solve in polynomial time unless $\mathbf{P} = \mathbf{NP}$.

If we remove the constraint that $x_S \in \{ 0, 1 \}$, and instead add the constraint that $0 \leq x_S \leq 1$, then we can solve the program in polynomial time. This is known as the {\bf relaxation} of the program. However, the solutions we obtain will no longer be $\{ 0, 1 \}$ valued, and will therefore not correspond to solutions to the original set cover problem. There are many ways of interpreting these solutions, and they lead to many different techniques used in the theory of approximation algorithms. For now, we consider these solutions as approximations to actual solutions to the set cover. If we can `snap' the solution to a $\{ 0, 1 \}$ valued solution, while bounding the value of the solution in terms of the optimal solution to the linear program, then we obtain an approximation algorithm to the set cover problem, because the optimal solution to the relaxation of the linear program is always going to lower bound the optimal solution to the set cover problem. A technique of snapping a solution to a proper solution is known as a {\bf rounding algorithm}.

Thus we consider an optimal solution $0 \leq x^* \leq 1$ to the relaxation of the set cover program. We begin by setting
%
\[ M = \max_{a \in X} |\{ S : a \in S \}| \]
%
to be the maximal number of sets which contain a particular element $a$, over all elements $a \in X$. We note that since $\sum_{a \in S} x^*_S \geq 1$ for each $x$, then $\mathbf{E}[x^*_S | x \in S] \geq 1/M$, and so there must exist a set $S$ with $a \in S$, and $x^*_S \geq 1/M$ (there must exist a value in a list of values greater than the average value of the list). This implies that we can obtain a feasible solution $x$ to the $\{ 0, 1 \}$ valued program by setting $x_S = 1$ if $x^*_S \geq 1/M$. This gives a bounded approximation to the solution.

\begin{lemma}
    $\sum w_S x_S \leq M \sum w_S x^*_S$.
\end{lemma}
\begin{proof}
    If $x_S = 1$, then $x_S^* \geq 1/M$, and so $x_S \leq Mx_S^*$, and therefore
    %
    \begin{align*}
        \sum w_S x_S \leq \sum w_S Mx_S^* = M \sum w_S x_S^*
    \end{align*}
    %
    and this gives the required inequality.
\end{proof}

It follows that this rounding algorithm produces an $M$ approximation algorithm to the set cover problem, since $\sum w_S x^*_S$ lower bounds the optimal solution to the set cover problem. In the case of vertex cover, each edge is covered by two vertices, so $M = 2$, and we obtain a 2-approximation algorithm. Assuming a result known as the Unique Games Conjecture, which is still an open problem in computing science, this is the best possible approximation we can achieve in polynomial time, unless $\mathbf{P} = \mathbf{NP}$.

\section{A Dual Formulation}

Often, it is useful to consider the dual of a linear program to verify optimality. In particular, we can use it to help with constructing rounding algorithms. Given the linear program relaxation to set cover, one can calculate that the dual of the set cover is to maximize $\sum y_a$, such that for each set $S$, $\sum_{a \in S} y_a \leq w_S$. One can think of the values $y_a$ as assigning a `cost' to each element of the set we are attempting to cover, and not only can we conclude that $y_a \leq \min_{a \in S} w_S$, so that every partial cover of $X$ includes $a$ must cost at least $y_a$, but also every partial cover of $X$ containing a subset $T$ must cost at least $\sum_{a \in T} y_a$. In particular, the total sum $\sum y_a$ lower bounds the optimal set cover solution. The theory of LP duality tells us that if $y^*$ optimize the dual LP, then $\sum y^*_a = \sum x^*_S$, where $x^*$ optimizes the linear relaxation to set cover.

Now consider the following algorithm. We construct an optimal solution $y^*$ to the dual program, and then construct a set cover $x$ where $x_S = 1$ if the dual constraint corresponding to $S$ is tight for $y^*$, i.e. $\sum_{a \in S} y^*_a = w_S$. In terms of our interpretation of the dual LP, we can interpreting this as choosing a set $S$ which gives us `the best deal for our money' relative to the costs $y^*$ for each element. It is a set $S$ such that $w_S$ is {\it exactly} the lowest cost cover of $S$. This gives us a set cover, because if $\sum_{a \in S} y^*_a < w_S$ for each set $S$ containing a particular element $b$, then we can increase $y^*_b$ while fixing all values, giving us a more optimal solution. What's more, it's weight is bounded.

\begin{lemma}
    $\sum w_S x_S \leq M \sum y_a^*$
\end{lemma}
\begin{proof}
    Because each element $a$ is in at most $M$ sets $S$,
    %
    \[ \sum w_S x_S = \sum_S x_S \sum_{a \in S} y^*_a \leq M \sum y^*_a \]
    %
    and this is the lower bound required.
\end{proof}

As we have argued, $\sum y_a^*$ gives us the lower bound to the set cover problem, and therefore $x$ gives an $M$ approximation to the set cover problem.

\section{Using Duals to Discretize LP Approximations}

The dual algorithm is always worse than the original LP approximation. We know that $\sum y_a \leq \sum w_S x_S$ for any feasible $y$ in the dual, and $x$ in the original LP. What's more, $\sum w_i x_i^* = \sum y_x^*$. Because of the theory of complementary slackness, we also know that $y_a^* > 0$ implies that $\sum_{a \in S} x_S^* = 1$, and whenever $x_S^* > 0$, then $\sum_{a \in S} y_a^* = w_S$. In the first rounding algorithm we considered, we include a set in our cover if $x_S^* \geq 1/M$, and complementary slackness implies that the set is included in the cover determined by the second algorithm. Thus the second algorithm always contains the first algorithms solution as a subcover, and is guaranteed to be less optimal.

However, the dual LP is still useful, because we can use it in tandem with the original LP to obtain a combinatorial approximation algorithm for set cover. These type of algorithms are known as {\bf dual primal algorithms}. This is analogous to the combinatorial algorithms for bipartite matching and max flow, which can be obtained by combinatorializing the linear program formulations of the problem. The algorithm consists as follows: begin with some feasible solution $y$ to the dual program, and consider some family of sets $S$ such that the corresponding constraint to $S$ is tight. If this family is a set cover, then we are done. Otherwise, there is some $a$ not covered by the family of sets, hence $y_a$ is not tight, and we may increase $y_a$ until it makes some dual constraint corresponding to some set $S$ tight, and we may add the set to our family of sets. Eventually we obtain a set cover, and the same argument for the dual algorithm implies that this solution is also an $M$ approximation (the argument depended very little on the fact that $y$ was optimal). We know from the last argument that this will always give worse results than the original algorithm, but it is likely much faster in practice to compute this solution than to optimize the LP formulation of set cover. What's more, we know of no better approximation bound for the other LP, so the other algorithm has no other guarantee of a `noticable' improvement asymptotically.

\section{A Greedy Solution}

We now present a natural greedy heuristic to set cover, which actually leads to a much better approximation to set cover. At each point in the algorithm, we have a partial set cover which covers some subset $T \subset S$ of all the points in the set. If $T \neq S$, then we add the set $S$ to the cover which maximizes the ratio
%
\[ \frac{w_S}{|S - T|}  \]
%
which is the `best value for the money' at the current point in time. If $H_n = \sum_{k = 1}^n k^{-1}$ is the $n$'th harmonic number, then we claim this algorithm gives a $H_n$ approximation algorithm to the set cover problem, where $n$ is the number of elements in $X$.

\begin{lemma}
    Given positive numbers $a_1, \dots, a_n$ and $b_1, \dots, b_n$,
    %
    \[ \min (a_i/b_i) \leq \frac{\sum a_i}{\sum b_i} \leq \max (a_i/b_i) \]
\end{lemma}
\begin{proof}
    We note that since the minimum of a set of elements is upper bounded by the average of the elements, and the maximum is lower bounded by the average, we conclude that
    %
    \[ \min(a_i/b_i) \leq \frac{\min(a_i)}{\max(b_i)} \leq \frac{\sum a_i/n}{\sum b_i/n} = \frac{\sum a_i}{\sum b_i} \]
    %
    and
    %
    \[ \max(a_i/b_i) \geq \frac{\max(a_i)}{\min(b_i)} \geq \frac{\sum a_i/n}{\sum b_i/n} = \frac{\sum a_i}{\sum b_i} \]
    %
    and this gives the inequality.
\end{proof}

Suppose the greedy algorithm takes $m$ iterations to form a cover. Let
%
\[ n = n_1 > n_2 > \dots > n_m > n_{m+1} > 0 \]
%
be the number of uncovered elements at the beginning of the individual iterations of the algorithm, let
%
\[ \emptyset = T_0 \subsetneq T_1 \subsetneq T_2 \subsetneq \dots \subsetneq T_m = X \]
%
denote the set of covered elements. For each set $S$, let $S_k = S - T_k$ denote the number of unconvered elements in $S$ at the beginning of the $k$th iteration. On the $k$th iteration, the greedy algorithm therefore takes the set $S$ minimizing $w_S/|S_k|$.

\begin{lemma}
    If $\text{OPT}$ is the value of the optimal set cover solution, then
    %
    \[ w_S \leq \frac{n_k - n_{k+1}}{n_k} \cdot \text{OPT} \]
\end{lemma}
\begin{proof}
    We first find that
    %
    \[ \min \left( \frac{w_S}{|S_k|} \right) \leq \frac{\sum w_S}{\sum |S_k|} \leq \frac{\text{OPT}}{n_k} \]
    %
    If we add $S$ to our set, then $n_{k+1} = n_k - |S_k|$, and so
    %
    \[ w_S \leq |S_k| \frac{\text{OPT}}{n_k} = \frac{n_k - n_{k+1}}{n_k} \text{OPT} \]
    %
    This gives the inequality.
\end{proof}

It then follows that the resulting solution we find satisfies
%
\begin{align*}
    \sum w_S &\leq \text{OPT} \sum_{k = 1}^m \frac{n_k - n_{k+1}}{n_k}\\
    &\leq \text{OPT} \sum_{k = 1}^m \left( \frac{1}{n_k} + \frac{1}{n_k - 1} + \dots + \frac{1}{n_{k+1} + 1} \right)\\
    &= \text{OPT} \sum_{i = 1}^n \frac{1}{i} = H_n \cdot \text{OPT}
\end{align*}
%
And therefore we obtain the desired approximation constant.

\section{Dual Fitting}

The $H_n$ algorithm is essentially an asymptotically optimal approximation algorithm for set cover in terms of the size of the input. If we can prove that there is a $c \lg n$ approximation algorithm for set cover, for any $c > 0$, then there is an $O(n^{O(\log \log n)})$ time algorithm for any problem in {\bf NP}, implying {\bf P} = {\bf NP}. However, we can actually show that in terms of another parameter related to the problem, the solution gives a better approximation than the bound we found above.

The theory of dual LPs returns again to allow us to slightly tighten the approximation ratio of the set cover algorithm we considered just previously. If $N$ bounds the cardinality of each set $S$, we claim that the algorithm above gives a $H_N$ approximation. In particular, we show that
%
\[ \sum w_S \leq H_N \cdot \text{OPT}_{\text{LP}} \]
%
where $\text{OPT}_{\text{LP}}$ is the optimal value of the LP relaxation of the set cover problem. To find this solution, we use the technique of dual fitting. The idea is that if $\sum w_S = \sum y_a$ for some {\it infeasible solution} $y$ to the dual of the set cover problem, but where $y/H_m$ {\it is} a feasable solution, then by weak duality we find that
%
\[ \sum_{S \in J} w_S = \sum y_a \leq H_m \cdot \text{OPT}_{\text{LP}} \]
%
giving us the required approximation bound. We construct the infeasible solution $y$ in this instance during the duration of the greedy algorithm. If we add the set $S$ on the $k$'th iteration, we set $y_a = w_S/|S_k|$ for each $a \in \smash{S_k}$. Then each $y_a$ is set exactly once. It is easy to see that if $S$ is added on the $k$'th iteration, then $w_S = \sum_{a \in S_k} y_a$ because of the way the weights are distributed over the LP. Now this implies that, unless $S_k = S$, that items will always `cost more' than they were meant.

Fix a set $S$, and let $U_k = S_k - S_{k+1}$ be the set of elements in $S$ covered in the $k$'th iteration. If $S'$ is chosen in the $k$th iteration, then for each $a \in U_k$,
%
\[ y_a = \frac{w_{S'}}{|S'_k|} \leq \frac{w_S}{|S_k|} \]
%
This means that
%
\begin{align*}
    \sum_{a \in S} y_a &= \sum_{k = 1}^n \sum_{a \in U_k} y_a \leq \sum_{k = 1}^n \frac{w_S}{|S_k|} = w_S \sum_{k = 1}^n \frac{|U_k| - |U_{k+1}|}{|S_k|} \leq w_S H_{|S|}
\end{align*}
%
It follows that $y/H_N$ is a feasible solution to the dual LP, and this gives the approximation ratio required.

\section{Randomized Rounding}

We now consider another technique for obtaining a solution to a combinatorial problem from an LP relaxation. Given an optimal solution $x^*$ to the LP relaxation to set cover, rather than interpreting the solutions to the LP as approximations to the discrete solution set, we shall interpret the solution as a {\it superposition of discrete solutions}. Consider the random family of sets which takes a set $S$ independent of other sets with probability $x^*_S$. As once can verify from linearity of expectation, this gives `a set cover in expectation', and the expected value of such a family is equal to $\text{OPT}_{\text{LP}}$. This should be immediately suspicious, since $\text{OPT}_{\text{LP}}$ lower bounds {\it all} discrete solutions to the algorithm. Indeed, we find that even though the random family satisfies the inequalities in expectation, it need not satisfy the inequalities pointwise. In other words, a particular choice of sets given by the random family need not be a cover. Nonetheless, we can still obtain good random approximation algorithms by choosing the probability values more carefully, and with a more developed analysis.

For each element $a$, the chance that $a$ is not covered by the randomized rounding algorithm is
%
\[ \prod_{a \in S} (1 - x^*_S) \]
%
Using the fact that $1 - x^*_S \leq e^{-x^*_S}$, we can bound the chance that $x$ is not covered by
%
\[ \prod_{a \in S} e^{-x^*_S} = e^{- \sum_{a \in S} x^*_S} \leq e^{-1} \]
%
because $\sum_{a \in S} x^*_S \geq 1$. Although $e^{-1}$ is only an upper bound on the probability of failure, it is possible to find instances of set cover which take this bound arbitrarily closely, so in the worst case we have $\approx 36 \%$ chance of not finding a set cover solution.

However, we can still use these ideas to obtain polynomial time algorithms whose chance of failure is $n^{-c}$, for any positive constant $c > 0$. This is known as an {\bf algorithm that works with high probability} (even though this is really a family of algorithms, and we normally find that the runtime of these algorithms increases as $c \to \infty$). If we can find a randomized procedure operating in polynomial time such that the probability that any $s$ is uncovered is bounded by $(1/n^c)$, then a union bound gives that the probability that a random choice is a set cover is $(1/n^{c-1})$, so we have obtained an algorithm giving a set cover with high probability. The idea is to consider $c \lg n$ Bernoulli trials which succeed with probability $x^*_S$, and to take a set $S$ in our cover if any of these trials pass. The probability that $S$ is not included is then $(1 - x_S^*)^{c \lg n}$, and therefore for each $a$,
%
\[ \prod_{a \in S} (1 - x^*_S)^{c \lg n} \leq \prod_{a \in S} e^{-x_S^* (c \lg n)} = e^{- (c \lg n) \sum_{a \in S_i} x^*_S} \leq \frac{1}{n^c} \]
%
Now we need only prove that this algorithm has a good expected value. Let $P_S(x^*) = 1 - (1 - x_S^*)^{c \lg n}$ be the probability that $S$ is included in our set. Since $x_S^* \in [0,1]$, we conclude that
%
\[ \frac{d P_S(x^*)}{d x^*_S} = (c \lg n) (1 - x_S^*)^{c \lg n - 1} \leq c \lg n \]
%
Furthermore, since $P_i(x^*) = 0$ if $x^*_S = 0$, we conclude that $P_S(x^*) \leq (c \lg n) x_S^*$. Let $X$ be the random variable with $X_S = 1$ if $S$ is included in our random set cover. Then the expected value of the given solution is
%
\[ \mathbf{E} \left(\sum w_S X_S \right) = \sum w_S P_S(x^*) \leq w_i (c \lg n) x_S^* = (c \lg n) \cdot \text{OPT}_{\text{LP}} \]
%
However, it is more important to bound the expected value, given that the given family we obtain is a set cover, because these are the actual feasible solutions we can output. Let $A$ be the event that the $X$ gives a valid set cover. Then $\mathbf{P}(A) \geq 1 - n^{c-1}$, and also
%
\[ \mathbf{E} \left(\sum w_S X_S \right) = \mathbf{E} \left(\sum w_S X_S | A \right) \mathbf{P}(A) + \mathbf{E} \left(\sum w_S X_S | A^c \right) \mathbf{P}(A^c) \]
%
Since $w,X \geq 0$, and $\mathbf{E}(\sum w_S X_S | A^c) \geq 0$, we conclude that
%
\[ \mathbf{E} \left( \sum w_S X_S | A \right) \leq \frac{\mathbf{E} \left( \sum w_S X_S \right)}{\mathbf{P}(A)} \leq \frac{(c \lg n) \cdot \text{OPT}_{\text{LP}}}{1 - n^{c-1}} \leq (2 c \lg n) \text{OPT}_{\text{LP}} \]
%
where we let $n,c \geq 2$. Though these types of randomized algorithms are not formally `approximation algorithms' in the sense of always giving an exact answer in a bounded, polynomial amount of time, these solutions are still practical in real life implementations of set cover approximations, and are also still interesting to consider.

\chapter{Greedy Approximations}

\section{Scheduling on a Single Machine}

Scheduling is one of the main archetypes of problems considered in combinatorial optimization. We are given work to be done, some resources to do the work, and a schedule at which these problems can be done. We want to optimize the time it takes to finish the work done. We now considered one of the simplest formulations of this problem. We have a sequence of task, each with a certain start time $s_i$, a deadline $d_i$, and a runtime $r_i$. We are given a single `worker' with which to accomplish the task, which can only focus on a single task at one time, and must process the task to completion once it is started. If task $i$ is started at time $t$, it finishes at time $f_i = t + r_i$, and we want to minimize the maximum lateness of any task, where the lateness is $l_i = f_i - d_i$.

This problem is NP hard, and even determining whether all jobs can be completed by their due date is NP hard. In the general case of the scheduling problem, we should not even expect an approximation algorithm to exist, because if we have an $f(n)$ approximation algorithm, then for any output which finishes all tasks before the deadline, the optimal output would be $0$, and our solution would output a solution of type $0 \cdot f(n) = 0$. However, if we assume that all deadlines are negative, so that we are late for every task, we can obtain a constant approximation algorithm. To introduce notation, for a subset $S$ of tasks, let $r(S) = \sum_{i \in S} r_i$ denote the minimum time to complete all tasks in $S$, $s(S) = \min_{i \in S} s_i$ the minimum time to start any of the tasks in $S$, and $d(S) = \max_{i \in S} d_i$ the latest deadline for any of the tasks. Let $\text{OPT}$ denote the value of the optimal solution to the scheduling problem.

\begin{prop}
    For any subset $S$ of tasks, $\text{OPT} \geq s(S) + r(S) - d(S)$.
\end{prop}
\begin{proof}
    In any solution to the scheduling problem, we can only start a task in $S$ at a later time than $s(S)$, and therefore we can only finish all the tasks in $S$ by time $s(S) + r(S)$. Since these tasks should all be finished before time $d(S)$, the last task completed must be, and we encumber a penalty of at least $s(S) + r(S) - d(S)$ for the solution. But then the optimal solution must also encumber this penalty, and we obtain the bound above.
\end{proof}

The standard heuristic for a single person to complete a set of tasks in real life is, when starting a new task, to choose the one with the earliest deadline. This is a greedy strategy known as the {\it earliest due date rule},  and gives a 2-approximation to the scheduling problem. Consider the schedule produced by the earliest due date rule, and let $j$ be the job with maximal lateness, finishing at time $f_j$. Consider the first time $t$ in which the worker never has to wait for a new task to be startable, so that the worker is working constantly from time $t$ onwards. Let $S$ be the set of jobs processed in this interval. Then $r(S) = f_j - t$, $s(S) = t$, because otherwise we could start the jobs in $S$ earlier, and $d(S) = d_j$, because otherwise we would have started job $j$ before the other jobs in $S$, and therefore the lemma above gives $f_j - d_j \leq \text{OPT}$.

\section{$K$ Center Problem}

Consider the $K$ center problem, in which we are given a metric space $(X,d)$, and the goal is to partition $X$ into $K$ clusters, each with a center point, in such a way that the maximum radius of the clusters is minimized, where the radius of a cluster is the largest distance in the cluster of a point from the center point. Once centers are chosen, the optimal way to partition $X$ into clusters is to place each point into the cluster whose center is cloest to that paint, so we might as well forget the whole partitioning set, and concentrate on choosing $K$ center points for the clusters.

We consider a greedy algorithm for this problem. We start by choosing an initial center arbitrarily. Then we pick the next point which lies furthest from this point, and then continuously choose a point whose distance from the current centers is maximized.

TODO:

\section{The Travelling Salesman Problem}

We now consider the travelling salesman problem, which takes a finite metric space $X$, and asks to find the enumeration $\{ x_1, \dots, x_n \}$ of $X$ which minimizes $\sum d(x_i,x_{i+1})$, where indices are modulo $n$. The greedy heuristic for the travelling salesman problem begins by finding the two points $x$ and $y$ in $X$ closest together, and forming the cycle that goes from $x$ to $y$, and then from $y$ to $x$. Given the current development of the cycle, we consider the point $z$ not yet in the tour closest to some point $x$ in the tour. If the tour goes from $x$ to $y$, then remove this edge in the tour, and instead go from $x$ to $z$ to $y$. After all points are added, we have found a tour.

It is clear that from the way that we have constructed the tour that this herustic has a relation to minimum spanning trees, since the growth of the tour is exactly the way that Prim's algorithm grows a minimum spanning tree. Consider the subsets $S_2, S_3, \dots, S_n$ of vertices identified in the tour successively, where $S_{k+1} = S_k \cup \{ x_k \}$, and where $S_2$ is the set of the first two vertices $x_1$ and $y_1$. If $y_k \in S_k$ is the closest vertex in $S_k$ to $x_k$, and $z_k$ is the vertex following $y_k$, then the difference in the length of the $k$'th and $k+1$'th tour is $d(x_k,y_k) + d(x_k,z_k) - d(y_k,z_k)$, and since the minimum spanning tree constructed by Prim's algorithm is $\sum d(x_k,y_k)$, we obtain that the length of $S_n$ is
%
\begin{align*}
    2d(x_1,y_1) + \sum_{k = 2}^{n-1} d(x_k,y_k) + d(x_k,z_k) - d(y_k,z_k)
\end{align*}
%
Using the triangle inequality, we find that $d(x_k,z_k) \leq d(x_k,y_k) + d(y_k,z_k)$, and as such the value above is bounded above by
%
\begin{align*}
    2d(x_1,y_1) + \sum_{k = 2}^{n-1} 2d(x_k,y_k) = 2 \text{OPT}_{MST}
\end{align*}
%
and we finish the algorithm by noting that $\text{OPT}_{MST} \leq \text{OPT}_{TSP}$, because if we remove a single edge from a tour, we obtain a minimum spanning tree. Thus we have a 2-approximation.

An alternate view of this algorithm, because of our use of the triangle inequality over the minimum spanning tree, is to take a minimum spanning tree, double all edges in the tree, and then form an Eulerian tour. This gives the same approximation bound as above in a more elegant way, but the solution we constructed above is always more optimal than this formulation. This is known as the {\it double tree algorithm}. What's more, we can use this kind of method to form a $3/2$ approximation, via a method known as {\it Christofides' Algorithm}

Consider the minimum spanning tree constructed above. Let $O$ be the set of points in the tree of odd degree. $O$ must contain an even number of points, because the sum of degrees of all nodes in a graph is always even (a consequence of the handshaking lemma). We can therefore compute a cheapest maximal matching of $O$ in polynomial time. If we add these edges to the minimum spanning tree, then we have an Eulerian graph, and we can therefore compute a tour on these edges, which is a feasible solution to the travelling salesman problem.

To see this is a $3/2$ approximation, we need only prove that the minimum cost perfect matching is $(1/2) \cdot \text{OPT}_{TSP}$. This is surprisingly easy. First, note that there is a tour on $O$ with cost at most $\text{OPT}$, because we can just `shortcut' a tour on $X$. If we color the nodes in $O$ by alternating the color on the tour, and then consider the two matchings of $O$ obtained by matching same colors on the tour, or my matching alternating colors, then one of these matchings must have cost at most $(1/2) \cdot \text{OPT}_{TSP}$, but then there must exist a perfect matching with this cost, and therefore the perfect matching we computed must have this cost.

Remarkably, this is the best known approximation ratio for the general metric travelling salesman problem. We only know that one cannot produce a better approximation ratio than $1.0045$, except if $\mathbf{P} = \mathbf{NP}$, so this is an open problem in theoretical computing science. Improvements for this problem have been constructed, including a $1 + \varepsilon$ ratio has been constructed in the case that our metric is embedable in the Euclidean plane. Because of this, we feel like more intelligent travelling salesman problem solutions exist.

\end{document}