%% The following is a directive for TeXShop to indicate the main file
%%!TEX root = HarmonicAnalysis.tex

\part{Distributional Methods}

\chapter{The Theory of Distributions}

Distribution theory is a tool which enables us to justify formal manipulations in harmonic analysis without having to worry about technical analytical issues. For instance, the theory allows us to take the formal derivative of a much more general class of functions than those differentiable pointwise by methods of the classical differential calculus. That such a consistant formal definition of the derivative is possible is hinted at by equations in Fourier analysis such as
%
\[ D^i f = \mathcal{F}^{-1} \left\{ M^i \mathcal{F} \{ f \} \right\}, \]
%
where $M^i f(x) = (2 \pi i x_i) f(x)$, and $D^i f$ is differentiation in the variable $x_i$. The quantity $D^i f$ can only be classically interpreted if $f$ is pointwise differentiable in the variable $x_i$, whereas the right hand side is defined for a much less regular class, namely the family of integrable functions $f$ such that $M^i \widehat{F} \{ f \}$ is also integrable. We will find a common domain, a class of generalized functions called tempered distributions, such that both sides of the equation above can be correctly interpreted.

If we consider solutions $u \in \loc{C^2}(\RR^d)$ to elliptic partial differential equations like the Laplace equation $\Delta u = 0$, then the regularity theory of such solutions shows the family of such functions is closed under locally uniform convergence, i.e. in the space $\loc{C}(\RR^d)$. This is no longer true for solutions to certain non-elliptic partial differential equations, such as solutions to the wave equation $\partial_t u = \Delta_x u$ in $\loc{C^2}(\RR \times \RR^d)$. For instance, the closure of such a class under uniform convergence would have to contain all functions of the form
%
\[ f(x+t) + g(x-t) \]
%
for any $f,g \in \loc{C}(\RR \times \RR^d)$. To fix this problem, in PDE it is often of interest to enlarge the class of solutions to consider \emph{weak solutions}, i.e. functions $u \in \loc{L^1}(\RR^d)$ such that for any $\phi \in C_c^\infty(\RR \times \RR^d)$,
%
\[ \int u(t,x) (\Delta \phi - \partial_t \phi)(t,x)\; dt\; dx = 0, \]
%
an equation true for any classical solution, as verified by integration by parts. This class is then clearly closed under uniform convergence, i.e. in $\loc{L^\infty}(\RR^d)$. The step to distributional solutions to partial differential equations from classical solutions is then not far off, since it allows us to apply differentiable operators to $u \in \loc{L^1}$, and we find that $u$ will be a weak solution to the wave equation if and only if $\partial_t u = \Delta_x u$. We will also find that the family of \emph{distributional} solutions to the wave equation is closed under a very general topology. These properties are one reason why distributions are a core tool to the formulation of many problems in partial differential equations and modern harmonic analysis.

\section{Distributions}

The path of modern analysis has extended analysis from the study of continuous and differentiable functions to the much larger class of measurable functions. The power of this approach is that we can study a very general class of functions. On the other hand, the more general the class of functions we work with, the more restricted the analytical operations we can perform, e.g. one cannot differentiate elements of $L^1(\RR^d)$ in a classical sense. Nonetheless, in pretty much all function spaces we consider on $\RR^d$, the space $C^\infty_c(\RR^d)$ forms a dense subclass, and in this subclass we can perform pretty much all possible analytical operations. One approach to studying the general class of measurable functions is to prove results for elements of $C^\infty_c(\RR^d)$, in which one can apply all necessary analytical operations, and then apply an approximation result to obtain the result for a wider class of measurable functions. The theory of distributions provides a complimentary approach, using \emph{duality} to formally extend analytical operations on $C^\infty_c(\RR^d)$ to larger sets.

From the perspective of set theory, a function $f: X \to Y$ is a rule for assigning a value in $Y$ to each point in $X$. However, in analysis this perspective is often not the most useful. This is most clear in measure theory, where we are very used to treating a function as defined `up to a set of measure zero', which has the peculiar feature of a function not actually being defined at any particular point. Distribution theory instead views functions as `integrands', whose properties are understood by integration against a family of `test functions'.

As an example of this phenomenon, since the dual space of $L^1(\RR^d)$is $L^\infty(\RR^d)$, we can think of elements $f \in L^\infty(\RR^d)$ as `integrands', which can be completely understood by `testing' $f$ against an element $\phi \in L^1(\RR^d)$, i.e. via a study of the quantities
%
\[ \int f(x) \phi(x)\; dx. \]
%
Similarily, if $K$ is a compact topological space, then the dual space to $C(K)$ is the space $M(K)$ of finite Borel measures on $K$. Thus we can think of the class of measures $M(K)$ as a family of `generalized integrands' which can be understood by testing these measures against continuous functions.

Notice that as we shrink the family of test functions, the resulting family of `generalized functions' becomes larger and larger, and so elements are allowed to behave more erratially. A distribution is a `generalized function' understood by testing against the very regular family of functions in the space $C_c^\infty(\RR^d)$, functions which in this situation are classical denoted as $\DD(\RR^d)$. Because $\DD(\RR^d)$ is very small, distributions can behave very erratically. But because most operations in analysis can be applied to elements of $\DD(\RR^d)$ and behave well under duality, we can then use duality to extend these operations to distributions. The class $\DD(\RR^d)$, or the slightly more general class $\SW(\RR^d)$ of Schwartz functions to be introduced shortly, is the natural class of functions for most problems studied in harmonic analysis. 

One can apply the ideas described in this chapter to many other classes of test functions. Provided that the test functions can be suitably localized, one will likely obtain similar results to that described in this chapter. On the other hand, if one deals with non localizable families of test functions, one is likely to obtain quite a different theory of generalized functions. This is encountered, for instance, if one takes the family of analytic functions as the test functions, which gives the theory of \emph{hyperfunctions}.

\begin{remark}
  From the perspective of experimental physics, viewing functions as integrands is more natural than viewing functions in the set-theoretic sense. Indeed, points in space are idealizations which do not correspond to real world phenomena. One can never measure the exact value of some quantity of a function at a point, but rather only understand the function by looking at it's averages over a small region around that point. Thus the only physically meaningful properties of a `function' are those obtained by testing that function against some family of test functions, obtained from some physical measurements. And indeed, physicists had experimented with using distributions in their calculations before a formal definition was introduced by mathematicians.
\end{remark}

\section{Test Functions and Distributions}

Fix an open subset $\Omega$ of $\RR^d$. There are various families of test functions we can use, which give a theory of distributions:
%
\begin{itemize}
    \item We can use the space $\DD(\Omega) = C_c^\infty(\Omega)$ of compactly supported smooth functions, under the topology of compact convergence, e.g. the LF space consisting of the inductive limit of the Fr\'{e}chet spaces $C^\infty_c(K)$ of smooth functions on $\RR^d$, which are compactly supported on a compact set $K$, equipped with the topology of uniform convergence of the functions are all of their derivatives. This gives the most general space of distributions $\DD(\Omega)^*$.

    \item We can use the space $\SW(\RR^d)$ of Schwartz functions on $\RR^d$, i.e. the Fr\'{e}chet space consisting of functions whose derivatives are all rapidly decreasing, i.e. all functions $f$ such that
    %
    \[ \| f \|_{\SW^{n,m}(\RR^d)} = \sup_{x \in \RR^d} \sup_{|\alpha| \leq m} |D^\alpha f(x)| \langle x \rangle^n \]
    %
    is finite for all integers $n$ and $m$. Schwartz space becomes a Fr\'{e}chet space if we treat these as continuous seminorms on the space. These test functions then give the space of \emph{tempered distributions} $\SW(\RR^d)^*$.

    \item We can use the space $\mathcal{E}(\Omega) = \loc{C^\infty}(\Omega)$ of all smooth functions on $\Omega$, with the topology of locally uniform convergence of the function and all of it's derivatives. Using these functions as test functions gives the space $\mathcal{E}(\Omega)^*$ of \emph{compactly supported distributions}.

    \item We can use the space $\DD^n(\Omega) = C_c^n(\Omega)$ of compactly supported $n$-times continuously differentiable functions, under the topology of compact convergence, e.g. with the LF topology analogous to that of $\DD(\Omega)$. This gives the space $\DD^n(\Omega)^*$ of \emph{distributions of order $n$}.
\end{itemize}
%
We begin by studying the space $\DD(\Omega)$. As an LF space, $\DD(\Omega)$ is \emph{not metrizable}. This is in some sense necessary; it is \emph{impossible} to make $C_c^\infty(\Omega)$ into a Fr\'{e}chet space, such that each of the subspaces
%
\[ C_c^\infty(K) = \{ f \in C_c^\infty(\Omega): \text{supp}(f) \subset K \} \]
%
are closed. The interior of each of these subspaces is empty because no translation of these subspaces is absorbing. The Baire category theorem would then imply that a countable union of such subspaces has nonempty interior, but $C_c^\infty(\Omega)$ is the countable union of such spaces. Nonetheless, as an LF space $\DD(\Omega)$ continues to have most of the useful properties of Fr\'{e}chet spaces.
%
%To see more practically what the problem is, let us see why $\DD(\RR^d)$ is not closed as subspace of the Fr\'{e}chet space $C^\infty_b(\RR^d)$ of smooth functions with bounded derivatives of all orders. Given $\psi \in \DD(\RR^d)$, the sum
%
%\[ \sum_{n = 1}^\infty 2^{-n} \cdot \text{Trans}_n \psi \]
%
%converges in $C^\infty_b(\RR^d)$ to a non compactly supported function. Thus under the topology of uniform convergence of a function and all of its derivatives, the limit of compactly supported functions can be made non compactly supported.
%
In particular,i t follows from the general theory of LF spaces that $\DD(\Omega)$ is a complete, locally convex space. The space $\DD(\Omega)$ is also a Montel space, since it is the inductive limit of Montel spaces.

We recall some useful properties of the inductive limit. A sequence $\{ \phi_n \}$ converges in $\DD(\Omega)$ to $\phi$ if $\bigcup_n \text{supp}(\phi_n)$ is precompact, and if $\{ \phi_n \}$ and all derivatives of this sequence converge uniformly to $\phi$. The space is Bornological, so that every bounded linear map $T: \DD(\Omega) \to Y$ is continuous, and so in particular, all sequentially continuous linear maps are continuous. Moreover, if $X$ is first countable, then for any bounded linear map $T: X \to \DD(\Omega)$, there exists a compact set $K \subset \Omega$ such that $T(X) \subset K$. The Arzela-Ascoli theorem implies that for any compact set $K$, each of the spaces $C_c^\infty(K)$ has the Heine-Borel property, and so $\DD(\Omega)$ also has the Heine-Borel property, i.e. $\DD(\Omega)$ is a Montel space.

%The process we perform here is quite general and can be viewed as a way to construct the `categorical limit' of a family of complete, locally convex spaces. For each compact set $K \subset \Omega$, the subspace $C_c^\infty(K) \subset \DD(\Omega)$ is a complete metric space under the family of seminorms $\| \cdot \|_{C^n(K)}$. We consider a convex topology on $\DD(\Omega)$ by considering the family of sets $\{ \phi + W \}$ as a basis, where $\phi$ ranges over all elements of $\DD(\Omega)$, and $W$ ranges over all convex, balanced subsets of $\DD(\Omega)$ such that $W \cap C_c^\infty(K)$ is open in $C_c^\infty(K)$ for each $K \subset \Omega$.

%\begin{theorem}
%    This gives a basis of a Hausdorff topology on $\DD(\Omega)$.
%\end{theorem}
%\begin{proof}
%    If $\phi_1 + W_1$ and $\phi_2 + W_2$ both contain $\phi$, then $\phi - \phi_1 \in W_1$ and $\phi - \phi_2 \in W_2$. The functions $\phi, \phi_1$, and $\phi_2$ are all supported on some compact set $K$. By continuity of multiplication on $C_c^\infty(K)$, and the fact that $W_n \cap C_c^\infty(K)$ is open, there is a small constant $\delta$ such that $\phi - \phi_n \in (1 - \delta) W_n$ for each $n \in \{ 1, 2 \}$. The convexity of the $W_n$ implies that $\phi - \phi_n + \delta W_n \subset W_n$. But then $\phi + \delta W_n \subset \phi_n + W_n$, and so $\phi + \delta (W_1 \cap W_2) \subset (\phi_1 + W_1) \cap (\phi_2 + W_2)$. Thus we have verified the family of sets specified above is a basis. Now we show $\DD(\Omega)$ is Hausdorff under this topology. Suppose $\phi$ is in every open neighbourhood of the origin, then in particular, for each $\varepsilon > 0$, $\phi$ lies in the set $W_\varepsilon = \{ f \in \DD(\Omega): \| f \|_{L^\infty(\Omega)} < \varepsilon \}$, and it is easy to see these sets are open. Since $\bigcap_{\varepsilon > 0} W_\varepsilon = \{ 0 \}$, this means $\phi = 0$.
%\end{proof}

%\begin{theorem}
%    $\DD(\Omega)$ is a locally convex space.
%\end{theorem}
%\begin{proof}
%    Fix $\phi$ and $\psi$, and consider any neighbourhood $W$ of the origin. By convexity, we have $(\phi + W/2) + (\psi + W/2) \subset (\phi + \psi) + W$. This shows addition is continuous. To show multiplication is continuous, fix $\lambda$, $\phi$, and a neighbourhood $W$ of the origin. Then $\phi$ is supported on some compact set $K$, and $W \cap C_c^\infty(K)$ is open, in particular absorbing, so there is $\varepsilon > 0$ such that if $|\alpha| < \varepsilon$, $\alpha \phi \in W/2$. Then if $|\gamma - \lambda| < \varepsilon$, then because $W$ is balanced and convex,
    %
%    \begin{align*}
%        \gamma \left(\phi + \frac{W}{2(|\lambda| + \varepsilon)} \right) &= \lambda \phi + (\gamma - \lambda) \phi + \frac{\gamma}{2(|\lambda| + \varepsilon)} W\\
%        &\subset \lambda \phi + W/2 + W/2 \subset \lambda \phi + W
%    \end{align*}
    %
%    so multiplication is continuous.
%\end{proof}

%\begin{theorem}
%    For each compact set $K \subset \Omega$, the canonical embedding of $C_c^\infty(K)$ in $\DD(\Omega)$ is continuous.
%\end{theorem}
%\begin{proof}
%    We shall prove a convex, balanced neighbourhood $V$ is open in $\DD(\Omega)$ if and only if $C_c^\infty(K) \cap V$ is open in $C_c^\infty(K)$ for each $K$. Since $V$ is open, $V$ is the union of convex, balanced sets $W_\alpha$ with $W_\alpha \cap C_c^\infty(K)$ open in $C_c^\infty(K)$ for each $K$. But then $V \cap C_c^\infty(K) = (\bigcup W_\alpha) \cap C_c^\infty(K)$ is open in $C_c^\infty(K)$. The converse is true by definition of the topology. But this statement means exactly that the map $C_c^\infty(K) \to \DD(\Omega)$ is an embedding, because it is certainly continuous, and if $W$ is a convex neighbourhood of the origin equal to the set of $\phi$ supported on $K$ with $\| \phi \|_{C^n(K)} \leq \varepsilon$ for some $n$, then the image is the intersection of $C_c^\infty(K)$ with the set of all $\phi$ supported on $\Omega$ satisfying the inequality, which is open. This shows that the map is open onto its image, hence an embedding.
%\end{proof}

%\begin{theorem}
%    Consider any $E \subset \DD(\Omega)$. Then $E$ is a bounded subset of $\DD(\Omega)$ if and only if $E$ is contained in $C_c^\infty(K)$ for some compact set $K$, and there is a sequence of constants $\{ M_n \}$ such that $\| \phi \|_{C^n(\Omega)} \leq M_n$ for all $\phi \in E$.
%\end{theorem}
%\begin{proof}
%    We shall now prove that if $E$ is not contained in some $C_c^\infty(K)$ for any compact set $K \subset \Omega$, then $E$ is not bounded. If our assumption is true, we can find functions $\phi_n \in E$ and a set of points $x_n \in X$ with no limit point such that $\phi_n(x_n) \neq 0$. For each $n$, set
    %
%    \[ W_n = \left\{ \psi \in \DD(\RR^d): |\psi(x_n)| < n^{-1} |\phi_n(x_n)| \right\}. \]
    %
%    Certainly $W_n$ is convex and balanced, and for each compact set $K$, if $\psi \in W_n \cap C_c^\infty(K)$, then there is $\varepsilon > 0$ such that $|\psi(x_n)| < n^{-1} |\phi_n(x_n)| - \varepsilon$. Thus if $\eta \in C_c^\infty(K)$ satisfies $\| \eta \|_{L^\infty(\RR^d)} < \varepsilon$, then $\psi + \eta \in W_n$. In particular, this means $W_n \cap C_c^\infty(K)$ is open in $C_c^\infty(K)$ for each $K$, so $W_n$ is open.

%    Now we claim $W = \bigcap_{n = 1}^\infty W_n$ is open. Certainly this set is convex and balanced. Moreover, each compact set $K$ contains finitely many of the points $\{ x_n \}$, so $W \cap C_c^\infty(K)$ can be replaced by a finite intersection of the $W_n$, and is therefore open. Since $\phi_n \not \in nW$ for all $n$, this implies that $E$ is not bounded. The fact that $\| \cdot \|_{C^n(\Omega)}$ specifies the topological structure of $C_c^\infty(K)$ for each compact $K$ now shows that if $E$ is bounded, there exists constants $\{ M_n \}$ such that $\| \phi \|_{C^n(\Omega)} \leq M_n$ for all $\phi \in E$. The converse property follows because $C_c^\infty(K)$ is embedded in $\DD(\Omega)$.
%\end{proof}

%\begin{theorem}
%    $\DD(\Omega)$ has the Heine Borel property.
%\end{theorem}
%\begin{proof}
%    This follows because if $E$ is bounded and closed, it is a closed and bounded subset of some $C_c^\infty(K)$ for some compact set $K$, hence $E$ is compact since $C_c^\infty(K)$ satisfies the Heine-Borel property (this can be proved by a technical application of the Arzela-Ascoli theorem).
%\end{proof}

%\begin{corollary}
%    $\DD(\Omega)$ is quasicomplete.
%\end{corollary}
%\begin{proof}
%    If $\phi_1, \phi_2, \dots$ is a Cauchy sequence in $\DD(\Omega)$, then the sequence is bounded, hence contained in some common $C_c^\infty(K)$. Since the sequence is Cauchy, they converge in $C_c^\infty(K)$ to some $\phi$, since $C_c^\infty(K)$ is complete, and thus the $\phi_n$ converge to $\phi$ in $\DD(\Omega)$.
%\end{proof}

TODO: Move to tensor products

\begin{theorem}
    For any two open sets $\Omega_1$ and $\Omega_2$, $C^\infty_{\text{loc}}(\Omega_1 \times \Omega_2)$ is naturally isomorphic to $C^\infty_{\text{loc}}(\Omega_2, C^\infty_{\text{loc}}(\Omega_1))$.
\end{theorem}
\begin{proof}
    The correspondence is obtained by mapping $f \in C^\infty_{\text{loc}}(\Omega_2, C^\infty_{\text{loc}}(\Omega_1))$ to $F \in C^\infty_{\text{loc}}(\Omega_1 \times \Omega_2)$ given by setting $F(x,y) = f(y)(x)$. The map is clearly one to one and continuous. If $F$ is an arbitrary element of $C^\infty_{\text{loc}}(\Omega_1 \times \Omega_2)$, and we define $f[y](x) = F(x,y)$, then the mean-value theorem implies that the quantities
    %
    \[ \frac{D^\alpha_x F(x,y + he_i) - D^\alpha_x F(x,y)}{h} \]
    %
    converge locally uniformly in $x$ to $D_y^i F$, and thus $f$ lies in $C^1(\Omega_2, C^\infty(\Omega_1))$. But iterating this process shows that $f$ is actually $C^\infty$. Thus the correspondence $f \mapsto F$ is a bijection. and since both spaces are Fr\'{e}chet spaces, the open mapping theorem implies that the correspondence is an isomorphism.
\end{proof}

\begin{theorem}
    Finite sums of tensor functions are dense in $\DD(\RR^d)$.
\end{theorem}
\begin{proof}
    Recall from the theory of multiple Fourier series that if $f \in C^\infty(\RR^d)$ is $N$ periodic, in the sense that $f(x + n) = f(x)$ for all $x \in \RR^d$ and $n \in (N \ZZ)^d$, then there are coefficients $a_m$ for each $m \in \ZZ^n$ such that $f = \lim_{M \to \infty} S_M f$, where the convergence is dominated by the sminorms $\| \cdot \|_{C^n(\RR^d)}$, for all $n > 0$, and
    %
    \[ (S_M f)(x) = \sum_{\substack{m \in \ZZ^d\\|m| \leq M}} a_m e^{\frac{2 \pi i m \cdot x}{N}}. \]
    %
    Note that since
    %
    \[ e^{\frac{2 \pi i m \cdot x}{N}} = \prod_{k = 1}^d e^{2 \pi i m_ix_i/N} \]
    %
    is a tensor product, $S_M f$ is a finite sum of tensor functions. If $\phi \in \DD(\RR^d)$ is compactly supported on $[-N,N]^d$, we let $f$ be a $10N$ periodic function which is equal to $\phi$ on $[-N,N]^d$. We then find coefficients $\{ a_m \}$ such that $S_M f$ converges to $f$. If $\psi: \RR \to \RR$ is a compactly supported bump function equal to one on $[-N,N]^d$, and vanishing outside of $[-2N,2N]^d$, then $\psi^{\otimes d} S_M f$ converges to $\psi$ as $M \to \infty$, and each is a finite sum of tensor functions.
\end{proof}

%We can actually go one step further than this.

%\begin{theorem}
%    For any open sets $\Omega$ and $\Psi$, $\DD(\Omega \times \Psi)$ is isomorphic to the \emph{projective} tensor product $\DD(\Omega) \otimes \DD(\Psi)$.
%\end{theorem}
%\begin{proof}
%    Let $X$ denote the \emph{algebraic} tensor product of $\DD(\Omega)$ and $\DD(\Psi)$. Then we have an injective map $i: X \to \DD(\Omega \times \Psi)$ with dense image. Because $i$ is bilinearly continuous, $i$ is continuous in the projective tensor product topology. Conversely, for any two compact sets $K_1 \subset \Omega$ and $K_2 \subset \Psi$, $i(C_c^\infty(K_1) \otimes C_c^\infty(K_2)) \subset \DD(K_1 \times K_2)$. For any $N > 0$, pick multi-indices $\alpha_i = \lambda_i + \gamma_i$ such that $\| f_i \|_{C^N(K_1)} = |D^{\lambda_i} f_i(x_i)|$ and $\| g_i \|_{C^N(K_2)} = |D^{\gamma_i} g_i(y_i)|$.

%    Doesn't diagonal example show this map is not open (i.e. $h(x) = \sum_{k = 1}^{n-1} \phi(nx - k) \phi(ny - k)$, with $\| h \|_{C^N} \sim n^N$, but with the projective tensor product norm proportional to $n^{2N}$)

%    find a multi-index $\alpha = \alpha_1 + \alpha_2$ with $|\alpha| \leq N$ and $(x_0,y_0)$ such that
    %
%    \[ \| h \|_{C^N(K_1 \times K_2)} = |D^\alpha h(x_0,y_0)|. \]
    %
%    Then $D^\alpha h(x_0,y_0) = \sum D^{\alpha_1} f_i(x_0) D^{\alpha_2} g_i(y_0)$



%    We claim this is continuous when we equip $X$ with the projective tensor topology. For any $f_1,\dots,f_n \in C_c^\infty(K_1)$, $g_1,\dots,g_n \in C_c^\infty(K_2)$, $h = f_1 \otimes g_1 + \dots + f_n \otimes g_n$ is supported on $K_1 \times K_2$, and for any $N > 0$,
    %
%    \begin{align*}
%        \| h \|_{C^N(K_1 \times K_2)} &\leq \| f_1 \otimes g_1 \|_{C^N(K_1 \otimes K_2)} + \dots + \| f_n \otimes g_n \|_{C^N(K_1 \times K_2)}\\
%        &\leq \| f_1 \|_{C^N(K_1 \otimes K_2)} \| g_1 \|_{C^N(K_1 \otimes K_2)} + \dots + \| f_n \|_{C^N(K_1 \otimes K_2)} \| g_n \|_{C^N(K_1 \otimes K_2)}.
%    \end{align*}
    %
%    Taking infima over all representations of $h$ as a sum of tensor products therefore gives that $\| h \|_{C^N(K_1 \times K_2)} \leq \| h \|_{C^N(K_1) \otimes C^N(K_2)}$. Conversely, given any function $h \in i(X)$, fix $N > 0$, and suppose $(x_0,y_0)$ and $\alpha$ are such that $\| h \|_{C^N(K_1 \times K_2)} = |D^\alpha h(x_0,y_0)|$. Then if $h = \sum f_i \otimes g_i$, $\sum D^\alpha (f_i \otimes g_i)(x) = h(x_0,y_0)$, so 
%\end{proof}

%Because $\DD(\Omega)$ is the limit of metrizable spaces, it's linear operators still have many of the same properties as metrizable spaces.

%\begin{theorem}
%    If $T: \DD(\Omega) \to X$ is a map from $\DD(\Omega)$ to some locally convex space $X$, then the following are equivalent:
    %
%    \begin{itemize}
%        \item[(1)] $T$ is continuous.
%        \item[(2)] $T$ is bounded.
%        \item[(3)] If $\{ \phi_n \}$ converges to zero, then $\{ T\phi_n \}$ converges to zero.
%        \item[(4)] For each compact set $K \subset \Omega$, $T$ is continuous restricted to $C_c^\infty(K)$.
%    \end{itemize}
%\end{theorem}
%\begin{proof}
%    We already known that (1) implies (2). If $T$ is bounded, and we have a sequence $\{ \phi_n \}$ converging to zero, then the sequence is bounded, hence contained in some $C_c^\infty(K)$. Then $T$ is bounded as a map from $C_c^\infty(K)$ to $X$, hence $\{ T\phi_n \} \to 0$. (3) implies (4) because each $C_c^\infty(K)$ is metrizable, and any convergent sequence is contained in some common $C_c^\infty(K)$. To prove that (4) implies (1), we let $V$ be a convex, balanced, open subset of $X$. Then $T^{-1}(V) \cap C_c^\infty(K)$ is open for each $K$, and $T^{-1}(V)$ is convex and balanced, so $T^{-1}(V)$ is an open set.
%\end{proof}

Because convergence is so strict in $\DD(\Omega)$, almost every operation we want to perform on smooth functions is continuous in this space.
%
\begin{itemize}
    \item Since $f \mapsto D^\alpha f$ is a continuous operator from $C_c^\infty(K)$ to itself, it is therefore continuous on the entire space $\DD(\Omega)$. More generally, any linear differential operator with coefficients in $\DD(\Omega)$ is a continuous operator on $\DD(\Omega)$.

    \item The inclusion $\DD(\Omega) \to L^p_c(\Omega)$ is continuous. To prove this, it suffices to prove for each compact $K$, the inclusion $C_c^\infty(K) \to L^p(K)$ is continuous, and this follows because if $f \in C_c^\infty(K)$, then
    %
    \[ \| f \|_{L^p(K)} \leq |K|^{1/p} \| f \|_{L^\infty(K)}. \]

    \item Multiplication gives a continuous operator
    %
    \[ \DD(\Omega) \times \DD(\Omega) \to  C^\infty_b(\Omega) \times \DD(\Omega) \to \DD(\Omega). \]
    %
    but the operator
    %
    \[ \loc{C^\infty}(\Omega) \times \DD(\Omega) \to \DD(\Omega) \]
    %
    is sequentially continuous, and thus bounded, but not continuous.

    \item The convolution operator
    %
    \[ \DD(\RR^d) \times \DD(\RR^d) \to L^1_c(\RR^d) \times \DD(\RR^d) \to \DD(\RR^d) \]
    %
    is continuous.
%    for any $g \in \DD(\RR^d)$, $f * g \in \DD(\RR^d)$. This is because $f * g$ is continuous since $g \in L^\infty(\RR^n)$, and it's support is contained in the algebraic sums of the support of $f$ and $g$, as well as the identity $D^\alpha(f * g) = f * (D^\alpha g)$. In fact, the map $g \mapsto f * g$ is a continuous operator on $\DD(\RR^n)$. This is because if we restrict our attention to $C_c^\infty(K)$, and $f$ has supported on $K'$, then our convolution operator maps into the compact set $K+K'$, and since
    %
%    \[ \| D^\alpha (g * f) \|_{L^\infty(K + K')} = \| D^\alpha g * f \|_{L^\infty(K + K')} \leq \| D^\alpha g \|_{L^\infty(K)} \| f \|_{L^1(K')}, \]
    %
%    we conclude
    %
%    \[ \| g * f \|_{C^n(K+K')} \leq \| g \|_{C^n(K)} \| f \|_{L^1(K')}, \]
    %
%    which gives continuity of the operator as a map from $C_c^\infty(K)$ to $C_c^\infty(K+K')$. Since the latter space embeds in $\DD(\RR^n)$, we obtain continuity of the operator on $\DD(\RR^n)$.
\end{itemize}
%
Thus $\DD(\Omega)$ is an ideal place to study many of the natural operations which occur in analysis.

%\begin{theorem}
%    If a map $T: C_c^\infty(K_0) \to \DD(\RR^n)$ is continuous, then the image of $C_c^\infty(K_0)$ is actually $C_c^\infty(K_1)$ for some compact set $K_1$.
%\end{theorem}
%\begin{proof}
%    Suppose there is a sequence $\{ x_i \}$ in $\RR^d$ with no limit point and smooth functions $\{ \phi_i \}$ compactly supported on $C_c^\infty(K_0)$ such that
    %
%    \[ (T\phi_i)(x_i) \neq 0. \]
    %
%    Then for any sequence $\{ \alpha_i \}$ of positive scalars, the sequence $\{ \alpha_i T\phi_i \}$ does not converge to zero, since the union of the supports of $\alpha_i T\phi_i$ is unbounded. This means $\alpha_i \phi_i$ does not converge to zero. But this is clearly not true, for if we let
    %
%    \[ \alpha_i = \frac{1}{2^i \| \phi_i \|_{C^i(\RR^d)}}, \]
    %
%    then for any fixed $n$, $\lim_{i \to \infty} \| \alpha_i \phi_i \|_{C^n(\RR^d)} = 0$, so the sequence $\{ \alpha_i \phi_i \}$ converges to zero. Thus there cannot exist a sequence $\{ x_i \}$, and so the union of the supports of $T(C_c^\infty(K_0))$ is supported on some compact set $K_1$.
%\end{proof}
%Thus the topology on the space $\DD(\RR^d)$ is as strict as can be. As a consequence, we shall see that the weak-$*$ topology on $\DD^*(\RR^d)$ is essentially the weakest topology available in analysis. This is surprising, because we are still able to obtain the continuity of many operators in the dual space to $\DD(\RR^d)$.

We now have the tools to explain the idea of a distribution. If $f \in \loc{L^1}(\Omega)$, then the linear functional $\Lambda[f]$ on $\DD(\Omega)$ defined for each $\phi \in \DD(\Omega)$ by setting
%
\[ \Lambda[f](\phi) = \int f(x) \phi(x)\; dx \]
%
is continuous. Moreover, $\Lambda[f]$ determines $f$ uniquely, and so we can safely identify $f$ with $\Lambda[f]$. We thus speak of `the distribution' $f$. The idea of the theory of distributions is to treat any continuous linear functional $\Lambda$ on $\DD(\Omega)$ as if it were given by integration against a test function. Thus for such a linear functional $\Lambda$, we often abuse notation by denoting the quantity $\Lambda(\phi)$ by
%
\[ \int_\Omega \Lambda(x) \phi(x)\; dx, \]
%
even if $\Lambda$ is not given by integration against some function. The space $\DD^*(\Omega)$ will be called the space of distributions on $\Omega$. The class of distributions induced by elements of $\loc{L^1}(\Omega)$ is a fundamental class of distributions, but we will soon see much more erratic distributions.

One huge advantage of this approach is that we can generalize many analytical operations defined on $\DD(\Omega)$ \emph{distributionally} to give an operation on $\DD^*(\Omega)$, even if the original analytical operations required some degree of smoothness to define. If $A: \DD(\Omega) \to \DD(\Omega)$ is a continuous operator, then we can consider it's adjoint $A^*: \DD(\Omega)^* \to \DD(\Omega)^*$. If $A^*$ maps $\DD(\Omega)$ continuously into itself, then we can define an extension of $A$ to a map from $\DD(\Omega)^* \to \DD(\Omega)^*$ by defining, for a distribution $\Lambda$, a distribution $A \Lambda$ by the formula
%
\[ \langle A \Lambda, \phi \rangle = \langle \Lambda, A^* \phi \rangle. \]
%
This definition has the property that $A(\Lambda[\phi]) = \Lambda[A \phi ]$ for any $\phi \in \DD(\Omega)$, so that we have really constructed an extension of $A$ so it is defined on all distributions.

For instance, we can use this idea to define the derivative of an arbitrary distribution. For $\phi,\psi \in \DD(\RR)$, integration by parts tells us that
%
\[ \int_{-\infty}^\infty \phi'(x) \psi(x)\; dx = - \int_{-\infty}^\infty \phi(x) \psi'(x)\; dx. \]
%
Thus if $A\phi = \phi'$ is the derivative operator then it's adjoint behaves on distributions by setting $A^* \psi = - \psi'$. Thus, for a distribution $\Lambda$ on $\RR$, we define it's derivative to be the distribution $\Lambda'$ such that for $\phi \in \DD(\RR)$,
%
\[ \Lambda'(\phi) = \Lambda(A^* \phi) = - \Lambda(\phi'). \]
%
More generally, a similar calculation allows us to set, for a distribution $\Lambda$ on $\RR^d$, and a multi-index $\alpha$, we define $D^\alpha \Lambda(\phi) = (-1)^{|\alpha|} \Lambda(D^\alpha \phi)$.

\begin{example}
    Let $H(x) = \mathbf{I}(x > 0)$ denote the {\it Heaviside step function}. Then $H$ is locally integrable, and so for any test function $\phi$, we calculate
    %
    \[ \int_{-\infty}^\infty H'(x) \phi(x)\; dx = - \int_{-\infty}^\infty H(x) \phi'(x) = - \int_0^\infty \phi'(x) = \phi(0) \]
    %
    Thus the \emph{distributional derivative} of the Heaviside step function is the Dirac delta function. It is not a function, but if we were to think of it as a `generalized function', it would be zero everywhere except at the origin, where it is infinitely peaked.
\end{example}

\begin{example}
    Consider the Dirac delta function at the origin, which is the distribution $\delta$ such that for any $\phi \in \DD(\RR)$,
    %
    \[ \int_{-\infty}^\infty \delta(x) \phi(x)\; dx = \phi(0). \]
    %
    Then
    %
    \[ \int_{-\infty}^\infty \delta'(x) \phi(x)\; dx = - \int_{\RR^d} \delta(x) \phi'(x)\; dx = - \phi'(0). \]
    %
    This is a distribution that does not arise from integration with respect to a locally integrable function nor integration against a measure, but it is an appropriate model of certain physical situations, i.e. for the distribution of electrical charge in a polarized point mass with positive charge infinitisimally to the left of the origin, and negative charge infinitisimally to the right of the origin.
\end{example}

In general, we define a \emph{distribution} to be a continuous linear functional on the space of test functions $\DD(\Omega)$, i.e. an element of $\DD(\Omega)^*$. In the last section, our exploration of continuous linear transformations on $\DD(\Omega)$ guarantees that a linear functional $\Lambda$ on $\DD(\Omega)$ is continuous if and only if for every compact $K \subset X$ there is an integer $n_k$ such that $|\Lambda \phi| \lesssim_K \| \phi \|_{C^{n_k}(K)}$ for $\phi \in C_c^\infty(K)$. If one integer $n$ works for all $K$, and $n$ is the smallest integer with such a property, we say that $\Lambda$ is a distribution of \emph{order $n$}. If such an $n$ doesn't exist, we say the distribution has infinite order. If such an $n$ doesn't exist, we say the distribution has infinite order. Applying the Hahn-Banach theorem shows that if $\Lambda \in \DD^*(\Omega)$ has order $n$, then $\Lambda$ extends uniquely to a continuous linear functional on $\DD^n(\Omega)$, since $\DD(\Omega)$ is dense in $\DD^n(\Omega)$. In fact, we see that we can identify $\DD^n(\Omega)^*$ with the space of distributions of order $n$.

In many other ways, distributions act like functions. For instance, any distribution $\Lambda$ can be uniquely written as $\Lambda_1 + i \Lambda_2$ for two distributions $\Lambda_1, \Lambda_2$ that are real valued for any real-valued smooth continuous function. However, we cannot write a real-valued distribution as the difference of two positive distributions, i.e. those which are non-negative when evaluated at any non-negative functional. This is because any non-negative distribution is actually given by integration against a Radon measure, and thus has order zero. To see this, given such a non-negative functional $\Lambda$ (which is automatically continuous),  we define $\Lambda f$ for a compactly supported continuous function $f \geq 0$ as
%
\[ \Lambda f = \sup \{ \Lambda g: g \in \DD(\RR^n), g \leq f \} \]
%
and then in general define $\Lambda (f^+ - f^-) = \Lambda f^+ - \Lambda f^-$. Then $\Lambda$ is obviously a positive extension of $\Lambda$ to all continuous functions, and is linear. But then the Riesz representation theorem implies that there is a positive Radon measure such that $\Lambda = \Lambda_\mu$, completing the proof.

\begin{example}
    If $\mu$ is a complex-valued Radon measure, then we can define a distribution $\Lambda[\mu]$ such that for each $\phi \in \DD(\RR^d)$.
    %
    \[ \Lambda[\mu](\phi) = \int_{\RR^d} \phi(x) d\mu(x) \]
    %
    Thus $\Lambda[\mu]$ is a distribution, since if $\phi$ is supported on $K$, then
    %
    \[ |\Lambda[\mu](\phi)| \leq \mu(K) \| \phi \|_{L^\infty(K)}. \]
    %
    The fact that this bound does not require information about the derivatives of $\phi$ implies that $\Lambda[\mu]$ is a distribution of order zero. The Riesz-Markov-Kakutani representation theorem, shows that \emph{any} distribution of order zero is given by a complex-valued Radon measure.
\end{example}

\begin{example}
    Let $U \subset \RR^d$ be any open set such that $\partial U$ is a $C^1$ hypersurface. Then we can find a $C^1$ function $\rho: \RR^d \to \RR$ such that $U = \{ x : \rho(x) > 0 \}$, and $\nabla \rho$ is non-zero on $\partial U$. Let us calculate the derivatives of the distribution $u = \mathbf{I}_U$. Clearly $\nabla u$ is supported on the surface $S = \partial U$. Let us work locally around a point $x_0 \in S$, and without loss of generality, let us rotate so that in a neighborhood of $x_0$, if we write $x = (x',x_d)$, then $S$ is described by the equation $x_d = \psi(x')$, where $\psi$ is $C^1$. Pick $f \in C^\infty(\RR)$ such that $f(t) = 0$ for $t < 0$ and $f(t) = 1$ for $t \geq 1$. Then, locally around $x_0$, $u$ is the distributional limit of the functions $u_\varepsilon$, where
    %
    \[ u_\varepsilon(x) = f \left( \frac{x_d - \psi(x')}{\varepsilon} \right). \]
    %
    But
    %
    \[ \nabla u_\varepsilon(x) = (1/\varepsilon) \cdot f' \left( \frac{x_d - \psi(x')}{\varepsilon} \right) \cdot \left( 1, - \nabla \psi(x') \right) \]
    %
    which implies that for any $\phi \in \DD(\RR^d)$ supported near $x_0$, we calculate using a change of variables that
    %
    \begin{align*}
        \int &\phi(x) \nabla u(x)\; dx\\
        &= \lim_{\varepsilon \to 0} \int \phi(x) (1/\varepsilon) \cdot f' \left( \frac{x_d - \psi(x')}{\varepsilon} \right) \cdot \left( 1, - \nabla \psi(x') \right)\\
        &= \lim_{\varepsilon \to 0} \int \phi(x', \psi(x') + \varepsilon t) f'(t) (1, -\nabla \psi(x'))\; dt\; dx'\\
        &= \int \phi(x', \psi(x')) (1, -\nabla \psi(x'))\; dx.
    \end{align*}
    %
    Since the surface measure on $S$ is given by $dS = (1 + |\psi'|^2)^{1/2}\; dx'$, and the normal vector to the surface is given by $n = (1, -\nabla \psi(x')) / (1 + |\psi'|^2)^{1/2}$, it follows that
    %
    \[ \int \phi(x) \nabla u(x)\; dx = \int_S \phi(x) n(x)\; dS. \]
    %
    Switching to the global viewpoint, we now see that $\nabla u = n \cdot dS$. In particular, we have proved that for any smooth, compactly supported vector field $X = (X_1,\dots,X_n)$ on $\RR^d$,
    %
    \[ \int_S (\nabla \cdot X) = \sum \int u(x) \partial_i X_i(x)\; dx = - \int_S X \cdot n\; dS, \]
    %
    so our proof amounts exactly to the Gauss-Green divergence theorem. The fact that $n \cdot dS$ is a distribution of order zero implies that the Gauss-Green formula continues to hold for any $C^1$ compactly supported vector field. Perhaps the most general setting in which the distributional derivative calculation continues to hold for any domain $U$ which is a \emph{Caccioppoli set}, i.e. such that $u = \mathbf{I}_U$ is a function with bounded variation. We can then define a Radon measure analogous to $n \cdot dS$ such that an analogue of the integration formulas continue to hold. In particular, we can still use the calculation for open sets with Lipschitz boundary, in which case the normal vector is only defined almost everywhere on the boundary.    
\end{example}

\begin{example}
    Consider a functional $\Lambda$ defined for functions $\phi \in \DD(\RR)$ vanishing in a neighbourhood of the origin by setting
    %
    \[ \Lambda(\phi) = \int_{-\infty}^\infty \frac{\phi(x)}{x}\; dx. \]
    %
    Such functions are \emph{not} dense in $\DD(\RR)$. But we claim $\Lambda$ is bounded on it's domain, and thus by the Hahn-Banach theorem, extends to at least one continuous functional on the entirety of $\DD(\RR)$. To prove this, fix $\phi \in C_c^\infty[-N,N]$ vanishing on a neighbourhood $(-\varepsilon,\varepsilon)$ of the origin. Then
    %
    \[ |\Lambda \phi| = \left| \int_{-\infty}^\infty \frac{\phi(x)}{x}\; dx \right| = \left| \int_{\varepsilon \leq |x| \leq N} \frac{\phi(x) - \phi(0)}{x}\; dx \right|. \]
    %
    Applying the mean-value theorem, we find
    %
    \[ |\Lambda \phi| \leq N \| \phi \|_{C^1[-N,N]}. \]
    %
    Since $N$ was arbitrary, it follows that $\Lambda$ is continuous in the topology induced by that of $\DD(\RR)$, and thus by the Hahn-Banach theorem, extends uniquely to at least one distribution on the entirety of $\DD(\RR)$.

    One canonical choice of $\Lambda$ is the \emph{principal value distribution} $\text{p.v}(1/x)$, defined such that
    %
    \[ \int_{-\infty}^\infty \text{p.v}(1/x) \phi(x)\; dx = \lim_{\delta \to 0} \int_{|x| \geq \delta} \phi(x) / x\; dx. \]
    %
    We essentially showed that this functional was continuous above. Another choice is the distribution $\lim_{\varepsilon \to 0} 1/(x + i \varepsilon)$, defined such that
    %
    \[ \int_{-\infty}^\infty \lim_{\varepsilon \to 0} 1/(x + i \varepsilon) \cdot \phi(x)\; dx = \lim_{\varepsilon \to 0} \int_{-\infty}^\infty \phi(x) / (x + i \varepsilon)\; dx. \]
    %
    If we pick $\delta = \varepsilon^{1/4}$, then we can show using the fact that $1/x$ and $1/(x + i \varepsilon)$ are not too different for large $x$ that
    %
    \begin{align*}
        \left| \int_{|x| \geq \delta} \frac{\phi(x)}{x} - \int_{-\infty}^\infty \frac{\phi(x)}{x + i\varepsilon} \right| \leq \| \phi \|_1 \cdot \varepsilon^{1/2}.
    \end{align*}
    %
    A contour integral shift shows that
    %
    \begin{align*}
        \int_{-\delta}^\delta \frac{\phi(x)}{x + i\varepsilon} &= \int_{-\delta}^\delta \frac{\phi(0)}{x + i \varepsilon} + O(\delta)\\
        &= -i \pi \phi(0) + O(\varepsilon / \delta) + O(\delta)\\
        &= -i \pi \phi(0) + O(\varepsilon^{1/4}).
    \end{align*}
    %
    Taking $\varepsilon \to 0$ shows that
    %
    \[ \text{p.v}(1/x) = i \pi \delta + \lim_{\varepsilon \to 0} 1/(x + i \varepsilon), \]
    %
    where $\delta$ is the Dirac delta distribution at the origin.

    More generally, if $\Lambda_1$ and $\Lambda_2$ are two distributions which extend the functional $\Lambda$, then one can show that for any function $\phi$ vanishing away from the origin, $\Lambda_1(\phi) - \Lambda_2(\phi) = 0$. We will later define the \emph{support} of a distribution, and so we have shown here that $\Lambda_1 - \Lambda_2$ is supported at $\{ 0 \}$. It follows from later theorems in this chapter than $\Lambda_1$ and $\Lambda_2$, applied to a function $\phi \in \DD(\RR)$, will differ by a finite linear combination of the values of $\phi$ and it's derivatives at the origin.

    The distribution $\text{p.v}(1/x)$ can also be described as the distributional derivative of the locally integrable function $\log |x|$, since an integration by parts shows that for each $\phi \in \DD(\RR^d)$,
    %
    \begin{align*}
        \int (\log |x|)'\; \phi(x)\; dx &= - \int \log |x| \phi'(x)\; dx\\
        &= \lim_{\varepsilon \to 0} \int_{|x| \geq \varepsilon} \log |x| \phi'(x)\\
        &= \lim_{\varepsilon \to 0} \left( \log(\varepsilon) \cdot \left( \phi(x) - \phi(-x) \right) + \int_{|x| \geq \varepsilon} \frac{\phi(x)}{x} \right)\\
        &= \text{p.v} \int \frac{\phi(x)}{x}\; dx.
    \end{align*}
    %
    An important analysis of these distributions arises in the theory of the Hilbert transform.
\end{example}

\begin{example}
    If $\Omega$ is an open subset of $\CC$, let us calculate $\partial E_{z_0} / \partial \overline{z}$ in $\Omega$, where $E_{z_0} = \text{p.v} \{ (z - z_0)^{-1} \}$. To begin with, we note that if $X$ is another open subset of $\CC$ containing $\Omega$, and if $\partial \Omega$ (the boundary of $\Omega$ \emph{relative to $X$}) is $C^1$, then for $\phi \in \DD(X)^*$, Green's formula gives
    %
    \[ \int_\Omega \frac{\partial \phi}{\partial \overline{z}} = (-i/2) \int_{\partial \Omega} \phi\; dz. \]
    %
    This formula implies the product rule, that for $\phi \in \DD(\Omega)^*$,
    %
    \begin{align*}
        \int_{\Omega - B_\varepsilon(z_0)} \frac{1}{z - z_0} \frac{\partial \phi}{\partial \overline{z}} &= \int_{\Omega - B_\varepsilon(z_0)} \frac{\partial}{\partial \overline{z}} \left\{ \frac{1}{z - z_0} \phi \right\}\\
        &= (i/2) \int_{\partial B_\varepsilon(z_0)} \frac{1}{z - z_0} \phi\; dz.
    \end{align*}
    %
    Letting $\varepsilon \to 0$ gives $\int E_{z_0} \partial_{\overline{z}} \phi = - \pi \phi(z_0)$, or in other words, we conclude that in $\Omega$, $\partial_{\overline{z}} E_{z_0} = \pi \delta_{z_0}$.
\end{example}

\begin{example}
    One reason we could define a distribution agreeing with $1/x$ away from the origin is because there is a lot of cancellation at the origin from either side of the origin, since $1/x$ switches sign here. One has to rely on other tricks to make sense of a distribution extending $1/x^2$. Indeed, if we write
    %
    \[ \Lambda(\phi) = \int \frac{\phi(x)}{x^2} \]
    %
    for $\phi$ ranging over all functions vanishing in the neighborhood of the origin, then we can use the mean value theorem to obtain a bound $|\phi(x)| \leq x^2 \| \phi'' \|_\infty$, from which it follows that
    %
    \[ \Lambda(\phi) \lesssim \| \phi'' \|_\infty, \]
    %
    and so Hahn-Banach extends $\Lambda$ to a family of distributions. But in this case the principal value
    %
    \[ \lim_{\delta \to 0} \int_{|x| \geq \delta} \frac{\phi(x)}{x^2} \]
    %
    rarely exists. Indeed, for any fixed $\phi \in \DD(\RR)$ we have
    %
    \[ \int_{|x| \geq \delta} \frac{\phi(x)}{x^2} = 2 \phi(0) / \delta + O(\delta). \]
    %
    which will only converge if $\phi(0) = 0$. Thus, to get around this, we define the \emph{finite part distribution} (or \emph{Hadamard regularization}) of $1/x^2$, i.e. the distribution $\text{f.p}(1/x^2)$ obtained by setting
    %
    \[ \int \text{f.p}(1/x^2)\; \phi(x)\; dx = \lim_{\delta \to 0} \left( \int_{|x| \geq \delta} \phi(x)/x^2 - \frac{2 \phi(0)}{\delta} \right), \]
    %
    which gets around the result that the distribution might explode near the origin if $\phi(0) \neq 0$. Note that we do not need to account for $\phi'(0)$ because of cancellation on both sides of the integral.

    Another approach to extending $\Lambda$ is to consider the derivative of the distribution $- \text{p.v}(1/x)$, since the derivative of this distribution agrees with integration against $1/x^2$ away from the origin. In fact, the derivative of $- \text{p.v}(1/x)$ is precisely $\text{f.p}(1/x^2)$. We leave it to the reader to use similar tricks to define the finite parts of higher order singularities, such as $1/x^3$.
\end{example}

\begin{example}
    Let $f$ be a left continuous function on the real line with bounded variation and with $f(-\infty) = 0$. Then $f'$ exists almost everywhere in the classical sense, and $f' \in L^1(\RR)$. By Fubini's theorem, if we let $\mu$ be the measure defined by $\mu([a,b)) = f(b) - f(a)$, then for any $\phi \in \DD(\RR)$,
    %
    \begin{align*}
        \int_{-\infty}^\infty \phi(x) d\mu(x) &= - \int_{-\infty}^\infty \int_x^\infty \phi'(y)\; dy\; d\mu(x)\\
        &= - \int_{-\infty}^\infty \phi'(y) \int_{-\infty}^y d\mu(x)\; dy\\
        &= - \int_{-\infty}^\infty \phi'(y) f(y) dy
    \end{align*}
    %
    Thus if $\Lambda$ is the distribution corresponding to integration with respect to $f(x)\; dx$, then $\Lambda'$ is given by integration with respect to $\mu$. In particular, $\Lambda'$ is given by integration with respect to $f'(x)\; dx$ precisely when $f$ is absolutely continuous.
\end{example}

\begin{example}
    If $f \in C^1(\RR - \{ 0 \})$, and if the function $v(x)$ defined to be $f'(x)$ for $x \neq 0$ is integrable, then the limits $f(0-)$ and $f(0+)$ both exist (a simple argument using the fundamental theorem of calculus), and the distributional derivative of $f$ is equal to
    %
    \[ f' = v + (f(0+) - f(0-)) \delta_0. \]
    %
    To see this, we calculate that for $\phi \in \DD(\RR)$,
    %
    \begin{align*}
        \int f'(x) \phi(x)\; dx &= -\int f(x) \phi'(x)\; dx\\
        &= \lim_{\varepsilon \to 0} - \int_{-\infty}^{-\varepsilon} f(x) \phi'(x)\; dx - \int_\varepsilon^\infty f(x) \phi'(x)\; dx\\
        &= \lim_{\varepsilon \to 0} f(\varepsilon) \phi(\varepsilon) - f(-\varepsilon) \phi(-\varepsilon) + \int_{-\infty}^{-\varepsilon} v(x) \phi(x)\; dx + \int_{-\infty}^{-\varepsilon} v(x) \phi(x)\; dx\\
        &= \int v(x) \phi(x)\; dx + [f(0+) - f(0-)] \phi(0).
    \end{align*}
    %
    As a particular example of this, the distributional derivative of $|x|$ is $\text{sgn}(x)$, and the distributional derivative of the Heaviside step function $H$ given above is evaluated to be $\delta_0$.
\end{example}

\begin{example}
    The boundary values of analytic functions are distributions, in the following manner. Let $\Omega \subset \RR^n$ be an open set, and let $\Gamma$ be a convex open cone in $\RR^n$. For $\gamma > 0$, let
    %
    \[ Z_\gamma = \{ z \in \CC^n: \text{Re}(z) \in \Omega, \text{Im}(z) \in \Gamma, |\text{Im}(z)| < \gamma \}. \]
    %
    If $f$ is analytic in $Z_\gamma$, and there is some $N > 0$ such that $|f(x + iy)| \lesssim |y|^{-N}$. We claim then that, if for each $y \in \Gamma$, we consider the analytic function $f_y(x) = f(x + iy)$, then as $y \to 0$, $\{ f_y \}$ converges distributionally to a distribution of order at most $N+1$ on $\Omega$. In fact, we will obtain an explicit formula for this distribution. For simplicity we deal with the case $n = 1$, where the higher dimensional case is similar (see H\"{o}rmander Theorem 3.1.15). Then we might as well assume that $Z_\gamma = \{ x + i y : x \in \Omega, 0 < y < \gamma \}$. Given $\phi \in C_c^{N+1}(\Omega)$, write
    %
    \[ \phi_y(x) = \phi(x + i y) = \sum_{\alpha = 0}^N \partial^\alpha_x \phi(x) \frac{(i y)^\alpha}{\alpha!}. \]
    %
    Then
    %
    \[ 2 \frac{\partial \phi}{\partial \overline{z}} = \partial^{N+1} \phi(x) \frac{(iy)^N}{N!}. \]
    %
    Thus if $0 < y_0 < \gamma$ is fixed, and if $0 < y < \gamma - y_0$, then the Cauchy integral formula implies that
    %
    \[ \int \phi(x) f(x + iy)\; dx - \int \phi(x + iy) f(x + iy + iy_0)\; dx = 2i \int \int_0^{y_0} f(x + iy + it) \frac{\partial \phi}{\partial \overline{z}} dt\; dx. \]
    %
    Thus
    %
    \[ \int \phi(x) f_y(x)\; dx = \int \phi_{y_0}(x) f_{y + y_0}(x)\; dx + \int \int_0^1 f_{y + ty_0}(x) \partial^{N+1} \phi(x) \frac{(i y_0)^{N+1} t^N}{N!}\; dt\; dx. \]
    %
    Thus we conclude that as $y \to 0$,
    %
    \[ \int \phi(x) f_y(x)\; dx \to \int \phi_{y_0}(x) f_{y_0}(x)\; dx + \frac{1}{N!} \int \int_0^1 f_{ty_0}(x) \partial^{N+1} \phi(x) (i y_0)^{N+1} t^N\; dt\; dx. \]
    %
    The right hand side clearly defines a distribution of order $N+1$. In higher dimensions, we obtain the similar representation formula
    %
    \[ \int \phi(x) f_y(x)\; dx \to \int \phi_{y_0}(x) f_{y_0}(x)\; dx + \frac{1}{N!} \int \int_0^1 f_{ty_0}(x) \sum_{|\alpha| = N+1} \partial^\alpha \phi(x) (iy_0)^\alpha t^N\; dt\; dx. \]
    %
    Thus we have a theory of boundary values of analytic functions.

    For $n = 1$, we denote the distribution obtained by $f(x + i 0)$. If we, by a similar process, obtained a distribution as a limit of an analytic function defined on the lower half plane, then we would denote that distribution by $f(x - i0)$. If $f$ is analytic above and below the half plane, and satisfies estimates of the form above, then the distribution $u$ given by
    %
    \[ \int \left( u(x + i y) \phi(x,y)\; dx \right)\; dy \]
    %
    (\emph{the order of integration matters}) defines a distribution of order $N$ in the complex plane, the derivative $\partial u / \partial \overline{z}$ is supported on the real line, and on that real line, we have
    %
    \[ \frac{\partial u}{\partial \overline{z}}(x) = (i/2) \left( f(x + i0) - f(x - i0) \right). \]
    %
    We can use this, for instance, to calculate the difference beteween the two distributions $(x + i0)^{-1}$ and $(x - i0)^{-1}$. We have calculated above that the distribution $u(z) = 1/z$ has
    %
    \[ \frac{\partial u}{\partial \overline{z}} = \pi \delta, \]
    %
    and so $(x - i0)^{-1} = (x + i0)^{-1} + 2 \pi i \delta$.
\end{example}

\begin{remark}
    Roughly speaking, this result is tight. If an analytic function $f$ has boundary values defining a distribution of order $N+1$, then it is necessary that $|f(z)| \lesssim |\text{Im}(z)|^{-N-2}$.
\end{remark}

There are many other important operations one can apply to distributions. If $\Omega$ is a conic subset of $\RR^d$, and $\phi,\psi \in \DD(\Omega)$, we find
%
\[ \int_{\Omega} \text{Dil}_\lambda \phi(x) \psi(x)\; dx = \lambda^{-d} \int_{\Omega} \phi(x) \cdot \text{Dil}_{1/\lambda} \psi(x)\; dx, \]
%
Thus if $\Lambda$ is a distribution on $\Omega$, then we define $\text{Dil}_\lambda \Lambda$ by setting
%
\[ \text{Dil}_\lambda \Lambda (\phi) = \lambda^{-d} \Lambda( \text{Dil}_{1/\lambda} \phi). \]
%
For $f \in C^\infty(\Omega)$, we have an operator $\phi \mapsto f \phi$ on $\DD(\Omega)$. The adjoint is clearly $\psi \mapsto f \psi$, so for a distribution $\Lambda$ on $\Omega$, we define $f \Lambda$ by setting $(f\Lambda)(\phi) = \Lambda(f \phi)$. Thus $\DD^*(\Omega)$ is naturally a $C^\infty(\Omega)$ module. Similarily, the family $\DD^*(\Omega)_k$ consisting of distributions of order $k$ form a $C^k(\Omega)$ module.

\begin{remark}
    H\"{o}rmander developed a sophisticated theory that enables us to define the product of two \emph{distributions} using the Fourier transform. In many basic situations, one can perform a spatial decomposition to define the product. Given a distribution $\Lambda$, we define it's \emph{singular support} $\text{supp}_{\text{sing}}(\Lambda)$ to be the \emph{complement} of the set of all points $x$ which have a neighborhood $U$ such that $\Lambda|_U \in C^\infty(U)$. For any two distributions $\Lambda$ and $\Psi$ whose singular supports are disjoint, a decomposition argument enables us to define the product $\Lambda \cdot \Psi$ in a natural way.
\end{remark}

\section{Topologies on the Space of Distributions}

As a dual space to an LF space, we can equip the space of distributions $\DD(\Omega)^*$ with several topologies. The most important for our purposes in the strong topology, which makes $\DD(\Omega)^*$ into a complete locally convex space, and the weak $*$ topology, which is quasi-complete. These topologies are roughly the same for many purposes. For instance, the uniform boundedness theorem implies that a sequence $\{ u_n \}$ of distributions converges in the weak $*$ topology if and only if it converges in the strong dual topology, and we call this convergence \emph{distributional convergence}. We note also that $\DD(\Omega)^*$, as the strong dual of a Montel space, is also a Montel space, i.e. it is locally convex, barelled, and satisfies the Heine-Borel property.

\begin{example}
    If $u \in \DD(\RR^d)$, and we set $u_\varepsilon(x) = \varepsilon^{-d} \text{Dil}_\varepsilon u(x)$, then for any $\phi \in \DD(\RR^d)$,
    %
    \[ \lim_{\varepsilon \to 0} \int u_\varepsilon(x) \phi(x)\; dx \to \phi(0) \int u(x)\; dx. \]
    %
    Thus $u_\varepsilon$ converges distributionally to $(\int u(x)\; dx) \cdot \delta_0$. Similarily, if $u \in \DD(\RR^d)$ and for any multi-index $\alpha$ with $|\alpha| \leq k$,
    %
    \[ \int u(x) x^\alpha\; dx = 0, \]
    %
    and we define $u_\varepsilon(x) = \varepsilon^{-d-k} \text{Dil}_\varepsilon u(x)$, then for any $\phi \in \DD(\RR^d)$,
    %
    \[ \lim_{\varepsilon \to 0} u_\varepsilon(x) \phi(x)\; dx \to \frac{1}{k!} \sum_{|\alpha| = k} \left( \int x^\alpha u(x)\; dx \right) \cdot D^\alpha \phi(0) \]
    %
    Thus $u_\varepsilon$ converges distributionally to an appropriate linear combination of $D^\alpha \delta_0$.
\end{example}

\begin{example}
    The distribution $\lim_{\varepsilon \to 0} 1/(x + i\varepsilon)$ defined above is precisely the limit of the distributions $1/(x + i \varepsilon)$ in the weak $*$ topology. Similarily, $\text{p.v}(1/x)$ is the weak $*$ limit of the functions $\mathbf{I}_{|x| \geq \delta}(x) \cdot (1/x)$. The distribution $\text{f.p}(1/x^2)$ is the distributional limit of $\mathbf{I}_{|x| \geq \delta}(x) \cdot (1/x^2) - 2 \delta_0 / \delta$, where $\delta_0$ is the Dirac delta function at the origin.
\end{example}

\begin{example}
    If $n$ is a positive integer, then integration by parts shows that for any $\phi \in \DD(\RR)$,
    %
    \[ \int_{-\infty}^\infty t^n e^{2 \pi itx} \phi(x)\; dx = i^{n+1} t^{-1} \int_{-\infty}^\infty e^{itx} \phi^{(n+1)}(x)\; dx, \]
    %
    which converges to zero as $t \to \infty$. Thus $t^n e^{2 \pi itx}$ converges distributionally to zero as $t \to \infty$. Another way to see this is to note that the distribution $\Lambda_t$ given by integration against $t^n e^{itx}$ can be written as $\Lambda_t(\phi) = t^n \widehat{\phi}(-t)$, and the Fourier transform of $\phi$ decays rapidly. Note that if we tested against functions that were less smooth (say, viewing these distributions as linear functionals on $L^1(\RR^d)$, or even $C^\infty$ functions that are only of polynomial decrease as they approach $\infty$) then this statement would no longer be true.
\end{example}

\begin{example}
    Let $u_t(x) = t^{1/k} e^{itx^k}$, where $k$ is an integer bigger than one. Let
    %
    \[ F(x) = \int_0^x e^{iy^k}\; dy. \]
    %
    When $x > 0$, a contour integration shift shows that
    %
    \[ F(x) = e^{i \pi / 2k} \int_0^x e^{-y^k}\; dy + O(|x|^{-(k-1)}). \]
    %
    If $k$ is even, then for $x < 0$,
    %
    \[ F(x) = - e^{i \pi / 2k} \int_0^x e^{-y^k}\; dy + O(|x|^{-(k-1)}) \]
    %
    and for $k$ odd,
    %
    \[ F(x) = - e^{-i \pi / 2k} \int_0^x e^{-y^k}\; dy + O(|x|^{-(k-1)}). \]
    %
    Thus given $\phi \in \DD(\RR)$, we can apply an integration by parts to write
    %
    \begin{align*}
        \int_{-\infty}^\infty u_t(x) \phi(x)\; dx &= \int_{-\infty}^\infty t^{1/k} e^{itx^k} \phi(x)\; dx \\
        &= \int_{-\infty}^\infty t^{1/k} F'(t^{1/k} x) \phi(x)\; dx\\
        &= - \int_{-\infty}^\infty F(t^{1/k} x) \phi'(x)\; dx,
     \end{align*}
     %
     By decomposing this integral into the region where $|x| \geq t^{1/k}$ and $|x| \leq t^{1/k}$ shows that this quantity converges to
     %
     \[ - F(\infty) \int_0^\infty \phi'(x)\; dx - F(-\infty) \int_{-\infty}^0 \phi'(x)\; dx = (F(\infty) - F(-\infty)) \phi(0). \]
     %
     Thus $u_t$ converges distributionally to $\left( 2 e^{i \pi / 2k} \int_0^\infty e^{-y^k}\; dy \right) \cdot \delta_0$ if $k$ is even, and to $\left( 2 \cos(\pi / 2k) \int_0^\infty e^{-y^k}\; dy \right) \cdot \delta_0$ if $k$ is odd.
\end{example}

It is often of interest to focus on subfamilies of $\DD(\Omega)^*$, equipped with a topology which is compatible with the behaviour of distributions. We thus define a \emph{space of distributions} to be a vector subspace $X$ of $\DD(\Omega)^*$, equipped with a topology which makes the inclusion map $X \to \DD(\Omega)^*_b$ continuous. Most function spaces are spaces of distributions, for instance, $\DD(\Omega)$, $\loc{L^1}$, and so on. We will later encounter the space of \emph{tempered distributions} $\mathcal{S}(\Omega)^*$, which will turn out to be a subspace of $\DD(\Omega)^*$ equipped with the relative topology.

Since convergence in $\DD(\Omega)$ is incredibly strict, a sequence of distributions can very easily converge distributinoally. It is therefore surprising that, since a differential operator $L: \DD(\Omega) \to \DD(\Omega)$ is continuous, it's extension to a map $L: \DD(\Omega)^* \to \DD(\Omega)^*$, roughly speaking, a rescaling of it's adjoint, is also continuous, both in the weak $*$ topologies and in the strong topology.

With either of the two topologies described above, the bilinear multiplication map
%
\[ \loc{C^\infty}(\Omega) \times \DD(\Omega)^* \to \DD(\Omega)^* \]
%
is sequentially continuous, but not jointly continuous.

%There is also an often useful result resulting from bounded countable families of distributions.

%\begin{theorem}
%    Suppose $\mathcal{U} \subset \DD^*(\Omega)$ is a family of distributions such that for each $\phi \in \DD(\Omega)$, $\sup_{u \in \mathcal{U}} |u(\phi)| < \infty$. Then for every compact set $K$, there exists $m$ such that for an $u \in \mathcal{U}$ and $\phi \in C_c^\infty(K)$,
    %
%    \[ |u(\phi)| \lesssim \| \phi \|_{C^m(K)}. \]
    %
%    If $\{ u_n \}$ is a sequence of distributions for which $\lim_n u_n(\phi)$ exists for every $\phi \in \DD(\Omega)$, then $u(\phi) = \lim_n u_n(\phi)$ defines a distribution, and for every compact set $K$ there is an integer $m$ such that for each $\phi \in C_c^\infty(K)$,
    %
%    \[ |u_n(\phi)| \lesssim \| \phi \|_{C^m(K)} \]
    %
%    and
    %
%    \[ \lim_{n \to \infty} \sup_{\phi \in C_c^\infty(K)} \frac{|u(\phi) - u_n(\phi)|}{\| \phi \|_{C^m(K)}} = 0. \]
%\end{theorem}
%\begin{proof}
%    Each distribution in $\mathcal{U}$ acts as a continuous operator on the Frech\'{e}t space $C_c^\infty(K)$, and this family satisfies the uniform boundedness principle, and the existence of an $n$ as above follows as a result of the uniform boundedness principle, i.e. it shows that restricted to $K$, the distributions in $\mathcal{U}$ are equicontinuous.

%    Now assume the second condition. This clearly implies the first, hence we get the uniform boundedness property above. Now a ball of finite radius in $C^{m+1}(K)$ is precompact in $C^m(K)$, by the Arzela-Ascoli theorem. Thus we can find $\phi_1,\dots,\phi_N \in C_c^\infty(K)$ such that if $\phi \in C_c^\infty(K)$ and $\| \phi \|_{C^{n+1}(K)} \leq 2$, then there exists $i$ such that $\| \phi - \phi_i \|_{C^n(K)} \leq \varepsilon$. Pick $n_0$ such that for any $n \geq n_0$ and $1 \leq i \leq N$, $|(u - u_n)(\phi_i)| \leq \varepsilon$. Then given any $\phi \in C_c^\infty(K)$ with $\| \phi \|_{C^{n+1}(K)} \leq 1$, we can find $i$ as above, and then
    %
%    \begin{align*}
%        |(u - u_n)(\phi)| &\leq |(u - u_n)(\phi - \phi_i)| + |(u - u_n)(\phi_i)|\\
%        &\lesssim \| \phi - \phi_i \|_{C^n(K)} + \varepsilon\\
%        &\lesssim \varepsilon.
%    \end{align*}
    %
%    Thus we have proven the required limiting statement.
%\end{proof}

\section{Homogeneous Distributions}

An important family of distributions are the \emph{homogenous distributions}, which are those distributions $\Lambda$ on $\RR^d - \{ 0 \}$ such that for each $\lambda > 0$, $\text{Dil}_\lambda \Lambda = \lambda^\alpha \Lambda$, where $\alpha$ is the \emph{order} of the homogenous distribution $\Lambda$.

\begin{example}
  If $f \in L_1^{\text{loc}}(\RR^d)$ and $f(\lambda x) = \lambda^\alpha f(x)$ for all $x \in \RR^d - \{ 0 \}$ then integration against $f(x)\; dx$ defines a homogenous distribution of order $\alpha$.
\end{example}

\begin{example}
  For any complex number $a$ with $\text{Re}(a) > -1$, if we define a distribution on $\RR - \{ 0 \}$ by setting
  %
  \[ x^a_+ = \begin{cases} x^a & x > 0 \\ 0 &: x <= 0 \end{cases}, \]
  %
  then $x^a_+$ is a homogeneous distribution of order $\alpha$, and $x \cdot x^a_+ = x^{a+1}_+$, and if $\text{Re}(a) > 0$,
  %
  \[ \frac{d}{dx} \left( x^a_+ \right) = a x^{a-1}_+. \]
  %
  Our goal is to extend this distribution to a larger range of values $a \in \CC$, such that the association $a \mapsto x^a_+$ is continuous. For any $\phi \in \DD(\RR)$, the function
  %
  \[ a \mapsto \langle x^a_+, \phi \rangle \]
  %
  is analytic in $a$ for $\text{Re}(a) > -1$. Integration by parts shows that
  %
  \[ \langle x^{a+1}_+, \phi' \rangle = - (a+1) \langle x^a_+, \phi \rangle. \]
  %
  The formula $\langle x^a_+, \phi \rangle = -(a+1)^{-1} \langle x^{a+1}_+, \phi' \rangle$ allows us to extend the definition of $x^a_+$ to all $a \in \CC$ with $\text{Re}(a) > -2$, except that we have a pole of order one when $a = -1$. Iterating this allows us to uniquely extend the definition of $x^a_+$ for all $a \in \CC - \ZZ^-$, and these distributions will all be homogeneous.

  Marcel Riesz also used some other complex analytic tricks to define $x^{-k}_+$ for all integers $k$, but then we lose some of the homogeneity. For any $\phi \in \DD(\RR)$, the function $a \mapsto \langle x^a_+, \phi \rangle$ is meromorphic, with simple poles at each integer $-k$ for $k > 0$, and the residue at $-k$ is equal to
  %
  \[ (-1)^k D^{k-1} \phi(0) / (k-1)! \]
  %
  Thus we conclude that, as $a \to -k$,
  %
  \[ \langle (a + k) x^a_+, \phi \rangle \to (-1)^{k-1} D^{k-1} \phi(0) / (k-1)! \]
  %
  In fact, expanding things out gives a constant $C_{-k}(\phi)$ such that as $a \to -k$,
  %
  \[ \langle x^a_+, \phi \rangle = \frac{(-1)^{k-1} D^{k-1} \phi(0)}{(k-1)!} \cdot \frac{1}{a + k} + C_{-k}(\phi) + O(a+k). \]
  %
  We define $\langle x^{-k}_+, \phi \rangle = C_{-k}(\phi)$, i.e. by keeping only the \emph{finite part} of the integral. Since, for $a$ close to $k$, we have
  %
  \begin{align*} \langle x^a_+, \phi \rangle &- \frac{(-1)^{k-1} D^{k-1} \phi(0)}{(k-1)!} \cdot (a+k)^{-1}\\
  &= (-1)^k (a+1)^{-1} \cdots (a+k-1)^{-1} \int_0^\infty \frac{x^{a + k} - 1}{a+k} D^k \phi(x)\; dx\\
  &\quad\quad + \frac{(-1)^{k-1}}{a+k} \left( (-1-a)^{-1} \dots (1-a-k) - \frac{1}{(k-1)!} \right) D^{k-1} \phi(0)\\
  &\to \frac{-1}{(k-1)!} \int_0^\infty \log(x) D^k \phi(x)\; dx + \frac{1}{(k-1)!} \left(\sum_{i = 1}^k 1/i \right) D^{k-1} \phi(0).
  \end{align*}
  %
  and thus
  %
  \[ \langle x^{-k}_+, \phi \rangle = \frac{-1}{(k-1)!} \int_0^\infty \log(x) D^k \phi(x)\; dx + \frac{1}{(k-1)!} \left(\sum_{i = 1}^k 1/i \right) D^{k-1} \phi(0). \]
  %
  Our extension of $x^a_+$ for all $a \in \CC$ continues to satisfy $x \cdot x^a_+ = x^{a+1}_+$ for all $a \in \CC$. The derivative formula is \emph{not} maintained, namely,
  %
  \[ D x^{-k}_+ = -k x^{-k-1}_+ + (-1)^k D^k \delta / k! \]
  %
  Moreover, $x^{-k}_+$ is no longer homogeneous of degree $-k$. Plugging into the formula shows that
  %
  \[ (tx)^{-k}_+ = t^{-k} x^{-k}_+ + \frac{\log t}{(k-1)!} \cdot D^{k-1} \delta. \]
  %
  One can also define $x^a_+$ by first removing the singularity, considering the distributions
  %
  \[ \langle x^a_\varepsilon, \phi \rangle = \int_\varepsilon^\infty x^a \phi(x)\; dx. \]
  %
  If $k$ is the smallest non-negative integer such that $k + \text{Re}(a) > -1$, then we can integrate by parts to conclude that there are constants $C_k$ such that
  %
  \[ \langle x^a_\varepsilon, \phi \rangle = \sum_{i = 0}^{k-1} C_i \varepsilon^{-i} + (-1)^k \int_0^\infty \frac{1}{(a+1) \dots (a+k)} x^{a + k} D^k \phi(x)\; dx + o(1). \]
  %
  Discarding the singular terms, and letting $\varepsilon \to 0$ gives the distributions $x^a_+$ above. One can analogously define the distributions $x^a_-$, and the distributions $|x|^a$, by reflecting and symmetrizing the distributions about the origin.
\end{example}

\begin{example}
    One can fix the singularities at the integers by normalizing. The appearance of the singularities in the extension of $x^a_+$ occured because of the $a$ term in the formula $D(x^a_+) = a x^{a-1}_+$. If we consider the normalization $\chi^a_+ = x^a_+ / \Gamma(a+1)$ when $\text{Re}(a) > -1$, then $D(\chi^a_+) = - \chi^{a-1}_+$, and because the Gamma function has no zeroes, this allows us to extend the definition of $\chi^a_+$ to an analytic function for all $a \in \CC$, each a homogeneous distribution of order $a$. But since $\chi^0_+$ is the Heaviside step function, we conclude that $\chi_+^{-k} = (-1)^k D^{k-1} \delta$.
\end{example}

\begin{example}
  Let $\delta$ be the Dirac-delta distribution at the origin in $\RR^d$. Then for $\phi \in \DD(\RR^d)$,
  %
  \begin{align*}
    \int_{\RR^d} (\text{Dil}_\lambda \delta)(x) \phi(x)\; dx = \lambda^{-d} \int_{\RR^d} \delta(x) \text{Dil}_{1/\lambda} \phi(x)\; dx = \lambda^{-d} \phi(0).
  \end{align*}
  %
  Thus $\text{Dil}_\lambda \delta = \lambda^{-d} \delta$, which implies $\delta$ is a homogenous distribution of order $-d$.
\end{example}

A homogeneous distribution is apriori defined by testing against a distribution in $\DD(\RR^n - \{ 0 \})$. But in most cases the distribution can be extended so that it can be tested against an arbitrary distribution in $\DD(\RR^n)$. Before we prove this, we begin with some simple observations. First is a formula due to Euler, which in the distributional setting states that for any distribution $u$ of degree $a$,
%
\[ \sum_{i = 1}^n x_i \partial_i u = (a + n) u. \]
%
When $n = 1$, this actually implies that $u$ is a multiple of $|x|^a$ on each coordinate axis. The identity also implies that for any $\psi \in \DD(\RR^n - \{ 0 \})$ with $\int_0^\infty r^{a + n-1} \psi(rx)\; dr = 0$ for all $x$, $\langle u, \phi \rangle = 0$ by rewriting the formula in polar coordinates.

\begin{theorem}
    Let $u$ be a homogeneous distribution on $\RR^n - \{ 0 \}$ of order
    %
    \[ a \in \CC - \{ -n, -(n+1), \dots \}. \]
    %
    Then $u$ has a unique extension to a distribution $E(u)$ on $\RR^n$, such that for any homogeneous polynomial $P$, $E(Pu) = P E(u)$, and if $u$ is not a distribution of order $1-n$, $E(\partial_i u) = \partial_i E(u)$. Moreover, the map $u \mapsto E(u)$ is continuous from $\DD(\RR^n - \{ 0 \})^*$ to $\DD(\RR^n)^*$.
\end{theorem}
\begin{proof}
    The uniqueness is obvious, because any distribution supported at the origin is a linear combination of derivatives of the Dirac delta function, which are all homogeneous of integer order $\leq -n$. To show existence, we note that if $u$ was locally integrable, then polar coordinates gives
    %
    \[ \int u(x) \phi(x) = \int_0^\infty \int_{|w| = 1} u(w) t^{a + n-1} \phi(t w)\; d\sigma(w)\; dt. \]
    %
    Thus we need only study the behaviour of $u$ near the unit sphere, which is supported away from the origin. Doing this more formally yields the extension map $E$. For any $\phi \in \DD(\RR^n)$, define $R_a \phi(x) = \langle t^{a + n-1}_+, \phi(tx) \rangle$. Then $R_a \phi$ is homogeneous of degree $-n-a$, and is continuous from $\DD(K)$ to $C^\infty(\RR^n - \{ 0 \})$ for any compact set $K$. If $\psi \in C_c(\RR^n - \{ 0 \})$ and $\int_0^\infty \psi(tx)/t\; dt = 1$ for all $x \neq 0$, then $\psi R_a \phi \in \DD(\RR^n - \{ 0 \})$, and $R_a(\psi R_a \phi) = R_a \phi$, so that (because of our observations before the proof) $E(u) = \langle u, \psi R_a \phi \rangle$ is independent of the choice of $\psi$. Moreover, $\langle u, \psi R_a \phi \rangle = \phi$ for each $\phi \in \DD(\RR^n - \{ 0 \})$. The continuity of the map $u \mapsto E(u)$ is not too difficult to see, completing the proof.
\end{proof}

When $a$ is an integer smaller than or equal to $-n$, one can still use the construction in the proof to define an operator $E(u)$, but then it can depend on $\psi$, may fail to be homogeneous near the origin, and may fail to satisfy the identities $E(Pu) = PE(u)$ and $E(\partial_i u) = \partial_i E(u)$ (See H\"{o}rmander, 3.2).

\section{Localization of Distribuitions}

Just as we can consider the local behaviour of functions around a point, we can consider the local behaviour of a distribution around points, and this local behaviour contains most of the information of the distribution. For instance, given an open subset $U$ of $X$, we say two distributions $\Lambda$ and $\Psi$ are equal on $U$ if $\Lambda \phi = \Psi \phi$ for every test function $\phi$ compactly supported in $U$. We recall the notion of a partition of unity, which, for each open cover $U_\alpha$ of Euclidean space, gives a family of $C^\infty$ functions $\psi_\alpha$ which are positive, {\it locally finite}, in the sense that only finitely many functions are positive on each compact set, and satisfy $\sum \psi_\alpha = 1$ on the union of the $U_\alpha$.

\begin{theorem}
    If $X$ is covered by a family of open sets $U_\alpha$, and $\Lambda$ and $\Psi$ are locally equal on each $U_\alpha$, then $\Lambda = \Psi$. If we have a family of distributions $\Lambda_\alpha$ which agree with one another on $U_\alpha \cap U_\beta$, then there is a unique distribution $\Lambda$ locally equal to each $\Lambda_\alpha$.
\end{theorem}
\begin{proof}
    Since we can find a $C^\infty$ partition of unity $\psi_\alpha$ compactly supported on the $U_\alpha$, upon which we find if $\phi$ is supported on $K$, then finitely many of the $\psi_\alpha$ are non-zero on $K$, and so
    %
    \[ \Lambda(\phi) = \sum \Lambda(\psi_\alpha \phi) = \sum \Psi(\psi_\alpha \phi) = \Psi(\phi) \]
    %
    Thus $\Lambda = \Psi$. Conversely, if we have a family of distributions $\Lambda_\alpha$ like in the hypothesis, then we can find a partition of unity $\psi_{\alpha \beta}$ subordinate to $U_\alpha \cap U_\beta$, and we can define
    %
    \[ \Lambda(\phi) = \sum \Lambda_\alpha(\psi_{\alpha \beta} \phi) = \sum \Lambda_\beta(\psi_{\alpha \beta} \phi) \]
    %
    The continuity is verified by fixing a compact $K$, from which there are only finitely many nonzero $\psi_{\alpha \beta}$ on $K$, and the fact that this definition is independant of the partition of unity follows from the first part of the theorem.
\end{proof}

In the language of commutative algebra, the association of $\DD^*(U)$ to each open subset $U$ of $\Omega$ gives the structure of a sheaf of modules on $\Omega$. Given a distribution $\Lambda$, we might have $\Lambda(\phi) = 0$ for every $\phi$ supported on some open set $U$. The complement of the largest open set $U$ for which this is true is called the \emph{support} of $\Lambda$. This agrees with the sheaf theoretic definition.

\begin{theorem}
    If a distribution $\Lambda \in \DD^*(\Omega)$ has compact support, then $\Lambda$ has some finite order $n$, and extends uniquely to a continuous linear functional on $C^n(\Omega)$.
\end{theorem}
\begin{proof}
    Let $\Lambda$ be a distribution supported on a compact set. If $\psi$ is a function with compact support with $\psi(x) = 1$ on the support of $\Lambda$, then $\psi \Lambda = \Lambda$, because for any $\phi$, $\phi - \phi \psi$ is supported on a set disjoint from the support of $\Lambda$. But if $\psi$ is supported on some compact set $K$, then there is $n$ such that for any $\phi \in C_c^\infty(K)$,
    %
    \[ |\Lambda(\phi)| \lesssim \| \phi \|_{C^n(K)}, \]
    %
    and so for any other compact set $K$,
    %
    \[ |\Lambda(\phi)| = |\Lambda(\phi \psi)| \lesssim \| \phi \psi \|_{C^n(K)} \lesssim \| \psi \|_{C^n(K)} \| \phi \|_{C^n(K)}. \]
    %
    which shows $\Lambda$ has order $N$. We have shown that $\Lambda$ is continuous with respect to the seminorm $\| \cdot \|_{C^N(K)}$ on $C^\infty(X)$, and so by the Hahn Banach theorem, $\Lambda$ extends uniquely to a continuous functional on $C^\infty(X)$.
\end{proof}

If $\mathcal{E}(\Omega)$ denotes $C^\infty(\Omega)$, equipped with the topology such that $f_n \to f$ if $D^\alpha f_n$ converges to $D^\alpha f$ locally uniformly for all $\alpha$, then the last theorem implies the family of compactly supported distributions embeds itself in $\mathcal{E}(\Omega)^*$. Conversely, \emph{every} element of $\mathcal{E}(\Omega)^*$ is a compactly supported distribution. Indeed, since $\mathcal{E}(\Omega)$ is a Frech\'{e}t space, if $\Lambda$ is a continuous linear functional on $\mathcal{E}(\Omega)$, then there exists a compact set $K$ and some $n > 0$ such that
%
\[ |\Lambda(\phi)| \lesssim \| \phi \|_{C^n(K)}. \]
%
It follows from this that $\Lambda$ is a distribution with support contained in $K$.

\begin{remark}
    For general compact sets $K$, it is \emph{not} true that if $\Lambda$ is a distribution supported on a set $K$, then there exists $n > 0$ such that
    %
    \[ |\Lambda(\phi)| \lesssim \| \phi \|_{C^n(K)}. \]
    %
    Suppose $K$ is not the union of finitely many compact connected sets. Then we can find a family of disjoint compact sets $\{ K_i \}$ in $K$ such that $K - (K_1 \cup \dots \cup K_n)$ is compact for any $n > 0$. Fix $x_i \in K_i$, let $x \in K$ be a limit point of this sequence, consider a sequence of numbers $\{ a_i \}$ such that $\sum a_i |x_i - x| = 1$, and $\sum a_i = \infty$, and let $\Lambda$ be the distribution
    %
    \[ \Lambda(\phi) = \sum_i a_i (\phi(x_i) - \phi(x)). \]
    %
    Then
    %
    \[ |\Lambda(\phi)| \leq \| \phi' \|_{L^\infty(\RR^d)}, \]
    %
    so $\Lambda$ is a distribution of order at most 1. On the other hand, if we choose a function $\phi \in \DD(\Omega)$ which is equal to one on a neighborhood of $K_1 \cup \dots \cup K_n$, and zero on a neighborhood of $K - (K_1 \cup \dots \cup K_n)$, then
    %
    \[ \Lambda(\phi) = \sum_{i = 1}^n a_i, \]
    %
    so we cannot have a bound of the form
    %
    \[ |\Lambda(\phi)| \lesssim \sum_{|\alpha| \leq k} \| D^\alpha \phi \|_{L^\infty(K)} \lesssim 1. \]
    %
    On the other hand, for any precompact neighborhood $U$ of $K$, we have a bound
    %
    \[ |\Lambda(\phi)| \lesssim \sum_{|\alpha| \leq 1} \| D^\alpha \phi \|_{L^\infty(U)}. \]
    %
    which is almost as good as the bound above.
\end{remark}

If $\Lambda$ is a distribution of order $k$ supported on $K$, though we do not have a uniform bound $|\Lambda(\phi)| \lesssim \sum_{|\alpha| \leq k} \| D^\alpha \phi \|_{L^\infty(K)}$, if the right hand side vanishes, so does the left hand side.

\begin{lemma}
    Suppose $\Lambda$ is a distribution of order $k$ supported on $K$, and $\phi \in C^k(\Omega)$ satisfies $D^\alpha \phi(x) = 0$ for all $|\alpha| \leq k$ and $x \in K$, then $\Lambda(\phi) = 0$.
\end{lemma}
\begin{proof}
    By a density argument, we may assume that $\phi \in C^\infty(\Omega)$ without loss of generality. Find $\chi_\varepsilon \in \DD(\Omega)$ such that $\chi_\varepsilon(x) = 1$ for $x \in K$, $\chi_\varepsilon(x) = 0$ if $d(x,K) \geq \varepsilon$, and $\| D^\alpha \chi_\varepsilon \|_{L^\infty} \lesssim \varepsilon^{-|\alpha|}$ for all $|\alpha| \leq k$. Then for any $\phi \in \DD(\Omega)$,
    %
    \[ |\Lambda(\phi)| = |\Lambda(\phi \chi_\varepsilon)| \lesssim \sum_{|\alpha| \leq k} \| D^\alpha(\phi \chi_\varepsilon) \|_{L^\infty} \lesssim \sum_{|\alpha| \leq k} \varepsilon^{|\alpha|-k} \| D^\alpha \phi \|_{L^\infty(K_\varepsilon)}. \]
    %
    For any $y \in K_\varepsilon$, we can pick $x \in K$ such that $|x - y| \leq \varepsilon$. Taylor's formula at $x$, together with the fact that all the derivatives of $\phi$ up to order $k$ vanish at $x$, implies that
    %
    \[ |(D^\alpha \phi)(y)| \lesssim \varepsilon^{k+1 - |\alpha|}. \]
    %
    Thus we conclude that $|\Lambda(\phi)| \lesssim \varepsilon$, and we can then take $\varepsilon \to 0$.
\end{proof}

The last lemma implies that the value of \emph{any distribution} $\Lambda$ of order $k$ supported on a point $x_0$ depends solely on the values $D^\alpha \phi(x_0)$ for $|\alpha| \leq k$. Thus there exists constants $\lambda_\alpha$ such that
%
\[ \Lambda(\phi) = \sum_{|\alpha| \leq k} \lambda_\alpha D^\alpha \phi(x_0). \]
%
This means that $\Lambda$ is a sum of Dirac delta functions and their derivatives. If we work harder, using the Whitney extension theorem as a black box, we can obtain a similar process for more general supports.

\begin{theorem}[Whitney]
    Let $K$ be a compact set in $\RR^d$, and for each $|\alpha| \leq k$, a function $u_\alpha \in C(K)$. If
    %
    \[ U_\alpha(x,y) = \sum_{|\alpha| \leq k} \sup_{x,y \in K} \left| u_\alpha(x) - \sum_{|\beta| \leq k - |\alpha|} u_{\alpha + \beta}(y) \cdot (x - y)^\beta / \beta! \right| \cdot |x - y|^{|\alpha| - k}  \]
    %
    for $x \neq y$, and $U_\alpha(x,x) = 0$, then provided $U_\alpha$ is continuous on $K \times K$, we can find a function $v \in C^k(\RR^d)$ such that $D^\alpha v = u_\alpha$ on $K$ for $|\alpha| \leq k$, and
    %
    \[ \sum_{|\alpha| \leq k} \| D^\alpha v \|_{L^\infty} \lesssim \sum_{|\alpha| \leq k} \| U_\alpha \|_{L^\infty(K \times K)} + \sum_{|\alpha| \leq k} \| u_\alpha \|_{L^\infty(K)}. \]
\end{theorem}

A consequence is the following strengthening of the last lemma.

\begin{lemma}
    For \emph{any} compact set $K$, and any distribution $\Lambda$ of order $k$ supported on $K$, we have
    %
    \begin{align*}
        |\Lambda(\phi)| &\lesssim \sum_{|\alpha| \leq k} \sup_{x,y \in K} \left| D^\alpha \phi(x) - \sum_{|\beta| \leq k - |\alpha|} D^{\alpha + \beta} \phi(y) \cdot (x - y)^\beta / \beta! \right| \cdot |x - y|^{|\alpha| - k}\\
        &\quad\quad + \sum_{|\alpha| \leq k} \| D^\alpha \phi \|_{L^\infty(K)}.
    \end{align*}
\end{lemma}
\begin{proof}
    To do this, we apply the Whitney extension theorem, setting $u_\alpha = D^\alpha \phi |_K$. We then apply the Whitney extension theorem to find $\psi \in C^k(\RR^n)$ extending $u_\alpha$ with the required bounds above. Then $D^\alpha(\phi - \psi) = 0$ on $K$ for all $|\alpha| \leq K$, from which it follows that $\Lambda(\phi) = \Lambda(\psi)$. The bound
    %
    \[ |\Lambda(\phi)| \lesssim \sum_{|\alpha| \leq k} \| D^\alpha \psi \|_{L^\infty}, \]
    %
    which gives the required bound above.
\end{proof}

Recall that a compact set $K$ is \emph{Whitney regular}, which means that $K$ is a finite union of compact, connected components, and for any two points $x,y \in K$ contained in a common component, there exists a rectifiable curve $\gamma$ from $x$ to $y$ with length $O(|x - y|)$.

\begin{lemma}
    If $K$ is Whitney regular, then for any distribution $\Lambda$ supported on $K$, there exists $k$ such that we have a bound
    %
    \[ |\Lambda(\phi)| \lesssim \sum_{|\alpha| \leq k} \| D^\alpha \phi \|_{L^\infty(K)}. \]    
\end{lemma}
\begin{proof}
    Fix a rectifiable unit velocity curve $\gamma: [0,L] \to K$ between two points $x$ and $y$ in $K$, and let
    %
    \[ F_\alpha(s) = D^\alpha \phi(\gamma(s)) - \sum_{|\beta| \leq k - |\alpha|} D^{\alpha + \beta} \phi(y) (\gamma(s) - y)^\beta / \beta! \]
    %
    Then $|F_\alpha(s)| \lesssim s^{k-|\alpha|} \sum_{|\beta| = k} \| D^\beta \phi \|_{L^\infty(K)}$. This is immediate if $|\alpha| = k$. For $|\alpha| < k$ we prove this result by induction, noting that the case for higher order $k$ implies that
    %
    \begin{align*}
        \left| dF_\alpha / ds \right| &\leq \sum_{i = 1}^d \left| \left( D^{\alpha + i} \phi(\gamma(s)) - \sum_{|\beta| \leq k - |\alpha|} D^{(\alpha + i) + (\beta - i)} \phi(y) (\gamma(s) - y)^{\beta - i} / (\beta - 1)! \right) \cdot \gamma_i'(s) \right|\\
        &\lesssim s^{k - |\alpha| - 1} \sum_{|\beta| = k} \| D^\beta \phi \|_{L^\infty(K)}
    \end{align*}
    %
    Integrating this inequality in $s$ together with the fact that $F_\alpha(0) = 0$ gives the higher order bound. But this means that
    %
    \begin{align*}
        \left| D^\alpha \phi(\gamma(s)) - \sum_{|\beta| \leq k - |\alpha|} D^{\alpha + \beta} \phi(y) (\gamma(s) - y)^\beta / \beta! \right| &= |F_\alpha(L)|\\
        &\lesssim L^{k-|\alpha|} \sum_{|\beta| = k} \| D^\beta \phi \|_{L^\infty(K)}.
    \end{align*}
    %
    Choosing $\gamma$ optimally gives
    %
    \[ \left| D^\alpha \phi(\gamma(s)) - \sum_{|\beta| \leq k - |\alpha|} D^{\alpha + \beta} \phi(y) (\gamma(s) - y)^\beta / \beta! \right| \lesssim |x - y|^{k - |\alpha|} \sum_{|\beta| = k} \| D^\beta \phi \|_{L^\infty(K)}. \]
    %
    The last Lemma, together with this bound, completes the proof.
\end{proof}

\begin{remark}
    Similar arguments can be used to show that if $\Lambda$ is a distribution of order $k$ supported on a compact set $K$, and there exists $\gamma \leq 1$ such that $K$ is a finite union of connected components, such that for any pair of points $x,y$ in that component, there exists a rectifiable path from $x$ to $y$ with length $O(|x - y|^\gamma)$, and $m \geq k / \gamma$, then
    %
    \[ |\Lambda(\phi)| \lesssim \sum_{|\alpha| \leq m} \| D^\alpha \phi \|_{L^\infty(K)}. \]
\end{remark}

Let us finish by considering a consequence of these results, applied to distributions supported on hyperplanes. For simplicity in notation, we assume this hyperplane is axis oriented.

\begin{theorem}
    Let $x = (x_0,x_1)$, where $x_0 \in \RR^{d_1}$, $x_1 \in \RR^{d_2}$, and $d = d_1 + d_2$. Let $H = \{ (x_0,x_1) \in \RR^d: x_1 = 0 \}$. If $\Lambda$ is a distribution of order $k$ compactly supported on $H$, then there exists distributions $\Lambda_\alpha$ of order $k - |\alpha|$ on $\RR^{d_1}$ for each $|\alpha| \leq k$, where $\alpha$ is a multi-index in the $\RR^{d_2}$ variables, and constants $\gamma_\alpha$ such that
    %
    \[ \Lambda(\phi) = \sum \Lambda_\alpha(D^\alpha \phi |_H). \]
\end{theorem}
\begin{proof}
    Fix a function $\psi \in \DD(\RR^{d_1})$ equal to one in a neighborhood of the origin. Given $\phi \in \DD(\RR^{d_1})$, all derivatives of the function
    %
    \[ \sum_{|\alpha| \leq k} D^\alpha \phi(x_0,0) \cdot (x_1^\alpha / \alpha!) \cdot \psi(x_1) = \sum_{|\alpha| \leq k} D^\alpha \phi |_H (x_0) \cdot (x_1^\alpha / \alpha!) \]
    %
    agree with $\phi$ on $H$, where $\alpha$ ranges over all derivatives in the $x_1$ direction. It follows that if we define a distribution $\Lambda_\alpha$ on $\RR^{d_1}$ such that for $\psi \in \DD(\RR^{d_1})$,
    %
    \[ \Lambda_\alpha(\psi) = \Lambda( \psi \otimes (x_1^\alpha / \alpha!)), \]
    %
    then
    %
    \[ \Lambda(\phi) = \sum_{|\alpha| \leq k} \Lambda_\alpha( D^\alpha \phi |_H ). \]
    %
    The hard part is showing that $\Lambda_\alpha$ has order $k - |\alpha|$. If the support of $\Lambda$ in the $x_0$ variable is contained in a compact ball $B$, then, because $B$ is Whitney regular,
    %
    \begin{align*}
        |\Lambda_\alpha(\psi)| &\lesssim \sum_{|\beta_1| + |\beta_2| \leq k} \| D^{\beta_1 + \beta_2} \left\{ \psi \otimes (x_1^\alpha / \alpha!) \right\} \|_{L^\infty(B \times \{ 0 \})}\\
        &= \sum_{|\beta| \leq k - |\alpha|} \| D^\beta \psi \|_{L^\infty(B)}.
    \end{align*}
    %
    This implies $\Lambda_\alpha$ has order $k-|\alpha|$.
\end{proof}

\begin{remark}
    This argument does not really need the power of the full extension theorem machinery, since the Whitney extension theorem is relatively trivial in the application we give (we can consider a simple convolution argument to extend a function on a hyperplane to the full space). But the more developed machinery can be applied to characterize distributions on more general sets, which we leave to the reader to experiment with.
\end{remark}

\section{Derivatives of Continuous Functions}

One of the main reasons to consider the theory of distributions is so that we can take the derivative of any function we want. We now show that, at least locally, every distribution is the derivative of some continuous function, which means the theory of distributions is essentially the minimal such class of objects which enable us to take derivatives of continuous functions.

\begin{theorem}
    If $\Lambda$ is a distribution on $\Omega$, and $K$ is a compact set, then there is a continuous function $f$ and $\alpha$ such that for every $\phi$,
    %
    \[ \Lambda \phi = (-1)^{|\alpha|} \int_\Omega f(x) (D^\alpha \phi)(x)\; dx \]
\end{theorem}
\begin{proof}
    TODO
\end{proof}

\begin{theorem}
    If $K$ is compact, contained in some open subset $V$, which in turn is a subset of $\Omega$, and $\Lambda$ has order $N$, then there exists finitely many continuous functions $f_\beta \in C(\Omega)$ supported on $V$, for each $|\beta| \leq N + 2$, with supports on $V$, and with $\Lambda = \sum D^\beta f_\beta$.
\end{theorem}

\begin{theorem}
    If $\Lambda$ is a distribution on $\Omega$, then there exists continuous functions $g_\alpha$ on $\Omega$ such that each compact set $K$ intersects the supports of finitely many of the $g_\alpha$, and $\Lambda = \sum D^\alpha g_\alpha$. If $\Lambda$ has finite order, then only finitely many of the $g_\alpha$ are nonzero.
\end{theorem}

\section{Convolutions of Distributions}

Using the convolution of two functions as inspiration, we will define the convolution of a distribution $\Lambda$ with a test function $\phi$, and under certain conditions, the convolution of two distributions. Recall that if $f,g \in L^1(\RR^n)$, then their convolution is the function in $L^1(\RR^n)$ defined by
%
\[ (f * g)(x) = \int f(y) g(x - y)\; dy \]
%
If we define the translation operators $T_y g(x) = \text{Trans}_y g(x) = g(x - y)$, then $(f * g)(x) = \int f(y) (T_x g^*)(y)\; dy$, where $g^*$ is the function defined by $g^*(x) = g(-x)$. Thus, if $\Lambda$ is any distribution on $\RR^n$, and $\phi$ is a test function on $\RR^n$, we can define a function $\Lambda * \phi$ by setting $(\Lambda * \phi)(x) = \Lambda(T_x \phi^*)$. Notice that since
%
\begin{align*}
    \int (T_x f)(y) g(y)\; dy &= \int f(y-x) g(y)\; dy = \int f(y) g(x+y)\; dy\\
    &= \int f(y) (T_{-x}g)(y)\; dy,
\end{align*}
%
so we can also define the translation operators on distributions by setting $(T_x \Lambda)(\phi) = \Lambda (T_{-x} \phi)$. One mechanically verifies that convolution commutes with translations, i.e. $T_x (\Lambda * \phi) = (T_x \Lambda) * \phi = \Lambda * (T_x \phi)$.

\begin{theorem}
    $\Lambda * \phi$ is $C^\infty$, and
    %
    \[ D^\alpha(\Lambda * \phi) = (D^\alpha \Lambda) * \phi = \Lambda * (D^\alpha \phi). \]
\end{theorem}
\begin{proof}
    It is easy to calculate that
    %
    \begin{align*}
        (D^\alpha \Lambda * \phi)(x) &= (D^\alpha \Lambda)(\phi^*_x) = (-1)^{|\alpha|} \Lambda(D^\alpha (T_x \phi^*))\\
        &= \Lambda(T_x (D^\alpha \phi)^*) = (\Lambda * D^\alpha \phi)(x)
    \end{align*}
    %
    If $k \in \{ 1, \dots, d \}$ and $h \in \RR$, we set
    %
    \[ (\Delta_h f)(x) = \frac{f(x + he_k) - f(x)}{h} \]
    %
    then $\Delta_h \phi$ converges to $D^k \phi$ in $\DD(\RR^d)$, and as such
    %
    \begin{align*}
      \Delta_h(\Lambda * \phi)(x) &= \frac{(\Lambda * \phi)(x + he_k) - (\Lambda * \phi)(x)}{ h}\\
      &= \Lambda \left( \frac{T_{-x - he_k} \phi^* - T_{-x} \phi^*}{h} \right)
    \end{align*}
    %
    As $h \to 0$, in $\DD(\RR^d)$ we have
    %
    \[ \frac{T_{-x - he_k} \phi^* - T_{-x} \phi^*}{h} \to - T_{-x} D_k \phi^* = T_{-x} (D_k \phi)^*. \]
    %
    Thus, by continuity,
    %
    \[ \lim_{h \to 0} \Delta_h(\Lambda * \phi)(x) = \Lambda(T_{-x} (D_k \phi)^*) = (\Lambda * D_k \phi)(x) \]
    %
    Iteration gives the general result that $\Lambda * \phi \in C^\infty(\RR^d)$. An easy calculation then shows that for each $x \in \RR^d$,
    %
    \begin{align*}
      [(D^\alpha \Lambda) * \phi](x) &= (D^\alpha \Lambda)(T_{-x} \phi^*)\\
      &= (-1)^{|\alpha|} \Lambda(T_{-x} D^\alpha \phi^*)\\
      &= \Lambda(T_{-x} (D^\alpha \phi)^*)\\
      &= (\Lambda * D^\alpha \phi)(x). \qedhere
    \end{align*}
\end{proof}

There is a certain duality going on here. Distributions can be viewed as linear functionals on $\DD(\RR^d)$, but one can also view them as a certain family of linear operators from $\DD(\RR^d) \to C^\infty(\RR^d)$ , and the convolution operator uniquely represents the distribuition. In fact, any such operator that is translation invariant and continuous can be represented as convolution by a distribution.

\begin{theorem}
  Let $T: \DD(\RR^d) \to \loc{C^\infty}(\RR^d)$ be a translation invariant continuous operator. Then there exists a distribution $\Lambda$ such that $T\phi = \Lambda * \phi$ for all $\phi \in \DD(\RR^d)$.
\end{theorem}
\begin{proof}
  If we knew $T\phi = \Lambda * \phi$ for some $\Lambda$, then we could recover $\Lambda$ since
  %
  \[ \int \Lambda(x) \phi(x)\; dx = T \phi(0). \]
  %
  Since $T$ is a continuous operator, the right hand side defines a distribution $\Lambda$, and translation invariance allows us to conclude that $T\phi = \Lambda * \phi$ for all $\phi \in \DD(\RR^d)$.
\end{proof}

\begin{example}
        A linear differential operator $P: \DD(\RR^d) \to \DD(\RR^d)$ is translation invariant, from which it follows that there exists a distribution $\Lambda$ such that $P\phi = \Lambda * \phi$. Of course, $\Lambda(\phi) = P\phi(0)$ is just given by applying the differential operator at the origin.
\end{example}

\begin{theorem}
    If $\phi, \psi \in \DD(\RR^n)$, then $\Lambda * (\phi * \psi) = (\Lambda * \phi) * \psi$.
\end{theorem}
\begin{proof}
  Let $K$ be a compact set containing the supports of $\phi$ and $\psi$. It is simple to verify that for each $x \in \RR^d$,
    %
    \[ (\phi * \psi)^*(x) = \int \phi^*(x + y) \psi(y)\; dy = \int (T_y \phi^*)(x) \psi(y)\; dy \]
    %
    since the map $y \mapsto (T_y \phi)^* \psi(y)$ is continuous, and vanishes out of the compact set $K$, we can consider the $C_c^\infty(K)$ valued integral
    %
    \[ (\phi * \psi)^* = \int_K \psi^*(y) T_y \phi^*\; ds \]
    %
    This means precisely that
    %
    \begin{align*}
        (\Lambda * (\phi * \psi))(0) &= \Lambda((\phi * \psi)^*) = \int_K \psi^*(y) \Lambda(T_y \phi^*)\; dy\\
        &= \int_K \psi^*(y) (\Lambda * \phi)(y)\; dy = ((\Lambda * \phi) * \psi)(0)
    \end{align*}
    %
    The commutativity in general results from applying the commutativity of the translation operators.
\end{proof}

A sequence $\{ \phi_n \}$ in $\DD(\RR^n)$ is known as an {\it approximate identity} in the space of distributions if $\Lambda * \phi_n$ converges to $\Lambda$ as $n \to \infty$, for every distribution $\Lambda$, and an approximate identity in the space of test functions if $\psi * \phi_n$ converges to $\psi$ in $\DD(\RR^n)$ for every $\psi \in \DD(\RR^n)$.

\begin{theorem}
    If $\{ \phi_n \}$ is a family of non-negative functions in $\DD(\RR^n)$ which are eventually supported on every neighbourhood of the origin, and all integrate to one, then $\{ \phi_n \}$ is an approximation to the identity in the space of test functions and in the space of distributions.
\end{theorem}
\begin{proof}
    It is easy to verify that if $f$ is a continuous function, then $f * \phi_n$ converges locally uniformly to $f$ as $n \to \infty$. But now we calculate that if $f \in \DD(\RR^n)$, then $D^\alpha(f * \phi_n) = (D^\alpha f) * \phi_n$ converges locally uniformly to $D^\alpha \phi$, which gives that $f * \phi$ converges to $f$ in $\DD(\RR^n)$. Now if $\Lambda$ is a distribution, and $\psi$ is a test function, then continuity gives
    %
    \begin{align*}
        \Lambda(\psi^*) &= \lim_{\delta \to 0} \Lambda(\phi_\delta * \psi) = \lim_{\delta \to 0} (\Lambda * (\phi_\delta * \psi))(0)\\
        &= \lim_{\delta \to 0} ((\Lambda * \phi_\delta) * \psi)(0) = \lim_{\delta \to 0} (\Lambda * \phi_\delta)(\psi^*)
    \end{align*}
    %
    and $\psi$ was arbitrary.
\end{proof}

If $\Lambda$ is a distribution on $\RR^n$, then the map $\phi \mapsto \Lambda * \phi$ is a linear transformation from $\DD(\RR^n)$ into $\EC(\RR^n)$, which commutes with translations. It is also continuous. To see this, we consider a fixed compact $K$, and consider the map from $C_c^\infty(K)$ to $\loc{C^\infty}(\RR^n)$. Both of these spaces are Fr\'{e}chet, so we can apply the closed graph theorem. Consider a sequence $\{ \phi_n \}$ converging in $C_c^\infty(K)$ to some $\phi$, and we suppose that $\{ \Lambda * \phi_n \}$ converges in $C_c^\infty(K)$ to a function $f$. It suffices to show that $\Lambda * \phi = f$. But we calculate that for each $x \in \RR^d$,
%
\[ f(x) = \lim_n (\Lambda * \phi_n)(x) = \lim \Lambda(T_x \phi^*_n) = \Lambda (\lim T_x \phi^*_n) = \Lambda(T_x \phi^*) = (\Lambda * \phi)(x). \]
%
Here we have used the fact that $T_x \phi_n^*$ converges to $T_x \phi^*$ in $\DD(\RR^n)$. Thus $\phi \mapsto \Lambda * \phi$ is a continuous, translation invariant operator from $\DD(\RR^n)$ to $\EC(\RR^n)$. Surprisingly, the converse is also true.

\begin{theorem}
    If $L: \DD(\RR^n) \to \EC(\RR^n)$ is a continuous linear transformation commuting with translations, then there is a distribution $\Lambda$ such that $L(\phi) = \Lambda * \phi$.
\end{theorem}
\begin{proof}
    If $L(\phi) = \Lambda * \phi$, then we would have
    %
    \[ \Lambda(\phi) = (\Lambda * \phi^*)(0) = L(\phi^*)(0) \]
    %
    and we take this as the definition of $\Lambda$ for an arbitrary operator $L$. Indeed, it then follows that $\Lambda$ is continuous because all the operations here are continuous, and because $L$ commutes with translations, we conclude
    %
    \[ (\Lambda * \phi)(x) = \Lambda(T_x \phi^*) = L(T_{-x} \phi)(0) = L(\phi)(x) \]
    %
    which gives the theorem.
\end{proof}

We now move onto the case where a distribution $\Lambda$ has compact support. Then $\Lambda$ extends to a continuous linear functional on $\EC(\RR^n)$, and we can define the convolution $\Lambda * \phi$ if $\phi \in \EC(\RR^n)$ as above. This is an extension of the convolution operator above to a continuous operator from $\mathcal{E}(\RR^n)$ to itself. The same techniques as before, or a density argument, verify that translations and derivatives are carried into the convolution.

\begin{theorem}
    If $\phi$ and $\Lambda$ have compact support on $\RR^d$, then $\Lambda * \phi$ has compact support, and moreover,
    %
    \[ \text{supp}(\Lambda * \phi) \subset \text{supp}(\Lambda) + \text{supp}(\phi). \]
    %
    The map $\phi \mapsto \Lambda * \phi$ is a continuous operator from $\mathcal{D}(\RR^d)$ to itself.
\end{theorem}
\begin{proof}
    Let $\phi$ be supported on a compact set $K$, and $\Lambda$ be supported on $K_0$. Then $(\Lambda * \phi)(x) = \Lambda(T_x \phi^*)$. Since $T_x \phi^*$ is supported on $x - K$, $(\Lambda * \phi)(x) = 0$ unless $K_0 \cap (x - K) \neq \emptyset$, i.e. $x \in K_0 + K$.

    To obtain continuity of the operator, it suffices to prove that $\phi \mapsto \Lambda * \phi$ is a continuous operator from $C_c^\infty(K)$ to $\DD(\RR^d)$ for any compact set $K$. Let $\Lambda$ be supported on a compact set $K_0$. Then $\Lambda * \phi$ is supported on $K + K_0$, and so it suffices to show $\phi \mapsto \Lambda * \phi$ is a continuous operator from $C_c^\infty(K)$ to $C_c^\infty(K + K_0)$. But this follows because the map is continuous into $\loc{C^\infty}(\RR^d)$.
\end{proof}

\begin{theorem}
    If $\Lambda$ and $\psi$ have compact support, and $\phi \in C^\infty(\RR^n)$, then
    %
    \[ \Lambda * (\phi * \psi) = (\Lambda * \phi) * \psi = (\Lambda * \psi) * \phi \]
\end{theorem}
\begin{proof}
    Let $\Lambda$ and $\psi$ be supported on some balanced compact set $K$. Let $V$ be a bounded, balanced open set containing $K$. If $\phi_0$ is a function with compact support equal to $\phi$ on $V + K$, then for $x \in V$,
    %
    \[ (\phi * \psi)(x) = \int \phi(x - y) \psi(y)\; dy = \int \phi_0(x - y) \psi(y)\; dy = (\phi_0 * \psi)(x) \]
    %
    Thus
    %
    \[ (\Lambda * (\phi * \psi))(0) = (\Lambda * (\phi_0 * \psi))(0) = ((\Lambda * \psi) * \phi_0)(0) \]
    %
    But $\Lambda * \psi$ is supported on $K + K$, so $((\Lambda * \psi) * \phi_0)(0) = ((\Lambda * \psi) * \phi)(0)$. Now we also calculate
    %
    \[ (\Lambda * (\phi * \psi))(0) = ((\Lambda * \phi_0) * \psi)(0) = ((\Lambda * \phi) * \psi)(0) \int (\Lambda * \phi_0)(-y) \psi(y) \]
    %
    where the last fact follows because $\Lambda * \phi_0$ agrees with $\Lambda * \phi$ on $K$. The general fact follows by applying the translation operators.
\end{proof}

Now we come to the grand finale, defining the convolution of two distributions. Given two distributions $\Lambda$ and $\Psi$, one of which has compact support, we define the linear operator
%
\[ L(\phi) = \Lambda * (\Psi * \phi) \]
%
This gives a continuous, translation invariant linear operator from $\DD(\RR^d)$ to $\EC(\RR^d)$; if $\Psi$ is compactly supported, then $\phi \mapsto \Psi * \phi$ is a continuous operator on $\DD(\RR^d)$, which gives continuity. If $\Lambda$ is compactly supported, then $\Lambda \mapsto \Lambda * \eta$ is a continuous linear operator on $\EC(\RR^d)$, which gives continuity. But this means that $L$ corresponds to convolution with a distribution, and we define this distribution to be $\Lambda * \Psi$.

\begin{theorem}
    If $\Lambda$ and $\Psi$ are distributions, one of which has compact support, then $\Lambda * \Psi = \Psi * \Lambda$. Let $S_\Lambda$ and $S_\Psi$, and $S_{\Lambda * \Psi}$ denote the supports of $\Lambda$, $\Psi$, and $\Lambda * \Psi$. Then $S_{\Lambda * \Psi} \subset S_\Lambda + S_\Psi$.
\end{theorem}
\begin{proof}
    We calculate that for any two test functions $\phi$ and $\psi$,
    %
    \[ (\Lambda * \Psi) * (\phi * \psi) = \Lambda * (\Psi * (\phi * \psi)) = \Lambda * ((\Psi * \phi) * \psi) \]
    %
    If $\Lambda$ has compact support, then
    %
    \[ \Lambda * ((\Psi * \phi) * \psi) = (\Lambda * \psi) * (\Psi * \phi) \]
    %
    Conversely, if $\Psi$ has compact support, then
    %
    \[ \Lambda * ((\Psi * \phi) * \psi) = \Lambda * (\psi * (\Psi * \phi)) = (\Lambda * \psi) * (\Psi * \phi) \]
    %
    We also calculate
    %
    \begin{align*}
        \Psi * ((\Lambda * \phi) * \psi) &= \Psi * (\Lambda * (\phi * \psi)) = \Psi * (\Lambda * (\psi * \phi))\\
        &= \Psi * ((\Lambda * \psi) * \phi) = (\Psi * \phi) * (\Lambda * \psi)
    \end{align*}
    %
    But since convolution is commutative, we have
    %
    \[ ((\Lambda * (\Psi * \phi)) * \psi) = \Lambda * ((\Psi * \phi) * \psi) = \Psi * ((\Lambda * \phi) * \psi) = (\Psi * (\Lambda * \phi)) * \psi \]
    %
    Since $\psi$ was arbitrary, we conclude
    %
    \[ (\Lambda * \Psi) * \phi = \Lambda * (\Psi * \phi) = \Psi * (\Lambda * \phi) = (\Psi * \Lambda) * \phi \]
    %
    and now since $\phi$ was arbitrary, we conclude $\Lambda * \Psi = \Psi * \Lambda$. Now we know convolution is commuatative, we may assume $S_\Psi$ is compact. The support of $\Psi * \phi^*$ lies in $S_\Psi - S_\phi$. But this means that if $S_\phi - S_\Psi$ is disjoint from $S_\Lambda$, which means exactly that $S_\phi$ is disjoint from $S_\Lambda + S_\Psi$, then
    %
    \[ (\Lambda * \Psi)(\phi) = (\Lambda * (\Psi * \phi))(0) = 0 \]
    %
    and this gives the support of $\Lambda * \Psi$.
\end{proof}

This means that the convolution of two distributions with compact support also has compact support. This means that if we have three distributions $\Lambda, \Psi$, and $\Phi$, two of which have compact support, then the distributions $\Lambda * (\Psi * \Phi)$ and $(\Lambda * \Psi) * \Phi$ are well defined, so convolution is associative and commutative. We calculate that for any test function $\phi$,
%
\[ (\Lambda * (\Psi * \Phi)) * \phi = \Lambda * (\Psi * (\Phi * \phi)) \]
\[ ((\Lambda * \Psi) * \Phi) * \phi = (\Lambda * \Psi) * (\Phi * \phi) \]
%
If $\Phi$ has compact support, then $\Phi * \phi$ has compact support, and so we can move $(\Lambda * \Psi)$ into the equation to prove equality. If $\Phi$ does not have compact support, then $\Lambda$ and $\Psi$ have compact support, and
%
\[ \Lambda * (\Psi * \Phi) = \Lambda * (\Phi * \Psi) \]
%
and we can apply the previous case to obtain that this is equal to $(\Lambda * \Phi) * \Psi$. Repeatedly applying the previous case brings this to what we want.

\begin{theorem}
    If $\Lambda$ and $\Psi$ are distributions, one of which having compact support, then
    %
    \[ D^\alpha(\Lambda * \Psi) = (D^\alpha \Lambda) * \Psi = \Lambda * (D^\alpha \Psi). \]
\end{theorem}
\begin{proof}
    The Dirac delta function $\delta$ satisfies
    %
    \[ (\delta * \phi)(x) = \int \phi(y) \delta(x-y)\; dy = \phi(x) \]
    %
    so $\delta * \phi = \phi$. Now $D^\alpha \delta$ is also supported at $x$, since
    %
    \[ (D^\alpha \delta)(\phi) = (-1)^{|\alpha|} \int \delta(x) (D^\alpha \phi)(x)\; dx = (-1)^{|\alpha|} (D^\alpha \phi)(0) \]
    %
    which means that for any distribution $\Lambda$, then $(D^\alpha \delta) * \Lambda$ has compact support,
    %
    \[ (((D^\alpha \delta) * \Lambda) * \phi)(0) = (D^\alpha \delta)((\Lambda * \phi)^*) = (-1)^{|\alpha|} D^\alpha (\Lambda * \phi)^* = ((D^\alpha \Lambda) * \phi)(0) \]
    %
    which verifies that $(D^\alpha \delta) * \Lambda = \delta * (D^\alpha \Lambda)$. But now we find
    %
    \[ D^\alpha(\Lambda * \Psi) = (D^\alpha \delta) * \Lambda * \Psi = ((D^\alpha \delta) * \Lambda) * \Psi = D^\alpha \Lambda * \Psi \]
    \[ D^\alpha(\Lambda * \Psi) = D^\alpha(\Psi * \Lambda) = (D^\alpha \Psi) * \Lambda = \Lambda * (D^\alpha \Psi) \]
    %
    which verifies the theorem in general.
\end{proof}

\section{Schwartz Space and Tempered Distributions}

We have already encountered the fact that Fourier transforms are well behaved under differentiation and multiplication by polynomials. If we let $\mathcal{S}(\RR^d)$ denote a class of functions under which to study this phenomenon, it must be contained in $L^1(\RR^d)$ and $C^\infty(\RR^d)$, and closed under multiplication by polynomials, and closed under applications of arbitrary constant-coefficient differential operators. A natural choice is then the family of functions which \emph{decays rapidly}, as well as all of it's derivatives; i.e. we let $\mathcal{S}(\RR^d)$ be the space of all functions $f \in C^\infty(\RR^d)$ such that for any integer $n$ and multi-index $\alpha$, $|x|^n D^\alpha f \in L^\infty(\RR^d)$. The space $\mathcal{S}(\RR^d)$ is then locally convex if we consider the family of seminorms
%
\[ \| f \|_{\mathcal{S}^{n,m}(\RR^d)} = \sup_{|\alpha| \leq n} \| \langle x \rangle^m D^\alpha f \|_{L^\infty(\RR^d)}. \]
%
Elements of $\mathcal{S}(\RR^d)$ are known as \emph{Schwartz functions}, and $\mathcal{S}(\RR^d)$ is often known as the \emph{Schwartz space}. The seminorms naturally give $\mathcal{S}(\RR^d)$ the structure of a Fr\'{e}chet space. Sometimes, it is more convenient to use the equivalent family of seminorms $\| f \|_{\mathcal{S}^{\alpha, \beta}(\RR^d)} = \| x^\alpha D^\beta f \|_{L^\infty(\RR^d)}$, because $x^\alpha$ often behaves more nicely under various Fourier analytic operations. It is obvious that $\mathcal{S}(\RR^d)$ is separated by the seminorms defined on it, because $\| \cdot \|_{L^\infty(\RR^d)} = \| \cdot \|_{\mathcal{S}^{0,0}(\RR^d)}$ is a norm used to define the space. We now show the choice of seminorms make the space complete.

\begin{theorem}
    $\mathcal{S}(\RR^d)$ is a complete metric space.
\end{theorem}
\begin{proof}
    Let $\{ f_i \}$ be a Cauchy sequence with respect to the seminorms $\| \cdot \|_{\mathcal{S}^{n,\alpha}(\RR^d)}$. This implies that for each integer $m$, and multi-index $\alpha$, the sequence of functions $\langle x \rangle^m D^\alpha f_k$ is Cauchy in $L^\infty(\RR^d)$. Since $L^\infty(\RR^d)$ is complete, there are functions $g_{m,\alpha}$ such that $\langle x \rangle^m D^\alpha f_k$ converges uniformly to $g_{m,\alpha}$. If we set $f = g_{0,0}$, then it is easy to see using the basic real analysis of uniform continuity that $f$ is infinitely differentiable, and $\langle x \rangle^m D^\alpha f = g_{m,\alpha}$. It is then easy to show that $f_i$ converges to $f$ in $\mathcal{S}(\RR^d)$.
\end{proof}

\begin{example}
    The Gaussian function $\phi: \RR^d \to \RR$ defined by $\phi(x) = e^{-|x|^2}$ is Schwartz. For any multi-index $\alpha$, there is a polynomial $P_\alpha$ of degree at most $|\alpha|$ such that $D^\alpha \phi = P_\alpha \phi$; this can be established by a simple induction. But this means that for each fixed $\alpha$, $|P_\alpha(x)| \lesssim_\alpha 1 + |x|^{|\alpha|}$. Since $e^{-|x|^2} \lesssim_{m,\alpha} \langle x \rangle^{-m -|\alpha|}$ for any fixed $m$ and $\alpha$, we find that for any $x \in \RR^d$,
    %
    \[ | (1 + |x|^m) D^\alpha \phi| \lesssim_{\alpha,m} 1. \]
    %
    Since $m$ and $\alpha$ were arbitrary, this shows $\phi$ is Schwartz.
\end{example}

\begin{example}
    Any compactly supported smooth function is Schwartz. In particular, the inclusion map
    %
    \[ \DD(\RR^d) \to \SW(\RR^d) \]
    %
    is bounded. The proof is left to the reader. An important consequence is that elements of $\SW(\RR^d)^*$, which we will call \emph{tempered distributions}, can be viewed as a subspace of the space $\DD(\RR^d)^*$ of all distributions.
\end{example} 

To show that an operator $T$ on $\mathcal{S}(\RR^d)$ is bounded, it suffices to show that for each $n_0$ and $m_0$, there is $n_1$, $m_1$ such that
%
\[ \| Tf \|_{\mathcal{S}^{n_0,m_0}(\RR^d)} \lesssim_{n_0,m_0} \| f \|_{\mathcal{S}^{n_1,m_1}(\RR^d)}. \]
%
For a functional $\Lambda: \mathcal{S}(\RR^d) \to \RR$, it suffices to show that there exists $n$ and $m$ such that $|\Lambda f| \lesssim \| f \|_{\mathcal{S}^{n,m}(\RR^d)}$. The minimal such choice of $n$ is known as the \emph{order} of the functional $\Lambda$. We normally do not care about the constant behind the operators for these norms, since the norms are not translation invariant and therefore highly sensitive to the positions of various functions. We really just care about proving the existence of such a constant.
%\begin{lemma}
%  The map $(f,g) \mapsto fg$ for $f,g \in \mathcal{S}(\RR^d)$ gives a bounded bilinear map from $\mathcal{S}(\RR^d) \times \mathcal{S}(\RR^d) \to \mathcal{S}(\RR^d)$.
%\end{lemma}
%\begin{proof}
%  A simple application of the Leibnitz formula shows that for any multi-index $\alpha$ with $|\alpha| = m$, and two non-negative integers $n_1$ and $n_2$ with $n_1 + n_2 = n$,
  %
%  \[ \| fg \|_{\mathcal{S}^{n,\alpha}(\RR^d)} \lesssim_n \| f \|_{\mathcal{S}^{n_1,m}(\RR^d)} \| g \|_{\mathcal{S}^{n_2,m}(\RR^d)}. \]
  %
%  More generally, this argument shows that the analogoue bilinear map from $C^\infty(\RR^d) \times \mathcal{S}(\RR^d) \to \mathcal{S}(\RR^d)$ is bounded.
%\end{proof}
%
Here are some examples:
%
\begin{itemize}
    \item Multiplication gives a continuous bilinear map
    %
    \[ \SW(\RR^d) \times \SW(\RR^d) \to \SW(\RR^d). \]
    %
    More generally, the bilinear multiplication map
    %
    \[ C^\infty_b(\RR^d) \times \SW(\RR^d) \to \SW(\RR^d) \]
    %
    is continuous.

    \item For each $x \in \RR^d$, the translation operator
    %
    \[ \text{Trans}_x: \SW(\RR^d) \to \SW(\RR^d) \]
    %
    is an isomorphism. For each $\xi \in \RR^d$, the modulation operator
    %
    \[ \text{Mod}_\xi: \SW(\RR^d) \to \SW(\RR^d) \]
    %
    is an isomorphism.

    \item The $L^p$ norms are continuous.

    \item The Fourier transform is an isomorphism of $\SW(\RR^d)$.
\end{itemize}
%
The last point follows by an application, e.g. of the compatibility of differentiation and polynomial multiplication with the Fourier transform.
%\end{theorem}
%\begin{proof}
%   Let $(T_h f)(x) = f(x - h)$. We calculate that if $|\alpha| \leq n$, then
    %
%   \begin{align*}
%       (1 + |x|^m) (T_h f)_\alpha &= T_h((1 + |x + h|^m) f_\beta)\\
%       &\leq 2^m T_h((1 + |x|^m + |h|^m) f_\alpha)\\
%       &\leq 2^m |h|^m \| f_\alpha \|_{n,0} + 2^m \| f \|_{n,m}.
%   \end{align*}
    %
%   Thus $\| T_h f \|_{n,m} \leq 2^m(1 + |h|^m) \| f \|_{n,m}$, so $T_h$ is continuous.

%   Similarily, we calculate using the Leibnitz formula and the formula for the derivatives of $e(\xi \cdot x)$ that if $|\alpha| \leq n$, then
    %
%   \[ (1 + |x|^m) |(e(\xi \cdot x) f)_\alpha| \leq 4^n (2\pi)^n (1 + |\xi|^n) \| f \|_{n,m} \]
    %
%   Thus $\| M_\xi f \|_{n,m} \leq (8 \pi)^n (1 + |\xi|^n) \| f \|_{n,m}$.

%   For any Schwartz function $f$, and $|\alpha| \leq n$,
    %
%   \[ f(x) \leq \frac{\| f \|_{0,d+1}}{1 + |x|^{d+1}} \]
    %
%   Integrating this equation gives
    %
%   \[ \| f_\alpha \|_{L^1(\RR^d)} \leq 2^d \| f \|_{0,d+1}. \]
    %
%   Thus $\| \cdot \|_1$ is a bounded norm on the space. Interpolation then shows that for any $1 < p < \infty$,
    %
%   \[ \| f \|_{L^p(\RR^d)} \leq \| f \|_{L^1(\RR^d)}^{1 - 1/p} \| f \|_{L^\infty(\RR^d)}^{1/p} \leq \| f \|_{L^1(\RR^d)} + \| f \|_{L^\infty(\RR^d)} \leq 2 \| f \|_{0,d+1}. \]
    %
%   This implies $\| \cdot \|_{L^p(\RR^d)}$ is bounded.

%   A simple calculation using the Leibnitz formula shows that if $|\alpha| \leq n$,
    %
%   \begin{align*}
%       (1 + |x|^m) |\mathcal{F}(f)_\alpha| &\leq |\mathcal{F}(f)_\alpha| + \sum_{k = 1}^d |x_k^m \mathcal{F}(f)_\alpha|\\
%       &\leq (2 \pi)^n \left( \| \mathcal{F} f \|_{L^\infty(\RR^d)} + \sum_{k = 1}^d |\mathcal{F}((x^\alpha f)_{me_k})| \right)\\
%       &\leq n! (2 \pi)^n 2^m (n+1) \max_{0 \leq k \leq d} \max_{1 \leq l \leq m} \left( \| \mathcal{F} f \|_{L^\infty(\RR^d)} + \sum_{k = 1}^n \max_{1 \leq l \leq m} \| \mathcal{F}(f_{le_k}) \|_{L^\infty(\RR^d)} \right)\\
%       &\leq n! (2 \pi)^n 2^m \left( \| f \|_{L^1(\RR^d)} + \sum_{k = 1}^n \max_{1 \leq l \leq m} \| f_{le_k} \|_{L^1(\RR^d)} \right)\\
%       &\leq n! (2 \pi)^n 2^m 2^d (n+1) \| f \|_{n,d+1}.
%   \end{align*}

%   there are constants $c_{\alpha \beta \gamma}$ for each $\gamma \leq \alpha \wedge \beta$ such that
    %
%   \begin{align*}
%       |x^\alpha \mathcal{F}(f)_\beta| &= (2 \pi)^{|\beta|} |x^\alpha \cdot \mathcal{F}(x^\beta f)|\\
%       &= (2\pi)^{|\beta| - |\alpha|} \mathcal{F}((x^\beta f)_\alpha)\\
%       &\leq (2\pi)^{|\beta| - |\alpha|} \sum_{\gamma \leq \alpha \wedge \beta} c_{\alpha \beta \gamma} |\mathcal{F}(x^{\beta - \gamma} f_{\alpha - \gamma})|.
%   \end{align*}
    %
%   This calculation shows
    %
%   \begin{align*}
%       \| \mathcal{F} f \|_{\alpha,\beta} &\lesssim_{\alpha,\beta} \sum \| \mathcal{F}(x^{\beta - \gamma} f_{\alpha - \gamma}) \|_{L^\infty(\RR^n)}\\
%       &\leq \sum \| x^{\beta - \gamma} f_{\alpha - \gamma} \|_{L^1(\RR^n)}.
%   \end{align*}
    %
%   The right hand side is a continuous function of $f$, so the Fourier transform is bounded. The smoothness of the Schwartz space implies that $\mathcal{F}$ is a bijective map. But then the open mapping theorem implies that $\mathcal{F}^{-1}$ is a bounded operation, and therefore $\mathcal{F}$ is a homeomorphism.

%    We leave all but the last point as exercises. Here it will be convenient to use the norms $\| \cdot \|_{\mathcal{S}^{\alpha,\beta}(\RR^d)}$ as well as the norms $\| \cdot \|_{\mathcal{S}^{n,m}(\RR^d)}$. If $|\alpha| \leq m$, $|\beta| \leq n$, then we can use the Leibnitz formula to conclude that
    %
%    \begin{align*}
%        \| \xi^\alpha D^\beta \mathcal{F}(f) \|_{L^\infty(\RR^d)} &\lesssim_{\alpha,\beta} \| \mathcal{F}(D^\alpha(x^\beta f)) \|_{L^\infty(\RR^d)}\\
%        &\lesssim_{\alpha,\beta} \max_{\gamma \leq \alpha \wedge \beta} \| \mathcal{F}(x^{\gamma} D^\gamma f) \|_{L^\infty(\RR^d)}\\
%        &\leq \max_{\gamma \leq \alpha \wedge \beta} \| x^\gamma D^\gamma f \|_{L^1(\RR^d)}\\
%        &\lesssim \| f \|_{\mathcal{S}^{\gamma,|\gamma| + d+1}(\RR^d)}.
%    \end{align*}
    %
%    Thus $\mathcal{F}$ is a bounded linear operator on $\mathcal{S}(\RR^d)$. Since all Schwartz functions are arbitrarily smooth, the Fourier inversion formula applies to all Schwartz functions, and so $\mathcal{F}$ is a bijective bounded linear operator with inverse $\mathcal{F}^{-1}$. The open mapping theorem then immediately implies that $\mathcal{F}^{-1}$ is bounded.
%\end{proof}

\begin{corollary}
    Convolution is a continuous operator
    %
    \[ \SW(\RR^d) \times \SW(\RR^d) \to \SW(\RR^d). \]
\end{corollary}
\begin{proof}
    We have
    %
    \[ f * g = \mathcal{F}^{-1} \{ \mathcal{F} \{ f \} \cdot \mathcal{F} \{ g \} \}, \]
    %
    and the result then follows from the previous examples.
\end{proof}

\begin{remark}
    This result allows us to define the convolution of any two tempered distributions in a natural way extending the theory of convolution of distributions defined before.
\end{remark}

Now we get to the interesting part of the theory of Schwartz functions. We have defined a homeomorphic linear transform from $\mathcal{S}(\RR^d)$ to itself. The theory of functional analysis then says that we can define a dual map, which is a homeomorphism from the dual space $\SW(\RR^d)^*$ to itself. Note the inclusion map $\DD(\RR^d) \to \mathcal{S}(\RR^d)$ is continuous, and $\DD(\RR^d)$ is dense in $\mathcal{S}(\RR^d)$. This implies that we have an injective, continuous map from $\SW(\RR^d)^*$ to $\DD(\RR^d)^*$, so every functional on the Schwarz space can be identified with a distribution. We call such distributions \emph{tempered}. They are precisely the linear functionals on $\DD(\RR^d)$ which have a continuous extension to $\mathcal{S}(\RR^d)$. Intuitively, this corresponds to having limited growth at infinity.

\begin{example}
    Recall that for any $f \in L^1_{\text{loc}}(\RR^d)$, we can consider the distribution $\Lambda[f]$ defined by setting
    %
    \[ \Lambda[f](\phi) = \int f(x) \phi(x)\; dx. \]
    %
    However, this distribution is not always tempered. If $f \in L^p(\RR^d)$ for some $p$, then, applying H\"{o}lder's inequality, we obtain that
    %
    \[ |\Lambda[f](\phi)| \leq \| f \|_{L^p(\RR^d)} \| \phi \|_{L^q(\RR^d)}. \]
    %
    Since $\| \cdot \|_{L^q(\RR^d)}$ is a continuous norm on $\mathcal{S}(\RR^d)$, this shows $\Lambda[f]$ is bounded. More generally, if $f \in L^1_{\text{loc}}(\RR^d)$, and $f(x) (1 + |x|)^{-m}$ is in $L^p(\RR^d)$ for some $m$, then $\Lambda[f]$ is a tempered distribution. If $p = \infty$, such a function is known as \emph{slowly increasing}.
\end{example}

\begin{example}
    For any Radon measure, $\mu$, we can define a distribution
    %
    \[ \Lambda[\mu](\phi) = \int \phi(x) d\mu(x) \]
    %
    But this distribution is not always tempered. If $|\mu|$ is finite, the inequality $\| \Lambda[\mu](\phi) \| \leq \| \mu \| \| \phi \|_{L^\infty(\RR^d)}$ gives boundedness. More generally, if $\mu$ is a measure such that for some $n$,
    %
    \[ \int_{\RR^d} \frac{d|\mu|(x)}{1 + |x|^n}\; dx < \infty \]
    %
    then $\mu$ is known as a \emph{tempered measure}, and acts as a tempered distribution since
    %
    \begin{align*}
      |\Lambda[\mu](\phi)| &\leq \int_{\RR^d} |\phi(x)|\; d|\mu|(x)\\
      &\leq \left( \int_{\RR^d} \frac{d|\mu|(x)}{1 + |x|^n}\; dx \right) \cdot \| \phi \|_{\mathcal{S}^{0,n}(\RR^d)}.
    \end{align*}
\end{example}

\begin{example}
  Any compactly supported distribution is tempered. Indeed, if $\Lambda$ is a distribution supported on a compact set $K$, then it has finite order $n$ for some integer $n$, and extends to an operator on $C^\infty(\RR^d)$. We then find
  %
  \[ |\Lambda(\phi)| \lesssim \| \phi \|_{C^n(\RR^d)} \leq \| \phi \|_{\mathcal{S}^{0,n}(\RR^d)}. \]
\end{example}

\begin{example}
  The distribution $\Lambda$ on $\RR$ given by
  %
  \[ \Lambda(\phi) = \text{p.v} \int_{-\infty}^\infty \frac{\phi(x)}{x}\; dx \]
  %
  is tempered, since
  %
  \[ \int_{|x| \geq 1} \frac{\phi(x)}{x} \lesssim \| \phi \|_{\mathcal{S}^{1,0}(\RR^d)} \]
  %
  and
  %
  \[ \text{p.v} \int_{-\infty}^\infty \frac{\phi(x)}{x}\; dx \lesssim \| \phi \|_{C^1(\RR^d)} = \| \phi \|_{\mathcal{S}^{0,1}(\RR^d)} \]
  %
  and so $\Lambda$ is tempered of order 1. This distribution is called the \emph{(Cauchy) principal value} of $1/x$, often denoted $\text{p.v}(1/x)$.
\end{example}

The derivative of a tempered distribution is tempered, and gives a continuous operator on $\SW(\RR^d)^*$. Multiplication gives a continuous map
%
\[ C^\infty_b(\RR^d) \times \SW(\RR^d)^* \to \SW(\RR^d)^*. \]
%
Furthermore, multiplication by a polynomial is also a continuous operator on $\SW(\RR^d)$, or more generally, by any smooth function whose derivatives are all slowly increasing.

Let us now apply the distributional method to define the Fourier transform of a tempered distribution. Recall that we heuristically think of $\Lambda$ as formally corresponding to a regular function $f$ such that
%
\[ \Lambda(\phi) = \int f(x) \phi(x)\; dx \]
%
The multiplication formula
%
\[ \int_{\RR^d} \widehat{f}(\xi) g(\xi)\; d\xi = \int_{\RR^d} f(x) \widehat{g}(x)\; dx \]
%
gives us the perfect opportunity to move the analytical operations on $f$ to analytical operations on $g$. Thus if $\Lambda$ is the distribution corresponding to a Schwartz $f \in \mathcal{S}(\RR^d)$, the distribution $\widehat{\Lambda}$ corresponding to $\widehat{f}$, then for any Schwartz $\phi \in \mathcal{S}(\RR^d)$,
%
\[ \widehat{\Lambda}(\phi) = \Lambda \left( \widehat{g} \right). \]
%
In particular, this motivates us to define the Fourier transform of \emph{any} tempered distribution $\Lambda$ to be the unique tempered distribution $\widehat{\Lambda}$ such that the equation above holds for all Schwartz $\phi$. This distribution exists because the Fourier transform is an isomorphism on the space of Schwartz functions. Clearly, the Fourier transform is a homeomorphism on the space of tempered distributions under the weak topology, and moreover, satisfies all the symmetry properties that the ordinary Fourier transform does, once we interpret scalar, rotation, translation, differentiation, etc, in a natural way on the space of distributions.

\begin{example}
    Consider the constant function $1$, interpreted as a tempered distribution on $\RR^d$. Then for any $\phi \in \mathcal{S}(\RR^d)$,
    %
    \[ 1(\phi) = \int \phi(x)\; dx, \]
    %
    Thus for any $\phi \in \mathcal{S}(\RR^d)$,
    %
    \[ \widehat{1} \left( \widehat{\phi} \right) = 1(\phi) = \int \phi(\xi)\; d\xi = \widehat{\phi}(0). \]
    %
    Thus $\widehat{1}$ is the Dirac delta function at the origin. Similarily, the Fourier inversion formula implies that
    %
    \[ \widehat{\delta} \left( \widehat{\phi} \right) = \phi(0) = \int \widehat{\phi}(\xi)\; d\xi = 1 \left( \widehat{\phi} \right) \]
    %
    so the Fourier transform of the Dirac delta function is the constant 1 function.
\end{example}

\begin{example}
    Let $u$ denote a compactly supported distribution. We claim that $\widehat{u} \in C^\infty(\RR^d)$ is a smooth function, such that
    %
    \[ \widehat{u}(\xi) = \langle u, e^{-2 \pi i \xi \cdot x} \rangle. \]
    %
    Indeed, formally speaking,
    %
    \[ \langle \widehat{u}, \phi \rangle = \langle u, \widehat{\phi} \rangle = \langle u, \int \phi(x) e^{-2 \pi i \xi \cdot x}\; dx \rangle = \int \phi(x) \langle u, e^{-2 \pi i \xi \cdot x} \rangle\; dx. \]
    %
    The proof that $\widehat{u}$ is smooth follows because we have control of the derivatives of $e^{-2 \pi i \xi \cdot x}$ on the support of $u$. 
\end{example}

\begin{example}
  The theory of tempered distributions enables us to take the Fourier transform of $f \in L^p(\RR^d)$, when $p > 2$ or when $p < 1$. The introduction of distributions is in some sense, essential to this process, because for each $p \not \in [1,2]$, there is $f \in L^p(\RR^d)$ such that $\widehat{f}$ is \emph{not} a locally integrable function. Otherwise, we could define an operator $T: L^p(\RR^d) \to L^1(\RR^d)$ given by
  %
  \[ Tf = \widehat{f} \mathbf{I}_{|\xi| \leq 1}. \]
  %
  If a sequence of functions $\{ f_n \}$ converges to $f$ in $L^p(\RR^d)$, and $Tf_n$ converges to $g$ in $L^1(\RR^d)$, then $Tf_n$ converges distributionally to $g$, which implies $Tf = g$. The closed graph theorem thus implies that $T$ is a continuous operator from $L^p(\RR^d)$ to $L^1(\RR^d)$, so there exists $M > 0$ such that
  %
  \[ \int_{|\xi| \leq 1} |\widehat{f}(\xi)| \leq M \| f \|_{L^p(\RR^d)}. \]
  %
  If $f_\alpha(x) = e^{-\pi \alpha |x|^2}$, then $\widehat{f_\alpha}(\xi) = \alpha^{-d/2} e^{-\pi |x|^2 / \alpha}$. We have
  %
  \begin{align*}
    \| f_\alpha \|_{L^p(\RR^d)} &= \left( \int_{\RR^d} e^{- \pi \alpha p |x|^2}\; dx \right)^{1/p}\\
    &= (\alpha p)^{-d/2p} \left( \int_{\RR^d} e^{- \pi |x|^2}\; dx \right)^{1/p} \lesssim_d (\alpha p)^{-1/2p}.
  \end{align*}
  %
  On the other hand, for $|\xi| \leq 1$, $|\widehat{f_\alpha}(\xi)| \geq \alpha^{-d/2} e^{-\pi/\alpha}$, so
  %
  \[ \int_{|\xi| \leq 1} |\widehat{f_\alpha}(\xi)| \gtrsim_d \alpha^{-d/2} e^{-\pi/\alpha}. \]
  %
  Thus we conclude that $\alpha^{-d/2} e^{-\pi/\alpha} \lesssim_d M (\alpha p)^{-d/2p}$, or equivalently,
  %
  \[ \alpha^{d/2(1/p-1)} e^{-\pi/\alpha} \lesssim_d M p^{-d/2p}. \]
  %
  Taking $\alpha \to \infty$ gives a contradiction if $p < 1$. For $p > 2$, we give the Gaussian an oscillatory factor that does not affect the $L^p$ norm but boosts the $L^1$ norm of the Fourier transform. We set
  %
  \[ g_\delta(x) = \prod_{k = 1}^d \frac{e^{- \pi x_k^2 / (1 + i \delta)}}{(1 + i \delta)^{1/2}}. \]
  %
  The Fourier transform formula of the Gaussian, when applied using the theory of analytic continuation, shows that
  %
  \[ \widehat{g_\delta}(\xi) = \prod_{k = 1}^d e^{- \pi (1 + i \delta) \xi_k^2}. \]
  %
  We have
  %
  \[ \int_{|\xi| \leq 1} |\widehat{g_\delta}(\xi)| = \int_{|\xi| \leq 1} e^{- \pi |\xi|^2} \gtrsim 1. \]
  %
  On the other hand, for $\delta \geq 1$,
  %
  \begin{align*}
    \| g_\delta \|_{L^p(\RR^d)} &= \left( \int |g_\delta(x)|^p\; dx \right)^{1/p}\\
    &= |1 + i \delta|^{-d/2} \left( \int_{-\infty}^\infty e^{- p \pi x^2/(1 + \delta^2)}\; dx \right)^{d/p}\\
    &\lesssim_d \delta^{-d/2} \delta^{d/p} p^{-d/p} = \delta^{d(1/p - 1/2)} p^{-d/p}.
  \end{align*}
  %
  Thus we conclude $1 \lesssim_d M \delta^{d(1/p - 1/2)} p^{d/p}$, which gives a contradiction as $\delta \to \infty$ if $p > 2$.
\end{example}

\begin{example}
  Consider the Riesz Kernel on $\RR^d$, for each $\alpha \in \CC$ with positive real part, as the function
  %
  \[ K_\alpha(x) = \frac{\Gamma(\alpha/2)}{\pi^{\alpha/2}} |x|^{-\alpha}. \]
  %
  Then for $0 < \text{Re}(\alpha) < d$, $\widehat{K_\alpha} = K_{d-\alpha}$. We recall that $\Gamma$ is defined by the integral formula
  %
  \[ \Gamma(s) = \int_0^\infty e^{-t} t^{s-1}\; ds, \]
  %
  where $\text{Re}(s) > 0$. We note that if $p = d/\text{Re}(\alpha)$, $K_\alpha \in L^{p,\infty}(\RR^d)$. The Marcinkiewicz interpolation theorem implies that if $d/2 < \text{Re}(\alpha) < d$, then $K_\alpha$ can be decomposed as the sum of a $L^1(\RR^d)$ function and a $L^2(\RR^d)$ function, and so we can intepret the Fourier transform of $\widehat{K_\alpha}$ using techniques in $L^1(\RR^d)$ and $L^2(\RR^d)$, and moreover, the Marcinkiewicz interpolation theorem implies that
  %
  \[ \| \widehat{K_\alpha} \|_{L^{q,\infty}(\RR^d)} \leq \| K_\alpha \|_{L^{p,\infty}(\RR^d)}. \]
  %
  where $q$ is the dual of $p$. In particualr, the Fourier transform of $K_\alpha$ is a function. We note that $K_\alpha$ obeys multiple symmetries. First of all, $K_\alpha$ is radial, so $\widehat{K_\alpha}$ is also radial. Moreover, $K_\alpha$ is homogenous of degree $-\alpha$, i.e. for each $x \in \RR^d$, $K_\alpha(\varepsilon x) = \varepsilon^{-\alpha} K_\alpha(x)$. This actually uniquely characterizes $K_\alpha$ among all locally integrable functions. Taking the Fourier transform of both sides of the equation for homogeneity, we find
  %
  \[ \varepsilon^{-d} \widehat{K_\alpha}(\xi/\varepsilon) = \varepsilon^{-\alpha} \widehat{K_\alpha}(x). \]
  %
  Thus $\widehat{K_\alpha}$ is homogenous of degree $\alpha - d$. But this uniquely characterizes $\widehat{K_{d-\alpha}}$ out of any distribution, up to multiplicity, so we conclude that for $d/2 < \text{Re}(\alpha) < d$, that $\widehat{K_\alpha}$ is a scalar multiple of $K_{d-\alpha}$. But we know that by a change into polar coordinates, if $A_d$ is the surface area of a unit sphere in $\RR^d$, then
  %
  \begin{align*}
    \int_{\RR^d} K_\alpha(x) e^{- \pi |x|^2}\; dx &= \frac{\Gamma(\alpha/2)}{\pi^{\alpha/2}} \int_{\RR^d} |x|^{-\alpha} e^{-\pi |x|^2}\; dx\\
    &= A_d \frac{\Gamma(\alpha/2)}{\pi^{\alpha/2}} \int_0^\infty r^{d-1-\alpha} e^{- \pi r^2}\; dr\\
    &= A_d \frac{\Gamma(\alpha/2)}{2 \pi^{d/2}} \int_0^\infty s^{(d-\alpha)/2 - 1} e^{-s}\; ds\\
    &= A_d \frac{\Gamma(\alpha/2) \Gamma((d-\alpha)/2)}{\pi^{d/2}}.
  \end{align*}
  %
  But this is also the value of
  %
  \[ \int_{\RR^d} K_{d - \alpha}(x) e^{- \pi |x|^2}, \]
  %
  so we conclude $\widehat{K_\alpha} = K_{d-\alpha}$ if $d/2 < \text{Re}(\alpha) < d$. We could apply Fourier inversion to obtain the result for $0 < \text{Re}(\alpha) < d/2$, but to obtain the case $\text{Re}(\alpha) = d/2$, we must apply something different. For each $s \in \CC$ with $0 < \text{Re}(s) < d$, and for each Schwartz $\phi \in \mathcal{S}(\RR^d)$ we define
  %
  \[ A(s) = \int K_s(\xi) \widehat{\phi}(\xi)\; d\xi = \frac{\Gamma(s/2)}{\pi^{s/2}} \int |\xi|^{-s/2} \widehat{\phi}(\xi)\; d\xi. \]
  %
  and
  %
  \[ B(s) = \int K_{d-s}(\xi) \widehat{\phi}(\xi)\; d\xi = \frac{\Gamma((d-s)/2)}{\pi^{(d-s)/2}} \int |\xi|^{(d-s)/2} \widehat{\phi}(\xi)\; d\xi. \]
  %
  The integrals above converge absolutely for $0 < \text{Re}(s) < d$, and the dominated convergence theorem implies that $A$ and $B$ are both complex differentiable. Since $A(s) = B(s)$ for $d/2 < \text{Re}(s) < d$, analytic continuation implies $A(s) = B(s)$ for all $0 < \text{Re}(s) < d$, completing the proof. For $\text{Re}(\alpha) \geq d$, $K_\alpha$ is no longer locally integrable, and so we must interpret the distribution given by integration by $K_\alpha$ in terms of principal values. The fourier transform of these functions then becomes harder to define.
\end{example}

\begin{example}
  Let us consider the complex Gaussian defined, for a given invertible symmetric matrix $T: \RR^d \to \RR^d$, as $G_T(x) = e^{- i \pi (Tx \cdot x)}$. Then
  %
  \[ \widehat{G_T} = e^{- i \pi \sigma/4} |\det(T)|^{-1/2} G_{-T^{-1}}, \]
  %
  where $\sigma$ is the \emph{signature} of $T$, i.e. the number of positive eigenvalues, minus the number of negative eigenvalues, counted up to multiplicity. Thus we need to show that for any Schwartz $\phi \in \mathcal{S}(\RR^d)$,
  %
  \[ e^{-i \pi \sigma/4} |\det(T)|^{-1/2} \int_{\RR^d} e^{i \pi (T^{-1}\xi \cdot \xi)} \widehat{\phi}(\xi)\; d\xi = \int_{\RR^d} e^{- i \pi (Tx \cdot x)} \phi(x)\; dx. \]
  %
  Let us begin with the case $d = 1$, in which case we also prove the theorem when $T$ is a complex symmetric matrix. If $T$ is given by multiplication by $-iz$, and if $\sqrt{\cdot}$ denotes the branch of the square root defined for all non-negative numbers and positive on the real-axis, then we note that when $z = \lambda i$,
  %
  \[ e^{- i \pi \sigma/4} |\det(T)|^{-1/2} = e^{- i \pi \text{sgn}(\lambda)/4} |\lambda|^{-1/2} = \sqrt{z}. \]
  %
  Thus it suffices to prove the analytic family of identities
  %
  \[ \int_{-\infty}^\infty e^{- (\pi/z) \xi^2} \widehat{\phi}(\xi)\; d\xi = \sqrt{z} \int_{-\infty}^\infty e^{-\pi z x^2} \phi(x)\; dx, \]
  %
  where both sides are well defined and analytic whenever $z$ has positive real part. But we already know from the Fourier transform of the Gaussian that this identity holds whenever $z$ is positive and real, and so the remaining identities follows by analytic continuation. We note that the higher dimensional identity is invariant under changes of coordinates in $SO(n)$. Thus it suffices to prove the remaining theorem when $T$ is diagonal. But then everything tensorizes and reduces to the one dimensional case. More generally, if $T = T_0 - i T_1$ is a complex symmetric matrix, which is well defined if $T_1$ is positive semidefinite, then
  %
  \[ \widehat{G_T} = \frac{1}{\sqrt{i \det(T)}} \cdot G_{-T^{-1}}, \]
  %
  which follows from analytic continuation of the case for real $T$.
\end{example}

\begin{example}
    The \emph{Airy function} on $\RR$ is the tempered distribution defined to be the inverse Fourier transform of $e^{2 \pi i \xi^3 / 3}$, i.e.
    %
    \[ \text{Ai}(x) = \int e^{2 \pi i (\xi^3 / 3 + \xi x)}\; d\xi. \]
    %
    In fact, $\text{Ai}$ is an analytic function on $\RR$. For $\varepsilon > 0$, let $\zeta = \xi + i \varepsilon$. Then
    %
    \[ 2 \pi i \zeta^3 / 3 = (2\pi) \left( i ( \xi^3 / 3 - \varepsilon^2 \xi) - (\varepsilon \xi^2 - \varepsilon^3 / 3) \right). \]
    %
    This shows that for each $\varepsilon > 0$, $e^{2 \pi i (\xi + i \varepsilon)^3 / 3}$ is a tempered distribution. Since
    %
    \[ e^{2 \pi i (\xi + i \varepsilon)^3 / 3} - e^{2 \pi i \xi^3 / 3} = e^{2 \pi i \xi^3 / 3} \left( e^{2 \pi i \varepsilon^2 \xi} e^{- 2 \pi \varepsilon \xi^2 + (2 \pi / 3) \varepsilon^3} - 1 \right).  \]
    %
    Thus these functions converge locally uniformly to zero, and are uniformly bounded, and thus $e^{2 \pi i(\xi + i \varepsilon)^3 / 3}$ converges in the sense of tempered distributions to $e^{2 \pi i \xi^3 / 3}$. But since $e^{2\pi i (\xi + i \varepsilon)^3 / 3}$ decreases exponentially, the Fourier transforms of these distributions are holomorphic in a strip of width $O(\varepsilon)$. TODO 7.6.8 of H\"{o}rmander Volume 1.
\end{example}

Thus we conclude that
%
\[ |\langle \widehat{u}, \phi \rangle| = |\langle u, \widehat{\phi} \rangle| \lesssim \| (1 + |x|)^K \phi \|_{L^\infty(\RR^d)}. \]
%
Thus $\widehat{u}$ is a distribution of order zero, and thus a measure. But $x^\alpha u$ is a compactly supported distribution for all $\alpha$, which implies that $D^\alpha \widehat{u}$ is a distribution of order zero, and thus a measure. 

\begin{example}
    We know $((-2 \pi i x)^\alpha)^\ft = ((- 2 \pi i x)^\alpha \cdot 1)^\ft = D^\alpha \delta$, which essentially provides us a way to compute the Fourier transform of any polynomial, i.e. as a linear combination of dirac deltas and the distribution derivatives of dirac deltas, which are derivatives evaluated at points.
\end{example}

\begin{example}
    Consider the Hilbert kernel $\Lambda = \text{p.v}(1/x)$. We have seen this distribution is tempered, so we can take it's Fourier transform. Now $x \Lambda = 1$, so the dericative of $\widehat{\Lambda}$ is $- 2 \pi i \delta$, where $\delta$ is the Dirac delta function at the origin. But this means there exists $A$ such that $\widehat{\Lambda}(\xi) = A - 2 \pi i \cdot \mathbf{I}(\xi > 0)$. But $\Lambda(-x) = - \Lambda(x)$, implying that $\widehat{\Lambda}(-\xi) = -\widehat{\Lambda}(\xi)$, and thus $A - 2 \pi i = -A$, i.e. $A = i \pi$. Thus
    %
    \[ \widehat{\Lambda}(\xi) = \pi i - 2 \pi i \cdot \mathbf{I}(\xi > 0) = - i \pi \cdot \text{sgn}(\xi). \]
\end{example}

\begin{theorem}
    If $\mu$ is a finite measure, $\widehat{\mu}$ is a uniformly continuous bounded function with $\| \widehat{\mu} \|_{L^\infty(\RR^d)} \leq \| \mu \|$, and
    %
    \[ \widehat{\mu}(\xi) = \int e(- 2 \pi i x \cdot \xi) d\mu(x) \]
    %
    The function $\widehat{\mu}$ is also smooth if $\mu$ has moments of all orders, i.e. $\int |x|^k d\mu(x) < \infty$ for all $k > 0$.
\end{theorem}
\begin{proof}
    Let $\phi \in \mathcal{S}(\RR^d)$. We must understand the integral
    %
    \[ \int_{\RR^d} \widehat{\phi}(x)\; d\mu(x). \]
    %
    Applying Fubini's theorem, which applies since $\mu$ has finite mass, we conclude that
    %
    \[ \int_{\RR^d} \widehat{\phi}(x)\; d\mu(x) = \int_{\RR^d} \int_{\RR^d} \phi(\xi) e^{-2 \pi i \xi \cdot x} d\mu(x)\; d\xi = \int_{\RR^d} \phi(\xi) f(\xi)\; d\xi, \]
    %
    where
    %
    \[ f(\xi) = \int_{\RR^d} e^{-2 \pi i \xi x} d\mu(x). \]
    %
    Thus $\widehat{\mu}$ is precisely $f$, and it suffices to show that $\| f \|_{L^\infty(\RR^d)} \leq \| \mu \|$, and that $f$ is uniformly continuous. The inequality follows from a simple calculation of the triangle inequality, and the second inequality follows because for some $y$,
    %
    \begin{align*}
      |f(\xi + \eta) - f(\xi)| &= \left| \int_{\RR^d} e^{-2 \pi i \xi \cdot x} (e^{-2 \pi i \eta \cdot x} - 1)\; d\mu(x) \right|\\
      &\leq \int_{\RR^d} |e^{-2 \pi i \eta \cdot x} - 1|\; d|\mu|(x).
    \end{align*}
    %
    As $\eta \to 0$, the dominated convergence theorem implies that this quantity tends to zero, which proves uniform continuity. On the other hand, if $x_i \mu$ is finite for some $i$, then
    %
    \begin{align*}
      \frac{f(\xi + \varepsilon e_i) - f(\xi)}{\varepsilon} &= \int_{\RR^d} e^{-2 \pi i \xi \cdot x} \frac{(e^{- 2 \pi \varepsilon i x_i} - 1)}{\varepsilon} d\mu(x).
    \end{align*}
    %
    We can apply the dominated convergence theorem to show that as $\varepsilon \to 0$, this quantity converges to the classical partial derivative $f_i$, which has the integral formula
    %
    \[ f_i(\xi) = (-2 \pi i) \int_{\RR^d} e^{-2 \pi i \xi \cdot x} x_i d\mu(x), \]
    %
    which is the Fourier transform of $x_i \mu$. Higher derivatives are similar.
\end{proof}

Not being compactly supported, we cannot compute the convolution of tempered distributions with all $C^\infty$ functions. Nonetheless, if $\phi$ is Schwartz, and $\Lambda$ is tempered, then the definition $(\Lambda * \phi)(x) = \Lambda(T_{-x} \phi^*)$ certainly makes sense, and gives a $C^\infty$ function satisfying $D^\alpha(\Lambda * \phi) = (D^\alpha \Lambda) * \phi = \Lambda * (D^\alpha \phi)$ just as for $\phi \in \DD(\RR^d)$. Moreover, $\Lambda * \phi$ is a slowly increasing function; to see this, we know there is $n$ such that
%
\[ |\Lambda \phi| \lesssim \| \phi \|_{\mathcal{S}^{n,m}(\RR^d)}. \]
%
Now for $|y| \geq 1$,
%
\[ \| T_y \phi \|_{\mathcal{S}^{n,m}(\RR^d)} \leq |x-y|^n \leq 2^n (1 + |y|^n) \| \phi \|_{\mathcal{S}^{n,m}(\RR^d)}, \]
%
and so
%
\[ (\Lambda * \phi)(x) = \Lambda(T_{-x} \phi^*) \lesssim_n (1 + |x|^n) \| \phi \|_{\mathcal{S}^{n,m}(\RR^d)}, \]
%
which gives that $\Lambda * \phi$ is slowly increasing. In particular, we can take the Fourier transform of $\Lambda * \phi$. Now for any $\psi \in \mathcal{S}(\RR^d)$ with $\widehat{\psi} \in \DD(\RR^d)$,
%
\begin{align*}
  \int_{\RR^d} \widehat{\Lambda * \phi}(\xi) \psi(\xi)\; d\xi &= \int_{\RR^d} (\Lambda * \phi)(x) \widehat{\psi}(x)\; dx\\
  &= \int_{\RR^d} \Lambda( \widehat{\psi}(x) \cdot T_{-x} \phi^*)\; dx\\
  &= \Lambda \left( \int_{\RR^d} \widehat{\psi}(x) T_{-x} \phi^*\; dx \right)\\
  &= \Lambda \left( \widehat{\psi} * \phi^* \right) = \Lambda \left( \widehat{\psi} * \widehat{\widehat{\phi}} \right)\\
  &= \Lambda \left( \widehat{\psi \widehat{\phi}} \right) = \widehat{\Lambda} \left( \psi \widehat{\phi} \right) = \widehat{\phi} \widehat{\Lambda}(\psi).
\end{align*}
%
We therefore conclude that $\widehat{\Lambda * \phi} = \widehat{\phi} \widehat{\Lambda}$.

Because of the dilation symmetry of the Fourier transform, the family of homogeneous distributions (which are all tempered) is invariant under the Fourier transform. More precisely, the Fourier transform of a distribution on $\RR^d$ which is homogeneous of degree $\sigma$ is a homogeneous distribution of degree $-d - \sigma$.

\begin{lemma}
    If $u$ is homogeneous and $\singsupp(u) \subset \{ 0 \}$, then $\widehat{u}$ is homogeneous, and $\singsupp(\widehat{u}) \subset \{ 0 \}$.
\end{lemma}
\begin{proof}
    Suppose first that $u$ is homogeneous of order $a$, with $\text{Re}(a) < -n$. Then we can write $u = u_0 + u_1$, where $u_0$ is supported in a neighborhood of the origin, and $u_1$ is an integrable function. But the Fourier transform of both of these terms is continuous. Thus $\widehat{u}$ is continuous in this case.

    To upgrade this fact, given any homogeneous function $u$ with $\singsupp(u) \subset \{ 0 \}$, then $D^\alpha u$ is homogeneous with real part less than $-n$ for sufficiently large $\alpha$, and $\singsupp(D^\alpha u) \subset \{ 0 \}$, which implies that $|\xi|^\alpha \widehat{u}$ is continuous, and thus $\widehat{u}$ is continuous away from the origin. But since $x^\beta u$ is homogeneous if $u$ is homogeneous, we conclude that $\widehat{x^\beta u} = D^\beta \widehat{u}$ is continuous in $\RR^n - \{ 0 \}$ for all $\beta$. Thus $\singsupp(\widehat{u}) \subset \{ 0 \}$.
\end{proof}


\section{Kernel Operators}

We have seen that all translation-invariant operators $T: \DD(\RR^d) \to \EC(\RR^d)$ are given by convolution by a distribution. Thus convolutions by a distribution are suitably general to represent all continuous translation-invariant operations. Surprisingly, when studying non translation invariant operators $T$ from $\DD(\RR^d)$ to $\EC(\RR^d)$, or more generally, from $\DD(\Omega_1)$ to $\DD(\Omega_2)^*$ for some open subsets $\Omega_1$ and $\Omega_2$ of $\RR^n$ and $\RR^m$ respectively, we can obtain a similar characterization. Instead of studying operators of the form
%
\[ T\phi(x) = \int \Lambda(x - y) \phi(y)\; dy \]
%
we instead study \emph{kernel} operators of the form
%
\[ T\phi(x) = \int K(x,y) \phi(y)\; dy \]
%
where $K$ is a distribution on $\Omega_2 \times \Omega_1$ acts as a continuous linear operator from $\DD(\Omega_1)$ to $\DD(\Omega_2)^*$ by defining, for $\psi \in \DD(\Omega_2)^*$,
%
\[ \int \psi(x) (T\phi)(x) = \int \psi(x) K(x,y) \phi(y)\; dx\; dy, \]
%
i.e. by testing $K$ against $\psi \otimes \phi$. The behaviour of the operator $T$ uniquely determines the distribution $K$, and we call $K$ the \emph{Schwartz kernel} of $T$. In 1953, Schwartz found the surprising result that kernels define all continuous operators from $\DD(\Omega_1)$ to $\DD(\Omega_2)^*$. This explains the prevalence of kernel operators in analysis.

\begin{theorem}
  Let $T: \DD(\Omega_1) \to \DD^*(\Omega_2)$ be a continuous linear operator. Then there exists a unique distribution $K \in \DD^*(\Omega_2 \times \Omega_1)$ such that for $\phi \in \DD(\Omega_1)$ and $\psi \in \DD(\Omega_2)$,
  %
  \[ \int T\phi(x) \psi(x)\; dy = \int K(x,y) \psi(x) \phi(y)\; dx\; dy. \]
\end{theorem}

\begin{remark}
    When $\Omega_1 = \Omega_2 = \RR^d$ it is sometimes more elegant to work with a representation of kernel operators in \emph{convolution form}, i.e. writing
    %
    \[ T \phi(x) = \int k(x,z) \phi(x-z)\; dz \]
    %
    which is just another way to represent the operators above by considering the change of variables $z = x - y$, i.e. we have $k(x,z) = K(x,x-z)$, or $K(x,y) = k(x,x-y)$. We call $k$ the \emph{Schwartz convolution kernel} associated with the operator $T$. If $k(z)$ is independent of the $x$-variable, then we have $T \phi = k * \phi$, i.e. the operator $T$ is really given by convolution.
\end{remark}

\begin{remark}
    There are many variants of Schwartz's result. for instance, if $T: \mathcal{S}(\RR^n) \to \mathcal{S}(\RR^m)^*$ is a continuous map, then it certainly restricts to a map from $\DD(\RR^n) \to \DD(\RR^m)^*$, and so the vanilla version of Schwartz's theorem shows that there is a distribution kernel $K$ for $T$. The fact that $\langle K, \phi \otimes \psi \rangle$ is well defined and bilinearly continuous for any $\phi \in \mathcal{S}(\RR^n)$ and $\psi \in \mathcal{S}(\RR^m)$ implies that $K$ is actually a tempered distribution. Thus every continuous map $T: \mathcal{S}(\RR^n) \to \mathcal{S}(\RR^m)^*$ is induced by a kernel in $\mathcal{S}(\RR^m \times \RR^n)^*$.
\end{remark}

\begin{remark}
    As another variant, if $K \in C^\infty(\Omega_2 \times \Omega_1)$ is a kernel, then we can define
    %
    \[ Tu(x) = \int K(x,y) u(y)\; dy \]
    %
    for any $u \in \mathcal{E}(\Omega_1)^*$, where we interpret the integral in a distributional sense, and then $Tu \in C^\infty(\Omega_2)$ because $u$ is a continuous linear functional, and $K$ can be viewed as an element of $C^\infty(\Omega_2, \loc{C^\infty}(\Omega_1))$. Moreover, the operator then extends to a continuous map
    %
    \[ T: \EC(\Omega_1)^* \to \loc{C^\infty}(\Omega_2), \]
    %
    Conversely, if $T: \EC(\Omega_1)^* \to \loc{C^\infty}(\Omega_2)$, then the continuity of $T$ implies that
    %
    \[ K(x,y) = (T \delta_y)(x) \]
    %
    lies in $C^\infty(\Omega_2 \times \Omega_1)$. Indeed, it is immediately verified from the continuity of $T$ that $K \in C(\Omega_1, \loc{C^\infty}(\Omega_2))$ since as $y \to y_0$, $\delta_y$ converges in the strong topology to $\delta_{y_0}$, and so $T \delta_y$ converges to $T \delta_{y_0}$ in $\loc{C^\infty}(\Omega_2)$. A similar argument shows that $K$ actually lies in $C^\infty(\Omega_1, \loc{C^\infty}(\Omega_2))$, where
    %
    \[ D^\alpha_x D^\beta_y K(x,y) = D^\alpha T \{ D^\beta \delta_y \}. \]
    %
    Thus $K$ is smooth.
\end{remark}

\begin{proof}
    Since Schwartz kernels act via a tensor product, understanding the Schwartz kernel theorem requires an understanding of these tensor products. The key property here is that $\DD^*$ is \emph{nuclear}. If $X$ and $Y$ are complete locally convex spaces, $X$ is barelled, and $X^*_b$ is nuclear, then $L_b(X,Y)$ is isomorphic to $X^* \CT Y$.
\end{proof}

\begin{remark}
    The fact that, e.g. the spaces $L^2(\Omega)$ are \emph{not} nuclear spaces (no infinite dimensional norm space can be nuclear) is one reason why, for instance, there are continuous linear operators $T: L^2(\Omega_1) \to L^2(\Omega_2)$ for which there does not exist a kernel $K \in L^2(\Omega_2 \times \Omega_1)$ such that
    %
    \[ Tf(x) = \int K(x,y) f(y)\; dy, \]
    %
    i.e. the class of Hilbert-Schmidt operators does not constitute the class of all continuous operators.
\end{remark}

Given an operator $T: \DD(\Omega_1) \to \DD(\Omega_2)^*$ with associated kernel $K(x,y)$, we can consider the transpose operator $T^t: \DD(\Omega_2) \to \DD(\Omega_1)^*$ given by the kernel $K(y,x)$. Then
%
\[ \langle T\phi, \psi \rangle = \int K(x,y) \psi(x) \phi(y)\; dx\; dy = \langle \phi, T^* \psi \rangle. \]
%
Thus $T^t$ is formally adjoint to $T$, in some sense. In particular, if we consider the adjoint $T^*: \DD(\Omega_2)^{**} \to \DD(\Omega_1)^*$, then $T^* \phi = T^t \phi$ for each $\phi \in \DD(\Omega_2)$.

Looking at the properties of kernels defining an operator is often a useful technique to gain insight in how an operator behaves, since one can study the singularities and smoothness of the kernel separately from the operator itself. A basic result along this form is that an operator $T$ has Schwartz kernel $K$, then for any test function $\phi$, $\text{supp}(T\phi) \subset \text{supp}(K) \circ \text{supp}(\phi)$. Let us consider some more general properties of operator that can be defined in terms of the kernel $K$ of the operator.

\begin{theorem}
    Let $T: \DD(\Omega_1) \to \DD(\Omega_2)^*$ be an operator with Schwartz kernel $K \in \DD(\Omega_2 \times \Omega_1)^*$:
    %
\begin{itemize}
    \item[(i)] $T$ maps $\DD(\Omega_1)$ into $\EC(\Omega_2)$ if and only if $K \in C^\infty(\Omega_2, \DD(\Omega_1)^*)$.

    \item[(ii)] $T$ extends to a continuous linear operator from $\EC(\Omega_1)^*$ to $\DD(\Omega_2)^*$ if and only if $K \in C^\infty(\Omega_1, \DD(\Omega_2)^*)$.

    \item[(iii)] $K$ is called \emph{separately regular} if it is an element of
    %
    \[ C^\infty(\Omega_1, \DD(\Omega_2)^*) \cap C^\infty(\Omega_2, \DD(\Omega_1)^*). \]
    %
    Then $K$ is completely regular if and only if $T$ maps $\DD(\Omega_1)$ into $\EC(\Omega_2)$, and extends to a map from $\EC(\Omega_1)$ to $\DD(\Omega_2)^*$.

    \item[(iv)] Suppose $\Omega_1 = \Omega_2$ is the same open set $\Omega$. Then $K$ is called \emph{very regular} if it is separately regular, and agrees with a $C^\infty$ function away from the diagonal
    %
    \[ \Delta = \{ (x,x) : x \in \Omega \}. \]
    %
    If $K$ is very regular, then $T$ is \emph{pseudolocal}, i.e. for any compactly supported distribution $u$, $\singsupp(Tu) \subset \singsupp(u)$.

    To make a remark, an operator $T: \DD(\Omega)^* \to \EC(\Omega)$ is called \emph{local} if $\text{supp}(Tu) \subset \text{supp}(u)$ for any $u \in \DD(\Omega)^*$. It is a result of Peetre that any local operator (continuous or non continuous) is a partial differential operator with coefficients in $C^\infty(\Omega)$.

    \item[(v)] We say $T$ is \emph{smoothing}, or \emph{regularizing}, if $T$ extends to a map from $\mathcal{E}(\Omega_2)^*$ to $C^\infty(\Omega_1)$. Then $T$ is smoothing if and only if $K \in C^\infty(\Omega_2 \times \Omega_1)$.

    \item[(vi)] We say $T$ is \emph{proper} if $T$ maps $\DD(\Omega_1)$ into $\EC(\Omega_2)^*$, and $T^t$ maps $\DD(\Omega_2)$ into $\EC(\Omega_1)^*$. This holds if and only if for any $\phi \in \DD(\Omega_1)$ and $\psi \in \DD(\Omega_2)$, $K \phi$ and $\psi K$ are in $\EC(\Omega_2 \times \Omega_1)^*$, or equivalently, if for any precompact open set $U \Subset \Omega_1$, there is a precompact open set $V \Subset \Omega_2$ such that if $\text{supp}(\phi) \subset U$, then $\text{supp}(T \phi) \subset V$, and conversely, for each such $V$, there is $U$ such that if $\text{supp}(\psi) \subset V$, then $\text{supp}(T^t \psi) \subset U$.

    Under the assumption that both (ii) and (vi) holds, i.e. $K \in C^\infty(\Omega_1, \DD(\Omega_2)^*)$ and $T$ is proper, the operator $T$ extends to a map from $\DD(\Omega_1)^*$ to $\DD(\Omega_2)^*$
\end{itemize}

\end{theorem}
\begin{proof}
    Let us begin with the proof of (i). Suppose $K \in C^\infty(\Omega_2, \DD(\Omega_1)^*)$. If $\phi \in \DD(\Omega_1)$, then the function
    %
    \[ u(x) = \int K(x,y) \phi(y)\; dy \]
    %
    then lies in $\EC(\Omega_2)$, since one calculates that
    %
    \[ D^\alpha u(x) = \int (D^\alpha K)(x,y) \phi(y)\; dy. \]
    %
    Conversely, if $T$ maps $\DD(\Omega_1)$ into $\EC(\Omega_2)$, then for any $\phi$,
    %
    \[ \int \frac{K(x + \delta e_i ,y) - K(x,y)}{\delta} \phi(y)\; dy = \int \Delta_{y_i,\delta} K(x,y) \phi(y)\; dy \]
    %
    converges locally uniformly in $x$ to $D^i T\phi(x)$. Thus $\{ \Delta_{y_i, \delta} K \}$ converges in the weak $*$ topology to $D^i T$, and the uniform boundedness theorem implies that it actually converges in the strong topology to $D^i T$, and that $D^i T: \DD(\Omega_1) \to \DD(\Omega_2)^*$ is continuous for all $i$. The Schwartz kernel of $D^i T$ is clearly the derivative $D^i K$, where $K$ is viewed as a map from $\DD(\Omega_1)$ to $\DD(\Omega_2)^*$. Iterating this argument shows that $K \in C^\infty(\DD(\Omega_1), \DD(\Omega_2)^*)$.

    Now let us prove (ii). Suppose $K \in C^\infty(\Omega_1, \DD(\Omega_2)^*)$. Then (i) implies that $T^t$ maps $\DD(\Omega_2)$ into $\EC(\Omega_1)$. Taking adjoints shows that the adjoint of $T^t$ is an extension of $T$ to a continuous map from $\EC(\Omega_1)^*$ to $\DD(\Omega_2)^*$. Conversely, if $T$ extends to a continuous map from $\EC(\Omega_1)^*$ to $\DD(\Omega_2)^*$, then it is simple to prove that if $K$ is the kernel of $T$, then $K(x,y) = T \delta_y(x)$, and thus lies in $C^\infty(\Omega_1, \DD(\Omega_2)^*)$.

    Result (iii) follows immediately from (i) and (ii).

    To prove result (iv), we note that if $K$ is very regular, then $T$ maps $\DD(\Omega)$ into $\EC(\Omega)$. Furthermore, if $\tilde{K} \in C^\infty(X^2 - \Delta)$ is the smooth function agreeing with $K$ away from the origin, then for any $u \in \EC(\Omega)^*$, and any $x \not \in \text{supp}(u)$,
    %
    \[ Tu(x) = \int K(x,x') u(x')\; dy = \int \tilde{K}(x,x') u(x')\; dy. \]
    %
    The right hand side defines a smooth function of $x$ for $x \in \Omega - \text{supp}(u)$. But this gives us the initial estimate $\singsupp(Tu) \subset \text{supp}(u)$. To prove the full statement, if $u \in \EC(\Omega)^*$, and $x_0 \not \in \singsupp(u)$, then there is $\phi \in \DD(\Omega)$ which equals one in a neighborhood $U$ of $x_0$, such that $\phi u \in \DD(\Omega)$. Thus $T(\phi u) \in \EC(\Omega)$. But $\text{supp}((1 - \phi) u)$ does not contain $x_0$, which implies that $x_0$ does not lie in $\singsupp(T((1 - \phi)u))$. But this means that both $T(\phi u)$ and $T((1 - \phi)u)$ are both smooth in a neighborhood of $x_0$, which means the same is true of $Tu$. Thus $x_0 \not \in \singsupp(Tu)$.

    We have already proved (v) in a remark following the statement of the Schwartz kernel theorem.

    The first part of the proof of (vi) is left to the reader. The assumptions of (ii) imply $T^t$ maps $\DD(\Omega_2)$ to $\EC(\Omega_1)$ continuously, and the assumptions of (vi) imply that $T^t$ maps $\DD(\Omega_2)$ to $\EC(\Omega_1)^*$ continuously. Taking both these properties into account, we conclude that $T^t$ maps $\DD(\Omega_2)$ into $\DD(\Omega_1)$ continuously. But taking adjoints gives an extension of $T$ as a continous map from $\DD(\Omega_1)^*$ to $\DD(\Omega_2)^*$.
\end{proof}
%
The pseudodifferential operators we will study later are very regular, so in particular, pseudolocal. In fact, they are \emph{microlocal}, in the sense that they never expand the \emph{wavefront set} of an operator, a subtle refinement of the singular support of a function.




\section{Test Functions on a Manifold}

For a general smooth manifold $M$, it is unnatural to define the space of test functions $\DD(M)$ to be $C_c^\infty(M)$. This is because there is no natural inclusion map from $C^\infty(M)$ to this dual, given that there is no natural definition of integration on a manifold $M$. Thus one cannot naturally think of the dual of $C_c^\infty(M)$ as being a space of `generalized functions' on $M$.

We remedy this, we work with \emph{scalar densities} rather than scalar-valued functions. Recall that on any manifold $M$, we can define a line bundle $\text{Vol}(TM)$, and scalar densities are sections of this line bundle. In coordinates, a scalar density corresponds to a family of functions $\omega_x \in C^\infty(U)$ for each coordinate chart $(x,U)$, such that for any other coordinate chart $(y,V)$, on $U \cap V$,
%
\[ \omega_y = \omega_x \cdot \left| \det \left( \frac{\partial y^i}{\partial x^j} \right) \right|. \]
%
We define $\DD(M)$ to be the space of all compactly supported smooth scalar densities, which we can equip with a natural locally convex structure analogous to the topology on the space of test functions in Euclidean space. Given any $f \in C^\infty(M)$, and any $\omega \in \mathcal{D}(M)$, the quantity
%
\[ \langle f, \omega \rangle = \int_M f \cdot \omega \]
%
is well defined, where the integral on the right is defined by working in local coordinates. Moreover, the map $\omega \mapsto \langle f, \omega \rangle$ is then a continuous linear functional on $\mathcal{D}(M)$. Thus we have a continuous inclusion $C^\infty(M) \to \mathcal{D}(M)^*$, which means it makes sense to define $\mathcal{D}(M)^*$ be the space of generalized functions on $M$.

\begin{remark} 
    If it helps to think more generally, given a smooth line bundle $E$ over $M$, we define $\mathcal{D}(M,E)$ to be the space of all smooth, compactly supported sections of $E^*$, the \emph{dual} line bundle to $E$. Then the space $\Gamma(E)$ of all smooth sections of $E$ is naturally included in $\mathcal{D}(M,E)^*$, which we call the space of \emph{generalized sections} of $E$.

    As an example of this approach, we can consider the line bundle $\text{Vol}^\alpha(TM)$ for each $0 \leq \alpha \leq 1$, whose sections $\omega$ consist of \emph{scalar densities of order $\alpha$}, which transform in coordinates via the relation
    %
    \[ \omega_y = \omega_x \cdot \left| \det \left( \frac{\partial y^i}{\partial x^j} \right) \right|^\alpha. \]
    %
    Thus $\text{Vol}(TM) = \text{Vol}^1(TM)$, and $TM = \text{Vol}^0(TM)$. The dual bundle to $\text{Vol}^\alpha(TM)$ can naturally be identified with $\text{Vol}^{1-\alpha}(TM)$ since if $\omega$ is a scalar density of order $\alpha$, and $\eta$ is a scalar density of order $1 - \alpha$, then $\omega \eta$ is a scalar density of order one, and thus can be integrated on $M$ in an invariant manner. Thus we can define the space $\mathcal{D}^\alpha(M)^*$ of generalized scalar densities of order $\alpha$ to be the dual to the space of compactly supported scalar densities of order $1 - \alpha$.
\end{remark}

Much of the basic theory of distributions follows for distributions on manifolds. In particular, a version of the Schwartz kernel theorem holds, i.e. for any smooth manifolds $M$ and $N$, and any continuous linear operator $T: \DD(M) \to \DD(N)^*$, there exists a generalized section $K$ of $\text{Vol}(M) \oplus TN$ such that for any smooth, compactly supported function $f$ on $M$, we formally have
%
\[ Tf(x) = \int K(x,y) f(y)\; dy, \]
%
in the sense that for any smooth, compactly supported scalar density $\omega$ on $N$,
%
\[ \langle Tf, \omega \rangle = \int \omega(x) K(x,y) f(y)\; dx\; dy. \]
%
The idea of the proof is to work locally in coordinates, applying the Schwartz kernel theorem in Euclidean space, and then noting that the kernels patch together if we viwe them as sections of $\text{Vol}(M) \oplus TN$.

It is more difficult to see how one would build a canonical definition of the space $\mathcal{S}^*(M)$ of tempered distributions on a manifold $M$, because there is no natural direction to measure the rate of decay of a function. Thus it is often wiser to do the things one does in Schwartz space (e.g. Fourier-type arguments) by working locally in coordinates, or on a compact manifold, where $C^\infty(M)$ should coincide with $\mathcal{S}(M)$.

\section{Paley-Wiener Theorem}

TODO: See Rudin, Functional Analysis, or Treves, Chapter 29, or H\"{o}rmander Vol 1, Section 7.3.

\begin{theorem}
    Let $K \subset \RR^n$ be a convex, compact subset with supporting function $H$. If $u$ is a distribution of order $m$, supported on $K$, then the Fourier transform of $u$ is pointwise defined by
    %
    \[ \widehat{u}(\xi) = \langle u, e^{- 2 \pi i \xi \cdot x} \rangle = \int u(x) e^{-2 \pi i \xi \cdot x}. \]
    %
    Moreover, this integral formula gives an extension of $\widehat{u}$ to an entire function on $\CC^n$, and
    %
    \[ |\widehat{u}(\xi)| \lesssim (1 + |\xi|)^N e^{2 \pi H(\text{Im}(\xi))}. \]
    %
    Conversely, any entire function $f: \CC^n \to \CC$ satisfying estimates of the form
    %
    \[ |f(\xi)| \lesssim (1 + |\xi|)^N e^{2 \pi H(\text{Im}(\xi))} \]
    %
    is tempered, and it's inverse Fourier transform is supported on $K$. In particular, if $f$ satisfies estimates of the form
    %
    \[ |f(\xi)| \lesssim_N (1 + |\xi|)^{-N} e^{2 \pi H(\text{Im}(\xi))} \]
    %
    for all $N > 0$, then the inverse Fourier transform of $f$ is smooth and compactly supported on $K$.
\end{theorem}
\begin{proof}
    Suppose $u$ is a distribution of order $N$ supported on $K$. For each $\delta > 0$, define a smooth function $\chi_\delta$ supported on $K_\delta$, equal to one on $K_{\delta/2}$, and with $\| \partial^\alpha \chi_\delta \| \lesssim_\alpha \delta^{-|\alpha|}$. Then
    %
    \begin{align*}
        |\widehat{u}(\xi)| &= \int u(x) \chi_\delta(x) e^{-2 \pi i \xi \cdot x}\\
        &\lesssim \sup_{|\alpha| \leq N} \| \partial^\alpha_x \left\{ \chi_\delta e^{-2 \pi i \xi \cdot x} \right\} \|\\
        &\lesssim e^{2 \pi H(\text{Im}(\xi)) + \delta |\text{Im}(\xi)|} \sum_{i = 0}^N \delta^{-i} (1 + |\xi|)^{N-i}.
    \end{align*}
    %
    Taking $\delta = (1 + |\xi|)^{-1}$ gives the result.

    Conversely, let us begin with the case where $f$ satisfies the decay estimates for all $N > 0$, and show this it's Fourier transform is smooth and compactly supported on $K$. Then we can define the inverse Fourier transform pointwise via the integral formula
    %
    \[ u(x) = \int f(\xi) e^{2 \pi i \xi \cdot x}. \]
    %
    To prove that $u$ is supported on $K$, we note that we can perform a contour shift, writing
    %
    \[ u(x) = \int f(\xi + i \eta) e^{2 \pi i (\xi + i \eta) \cdot x}. \]
    %
    Applying the required decay estimate with $N = n+1$ gives
    %
    \[ |u(x)| \leq e^{2 \pi (H(\eta) - \eta \cdot x)} \int (1 + |\xi|)^{-n-1}\; d\xi \]
    %
    Thus if, for a given $x$, there exists $\eta$ with $H(\eta) \leq \eta \cdot x$, then scaling $\eta$ gives $u(x) = 0$. But if $H(\eta) \geq \eta \cdot x$ for all $\eta \in \RR^d$, it follows that $x \in K$.

    Now we consider the general case. Consider an entire function $f$ satisfying estimates of the form above for a fixed $N$, and let $u$ denote it's Fourier transform. Consider a smooth function $\phi$ supported in the unit ball, with $\int \phi(x)\; dx = 1$, and let $\phi_\delta(x) = \delta^{-n} \phi(x/\delta)$. Then the inverse Fourier transform of $f_\delta(\xi) = f(\xi) \widehat{\phi}(\delta \xi)$ is $u * \phi_\delta$. Now because $\phi$ is supported in a unit ball, it's Fourier transform has the estimates proved above, and so in particular,
    %
    \[ |f_\delta(\xi)| \lesssim_M (1 + |\xi|)^{N-M} e^{2 \pi (H(\text{Im}(\eta)) + \delta |\text{Im}(\xi)|)}. \]
    %
    Thus by the last paragraph, we conclude that $u * \phi_\delta$ is supported on $K_\delta$, and taking $\delta \to 0$ completes the proof since $u * \phi_\delta$ converges to $u$ distributionally.
\end{proof}


\section{Hyperfunctions}

TODO: Treves Chapter 22, and H\"{o}rmander 7.3,7.4,7.5,8.4,8.5,8.6,8.7,Chapter 9.







\chapter{Microlocal Analysis of Singularities}

Suppose $u$ is a distribution on $\RR^d$. The \emph{singular support} of $u$ is the set of points $x_0 \in \RR^d$ which \emph{do not} have an open neighbourhood upon which $u$ acts as integration against a $C^\infty$ function. Understanding the singular support of a distribution, and how to control it, is often a useful perspective in harmonic analysis; to reduce the study of $u$ to the study of a $C^\infty$ function one need only smoothen around the singular support of $u$.

The smoothess of a distribution is linked to the decay of it's Fourier transform. In particular, suppose there is a compactly supported bump function $\phi \in C^\infty(\RR^d)$ with $\phi(x) = 1$ in a neighbourhood of some point $x_0 \in \RR^d$. Since $\phi u$ is compactly supported, the Paley-Wiener theorem implies $\widehat{\phi u}$ is an entire function with polynomial growth at infinity. The Fourier inversion formula implies that $\phi u \in \DD(\RR^d)$ if and only if for all $N \geq 0$, $|\widehat{\phi u}(\xi)| \lesssim_N |\xi|^{-N}$. Thus we can infer the singular support of $u$ via purely spectral means, provided we localize about a point.

We can therefore gain more detailed information about singularities of a distribution $u$ through the Fourier transform. If $x_0$ is a singularity of $u$, then for any bump function $\phi \in C^\infty(\RR^d)$ with $\phi(x) = 1$ in a neighbourhood of $x_0$, there must exist some direction in frequency space on which $\widehat{\phi u}$ does not decay. However, this does not mean that $\phi$ is unable to decay in certain directions; there might exist a conical neighbourhood $U$ about the origin containing some frequency $\xi_0$ such that for all $\xi \in U$ and all $N > 0$,
%
\begin{equation} \label{nonsingularfourierdecay}
  |\widehat{u \phi}(\xi)| \lesssim_N |\xi|^{-N}.
\end{equation}
%
the set of values $\xi_0$ which do \emph{not} satisfy \eqref{nonsingularfourierdecay} for any choice of bump function $\phi$ about $x_0$ forms a closed conical subset of $\RR^d$, and we call this the \emph{wavefront} of $u$ about the singularity $x_0$. The set
%
\[ \text{WF}(u) = \{ (x_0,\xi_0) : \xi_0\ \text{is in the wavefront of $u$ at $x_0$} \} \]
%
is the \emph{wavefront set} of the distribution, and provides a deeper characterization of the singularities of $u$. For instance, in order to smoothen out a distribution $u$ one need only average along the directions in the wave-front set. This is the beginning of \emph{microlocal analysis}, localization not only in space, but also localization in a conic subset of space / frequency, e.g. the tangent bundle.

\begin{remark}
    The term \emph{wavefront set} is meant to be in analogy to the to the theory of Huygens on wave propogation. If one knows the location and tangent plane to a wave, then it is translated along the normal direction to the plane. We will later see that if the principal symbol of a linear partial differential operator is real with constant coefficients, then the wavefront set of linear partial differential equations is invariant under the bicharacteristic flow, which is the case of the wave equation reduces to Huygen's theory.
\end{remark}

Let us now discuss the wavefront set a little more precisely. If $u$ is a compactly supported distribution on $\RR^d$, we define $\Gamma(u)$ to be the set of $\xi_0 \in \RR^d$ which have no conical neighbourhood $U$ such that for each $N > 0$ and $\xi \in U$,
%
\begin{equation} \label{fastDecayEquation}
    |\widehat{u}(\xi)| \lesssim_N |\xi|^{-N}.
\end{equation}
%
It is simple to verify via a compactness argument that if $\Gamma(u) = \emptyset$, then $u \in C^\infty(\RR^d)$.

\begin{lemma} \label{wavefrontlocalizationlemma}
  If $u$ is a compactly supported distribution and $\phi \in \DD(\RR^d)$, then
  %
  \[ \Gamma(\phi u) \subset \Gamma(u). \]
\end{lemma}
\begin{proof}
  Suppose $\xi_0 \not \in \Gamma(u)$, so $\xi_0$ has a conical neighbourhood $U$ such that \eqref{fastDecayEquation} holds. Then there exists $\varepsilon > 0$ such that $U$ contains
  %
  \[ \left\{ \eta \in \RR^d : \frac{\xi_0 \cdot \eta}{|\xi_0| |\eta|} \geq 1 - 2\varepsilon \right\} \]
  %
  Let $V$ be the conical neighbourhood of $\xi_0$ defined by setting
  %
  \[ V = \left\{ \eta \in \RR^d : \frac{\xi_0 \cdot \eta}{|\xi_0| |\eta|} \geq 1 - \varepsilon \right\}. \]
  %
  We claim $V$ satisfies \eqref{fastDecayEquation}. Fix $\xi \in V$. Then
  %
  \[ |\widehat{\phi u}(\xi)| = (\widehat{\phi} * \widehat{u})(\xi) = \int_{\RR^d} \widehat{\phi}(\eta) \widehat{u}(\xi - \eta)\; d\xi. \]
  %
  If $|\xi - \eta| \leq 0.25 \varepsilon |\xi|$, then it is simple to verify that
  %
  \[ (\xi_0 \cdot \eta) \geq (1 - 2\varepsilon) |\xi_0| |\eta| \]
  %
  so $\eta \in U$. Thus for any $N > 0$, $\widehat{u}(\eta) \lesssim_N 1/(1 + |\eta|)^N$. Since $\phi \in L^\infty(\RR^d$, we conclude
  %
  \begin{align*}
    \int_{|\eta| \leq 0.25 \varepsilon |\xi|} \widehat{\phi}(\eta) \widehat{u}(\xi - \eta)\; d\xi &\lesssim_{\phi} \int_{|\eta| \leq 0.25 \varepsilon |\xi|} \frac{1}{1 + |\xi - \eta|^N}\\
    &\lesssim_{\varepsilon,d} \frac{|\xi|^d}{(1 + 2 |\xi|^{N})} \lesssim \frac{1}{1 + |\xi|^{N-d}}.
  \end{align*}
  %
  On the other hand, since $u$ is compactly supported, $\widehat{u}$ is slowly increasing, i.e. there exists $m > 0$ such that
  %
  \[ |\widehat{u}(\xi)| \leq 1 + |\xi|^m. \]
  %
  Since $\phi \in \DD(\RR^d)$, we have $|\widehat{\phi}(\eta)| \lesssim_M 1/(1 + |\eta|^M)$ for all $M > 0$ and thus we conclude that if $M > m + d$
  %
  \begin{align*}
    \int_{|\eta| \geq 0.25 \varepsilon |\xi|} \widehat{\phi}(\eta) \widehat{u}(\xi - \eta) &\lesssim_M \int_{|\eta| \geq 0.25 \varepsilon |\xi|} \frac{1 + |\xi - \eta|^m}{1 + |\eta|^M}\\
    &\lesssim_{\varepsilon,m} \int_{|\eta| \geq 0.25 \varepsilon |\xi|} \frac{1 + |\eta|^m}{1 + |\eta|^M}\\
    &\lesssim_{\varepsilon,d} \frac{1}{1 + |\xi|^{M-m-d}}.
  \end{align*}
  %
  Choosing the parameter $M$ and $N$ appropriately, we obtain the required bound which shows that $\xi_0 \not \in \Gamma(\phi u)$.
\end{proof}

This fact means that for each distribution $u$, and any pair of distributions $\phi_1,\phi_2 \in \DD(\RR^d)$ such that $\text{supp}(\phi_2)$ is contained in the interior of the support of $\phi_1$, then $\phi_2/\phi_1 \in \DD(\RR^d)$, and so we conclude that
%
\[ \Gamma(\phi_2 u) = \Gamma((\phi_2/\phi_1) \phi_1 u) \subset \Gamma(\phi_1 u). \]
%
Thus if $u$ is a distribution, and $x \in \RR^d$, then we define $\Gamma_x(U)$ to be equal to
%
\[ \bigcap \left\{ \Gamma(\phi u) : \phi \in \DD(\RR^d), x \in \text{supp}(\phi) \right\}. \]
%
If $\{ \phi_n \}$ is a sequence in $\DD(\RR^d)$ such that $\text{supp}(\phi_{n+1})$ is compactly supported in $\text{supp}(\phi_n)$ for each $n$, and if $\bigcap \text{supp}(\phi_n) = \{ x \}$, then $\Gamma_x(u) = \lim_{n \to \infty} \Gamma(\phi_n u)$. Finally, we define
%
\[ \text{WF}(u) = \{ (x,\xi): \xi \in \Gamma_x(u) \}. \]
%
This is the \emph{wavefront set} of $u$.

\begin{remark}
    The theory of wavefront sets can also be defined via a sheaf theoretic framework. The distributions on a manifold form a sheaf $\DD^*$, since one can restrict and glue distributions defined on open subsets of a manifold. Similarily, the smooth functions on a manifold form a sheaf $C^\infty$, which is a subsheaf of $\DD^*$. Thus we can consider the quotient sheaf $\mathcal{F} = \DD^* / C^\infty$. The support of a distribution in $\mathcal{F}$ is then precisely the singular support of that distribution. Given a manifold $M$, we can consider a natural sheaf on $T^*M$. For each open set $U \subset T^* M$, we consider the family of all distributions on $M$, modulo the space of distributions $u$ with $\text{WF}(u) \cap U = \emptyset$. This is called the \emph{sheaf of microdistributions on $M$}. The support of a distribution in this sheaf is precisely the wavefront set of the distribution.
\end{remark}

\begin{lemma}
    If $u$ is a distribution, then $\pi_x(\text{WF}(u))$ is the singular support of $u$. If $u$ is compactly supported, then $\pi_\xi(\text{WF}(u)) = \Gamma(u)$.
\end{lemma}
\begin{proof}
    Fix $x_0 \in \RR^d$. If $(x_0,\xi_0) \not \in \text{WF}(u)$ for all $\xi_0 \in \RR^d$, then there exists $\phi \in \DD(\RR^d)$ such that $\phi(x_0) \neq 0$ and $\Gamma(\phi u) = \emptyset$. But this means $\phi u \in \DD(\RR^d)$, so $x_0$ is not in the singular support of $u$. This shows $\pi_x(\text{WF}(u))$ is a subset of the singular support. The converse is obvious.

    On the other hand, let us assume $u$ is compactly supported, and that $\xi_0 \not \in \Gamma(u)$. Then $(x_0,\xi_0) \not \in \text{WF}(u)$ for any $x_0 \in \RR^d$ since $\Gamma(\phi u) \subset \Gamma(u)$ for any $\phi \in \DD(\RR^d)$. But if $(x_0.\xi_0) \not \in \text{WF}(u)$ for any $x_0 \in \RR^d$ we can cover the support of $u$ by a partition of unity $\phi_1,\dots,\phi_N \in \DD(\RR^d)$ such that $\xi_0 \not \in \Gamma(\phi_i u)$ for each $i$, and summing up shows $\xi_0 \not \in \Gamma(u)$.
\end{proof}

\begin{example}
  Suppose $u$ is a homogenous distribution which is $C^\infty$ away from the origin. Then $\widehat{u}$ is homogenous and $C^\infty$ away from the origin, and we claim that
  %
  \[ \text{WF}(u) = \{ (0,\xi): \xi \in \text{supp}(\widehat{u}) \} = \{ 0 \} \times \Gamma(u). \]
  %
  Since the singular support of $u$ is $\{ 0 \}$, we know $\text{WF}(u) \subset \{ 0 \} \times \RR^d$, and so it suffices to calculate $\Gamma_0(u)$. Fix a non-negative radial function $\phi \in \DD(\RR^d)$ such that $\widehat{\phi}$ is a non-negative Schwartz function with total mass one. Let $\phi_t(x) = \phi(tx)$, let $\psi(\xi) = \widehat{\phi}(\xi)$, and let $\psi_t(\xi) = \widehat{\phi_t}(\xi) = t^{-d} \psi(\xi/t)$. Set $u_t = \psi_t u$. Then
  %
  \[ \Gamma_0(u) = \lim_{t \to \infty} \Gamma(u_t). \]
  %
  If $v$ is the homogeneous distribution given by the Fourier transform of $u$, and $v_t$ is the Fourier transform of $u_t$, then
  %
  \[ v_t = \widehat{\phi_t u} = \psi_t * v. \]
  %
  Since $\phi_t u$ is no longer homogeneous, neither are the distributions $\{ v_t \}$. But they do satisfy the homogeneity relations
  %
  \[ v_t(\xi) = (t/r)^s \text{Dil}_{t/r} v_r \]
  %
  for any $t,r > 0$. In particular, we see from this that $\Gamma(u_t) = \Gamma(u_r)$ for all $t,r > 0$. Thus $\Gamma_0(u) = \Gamma(u_t)$ for any $t > 0$. If $v$ vanishes in a neighborhood of $\xi_0$, then it vanishes in a conical neighborhood of $\xi_0$. It thus follows that $v_t$ vanishes in a conical neighborhood of $\xi_0$ if we pick $t$ to be sufficiently small, and thus $\xi_0 \not \in \Gamma(u_t) = \Gamma_0(u)$. Conversely, if $\xi_0 \not \in \Gamma_0(u)$, then for all $t > 0$, there is an open cone $U_t \subset \RR^d$ containing $\xi_0$ such that $|v_t(\eta)| \lesssim_{t,N} |\eta|^{-N}$ for all $\eta \in U_t$. The homogeneity relation above implies that we actually have
  %
  \[ |v_t(\eta)| \lesssim_N t^{N + |s|} |\eta|^{-N}. \]
  %
  for all $\eta \in U_1$. As $t \to 0$, $|v_t(\eta)| \to |v(\eta)|$, and if $N$ is taken larger than $|s|$ in the inequality above, this implies that
  %
  \[ |v(\eta)| \lesssim_N \limsup_{t \to 0} t^{N + |s|} |\eta|^{-N} = 0. \]
  %
  Thus $v(\eta) = 0$ for all $\eta \in U_1$, so in particular, $\xi_0 \not \in \text{supp}(v)$.

  Here are some special cases of the above example:
  %
  \begin{itemize}
    \item If $\delta$ is the Dirac delta distribution at the origin in $\RR^d$, then $\widehat{\delta}$ is the constant function, and so
    %
    \[ \text{WF}(\delta) = \{ 0 \} \times \RR^d. \]

    \item If $u(x) = \text{p.v}(1/x)$, then $\widehat{u}(\xi) = -i \pi \text{sgn}(\xi)$, and so
    %
    \[ \text{WF}(u) = \{ 0 \} \times \RR. \]

    \item If $H \subset \RR^d$ is a plane, and $\sigma_H$ is the distribution given by integration against the surface measure on $H$, then $\sigma_H$ is homogeneous, and we claim that $\widehat{\sigma}_H = \sigma_{H^\perp}$, from which the argument above implies that
    %
    \[ \Gamma_0(\sigma_H) = H^\perp. \]
    %
    Since $\sigma_H$ is invariant under translation by elements of $H$, it follows that for any $x \in H$, $\Gamma_x(\sigma_H) = H^\perp$. But $\sigma_H$ is $C^\infty$ away from $H$, so
    %
    \[ \text{WF}(u) = H \times H^\perp. \]
  \end{itemize}
\end{example}

\begin{example}
    Let $f_0$ be a distribution of order $N+1$ defined on an open set $\Omega \subset \RR^n$ by taking the boundary values of an analytic function $f$ defined on the set $\Omega \times \Gamma \subset \CC^n$, where $\Gamma$ is a convex open cone in $\RR^n$, and $|f(x + iy)| \lesssim |y|^{-N}$. Let $\Gamma^\circ = \{ \xi \in \RR^n: y \circ \xi \geq 0\ \text{for all $y \in \Gamma$} \}$ denote the dual cone of $\Gamma$. Then $\text{WF}(f_0) \subset \Omega \times (\Gamma^\circ - \{ 0 \})$. To figure this out, we have that for any $\nu \geq N$,
    %
    \begin{align*}
        \widehat{f_0 \phi}(\xi) &= \int \phi_{y_0}(x) f_{y_0}(x) e^{-2 \pi i (x + i y_0) \cdot \xi}\; dx\\
        &\quad + \frac{1}{\nu!} \int \int_0^1 f_{ty_0}(x) e^{-2 \pi i (x + i ty_0) \cdot \xi} \sum_{|\alpha| = \nu + 1} \partial^\alpha \phi(x) (i y_0)^\alpha t^\nu\; dx\; dt.
    \end{align*}
    %
    If $y_0 \cdot \xi < 0$, the first quantity decays exponentially as $\xi$ scales, and the second term is $O((1 + |\xi|)^{N-\nu})$. Taking $\nu \to \infty$ yields the claim.
\end{example}

The fact that $(x_0,\xi_0) \not \in \text{WF}(u)$ implies precisely that there exists a neighbourhood $U_0$ of $x_0$ such that for any $\phi \in \DD(U_0)$, and any $N > 0$,
%
\[ \int_{\RR^d} u(x) \phi(x) e^{-2 \pi i \lambda \xi \cdot x}\; dx \lesssim_N \langle \xi \rangle^{-N}. \]
%
It will be useful to consider a nonlinear analogue of this statement, which will be useful for showing the invariance of the wavefront set under changes of variables.

\begin{theorem}
    Let $u$ be a distribution, and let $(x_0,\xi_0) \not \in \text{WF}(u)$. Let $U$ be an open subset of $\RR^d$ containing $x_0$, let $V$ be an open subset of $\RR^p$ containing $a_0$, and let $\psi: U \times V \to \RR$ be a $C^\infty$ function with $\nabla_x \psi(x_0,a_0) = \xi_0$. Then there is an open set $U_0$ of $x_0$, an open set $V_0$ of $a_0$ such that for any $\phi \in \DD(U_0)$, and any $N > 0$,
    %
    \[ \left| \int u(x) \phi(x) e^{-2 \pi i \lambda \psi(x,a)}\; dx \right| \lesssim_N \lambda^{-N} \]
    %
    where the bound is uniform on $V_0$.
\end{theorem}
\begin{proof}
    Fix $\varepsilon > 0$, to be chosen later, and choose $U_0$ and $V_0$ such that $|\nabla_x \psi(x,a) - \xi_0| \leq \varepsilon/2$ for $x \in U_0$ and $a \in V_0$. For any given $\phi \in \DD(U_0)$, consider $\tilde{\phi} \in \DD(U_0)$ with $\tilde{\phi} \phi = \phi$. Then
    %
    \begin{align*}
        \int u(x) \phi(x) e^{-2 \pi i \lambda \psi(x,a)}\; dx &= \int u(x) \phi(x) \phi_1(x) e^{-2 \pi i \lambda \psi(x,a)}\; dx\\
        &= \int \widehat{u \phi}(\xi) \left( \int \phi_1(x) e^{-2 \pi i (\lambda \psi(x,a) - \xi)}\; dx \right)\; d\xi\\
        &= \lambda^d \int \widehat{u \phi}(\xi) \left( \int \phi_1(x) e^{-2 \pi \lambda i(\psi(x,a) - \xi)}\; dx \right)\; d\xi\\
        &= \lambda^d \int \widehat{u \phi}(\lambda \xi) J(\lambda,\xi,a)\; d\xi.
    \end{align*}
    %
    Let $\eta \in \DD(\RR^d)$ be a smooth bump function supported on $|\xi| \leq 1$ and with $\eta(\xi) = 1$ for $|\xi| \leq 1/2$. Fix $\varepsilon > 0$, and write $J(\lambda,\xi,a) = J_1(\lambda,\xi,a) + J_2(\lambda,\xi,a)$, where
    %
    \[ J_1(\lambda,\xi,a) = \eta \left( \frac{\xi - \xi_0}{\varepsilon} \right) \int_{\RR^d} \phi_1(x) e^{-2 \pi \lambda i(\psi(x,a) - \xi)}\; dx \]
    %
    and
    %
    \[ J_2(\lambda,\xi,a) = \left(1 - \eta \left( \frac{\xi - \xi_0}{\varepsilon} \right) \right) \int_{\RR^d} \phi_1(x) e^{-2 \pi \lambda i (\psi(x,a) - \xi)}\; dx. \]
    %
    If $\varepsilon$ is chosen appropriately small, then $|\widehat{u \phi}(\lambda \xi)| \lesssim_N \lambda^{-N}$ uniformly for $|\xi - \xi_0| \leq \varepsilon$. Since $|J_1(\lambda,\xi,a)| \lesssim 1$, this implies
    %
    \[ \left| \int \widehat{u \phi}(\lambda \xi) J_1(\lambda,\xi,a) \right|\; d\xi \lesssim_N \lambda^{-N}. \]
    %
    On the other hand, if $|\xi - \xi_0| \geq \varepsilon$, then $|\nabla_x \phi(x,a) - \xi| = |\xi_0 - \xi| - \varepsilon/2 \geq \varepsilon/2$. Thus the method of stationary phase implies that
    %
    \[ |J_2(\lambda,\xi,a)| \lesssim_N \lambda^{-N}, \]
    %
    uniformly in $a$. Combined with the fact that $\widehat{u \phi}$ is of polynomial growth, this implies that
    %
    \[ \left| \int \widehat{u \phi}(\lambda \xi) J_1(\lambda,\xi,a)\; d\xi \right| \lesssim_N \lambda^{-N}. \]
    %
    Combining these two estimates completes the proof.
\end{proof}

For a smooth function $\phi \in \DD(V)$ and a smooth diffeomorphism $f: U \to V$, we can define $f^* \phi \in \DD(U)$ by setting $f^* \phi(x) = \phi(f(x))$. Then for $\psi \in \DD(V)$,
%
\[ \int f^* \phi(x) \psi(x) = \int \phi(f(x)) \psi(x) = \int \phi(y) \psi(f^{-1}(y)) \cdot \frac{1}{|f'(f^{-1}(y))|} \; dy. \]
%
Thus for a distribution $u$ on $V$, to define a distribution $f^* u$ on $U$ such that for $\psi \in \DD(\RR^d)$,
%
\[ \int (f^* u)(x) \psi(x) = \int u(y) \phi(f^{-1}(y)) \cdot \frac{1}{|f'(f^{-1}(y))|}\; dy. \]
%
There is a simple relation between the wavefront set of $u$ and $f^* u$. We consider $f^*: V \times \RR^d \to U \times \RR^d$ by defining $f^*((y,v)) = (f^{-1}(y), f'(y)^T v)$. This agrees with the definition of $f^*$ encountered in differential geometry if we identify $V \times \RR^d$ and $U \times \RR^d$ with the cotangent bundle $T^* V$ and $T^* U$.

\begin{theorem}
    For any distribution $u$ on $V$, $\text{WF}(f^* u) = f^*(\text{WF}(U))$.
\end{theorem}
\begin{proof}
    Assume $(y_0,\eta_0) \not \in \text{WF}(u)$, let $(x_0,\xi_0) = f^*((y_0,\eta_0))$, and then define $\psi(y,\xi) = f^{-1}(y) \cdot \xi$. Then
    %
    \[ \nabla_y \psi(y_0,\xi_0) = (f^{-1}(y_0)')^T(\xi_0) = \eta_0. \]
    %
    Thus, applying the previous theorem, since
    %
    \[ \widehat{f^*(u \phi)}(\lambda \xi) = \int u(y) \frac{\phi(f^{-1}(y))}{|f'(f^{-1}(y))|} e^{-2 \pi i \lambda \xi \cdot f^{-1}(y)} = \int u(y) \tilde{\phi}(y) e^{-2 \pi i \lambda \psi(y,\xi)}\; dy, \]
    %
    we conclude that $|\widehat{f^*(u \phi)}(\lambda \xi) \lesssim_N \lambda^{-N}$, which implies $(x_0,\xi_0) \in \text{WF}(f^*(u))$. Thus $\text{WF}(f^* u) \subset f^*(\text{WF}(u))$. The converse statement that $f^*(\text{WF}(u)) \subset \text{WF}(f^* u)$ is obtained by symmetry.
\end{proof}

Using this change of variables formula, we see that the wavefront set transforms under a change of coordinates like a covector. Since this gives an invariant definition, we can define the wavefront set of distributions on any smooth manifold $M$, and the wavefront set will then be a closed, conical subset of $T^* M$.

\section{Oscillatory Integral Distributions}

In this section, we consider distributions on an open subset $U$ of $\RR^d$, formally defined by the formula
%
\[ I_{a,\phi}(x) = \int a(x,\theta) e^{2 \pi i \phi(x,\theta)}\; d\theta. \]
%
Here $a$ is a \emph{symbol} lying in some class $\mathcal{S}^t(U \times \RR^p)$, i.e. a smooth function satisfying bounds of the form
%
\[ |\nabla_x^n \nabla_\theta^m a(x,\theta)| \lesssim_{n,m} \langle \theta \rangle^{t - m}, \]
%
and $\phi \in C^\infty(U \times (\RR^p - \{ 0 \}))$ is homogeneous of degree one in $\theta$, and $d\phi$ is nonvanishing on the support of $a$.

If $t < -d$, then the integrand formally defining $I_{a,\phi}$ is absolutely integrable, and interpreting $I_{a,\phi}$ as a Lebesgue integral gives us a locally integrable function $I_{a,\phi}$. But for $t \geq -d$, the integral defining $I_{a,\phi}$ need not be locally integrable; for instance, our definition will show that the distribution
%
\[ \int_{\RR^d} \xi^t e^{2 \pi i x \cdot \xi}\; d\xi \]
%
acts on functions as a constant multiple of the differential operator $D^t$.

To define the oscillatory integral distribution rigorously, we fix $\psi \in \DD(\RR^d)$, and $\rho \in \DD(\RR^p)$, equal to one in a neighborhood of the origin. The integral
%
\[ \int a(x,\theta) \psi(x) \rho(\theta / R) e^{2 \pi i \phi(x,\theta)}\; d\theta \]
%
is then well defined, and we claim that the limit
%
\[ \lim_{R \to \infty} \int a(x,\theta) \psi(x) \rho(\theta / R) e^{2 \pi i \phi(x,\theta)}\; d\theta \]
%
exists and is independent of the choice of $\rho$. We can then define this limit to be
%
\[ \int I_{a,\phi}(x) \psi(x)\; dx \]
%
and this defines $I_{a,\phi}$ as a distribution. To prove the limit exists, we fix $R_1 \leq R_2$, and let $\tilde{\rho}(\theta) = \rho(\theta/R_2) - \rho(\theta/R_1)$. Then $\tilde{\rho}$ is supported on $R_1 \lesssim |x| \lesssim R_2$. Assume first that $R_2 \leq 2R_1$. Rescaling, we find that if $\eta(x,\theta) = a(x,R_2 \theta) \psi(x) \rho(\theta)$, then
%
\begin{align*}
    \int_{\RR^n} \int_{\RR^p} e^{2 \pi i \phi(x,\theta)} a(x,\theta) \psi(x) \tilde{\rho}(\theta) &= R_2^m \int_{\RR^n} \int_{\RR^p} e^{2 \pi i R_2 \phi(x,\theta)} a(x,R_2 \theta) \psi(x) \tilde{\rho}(\theta)\\
    &= R_2^p \int_{\RR^n} \int_{\RR^p} e^{2 \pi i R_2 \phi(x,\theta)} \eta(x,\theta).
\end{align*}
%
Then $\eta$ is supported on $1/2 \lesssim |\theta| \lesssim 1$ and $|x| \lesssim 1$. Thus the support of $\eta$ is independant of $R_1$ and $R_2$. It is simple to verify that
%
\[ |\nabla^n_x \nabla^m_\theta \eta(x,\theta)| \lesssim_{n,m} R_2^t \cdot |\nabla^n_x \psi(x)|, \]
%
where the bound is independant of $R_1$ and $R_2$. Since $\nabla_x \phi$ and $\nabla_\theta \phi$ have no common zeroes on the support of $a$, we can apply the principle of stationary phase to conclude that
%
\[ \left| R_2^p \int_{\RR^n} \int_{\RR^p} e^{2 \pi i R_2 \phi(x,\theta)} \eta(x,\theta) \right| \lesssim_N R_2^{p + m - N} \cdot \| \nabla^{\leq N} \psi \|_{L^\infty(\RR^d)}. \]
%
In general, if $R_2 \geq 2R_1$, we consider the largest $l$ such that $2^l R_1 \leq R_2$. If we set $a_k = 2^k R_1$ for $0 \leq k \leq l$, and $a_{l+1} = R_2$, then we write
%
\begin{align*}
    &\left| \int_{\RR^n} \int_{\RR^p} e^{2 \pi i \phi(x,\theta)} a(x,\theta) \phi(x) \tilde{\rho}(\theta) \right|\\
    &\quad\quad= \left| \sum_{k = 0}^l \int_{\RR^n} \int_{\RR^p} e^{2 \pi i \phi(x,\theta)} a(x,\theta) \phi(x) (\rho(\theta/a_{k+1}) - \rho(\theta / a_k)) \right|\\
    &\quad\quad\lesssim \sum_{k = 0}^l a_{k+1}^{p + m - N} \| \nabla^{\leq N} \psi \|_{L^\infty(\RR^d)}. 
\end{align*}
%
If we choose $N > p + m$, then we conclude that
%
\begin{align*}
    \left| \int_{\RR^n} \int_{\RR^p} e^{2 \pi i \phi(x,\theta)} a(x,\theta) \phi(x) \tilde{\rho}(\theta) \right| &\lesssim (R_1^{p + m - N} + R_2^{p + m - N}) \| \nabla^{\leq N}_x \psi \|_{L^\infty(\RR^d)}\\
    &\lesssim R_1^{p + t - N} \| \nabla^{\leq N} \psi \|_{L^\infty(\RR^d)}.
\end{align*}
%
In particular, this quantity tends to zero as $R_1 \to \infty$, which gives convergence of the limit, and also gives boundedness, showing $I_{a,\phi}$ is a distribution of order $\leq N$, where $N$ is the smallest integer bigger than $p + m$. A very similar argument shows that if $\rho \in \DD(\RR^p)$ is equal to zero in a neighborhood of the origin, then
%
\[ \lim_{R \to \infty} \int_{\RR^n} \int_{\RR^p} e^{2 \pi i \phi(x,\theta)} a(x,\theta) \psi(x) \rho(\theta / R)\; dx = 0. \]
%
It follows from the above observation that the definition is independent of the original choice of $\rho$. It is left as an exercise to show that the map $a \mapsto I_{a,\phi}$ is continuous from $S^t(U \times \RR^p)$ to $\DD^*(U)$.

\begin{remark}
    If $M$ is a manifold, and $E$ is a vector bundle over $M$, then for any homogeneous phase $\phi \in C^\infty(E - \{ 0 \})$ and any symbol $a \in \mathcal{S}^t(E^*)$, we can consider the oscillatory integral distribution formally defined by the integral
    %
    \[ I_{a,\phi}(x) = \int_{E_x} a(x,\theta) e^{2 \pi i \phi(x,\theta)}\; d\theta, \]
    %
    which is a distribution on $M$.
\end{remark}

Let us now consider the wavefront set of $I_{a,\phi}$. If $\psi$ is a bump function supported in a neighbourhood of some point $x_0$, then rescaling gives
%
\[ \widehat{I_{a,\phi} \psi}(\lambda \xi_0) = \lambda^d \int \int e^{2 \pi i \lambda (\phi(x,\theta) - x \cdot \xi_0)} a(x,\lambda \theta) \psi(x)\; dx\; d\theta. \]
%
This is an oscillatory integral, and the phase is non-stationary in the $x$ and $\theta$ variables provided that the support of $\psi$ is small enough, and $\nabla_\theta \phi(x_0,\theta_0) \neq 0$, or if $\nabla_\theta \phi(x_0,\theta_0) = 0$, but $\nabla_x \phi(x_0,\theta_0) \neq \xi_0$, provided that $\psi$ is supported on a small enough neighborhood of $x_0$. Thus we are lead to conclude that
%
\[ \text{WF}(I_{a,\phi}) \subset \{ (x_0,\nabla_x \phi(x_0,\theta_0)): (x_0,\theta_0) \in \msupp(a)\ \text{and}\ \nabla_\theta \phi(x_0,\theta_0) = 0 \}. \]
%
Here $\msupp(a)$ is the \emph{microsupport} of $a$, the complement of the set of pairs $(x_0,\theta_0)$ which have a conical neighborhood upon which $a$ lies in $S^{-\infty}$.

\begin{theorem}
    If $a \in \mathcal{S}^t(U \times \RR^p)$, and $\phi \in C^\infty(U \times \RR^p)$ is a phase satisfying the standard conditions to define the oscillatory integral distribution $I_{a,\phi}$, then
    %
    \[ \text{WF}(I_{a,\phi}) \subset \{ (x_0,\nabla_x \phi(x_0,\theta_0)): (x_0,\theta_0) \in \msupp(a)\ \text{and}\ \nabla_\theta \phi(x_0,\theta_0) = 0 \} \]
\end{theorem}
\begin{proof}
    Fix $(x_0,\xi_0) \in T^* U$, and suppose that for all $(x_0,\theta_0) \in U \times \RR^p$, either $\nabla_\theta \phi(x_0,\theta_0) \neq 0$, $\nabla_x \phi(x_0,\theta_0) \neq \xi_0$, or $(x_0,\theta_0) \not \in \msupp(a)$. Without loss of generality, we can assume that this third condition never holds, by decomposing $a$ as the sum of a symbol in $\mathcal{S}^{-\infty}$ and a symbol where $a$ vanishes in a neighborhood of any point $(x_0,\theta_0)$ with $\nabla_\theta \phi(x_0,\theta_0) = 0$ and $\nabla_x \phi(x_0,\theta_0) = \xi_0$, and if $a \in \mathcal{S}^{-\infty}$, then $I_{a,\phi} \in C^\infty(U)$. Write
    %
    \[ \widehat{I_{a,\phi} \psi}(\lambda \xi_0) = \lambda^d \int \int e^{2 \pi i \lambda \tilde{\phi}(x,\theta;\xi_0)} \tilde{a}_\lambda(x,\theta)\; dx\; d\theta \]
    %
    where $\tilde{a}_\lambda(x,\theta) = a(x, \lambda \theta) \psi(x)$, and write $\tilde{\phi}(x,\theta;\xi) = \phi(x,\theta) - x \cdot \xi$. If $Z$ is the set of all $\theta \in \RR^p$ with $|\theta| = 1$ such that $\nabla_\theta \phi(x_0,\theta) = 0$, then $Z$ is closed, and thus compact. Since $\nabla_x \phi(x_0,\theta) \neq \xi_0$ for all $\theta \in Z$, it follows by compactness that
    %
    \[ |\nabla_x \phi(x_0,\theta) - t \xi_0| \gtrsim 1 \]
    %
    for all $\theta \in Z$ and all $t > 0$. By homogeneity, for any $\theta \in \RR^p - \{ 0 \}$ such that $\nabla_\theta \phi(x_0,\theta) = 0$,
    %
    \[ |\nabla_x \phi(x_0,\theta) - \xi_0| \gtrsim |\theta|. \]
    %
    By reducing the implicit constant slightly, we may assume that there is $\varepsilon > 0$ such that for any $x \in \RR^d$ and $\xi \in \RR^d$ with $|x - x_0| \leq \varepsilon$ and $|\xi - \xi_0| \leq \varepsilon$, and any $\theta \in \RR^p - \{ 0 \}$ such that $|\nabla_\theta \phi(x_0,\theta)| \leq \varepsilon$, then
    %
    \[ |\nabla_x \phi(x,\theta) - \xi| \gtrsim |\theta|. \]
    %
    Now if $\psi$ has support in a $\varepsilon$ neighborhood of $x_0$, it follows that for all $x \in \text{supp}(\psi)$ and any $\theta \in \RR^p$,
    %
    \[ |\nabla_{x,\theta} \tilde{\phi}(x,\theta,\xi)| \gtrsim 1. \]
    %
    The principle of nonstationary phase thus guarantees that for any $N > 0$,
    %
    \[ |\widehat{I_{a,\phi} \psi}(\lambda \xi_0)| \lesssim_{\phi,N} \lambda^{-N} \sup_{|\alpha| \leq N} \| D^\alpha_{x,\theta} \tilde{a}_\lambda(x,\theta) |\nabla_{x,\theta} \tilde{\phi}(x,\theta;\xi)|^{|\alpha| - 2N} \|_{L^\infty(U \times \RR^p)}. \]
    %
    Now
    %
    \[ |\nabla_{x,\theta} \tilde{\phi}(x,\theta;\xi)| = |\nabla_x \phi(x,\theta) - \xi| \lesssim 1, \]
    %
    and if $D^\alpha_{x,\theta} = D^{\alpha_1}_x D^{\alpha_2}_\theta$, then $D^\alpha_{x,\theta} \tilde{a}_\lambda(x,\theta)$ is a finite sum of $O(\alpha)$ terms, each of the form
    %
    \[ \lambda^{|\alpha_2|} D^{\beta_1}_x D^{\alpha_2}_\theta a(x,\lambda \theta) D^{\beta_2}_x \psi(x), \]
    %
    and
    %
    \[ |\lambda^{|\alpha_2|} D^{\beta_1}_x D^{\alpha_2}_\theta a(x,\lambda \theta) D^{\beta_2}_x \psi(x)| \lesssim_{\beta_1,\beta_2} \lambda^{|\alpha_2|} \langle \lambda \rangle^{t - |\alpha_2|} \lesssim \langle \lambda \rangle^t \]
    Thus we conclude that
    %
    \[ |\widehat{I_{a,\phi} \psi}(\lambda \xi)| \lesssim_{\phi,\psi,N} \langle \lambda \rangle^{t - N}. \]
    %
    Thus $\xi_0 \not \in \text{WF}(I_{a,\phi})$.
\end{proof}

\begin{example}
    If we set $\phi(x,\xi) = - 2 \pi x \cdot \xi$, and $a(x,\xi) = a(\xi)$ is a symbol depending only on the $\xi$ variable, then
    %
    \[ I_{a,\phi}(x) = \int a(\xi) e^{- 2 \pi i x \cdot \xi}\; d\xi \]
    %
    is the Fourier transform of $a$. The result we have just calculated shows that for any symbol $a$,
    %
    \[ \text{WF}(I_{a,\phi}) \subset \{ 0 \} \times \RR^n. \]
    %
    Thus the Fourier transform of any symbol is smooth away from the origin. This result should be compared to the standard result that the Fourier transform of a homogeneous distribution which is $C^\infty$ away from the origin is also homogeneous and $C^\infty$ away from the origin.
\end{example}

Near the wavefront set of an oscillatory integral distribution, we can compute an asymptotic formula which characterizes the behaviour of the distribution near the wavefront set up to integration against a function in $C^\infty(U)$.

\begin{theorem}
    Consider a phase function $\phi_1$. Fix $(x_0,\theta_0) \in U \times \RR^p$ such that $\nabla_\theta \phi_1(x_0,\theta_0) = 0$. Let $\xi_0 = \nabla_x \phi_1(x_0,\theta_0)$. Consider any phase function $\phi_2 \in C^\infty(U \times \RR^q)$ and $\sigma_0 \in \RR^q$ with
    %
    \[ \nabla_x \phi_1(x_0,\theta_0) = \nabla_x \phi_2(x_0,\sigma_0). \]
    %
    Furthermore, assume that the Hessian $H_{x,\theta} (\phi - \psi)$ is nondegenerate at $(x_0,\theta_0,\sigma_0)$. Then there exists a conical neighborhood $\Gamma$ of $(x_0,\theta_0)$, an open neighborhood $V$ of $x_0$, and an open neighborhood $\Sigma$ of $\sigma_0$, such that if $\psi \in \DD(V)$, and $a$ is a symbol on $U \times \RR^p$ with $\text{ess sup}(a) \subset \Gamma$, then there exists a family of smooth functions $a_k$ such that as $\lambda \to \infty$,
    %
    \begin{align*}
        \int I_{a,\phi_1}(x) & \psi(x) e^{-2 \pi i \lambda \phi_2(x,\sigma)}\; dx\\
        &\sim e^{-2 \pi i \lambda \phi_2(x(\sigma),\sigma)} |\det Q(\sigma)|^{-1/2} e^{(i \pi/4) \text{sgn}(Q(\sigma))} \lambda^{(p-d)/2} \sum_{k = 0}^\infty a_k(\sigma,\lambda) \cdot \lambda^{-k},
    \end{align*}
    %
    Here $a_k(\sigma,\lambda)$ is a linear differential operator in $a$ and $u$ at $(x(\sigma), \theta(\sigma), \sigma)$
\end{theorem}

The phase $\phi$ of an oscillatory integral distribution is called \emph{nondegenerate} if whenever $\nabla_\theta \phi(x,\theta) = 0$, the matrix $D_{x,\theta}(\nabla_\theta \phi)(x,\theta)$ has full rank $p$. It follows that
%
\[ \Sigma_\phi = \{ (x,\theta): \nabla_\theta \phi(x,\theta) = 0 \} \]
%
is a $d$ dimensional submanifold of $U \times \RR^p$. Moreover, the map $f$ from $\Sigma_\phi$ to $U \times \RR^d$ given by $(x,\theta) \mapsto (x,\nabla_x \phi(x,\theta))$ is an immersion, the immersed submanifold in the image being denoted by $\Lambda_\phi \subset T^* U$. To verify the map is an immersion, we note that at a point $(x,\theta)$ the tangent space of $\Sigma_\phi$ consists of vectors $(v,w) \in \RR^d \times \RR^p$ such that
%
\[ D_x \nabla_\theta \phi(x,\theta) \cdot v + D_\theta \nabla_\theta \phi(x,\theta) \cdot w = 0. \]
%
Now
%
\[ D_{x,\theta}f(x,\theta)(v,w) = (v, D_x \nabla_x \phi(x,\theta) \cdot v + D_\theta \nabla_x \phi(x,\theta) \cdot w ). \]
%
Thus if $(v,w)$ lies in the tangent space and $Df(x,\theta)(v,w) = 0$, then $v = 0$, which implies
%
\[ D_\theta \nabla_\theta \phi(x,\theta) \cdot w = D_\theta \nabla_x \phi(x,\theta) \cdot w = 0. \]
%
Since mixed partials commute, this says exactly that $D(\nabla_\theta \phi)^T \cdot w = 0$. The full rank condition thus implies that $w = 0$. Thus $(v,w) = 0$, completing the argument that $f$ is an immersion.

%Many properties about the phase function can be summarized via the immersed manifold $\Lambda_\phi$. For instance, given a function $\psi(x,\sigma)$, the function $\eta(x,\theta,\sigma) = \phi(x,\theta) - \psi(x,\sigma)$ has a nondegenerate stationary point as a function of $x$ and $\theta$ at a point $(x_0,\theta_0,\sigma_0)$ if and only if $\phi$ is nondegenerate phase function in a neighborhood of $(x_0,\theta_0)$, and the covector field $d_x \psi$ intersects $\Lambda_\phi$ transversally at $(x_0,\xi_0)$, where $\xi_0 = \nabla_x \phi(x_0,\theta_0) = \nabla_x \psi(x_0,\sigma_0)$. In particular, we see that nondegenerate phase functions are `generic'.

TODO: If $I_{a_1,\phi_1} = I_{a_2,\phi_2}$, prove that $\Lambda_{\phi_1} = \Lambda_{\phi_2}$.

The converse is also true.

\begin{theorem}
    Suppose $\phi_1$ and $\phi_2$ are nondegenerate phase functions on $U \times \RR^{p_1}$ and $U \times \RR^{p_2}$. Let 
\end{theorem}

The manifold $\Lambda_\phi$ of $U \times \RR^d$ actually has special geometric structure. Consider the two form
%
\[ \sigma = d\xi^1 \wedge dx^1 + \dots + d\xi^d \wedge dx^d. \]
%
The $\sigma = d\omega$, where $\omega = \xi^1 dx^1 + \dots + \xi^d dx^d$. We claim that for any $p =(x,\theta) \in \Sigma_\phi$, and any $v,w \in T_p \Sigma_\phi$, $\sigma(f_* v, f_* w) = 0$. To see this, we calculate that
%
\[ f^*(\sigma) = f^*(d \omega) = d(f^* \omega), \]
%
and
%
\[ f^* \omega = \nabla_x \phi \cdot dx = d \phi - \nabla_\theta \phi \cdot d\theta. \]
%
On $\Sigma_\phi$, $\nabla_\theta \phi = 0$, so $f^* \omega = d \phi$, and so $f^*(\sigma) = d(f^* \omega) = d^2 \phi = 0$. Thus $\Lambda_\phi$ is a \emph{Lagrangian submanifold} of $T^* \RR^d$ with respect to the two form $\sigma$.

For any phase function $\phi$ (possibly degenerate), we can define
%
\[ \Sigma_\phi = \{ (x,\theta): \nabla_\theta \phi(x,\theta) = 0 \} \]
%
\[ \Lambda_\phi = \{ (x,\nabla_x \phi(x,\theta)) : \nabla_\theta \phi(x,\theta) = 0 \}. \]
%
If $\Lambda_\phi$ is an immersed Lagrangian manifold (though not necessarily an immersion through the map $f: \Sigma_\phi \to \Lambda_\phi$), we say $I_{a,\phi}$ is a \emph{Lagrangian distribution}. The phase $\phi$ might be degenerate in this case.

\begin{example}
    A degenerate example of a Lagrangian distribution is given for $p = d + 1$, $\theta = (\theta_0, \theta_1)$ with $\theta_0 \in \RR^d$ and $\theta_1 \in \RR$, and
    %
    \[ \phi(x,\theta) = x \cdot \theta_0, \]
    %
    then $\phi$ defines a Lagrangian distribution $I_{a,\phi}$ for any symbol $a$, provided that $a(x,\theta)$ vanishes for any $\theta$ in a cone containing the $\theta_1$ axis. Now $\Sigma_\phi = \{ 0 \} \times \RR^{d+1}$, and $\Lambda_\phi = \{ 0 \} \times \RR^d$, which is a Lagrangian manifold. Thus the distributions
    %
    \[ I_{a,\phi}(x) = \int a(x,\theta_0,\theta_1) e^{2 \pi i x \cdot \theta_0}\; d\theta_0\; d\theta_1 \]
    %
    are Lagrangian.
\end{example}

If $f: U \to V$ is a diffeomorphism between open subsets of $\RR^d$, and we equip $T^* U$ with coordinates $(x,\xi)$, and $T^* V$ with coordinates $(y,\eta)$, then we obtain an isomorphism $g: T^* U \to T^* V$ mapping $(x,\xi)$ in $T^* V$ to $(x, (Df(x)^T)^{-1} \eta)$ in $T^* U$. Under this correspondence, if we consider the two-form $\omega_V = \sum \eta_i \wedge dy^i$ on $U$, then
%
\[ g^* \omega_V = \sum (\eta^i \circ g) \cdot d(y^i \circ g) = \sum ((Df(x)^T)^{-1} \xi)_i df^i(x) = \sum \xi_i dx^i. \]
%
Thus the Lagrangian form is invariant under coordinate changes, and can thus be well defined on the cotangent bundle of any manifold $M$. Thus we can discuss Lagrangian submanifolds of $T^* M$ for any manifold $M$, and Lagrangian distributions on manifolds.

\begin{example}
    Consider a one-form $\psi$ on $M$, i.e. a smooth function $\psi: M \to T^* M$. Working in coordinates $(x,\xi)$ on $T^* M$, we have
    %
    \[ \psi^* \omega = \sum \psi^i dx^i = d\psi. \]
    %
    Thus we see that $\psi^* \sigma = 0$ if and only if $d\psi = 0$, so $\psi$ defines a Lagrangian submanifold of $T^* M$ if and only if it is closed.
\end{example}


\section{Singular Operations on Distributions}

A subset $\Gamma$ of $\Omega \times \RR^d$ is \emph{conic} if $(x,\xi) \in \Gamma$ implies that $(x,\lambda \xi) \in \Gamma$. Given a closed conic set $\Gamma$, let $\DD^*_\Gamma(\Omega)$ denote the family of all distributions $u$ with $\text{WF}(u) \subset \Gamma$, with the seminorms
%
\[ u \mapsto \sup_{\substack{\xi \in V\\\lambda > 0}} \lambda^N |\widehat{\phi u}(\lambda \xi)| \]
%
where $V$ is a closed conic set disjoint from $\Gamma$. Then $\DD^*_\Gamma(\Omega)$ is a Fr\'{e}chet space. To analyze this space, we require a lemma about general distributions.

\begin{lemma}
    Let $\mathcal{U}$ be a family in $\DD^*(\Omega)$ such that $\sup_{u \in \mathcal{U}} |u(\phi)| < \infty$ for all $\phi \in \DD(\Omega)$. Then for any $\phi \in \DD(\Omega)$, there exists $m > 0$ such that for any $\xi \in \mathbf{R}^d$,
    %
    \[ |\widehat{\phi u}(\xi)| \lesssim (1 + |\xi|)^m, \]
    %
    uniformly in $u$ and $\xi$. If we have a sequence $\{ u_n \}$ converging to some distribution $u$, then as $n \to \infty$,
    %
    \[ |\widehat{\phi u_n}(\xi) - \widehat{\phi u}(\xi)| = o((1 + |\xi|^m)). \]
    %
    In particular, $\widehat{\phi u_n}$ converges to $\widehat{\phi u}$ on compact subsets.
\end{lemma}
\begin{proof}
    For any compact set $K \subset \Omega$, the assumptions imply there exists $m$ such that for $u \in \mathcal{U}$ and $\phi \in C_c^\infty(K)$,
    %
    \[ |u(\phi)| \lesssim \| \phi \|_{C^m(K)}. \]
    %
    This is a result we have proven earlier following from the uniform boundedness principle. If $\phi$ is supported on $K$, then
    %
    \[ |u(\phi e^{-2 \pi i \xi \cdot x})| \lesssim \| \phi e^{-2 \pi i \xi \cdot x} \|_{C^m(K)} \lesssim (1 + |\xi|)^m. \]
    %
    A similar argument shows the result for a convergent sequence.
\end{proof}

\begin{theorem}
    A sequence of distributions $u_n$ converges to $u$ in $\DD^*_\Gamma(U)$ if and only if $u_n$ converges to $u$ distributionally, and for any conic set $V$ disjoint from $\Gamma$, and $N > 0$,
    %
    \[ \sup_{\substack{\xi \in V}} \lambda^N |\widehat{\phi u_n}(\lambda \xi)| \]
    %
    is bounded independantly of $n$.
\end{theorem}
\begin{proof}
    This conditions are certainly necessary for convergence. Conversely, if these conditions are satisfied, the previous lemma implies that as $n \to \infty$,
    %
    \[ \sup_{\lambda \geq 1} \sup_{\xi \in V} \lambda^{-m} |\widehat{\phi u_n}(\lambda \xi) - \widehat{\phi u}(\lambda \xi)| = o(1). \]
    %
    We also know that
    %
    \[ \sup_{\lambda \geq 1} \sup_{\xi \in V} \lambda^{N+1} |\widehat{\phi u_n}(\lambda \xi) - \widehat{\phi u}(\lambda \xi)| < \infty. \]
    %
    Let us call this supremum $C > 0$. Given any $\varepsilon > 0$, we find that
    %
    \[ \sup_{\lambda \geq C/\varepsilon} \lambda^N |\widehat{\phi u_n}(\lambda \xi) - \widehat{\phi u}(\lambda \xi)| \leq \varepsilon. \]
    %
    But
    %
    \[ \sup_{1 \leq \lambda \leq C/\varepsilon} \sup_{\xi \in V} \lambda^N |\widehat{\phi u_n}(\lambda \xi) - \widehat{\phi u}(\lambda \xi)| = o((C/\varepsilon)^{N+m}). \]
    %
    Taking $n$ suitably large, depending on $C$, $\varepsilon$, $N$, and $m$, we conclude that
    %
    \[ \sup_{1 \leq \lambda \leq C/\varepsilon} \sup_{\xi \in V} \lambda^N |\widehat{\phi u_n}(\lambda \xi) - \widehat{\phi u}(\lambda \xi)| \leq \varepsilon. \]
    %
    Combining this with the supremum above shows that we have convergence in $\DD^*_\Gamma(U)$.
\end{proof}

\begin{theorem}
    $\DD(\Omega)$ is sequentially dense in $\DD^*_\Gamma(\Omega)$.
\end{theorem}
\begin{proof}
    Consider a distribution $u \in \DD^*_\Gamma(\Omega)$. Without loss of generality we may assume $u$ is compactly supported. Consider an approximation to the identity $\{ \phi_\delta \}$. Then $u * \phi_{1/n} \in C^\infty(\Omega)$, and thus an element of $\DD^*_\Gamma(\Omega)$. $u * \phi_{1/n}$ converges to $u$ in $\DD^*(\Omega)$, so by the last result, it suffices to show that for any admissable choice of $\psi$, $V$, and $N > 0$,
    %
    \[ \sup_{\substack{\xi \in V}} \sup_n \lambda^N |\widehat{\psi (u * \phi_n)}(\lambda \xi)| < \infty. \]
    %
    But this is simple, for we get arbitrarily fast decay if $n$ is suitably large, depending on the distance from $V$ to $\Gamma$, and the finitely many smaller choices of $n$ are negligible.
\end{proof}

We have a continuous map $(\phi,\psi) \to \phi \psi$ from $\DD(\Omega) \times \DD(\Omega) \to \DD(\Omega)$, which extends to a continuous map $(\phi,u) \to \phi u$ from $\DD(\Omega) \times \DD^*(\Omega) \to \DD^*(\Omega)$. However, it is \emph{not} possible to extend this to a continuous map $(u,v) \mapsto uv$ from $\DD^*(\Omega) \times \DD^*(\Omega) \to \DD^*(\Omega)$. For instance, if $\{ \phi_\varepsilon \}$ is an approximation to the identity, then $\phi_\varepsilon$ converges to the Dirac delta distribution $\delta$ at the origin, so we would expect $\phi_\varepsilon^2$ to converge to a distribution representing the product $\delta \cdot \delta$, but this does not happen because if $\psi \in \DD(\RR^d)$ and $\psi(x) = 1$ for $|x| \leq 1$, then
%
\[ \left| \int \phi_\varepsilon^2(x) \psi(x) \right| \gtrsim 1/\varepsilon \]
%
and thus does not converge. It is a surprising fact that we can use the wavefront set of a distribution to define the product of two distributions, \emph{provided that the wavefront sets satisfy a disjointness relation}.

To see how this is possible, we note that for $\phi,\psi \in \DD(\RR^d)$, we might expect us to be able to take Fourier transforms, so that
%
\begin{align*}
    \int u(x) v(x) \phi(x) \psi(x)\; dx &= \int (\phi u)(x) (\psi v)(x)\; dx\\
    &= \int \int (\widehat{\phi u} * \widehat{\psi v})(\xi) e^{2 \pi i \xi \cdot x}\; d\xi\; dx\\
    &= \int \int \widehat{\phi u}(\eta) \widehat{\psi v}(\xi - \eta) e^{2 \pi i \xi \cdot x}\; d\xi\; dx.
\end{align*}
%
The only problem with taking this as the \emph{definition} of the product is that the integral we have obtained might not converge in general. However, if at least one of the Fourier transforms decreases rapidly in the right directions.

\begin{theorem}
    Fix conic sets $\Gamma_1,\Gamma_2 \subset \Omega \times \mathbf{R}^d - \{ 0 \}$. If $\Gamma_3 = \Gamma_1 + \Gamma_2$ does not contain any points in $0_\Omega = \Omega \times \{ 0 \}$, then we have a unique continuous map from $\DD^*_{\Gamma_1}(\Omega) \times \DD^*_{\Gamma_2}(\Omega) \to \DD^*_{\Gamma_3}(\Omega)$ which agrees with multiplication for elements of $C^\infty(\Omega)$.
\end{theorem}

\begin{example}
    Consider the distributions $\Lambda_1$ and $\Lambda_2$ on $\RR^2$, given by integration along the $x$ and $y$ axis respectively, i.e.
    %
    \[ \int \Lambda_1(x,y) \phi(x,y)\; dx\; dy = \int \phi(x,0)\; dx \]
    %
    and
    %
    \[ \int \Lambda_2(x,y) \phi(x,y)\; dx\; dy = \int \phi(0,y)\; dy. \]
    %
    We have seen that $\text{WF}(\Lambda_1) = \{ (x,0;0,\eta) : \eta \neq 0 \}$ and $\text{WF}(\Lambda_2) = \{ (0,\xi;y,0) : \xi \neq 0 \}$. Now
    %
    \[ \text{WF}(\Lambda_1) + \text{WF}(\Lambda_2) = \{ (x,\xi,y,\eta): \xi, \eta \neq 0 \}, \]
    %
    which is disjoint from $0_{\RR^2}$, and so a product $\Lambda_1 \cdot \Lambda_2$ is well defined. To determine what the product is, we consider a non-negative bump function $\phi \in \DD(RR^d)$ equal to one in a neighborhood of the origin, and define
    %
    \[ \phi_{x,\delta}(x,y) = (1/2\delta) \mathbf{I}(|x| \leq 1/\delta, |y| \leq \delta) \]
    %
    %
    \[ \phi_{y,\delta}(x,y) = (1/2\delta) \mathbf{I}(|x| \leq \delta, |y| \leq 1/\delta). \]
    %
    Then as $\delta \to 0$, $\phi_{x,\delta} \to \Lambda_1$ and $\phi_{y,\delta} \to \Lambda_2$. We find that
    %
    \[ \phi_{x,\delta} \phi_{y,\delta} = (1/4\delta^2) \mathbf{I}(|x| \leq \delta, |y| \leq \delta). \]
    %
    As $\delta \to 0$, $\phi_{x,\delta} \phi_{y,\delta}$ thus converges to the Dirac delta distribution $\delta$ at the origin. Thus $\Lambda_1 \cdot \Lambda_2 = \delta$.
\end{example}

\begin{example}
    Let $\Lambda = (x + i0^+)^{-1}$, i.e. the distribution
    %
    \[ \int \Lambda(x) \phi(x)\; dx = \lim_{y \to 0^+} \int \frac{\phi(x)}{x + iy}\; dx = \lim_{y \to 0^+} \Lambda_y(\phi). \]
    %
    The $\Lambda$ is homogeneous. Moreover, some formal manipulations, plus some contour integrals, show that
    %
    \[ \widehat{\Lambda}(\xi) = - 2 \pi i \cdot \mathbf{I}(\xi < 0). \]
    %
    In particular, $\text{WF}(\Lambda) = \{ (0,\xi) : \xi < 0 \}$. this means $\text{WF}(\Lambda) + \text{WF}(\Lambda)$ does not contain any zero vectors, so the product $\Lambda \cdot \Lambda$ is well defined. Now $\Lambda$ is the limit of the $C^\infty$ functions $\phi_y(x) = 1/(x + iy)$ in $\DD^*(\RR)$, and it requires only a simple calculation to show that $\Lambda$ is also the limit in $\DD^*_\Gamma(\RR)$, where $\Gamma = \{ (0,\xi): \xi < 0 \}$. Since
    %
    \[ \phi_y(x)^2 = 1/(x + iy)^2, \]
    %
    we find by continuity that
    %
    \[ \Lambda \cdot \Lambda = (x + i0^+)^{-2}, \]
    %
    i.e.
    %
    \[ \int \Lambda(x) \Lambda(x) \phi(x)\; dx = \lim_{y \to 0} \int \frac{\phi(x)}{(x + iy)^2}\; dx. \]
\end{example}

To define more sophisticated operations on distributions, we define the generic operations of \emph{pullback}, \emph{pushforward}, and \emph{tensoring}. Intuitively, the pullback of a distribution gives a way to `compose' a distribution with a smooth function in the domain, the push forward enables one to `integrate a distribution along fibres', and tensoring enables us to take the product of distributions.

Let us begin with the pullback. For a smooth map $f: \Omega \to \Psi$, not necessarily a diffeomorphism, and $\phi \in \DD(\Psi)$, we can define $f^* \phi = \phi \circ f \in C^\infty(\Omega)$. This map is continuous in the appropriate topology, and if $f$ is a proper map, $f^*$ is continuous from $\DD(\Psi) \to \DD(\Omega)$. To obtain a distributional definition, we apply the Fourier inversion formula; if $\psi \in \DD(\RR^d)$, then
%
\[ \int (f^* \phi)(x) \psi(x)\; dx = \int \phi(f(x)) \psi(x)\; dx = \int \int \widehat{\phi}(\eta) \psi(x) e^{2 \pi i \eta \cdot f(x)}\; d\xi\; dx. \]
%
For a compactly supported distribution $u$ on $\Psi$, it is therefore natural to define $f^* u$ on $\Omega$ such that
%
\[ \int (f^* u)(x) \psi(x)\; dx = \int \widehat{u}(\eta) \left( \int \psi(x) e^{2 \pi i \eta \cdot f(x)}\; dx \right)\; d\eta. \]
%
We can decompose this integral so that $\psi$ is supported on various small sets. If $\psi$ is supported in a neighbourhood of $x_0$, then the oscillatory integral on the inside decays fast as $\eta \to \infty$ provided that $Df(x_0)^T \eta \neq 0$. Thus, provided that $\widehat{u}(\eta)$ decays fast whenever $Df(x_0)^T \eta = 0$, the integral above is well defined. Proceeding through this argument more rigorously gives the following result, left as an exercise.

\begin{theorem}
    Given a smooth map $f: \Omega \to \Psi$, let
    %
    \[ N = \{ (f(x),\eta): Df(x)^T \eta = 0 \}. \]
    %
    Fix a closed cone $\Gamma$ with $\Gamma \cap N = \emptyset$. Then $f^*: \DD(\Psi) \to C^\infty(\Omega)$ extends to a continuous map from $\DD^*_\Gamma(\Psi) \to \DD^*_{f^* \Gamma}(\Omega)$, where
    %
    \[ f^* \Gamma = \{ (x,Df(x)^T \xi) : (f(x), \xi) \in \Gamma \}. \]
\end{theorem}

\begin{example}
    Given a smooth map $f: \Omega \to \RR^d$, with the property that for any $x \in f^{-1}(0)$, $Df(x)$ has rank $d$, the set $N$ above is disjoint from $\{ 0 \} \times \RR^d$, and so we can define the pullback of the Dirac delta function at the origin, $f^* \delta$, also denoted $\delta(f(x))$.

    As examples of this construction, we can consider the distributions $\delta(x)$ and $\delta(y)$ on $\RR^2$. This are equal to $\pi_x^* \delta$ and $\pi_y^* \delta$, where $\pi_x: \RR^2 \to \RR$ and $\pi_y: \RR^2 \to \RR$ are the obvious projection maps, then we have
    %
    \begin{align*}
        \int (\pi_x^* \delta)(x,y) \phi(x,y)\; dx\; dy &= \int \widehat{\delta}(\xi) \phi(x,y) e^{2 \pi i \xi \cdot x}\; dx\; dy\; d\xi\\
        &= \int \phi(x,y) e^{2 \pi i \xi \cdot x}\; dx\; dy\; d\xi\\
        &= \int \phi(0,y)\; dy.
    \end{align*}
    %
    Thus $\delta(x) = \pi_x^* \delta$ is the distribution given by integration on the $y$-axis. Similarily, one can calculate that $\delta(y) = \pi_y^* \delta$ is the distribution given by integration on the $x$-axis. It is simple to calculate explicitly, or using the properties of pullback, that
    %
    \[ \text{WF}(\pi_x^* \delta) \subset \{ (0,y;\xi,0) : \xi \neq 0 \} \]
    %
    and
    %
    \[ \text{WF}(\pi_y^* \delta) \subset \{ (x,0;0,\eta): \eta \neq 0 \}. \]
    %
    In fact, in these two cases these equations are equalities.
\end{example}

Next, let us define the tensor product. Given a distribution $u_1$ on $\Omega_1$ and a distribution $u_2$ on $\Omega_2$, we define a distribution $u_1 \otimes u_2$ on $\Omega_1 \times \Omega_2$ such that for $\phi \in \DD(\Omega_1 \times \Omega_2)$,
%
\[ \int (u_1 \otimes u_2)(x_1,x_2) \phi(x_1,x_2)\; dx_1\; dx_2 = \int u_1(x_1) \left( \int u_2(x_2) \phi(x_1,x_2)\; dx_2 \right)\; dx_1, \]
%
where the function
%
\[ \tilde{\phi}(x_1) = \int u_2(x_2) \phi(x_1,x_2)\; dx_2 \]
%
is smooth, where one can easily verify that
%
\[ D^\alpha \tilde{\phi}(x_1) = \int u_2(x_2) D^\alpha \phi(x_1,x_2)\; dx_2. \]
%
Thus the tensor product of any two distributions is well defined. It is simple to check that
%
\[ \text{WF}(u_1 \otimes u_2) \subset \text{WF}(u_1) \times \text{WF}(u_2) \cup \text{WF}(u_1) \times \{ 0 \} \cup \{ 0 \} \times \text{WF}(u_1). \]
%
obtained by isolating each variable separately with a bump function and then tensoring the Fourier transform.

\begin{example}
    Given $\phi,\psi \in \DD(\Omega)$, we have
    %
    \[ \phi \cdot \psi = i^*(\phi \otimes \psi), \]
    %
    where $i(x) = (x,x)$, which gives us another way to define the product of distributions by a tensoring, combined with a pullback.
\end{example}

Finally, we define the pushforward of a distribution. This is most naturally defined distributionally. Given a smooth map $f: \Omega \to \Psi$, $\phi \in \DD(\Omega)$, and $\psi \in \DD(\Psi)$, we define
%
\[ \int f_* \phi(y) \psi(y) dy = \int \phi(x) \psi(f(x))\; dx. \]
%
Thus $f_*$ is just the adjoint of $f^*$. One problem which prevents us from directly using this definition to extend the definition to distributions is that $\psi \circ f$ need not be compactly supported if $\psi$ is compactly supported. One way to resolve this is to consider only the pushforwards of compactly supported distributions. Another way to solve the problem is to assume $f$ is a \emph{proper map}, i.e. inverse images of compact sets are compact. It is then simple to define
%
\[ \int f_* u(y) \psi(y)\; dy = \int u(x) \psi(f(x))\; dx. \]
%
for a distribution $u$ on $\Omega$ and $\psi \in \DD(\Psi)$. To understand the wavefront set of $u$, we consider a bump function $\phi$ supported in a neighbourhood  on $\Omega$ and consider
%
\[ \int f_*(u \phi)(y) e^{-2 \pi i \eta \cdot y}\; dy = \int u(x) \phi(x) e^{-2 \pi i \eta \cdot f(x)}\; dx. \]
%
We have already show that for such an oscillatory integral, provided that $(f(x_0),Df(x_0)^T \eta) \not \in \text{WF}(u)$, this integral converges. Thus
%
\[ \text{WF}(f_* u) \subset \{ (y,\eta) : \text{There is $(x,\xi) \in \text{WF}(u)$ and $Df(x)^T \eta = \xi$} \}. \]
%
Now we have defined pushforward, pullback, and tensoring, let us see how they can be used to define useful operations on distributions.

\begin{example}
    Let $M$ be a manifold with a $C^\infty$ immersed submanifold $M_0$. Then we have a smooth inclusion map $f: M_0 \to M$. That $f$ is proper is equivalent to the assumption that $M_0$ is a closed submanifold of $M$. If $u \in \DD(M_0)^*$, then $f_* u$ is precisely the distribution on $M$ given by $\langle f_* u, \phi \rangle = \langle u, \phi|_{M_0} \rangle$. The wavefront set $\text{WF}(\phi_* u)$ is equal to the unique of the \emph{conormal bundle of $M_0$}, i.e. the set
    %
    \[ N^*(M_0) = \{ (x,\xi) \in T^*M: x \in M_0\ \text{and}\ f^*(x)(\xi) = 0 \}, \]
    %
    and $\text{WF}(u)$.
\end{example}

Let us consider an important example which occurs in the theory of kernel operators. Recall that the Schwartz kernel theorem says that if $T: \DD(Y) \to \DD^*(X)$ is any continuous linear map, then there exists a distribution $K \in \DD^*(X \times Y)$ such that, formally speaking, if $\phi \in \DD(Y)$ and $\psi \in \DD(X)$,
%
\[ \int T\phi(x) \psi(x)\; dx = \int \psi(x) K(x,y) \phi(y)\; dy. \]
%
In other words, if $\pi(x,y) = x$ and $\Delta(x,y) = (x,y,y)$, then, unwinding the definition, we find
%
\[ T\phi = \pi_*(\Delta^* (K \otimes \phi)). \]
%
Now define the \emph{wavefront} or \emph{canonical relation}
%
\[ \text{WF}(K)' = \{ (x,\xi;y,\eta) : (x,\xi;y,-\eta) \in \text{WF}(K) \}, \]
%
and
%
\[ \text{WF}_X(K)' = \text{WF}(K)' \circ 0_Y = \{ (x,\xi): (x,\xi;y,0) \in \text{WF}(K)\ \text{for some $y \in Y$}  \}. \]
%
Working through the definition shows that $\text{WF}(T\phi) \subset \text{WF}_X(K)'$. We can also use the pushforward equation to extend the domain of $T$ to certain compactly supported distributions. Going through the definition shows that for a compactly supported distribution $u$, the expression $\pi_*(\Delta^*(K \otimes u))$ is well defined provided that $\text{WF}(u)$ is disjoint from $\text{WF}_Y(K)'$, where
%
\[ \text{WF}_Y(K)' = \{ (y,\eta) : (x,0;y,-\eta) \in \text{WF}(K)\ \text{for some $x \in X$} \}. \]
%
In this case, we define $Tu = \pi_*(\Delta^*(\phi \otimes K))$. This gives a sequentially continuous map from the subspace of compactly supported distributions in $\DD^*_\Gamma(\Omega)$ to $\DD^*(\Omega)$ for any $\Gamma$ with $\text{WF}_Y(K)' \cap \Gamma = \emptyset$. If, in addition, the projection map $\pi(x,y) = x$ is proper on $\text{supp}(K)$, then this can be extended to a sequentially continuous map from $\DD^*_\Gamma(\Omega)$ to $\DD^*(\Omega)$. Again, working through the definitions shows that
%
\[ \text{WF}(Tu) \subset \text{WF}(K)' \circ \{ \text{WF}(u) \cup 0_Y \}. \]
%
A simple way to remember the results of this construction is that $K$ can be applied to any compactly supported distribution $u$ such that
%
\[ \text{WF}(K)' \circ (\text{WF}(u) \cup 0_\Omega) \]
%
contains no zero vector, and then $\text{WF}(Ku)$ is equal to this composition.

\begin{example}
    Consider a pseudodifferential operator $T$ given by a symbol $a$, i.e.
    %
    \[ T\phi(x) = \int a(x,\xi) \widehat{\phi}(\xi) e^{2 \pi i \xi \cdot x}\; d\xi = \int a(x,\xi) \phi(y) e^{2 \pi i \xi \cdot (x - y)}\; d\xi\; dy. \]
    %
    We can also think of $T$ as a kernel operator with kernel
    %
    \[ K(x,y) = \int a(x,\xi) e^{2 \pi i \xi \cdot (x - y)} \; d\xi. \]
    %
    The kernel is a distribution defined by an oscillatory integral distribution, and our calculations for such distributions show that
    %
    \[ \text{WF}(K) \subset \{ (x,-\xi;x,\xi) : x \in \Omega, \xi \in \RR^n - \{ 0 \} \}. \]
    %
    Thus
    %
    \[ \text{WF}(K)' \subset \{ (x,\xi;x,\xi) : x \in \Omega, \xi \in \RR^n - \{ 0 \} \}. \]
    %
    In particular, $\text{WF}(K)'$, viewed as a relation, contains no zero vectors, and so $T$ extends to a continuous operator from $\mathcal{E}(\RR^d)^*$ to $\DD(\RR^d)^*$, and if $a$ is compactly supported in the $x$ variable, $T$ extends to a continuous operator from $\DD(\RR^d)^*$ to itself. For any distribution $u$, we find that $\text{WF}(Tu) \subset \text{WF}(u)$. This is part of the \emph{pseudolocal} nature of pseudodifferential operators; when $T$ is applied to some distribution $u$ supported near $(x_0,\xi_0)$ in phase space, we should expect the same will be true of $Tu$.
\end{example}

\begin{example}
    Given a distribution $u$, the operator given by convolution by $u$ has kernel
    %
    \[ K(x,y) = u(x-y). \]
    %
    In other words, $K = f^* u$, where $f: \RR^{2d} \to \RR^d$ is given by $f(x,y) = x - y$. Since $f$ is surjective, the pullback $f^* u$ is always well defined, and moreover,
    %
    \[ \text{WF}(K) \subset f^* \text{WF}(u) = \{ (x_1,-\xi,x_2,\xi): (x_2 - x_1, \xi) \in \text{WF}(u) \}, \]     
    %
    and therefore
    %
    \[ \text{WF}(K)' \subset \{ (x+a,\xi;x,\xi): (a,\xi) \in \text{WF}(u) \}. \]
    %
    We actually have equality here. To see this, for $a \in \RR^d$, and let $g: \RR^d \to \RR^{2d}$ such that $g(x) = (x + a, a)$. Then $u = g^* K$, and so it follows that
    % I 0
    \[ \text{WF}(u) \subset g^* \text{WF}(K) = \{ (x,\xi): (x+a, \xi; a, \eta) \in \text{WF}(K) \}. \]
    %
    It follows from this that we have equality. Thus the operation $v \mapsto u * v$ is a continuous operator from $\mathcal{E}^*(\RR^d)$ to $\DD(\RR^d)^*$, and
    %
    \[ \text{WF}(u * v) \subset \{ (x+a,\xi): (a,\xi) \in \text{WF}(u), (x,\xi) \in \text{WF}(v) \}. \]
    %
    If $u$ is a distribution with singular support only at the origin, then $\text{WF}(K)'$ is a subset of the diagonal in $T^* \RR^d \times T^* \RR^d$, and so $\text{WF}(u * v) \subset \text{WF}(v)$ for any distribution $v$ for which the convolution can be computed.
\end{example}

How about the \emph{composition} of kernel operators? Intuitively, if $A: \DD(Y) \to \DD^*(X)$ and $B: \DD(Z) \to \DD^*(Y)$, with kernels $K_A(x,y)$ and $K_B(y,z)$, then, if we could define a kernel operator $C = A \circ B: \DD(Z) \to \DD^*(X)$, it should have kernel
%
\[ K_C(x,z) = \int K_A(x,y) K_B(y,z)\; dy. \]
%
This would be possible and well defined, for instance, if the kernel of $B$ lay in $\DD(Y \times Z)$. The integral of $K_C$ would be well defined, the composition of operators would be well defined, for if $\phi \in \DD(Z)$, then $B\phi \in \DD(Y)$, and so $A(B\phi)$ is well defined, and one finds that $K_C$ is the kernel of this operator. To generalize this composition, we might want to define $K_C = \pi_* \Delta^*(K_A \otimes K_B)$, for the appropriate maps $\Delta$ and $\pi$. Now this operation is well defined provided that $(x,0;y,\eta) \in \text{WF}(K_A)'$ and $(y,\eta;z,0) \in \text{WF}(K_B)'$ for some $x \in X$, $y \in Y$, $z \in Z$, and $\eta \in \RR^d$, and the projection map $\pi(x,y,z) = (x,z)$ is proper on the support of $\Delta^* (K_A \otimes K_B)$, which would hold, for instance, if either $\pi(x,y) = x$ was proper on the support of $K_A$, or $\pi(y,z) = z$ was proper on the support of $K_B$. In this situation, we have
%
\[ \text{WF}(K_C)' \subset \{ \text{WF}(K_A)' \circ \text{WF}(K_B)' \} \cup \{ \text{WF}_X(K_A)' \times 0_Z \} \cup \{ 0_X \times \text{WF}_Z(K_B)' \}. \]
%
Thus we have defined a fairly general composition operator on distributions, which is the unique bilinear continuous extension of the composition of two operators with kernels in $\DD(X \times Y)$ and $\DD(Y \times Z)$ to the pair $\DD_{\Gamma_A}^*(X \times Y)$ and $\DD_{\Gamma_B}^*(Y \times Z)$ for any pair of conic sets $\Gamma_A$ and $\Gamma_B$ for which there does not exist any $(x,0;y,-\eta) \in \Gamma_A$ and $(y,\eta;z,0) \in \Gamma_B$.

%Recall first that an operator $Q$ on $U$ is \emph{proper} if, for any compact set $C_1 \subset U$, there is another compact set $C_2$ such that if $\phi \in \DD(U)$ and $\text{supp}(\phi) \subset C_1$, then $\text{supp}(Q\phi) \subset C_2$. Equivalently, if $K$ is the kernel of $Q$, then for any $(x,y) \in \text{supp}(K)$ with $y \in C_1$, we have $x \in C_2$. Under these assumptions, we claim that the projection map $\pi(x,y,z) = (x,z)$ is proper on the support of
    %
%    \[ K_3(x,y,z) = \Delta^*(K_1 \otimes K_2)(x,y,z) = K_1(x,y) K_2(y,z). \]
    %
%    Indeed, if $C_1$ and $C_2$ are compact sets of $U$, then there is a third compact set $C_3$ such that if $(y,z) \in \text{supp}(K_2)$, and $z \in C_2$, then $y \in C_3$. Since $(x,y,z) \in \text{supp}(K_3)$ only if $(x,y) \in \text{supp}(K_1)$ and $(y,z) \in \text{supp}(K_2)$, this implies that
    %
%    \[ \pi^{-1}(C_1 \times C_2) \cap \text{supp}(K_3) \subset C_1 \times C_3 \times C_2. \]

\begin{example}
    Let us use this construction to define the composition of two pseudodifferential operators $P$ and $Q$ on some domain $U$, such that one of these operators is proper in the sense above. Indeed, the wave front sets of $P$ and $Q$ are never incompatible, so we can always define $P \circ Q$ using the theory above, and we find that
    %
    \[ \text{WF}(P \circ Q)' = \text{WF}(P) \circ \text{WF}(Q). \]
    %
    In particular, since $\text{WF}(P)$ and $\text{WF}(Q)$ are both subsets of the diagonal in $T^*U \times T^*U$, it follows that $P \circ Q$ is also pseudolocal in the sense that $\text{WF}((P \circ Q)(u)) \subset \text{WF}(u)$. Of course, in the theory of pseudodifferential operators one shows that $P \circ Q$ is also a pseudolocal operator for any two psuedolocal operators $P$ and $Q$, and then this result follows from that theory.
\end{example}

\begin{example}
    Suppose $T: \DD(Y) \to \DD^*(X)$ is continuous, and thus has a Schwartz kernel $K \in \DD^*(X \times Y)$. Then if we consider the operator $T^*: \DD(X) \to \DD(Y)^*$ induced by the kernel $K^*(y,x) = \overline{K(x,y)}$, then we find that
    %
    \[ \langle T\phi, \psi \rangle = \langle \phi, T^* \psi \rangle \]
    %
    for any $\phi \in \DD(Y)$ and $\psi \in \DD(X)$. Thus $T^*$ is the formal adjoint of $T$. It is simple to verify that
    %
    \[ \text{WF}(K^*)' = \{ (y,\eta; x, \xi) : (x,\xi;y,\eta) \in \text{WF}(K)' \}. \]
    %
    Thus if $\pi(x,y) = y$ is a proper map on $\text{supp}(K)$ and $\text{WF}_X(K)' = \emptyset$, we can define $T^* \circ T$, and we have
    %
    \[ \text{WF}(T^* \circ T)' \subset \{ (y_1,\eta_1;y_2,\eta_2) : (x,\xi;y_1,\eta), (x,\xi;y_2,\eta_2) \in \text{WF}(K)' \}. \]
    %
    In particular, if $\text{WF}(K)$ is contained in the graph of a function, i.e. there exists a function $f$ such that
    %
    \[ \text{WF}(K) \subset \{ (x,\xi;y,\eta) : (y,\eta) = f(x,\xi) \}, \]
    %
    then $\text{WF}(K^* \circ K)$ is a subset of the diagonal of $T^* Y \times T^* Y$, and thus $T^* \circ T$ has various pseudolocal properties. This idea comes up in the theory of Fourier integral operators, for then if $T$ is a Fourier integral operator defined by a Lagrangian distribution that is locally a graph of a function, then $T^* T$ will be a pseudodifferential operator.
\end{example}








\section{Propogation of Singularities}

One important relation between $u$ and $\text{WF}(u)$ is the \emph{propogation of singularities theorem}. If $u$ is a solution to a linear partial differential equation
%
\[ \sum_{|\alpha| \leq K} a_\alpha(x) (\partial_\alpha u)(x) = v \]
%
where $v$ is a distribution, then for any $(x,\xi) \in \text{WF}(u) - \text{WF}(v)$,
%
\[ q(x,\xi) = \sum_{|\alpha| \leq K} a_\alpha(x) \xi^\alpha = 0, \]
%
and $\text{WF}(u) - \text{WF}(v)$ is invariant under the flow generated by the Hamiltonian vector field
%
\[ H_{x,\xi} = \sum_{i = 1}^d \frac{\partial q}{\partial x^j} \frac{\partial}{\partial \xi^j} - \frac{\partial q}{\partial \xi_j} \frac{\partial}{\partial x^j}. \]
%
As a particular example, if $u(t,x,y)$ is a distributional solution to the wave equation $u_{tt} = \Delta u$ and we let $v_t(x,y) = u(t,x,y)$, then $\Delta v_t = u_{tt}$, and so by the propogation of singularities theorem $\text{WF}(v_t) \subset \text{WF}(u_{tt})$.

Then the Paley-Wiener theorem implies that $\widehat{u}$ is an analytic function on $\RR^d$. If $\widehat{u}$ decays rapidly, then $u$ is also a smooth function. However, even if $u$ is not smooth, $\widehat{u}$ may still decrease rapidly in certain directions, which implies that the singularities of $u$ `propogate' in certain directions and understanding these directions is often useful to understanding the distribution $u$. We can also get even more information about the distribution $u$ by looking at the singular frequencies.

To begin with, let 

To begin with, a distribution $u$ is \emph{nonsingular} at a point $x \in \RR^d$ if $u$ is locally a $C^\infty$ function in a neighbourhood of $x$, i.e. there exists a bump function $\phi \in C^\infty(\RR^d)$ with $\phi(x) \neq 0$ such that $\phi u \in C^\infty(\RR^d)$. The  \emph{singular support} of a compactly supported distribution $u$ to be the set of all points $x \in \RR^d$ upon which $u$ is not nonsingular.

\newpage

A degree $m$ constant coefficient linear differential operator $P(D)$ on $\RR^n$ is said to be of \emph{real principal type} if the principal symbol $P_m$ is a real-coefficient polynomial, and $\nabla P_m$ is non-vanishing on $\RR^n - \{ 0 \}$.

\begin{lemma}
    If $P(D)$ is a differential operator of real principal type, then there exists distributions $E_+$ and $E_-$ which are parametrices for $P$,
    %
    \[ \text{WF}(E_+) \subset (\{ 0 \} \times \RR^n) \cup \{ (t \cdot \nabla P_m(\xi), \xi) : t > 0, \xi \in \text{Char}(P), P_m(\xi) = 0 \}, \]
    %
    and
    %
    \[ \text{WF}(E_+) \subset (\{ 0 \} \times \RR^n) \cup \{ (t \cdot \nabla P_m(\xi), \xi) : t < 0, \xi \in \text{Char}(P), P_m(\xi) = 0 \}. \]
\end{lemma}











\chapter{Distributional Methods to PDEs}

Distribution theory was originally invented to provide a more amenable setting to the theory of existence for linear partial differential equations. Let us use the theory we have now established to solve some differential equations in the language of distributions. We begin with the most basic differential equation, namely solutions to the transport equation $D^i u = 0$.

\begin{theorem}
  If $u \in \DD^*(\RR^d)$ and there exists an index $i$ such that $D^i u = 0$, then there exists $v \in \DD^*(\RR^{d-1})$ such that
  %
  \[ \int_{\RR^d} u(x) \phi(x)\; dx = \int_{\RR^{d-1}} v(x) \left( \int_{-\infty}^\infty \phi(x)\; dx^i \right)\; dx, \]
  %
  i.e. $u$ is `constant' in the direction $i$. In particular, if $d = 1$, and $D u = 0$, then $u$ is a constant.
\end{theorem}
\begin{proof}
  Suppose without loss of generality that $i = d$. Suppose $\phi \in \DD(\RR^d)$ and for each $x \in \RR^{d-1}$,
  %
  \[ \int_{-\infty}^\infty \phi(x,t)\; dt = 0. \]
  %
  Then the function
  %
  \[ \psi(x,t) = \int_{-\infty}^t \phi(x,s)\; ds = 0 \]
  %
  has compact support and $D^i \psi = \phi$. Thus
  %
  \begin{align*}
    \int_{-\infty}^\infty u(x,t) \phi(x,t)\; dx\; dt &= \int_{-\infty}^\infty u(x,t) D^i \psi(x,t)\; dx\; dt\\
    &= - \int_{-\infty}^\infty D^i u(x,t) \psi(x,t)\; dx\; dt = 0.
  \end{align*}
  %
  Now fix $\phi_0 \in \DD(\RR)$ with $\int_{-\infty}^\infty \phi_0(x) = 1$. Then given any $\phi \in \DD(\RR^d)$,
  %
  \[ \int_{-\infty}^\infty u(x,t) \phi(x,t)\; dx\; dt = \int_{-\infty}^\infty u(x,t) \phi_0(t) \left( \int_{-\infty}^\infty \phi(x,s)\; ds \right)\; dx\; dt. \]
  %
  Thus it suffices to set
  %
  \[ v(x) = \int_{-\infty}^\infty u(x,t) \phi_0(t)\; dt. \qedhere \]
\end{proof}

\begin{remark}
    Applying this result repeatedly shows that if $u \in \DD^*(\RR^d)$ is a distribution, and $\nabla u = 0$, then $u$ is a constant.
\end{remark}

A change of variables gives versions of this result for any transport equation of the form $w \cdot \nabla u = 0$ for a fixed vector $w \in \RR^d$. A corollary is a regularity result for distributional solutions to the equation $w \cdot \nabla u(x) + a \cdot u = f$, where $f \in C(\RR)$, and $a \in C^\infty(\RR)$. We begin with the case $d = 1$.

\begin{lemma}
    If $u \in \DD(\RR)^*$ and $Du + au = f$, for $f \in C(\RR)$, and $a \in C^\infty(\RR)$, then $u \in C^1(\RR)$, and so $u$ is a classical solution to the equation.
\end{lemma}
\begin{proof}
    We just apply classical techniques distributionally. First assume $a = 0$. If $F$ is an antiderivative of $f$ in the $i$th direction, then $F \in C^1(\RR)$, and $D(u - F) = 0$, so $u$ differs from $F$ by a constant, and is therefore also in $C^1(\RR)$. For $a \neq 0$, let $A$ be an antiderivative of $a$, and set
    %
    \[ E(x) = e^{A(x)}. \]
    %
    Then $E \in C^\infty(\RR)$. Thus if $u$ is a distribution solving $D u + a u = f(x)$, and if $v = E u$ then the product rule shows that $Dv = Ef$. The $a = 0$ case implies that $v \in C^1(\RR)$, and so $u \in C^1(\RR)$.
\end{proof}

\begin{remark}
    The idea of this result generalizes to a system of differential equations given by a matrix $a$ with $C^\infty$ entries, and where $f$ is a vector with continuous entries, by finding an invertible matrix $E(x)$ such that $E'(x) = E(x) \cdot a(x)$. In particular, since higher order ordinary differential equations can be reduced to one dimensional systems of ordinary differential equations, we conclude that if $u$ is a distribution satisfying the equation
    %
    \[ D^m u + a_{m-1} D^{m-1} u + \dots + au = f, \]
    %
    for $a_i \in C^\infty(\RR)$, and $f \in C(\RR)$, then $u$ actually lies in $C^m(\RR)$, and satisfies this equation pointwise in the classical sense.
\end{remark}

Higher dimensional analogues of these results are not as strong. Indeed, we have already seen that distributional solutions to $D^i u = 0$ may fail to be classical solutions `normal to the direction $i$'. On the other hand, we can `almost' obtain such a result if we assume apriori that $u$ is a continuous function.

\begin{lemma}
    Suppose $u$ and $f$ are continuous functions in $C(\RR^d)$, and $u$, viewed as a distribution, satisfies the equation $D^i u = f$. Then $D^i u$ exists pointwise, in the classical sense, for all $x \in \RR^d$, and $D^i u (x) = f(x)$ for all $x \in \RR^d$.
\end{lemma}
\begin{proof}
    Assume $i = d$ without loss of generality, and write $x = (x_0,t)$, for $x_0 \in \RR^{d-1}$ and $t \in \RR$. Set
    %
    \[ v(x_0,t) = \int_0^t f(x_0,s)\; ds. \]
    %
    Then $v$ is a distributional solution to the equation $D^i v = f$, and so $D^i(u - v) = 0$. It follows that there exists a distribution $w \in \DD(\RR^{d-1})^*$ such that $u(x,t) - v(x,t) = w(x)$. The proof of the existence of $w$ actually implies that $w$ is continuous, since $u$ and $v$ are continuous. But then $u(x,t) = v(x,t) + w(x)$ is differentiable in the $t$-variable by the fundamental theorem of calculus.
\end{proof}

\begin{theorem}
    Fix $a_1,\dots,a_n \in C^1(\RR^d)$, and $b \in C^0(\RR^d)$, and suppose that $u \in C(\RR^d)$ is pointwise differentiable, and in a pointwise sense, the equation $a_1 D_1 u + \dots + a_n D_n u + b u = f$, where $f \in C(\RR^d)$. Then the same equation holds in a distributional sense (which makes sense because $D_i u$ is a distribution of order one, and thus can be multiplied against $C^1$ functions).
\end{theorem}
\begin{proof}
    We adapt the proof of Cauchy's theorem due to Goursat. Our goal is to prove, given the assumptions in the theorem, that for any $\phi \in C_c^\infty(\RR^d)$,
    %
    \[ \int (b \phi - D_1(a_1 \phi) + \dots + D_n(a_n \phi)) \cdot u = \int f \phi. \]
    %
    For any cube $I$, let
    %
    \[ A_I = \int_I (f \phi - (b \phi - D_1(a_1 \phi) - \dots - D_n(a_n \phi)) \cdot u) + \int_{\partial I} u (a \cdot n)\; dS. \]
    %
    For any fixed $x_0 \in \RR^d$, we claim that
    %
    \[ \lim_{\substack{x_0 \in I\\|I| \to 0}} \frac{|A_I|}{|I|} = 0. \]
    %
    To prove this, we may replace $u$ in the formula above with it's first order Taylor expansion, and $f$ with the resultant formula $a_1 D_1 u + \dots + a_n D_n u + bu$, without loss of generality, since the error term here disappears in the limit. But then the formula is easy to verify. But now it follows that we must have $A_I = 0$ for all $I$, because otherwise we could find a nested family of cubes $A_I$ with $|A_I| \gtrsim |I|$. But if we now take $I$ large enough that it contains the support of $\phi$, the theorem is proved.
\end{proof}

Using techniques from the next section, we obtain a higher dimensional variant of the ODE result above.

\begin{theorem}
    Let $U = V_x \times I_t$, where $V \subset \RR^n$ is open, and $I \subset \RR$ is an open interval. If $u \in \DD(U)^*$ satisfies
    %
    \[ \partial_t^m u + L_{m-1} \{ \partial_t^{m-1} u \} + \dots + L_0 \{ u \} = f, \]
    %
    where $f \in C(I, \DD(V)^*)$, and $L_0,\dots,L_{m-1}$ are differential operators with coefficients in $C^\infty(U)$, then $u \in C^m(I,\DD(V)^*)$.
\end{theorem}
\begin{proof}
    Assuming we are working with vector-valued inputs reduces us to the study of $m = 1$, i.e. an equation of the form $\partial_t u + Lu = f$, and it suffices to show that $u \in C^1(I,\DD(V)^*)$. The case $n = 0$ has already been considered above. Localizing if necessary, write $u = \sum_{\beta_1,\beta_2} \partial_t^{\beta_1} D^{\beta_2}_x u_\beta$, where $u_\alpha \in C(U)$, and write $f = \sum_{|\alpha| \leq N} D^\alpha_x f_\alpha$, where $f_\alpha \in C(U)$. If $u_\beta = 0$ for any $\beta$ with $\beta_1 > 0$, then $u \in C(I, \DD(V)^*)$. Otherwise, let $M$ be the smallest integer such that $u_\beta = 0$ for any $\beta$ with $\beta_1 > M$. Then
    %
    \[ \partial_t u = f - Lu = \sum D^\alpha f_\alpha - \sum L D^\alpha u_\alpha. \]
    %
    But, antidifferentiating, we can write
    %
    \[ \sum D^\alpha f_\alpha - \sum L D^\alpha u_\alpha = \partial_t \left\{ \sum_\beta \partial_t^{\beta_1} D^{\beta_2} g_\beta \right\}, \]
    %
    where $g_\beta = 0$ for $\beta_1 > M-1$. But we have already seen from the fact that these quantities are equal, that
    %
    \[ u - \sum_\beta \partial_t^{\beta_1} D^{\beta_2} g_\beta \]
    %
    lies in $C(I,\DD(V)^*)$. Applying induction on $M$, we conclude $u \in C(I,\DD(V)^*)$. But $\partial_t u = u - Lf$ also lies in $C(I,\DD(V)^*)$, and so $u \in C^1(I,\DD(V)^*)$.
\end{proof}

\begin{remark}
    If $f$ extends to an element of $C(\overline{I}, \mathcal{D}(V)^*)$, and $u$ extends to a distribution defined in an open neighborhood of $\overline{I} \times V$, then the proof actually shows that $u \in C^m(\overline{I}, \DD(V)^*)$.
\end{remark}

\section{Fundamental Solutions}

Let us now discuss the idea of a \emph{fundamental solution} to a partial differential equation, a technique very useful to the study of the existence and uniqueness of such equations. For a differential operator $L = \sum c_\alpha D^\alpha$ with constant coefficients on $\RR^n$, a \emph{fundamental solution} for $L$ is a distribution $\Phi \in \DD(\RR^n)^*$ such that $L\Phi = \delta$ is the Dirac delta function at the origin. The reason that fundamental solutions are so useful to the study of constant coefficient partial differential equations is that for any compactly supported distribution $v \in \DD(\RR^d)^*$, the distribution $u = \Phi * v$ is a solution to the equation $Lu = v$, since $L u = (L \Phi) * v = \delta * v = v$. Thus fundamental solutions give rise to a natural right inverse to a differential operator. It is a general result that \emph{all} constant coefficient differential equations have fundamental solutions, though here we only consider particular examples.

\begin{example}
    Consider the differential operator $\Delta$, defining Poisson's equation $\Delta u = v$. To guess a fundamental solution for $\Delta$, we first note that since $\Delta$ is invariant under rotations, as is the Dirac delta, the equation $\Delta u = \delta$ is radially symmetric. Thus we might expect to find a radially symmetric fundamental solution. Since $\Delta$ is \emph{elliptic}, we expect fundamental solutions to be smooth away from the origin. Thus to make a guess on the fundamental solution, it will be a useful calculation to determine all smooth, radially symmetric functions $u: \RR^n - \{ 0 \} \to \CC$ such that $\Delta u = 0$. If $u(x) = f(|x|)$ for $x \neq 0$ and $f \in C^\infty((0,\infty))$, then for $r > 0$,
    %
    \[ f''(r) + \frac{n-1}{r} f'(r) = 0. \]
    %
    For $n > 2$, this implies that $f(r) = a_1 r^{2-n} + a_2$, and for $n = 2$, $f(r) = a_1 \log(r) + a_2$. We might expect that some choice of constants gives the fundamental solution if we extend these functions to distributions on $\RR^n$. We may without loss of generality pick $a_2 = 0$, since constants do not factor into the output of the operator $\Delta$. The correct choice of the constant $a_1$ gives the \emph{Poisson kernel}
    %
    \[ \Phi(x) = \begin{cases} - \frac{1}{2 \pi} \log |x| &: \text{if $n = 2$} \\ - \frac{1}{c_n} \frac{1}{n-2} \frac{1}{|x|^{n-2}} &: \text{if $n > 2$.} \end{cases} \]
    %
    Here $c_n$ is the surface area of the unit sphere in $\RR^n$. These distributions are all locally integrable near the origin and thus extend uniquely to distributions on $\RR^n$. To prove that $\Phi$ is a fundamental solution, it suffices to show for any $\phi \in \DD(\RR^d)$,
    %
    \[ \phi(0) = \int \Phi(x) \Delta \phi(x)\; dx. \]
    %
    Since $\Phi$ is locally integrable, Gauss' formula implies that as $\varepsilon \to 0$,
    %
    \[ \int_{|x| \geq \varepsilon} \Phi(x) \Delta \phi(x)\; dx = \int_{|x| = \varepsilon} \Phi(x) [\nabla \phi(x) \cdot n(x)] dS - \int \nabla \Phi(x) \cdot \nabla \phi(x)\; dx. \]
    %
    Now a simple estimate gives that as $\varepsilon \to 0$,
    %
    \[ \int_{|x| = \varepsilon} \Phi(x) [\nabla \phi(x) \cdot n(x)] dS \to 0. \]
    %
    Thus we conclude that
    %
    \[ \int \Phi(x) \Delta \phi(x)\; dx = - \int \nabla \Phi(x) \cdot \nabla \phi(x)\; dx. \]
    %
    Again, we approximate away from the origin by a $\varepsilon$ ball, integrate by parts, and calculate that
    %
    \begin{align*}
        \int_{|x| \geq \varepsilon} \nabla \Phi(x) \cdot \nabla \phi(x)\; dx &= \int_{|x| = \varepsilon} \phi(x)\ [\nabla \Phi(x) \cdot n(x)] dS - \int_{|x| \geq \varepsilon} \Delta \Phi(x) \phi(x)\; dx\\
        &= \int_{|x| = \varepsilon} \phi(x)\ [\nabla \Phi(x) \cdot n(x)] dS.
    \end{align*}
    %
    For $n = 2$, we have $\nabla \Phi(x) = (2 \pi)^{-1} (x/|x|^2)$, hence
    %
    \begin{align*}
        \int_{|x| = \varepsilon} \phi(x) [\nabla \Phi(x) \cdot n(x)] dS &= \frac{1}{2 \pi \varepsilon} \int_{|x| = \varepsilon} \phi(x)\; dx\\
        &= \phi(0) + O(\varepsilon).
    \end{align*}
    %
    For $n > 2$, $\nabla \Phi(x) = c_d^{-1} (x / |x|^d)$, so
    %
    \begin{align*}
        \int_{|x| = \varepsilon} \phi(x) [\nabla \Phi(x) \cdot n(x)] dS &= \frac{1}{c_d \varepsilon^{d-1}} \int_{|x| = \varepsilon} \phi(x)\; dS = \phi(0) + O(\varepsilon).
    \end{align*}
    %
    Taking $\varepsilon \to 0$ gives the result.

    An alternate approach to obtaining this fundamental solution is to take Fourier transforms, assuming that we can find a \emph{tempered} fundamental solution $\Phi$. If $\Psi = \widehat{\Phi}$, then we conclude that $\Phi$ is a fundamental solution if and only if $\Psi$ is tempered and
    %
    \[ - 4\pi^2 |\xi|^2 \cdot \Psi(\xi) = 1. \]
    %
    For $n > 2$, we can interpret the formula $\Psi(\xi) = (-1/4\pi^2) |\xi|^{-2}$ as defining a distribution by integration against a locally integrable function. For $n = 2$ a version of this equation remains true provided that we interpret the distribution $1/|\xi|^2$ at the origin in the right way. Thus we have the alternate expression for the fundamental solution above.
\end{example}

\begin{example}
    Next, consider the operator $L = \partial_t - \Delta$ on $\RR^{n+1}$, which gives rise to the heat equation. Set
    %
    \[ \Phi_+(x,t) = \frac{1}{(4 \pi t)^{n/2}} \exp \left( - \frac{|x|^2}{4t} \right) \]
    %
    for $t > 0$, and $\Phi(x,t) = 0$ for $t \leq 0$. Then $\Phi$ is locally integrable, and a fundamental solution for $L$. Indeed, $\Phi$ is tempered in the $x$-variable, and the Fourier transform of $\Phi$ in the $x$ variable is verified to be
    %
    \[ U(\xi,t) = e^{- 4 \pi^2 t |\xi|^2} \]
    %
    which satisfies $\partial_t U = - 4 \pi^2 |\xi|^2 U$, which is equivalent to the equation $L \Phi = \delta$. We see immediately that $\Phi(0+) = \lim_{t \to 0} \Phi(t)$ is the Dirac delta function at the origin.

    More generally, similar calculations enable us to find fundamental solutions to any operator of the form $L = \partial_t - S$, where $S = \sum A_{ij} \partial_i \partial_j$. If we assume that there exists a fundamental solution $\Phi_+$ to $L$, tempered in the $x$-variable and supported on $t \geq 0$, then taking Fourier transforms of the equation $L\Phi = \delta$ in the $x$-variable, and setting $\Psi_+$ to be this Fourier transform, we are lead to conclude that $\partial_t \Psi + 4\pi^2 (\xi^T A \xi) \Psi = \delta(t)$. Let us assume that $\Phi$ is supported on $t \geq 0$. Thus we can write
    %
    \[ \Psi(\xi,t) = H(t) A(\xi) e^{- 4 \pi^2 (\xi^T A \xi) t} \]
    %
    for some distribution $A \in \DD(\RR^n)^*$. But we then calculate that
    %
    \[ L \Psi = \delta(t) A(\xi) \]
    %
    and so we must have $A(\xi) = 1$, i.e.
    %
    \[ \Psi(\xi,t) = H(t) e^{-4 \pi^2 (\xi^T A \xi) t}. \]
    %
    But this means that, in order to get a tempered fundamental solution we must assumed $\text{Re}(A)$ is positive semidefinite. If we, in addition, assume that $A$ is invertible, then we have the Fourier transform for $\Psi$, namely, we conclude that
    %
    \[ \Phi(x,t) = \frac{H(t)}{(4 \pi t)^{n/2} ( \det(A))^{1/2}} e^{- x^T A^{-1} x / 4t}. \]
    %
    If the real part of $A$ is positive \emph{definite}, then the same is true for $A^{-1}$, and so $\Phi$ can be interpreted as locally integrable on $\RR \times \RR^n$. On the other hand, if $A$ is positive \emph{semidefinite} this is not the case, for instance, the Schr\"{o}dinger equation $L = \partial_t - i \Delta_x$ has fundamental solution
    %
    \[ \Phi(x,t) = \frac{H(t)}{(4 \pi i t)^{n/2}} e^{i |x|^2 / 4t},  \]
    %
    which is not locally integrable, but we must treat this solution as a principal value in the $t$-variable, using the principle of stationary phase to show the limit is well defined, which allows us to conclude $\Phi$ is a distribution of order $n+1$.
\end{example}

Before we consider other quadratic partial differential operators, let us perform a few calculations. Take an operator of the form
%
\[ L_A = \sum_{i,j = 1}^n A_{ij} D^{ij}_x \]
%
for some symmetric $n \times n$ matrix $A$. If $T$ is an invertible linear transformation, then the chain rule implies that
%
\[ D^i \{ f \circ T \} = \sum_{k = 1}^n T_{ki} \cdot (D^k \{ f \} \circ T), \]
%
i.e. $\nabla \{ f \circ T \} = T^t \{ (\nabla f) \circ T \}$. Iterating the chain rule, we find that
%
\[ D^{ij} \{ f \circ T \} = \sum_{k = 1}^n \sum_{l = 1}^n T_{lj} T_{ki} \cdot (D^{kl} \{ f \} \circ T). \]
%
Thus the Hessian is $H \{ f \circ T \} = T^t \cdot \{ Hf \circ T \} \cdot T$. This means that if $T^* L_A$ is the operator given by the equation $T^*L_A \{ f \} = L_A \{ f \circ T \} \circ T^{-1}$, then
%
\[ T^*L_A f = (L_{TAT^t} f) \circ T. \]
%
The use of these pullbacks is that if $f \in C^\infty(\RR^n)$, and if $T^* f = f \circ T^{-1}$, then
%
\[ T^* \{ L_A f \} = L_A f \circ T^{-1} = L_A \{ T^* f \circ T \} \circ T^{-1} = (T^* L_A) \{ T^* f \}. \]
%
We can take several important points away from this result:
%
\begin{itemize}
    \item $T^*L = L$ precisely when $TAT^t = A$, which is equivalent to the condition that if $B_A(x,y) = x^t A y$ is the symmetric bilinear form specified by $A$, then $B_A(Tx,Ty) = B(x,y)$. The family of such linear maps from a \emph{generalized orthogonal group}.

    \item Sylvester's law of inertia implies we can write any symmetric $n \times n$ matrix $A$ as $T^{-1} B T^{-t}$, where $B$ is a diagonal matrix consisting of zeroes, ones, and negative ones. But this means that
    %
    \[ T^* L = \sum B_{ii} (D^{ii} \{ f \} \circ T). \]
    %
    Thus the theory of a real-coefficient order two differential equation with constant coefficients is equivalent to one of the form
    %
    \[ Lf = \sum_{1 \leq i < n_1} \frac{\partial^2 f}{\partial x_i^2} - \sum_{n_1 \leq i < n_2} \frac{\partial^2 f}{\partial x_i^2}, \]
    %
    where $n_1 + n_2 \leq n$. This simplifies our calculations considerably, e.g. the theory of an operator $L$ given by a positive definite matrix $A$, i.e. one which has signature $(n,0)$, is equivalent to the theory of the Laplacian $\Delta$. The theory of $L$ given by a $(n+1) \times (n+1)$ matrix with signature $(1,n)$ is equivalent to the theory of the wave operator $\Box = \partial_t^2 - \Delta_x$, and so on and so forth.
\end{itemize}
%
Inspired by these calculations, we now find a fundamental solution to any quadratic partial differential operator $L = \sum B_{ij} \partial^{ij}$, such that $B$ is invertible.

\begin{lemma}
    Let $A$ be a real, invertible matrix with signature $(n_+,n_-)$. If $c_n$ is the surface area of the unit sphere in $\RR^n$, then for $n > 2$, if we let $B = A^{-1}$, and
    %
    \[ L_B = \sum B_{ij} D^{ij}, \]
    %
    then $L_B \{ (A \pm i 0)^{1 - n/2} \} = - (n-2) c_n |\det(A)|^{-1/2} e^{\mp i \pi n_- / 2} \delta_0$.
\end{lemma}
\begin{proof}
    By conjugation, it suffices to show that
    %
    \[ L_B \{ (A + i 0)^{1-n/2} \} = - (n-2) c_n |\det A|^{-1/2} e^{- i \pi n_- / 2} \delta_0. \]
    %
    We will begin by showing that if $A$ is a symmetric matrix with complex coefficients such that $\text{Re}(A)$ is positive definite, and if $B$ is the inverse of $A$, then
    %
    \[ L_B \{ A^{1-n/2} \} = -(n-2) c_n (\det A)^{-1/2} \delta \]
    %
    where the determinant power is defined to be the unique analytic branch such that $(\det A)^{-1/2} > 0$ if $A$ is positive definite. Analytic continuation means we only need to prove this formula if $A$, and thus $B$, is positive definite. But then we can find a matrix $S$ such $S B S^t = I$, and thus $S^{-t} A S^{-1} = I$. But then
    % \det(S) = det(A)^{1/2}
    \begin{align*}
        S^* \{ L_B \{ A^{1-n/2} \} \} &= (S^* L_B) \{ S^* A^{1-n/2} \}\\
        &=  L_I \{ |x|^{2-n} \}\\
        &= \Delta \{ |x|^{2-n} \}\\
        &= - (n-2) c_n \delta_0.
    \end{align*}
    %
    The result then follows because
    %
    \[ (S^{-1})^* \delta_0 = \det(S)^{-1} \delta_0 = \det(A)^{-1/2}. \]
    %
    To obtain the general result, we now note that if
    %
    \[ A_\varepsilon = \varepsilon - i A \quad\text{and}\quad B_\varepsilon = (\varepsilon - iA)^{-1}. \]
    %
    Then
    %
    \[ \det(A_\varepsilon)^{-1/2} = |\det(A)|^{-1/2} e^{i \pi \text{sgn}(A) / 4}, \]
    %
    and $B_\varepsilon \to i B$ and $A_\varepsilon^{1-n/2} \to i^{-1} e^{i \pi n / 4} (A + i0)^{1-n/2}$, so that this gives the result in the limit.
\end{proof}
 
We would hope to define a fundamental solution to the equation $L_B$ as the pull back $A^* \chi_+^{1-n/2}$. This is not quite possible using the spectral analysis of singularities since $A$ is singular at the origin. But provided $A$ is nonsingular, we can define the pullback viewing $A$ as a map from $\RR^n - \{ 0 \} \to \RR$. Since $A^* \chi_+^{1-n/2}$ is then a homogeneous distribution of degree $2-n$ on $\RR^n - \{ 0 \}$, this distribution then uniquely extends to a distribution on $\RR^n$, and we claim this is a constant multiple of a fundamental solution.

\begin{corollary}
    Using the setup to the previous lemma,
    %
    \[ L_B \{ A^* \chi_{\pm}^{1-n/2} \} = \pm 4 \pi^{n/2 - 1} \sin (\pi n_{\pm} / 2) |\det A|^{-1/2} \delta_0. \]
\end{corollary}
\begin{proof}
    The result follows from the previous result given that
    %
    \[ \chi_+^a(s) = \frac{i}{2\pi} \Gamma(-a) ((s - i0)^a e^{i \pi a} - (s + i0)^a e^{-i \pi a}) \]
    %
    for all $a$ that is not a positive integer. To obtain this, we use the fact that $\Gamma(a+1) \Gamma(-a) = - \pi / \sin(\pi a)$, which means that
    %
    \begin{align*}
        \chi_+^a(s) &= \frac{1}{\Gamma(a+1)} s_+^a\\
        &= - \frac{1}{\pi} \Gamma(-a) \sin(\pi a) \cdot s_+^a\\
        &= \frac{i}{2 \pi} \Gamma(-a) (e^{i \pi a} - e^{- i \pi a}) s_+^a.
    \end{align*}
    %
    For $\text{Re}(a) > 0$, $s_+^a$ is locally integrable, and
    %
    \[ (s \pm i0)^a = \begin{cases} s^a &: s \geq 0, \\ |s| e^{\pm i \pi a} &: s < 0. \end{cases} \]
    %
    Thus we conclude that
    %
    \[ (e^{i \pi a} - e^{-i \pi a}) s_+^a = e^{i \pi a} (s - i 0)^a - e^{-i \pi a} (s + i0)^+. \]
    %
    The general case follows by analytic continuation.
\end{proof}

%These calculations are closed related to the behaviour of operators under changes of variables, i.e. working in the language of differential geometry, if we consider two coordinate systems $x: \RR^n \to \RR^n$ and $y: \RR^n \to \RR^n$ on $\RR^n$, where $y$ is equal to $T^{-1}$, then we have the two operators $\nabla_x$ and $\nabla_y$, where $\nabla_x$ is the usual gradient, and
%
%\[ \nabla_y f = \nabla \{ f \circ y^{-1} \} \circ y, \]
%
%then our calculations show that $\nabla_y f = T^t \cdot \nabla_x f$. If we define the Hessians $H_x$ and $H_y$, where $H_x$ is the usual Hessian, and
%
%\[ H_y f = H \{ f \circ y^{-1} \} \circ y, \]
%
%then our calculations show that $H_y = T^t \cdot H_x f \cdot T$. In particular, if we consider a second order constant coefficient partial differential equation
%
%\[ L = \sum g_{ij} D^{ij}, \]
%
%whre $G = \{ g_{ij} \}$ is a real-valued symmetric matrix. Sylvester's law of inertia shows that we can write $G = T A T^t$, where $A$ is a diagonal matrix consisting of zeroes, ones, and minus ones. If we let $y = T$, then

\begin{example}
    Consider the d'Alembertian, or wave operator $\Box f = \partial_t^2 f - \Delta_x f$. The fundamental solutions for this equation are significantly richer than for the Laplace equation. There is no single preferred fundamental solution in this setting. Physical intuition tells us that waves travel at a \emph{finite speed of propagation}, in this case, at a unit velocity. Thus we expect to find a fundamental solution supported on the interior of the \emph{light cone}
    %
    \[ \Sigma = \{ (x,t) \in \RR^n \times \RR: Q(x,t) = 0 \}, \]
    %
    i.e. supported on $\Sigma_+ = \{ (x,t) \in \RR^n \times \RR: Q(x,t) \geq 0 \}$, where $Q(x,t) = t^2 - |x|^2$. In fact, for $n \geq 2$, we have already done the calculations to find such a fundamental solution, namely we have a fundamental solution of the form
    %
    \[ \Phi = \frac{1}{4 \pi^{\frac{n-1}{2}}} Q^* \left(\chi_+^{-\frac{n-1}{2}} \right). \]
    %
    For $n = 1$, this formula actually continues to hold, namely, we have
    %
    \[ \Phi(x,t) = (1/2) \mathbf{I}((x,t) \in \Sigma_+) = (1/2) H(t - x) H(t + x), \]
    %
    though we have to make do with more rudimentary calculations. These choices give fundamental solutions to the wave equation. To verify the equation gives a fundamental solution to the wave operator for $n = 1$, we calculate that
    %
    \begin{align*}
        \partial_t \Phi(x,t) &= (1/2) \delta(t - x) H(t + x) + (1/2) \delta(t + x) H(t - x)\\
        &= (1/2) \left( \delta(t - x) + \delta(t + x) \right) H(t),
    \end{align*}
    %
    and so
    %
    \begin{align*}
        \partial_t^2 \Phi(x,t) &= (1/2) \left( \delta(t - x) + \delta(t + x) \right) \delta(t) + (1/2) \left( \delta'(t-x) + \delta'(t + x) \right) H(t)\\
        &= \delta(x,t) + (1/2) \left( \delta'(t-x) + \delta'(t + x) \right) H(t).
    \end{align*}
    %
    Next, we calculate that
    %
    \begin{align*}
        \partial_x \Phi(x,t) &= (1/2) H(t - x) \delta(t + x) - (1/2) H(t + x) \delta(t-x)\\
        &= (1/2) \left( \delta(t + x) - \delta(t - x) \right) H(t),
    \end{align*}
    %
    and thus
    %
    \begin{align*}
        \partial_x^2 \Phi(x,t) &= (1/2) \left( \delta(t + x) - \delta(t - x) \right) \delta(t) + (1/2) \left( \delta'(t + x) + \delta'(t - x) \right) H(t)\\
        &= (1/2) \left( \delta'(t + x) + \delta'(t - x) \right) H(t).
    \end{align*}
    %
    Thus $\Box \Phi(x,t) = \delta(t,x)$. In particular, we note that when $n$ is odd, then the support of
    %
    \[ \chi_+^{- \frac{n-1}{2}} \]
    %
    is equal to $\{ 0 \}$. This means that when $n \geq 3$ is odd, then $\Phi$ is actually \emph{supported} on $\Sigma$, which hints at \emph{Huygen's principle}, i.e. that in odd dimensions, if $u$ is a solution to the Cauchy problem $\Box u = 0$ with initial conditions $u_0$, then the behaviour of $u$ at a point $(t,x)$ is determined by the values of $u_0$ on a sphere of radius $t$ around $x$.

    There are two important alternate fundamental solutions, the \emph{forward} and \emph{backward}, or \emph{advanced} and \emph{retarded} fundamental solution
    %
    \[ \Phi_+(x,t) = 2 H(t) \Phi(x,t) \quad\text{and}\quad \Phi_-(x,t) = 2 H(-t) \Phi(x,t). \]
    %
    That these two distributions are also fundamental solution follows from the symmetry of all objects involved under time reflection. These are actually the \emph{unique} fundamental solutions supported on the forward and backward light cone and their interior. Indeed, if $u \in \DD^*(\RR^n \times \RR)$ is supported on the forward light cone and $\Box u = 0$, then
    %
    \[ u = (\Box \Phi_+) * u = \Phi_+ * \Box u = 0, \]
    %
    where the convolution is well defined because the projection
    %
    \[ ((x,t), (y,s)) \mapsto (x + y, t + s) \]
    %
    is then proper on $\text{supp}(u) \times \text{supp}(\Phi_+)$.

    The wave equation plays nicely with the Fourier transform. If $\Phi$ is any fundamental solution, and we assume that it is tempered in the $x$-variable (which is the case for the distributions we consider above), supported on $\Sigma$, then applying the Fourier transform in this variable, i.e. setting $\Psi = \mathcal{F}_x \Phi$, then
    %
    \[ \partial_t^2 \Psi(\xi,t) + 4 \pi^2 |\xi|^2 \Psi(\xi,t) = \delta(t) \]
    %
    This implies that, for $t > 0$, we have
    %
    \[ \Psi(\xi,t) = A_+(\xi) e^{2 \pi i |\xi| t} + B_+(\xi) e^{-2 \pi i |\xi| t}, \]
    %
    and a similar formula, allowing different functions, for $t < 0$. Because $\Psi$ is smooth away from the boundary of the light cone, this implies that $A_+$ and $B_+$ are both smooth functions. We calculate that
    %
    \[ \partial_t \Psi(\xi,0) = 2 \pi i |\xi| [A_+(\xi) - B_+(\xi)] \]
    %
    If we assume that $\Psi$ is supported on $t \geq 0$, and we denote it by $\Psi_+$, then for any $\phi \in \DD(\RR^n \times \RR)$, an integration by parts shows that
    %
    \begin{align*}
        \int_0^\infty &\int_{\RR^n} (\partial_t^2 \phi(\xi,t) + 4 \pi^2 |\xi|^2 \phi(\xi,t)) \Psi_+(\xi,t)\\
        &= - [A_+(\xi) + B_+(\xi)] \partial_t \phi(\xi,0) + 2 \pi i |\xi| [A_+(\xi) - B_+(\xi)] \phi(\xi,0).
    \end{align*}
    %
    Thus to get a fundamental solution, we should have $A_+(\xi) + B_+(\xi) = 0$, and $A_+(\xi) - B_+(\xi) = 1 / 2 \pi i |\xi|$. Thus we conclude that
    %
    \[ \Psi_+(\xi,t) = \frac{\sin(2 \pi |\xi| t)}{4 \pi |\xi|}. \]
    %
    This implies that there is a \emph{unique} fundamental solution supported on $t > 0$ and tempered in the $x$-variable, i.e. the fundamental solution $\Phi_+$ we define above, and moreover, we have the Fourier representation
    %
    \[ \Phi_+(x,t) = \frac{H(t)}{2} \int \frac{\sin(2 \pi |\xi| t)}{2 \pi |\xi|} e^{2 \pi i \xi \cdot x}\; d\xi. \]
    %
    % (x,t) -> (0,c)
    % t^2 - |x|^2 = c^2
    %
    % T(0,c) = (x,t)
    %
    % T is [A B]
    %      [C D]
    % A is n x n
    % B is n x 1
    % C is 1 x n
    % D is 1 x 1
    %
    % Bc = x
    % Dc = t
    %
    % D = t/c
    % B = x/c
    %
    % [A  x/c]
    % [C  t/c]
    % [A  x/c] [-1 0] [A^T   C^T]
    % [C  t/c] [ 0 1] [x^T/c t/c]
    % [-A x/c] [A^T   C^T]
    % [-C t/c] [x^T/c t/c]
    % [-AA^T + |x|^2/c^2     -AC^T + xt/c^2]
    % [-CA^T + tx^T/c^2      -CC^T + t^2/c^2]
    % AA^T = 1 + |x|^2/c^2 = t^2/c^2
    % AC^T = xt/c^2
    % CC^T = t^2/c^2 - 1 = |x|^2/c^2
    %
    % so A = (t/c) * O(n)
    % |C| = |x|/c
    % AC^T = xt/c^2
    % Very easy to do this
    % We can pick any C (n degrees of freedom)
    % then there are dim(O(n-1)) degrees of freedom to pick A.
    % Thus total degrees of freedom in picking are n + (n-1)(n-2)/2 = (n^2 - n + 2)/2
    % Total degrees of freedom of Lorenz transformations are (n+1)n/2 = (n^2 + n)/2
    % Thus picking a vector to do this to requires losing 2(n-1) degrees of freedom
    % Thus we should be able to map m <= (n^2 - n + 2) / 4(n-1) approx n/4 vectors to arbitrary positions.
    %
    % A = (t/c) * O(n)
    % B = x/c
    % |C| = |x|/c
    % D = t/c
    %
    % det((t/c)*M  x/c)
    %    (|x|/c    t/c)
    % ( (t/c) * M    x/c ) (M^{-1}   0)
    % (  v    t/c        )  ( 0       1)
    %  (t/c)      x/c
    % vM^{-1}     t/c
    % Has determinant (t/c)^{n+1}
    %
    A similar formula holds for the other fundamental solutions we constructed above, namely
    %
    \[ \Phi_-(x,t) = \frac{H(-t)}{2} \int \frac{\sin(- 2 \pi |\xi| t)}{2 \pi |\xi|} e^{2 \pi i \xi \cdot x}\; d\xi, \]
    %
    and
    %
    \[ \Phi(x,t) = \frac{1}{4} \int \frac{\sin(2 \pi |\xi t|)}{2 \pi |\xi|} e^{2 \pi i \xi \cdot x}\; d\xi. \]
    %
    However, the Fourier calculation above yields some other interesting fundamental solutions, such as the \emph{Feynman fundamental solution}
    %
    \[ \Phi_F(x,t) = \frac{1}{4i} \int \frac{1}{2 \pi |\xi|} e^{2 \pi i (\xi \cdot x + |\xi t|)}\; d\xi  \]
    %
    One quirk of this fundamental solution is that it is \emph{not} contained in the light cone, despite the finite speed of propogation of the d'Alembertian. In fact, $\text{supp}(\Phi_F) = \RR \times \RR^n$. To see this, let $u(t) = \Phi_F(0,t)$ and $v(x) = \Phi_F(x,0)$. That $\text{supp}(\Phi_F) = \RR \times \RR^n$ will follow from the symmetry of $\Phi_F$ under the Lorentz transformation, and that $\text{supp}(u) = \RR$, $\text{supp}(v) = \RR^n$. To obtain this, we can explicitly calculate that for $n = 1$, $v(x)$ is a multiple of $\text{sgn}(x)$, and for $n > 1$, $v(x)$ is a multiple of $|x|^{1-n}$. In both cases, $\text{supp}(v) = \RR^n$. Next, we calculate that
    %
    \[ u(t) = \frac{1}{4i} \int \frac{e^{2 \pi i |\xi t|}}{2\pi|\xi|}\; d\xi = \frac{A_{n-1}}{4i} \int_0^\infty \tau^{n-2} e^{2 \pi i \tau t}\; d\tau. \]
    %
    Thus $u$ is the Fourier transform of a homogeneous function of order $n-2$ in the $\tau$ variable, and is thus a homogeneous function of order $1-n$ in the $t$ variable. Time reversal symmetry, and the fact that $u(t)$ is not supported at the origin (since it's Fourier transform is not a polynomial) imply that $\text{supp}(u) = \RR$.

    TODO: Move to Cauchy Problem For the study of the Cauchy problem for the wave operator, it is useful to study the behaviour of $\Phi_+$ for $t > 0$. Since $\partial_t^2 \Phi - \Delta_x \Phi = 0$, it follows from the general theory of distributional solutions to ODEs that $\Phi \in C^2([0,\infty), \DD(\RR^n)^*)$. In particular, if we consider the distributions $\Phi(t)$ on $\DD(\RR^n)^*$, then both $\Phi(0+) = \lim_{t \to 0^+} \Phi(t)$ and $\partial_t \Phi(0+) = \lim_{t \to 0^+} \partial_t \Phi(t)$ are well defined. Moreover, for $\phi \in \DD(\RR^{n+1})$ we have the representation formula
    %
    \[ \langle \Phi_+, \phi \rangle = \int_0^\infty \langle \Phi_+(t), \phi(t) \rangle\; dt, \]
    %
    since both sides of the equation define homogeneous distributions of order $1-n$ which agree for $\phi \in \DD(\RR^{n+1} - \{ 0 \})$, and thus agree for all $\phi \in \DD(\RR^{n+1})$. But integration by parts implies that
    %
    \begin{align*}
        \phi(0) &= \langle \Box \Phi_+, \phi \rangle\\
        &= \int_0^\infty \langle \Phi_+(t), \Box \phi(t) \rangle\; dt\\
        &= \int_0^\infty \langle \Phi_+(t), \partial_t^2 \phi(t) \rangle - \langle \Phi_+(t), \Delta \phi(t) \rangle\; dt\\
        &= \langle \partial_t \Phi_+(0+), \phi(0) \rangle - \langle \Phi_+(0+), \partial_t \phi(0) \rangle.
    \end{align*}
    %
    Since this is true for arbitrary $\phi \in \DD(\RR^{n+1})$, we conclude that $\Phi_+(0+) = 0$ and $\partial_t \Phi_+(0+) = \delta$. For $t > 0$, we can compute a formula for the distribution $\Phi_+$. The map $(t,x) \mapsto (t^2 - |x|^2, x)$ then has an inverse
    %
    \[ H(s,x) = ( (s + |x|^2)^{1/2} , x). \]
    %
    Thus $|\det(DH)| = (1/2) (s + |x|^2)^{-1/2}$. If $\phi \in \DD(\RR^n \times \RR)$ is supported on $t > 0$, and we set
    %
    \[ \langle \Phi_+, \phi \rangle = \frac{1}{4 \pi^{\frac{n-1}{2}}} \langle \chi_+^{- \frac{n-1}{2}}, \psi \rangle, \]
    %
    where
    %
    \[ \psi(s) = \int \phi(H(s,x)) (s + |x|^2)^{-1/2}\; dx. \]
    %
    If we set
    %
    \[ \tilde{\phi}(t,r) = \frac{1}{r} \int_{|x| = r} \phi(t, x)\; dx, \]
    %
    then
    %
    \[ \psi(s) = \int_{s^{1/2}}^\infty \tilde{\phi}(t, (t^2 - s)^{1/2})\; dt \]
    %
    If we consider the compactly supported distribution $\Phi_+(t)$ on $\RR^n$, then for $\eta \in C^\infty(\RR^n)$,
    %
    \[ \langle \Phi_+(t), \eta \rangle = \frac{1}{4 \pi^{\frac{n-1}{2}}} \int_0^{t^2} \chi_+^{-\frac{n-1}{2}}(s) \tilde{\eta}((t^2 - s)^{1/2})\; ds, \]
    %
    where
    %
    \[ \tilde{\eta}(r) = \frac{1}{r} \int_{|x| = r} \eta(x)\; dx.  \]
    %
    For $n = 1$, we have $\chi_+^0(s) = H(s)$, and
    %
    \[ \Phi_+(t) = (1/2) \mathbf{I}_{[-t,t]}. \]
    %
%    \begin{align*}
%        \langle \Phi_+(t), \eta \rangle &= \frac{1}{4} \int_0^{t^2} \tilde{\eta}((t^2 - s)^{1/2})\; ds\\
%        &= \frac{1}{2} \int_0^t \tilde{\eta}(r) r\; dr\\
%        &= \frac{1}{2} \int_0^t \left( \eta(r) + \eta(-r) \right)\; d\\
%        &= \frac{1}{2} \int_{-t}^t \eta(s)\; ds.
%    \end{align*}
    %
    For $n = 2$, we have $\chi_+^{-1/2} = x_+^{-1/2} \pi^{-1/2}$, and so
    %
    \[ \Phi_+(t) = (1/2 \pi) \cdot \max(t^2 - |x|^2, 0)^{-1/2}. \]
    %
    For odd values $n = 2 m + 1$, we have $\chi_+^{- \frac{n-1}{2}} = \chi_+^{-m} = D^{m-1} \delta_0$, and this leads to the conclusion that
    %
    \[ \langle \Phi_+(t), \eta \rangle = \frac{1}{4 \pi^m} \left. \frac{d^{m-1}}{ds^{m-1}} \left( \tilde{\eta}(s^{1/2}) \right) \right|_{s = t^2}. \]
    %
    Thus this value depends on the values of $\eta$, and their normal derivatives up to order $m-1$, on a sphere of radius $t$. In particular, for $n = 3$, we have
    %
    \[ \langle \Phi_+(t), \eta \rangle = \frac{1}{4 \pi} \tilde{\eta}(t) = \frac{1}{4 \pi t} \int_{|x| = t} \eta(x)\; dx. \]
\end{example}

TODO: Move to Cauchy Problem The existence and uniqueness of solutions to the Cauchy problem for the wave equation follow immediately from this calculation. Namely, if $\phi_0, \phi_1 \in C^\infty(\RR^n)$, and $f \in C^\infty(\RR^n \times [0,\infty))$, then there exists a unique $u \in C^\infty(\RR^n \times [0,\infty))$ such that $\Box u = f$ on $\RR^n \times [0,\infty)$, and we have
%
\[ u(t) = \Phi_+(t) * \phi_1 + (\partial_t \Phi_+)(t) * \phi_0 + \int_0^t (\Phi_+(t-s) * f(s))\; ds. \]
%
Uniqueness follows the fact the fact that if $u$ is a distribution supported on the forward light cone and $\Box u = 0$, then $u = 0$. Conversely, the equation above certainly defines an element of $C^\infty([0,\infty) \times \RR^n)$ and gives a solution since
%
\[ \Box (\Phi_+ * \phi_1) = \delta * \phi_1 = \phi_1 \]
%
\[ \Box (\partial_t \Phi_+ * \phi_0) = \partial_t (\Box \Phi_+ * \phi_0) = \partial_t \phi_0 = 0 \]
%
and since $\Phi_+(0) = 0$, and $\partial_t \Phi_+(0) = \delta$, we find that
%
\begin{align*}
    \partial_t \left\{ \int_0^t (\Phi_+(t-s) * f(s))) \right\} = \int_0^t (\partial_t \Phi_+(t-s) * f(s))
\end{align*}
%
and so
%
\[ \partial_t^2 \left\{ \int_0^t (\Phi_+(t-s) * f(s))\; ds \right\} = f(t) + \int_0^t (\partial_t^2 \Phi_+(t-s) * f(s)). \]
%
Since $\Box \Phi_+$ vanishes away from the origin,
%
\[ \Box \left\{ \int_0^t (\Phi_+(t-s) * f(s))\; ds \right\} = f(t), \]
%
which completes the proof.

\begin{example}
    Consider the Cauchy-Riemann equations on $\RR^2$, i.e. the operators $\partial_z = (1/2)(\partial_x - i \partial_y)$ and $\partial_{\overline{z}} = (1/2)(\partial_x + i \partial_y)$. The divergence theorem tells us that
    % F = (phi/2, -i \phi / 2)
    \[ \int_\Omega \frac{\partial \phi}{\partial z}\; dx\; dy = \frac{1}{2} \int_{\partial \Omega} \phi\; d\overline{z} \]
    %
    and
    %
    \[ \int_\Omega \frac{\partial \phi}{\partial \overline{z}}\; dx\; dy = \frac{1}{2} \int_{\partial \Omega} \phi\; dz \]
    %
    The variant of Green's formula then tells us that
    %
    \[ \int_\Omega \left( u \frac{\partial v}{\partial z} + \frac{\partial u}{\partial z} v \right)\; dx\; dy = \frac{1}{2} \int_{\partial \Omega} uv\; d\overline{z}. \]
    %
    Thus the operators $\partial_z$ and $\partial_{\overline{z}}$ are self adjoint. In particular, applying this formula gives that for any $\phi \in \DD(\RR^2)$,
    %
    \begin{align*}
        \int_{|z| \geq \varepsilon} \frac{\partial \phi}{\partial \overline{z}} \left( \frac{1}{\pi i} \frac{1}{z} \right)\; dx\; dy &= \frac{1}{2 \pi i} \int_{|z| = \varepsilon} \frac{\phi}{z}\; dz\\
        &= \frac{1}{2 \pi} \int_0^{2\pi} \phi(\varepsilon e^{i \theta})\; d\theta\\
        &= \phi(0) + O(\varepsilon).
    \end{align*}
    %
    Taking $\varepsilon \to 0$ allows us to conclude that
    %
    \[ \Phi(x,y) = \frac{1}{i \pi} \frac{1}{x + iy} \]
    %
    is a fundamental solution to the operator $\partial_{\overline{z}}$. Similarily,
    %
    \[ \Phi(x,y) = \frac{1}{i \pi} \frac{1}{x - iy} \]
    %
    is a fundamental solution to the operator $\partial_z$. The first fundamental solution is called the \emph{Cauchy Kernel}.
\end{example}

A simple consequence of the fundamental solution technique is the intriguing statement that \emph{any} distribution is a successive derivative of a family of continuous functions.

\begin{theorem}
    For any open set $U \subset \RR^n$, and any distribution $u \in \DD(U)^*$, there exists $f_\alpha \in C(U)$, such that the cover $\{ \text{supp}(f_\alpha) \}$ is locally finite on $U$, and such that $u = \sum_\alpha D^\alpha f_\alpha$.
\end{theorem}
\begin{proof}
    Suppose first that $U = \RR^n$. For any $m$, the function
    %
    \[ \Phi(x) = \max(x_1,0)^m \dots \max(x_n,0)^m / (m!)^n \]
    %
    is a fundamental solution to the operator $L = (\partial_1 \cdots \partial_n)^{m+1}$, and lies in $C^{m-1}(\RR^d)$. If $u$ is a distribution of order $m-1$, then it follows that $v = \Phi * u$ lies in $C(\RR^d)$, and $Lv = u$, which completes the proof in this case. In general, we just localize.
\end{proof}

The fundamental solutions to the heat equation and Cauchy-Riemann equation are smooth away from the origin. It will turn out that this is true for any constant coefficient partial differential equation which is \emph{hypoelliptic}. Here are several consequences.

\begin{lemma}
    Suppose $\Phi$ is a fundamental solution to a constant coefficient partial differential operator $L$ on $\RR^n$, and suppose that $\singsupp(\Phi) = \{ 0 \}$. Then for any open set $U \subset \RR^n$, and any distribution $u$ on $U$, $\singsupp(Lu) = \singsupp(u)$. Thus any distribution $u$ on $U$ with $Lu = 0$ lies in $C^\infty(U)$. If $\{ u_n \}$ is a sequence of solutions to the equation $Lu = 0$, with $u_n \to u$ distributionally, then $u_n \to u$ in $C^\infty(U)$.
\end{lemma}
\begin{proof}
    It is simple to see that $\singsupp(Lu) \subset \singsupp(u)$. Conversely, fix a compact set $K \subset U$, and consider $\psi \in \DD(U)$ equal to one on a neighbourhood of $K$. Then for any distribution $u$ on $U$ with $Lu = 0$,
    %
    \begin{align*}
        K \cap \singsupp(Lu) &= K \cap \singsupp(L(\psi u))\\
        &= K \cap \singsupp(\Phi * (\psi u))\\
        &= K \cap [\singsupp(\Phi) + \singsupp(\psi u)]\\
        &= K \cap \singsupp(u).
    \end{align*}
    %
    The first part is therefore completed, since $K$ was arbitrary. Now given the Fr\'{e}chet space $X = \{ u \in \DD(U)^* : Lu = 0 \}$, we have a natural inclusion map $i: X \to C^\infty(U)$. It's graph is obvious closed, so by the closed graph theorem, this map is continuous, and this implies the sequence property above.
\end{proof}

The fundamental solutions above are actually better than just being smooth away from the origin, they are \emph{analytic} away from the origin. This is true of any \emph{elliptic} partial differential equation with constant coefficietns. The following lemma then applies.

\begin{lemma}
    Suppose $\Phi$ is a fundamental solution to a constant coefficient partial differential operator $L$ on $\RR^n$, and suppose that $\Phi$ is real analytic away from the origin. Then for any open set $U \subset \RR^n$, there is an open set $V \subset \CC^n$ such that if $u$ is a distribution on $U$, and $Lu = 0$, then $u$ is an analytic function extending to a complex analytic function on $V$. In particular, we conclude that if $u$ vanishes in a neighborhood of a point $x_0 \in U$, then $u$ vanishes on the connected component of $U$ containing $x_0$. Moreover, if $u_n$ is a sequence in $C^\infty(U)$ with $Lu_n = 0$, and with $u_n \to u$, then the extension of the functions $u_n$ to analytic functions on $V$ converge locally uniformly to the extension of $u$.
\end{lemma}
\begin{proof}
    Choose any open set $\RR^n - \{ 0 \} \subset V_0 \subset \CC^n$ such that $\Phi$ extends to a complex analytic function on $V_0$. Since the question is local, consider a compact set $K$, and $\psi \in \DD(U)$ equal to one on a neighborhood of $K$. Then for $x \in K$, we have
    %
    \[ u(x) = (\psi u)(x) = (\Phi * L(\psi u))(x) = \int \Phi(x-y) L(\psi u)(y)\; dy, \]
    %
    and this formula continues to define a complex analytic function if we replace $x$ with a complex parameter $z$ provided that for any $y \in \text{supp}(\nabla (\psi u))$, $z - y \in V_0$. This is a neighborhood of $K$. We can then perform this extension for many different choices of $\psi$, and thus of $K$, to show that $u$ extends to an analytic function on some set $V$ depending only on $U$. To get the latter part of the Lemma, we just apply the closed graph theorem.
\end{proof}

Let us now extend the approximation theorem of Runge to the case of general elliptic partial differential equations.

\begin{lemma}
    Suppose $\Phi$ is a fundamental solution to a constant coefficient partial differential operator $L$ on $\RR^n$, and suppose that $\Phi$ is real analytic away from the origin. Consider two open sets $V \subset U \subset \RR^n$, such that $U - V$ does not have a compact connected component. Then every solution $v \in C^\infty(V)$ to the equation $Lv = 0$ is a limit in $C^\infty(V)$ of the restriction of solutions $u \in C^\infty(U)$ to the equation $Lu = 0$.
\end{lemma}
\begin{proof}
    By duality, it suffices to show that if $w$ is a compactly-supported distribution in $V$, and
    %
    \[ \int_V w(x) u(x)\; dx = 0\ \]
    %
    for any $u \in C^\infty(U)$ such that $Lu = 0$, then for any $v \in C^\infty(V)$ with $Lv = 0$,
    %
    \[ \int_V w(x) v(x)\; dx = 0. \]
    %
    If $L^*$ is the adjoint of the operator $L$, and we set $\Psi(x) = \Phi(-x)$, then $\Psi$ is an analytic fundamental solution to $L^*$. If we can show that $\Psi * w$ is a compactly supported distribution in $V$, then for any $v \in C^\infty(V)$ with $Lv = 0$,
    %
    \[ \int w(x) v(x)\; dx = \int L^* (\Psi * w)(x) v(x)\; dx = \int (\Psi * w)(x) Lv(x)\; dx = 0. \]
    %
    This would therefore complete the proof.

    Let $K$ denote the support of $w$. The distribution $\Psi * w$ is analytic function outside of $K$. Moreover, the support of $\Psi * w$ is certainly contained in $U$, because if $x \not \in U$, then the function $u(y) = \Psi(x-y)$ satisfies $Lu = 0$, and therefore by assumption on $w$, $(\Psi * w)(x) = 0$. Applying analyticity, it follows that $\Psi * w$ vanishes in every component of $K^c$ containing a point in $U^c$. We claim this is sufficient to prove the result. If $W$ is a bounded connected component of $K^c$ contained in $U$, then the set $W \cap V^c$ is a compact connected component of $U - V$. This contradicts the hypothesis of the Lemma, and therefore $W$ cannot exist. Because $K^c$ has at most one unbounded connected component, we conclude that the theorem is proved, except in the case where $U$ contains the unbounded connected component of $K^c$. We will address this case in two paragraphs. For now, note that this proves the theorem when $U$ is a open ball.

    Next, we address the case where $U = \RR^d$, and $V$ is a ball $B$, without loss of generality, centred at the origin. This is because if $v \in C^\infty(B)$ solves $Lv = 0$, then for each $k$, we can recursively find a sequence $u_{k,i} \in C^\infty(k B)$, such that $Lu_{k,i} = 0$, and such that $u_{1,i}$ converges to $v$ in $C^\infty(B)$ as $i \to \infty$, such that $\| u_{k+1,i} - u_{k,i} \|_{L^\infty(k B)} \leq 1/2^{k+i}$ for any $|\alpha| \leq k + i$. It follows that we can define $u_i \in C^\infty(\RR^d)$ as the pointwise limit of the functions $u_{k,i}$ (smooth because of analyticity). Then $Lu_i = 0$, and $u_i \to u$ in $C^\infty(B)$.

    Now returning to the general case of the proof, if $K$ is contained in a ball $B$ of radius $R$, then for $|x| > R$, the function $u_x(y) = \Phi(y-x)$ solves the equation $Lu = 0$ in $B$, and thus is the limit in $C^\infty(B)$ of functions $u_i$ solving $Lu_i = 0$ in $C^\infty(\RR^n)$. But then
    %
    \[ (\Phi * w)(x) = \int w(x) u(x)\; dx = \lim_{i \to \infty} \int w(x) u_i(x)\; dx = 0, \]
    %
    which completes the proof.
\end{proof}

\begin{remark}
    This result is \emph{only} possible given the topological hypothesis. Suppose the result is true, and $K$ is a compact connected component of $U - V$. If $u_i \in C^\infty(U)$ is a sequence of solutions to the equation $Lu_i = 0$ such that $u_i$ converges to a function $u \in C^\infty(V)$, then we claim that $u_i$ converges to some $u \in C^\infty(V \cup K)$. In particular, we conclude every solution on $C^\infty(V)$ extends to a solution on $C^\infty(V \cup K)$. This causes a contradiction, since if $x_0 \in K$, then the function $u(y) = \Phi(y - x_0)$ solves the equation on $V$, but cannot be extended to a solution on $V \cup K$ (e.g. because any such solution would be analytic).

    Let us now prove that $u$ can be extended under our assumptions. If we consider $\phi \in \DD(V \cup K)$ equal to one on a neighborhood $W$ of $K$, then $L(\phi u_i)$ vanishes on $W$, and hence is supported in $V$. Since $u_i$ converges to $u$ in $V$, so $L(\phi u_i)$ converges to $Lu$ in $V \cup K$, and this means that $u_i = \Phi * L(\phi u_i)$ converges in $C^\infty(V \cup K)$ to some function $u$ in $C^\infty(V \cup K)$, which completes the proof of our claim.
\end{remark}

Using Runge's approximation theorem, we can now finally prove our first major existence theorem for partial differential equations on proper subsets $U$ of $\RR^n$, assuming the existence of a fundamental solution (i.e. the existence of solutions to the equation $Lu = f$ for compactly supported distributions $f$ on $\RR^n$. To obtain the idea of the proof of the existence of solutions to the equation $Lu = f$, for a distribution $f$ on $U$ (not necessarily compactly supported), imagine that such a solution $u$ existed. For any $\phi \in \DD(U)$ equal to one on some open set $W$, we can find a distribution $u_W$ such that $L(u_W) = \phi f$. Then $L(u - u_W) = 0$ on $W$, which implies that $u - u_W$ is analytic on $W$. If $u_V - u_W = 0$ on $V \cap W$ for any $V,W \subset U$, these solutions would combine to give us a choice of $u$ on the whole domain. Instead, we only have $u_V - u_W$ \emph{analytic} on $V \cap W$. The Runge approximation theorem will enable us to adjust these functions so that $u_V - u_W$ is \emph{almost} equal to zero, and then we can take limits.

\begin{theorem}
    Suppose $\Phi$ is a fundamental solution to a constant coefficient partial differential operator $L$ on $\RR^n$, and suppose that $\Phi$ is real analytic away from the origin. Then for any open set $U$ in $\RR^n$, and any distribution $f \in \DD(U)^*$, then there exists $u \in \DD(U)^*$ such that $Lu = f$.
\end{theorem}
\begin{proof}
    Let $U_R = \{ x \in U: |x| < R\ \text{and}\ d(x,U^c) > 1/R \}$. Then we can apply Runge's approximation theorem for $U_R \subset U$. We leave this proof of this to the end for it is fairly technical and unenlightening. Now if $\phi_i \in \DD(U)$ equals one on $U_i$, we set $u_i = \Phi * (\phi_i f)$. Then $Lu_i = \phi_i f$, in particular, $Lu_i = f$ on $U_i$. It suffices to modify $u_i$ so that it converges distributionally to some function $u$, since it will then follow that $Lu = f$ on $U$. we note that $L(u_{i+1} - u_i) = 0$ in $U_i$. Thus Runge's theorem implies that there exists an analytic function $w_i$ defined on $U$ such that $\| (u_{i+1} - u_i) - w_i \|_{L^\infty(U_i)} \leq 1/2^i$. But then if we define $v_i = u_i - w_1 - \dots - w_{i-1}$, then $Lv_i = \phi_i f$ on $U$, and
    %
    \[ \| v_{i+1} - v_i \|_{L^\infty(U_i)} = \| (u_{i+1} - u_i) - w_i \|_{L^\infty(U_i)} \leq 1/2^i, \]
    %
    which implies the required convergence property.

    To show that $U - U_R$ has no compact connected components, consider a potential compact connected component $K$ of $U - U_R$. If $B_R$ is the ball of radius $R$ centred at the origin, $K \cap B_R^c$ is compact. It is also open in $B_R^c$, because we can write $K = V \cap U \cap U_R^c$ for some open set $V$, and then $K \cap B_R^c = V \cap U \cap U_R^c \cap B_R^c = V \cap U \cap B_R^c$ since $U_R \subset B_R$, and $U$ and $V$ are both open. But this means that $K \cap B_R^c = \emptyset$, i.e. $K \subset B_R$. Next, we claim $d(x,y) > 1/R$ for any $y \in U^c$. Otherwise, take $y$ with $d(x,y) \leq 1/R$, and consider the line segment $L$ between $x$ and $y$. Since $U^c$ is closed, we may assume without loss of generality (by picking the closest point in $U^c$ to $x$ on the line and replacing $y$ with it), that $L \cap U^c = \{ y \}$. Since $x \not \in U_R$, and $d(x_1,y) < 1/R$ for any other point $x_1$ on the line, $L \cap U_R = \emptyset$. Thus $L - \{ y \} \subset U - U_R$. But $L - \{ y \}$ is connected and contains $x \in K$, so $L - \{ y \} \subset K$. But this is impossible for it implies that $\partial K$ contains $y$, which is not an element of $K$, so $K$ cannot be closed. Put together, this means that for any $x \in K$, $d(x,U^c) = 1/R$. But then it is impossible for $K$ to be an open subset of $U - U_R$, so $K$ cannot exist.
\end{proof}

\begin{example}
    Suppose $L = \sum_{|\alpha|} c_\alpha D^\alpha = P(D)$ is a homogeneous elliptic partial differential equation of degree $m$ on $\RR^n$. Then $1/P(\xi)$ is a homogeneous function defined on $\RR^n - \{ 0 \}$, of degree $-m$, analytic in $\RR^n - \{ 0 \}$. Thus we have a fundamental solution $\Phi$ defined such that $\widehat{\Phi} = 1/P$.
    %
    \begin{itemize}
        \item If $n > m$, then $L$ has a fundamental solution $\Phi = \Phi_0$ is homogeneous of degree $m-n$, and $C^\infty$ in $\RR^n - \{ 0 \}$.

        \item If $n \leq m$, then $L$ has a fundamental solution $\Phi = \Phi_0 - Q(x) \log |x|$, where $\Phi_0$ is homogeneous of degree $m - n$ and
        %
        \[ Q(x) = \frac{1}{(2\pi)^m (m-n)!} \int_{|\xi| = 1} \frac{(2\pi i x \cdot \xi)^{m-n}}{P(\xi)}\; d\xi. \]
        %
        TODO: Technical FT Calculation in H\"{o}rmander Volume 1, Theorem 7.1.20.
    \end{itemize}
    %
    In either case, $\Phi_0$ is analytic in $\RR^n - \{ 0 \}$.
\end{example}








\section{The Cauchy Problem}

Recall the Cauchy problem: Given a linear differential operator $L = \sum c_\alpha D^\alpha$ of order $m$, defined on an open set $\Omega_0 \subset \RR^{n+1}$, with coefficients in $C^\infty(\Omega_0)$. We suppose that $\Omega_0$ is divided into two parts by a smooth hypersurface $S$. Thus $\Omega_0$ is divided into the union of three disjoint set $\Omega_0^- \cup S \cup \Omega_0^+$, where $\Omega_0^-$ and $\Omega_0^+$ are connected, open sets.

Classically, the goal of the Cauchy problem, given some data $f$ on $\Omega_0$, and $\phi_0,\dots,\phi_{m-1}$ on $S$, is to find a solution $Lu = f$, such that the first $m-1$ derivatives of $u$ in the direction tangent to $S$ agrees with the functions $\{ \phi_i \}$. For smooth data this is often a fine way to pose the problem; but for more general data. Instead, we consider $f \in \DD(\Omega_0)$, and 

Consider a linear differential operator $L = \sum c_\alpha D^\alpha$ of order $m$, defined on an open set  Finally, we consider some $f \in \DD(\Omega_0)^*$.

For rough data, the best way to formulate the Cauchy problem is as follows: We consider some initial data $\phi \in \DD$


\emph{Cauchy problem} for this data is, given some set $X \subset \DD(\Omega_0)^*$, to find $u \in X$ such that $Lu = f$.

\subsection{First Order Operators}

TODO

\subsection{Calderon's Uniqueness Theorem}

Let us discuss the phenomenon of \emph{uniqueness} for the Cauchy problem. Thus our goal is to determine, for a given set $X \subset \DD(\Omega_0)^*$, whether $L: X \to \DD(\Omega_0)^*$ is injective. If $X$ is a linear subspace of $\DD(\Omega_0)^*$, it suffices to show that if $u \in X$, and $Lu = 0$, then $u = 0$. This simplifies the notation in the problem. To make the problem possible to study microlocally, we work locally. Thus we focus on the case where $X$ solely contains functions which vanish on $\Omega_0^-$, and will try and find conditions which guarantee that if $Lu = 0$, then $u$ vanishes on a neighborhood of $S$.

Let $P(x,\xi)$ denote the principal symbol of $L$. We begin by assuming that $S$ is \emph{non characteristic} at any of it's points, i.e. that if $x_0 \in S$, and $\eta_0 \in T_{x_0}^* \RR^{n+1} - \{ 0 \}$ has $T_{x_0} S$ as it's kernel (conormal to $S$), then $P(x_0,\eta_0) \neq 0$. We actually assume $S$ is \emph{strongly non characteristic}, i.e. that for any $\eta_0 \in T_{x_0}^* \RR^{n+1}$ conormal to $S$, and any $\eta_1 \in T_{x_0}^* \RR^{n+1}$ not conormal to $S$, the polynomial $P(x_0, \eta_0 + z \eta_1)$ has $m$ simple roots in the $z$-variable. For $m = 1$, this is equivalent to being noncharacteristic. For $m > 1$, the surface $S$ can be noncharacteristic but not satisfy this property, for instance, if $S$ is the $x$-axis, and $L = (D_x + i D_y)^2 = D_x^2 + 2i D_x D_y - D_y^2$.

Being strongly noncharacteristic is a \emph{local} conditions on $L$ and of $S$; if we replace $S$ with any other surface tangent to $S$ at $x_0$, the condition will continue to hold in a neighborhood of $x_0$. Thus, working locally, we may switch to a coordinate system in which it suffices to study a more familiar version of the Cauchy problem, namely, we consider a set $\Omega_0 = (-T,T) \times \Omega$, where $\Omega$ is a connected, open subset of $\RR^n$, $T > 0$, where $S = \{ 0 \} \times \Omega$, and where
%
\[ \Omega^- = \{ (t,x) \in (-T,T) \times \Omega : t < 0 \} \quad\text{and}\quad \Omega^+ = \{ (t,x) \in (-T,T) \times \Omega: t > 0 \}. \]
%
The principal symbol will then be written in the $(t,x)$ variables, and their dual variables $(\tau,\xi)$ variables, i.e. as the polynomial $P(t,x,\tau,\xi)$. The assumption slightly stronger than being noncharacteristic then locally amounts to the fact that for any $\xi \in \RR^n - \{ 0 \}$, and any $z \in \CC$, if $P(t,x,z,\xi) = 0$, then $\partial_z P(t,x,z,\xi) \neq 0$. It follows that locally on $\Omega$ we can write the solutions to $P(t,x,\tau,\xi) = 0$ in the $\tau$ variable for $1 \leq i \leq m$ as the smooth functions $\tau_1,\dots,\tau_m: (-T,T) \times \Omega_x \times (\RR^n_\xi - \{ 0 \}) \to \CC$. Given that $P$ is homogeneous in the $\xi$ variable of degree $m$, $\tau_1,\dots,\tau_m$ will be homogeneous of degree one.

\begin{example}
    Let $L = \partial_t^2 + \Delta_x$. Then $L$ has principal symbol $-4 \pi^2 (\tau^2 + |\xi|^2)$, and has roots $\pm i |\xi|$. The principal symbol factors as $- 4 \pi^2 (\tau - i |\xi|) (\tau + i |\xi|)$.
\end{example}

Since $P$ is a homogeneous polynomial of degree $m$, the coefficient of $\tau^n$ in the expansion of $P$ must be a non-vanishing, $C^\infty$ function of $P$. If it vanished at a point $(t_0,x_0)$, then we would have $P(t_0,x_0,\tau,0) = 0$ for all $\tau \in \RR$, which contradicts the fact that $L$ is strongly noncharacteristic since then $\partial_\tau P(t_0,x_0,\tau,0) = 0$. If $a \in C^\infty((-T,T) \times \Omega)$ is the coefficient of $\tau^n$, it is therefore clear that the solutions of the operator $Lu = 0$ are the same as solutions of the operator $a^{-1} \cdot Lu = 0$. Thus in the sequel, we will assume without loss of generality that $a(t,x) = 1$ for all $t \in (-T,T)$ and all $x \in \Omega$.

\begin{lemma}
    Suppose that $L$ is a differential operator of order $m$ on $(-T,T) \times \Omega$, such that $S = \{ (t,x): t = 0 \}$ is strongly noncharacteristic with respect to $L$. Then there exists $m$ classical pseudodifferential operators $Z_1,\dots,Z_m$, each of order one, with principal symbols $z_1,\dots,z_m$ such that
    %
    \[ L = (D_t - Z_1) \dots (D_t - Z_m) + R, \]
    %
    where $R$ is a smoothing operator.
\end{lemma}
\begin{proof}
    The composition calculus shows that $L - (D_t - z_1(t,x,D_x)) \dots (D_t - z_m(t,x,D_x))$ is a pseudodifferential operator $R_1$ of order $m-1$. As with many regularizing calculi. We proceed asymptotically, like in the construction of parametrices. Given a choice of operators $\{ Z_i^N \}$ such that $R_N = L - (D_t - Z_1^N) \dots (D_t - Z_m^N)$ is order $m-N$, we must consider operators $W_1,\dots,W_m$ of order $1-N$ such that $R_{N+1} = L - (D_t - Z_1^N - W_1) \dots (D_t - Z_m^N - W_m)$ is an operator of order $m - N - 1$. If $w_1,\dots,w_n$ are the symbols of $W_1,\dots,W_m$, then it clearly suffices to choose these symbols such that if $r_N$ is the principal symbol of $R_N$, then
    %
    \begin{align*}
        r_N(t,x,\tau,\xi) &= \sum_{i = 1}^n w_i \left[ \prod_{j \neq i} (\tau - z_j) \right].
    \end{align*}
    %
    If we pick
    %
    \[ w_i = \frac{r_N}{\partial_z P(t,x,z_i,\xi)} = \frac{r_N}{\prod_{j \neq i} (z_i - z_j)}, \]
    %
    then $w_k$ is homogeneous of degree $m-N-(m-1) = 1-N$, and since the rational function
    %
    \[ f(\tau) = \sum_{i = 1}^n \prod_{j \neq i} \frac{\tau - z_j}{z_i - z_j} \]
    %
    is a degree $m-1$ polynomial with $f(z_1) = \dots = f(z_m) = 1$, and so $f(\tau) = 1$ for all $\tau \in \RR$, which implies the required formula for $r_N$. Now we just take $Z_i = \lim Z_i^N$, and we get the required result.
\end{proof}

The next remark shows that we can assume that there exists a compact set $K'$ such that $\text{supp}_x(u) \subset K'$, such that the coefficients of $L$ vanish outside of $K'$, and the $(x,y)$ support of the kernels of $Z_1,\dots,Z_m$, and of $R$ vanish outside of $K'$, uniformly in $t$.

\begin{remark}
    Switch to a coordinate system in which $S = \{ (t,x) \in (-T,T) \times \Omega : t = |x|^2 \}$, after possibly shrinking $\Omega$. In this coordinate system, if $u$ vanishes on $\Omega_0^-$, then $u(t,x)$ vanishes for $t < |x|^2$. Thus, after shrinking $T$, we may assume that $\Omega$ contains a ball of radius $T^2$, and then $\text{supp}_x(u)$ is contained in that ball of radius $T^2$. Switching back to our original coordinate system, we may assume without loss of generality that there exists a compact set $K \subset \Omega$ such that $\text{supp}_x(u) \subset K$. If $g \in C_c^\infty(\Omega)$ is equal to one on a neighborhood of $K$, then $0 = Lu = g \cdot Lu = (D_t - g Z_1) \dots (D_t - g Z_m) + R^\# u$, which is a similar expansion to the original except that we can now assume that the $(x,y)$ supports of the kernels of all operators involved are contained in $K' \times K'$ for some compact set $K'$.
\end{remark}

Define $u_1 = u$, and for $1 \leq i \leq m-1$, iteratively define
%
\[ u_{i+1} = (D_t - Z_i) u_i. \]
%
Then since $L = (D_1 - Z_1) \dots (D_n - Z_m) u + R$, it follows that
%
\[ (D_t - Z_m) u_m = Lu - Ru = -Ru. \]
%
We can then consider the diagonal matrix valued pseudodifferential operator
%
\[ A = i \cdot \text{diag}(Z_1,\dots,Z_m). \]
%
Let $N$ be the standard nilpotent matrix, with ones directly above the diagonal. Let $\tilde{R}(t)$ be the $m \times m$ equal to $R(t)$ in the lower left corner. Then if $U = (u_1,\dots,u_m)$, then
%
\[ \mathcal{L}U = \frac{\partial U}{\partial t} - A(t) U - i [N - \tilde{R}(t)] U = 0. \]
%
Let us now assume that $u \in C^i((-T,T), H^{m-j}_c(\Omega))$. Then $u_k \in C^j((-T,T), H^{m-k+1-j}_c)$. Thus for $j = 0$ and $j = 1$, $U \in C^j((-T,T), H^{1-j}_c(\Omega) \otimes \CC^m)$.

The uniqueness will now follow from an \emph{inequality of Carleman type}. Namely, if $V(t,x) = \chi(t) U(t,x)$, where $\chi(t) = 1$ for $t < 8T/10$, and vanishes for $t > 9T/10$, then for all suitably large positive values $\rho > 0$,
%
\[ \rho \int_0^T \int_\Omega e^{\rho(T-t)^2} |V(t,x)|^2\; dx\; dt \lesssim \int e^{\rho (T - t)^2} |\mathcal{L}V(t,x)|^2\; dx\; dt. \]
%
We then claim that $U$ vanishes for $t < T/2$. Indeed, we can restrict the integration on the right hand side to the region $8T/10 \leq t \leq T$, because $\mathcal{L} U$ because it satisfies the ODE. But if we limit the integration on the left hand side to $0 \leq t \leq t/2$, we get that
%
\[ \rho \cdot e^{\rho (24/25) T^2} \int_0^{t/2} \int_\Omega |V(t,x)|^2\; dx \;dt \lesssim \int_{8T/10}^T \int_\Omega | \mathcal{L} V |^2 \; dx\; dt. \]
%
Unless the right hand side is equal to zero, it will become arbitrarily large as $\rho \to \infty$, which eventually gives a contradiction. Thus $V$ must vanish on $0 \leq t \leq T/2$.

\begin{remark}
    Note that we did not use $\rho$ in the proof. It's utility is that if $\rho$ is suitably large, then we can replace $\mathcal{L}$ by the diagonal part $\mathcal{L}_0 = I_m (\partial / \partial t) - A(t)$, and thus discard the other terms as extraneous.
\end{remark}

Calderon proved this result (Uniqueness in the Cauchy problem of partial differential equations, 1958) under two assumptions:
%
\begin{itemize}
    \item The roots $z_1,\dots,z_m$ are all valued in $\CC - \RR$, i.e. so that $P$ is elliptic.

    \item The roots $z_1,\dots,z_m$ are all valued in $\RR$. Under this condition, we say that $z_1,\dots,z_m$ are \emph{strongly hyperbolic}.
\end{itemize} 
%
TODO: Rest of argument (Treves, Vol 1. Starting on page 112).














\chapter{Symbol Classes}

In various settings in harmonic analysis, especially generalizations of settings where \emph{homogeneous functions} are the classical objects of study, it is useful to study various \emph{symbol classes}. For instance, pseudodifferential operators historically dealt with operators $a(x,D)$, where $a: \RR^d_x \times \RR^d_\xi \to \CC$ is a smooth function defined by a finite, or asymptotic sum of homogeneous functions of various orders in the $\xi$ variable. If the highest degree of the terms in the sum was $t$, then for any $n$ and $m$, $\nabla^n_x \nabla^m_\xi a$ is alsoa sum of homogeneous functions, with highest degree $t - m$. Thus we have bounds of the form
%
\[ | \nabla^n_x \nabla^m_\theta a(x,\xi) | \lesssim \langle \xi \rangle^{t - m}. \]
%
Given a quantity $t$, a non-negative integer $p$, and an open subset $\Omega \subset \RR^d$, we define the symbol class $\mathcal{S}^t(\Omega \times \RR^p)$, consisting of \emph{symbols of order $t$}, as the family of all functions $a \in C^\infty(\Omega_x \times \RR^p_\theta)$ such that
%
\[ |\nabla^n_x \nabla^m_\theta a(x,\theta)| \lesssim_{n,m} \langle \theta \rangle^{t-m}. \]
%
where the implicit constant is uniform in $x$. We take the optimal constants in these inequalities as a family of seminorms which gives $\mathcal{S}^t(\Omega \times \RR^p)$ the structure of a Frech\'{e}t space. Similar to other function spaces, we can also consider the local symbol classes $\loc{\mathcal{S}^t}(\Omega \times \RR^p)$.

The classes $\mathcal{S}^t(\Omega \times \RR^p)$ are decreasing as $t \to -\infty$, and we define $\mathcal{S}^{-\infty}(\Omega \times \RR^p)$ to be the intersection of all these classes of symbols. Operators defined by such functions are often highly regular. For instance, a pseudodifferential operator defined by such a symbol is called a \emph{smoothing operator}, and maps any compactly supported distribution to a smooth function. The class $\mathcal{S}^{-\infty}(\Omega \times \RR^p)$ is dense in any of the classes $\mathcal{S}^t(\Omega \times \RR^p)$, since it contains any symbol compactly supported in $\theta$, and we can take cutoffs as $\theta \to \infty$.

A useful strategy to understand a symbol is to break it down into an asymptotic series of simpler symbols. Suppose $\{ a_n \}$ is a sequence of symbols, then we write
%
\[ a \sim \sum_{n = 0}^\infty a_n \]
%
for some symbol $a$, if for any $t \in \RR$, there exists $N_0$ such that for $N \geq N_0$, $a - \sum_{n = 0}^N a_n$ is a symbol of order $t$. If $a_n$ is a symbol of order $t_n$, and $\lim_{n \to \infty} t_n = -\infty$, then a symbol $a$ always exists satisfying these asymptotics.

\begin{theorem}
    Consider a sequence of symbols $\{ a_n \}$, with $a_n \in \mathcal{S}^{t_n}(\Omega \times \RR^p)$, where $\lim_{n \to \infty} a_n = -\infty$, and let $t = \max t_n$. Then there exists a symbol $a \in \mathcal{S}^t(\Omega \times \RR^p)$ such that $a \sim \sum a_n$.
\end{theorem}
\begin{proof}
    Fix a bump function $\phi \in \DD(\RR^p)$ equal to 0 when $|x| \leq 1/2$, and equal to one when $|x| \geq 1$. Find a rapidly increasing sequence $\{ r_n \}$ such that
    %
    \[ | \nabla_x^j \nabla_\lambda^k \{ \phi( \theta / r_n ) a_n(x,\theta) \} | \leq 2^{-n} \langle \theta \rangle^{t_n + 1 - k} \]
    %
    for $x \in \Omega$, where $i,j \leq n$. We define
    %
    \[ a(x,\theta) = \sum_{n = 0}^\infty \phi(\theta / r_n) \cdot a_n(x,\theta), \]
    %
    which is smooth, since it is a locally finite sum. For any $N$, if we set
    %
    \[ R_N(x,\theta) = \sum_{n = N}^\infty \phi(\theta / r_n) \cdot a_n(x,\theta), \]
    %
    then
    %
    \[ a - \sum_{n = 0}^{N-1} a_n = \sum_{n = 0}^{N-1} (\phi(\theta/r_n) - 1) a_n(x,\theta) + R_N(x,\theta) \]
    %
    If $x \in \Omega$, we find that
    %
    \[ | \nabla_x^j \nabla_\lambda^k R_N(x,\theta) | \lesssim_{N,i,j} \langle \theta \rangle^{\max_{n \geq N} t_n + 1 - k}. \]
    %
    Thus $R_N \in \mathcal{S}^{\beta_N}(\Omega \times \RR^p)$, where $\beta_N = \max_{n \geq N} t_n + 1$. On the other hand,
    %
    \[ E_N(x,\theta) = \sum_{n = 0}^{N-1} (\phi(\theta/r_n) - 1) a_n(x,\theta) \]
    %
    vanishes for $|\theta| \geq r_n$, and is thus compactly supported in $\theta$, which implies that $E_N \in \mathcal{S}^{-\infty}(\Omega \times \RR^p)$.
\end{proof}

\begin{remark}
    A similar formula holds for local families of symbols.
\end{remark}

To verify asymptotic formulae, the following Lemma is often helpful.

\begin{lemma}
    Suppose $a \in C^\infty(\Omega \times \RR^p)$, and for any $n,m > 0$, there exists $t_{nm}$ such that
    %
    \[ |\nabla^n_x \nabla^m_\theta a(x,\theta)| \lesssim_{n,m} \langle \theta \rangle^{t_{nm}}. \]
    %
    If, for any $t \in \RR$,
    %
    \[ |a(x,\theta)| \lesssim_t \langle \theta \rangle^t, \]
    %
    then $a \in \mathcal{S}^{-\infty}(\Omega \times \RR^p)$.
\end{lemma}
\begin{proof}
    We begin by showing that if $f \in C^2(\RR)$, $\| f \|_{L^\infty(\RR)} \leq A$, and $\| f'' \|_{L^\infty(\RR)} \leq B$, then $\| f' \|_{L^\infty(\RR)} \leq \sqrt{2AB}$. this follows because for any $x$, and $\varepsilon > 0$, there exists $\theta_1$ lying between $x$ and $x - \varepsilon$ such that
    %
    \[ f(x) - f(x-\varepsilon) = \varepsilon f'(x) + \varepsilon^2 f''(\theta_1) / 2 \]
    %
    and $\theta_2$ lying between $x$ and $x + \varepsilon$ such that
    %
    \[ f(x + \varepsilon) - f(x) = \varepsilon f'(x) + \varepsilon^2 f''(\theta_2)/2. \]
    %
    Thus
    %
    \[ f(x+\varepsilon) - f(x-\varepsilon) = 2 \varepsilon f'(x) + \varepsilon^2 / 2 (f''(\theta_1) + f''(\theta_2)). \]
    %
    Rearranging gives
    %
    \[ f'(x) = (f(x+\varepsilon) - f(x-\varepsilon))/2 \varepsilon - (\varepsilon / 4)(f''(\theta_1) + f''(\theta_2)), \]
    %
    and thus
    %
    \[ |f'(x)| \leq A/\varepsilon + B \varepsilon / 2. \]
    %
    Taking $\varepsilon = \sqrt{2A/B}$ completes the proof.

    It follows from this that if $K$ and $K'$ are compact sets, with $K$ contained in the interior of $K'$, then
    %
    \[ \| \nabla_\theta \phi \|_{L^\infty(K)} \lesssim_K \sqrt{\| \phi \|_{L^\infty(K')} \| \nabla_\theta^2 \phi \|_{L^\infty(K'')} }. \]
    %
    The theorem then follows by successively differentiating in $\theta$.
\end{proof}

\begin{corollary}
    Suppose $\{ a_n \}$ are a family of symbols, with $a_n \in \mathcal{S}^{t_n}(\Omega \times \RR^p)$ for each $n$, and $\lim_{n \to \infty} t_n = -\infty$. Then if $a \in C^\infty(\Omega \times \RR^p)$, and for each $N$ and $M$, there exists $t_{NM}$ such that
    %
    \[ |\nabla^N_x \nabla^M_\theta a(x,\theta)| \lesssim \langle \theta \rangle^{t_{NM}}. \]
    %
    If for each $n$, there exists $\beta_n$ such that
    %
    \[ |a(x,\theta) - \sum_{k = 0}^n a_n(x,\theta)| \lesssim_n \langle \theta \rangle^{\beta_n}, \]
    %
    and $\lim_{n \to \infty} \beta_n = -\infty$, then $a \sim \sum a_n$.
\end{corollary}

Sometimes one has to use more powerful notions of homogeneity than the simple decay estimates above. In this case, it is useful to focus on \emph{classical symbols}, i.e. symbols which satisfy an asymptotic formula of the form
%
\[ a(x,\theta) \sim \sum_{n = 0}^\infty a_{t-n}(x,\theta), \]
%
where $a_{t-n} \in S^{\text{Re}(t) - n}(\Omega \times \RR^p)$ is homogeneous of degree $t-n$ in the $\theta$ variables for suitably large $\theta$ (the symbol must be smooth, and so cannot be homogeneous for small $\theta$ unless it is a polynomial). We denote the class of such symbols of order $t$ by $\mathcal{S}^t_{\text{cl}}(\Omega \times \RR^p)$. Sometimes this class is also called the class of \emph{polyhomogeneous symbols}, and denoted $\mathcal{S}^t_{\text{phg}(\Omega \times \RR^p)}$. We can also consider polyhomogeneous symbols with non integer step sizes, i.e. the class $\mathcal{S}^{t,h}_{\text{phg}}(\Omega \times \RR^p)$, i.e. those symbols that satisfy an asymptotic expansion of the form
%
\[ a(x,\theta) \sim \sum_{n = 0}^\infty a_{t - hn}(x,\theta) \]
%
where $a_{t-hn} \in S^{\text{Re}(t) - hn}(\Omega \times \RR^p)$ is homogeneous of degree $t-hn$.

\begin{remark}
    Let $M$ be a manifold, and let $E$ be a vector bundle over $M$. We can define the space $\loc{\mathcal{S}^t}(E)$ of symbols of order $t$ on $E$ to be the family of all scalar functions $a$ on $E$ which are symbols of order $t$ in local coordinates. These have a very similar theory of the theory we have expounded above. In particular, one can consider asymptotic developments of symbols.
\end{remark}












\chapter{Pseudodifferential Operators}

The goal of this chapter is to define the calculus of pseudodifferential operators, a general family of operators which allows us to manipulate the spatial and frequential properties of functions simultaneously. It is impossible to do this completely pointwise because of the uncertainty principle, but one can do things \emph{pseudolocally}, i.e. the position of the support in time and space is approximately preserved, up to a rapidly decaying error. Roughly speaking, we will define a family of operators $a(x,D)$, associated with functions $a(x,\xi)$, such that if the support of a function $f$ is concentrated near a point $x_0$, and the support of $\widehat{a}$ is concentrated near $\xi_0$, then we will find $a(x,D) f \approx a(x_0,\xi_0) f$. Before we begin, let us consider some basic examples that allow us to control space or time exclusively, to get an idea of what we want out of such a theory.

The most basic spatial multipliers in analysis are the \emph{position operators}, which are the family of operators $X^\alpha: \mathcal{S}(\RR^d) \to \mathcal{S}(\RR^d)$, defined by setting
%
\[ X^\alpha f(x) = x^\alpha f(x), \]
%
and the \emph{momentum operators} $D^\alpha: \mathcal{S}(\RR^d) \to \mathcal{S}(\RR^d)$, which is the Fourier multiplier
%
\[ \widehat{D^\alpha f}(\xi) = \xi^\alpha \widehat{f}(\xi). \]
%
Note that in this chapter, the operators $\{ D^\alpha \}$ will be normalized as such, and thus differ from the usual differential operators, which we will here denote by $\partial^\alpha$, by the constant $(2 \pi i)^{-|\alpha|}$. For each $m \in C^\infty(\RR^d)$ such that $m$ and all of it's derivatives are slowly increasing, we can define a bounded operator $m(X): \mathcal{S}(\RR^d) \to \mathcal{S}(\RR^d)$ by setting
%
\[ m(X) f(x) = m(x) f(x). \]
%
We can also define an operator $m(D): \mathcal{S}(\RR^d) \to \mathcal{S}(\RR^d)$ by setting
%
\[ \widehat{m(D) f}(\xi) = m(\xi) \widehat{f}(\xi). \]
%
Thus we have found two homomorphisms from a ring of smooth functions on $\RR^d$ to the ring of bounded operators on $\mathcal{S}(\RR^d)$.

The family of such operators is very useful in analysis, since families of functions are more amenable to intuition than families of operators, and so we can try and understand what features of the function $m$ tell us about the resulting operators $m(X)$ and $m(D)$. For instance, a very important application of operators of the form $m(D)$ is to the study of \emph{elliptic} differential operators with constant coefficients. Recall that a partial differential operator $L = \sum c_\alpha D^\alpha$ of degree $k$ is \emph{elliptic} if the homogeneous polynomial $\sum_{|\alpha| = k} c_\alpha \xi^\alpha$ is non-vanishing away from the origin.

\begin{theorem}
    If $L$ is an elliptic partial differential operator on $\RR^d$, with constant coefficients, then $L$ has a fundamental solution, i.e. there exists a distribution $\Phi \in \DD(\RR^d)^*$ such that $L(\Phi) = \delta$.
\end{theorem}
\begin{proof}
    Suppose $L$ has degree $k$, and write $L = \sum c_\alpha D^\alpha$ for some constants $\{ c_\alpha \}$. Since $L$ is elliptic, there exists $R > 0$ such that the polynomial $P(\xi) = \sum_\alpha c_\alpha \xi^\alpha$ satisfies $|P(\xi)| \sim |\xi|^k$ for $|\xi| \geq R$. If $\chi \in \DD(\RR^d)$ is chosen such that $\chi(\xi) = 1$ for $|\xi| \leq R$, and we define a distribution $\Phi_0$ such that
    %
    \[ \widehat{\Phi_0}(\xi) = \frac{(1 - \chi(\xi))}{P(\xi)}. \]
    %
    Then $\widehat{\Phi_0}$ is a smooth function with $|\widehat{\Phi_0}(\xi)| \lesssim \langle \xi \rangle^{-k}$ for all $\xi \in \RR^d$, which means that $\widehat{\Phi_0}$ is a well defined tempered distribution, and thus $\Phi_0$ is also a well defined tempered distribution. But then
    %
    \[ \widehat{L \Phi_0} = 1 - \chi(\xi). \]
    %
    Since $\chi \in \DD(\RR^d)$, it follows by the Paley-Wiener theorem, taking the inverse Fourier transform that $L \Phi_0 = \delta - w$, where $w$ is an entire analytic function of at most polynomial increase. The Cauchy-Kovalevskaya theorem (i.e. solving the equation by expanding out power series) allows us to find an entire analytic function $u$ of at most exponential increase such that $Lu = w$. Then $\Phi = \Phi_0 + u$ is a fundamental solution for $L$.
\end{proof}

The theory of pseudodifferential operators was introduced primarily to generalize these kinds of constructions to elliptic linear partial differential equations with \emph{non constant} coefficients. A \emph{(left) parametrix} for a linear differential operator $L$ with smooth coefficients on a domain $\Omega$ is an operator $S: \DD(\Omega) \to \DD(\Omega)^*$ such that $1 - S \circ L$ is a \emph{smoothing operator}. We think of $S$ as given an `approximate inverse' for the operator $T$. The existence of a regular parametrix for an elliptic linear differential operator, which will be justified by the theory of pseudodifferential operators, is quite important in the theory of differential equations. In particular, it proves that certain differential operators are \emph{hypoelliptic}, i.e. that if $u \in \DD(\Omega)$, then $\singsupp(Lu) = \singsupp(u)$; it suffices to show $\singsupp(u) \subset \singsupp(Lu)$, since the inclusion $\singsupp(Lu) \subset \singsupp(u)$ is true for any differential operator $L$ with smooth coefficients.

\begin{theorem}
    Let $L$ be a differential operator with smooth coefficients. If $L$ has a very regular left parametrix $S$, then $L$ is hypoelliptic.
\end{theorem}
\begin{proof}
    For any very regular operator $S$, $\singsupp(Su) \subset \singsupp(u)$. It suffices to prove that for any \emph{compactly supported} distribution $u$, we have $\singsupp(u) \subset \singsupp(Lu)$, since the general case follows by localization. Since $1 - S \circ L$ is a smoothing operator, we have
    %
    \begin{align*}
        \singsupp(u) &\subset \singsupp((1 - S \circ L) u) \cup \singsupp((S \circ L) u)\\
        &\subset \singsupp((S \circ L) u)\\
        &\subset \singsupp(Lu). \qedhere
    \end{align*}
\end{proof}

The family of pseudodifferential operators we will study are \emph{microlocal}, i.e. not only do they not expand the singular support of distributions, but they also do not expand the wavefront set of distributions. It will therefore follow from the theory that any elliptic differential operator has a pseudodifferential parametrix, and an analogous argument to that given above gives the strong equation $\text{WF}(Lu) = \text{WF}(u)$ for any distribution $u$.

Returning to the general question of constructing a functional calculus which includes both the position and momentum operators, we recall the \emph{spectral calculus}, whose goal, for a suitable algebra of normal operators $A$, is to produce an isomorphism of $A$ with an algebra of functions on some space $X$, called the \emph{spectrum} of $A$. A natural hope would be to find such a calculus for an algebra $A$ of operators which includes the position and momentum operators. This would, in particular, enable us to analyze linear differential operators with non-constant coefficients. However, we quickly see that such a theory would not quite work in as standard a way as the spectral calculus provides, because the families of operators $\{ X^\alpha \}$ and $\{ D^\alpha \}$ do \emph{not} commute with one another, i.e. the chain rule implies that
%
\[ [D^i,X^i] = D^i X^i - X^i D^i = 1. \]
%
The key thing we should notice from this equation, however, is that this equation indicates that position and momentum operators commute `up to lower order terms'. In other words, if we think of $X^\alpha$ and $D^\alpha$ as being operators of \emph{order $|\alpha|$}, then $[D^\alpha,X^\beta]$ is equal to zero, \emph{modulo terms of order $|\alpha| + |\beta| - 1$}. This fact will enable us to obtain an `approximate' functional calculus for the desired algebra of operators. This is precisely the \emph{calculus of pseudodifferential operators}.

We will associate, with each suitably regular function $a(x,\xi)$, an operator $a(x,D)$, which is a homomorphism 'modulo lower order terms'. This association will have the property that if $a(x,\xi) = \sum c_\alpha(x) \xi^\alpha$, then $a(x,D)$ will be the differential operator $\sum c_\alpha(x) D^\alpha$. Indeed, this is where the notation $a(x,D)$ comes from. The association will also generalize the two families of multiplier operators; if $a(x,\xi) = m(x)$, then $a(x,D)$ is equal to $m(X)$, and if $a(x,\xi) = m(\xi)$, then $a(x,D)$ is equal to $m(D)$. To get an idea for what this operator should look like, we calculate that if $a(x,\xi) = \sum_\alpha c_\alpha(x) \xi^\alpha$ is the symbol of a differential operator with nonconstant coefficients, then the corresponding differential operator satisfies
%
\begin{align*}
    a(x,D) f &= \sum c_\alpha(x) D^\alpha f(x)\\
    &= \int_{\RR^d} \sum_\alpha c_\alpha(x) \xi^\alpha \widehat{f}(\xi) e^{2 \pi i \xi \cdot x}\; d\xi\\
    &= \int_{\RR^d} a(x,\xi) e^{2 \pi i \xi \cdot x} \widehat{f}(\xi)\; d\xi.
\end{align*}
%
We use this integral formula to define $a(x,D)$ for a much more general family of functions $a(x,\xi)$.

Fix an open set $\Omega \subset \RR^d$. Given any distribution $a(x,\xi)$ which is tempered in the $\xi$ variable, i.e. any continuous, bilinear map $a: \DD(\Omega_x) \times \mathcal{S}(\RR^d_\xi) \to \CC$, or more technically, any element of $\DD(\Omega_x)^* \CT \SW(\RR^d_\xi)^*$, we can associate an operator $a(x,D): \DD(\Omega) \to \DD(\Omega)^*$, such that for any $f,g \in \DD(\Omega)$,
%
\[ \langle a(x,D) f, g \rangle = \int a(x,\xi) \widehat{f}(\xi) e^{2 \pi i \xi \cdot x} g(x)\; dx\; d\xi. \]
%
We call any operator $T$ which can be given in the form $a(x,D)$ a \emph{pseudodifferential operator}. The symbol $a$ is uniquely determined by the operator $T$, since the action of $a(x,D)$ on $\DD(\Omega)$ determines the behaviour of $a$, viewed as a bilinear map, on an arbitrary element of $\DD(\Omega_x) \times \SW(\RR^d_\xi)$. For any set $S \subset \DD(\Omega_x)^* \CT \SW(\RR^d_\xi)^*$, the notation $\text{Op}(S)$ is often used to denote the family of all pseudodifferential operators defined by an element of $S$.

\begin{example}
    Consider the Laplacian $\Delta$ on $\RR^n$. Then $\Delta$ is the pseudodifferential operator on $\RR^n$ of order two, corresponding to the symbol $a(x,\xi) = - 4\pi^2 |\xi|^2$. Since $\Delta$ is a constant coefficient operator, it just acts as a Fourier multiplier. If $\Delta u = f$, then, modulo a harmonic function, which is arbitrarily smooth, $u = \Phi * f$, where $\Phi$ is a fundamental solution to the Laplacian. The operator $f \mapsto \Phi * f$ is a pseudodifferential operator with symbol $b(x,\xi) = \text{f.p}(1/4\pi^2 |\xi|^2)$. 
\end{example}

\begin{example}
    The Cauchy-Riemann operator on $\RR^2$ given by
    %
    \[ \frac{\partial}{\partial \overline{z}} = \frac{1}{2} \left( \frac{\partial}{\partial x} + i \frac{\partial}{\partial y} \right) \]
    %
    is a pseudodifferential operator of order one with symbol $i \pi(\xi + i \eta)$.
\end{example}

Any pseudodifferential operator $T = a(x,D)$ is continuous from $\DD(\Omega)$ to $\DD(\Omega)^*$, and has Schwartz kernel
%
\[ K_a(x,y) = \int a(x,\xi) e^{2 \pi i \xi \cdot (x - y)}\; d\xi, \]
%
where in general the oscillatory integral must be interpreted formally. It is also useful to write this kernel in the convolution form $k_a(x,z) = K_a(x,x-z)$, because we then have
%
\[ T\phi(x) = \int k_a(x,z) f(x-z)\; dz, \]
%
which reflects the fact that when $a(x,\xi)$ is independant of $x$, $k_a$ is a function of $z$, and then $Tf = k_a * f$ is a convolution operator. In fact, all pseudodifferential operators are infinite sums of convolution operators, in the following sense: if $a \in \DD(\RR^d)^* \CT \SW(\RR^d)^*$, then we can write $a$ as a sum of the form
%
\[ \sum_{i = 1}^\infty u_i \otimes v_i, \]
%
where $\{ u_i \}$ are in $\DD(\RR^d)^*$, and $\{ v_i \} \in \SW(\RR^d)^*$, and the convergence occurs unconditionally, in the distributional topology. It then follows that for $\phi \in \DD(\Omega)$,
%
\[ a(x,D) \phi = \sum_{i = 1}^\infty u_i \cdot (v_i * \phi). \]
%
Thus $a(x,D)$ is, in some sense, a non-constant coefficient sum of convolution operators.

As we increase the regularity of $a$, we no longer need to treat the definition of a pseudodifferential operator quite as formally, and so we can define the operator on a more general family of functions. Here are some non-comprehensive examples of this phenomenon:
%
\begin{itemize}
    \item If $a \in \mathcal{S}(\RR^d \times \RR^d)^*$, then $a(x,D)$ extends to a continuous linear operator from $\mathcal{S}(\RR^d)$ to $\mathcal{S}(\RR^d)^*$. The Schwartz kernel theorem implies that any continuous linear operator from $\mathcal{S}(\RR^d)$ to $\mathcal{S}(\RR^d)^*$ is of this form, which probably indicates that the family of such operators is too general to obtain interesting results.

    \item If $a \in \loc{\mathcal{S}^t}(\Omega \times \RR^d)$, then we will see later on that $a(x,D)$ extends to a continuous operator from $\EC(\Omega)^*$ to $\DD(\Omega)^*$ and from $\DD(\Omega)$ to $\EC(\Omega)$.

    \item If $a \in \mathcal{S}^t(\RR^d \times \RR^d)$, then $a(x,D)$ extends to a continuous operator from $\SW(\RR^d)$ to $\SW(\RR^d)$.

    \item If $a \in \loc{\mathcal{S}^{-\infty}}(\Omega \times \RR^d)$, then we will see later in this chapter that $a(x,D)$ has a kernel in $C^\infty(\RR^d \times \RR^d)$, and is therefore a smoothing operator, thus extending to a continuous operator from $\EC(\RR^d)^*$ to $\EC(\RR^d)$.

    Conversely, we will also see that if $T$ is \emph{any} `pseudolocal' smoothing operator, in the sense that it has a kernel $K \in C^\infty(\Omega \times \Omega)$ satisfying bounds of the form
    %
    \[ |\nabla^n_x \nabla^m_y K(x,y)| \lesssim_N \langle x - y \rangle^{-N}, \]
    %
    then $T \in \text{Op}(\loc{\mathcal{S}^{-\infty}}(\Omega \times \RR^d))$. In particular, any proper smoothing operator is of this form.

    \item If $a(x,D)$ is a proper operator, then it maps $\DD(\Omega)$ into $\EC(\Omega)^*$ and from $\EC(\Omega)$ into $\DD(\Omega)^*$. In combination with the previous results, we conclude that if $a \in \loc{\mathcal{S}^t}(\Omega \times \RR^d)$, then $a(x,D)$ is an operator from $\DD(\Omega)^*$ to itself, $\EC(\Omega)^*$ into itself, and from $\DD(\Omega)$ into itself. If $a \in \loc{\mathcal{S}^{-\infty}}(\Omega \times \RR^d)$, then $a(x,D)$ maps $\DD(\Omega)$ to itself and maps $\DD(\Omega)^*$ into $\EC(\Omega)$.

    Conversely, let $T: \DD(\Omega) \to \EC(\Omega)^*$ be \emph{any} proper operator, and let $K$ be it's kernel. Then $K$ lies in $\DD(\Omega_x)^* \CT \EC(\Omega_y)^*$. Thus we can define a symbol $a \in \DD(\Omega_x)^* \CT \SW(\RR^d_\xi)^*$ by setting
    %
    \[ a(x,\xi) = \int K(x,y) e^{2 \pi i \xi \cdot (x - y)}\; dy. \]
    %
    We verify using the Fourier multiplication formula that
    %
    \[ T_a \phi(x) = \int a(x,\xi) \widehat{f}(\xi) e^{2 \pi i \xi \cdot x}\; d\xi = \int K(x,y) f(y)\ dy = T\phi(x). \]
    %
    Thus \emph{any} proper operator is a pseudodifferential operator.
\end{itemize}
%
That every proper operator, and that every operator on Schwartz space, is a pseudodifferential operator, indicates that the theory of pseudodifferential operators is too general to study more detailed in the form above. We will mostly focus on pseudodifferential operators defined by various symbol classes, since most practical operator occuring in PDE and analysis are of this form, and we can still get a sophisticated calculus.

\section{Basic Definitions}

There are two varieties of the theory of pseudodifferential operators, whose basic results are roughly analogous to one another. The first, which works best when considering pseudodifferential operators on $\RR^d$, works with operators specified by symbols $a: \RR^d \times \RR^d \to \CC$ in $\mathcal{S}^t(\RR^d \times \RR^d)$, i.e. satisfying estimates of the form
%
\[ |\nabla^n_x \nabla^m_\xi a(x,\xi)| \lesssim_{n,m} \langle \xi \rangle^{t-m} \]
%
where the bound holds uniformly in both $x$ and $\xi$, for any integers $n$ and $m$. As mentioned above, $a(x,D)$ then extends to a continuous operator from $\SW(\RR^d)$ to itself, which leads to an elegant theory. However, this theory is less easy to work with locally, e.g. working in various different coordinate systems, or obtaining a definition of pseudodifferential operator that applies to operators on manifolds. The approach here is best taken by describing a theory described by symbols $a \in \loc{\mathcal{S}^t}(\Omega \times \RR^d)$, i.e. satisfying an inequality of the form above uniformly in $\xi$, but only \emph{locally uniformly} in $x$.

Fix an open set $\Omega \subset \RR^d$, and consider a symbol $a \in \loc{\mathcal{S}^t}(\Omega \times \RR^d)$. From this symbol, we can define a continuous operator $T_a: \DD(\Omega) \to \EC(\Omega)$ by setting
%
\[ T_a f(x) = \int a(x,\xi) e^{2 \pi i \xi \cdot x} \widehat{f}(\xi)\; d\xi, \]
%
where the integral can now be interpreted in the usual Riemann / Lebesgue sense, and $T_a f$ is smooth since $a \in C^\infty(\Omega_x, \SW(\RR^d_\xi)^*)$. We then say $T_a$ is a \emph{pseudodifferential operator of order $t$}. The family of all such operators is denoted $\loc{\Psi^t}(\Omega)$; we reserve the notation $\Psi^t(\Omega)$ to denote those operators defined by elements of $\mathcal{S}^t(\Omega \times \RR^d)$.

The kernel of a pseudodifferential operator given by a symbol $a(x,\xi)$ of the class above is of the form
%
\[ K_a(x,y) = \int a(x,\xi) e^{2 \pi i \xi \cdot (x - y)}\; d\xi, \]
%
where the integral is now an oscillatory integral distribution. In particular, we know from our discussion of oscillatory integral distributions that
%
\[ \text{WF}(K_a) \subset \{ (x,x;\xi,\xi) : x, \xi \in \RR^d \}. \]
%
The microlocal analysis of distributions implies the existence of a continuous extension $T_a: \EC(\Omega) \to \DD(\Omega)^*$, and we find that $\text{WF}(T_au) \subset \text{WF}(u)$. Thus the operator $T_a$ preseres the location of singularities of a distribution in both position and frequency. This is the first instance of the \emph{microlocal nature} of pseudodifferential operators; these operators roughly preserve the location of the mass and frequency support of a function, but with some additional `fuzz' that is usually neglible to the problem, but must be managed.

The symbol $a$ is uniquely determined from the operator $T_a$. To actually recover the symbol from an operator, we have several methods. Formally, we can calculate that
%
\[ a(x,\xi) = e^{-2 \pi i \xi \cdot x} T_a(e^{2 \pi i \xi \cdot y}). \]
%
The wavefront calculation above shows that the convolution kernel $k_a(x,z)$ of a pseudodifferential operator agrees with a smooth function away from the line $z = 0$. We will see very shortly that it decays rapidly away from this line, and therefore $k_a$ is tempered in the $z$-variable. The above formal equation then implies the less formal equation
%
\[ a(x,\xi) = \int k_a(x,z) e^{2 \pi i \xi \cdot z}\; dy. \]
%
Thus the symbol $a$ is obtained by taking the Fourier transform of the convolution kernel $k_a$ in the $z$ variable.

Here is a quantitative estimate on the kernel of a pseudodifferential operator, which show another instance of it's pseudolocal nature, i.e. localization on the spatial side of things. In particular, the result implies that if $\Omega = \RR^d$, then $T_a$ extends to a continuous operator from $\SW(\RR^d)$ to itself. As mentioned above, $K_a \in \DD(\Omega \times \Omega)^*$ agrees with a $C^\infty$ function away from the diagonal $\Delta_\Omega$. Thus if $f \in \DD(\RR^d)$, and $x \not \in \text{supp}(f)$, then the multiplication formula for tempered distributions implies that
%
\[ T_af(x) = \int a(x,\xi) \widehat{f}(\xi) e^{2 \pi i \xi \cdot x}\; d\xi = \int K(x,y) f(y)\; dy, \]
%
where, since the integral on the right hand side vanishes in a neighborhood of $x$, we can actually interpret the right hand integral as a Lebesgue integral, rather than a formal integral. Moreover, we have even better estimates for the behaviour of $K_a$ away from the origin, which reflects the pseudolocal behaviour of the operator. To discuss these estimates, we introduce the differential operators $\partial^i_z = \partial^i_x - \partial^i_y$, and the induced operators $\nabla^m_z$, which measures the derivatives measured in the direction normal to the diagonal in $\Omega \times \Omega$.

\begin{theorem}
    Let $a \in \loc{\mathcal{S}^t}(\Omega)$. Then for any pair of integers $n,m \geq 0$, and any $N \geq 0$ such that $t + d + m + N \geq 0$,
    %
    \[ |\nabla^n_x \nabla^m_z K_a(x,y)| \lesssim_{n,m,N} |x - y|^{-t-d-m-N}, \]
    %
    where the implicit constant is locally uniform in $x$. If $a \in \mathcal{S}^t(\Omega)$, then we can choose the implicit constant to be uniform in $x$.
\end{theorem}
\begin{proof}
    If $\text{supp}_\xi(a)$ is compact, then $a \in \loc{\mathcal{S}^{-\infty}}(\Omega)$, and by the compactness, we conclude that for any $N \geq 0$,
    %
    \[ |\nabla^n_x \nabla^m_z K_a(x,y)| = |\mathcal{F}_\xi\{ a \} (x,x-y)| \lesssim_N 1 / |x-y|^N, \]
    %
    where the uniform estimate happens because we $a$ is compactly supported in the $\xi$ variable, uniformly in $x$, and smooth, locally uniformly in $x$. Thus, without loss of generality, in the remainder of the proof we may assume $a(x,\xi) = 0$ for $|\xi| \leq 1$. We can then perform a Littlewood-Paley decomposition, i.e. writing
    %
    \[ a(x,\xi) = \sum_{i = 0}^\infty a_i(x,y,\xi), \]
    %
    where $a_i(x,\xi) = \rho(\xi / 2^i) a(x,\xi)$ is supported on $|\xi| \sim 2^n$. Let $K_i$ be the kernel of the pseudodifferential operator $a_i(x,D)$. Then
    %
    \[ K(x,z) = \sum_{i = 0}^\infty K_i(x,z), \]
    %
    where the convergence is distributional. We claim that for any $i$, and any $n,m,N \geq 0$,
    %
    \[ |\nabla^n_x \nabla^m_z K_i(x,y)| \lesssim_{n,m,N} |x-y|^{-N} 2^{i(t + d + m - N)}, \]
    %
    where the implicit constant is locally uniform in $x$. This follows from a simple integration by parts, applied to the integral
    %
    \[ K_i(x,y) = \int \rho(\xi / 2^i) a(x,\xi) e^{2 \pi i \xi \cdot (x-y)}\; d\xi. \]
    %
    These bounds, if $N$ is taken large enough, imply that the sum $K = \sum K_i$ converges uniformly on any set of the form
    %
    \[ \{ (x,y) \in \Omega \times \Omega: x \in K, |y-x| > \varepsilon \}, \]
    %
    where $K \subset \Omega$ is compact. Summing up these bounds for sufficiently large $N$ gives the required inequality for $|x-y| \geq 1$. For $0 < |x-y| \leq 1$, we break the sum into two parts, i.e. writing
    %
    \[ K(x,y) = \sum_{2^i \leq 1/|x-y|} K_i(x,y) + \sum_{2^i > 1/|x-y|} K_i(x,y). \]
    %
    For the first sum, we take $N = 0$, and for the second sum, we take $N > t + d + m$, which gives the required bounds.
\end{proof}

These singularity conditions characterize those kernels $K_a$ induced by pseudodifferential operators of some order $t < 0$. For $t \geq 0$, we must also assume an additional cancellation condition.

\begin{lemma}
    Let $K \in \DD(\Omega \times \Omega)^*$ be a Schwartz kernel with
    %
    \[ \singsupp(K) \subset \{ (x,x): x \in \Omega \}. \]
    %
    Suppose that for some $t$, and any non-negative integers $n,m$, and $N$, the kernel $K$ satisfies the growth condition
    %
    \[ |\nabla^n_x \nabla^m_z K(x,y)| \lesssim |x-y|^{-t-d-m-N} \]
    %
    locally uniformly in $x$. Then:
    %
    \begin{itemize}
        \item If $t < 0$, and if, for each $x \in \Omega$, and any $\phi \in \DD(\Omega)$,
        %
        \[ \int K(x,y) \phi(y)\; dy = \lim_{\varepsilon \to 0} \int_{|x-y| > \varepsilon} K(x,y) \phi(y)\; dy, \]
        %
        where the right hand side exists, and defines a distribution induced by a locally integrable function on $\Omega$ by virtue of the growth condition with $m$ and $N$ equal to zero, then $K$ is the Schwartz kernel of a pseudodifferential operator given by a local symbol of order $t$.

        \item If $t \geq 0$, and any $\phi \in \DD(\Omega)$, the kernel $K$ satisfies the \emph{cancellation conditions}
        %
        \[ \left| \int D^\alpha_x K(x,y) \phi((x-y)/R)\; dy \right| \lesssim_{\phi,\alpha} R^t, \]
        %
        where the implicit constant is independent of $R$, locally uniform in $x$, and a continuous seminorm on $\DD(\Omega)$ for each multi-index $\alpha$, then $K$ is the kernel of some pseudodifferential operator given by a local symbol of order $t$.
    \end{itemize}
    %
    If we replace the inequalities that locally uniformly depend on $x$ with inequalities that are uniform in $x$, then the symbols we find can also be chosen to be uniform.
\end{lemma}
\begin{proof}
TODO: Ask Andreas about this.
\begin{comment}
    For $t < 0$, we make the family of all such kernels above into a Fr\'{e}chet space $X_t$ by taking the optimal implicit constants in the growth condition inequalities above as seminorms. For $t \geq 0$, we make $X_t$ into a locally convex space by taking those implicit constants as seminorms, as well as taking, for each bounded set $\mathcal{B}$, the implicit constants in the cancellation condition as a seminorm. For $t < 0$, the growth conditions on elements of $X_t$ imply we have a continuous inclusion $X_t \to L^\infty_x(\RR^d) L^1_z(\RR^d) \to \mathcal{S}(\RR^d \times \RR^d)^*$. For $t \geq 0$,

    Let $k \in X_t$ be a kernel, and let $a(x,\xi)$ be the Fourier transform of the kernel in the $z$-variable. If we split up $k = k_0 + k_\infty$ and thus write $a = a_0 + a_\infty$, where $k_0$ is supported on $|z| \leq 2$, and $k_\infty$ on $|z| \geq 1$, then we see that, because $k_0$ is compactly supported, $a_0$ is smooth, and because $k_\infty$ is rapidly decaying away from the origin, $a_\infty$ is smooth. Thus $a \in C^\infty(\RR^d \times \RR^d)$ for any $k \in X_t$.

    Now set $k_R(x,z) = k(x,z/R)$ for each $R > 0$. The family
    %
    \[ \{ R^{-t-d} k_R : R > 0 \} \]
    %
    is then a bounded set in $X_t$, since, for instance,
    %
    \[ \sup \{ |z|^{t+d+n_2+N} |\nabla_x^{n_1} \nabla_z^{n_2} k_R(x,z)| \} = R^{t + d + N} \sup \{ |z|^{t + d + n_2 + N} |\nabla_x^{n_1} \nabla_z^{n_2} k(x,z)| \} \]


    Suppose we can prove that $|a(x,\xi)| \leq C(a)$ for all $x \in \RR^d$, and $1/2 \leq |\xi| \leq 2$, where $a \mapsto C(a)$ is a continuous seminorm on $X$. If we set
    %
    \[ a_R(x,\xi) = \int k_R(x,z) e^{-2 \pi i \xi \cdot z}, \]
    %
    then we will then have actually proved that $|a_R(x,\xi)| \leq C(a) R^{t+d}$ for all $R > 0$ and $1/2 \leq |\xi| \leq 2$. Since $a_R(x,\xi) = R^d a(x,R \xi)$, this implies that
    %
    \[ |a(x,\xi)| \lesssim_t C(a) |\xi|^t \]
    %
    for all $\xi$. Since $k \mapsto D^\alpha_x D^\beta_z k$ is a continuous operator on $X_t$ to $X_{t - |\beta|}$, this means we will have actually proved that
    %
    \[ |\nabla^{n_1}_x \nabla^{n_2}_z a(x,\xi)| \lesssim_t C_{n_1,n_2}(a) |\xi|^{t-n_2}. \]
    %
    Thus we have proved that $a \in \mathcal{S}^t(\RR^d \times \RR^d)$. For $t < 0$, to show that the bounds on $1/2 \leq |\xi| \leq 2$ hold, we just note that we can take $C(a) = \| k \|_{L^\infty_x L^1_z}$, which is a continuous seminorm on $X_t$ because
    %
    \[ \int |k(x,z)|\; dz \leq \sup_{x \in \RR^d, |z| \leq 1} |k(x,z)| |z|^{t + d} + \sup_{x \in \RR^d, |z| \geq 1} |k(x,z)| |z|^{d+1}. \]
    %

    TODO: Ask Andreas about this.
\end{comment}
\end{proof}

In addition to studying the behaviour of $\Psi$DOs away from the diagonal, which reflects the pseudolocal behaviour of the distribution, it is also of interest to determine the behaviour of the operator under highly oscillatory, but non-stationary, phenomena, which is related to it's microlocal nature. Consider a symbol $a(x,\xi)$, a smooth function $f(y)$, and a smooth phase $\phi(y)$ with $\nabla \phi(y)$ nonvanishing on $\text{supp}_x(a)$. Our goal is to try to determine the asymptotic behaviour of the function $T_a(f e^{2 \pi i \lambda \phi})$ as $\lambda \to \infty$. Since $T_a$ is pseudolocal, the value at a point $x$ should be determined to a large degree by the behaviour of $f e^{2 \pi i \lambda \phi}$ near $x$, which, roughly speaking, oscillates near the frequency $\lambda \nabla \phi(x)$. Thus we might expect that
%
\[ T_a \{ f e^{2 \pi i \lambda \phi} \} (x) \approx a(x,\lambda \nabla \phi(x)) f(x) e^{2 \pi i \lambda \phi(x)}. \]
%
This is correct up to first order in $\lambda$, and in fact, we can obtain a complete asymptotic development as $\lambda \to \infty$. For simplicity, we assume $\text{supp}_x(a)$ is compact.

\begin{theorem}
    Fix a symbol $a \in \mathcal{S}^t(\Omega \times \RR^d)$, compactly supported in the $x$-variable, a smooth function $f \in \DD(\Omega)$, and a smooth, real-valued function $\phi \in C^\infty(\Omega)$ with $\nabla \phi$ nonvanishing on $\text{supp}_x(a)$. Let
    %
    \[ r_x(y) = \nabla \phi(x) \cdot (x - y) - (\phi(x) - \phi(y)). \]
    Then for any $N > 0$ and $\lambda > 0$, we can write $e^{-2 \pi i \lambda \phi(x)} T_a \{ f e^{2 \pi i \lambda \phi} \}(x)$ as
    %
    \begin{align*}
        \sum_{|\beta| < N} \frac{1}{\beta! \cdot (2 \pi i)^{\beta}} \cdot \partial_\xi^\beta a(x,\lambda \nabla \phi(x)) \left. \partial^\beta_y \{ e^{2 \pi i \lambda r_x} f \} \right|_{y = x} + R_N(x,\lambda),
    \end{align*}
    %
    where $\lambda^{t - \lceil N/2 \rceil} R_N \in L^\infty(\RR^d \times (0,\infty))$. In particular, for $N = 3$, we find that $e^{-2 \pi i \lambda \phi(x)} T_a \{ f e^{2 \pi i \lambda \phi} \}(x)$ is equal to
    %
    \begin{align*}
        & a(x, \lambda \nabla \phi(x)) f(x)\\
            &\quad\quad\quad\quad + \frac{1}{2\pi i} \sum_{k = 1}^d (\partial^k_\xi a)(x,\lambda \nabla \phi(x)) \cdot (\partial^k_x f)(x)\\
            &\quad\quad\quad\quad - \frac{i \lambda}{4 \pi} \sum_{|\beta| = 2} (\partial^\beta_\xi a)(x,\lambda \nabla \phi(x)) (\partial^\beta_x \phi)(x) f(x)\\
            &\quad\quad\quad\quad\quad\quad\quad + O(\lambda^{t - 2}).
    \end{align*}
    %
    If $\phi(x) = \xi \cdot x$ for some $\xi$, then we find
    %
    \[ T_a \{ f e^{2 \pi i \lambda \phi} \}(x) \sim \sum_{\beta} \frac{1}{\beta! (2 \pi i )^\beta} \partial_\xi^\beta a(x,\lambda \xi) \cdot \partial^\beta_x f(x). \]
\end{theorem}
\begin{proof}
    We write
    %  xi [x - y] + lambda [phi(y) - phi(x)]
    %  xi [x - y] + lambda [nabla phi(x) (y - x) + r_x(y)]
    % lambda [ (xi - nabla phi(x)) (x - y) + r_x(y)]
    \[ e^{-2 \pi i \lambda \phi(x)} T_a \{ f e^{2 \pi i \lambda \phi} \}(x) = \lambda^d \int a(x, \lambda \xi) f(y) e^{2 \pi i \lambda [ (\xi - \nabla \phi(x)) \cdot (x - y) + r_x(y) ]}\; dy\; d\xi. \]
    %
    This integral is oscillatory, with phase
    %
    \[ \Phi_x(\xi,y) = (\xi - \nabla \phi(x)) \cdot (x - y) + r_x(y). \]
    %
    Now $\nabla_\xi \Phi_x(\xi,y) = 0$ precisely when $y = x$, and $\nabla_y \Phi_x(\xi,x) = 0$ precisely when $\xi = \nabla \phi(x)$. If we consider a smooth cutoff $\psi \in C_c^\infty(\RR^d_\xi)$ supported on
    %
    \[ (1/2) \cdot \inf_{x \in \text{supp}_x(a)} |\nabla \phi(x)| \leq |\xi| \leq 2 \cdot \sup_{x \in \text{supp}_x(a)} |\nabla \phi(x)| \]
    %
    then we can write $e^{-2 \pi i \lambda \phi(x)} T_a \{ f e^{2 \pi i \lambda \phi} \}(x) = I_1 + I_2$, where $I_1$ is obtained by substituting $\psi$ into the integrand, and $I_2$ is obtained by substituting $1 - \psi$ into the integrand. Nonstationary phase tells us that
    %
    \[ |I_2| \lesssim_N \lambda^{-N} \]
    %
    for all $N > 0$, where the implicit constant is independent of $\lambda$. Thus it suffices to concentrate on $I_1$. In the sequel, we will therefore assume without loss of generality that $a(x, \lambda \xi) = a(x, \lambda \xi) \psi(\xi)$, i.e. $a(x, \lambda \xi)$ is supported near $|\xi| \sim 1$. A change of variables $\xi \mapsto \xi + \nabla \phi(x)$ allows us to rewrite $I_1$ as
    %
    \[ \lambda^d \int e^{2 \pi i \lambda ( \xi \cdot (x - y) + r_x(y) )} a(x, \lambda [\nabla \phi(x) + \xi]) f(y)\; d\xi\; dy. \]
    %
    Using Taylors formula, we write
    %
    \[ a(x, \lambda \nabla \phi(x) + \lambda \xi) = \sum_{|\beta| < N} \frac{\lambda^\beta}{\beta !} \partial_\xi^\beta a(x, \lambda \nabla \phi(x)) \xi^\beta + R_{N,\lambda}(x,\xi), \]
    %
    where
    % f(xi) = a(x, lambda nabla phi(x) + lambda xi)
    \[ R_{N,\lambda}(x,\xi) = \sum_{|\beta| = N} \xi^\beta \frac{N}{\beta!} \int_0^1 (1 - t)^{N-1} \lambda^N \partial^\beta_\xi a(x, \lambda \nabla \phi(x) + t \lambda \xi)\; dt. \]
    %
    We only care about this formula when $|\xi| \sim 1$, and differentiating this formula gives that, for such $\xi$,
    %
    \[ |\partial_\xi^\alpha R_{N,\lambda}(x,\xi)| \lesssim_\alpha \lambda^{t - |\alpha|}, \]
    %
    and $\partial_\xi^\alpha R_{N,\lambda}(x,0) = 0$ for all $|\alpha| < N$. Stationary phase thus implies that
    %
    \[ \left| \lambda^d \int \int e^{2 \pi i \lambda ((x - y) \cdot \xi + r_x(y))} R_{N,\lambda}(x,\xi) f(y)\; d\xi\; dy \right| \lesssim \lambda^{t - \lceil N/2 \rceil}. \]
    %
    Finally, we note that via an integration by parts,
    %
    \begin{align*}
        \int & e^{2 \pi i \lambda (\xi \cdot (x - y) + r_x(y))} \xi^\beta f(y)\; d\xi\; dy\\
        &= \lambda^{-(d + \beta)} (2 \pi i)^{-\beta} \left. \partial_y^\beta \{ e^{2 \pi i \lambda r_x(y)} f(y) \} \right|_{y = x}
    \end{align*}
    % 
    and substituting them into the formula completes the proof.
\end{proof}

\begin{remark}
    If $T$ is a pseudodifferential operator on $\Omega \subset \RR^d$ defined by a symbol $a$ with $\text{supp}_x(a)$ compact, and $\kappa: \Psi \to \Omega$ is a diffeomorphism, then we can consider the operator $S: \DD(\Psi) \to \DD(\Psi)^*$ given by `changing the coordinates of $T$', i.e.
    %
    \[ S\phi(x) = T(\phi \circ \kappa^{-1})(\kappa(x)). \]
    %
    We can thus write, after a change of coordinates, with $\tilde{a}(x,\xi) = a(\kappa(x),\xi)$,
    %
    \begin{align*}
        S\phi(x) &= \int \tilde{a}(x,\xi) e^{2 \pi i \xi \cdot (\kappa(x) - \kappa(y))} |\det(\kappa(y))| \phi(y)\; dy\; d\xi.
    \end{align*}
    %
    This leads to an analysis of quantities similar to that obtained in the above proof, which will lead us to conclude that $S$ itself is a pseudodifferential operator. Note, however, that purely from the spectral anlaysis of singularities, we see immediately from the integral kernel that $S$ is a microlocal operator, i.e. it preserves the wavefront set of distributional inputs.

    This calculation will also prove handy when understanding the composition of a pseudodifferential operator with a \emph{Fourier integral operator}.
\end{remark}

As the order of the symbol $a$ decreases, we expect the behaviour of the corresponding pseudodifferential operator to become more and more regular. In particular, for $t < - d$, the symbol is actually \emph{integrable} in $\xi$. A $\Psi$DO of order $-\infty$ has a kernel $K$ lying in $C^\infty(\Omega \times \Omega)$, and satisfying estimates of the form
%
\[ | \nabla^n_x \nabla^m_z K(x,y)| \lesssim_{n,m,N} \langle x-y \rangle^{-N} \]
%
for any $N \geq 0$. Thus a $\Psi$DO of order $-\infty$ is smoothing. Conversely, if $K \in C^\infty(\Omega \times \Omega)$ satisfies estimates of the form above, then it is the $\Psi$DO corresponding to the symbol
%
\[ a(x,\xi) = \int K(x,y) e^{2 \pi i \xi \cdot (x-y)}\; dy, \]
%
which is a symbol of order $-\infty$. In particular, all properly supported smoothing operators are $\Psi$DOs of order $-\infty$.

Much of the theory of pseudodifferential operators to come is most elegantly explained \emph{modulo smoothing operators}. Working modulo smoothing operators is usually fine in harmonic analysis provided that we are trying to establish localized estimates for certain objects, since smoothing operators, once localized, satisfy all the inequalities we might be interested in. In particular, working modulo smoothing operators often makes the theory more flexible. For instance, one interesting operator to study is the square root of the Laplacian, which is the pseudodifferential operator with symbol $a(x,\xi) = 2 \pi |\xi|$. The fact that $a$ is not smooth near the origin means that $a(x,D)$ does not quite fit the theory of pseudodifferential operators given above. Nonetheless, for any bump function $\rho \in \DD(\RR^d_\xi)$ equal to one in a neighborhood of the origin, the symbol $\tilde{a}(x,\xi) = 2 \pi |\xi| (1 - \rho(\xi))$ lies in $\mathcal{S}^1(\RR^d \times \RR^d)$. Since $\text{supp}_\xi(a - \tilde{a})$ is compact, it follows that $a(x,D) - \tilde{a}(x,D)$ is a smoothing operator. Thus, modulo smoothing operators, it makes sense to identify $a$ with a symbol of order one. Thus we define the symbol class $\dot{\mathcal{S}}^t(\Omega \times \RR^d)$ to be the space of all distributional symbols which differ from an element of $\loc{S}^t(\Omega \times \RR^d)$ by a distributional symbol which induces a smoothing operator. Then $a \in \dot{\mathcal{S}}^1$.

One can already see from this example, that this theory works well with pseudodifferential operators defined in terms of homogeneous functions, of which the square root of the Laplacian is a special example, since homogeneous functions often fail to be smooth at the origin. Another case where this works well is that is only suffices to specify an element of $\dot{\mathcal{S}}^t$ by a symbol $a(x,\xi)$ only defined for suitably large $\xi$, since any two extensions of the symbols near the origin will differ by a symbol compactly supported in the $\xi$ variable, and thus smoothing.

Another virtue of working modulo smoothing operators is that asymptotics can be used to specify symbols. In other words, for any family of symbols $a_k \in \loc{\mathcal{S}^{t_k}}(\Omega \times \RR^d)$, provided that $t_k \to -\infty$ and with $t = \max(t_k)$, we can define a symbol $a \in \dot{\mathcal{S}}^t(\Omega \times \RR^d)$ by an asymptotic formula of the form
%
\[ a(x,\xi) \sim \sum_{k = 0}^\infty a_k(x,\xi), \]
%
since the difference of any two symbols satisfying this asymptotic formula lies in $\loc{\mathcal{S}^{-\infty}}(\Omega \times \RR^d)$, and thus induces a smoothing operator.

\section{Compound Operators and Quantization}

It is natural to wish to study a more general family of operators with a \emph{compound symbol} of the form $a(x,y,\xi)$, i.e. an operator of the form
%
\[ T_a f(x) = \int a(x,y,\xi) e^{2 \pi i \xi \cdot (x - y)} f(y)\; d\xi\; dy. \]
%
However, any such operator is already a pseudodifferential operator, and we can calculate an explicit asymptotic expansion for the symbol of this operator.

\begin{lemma}
    For any $a \in \loc{\mathcal{S}^t}(\Omega_x \times \Omega_y \times \RR^d)$, the operator $T_a$ lies in $\loc{\Psi^t}(\Omega)$, and the symbol of $T_a$ has the asymptotic expansion
    %
    \[ \sum_\beta \frac{1}{\beta!} \frac{1}{(2 \pi i)^{|\beta|}} \partial^\beta_\xi \partial^\beta_y a(x,x,\xi). \]
\end{lemma}
\begin{proof}
    We begin by noting that, by very similar techniques to those above, the kernel $K$ of an operator defined by a compound symbol of order $-\infty$ is smooth, and satisfies estimates of the form
    %
    \[ |\nabla^n_x \nabla^m_z K(x,y)| \lesssim_{n,m,N} \langle x - y \rangle^{-N} \]
    %
    locally uniformly in $x$ and $y$, which means that $K$ is the kernel of an operator in $\loc{\Psi^{-\infty}}(\Omega)$. Thus we obtain the result for $t = -\infty$. In general, we perform a Taylor expansion, writing
    %
    \[ a(x,y,\xi) = \sum_{|\beta| \leq N} \frac{1}{\beta !} \partial^\beta_y a(x,x,\xi) \cdot (y - x)^\beta + R_N(x,y,\xi), \]
    %
    where $\partial^\beta_y R_N(x,x,\xi) = 0$ for all $|\beta| \leq N$. We can find $C^\infty$ functions $b_\beta(x,y,\xi)$, for $|\beta| = N+1$, such that
    %
    \[ R_N(x,y,\xi) = \sum_{|\beta| = N+1} (2\pi i)^{N+1} (y - x)^\beta b_\beta(x,y,\xi). \]
    %
    Now integration by parts shows that
    %
    \begin{align*}
        \int & R_N(x,y,\xi) e^{2 \pi i \xi \cdot (x - y)}\; d\xi\\
        &= (-1)^{N+1} \sum_{|\alpha| = N+1} \int (D_\xi^\alpha b_\alpha)(x,y,\xi) e^{2 \pi i \xi \cdot (x - y)}\; d\xi.
    \end{align*}
    %
    The functions $b_\beta$ are symbols of order $t$, so $D_\xi^\alpha b_\beta$ are symbols of order $t - (N+1)$. Thus the operators specified by the compound symbol $R_N$ have order at most $t - (N+1)$. On the other hand, another integration by parts again shows that
    %
    \begin{align*}
        \int &\partial^\beta_y a(x,x,\xi) \cdot (y - x)^\beta \cdot e^{2 \pi i \xi \cdot (x - y)}\; d\xi\\
        &= \frac{1}{(2 \pi i)^{|\beta|}} \int \partial^\beta_\xi \partial^\beta_y a(x,x,\xi) e^{2 \pi i \xi \cdot (x - y)}\; d\xi.
    \end{align*}
    %
    Thus the operator corresponding with symbol $\partial^\beta_y a(x,x,\xi) \cdot (y - x)^\beta$ also corresponds to the symbol $1 / (2 \pi i)^{|\beta|} \cdot \partial^\beta_\xi \partial^\beta_y a(x,x,\xi)$. Thus if we consider a symbol $\tilde{a}(x,\xi)$ satisfying the asymptotic equation defined in the theorem, then we see that $\tilde{a}(x,D)$ differs from $T_a$ by a compound symbol of order $-\infty$, which, as previously discussed, is an element of $\loc{\Psi^{-\infty}}(\Omega)$.
\end{proof}

\begin{remark}
    If $b(x,\xi) = a(x,x,\xi)$, then the formula above can be written more formally as $a(x,y,\xi) \sim e^{2 \pi i D_x \cdot D_\xi} b$. This makes sense, since if $K$ is the kernel of $T_a$, and $T_a$ corresponds to a pseudodifferential operator, then it's symbol would correspond precisely to
    %
    \begin{align*}
        a(x,\xi) &= \int K(x, x - y) e^{-2 \pi i \xi \cdot y}\; dy\\
        &= \int \int a(x,x-y,\xi-\eta) e^{- 2 \pi i \eta \cdot y}\; d\eta\; dy.
    \end{align*}
    %
    If we define $a_x(y,\xi) = a(x,y,\xi)$, and $c(x,\xi) = e^{-2 \pi i \xi \cdot x}$, then $a(x,\xi) = (a_x * c)(x,\xi)$. But $c$ is a Gaussian, and thus if $(\xi', x')$ are the dual variables to $(x,\xi)$, then $\widehat{c}(\xi', x') = e^{2 \pi i \xi' \cdot x'}$. Thus
    %
    \[ a(x,\xi) = e^{2 \pi i D_x \cdot D_\xi} a_x(x,\xi), \]
    %
    where $e^{2 \pi i D_x \cdot D_\xi}$ is the Fourier multiplier operator with symbol $e^{2 \pi i \xi' \cdot x'}$. Taking the power series expansion of the exponential gives the expansion above.
\end{remark}

For any pseudodifferential operator $T$ with kernel $K_a(x,y)$, we can consider it's formal adjoint $T^*$ with kernel $K(x,y) = \overline{K_a(y,x)}$. We now use the calculation above to show the adjoint of a pseudodifferential operator $a(x,\xi)$; it is simple to calculate that the adjoint of any $\Psi DO$ $a(x,D)$ is a pseudodifferential operator with compound symbol $(x,y,\xi) \mapsto \overline{a(y,\xi)}$. Nonetheless, the above theorem implies that the adjoint can be given by a symbol $a^*(x,\xi)$ where $a^*(x,\xi) = e^{2 \pi i D_x \cdot D_\xi} \overline{a}(x,\xi)$, which we can write explicitly as an asymptotic expansion as
%
\[ a^*(x,\xi) \sim \sum_\beta \frac{1}{\beta!} \frac{1}{(2 \pi i)^{|\beta|}} \overline{\partial^\beta_\xi \partial^\beta_x a(x,\xi)}. \]
%
In particular, if $a$ is a symbol of order $t$, then $a^*(x,\xi) - \overline{a(x,\xi)}$ is a symbol of order $t - 1$, which we might write as saying that $a^* \approx \overline{a}$, up to lower order terms. In particular, if $a$ is a symbol correspond to a \emph{self adjoint} pseudodifferential operator, then $a \approx \text{Re}(a)$, up to lower order terms.

The choice of $(x,\xi)$ variables is common, but certainly not standard. The association of the pseudodifferential operator $a(x,D)$ with any symbol $a(x,\xi)$ is called the \emph{Kohn-Nirenberg quantization}. We could also use the \emph{adjoint Kohn-Nirenberg quantization} to associate an operator with every symbol $a$ in two variables, using the $(y,\xi)$ variables instead of the $(x,\xi)$ variables. We find, using the expansion above, that modulo smoothing operators, any symbol in the $(y,\xi)$ variables can be written in the $(x,\xi)$ variables, and moreover,
%
\[ a(y,\xi) \sim \sum_\beta \frac{1}{\beta!} \frac{1}{(2 \pi i)^{|\beta|}} \partial^\beta_\xi \partial^\beta_x a(x,\xi). \]
%
In particular, the symbol $(x,y,\xi) \mapsto a(y,\xi) - a(x,\xi)$ is a pseudodifferential operator of order $t - 1$. Thus the two quantizations describe exactly the same family of operators, and the association of an operator with a symbol only matters up to lower order terms.

The family of operators one can describe via the adjoint Kohn-Nireberg quantization is the same as the Kohn-Niernberg quantization. Thus, in the sequel, there is no harm in sticking with the Kohn-Nirenberg quantization. On the other hand, the symbols representing various operators change. For instance, we previously found that under the Kohn-Nirenberg quantization, the symbol $a(x,\xi) = \sum c_\alpha(x) \xi^\alpha$ corresponded to the differential operator $Lf = \sum c_\alpha D^\alpha f$. Under the adjoint Kohn-Nirenberg quantization, the symbol $a(y,\xi) = \sum c_\alpha(y) \xi^\alpha$ corresponds to the differential operator $Lf = \sum D^\alpha( c_\alpha f)$. If $t$ is the order of these operators, then the difference of these operators is a differential operator of order $t-1$, which reflects the equivalence described above.

Thus we see that, roughly speaking, the operators differ in the order in which they apply spatial and frequency modulation. It is sometimes useful to deal with a quantization that does both in a `symmetric' manner. To do this, we introduce the \emph{Weyl quantization}, which associates with each symbol $a(x,\xi)$ gives the Pseudodifferential operator $T$ with compound symbol $(x,y,\xi) \mapsto a((x + y)/2, \xi)$. This approach has the advantage that $T$ will be self-adjoint if and only if $a$ is real-valued, rather than just this only being true up to lower order terms. The Weyl quantization is the approach that works best in a generalization of a functional calculus for any finite family of noncommuting operators (there are notes by Tao which describes this process in detail, but it is beyond the scope of these notes).

\begin{comment}
\begin{remark}
    Here, we have worked with symbols satisfying uniform estimates in $x$. But often one can only work with symbols which \emph{locally} satisfy these estimates in $x$, i.e. working in the symbol classes $\loc{\mathcal{S}^t}(\RR^d \times \RR^d)$. The kernels of operators formed from these symbols satisfy bounds of the form
    %
    \[ | \nabla^{n_1}_x \nabla^{n_2}_z K(x,y)| \lesssim_{n_1,n_2,N} \frac{1}{|x-y|^{t + d + n_2 + N}}, \]
    %
    where the implicit constant is \emph{locally uniform} in $x$, and uniform in $y$. On a related note, such operators can be applied to any compactly supported distribution, and satisfy the microlocalization statement $\text{WF}(Tu) \subset \text{WF}(u)$. On the other hand, unless one has a bound such as
    %
    \[ |\nabla_x^{n_1} \nabla_y^{n_2} \nabla_\xi^m a(x,\xi)| \lesssim_{n,m} (\langle x \rangle^{k_{1n}} + \langle y \rangle^{k_{2n}}) \cdot \langle \xi \rangle^{k_{nm}}, \]
    %
    for all $n$ and $m$, it is not necessarily possible to apply the operator to Schwartz functions, and tempered distributions. One can consider asymptotics, as long as we work modulo a weaker family of smoothing operators, i.e. those whose kernels lie in $\EC(\RR^d \times \RR^d)$.
\end{remark}
\end{comment}

\section{Compositions of $\Psi$DOs}

Let $a(x,D)$ and $b(x,D)$ be pseudodifferential operators defined by local symbols of order $t$ and $s$. It is not always possible to define the composition $a(x,D) \circ b(x,D)$. This is because the image of an element of $\DD(\Omega)$ under the operator $b(x,D)$ in general lies in $\EC(\Omega)$, and one cannot necessarily apply $a(x,D)$ to elements of $\EC(\Omega)$. If we were working on $\RR^d$ and working with uniform symbols this wouldn't be a problem, since if $a \in \mathcal{S}^t(\RR^d \times \RR^d)$ and $b \in \mathcal{S}^t(\RR^d \times \RR^d)$, then $a(x,D)$ and $b(x,D)$ are both continuous operators from $\SW(\RR^d)$ to itself, and so the composition is well defined. In our approach, we must make a slightly technical assumption: we assume that either $a(x,D)$ or $b(x,D)$ maps $\DD(\RR^d)$ to itself, which is true in particular if $a(x,D)$ or $b(x,D)$ is a \emph{proper} pseudodifferential operator.

Regardless of which method we use, the composition of a $\Psi$DO of order $t$ and a $\Psi$DO of order $s$ will then be a $\Psi$DO of order $t + s$, and we have an asymptotic formula for the symbol of such an expansion, reflecting the lack of commutivity between the spatial and frequential variables. In particular, the symbol of the composition is, to first order, the product of the symbols of the two operators.

\begin{theorem}
    Let $a(x,\xi)$ and $b(x,\xi)$ be symbols of order $t$ and $s$, corresponding to operators $T_a$ and $T_b$. Then $T_a \circ T_b$ is a $\Psi$DO of order $t + s$, and has symbol
    %
    \[ (a \circ b)(x, \xi) = \left. e^{2 \pi i D_\xi \cdot D_y} \{ a(x,\xi) b(y,\eta) \} \right|_{y = x, \eta = \xi}. \]
    %
    In particular, we have the asymptotic expansion
    %
    \[ (a \circ b)(x,\xi) \sim \sum_\alpha \frac{1}{\alpha!} \frac{1}{(2 \pi i)^{|\alpha|}} \partial^\alpha_\xi a(x,\xi) \cdot \partial^\alpha_x b(x,\xi). \]
    %
    Thus $(a \circ b)(x,\xi) - a(x,\xi) b(x,\xi)$ is a symbol of order $t + s - 1$.
\end{theorem}

\begin{remark}
    If we consider the two form
    %
    \[ \omega = dx \wedge d \xi - d\xi \wedge dx \]
    %
    on $T^* \RR^n$, then we can define the \emph{Poisson bracket} of two functions on $T^* \RR^n$ by setting
    %
    \[ \{ a, b \} = \omega ( \nabla a, \nabla b ) = \sum_{i = 1}^d \frac{\partial a}{\partial \xi^i} \frac{\partial b}{\partial x^i} - \frac{\partial a}{\partial x^i} \frac{\partial b}{\partial \xi^i}. \]
    %
    For any two symbols $a$ and $b$, the result above implies that the commutator $[a(x,D), b(x,D)]$ of the two operators is a pseudodifferential operator of order $t + s$ with some symbol $[a,b]$, such that the symbol $[a,b] - (4 \pi i)^{-1} \{ a, b \}$ has order $t + s - 2$.
\end{remark}

\begin{proof}
    We can write
    %
    \begin{align*}
        (T_a \circ T_b) f(x) &= \int a(x,\eta) e^{2 \pi i \eta \cdot (x - z)} T_b f(z)\; dz\; d\eta\\
        &= \int a(x,\eta) b(z,\xi) e^{2 \pi i (\eta - \xi) \cdot (x - z)} e^{2 \pi i \xi \cdot (x - y)} f(y)\; dy\; dz\; d\xi\; d\eta.
    \end{align*}
    %
    Thus we see that we can view the composition as a $\Psi$DO with kernel
    %
    \[ c(x,\xi) = \int \int a(x,\eta) b(z,\xi) e^{2 \pi i (\eta - \xi) \cdot (x - z)}\; d\eta\; dz. \]
    %
    This is an oscillatory integral, with stationary point when $z = x$ and $\eta = \xi$. Thus we expand power series near this point, i.e. writing
    %
    \[ a(x,\eta) = \sum_\alpha \frac{1}{\alpha!} \partial^\alpha_\xi a(x,\xi) (\eta - \xi)^\alpha \]
    %
    and
    %
    \[ b(z,\xi) = \sum_\beta \frac{1}{\beta!} \partial^\beta_x b(x,\xi) (z - x)^\beta. \]
    %
    Using the Fourier inversion formula, we calculate that
    %
    \begin{align*}
        \int &(\eta - \xi)^\alpha (z - x)^\beta e^{2 \pi i (\eta - \xi) \cdot (x - z)}\; d\eta\; dz\\
        &= \int \tau^\alpha y^\beta e^{-2 \pi i \tau \cdot y}\; d\tau\; dy\\
        &= \begin{cases} 0 & \alpha \neq \beta, \\ \alpha! / (2 \pi i)^\alpha & \alpha = \beta. \end{cases}
    \end{align*}
    %
    Working like in our analysis of compound symbols, it suffices to show that if $g_1$ and $g_2$ are symbols of order $t$ and $s$, then
    %
    \[ f(x,\xi) = \int \int (\eta - \xi)^\alpha (z - x)^\beta g_1(x,\eta) g_2(z,\xi) e^{2 \pi i (\eta - \xi) \cdot (x - z)}\; d\eta\; dz \]
    %
    is a symbol of order $t + s - M - 1$. Applying sufficiently many integration by parts, it actually suffices to show integrals of the form
    %
    \[ f(x,\xi) = \int \int g_1(x,\eta) g_2(z,\xi) e^{2 \pi i (\eta - \xi) \cdot (x - z)}\; d\eta\; dz, \]
    %
    have order $t + s$, where $g_1$ has order $t$, and $g_2$ has order $s$. We write $\lambda = |\xi|$, and $\xi = \lambda \tilde{\xi}$, and write
    %
    \[ f(x,\xi) = \lambda^d \int \int g_1(x, \lambda \eta) g_2(z, \xi) e^{2 \pi i \lambda (\eta - \tilde{\xi}) \cdot (x - z)}\; d\eta \]
    %
    We can decompose the domain dyadically. For $|\eta| \leq 1/2$ and $|x - z| \leq 1$, an integration by parts in $z$ gives rapid decay in $t$. Similarily, we can dyadically sum over the regions where $|\eta| \leq 1/2$ and $|x - z| \sim 2^k$ by first integrating in $\eta$ using integration by parts, then integration in parts in $z$. This also gives rapid decay in $t$. Similar arguments give rapid decay in $\xi$ for $|\eta| \sim 2^l$, in fact giving estimates which are summable in $l$. Thus we are left with giving decay for an integral of the form
    %
    \[ t^d \int \int g_1(x, t \eta) g_2(z,\xi) \rho(|x - z|) \rho(|\eta| - 1) e^{2 \pi i t (\eta - \tilde{\xi}) \cdot (x - z)}\; d\eta\; dz. \]
    %
    This domain has a stationary point when $\eta = \xi$ and $z = x$. However, the stationary point is nondegenerate. Thus the integral is $O(\lambda^{t + s} \lambda^{-d})$ and so $|f(x,\xi)| \lesssim \langle \xi \rangle^{t + s}$. Replacing $g_1$ and $g_2$ with appropriate derivatives gives a full argument that $f$ is a symbol of order $t + s$.
\end{proof}

\begin{remark}
    This result shows that $\Phi = \bigcup_t \Phi^t$ and the subfamily $\Phi_{\text{loc},\text{prop}}$ of proper pseudodifferential operators in $\Phi_{\text{loc}} = \bigcup_t \Phi^t_{\text{loc}}$ form graded algebras. We note for any pseudodifferential operator $T \in \Phi^t_{\text{loc}}$, there is a properly supported pseudodifferential operator $\tilde{T} \in \Phi_{\text{loc},\text{prop}}^t$ such that $T - \tilde{T}$ is a smoothing operator. If we define $\dot{\Phi}^t$ to be the space of all pseudodifferential operators which differ from an element of $\loc{\Phi^t}$ by a pseudodifferential smoothing operator, modulo the set of smoothing operators, then $\dot{\Phi} = \bigcup_t \dot{\Phi}^t$ is a graded algebra under addition and composition, which induces, by bijection, an alternate graded algebra structure on $\dot{\mathcal{S}}$, agreeing up to first order with the standard algebra structure on this space given by multiplication.
\end{remark}

\section{Parametrices for Elliptic Operators}

A \emph{pseudodifferential parametrix} for a pseudodifferential operator $T$ is a pseudodifferential operator $S$ such that $S \circ T$ and $T \circ S$ are the identity operator, modulo smoothing operators. One useful result of our calculations is that we can easily construct \emph{parametrices} for suitable pseudodifferential operators. A symbol $a \in \dot{\mathcal{S}}^t(\Omega \times \RR^d)$ is called \emph{elliptic} if, locally in the $x$ variable, we can find $R > 0$ such that for $|\xi| \geq R$,
%
\[ |a(x,\xi)| \sim \langle \xi \rangle^t, \]
%
where the implicit constant is also locally uniform. In this case, we can interpret $b_0(x,\xi) = 1 / a(x,\xi)$ as an element of $\dot{\mathcal{S}}^{-t}$, since the reciprocal is well defined for large $\xi$, and satisfies the required symbol estimates. By the composition calculus, $1 - (a \circ b_0)(x,\xi)$ is a symbol $r_1 \in \dot{\mathcal{S}}^{-1}$. For $i \geq 1$, given $r_i \in \dot{\mathcal{S}}^{-i}$, if we define $b_i = r_i / a$ in $\dot{\mathcal{S}}^{t-i}$, then the composition calculus tells us that $r_{i+1} = 1 - (a \circ (b_0 + \dots + b_i))$ lies in $\dot{\mathcal{S}}^{-i-1}$, and we can continue the iteration. If we consider $b \in \dot{\mathcal{S}}^{-t}$ defined by the asymptotic development
%
\[ b \sim \sum_{i = 0}^\infty b_i \]
%
then $b(x,D)$ is a right pseudodifferential parametrix for $a(x,D)$. Similarily, we can construct $c \in \dot{\mathcal{S}}^{-t}$ such that $c(x,D)$ is a left pseudodifferential parametrix for $a(x,D)$. But this means that, in $\dot{\mathcal{S}}^{-t}$,
%
\[ b = b \circ (a \circ c) = (b \circ a) \circ c = c \]
%
so $b = c$, modulo smoothing operators. Thus a left parameterix for an elliptic operator is automatically a right parameterix, and we have constructed such a parametrix.

\begin{remark}
    The condition that $|a(x,\xi)| \sim \langle \xi \rangle^t$ for large $\xi$ is necessary in order to construct a parametrix of order $-t$. Without loss of generality, it suffices to analyze the case $t = 0$, since in general we can replace $a(x,\xi)$ with $a(x,\xi) \langle \xi \rangle^{-t}$ and $b(x,\xi)$ with $b(x,\xi) \langle \xi \rangle^t$. If there is $b \in \dot{\mathcal{S}}^0$ such that $b(x,D)$ is a parametrix for $a(x,D)$, then the composition calculus tells us that $1 - a(x,\xi) b(x,\xi)$ is a local symbol of order $-1$. Thus we find that $|a(x,\xi) b(x,\xi) - 1| \lesssim \langle \xi \rangle^{-1}$ locally uniformly in $x$, which implies that, locally in $x$, there exists $R > 0$ such that for $|\xi| \geq R$,
    %
    \[ |a(x,\xi) b(x,\xi) - 1| \leq 1/2. \]
    %
    Thus
    %
    \[ |a(x,\xi)| \geq \frac{0.5}{|b(x,\xi)|}, \]
    %
    and combined with the fact that $|b(x,\xi)| \lesssim 1$, we conclude that $|a(x,\xi)| \gtrsim 1$.
\end{remark}

It follows from our theory that if $T$ is an elliptic pseudodifferential operator on $\Omega$, then for any $u \in \EC(\Omega)^*$, $\singsupp(Tu) = \singsupp(u)$. This is very similar to hypoellipticity, except that for differential operators, this condition can be applied to arbitrary distributions, not just the compactly supported distributions. More generally, we actually find that $\text{WF}(Tu) = \text{WF}(u)$ for $u \in \EC(\Omega)^*$ in virtue of the microlocal nature of pseudodifferential operators.

We can use similar asymptotic tools to construct formal fractional powers of an elliptic pseudodifferential operator. For simplicity, we assume we are constructing the fractional powers of an elliptic symbol $a(x,\xi)$ such that $a(x,\xi)$ is never a negative, real number, so that $a(x,\xi)^{p/q}$ is well defined as a principal branch of $z^{p/q}$. Of course, one can consider a similar development for any other choice of branch, assuming an appropriate constraint on the range of $a$.

\begin{theorem}
    Let $a \in \dot{S}^t$ be an elliptic symbol such that $a(x,\xi)$ is negative a negative, real number, and suppose that
    %
    \[ | \text{arg}(a(x,\xi)) - \pi | \gtrsim 1 \]
    %
    for $\xi \gtrsim 1$, where the implicit constants are locally uniform in $x$. Then for any pair of positive integers $p$ and $q$, there exists a unique $b \in \dot{S}^{t(p/q)}$ with principal symbol $a(x,\xi)^{p/q}$, such that $a(x,D)^p = b(x,D)^q$, modulo smoothing operators.
\end{theorem}
\begin{proof}
    Let $b_0(x,\xi) = a(x,\xi)^{p/q}$. Then $b_0$ is a symbol of order $t(p/q)$, and the composition calculus implies that $a(x,D)^p - b_0(x,D)^q$ is a pseudodifferential operator of order $tp - 1$, say, with symbol $r_1$. Given that we have chosen $b_0,\dots,b_N$ such that $a(x,D)^p - (b_0 + \dots + b_N)(x,D)^q$ is a pseudodifferential operator of order $tp - (N+1)$, say, with symbol $r_N$. The composition calculus means if we choose $b_{N+1} \in \dot{\mathcal{S}}^{t(p/q) - (N+1)}$ by setting
    %
    \[ b_{N+1} = q^{-1} r_N / (b_0 + \dots + b_N)^{q-1}. \]
    %
    Here we rely on the fact that $a$ is elliptic, so that the denominator is non-vanishing for large $\xi$. Then
    %
    \[ a(x,D)^p - (b_0 + \dots + b_N + b_{N+1})^q \]
    %
    is a pseudodifferential operator of order $tp - (N+2)$, allowing us to continue the construction. If we pick $b \sim \sum_{n = 0}^\infty b_n$, then $a(x,D)^p - b(x,D)^q$ will be a smoothing operator. Moreover, it is clear that the choice of such symbols at each step is essentially unique, which shows that $b$ is unique up to a smoothing operator.
\end{proof}

Given the assumptions of the theorem, we denote the unique operator $b$ by $a^{p/q}$. Using $a^{-1}$ instead of $a$, one can also construct negative fractional powers of the symbol $a$. It is simple to see that for $r_1,r_2 \in \QQ$, we have $a^{r_1} \circ a^{r_2} = a^{r_1 + r_2}$.

%\begin{theorem}
%    $T_a$ maps $\mathcal{S}(\RR^d)$ to $\mathcal{S}(\RR^d)$ continuously.
%\end{theorem}
%\begin{proof}
%    Integration by parts shows that 
%    Since $f \mapsto \widehat{f}$ is an isomorphism of $\mathcal{S}(\RR^d)$, it suffices to prove the operator
    %
%    \[ Sg(x) = \int a(x,\xi) g(\xi) e^{2 \pi i \xi \cdot x}\; d\xi. \]
    %
%    is continuous. Fix a multi-index $\alpha$ with $|\alpha| = m$. Now using the fact that $g$ is Schwartz, one finds
    %
%    \[ D^\alpha_x(Sg)(x) = \int D^\alpha_x a(x,\xi) g(\xi) e^{2 \pi i \xi \cdot x}\; d\xi. \]
    %
%    If we write $a_x(\xi) = a(x,\xi)$, then $D^\alpha_x(Sg)(x) = ((D^\alpha_x a_x) g)^\vee(x)$. Now
    %
%    \[ \nabla^n_\xi ( (D^\alpha_x a_x) \cdot g)(\xi) \lesssim_{n,m} \langle x \rangle^{k_m} \langle \xi \rangle^{l_{nm} - k} \| f \|_{\mathcal{S}^{n,k}(\RR^d)}. \]
    %
%    If $k$ is chosen larger than $l_{nm} + d$, then integration by parts implies that
    %
%    \[ |D^\alpha(Sg)(x)| = |((D^\alpha_x a_x) g)^\vee(x)| \lesssim_{n,m} \langle x \rangle^{k_m - n} \| f \|_{\mathcal{S}^{n,k}(\RR^d)}. \]
    %
%    Thus for any fixed $k_1$, there exists $k_2$ such that
    %
%    \[ \| Sg \|_{\mathcal{S}^{m,k_1}(\RR^n)} \lesssim_{m,k_1} \| g \|_{\mathcal{S}^{k_m + k_1, k_2}(\RR^d)}. \]
    %
%    This gives the required continuity of the operator.
%\end{proof}

\section{Regularity Theory}

Let us now discuss the boundedness of certain pseudodifferential operators with respect to various norm spaces. We first note that a differential operator of degree $m$ given by
%
\[ L = \sum c_\alpha(x) D^\alpha, \]
%
where $c_\alpha$ is bounded, maps $H^s(\RR^d)$ to $H^{s-m}(\RR^d)$ for each $s$. This feature remains true for a general pseudodifferential operator.

\begin{theorem}
    For any $a \in \loc{\mathcal{S}^t}(\RR^d)$, $a(x,D)$ extends uniquely to a continuous operator from $H^s_c(\RR^d)$ to $H^{s-m}_{\text{loc}}(\RR^d)$.
\end{theorem}
\begin{proof}
    Let $T$ have symbol $a(x,\xi)$. Without loss of generality, since we need only prove local estimates in the output we may assume that $a$ is compactly supported in the $x$-variable. Then $a$ is uniformly integrable in the $x$-variable, and we let
    %
    \[ A(\lambda,\xi) = \int a(x,\xi) e^{-2 \pi i \lambda \cdot x}\; dx \]
    %
    denote the Fourier transform of $a$ in the $x$-variable. For any input $\phi \in \mathcal{S}(\RR^d)$, $T\phi \in \mathcal{S}(\RR^d)$, and we may calculate that
    %
    \[ \widehat{T\phi}(\lambda) = \int A(\lambda - \xi,\xi) \widehat{\phi}(\xi)\; d\xi. \]
    %
    The assumptions on the symbol $a$ imply that
    %
    \[ |A(\lambda,\xi)| \lesssim_N \langle \xi \rangle^t \langle \lambda \rangle^{-N}. \]
    %
    Without loss of generality, assume that $s = t$. Applying Schur's lemma, the operator
    %
    \[ Sf(\lambda) = \int \langle \lambda - \xi \rangle^{-N} f(\xi)\; d\xi, \]
    %
    somewhat analogous to the Hardy-Littlewood-Sobolev fractional integration operator, is bounded from $L^2(\RR^d)$ to itself provided that $N > d$. But this means that if we pick $N > d$, then
    %
    \begin{align*}
        \| T\phi \|_{L^2(\RR^d)} &\lesssim_N \left\| \int \langle \lambda - \xi \rangle^{-N} \langle \xi \rangle^t \widehat{\phi}(\xi)\; d\xi \right\|_{L^2(\RR^d)}\\
        &\lesssim \| \langle \xi \rangle^t \widehat{\phi} \|_{L^2(\RR^d)}\\
        &\lesssim \| \phi \|_{H^t(\RR^d)}.
    \end{align*}
    %
    Thus $T$ is bounded from $H^t(\RR^d)$ to $L^2(\RR^d)$. But for general $s \in \RR^d$, $T$ will be bounded from $H^s(\RR^d)$ to $H^{s-t}(\RR^d)$ if and only if $(1 - \Delta)^{s-t} T (1 - \Delta)^{t-s}$ is bounded from $H^t(\RR^d)$ to $L^2(\RR^d)$, and this follows because $(1 - \Delta)^{s-t} T (1 - \Delta)^{t-s}$ is also a pseudodifferential operator of order $t$. The bounds we have proven show that there is a unique extension of $T$ to $H^s(\RR^d)$ for all $s \in \RR^d$ such that $\| Tf \|_{H^{s-t}(\RR^d)} \lesssim \| f \|_{H^s(\RR^d)}$ for all $f \in H^s(\RR^d)$. The closed graph theorem shows that this extension agrees with the definition of $Tf$ given for any compactly supported $f$, which we can view as an element of $\mathcal{E}(\RR^d)^*$.
\end{proof}

\begin{remark}
    For symbols of order zero, a simpler proof follows by writing
    %
    \[ T^\lambda \phi(x) = \int A(\lambda, \xi) \widehat{\phi}(\xi) e^{2 \pi i (\lambda + \xi) \cdot x}\; d\xi. \]
    %
    then the Fourier inversion formula shows that
    %
    \[ T \phi(x) = \int T^\lambda \phi(x)\; d\lambda. \]
    %
    Now the operators $\{ T^\lambda \}$ are just Fourier multiplier operators with symbols $\{ m_\lambda \}$, where $m_\lambda(\xi) = A(\lambda, \xi) e^{2 \pi i \lambda \cdot x}$, the bounds on $A$ imply that $\| m_\lambda \|_{L^\infty} \lesssim \langle \lambda \rangle^{-N}$, and this immediately gives boundedness from $L^2(\RR^d)$ to itself. The general result follows from the same trick as in the end of the last proof using the composition calculus.
\end{remark}

\begin{remark}
    If $a$ is \emph{properly supported} pseudodifferential operator of order $t$, then $a(x,D)$ extends to a continuous operator from $\loc{H^s}(\Omega)$ to $\loc{H^{s-t}}(\Omega)$.
\end{remark}

Using Calderon-Zygmund theory, we can obtain better estimates. Let us restrict ourselves at first to pseudodifferential operators of order $0$. The kernel of a $\Psi$DO of order zero satisfies estimates of the form
%
\[ |K(x,y)| \lesssim \frac{1}{|x - y|^d}. \]
%
Thus we focus on obtaining $L^2 \to L^2$ estimates, so that the standard theory of singular integrals gives $L^p \to L^p$ estimates for all $1 < p < \infty$.

\begin{theorem}
    If $a \in \mathcal{S}^0(\RR^d \times \RR^d)$, then for any $f \in \mathcal{S}$,
    %
    \[ \| T_af \|_{L^2(\RR^d)} \lesssim \| f \|_{L^2(\RR^d)}. \]
\end{theorem}
\begin{proof}
    If $\text{supp}_x(a)$ is compact, then we have already proven this result. To prove the result for more general symbols, we work with a kernel representation of $T_a$. Thus we write
    %
    \[ T_af(x) = \int_{\RR^d} K(x,y) f(x)\; dx, \]
    %
    where
    %
    \[ K(x,y) = \int_{\RR^d} a(x,\xi) e^{2 \pi i \xi \cdot (x - y)}\; d\xi. \]
    %
    We have already shown that the kernel $K$ is $C^\infty$ away from the diagonal, and decays rapidly away from that diagonal. This is one instance of the pseudolocal nature of these operators. Another quantitative result reflecting this nature is that for each $N > 0$ and $x_0 \in \RR^d$,
    %
    \[ \int_{|x - x_0| \leq 1} |T_af(x)|^2\; dx \lesssim_N \int_{\RR^d} \frac{|f(x)|^2}{\langle x - x_0 \rangle^N}\; dx. \]
    %
    Thus we can `almost' bound the magnitude of $T_af$ in a neighbourhood of $x_0$ by the magnitude of $f$ in a neighbourhood of $x_0$. We focus on the case $x_0 = 0$, the other cases treated in much the same way. Write $f = f_1 + f_2$, where $f_1$ is supported on $|x| \leq 3$, $f_2$ is supported on $|x| \geq 2$, and $|f_1|, |f_2| \leq |f|$. If $\eta(x)$ is a smooth cuttoff supported on $|x| \leq 3$, then the symbol $\eta(x) a(x,\xi)$ is compactly supported, and so
    %
    \begin{align*}
        \int_{|x| \leq 1} |T_a f_1(x)|^2\; dx &= \int_{|x| \leq 1} |T_{\eta a} f_1(x)|^2 \lesssim \| f_1 \|_{L^2(\RR^d)}^2\\
        &\lesssim_N \int_{\RR^d} \frac{|f_1(x)|^2}{\langle x \rangle^N}\; dx \leq \int_{\RR^d} \frac{|f(x)|^2}{\langle x \rangle^N}\; dx.
    \end{align*}
    %
    On the other hand, since $f_2(x)$ is supported on $|x| \geq 2$, we find that
    %
    \begin{align*}
        \int_{|x| \leq 1} |T_a f_2(x)|^2\; dx &= \int_{|x| \leq 1} \left| \int K(x,y) f_2(y)\; dy \right|^2\; dx\\
        &\leq \int \int_{|x| \leq 1} |K(x,y)|^2 |f_2(y)|^2\; dy\; dx\\
        &\lesssim_N \int \int_{|x| \leq 1} \frac{|f_2(y)|^2}{|x - y|^N}\; dy\; dx\\
        &\lesssim \int \int_{|x| \leq 1} \frac{|f_2(y)|^2}{\langle y \rangle^N}\; dy\\
        &\lesssim \int \frac{|f_2(y)|^2}{\langle y \rangle^N}\; dy.
    \end{align*}
    %
    But we now find that if $N > d$, then
    %
    \begin{align*}
        \int |T_af(x)|^2\; dx &\lesssim \int \int_{|x - y| \leq 1} |T_af(y)|^2\; dy\; dx\\
        &\lesssim_N \int \int \frac{|f(y)|^2}{\langle x - y \rangle^N}\; dy\; dx\\
        &\lesssim \int |f(y)|^2\; dy,
    \end{align*}
    %
    which gives $L^2$ boundedness.
\end{proof}

Sobolev norms follow simply from these bounds. Namely, it follows simply from this that if $a(x,\xi)$ is a symbol of order $t$, then for $1 < p < \infty$, and any $s$, we have bounds of the form
%
\[ \| T_a f \|_{L^p_s(\RR^d)} \lesssim_{p,s} \| f \|_{L^p_{t + s}(\RR^d)}. \]
%
If $T_a$ is an elliptic pseudodifferential operator of order $t$, then it has a parametrix $S$ of order $-t$, and so for any $r > 0$
%
\begin{align*}
    \| f \|_{L^p_{t + s}(\RR^d)} &= \| ST_a f + (I - ST_a) f \|_{L^p_{t+s}(\RR^d)}\\
    &\leq \| ST_a f \|_{L^p_{t+s}(\RR^d)} + \| (I - ST_a) f \|_{L^p_{t+s}(\RR^d)}\\
    &\lesssim_{p,s,r} \| T_a f \|_{L^p_s(\RR^d)} + \| f \|_{L^p_{-r}(\RR^d)} 
\end{align*}
%
Thus $T$ is \emph{almost} invertible as a map from $L^p_{t+s}(\RR^d)$ to $L^p_t(\RR^d)$, except that we cannot quite obtain the bound $\| T f \|_{L^p_s(\RR^d)} \sim_{p,s} \| f \|_{L^p_{t + s}(\RR^d)}$, for instance, by virtue of the fact that $T$ might not even be invertible. But we can obtain a less quantitative result.

\begin{theorem}
    Let $T$ be an elliptic pseudodifferential operator of order $t$. If $f \in \mathcal{E}(\RR^d)^*$, and $Tf$ lies in $L^p_s(\RR^d)$, then $f$ lies in $L^p_{s + t}(\RR^d)$.
\end{theorem}
\begin{proof}
    Since $T$ is elliptic, we can find a parametrix $S$, which is a pseudodifferential operator of order $-t$. Thus there exists a smoothing operator $U$ such that $1 = ST + U$. Since $Tf \in L^p_s(\RR^d)$, $STf \in L^p_{s + t}(\RR^d)$. Since $f$ is compactly supported, $Uf \in C^\infty(\RR^d)$, and thus in $L^p_{s+t,\text{loc}}(\RR^d)$. This means $f \in L^p_{s+t,\text{loc}}(\RR^d)$, and since $f$ is compactly supported, this means $f \in L^p_{s+t}(\RR^d)$.
\end{proof}















\section{Pseudodifferential Operators on Manifolds}

It is an important fact that the class of pseudodifferential operators whose kernels are compactly supported is invariant under a change of coordinates, modulo smoothing operators. By an analysis of how these pseudodifferential operators change under a change of coordinates, we will be able to obtain a theory of pseudodifferential operators on manifolds.

Let $\kappa: U \to V$ be a diffeomorphism, where $U$ and $V$ are open sets in $\RR^n$, and consider $a \in \loc{S}^t(U)$ which gives a pseudodifferential operator $T: \DD(U) \to \EC(U)$ of order $t$. Define an operator $S: \DD(V) \to \EC(V)$ by setting
%
\[ Sf(\kappa(x)) = T(f \circ \kappa)(x) = \int a(x,\xi) e^{2 \pi i \xi \cdot (x - y)} f(\kappa(y))\; dy\; d\xi, \]
%
i.e. $S$ is defined such that $\kappa_* \circ S = T \circ \kappa_*$. One can verify that $S$ is pseudolocal, so one might expect $S$ to be a pseudodifferential operator, of order $t$ as well. By localizing, we may assume that $\text{supp}_x(a)$ is compact, so that $T$ extends to a continuous operator from $\mathcal{D}(U)^*$ to $\mathcal{D}(U)^*$, and $S$ to a continuous operator from $\mathcal{D}(V)^*$ to $\mathcal{D}(V)^*$. If a symbol $b \in \loc{S}^t(V)$ existed such that $S = b(y,D)$, then we would find
%
\[ b(\kappa(x_0),\eta) = \left\{ S e^{2 \pi i \eta \cdot z} \right\}(\kappa(x_0)) = \int a(x_0,\xi) e^{2 \pi i [\xi \cdot (x_0 - x) + \eta \cdot \kappa(x)]}\; dx\; d\xi \]
%
Provided we can show that the right hand side defines a symbol $b \in \loc{S}^t(V)$, then $b(y,D)$ will be a continuous operator whose kernel is compactly supported in $V \times V$, and the fact that $S e^{2 \pi i \eta \cdot z} = b(y,D) e^{2 \pi i \eta \cdot z}$ for all $\eta$ implies, by continuity and the fact that linear combinations of exponentials are dense in $\mathcal{D}(V)^*$, that $S = b(y,D)$. Thus, given $a \in \loc{S}^t(U)$, such that $\text{supp}_x(a)$ is compact, our goal is an analysis of the function
%
\[ b(\kappa(x_0),\eta) = \int a(x_0,\xi) e^{2 \pi i [ \xi \cdot (x_0 - x) + \eta \cdot \kappa(x) ]}\; dx\; d\xi. \]
%
If $\phi \in \DD(V)$ is a bump function equal to one in a neighborhood of the projections of the support of the kernel of $S$ onto each coordinate, then
%
\[ b(\kappa(x_0),\eta) = \left\{ \phi S \left\{ \phi \cdot e^{2 \pi i \eta \cdot z} \right\} \right\}(\kappa(x_0)), \]
%
which implies that $b \in C^\infty(T^* V)$. If we set
%
\[ \Phi(\xi,\eta) = \int \phi(x) e^{2 \pi i [ \eta \cdot \kappa(x) - \xi \cdot x ]}\; dx, \]
%
then
%
\[ b(\kappa(x_0), \eta) = \int a(x_0,\xi) \Phi(\xi,\eta) e^{2 \pi i \xi \cdot x_0}\; d\xi. \]
%
Now $\Phi$ is a standard oscillatory integral, whose phase $\phi(x) = \eta \cdot \kappa(x) - \xi \cdot x$ has a stationary point for values of $x$ such that $D\kappa(x)^T \eta = \xi$. Since $D\kappa(x)^T$ is invertible, this can only happen when $|\eta| \sim |\xi|$. More precisely, suppose $|D\kappa(x)|, |D\kappa(x)^{-1}| \leq C$. Then $|D\kappa(x)^T \eta - \xi| \geq |\xi|/2 \gtrsim_C |\xi| + |\eta|$, and $|D\kappa(x)^T \eta - \xi| \geq |\eta|/2 \gtrsim_C |\xi| + |\eta|$ for $|\xi| \leq |\eta|/2C$. Thus unless $2C|\eta| \geq |\xi| \geq |\eta|/2C$, we conclude by the principle of nonstationary phase that for each $N > 0$,
%
\[ |\Phi(\xi,\eta)| \lesssim_N [1 + |\xi| + |\eta|]^{-N}, \]
%
where the implicit constant will be uniformly bounded over a family of diffeomorphisms $\kappa$ and a family of pseudodifferential operators $T$ if the kernels of the resulting operators $S$ are uniformly supported on a common compact subset of $V \times V$, and if we have uniform upper and lower bounds on the derivative of $\kappa$ on the support of these kernels. If we write $1 = \chi_1 + \chi_2$, for $\chi_1,\chi_2 \in C^\infty(\RR^d)$ with $\chi_1(\alpha)$ supported on $1/2C \leq |\alpha| \leq 2C$ and equal to one when $1/C \leq |\alpha| \leq C$, then this induces a decomposition $b(\kappa(x_0),\eta) = b_1(\kappa(x_0),\eta) + b_2(\kappa(x_0),\eta)$, where
%
\[ b_i(\kappa(x_0),\eta) = \int \chi_i(\xi/|\eta|) a(x_0,\xi) \Phi(\xi,\eta) e^{2 \pi i \xi \cdot x_0}\; d\xi. \]
%
The bound on $\Phi(\xi,\eta)$ above implies that $b_1 \in S^{-\infty}$. To analyze $b_2$, we apply the method of stationary phase. To simplify our formulas, let us assume without loss of generality that $x_0 = \kappa(x_0) = 0$ and that $B = D\kappa(x_0)$. If we fix $|\eta| = 1$, consider $\lambda > 1$, and do a change of variables, replacing $\xi$ with $\lambda (\xi + B^T \eta)$, we find that
%
\[ b_2(\kappa(x_0),\lambda \eta) = \lambda^d \int \psi(x,\xi) a(x,\lambda B^T \eta + \xi)) e^{2 \pi i \lambda \phi(x,\xi)}\; d\xi\; dx, \]
%
where $\psi(\xi,x) = \chi_2(\xi + B^T \eta) \phi(x_0 + x)$ and $\phi(x,\xi) = \eta \cdot (\kappa(x) - Bx) - \xi \cdot x$. This is a continuous family of non-degenerate stationary phase integrals, each with a unique stationary point when $(x,\xi) = (0,0)$. Since $\psi$ is equal to one in a neighborhood of this point, it will not show up in the corresponding asymptotics. The Hessian at the origin is precisely $A + B$, where
%
\[ A = \begin{pmatrix} 0 & -I \\ -I & 0 \end{pmatrix} \]
%
and
%
\[ C = \begin{pmatrix} \tilde{C} & 0 \\ 0 & 0 \end{pmatrix} \]
%
where $\tilde{C} = \tilde{C}(\eta)$ is the Hessian of the map $x \mapsto \eta \cdot \kappa(x)$ at the origin. If we set
%
\[ r(x) = \eta \cdot (\kappa(x) - Bx) - (1/2)(x^T A x) \]
% - 0.5 Dkappa(x)^T eta + D kappa(x)^T eta
%
which satisfies $\partial^\alpha r(0) = 0$ for all $|\alpha| \leq 2$, then it follows that if $a_\lambda(x_0,\xi) = a(x_0,\lambda \xi)$, then
% A -I    0  -I
% -I 0    -I -A
% - D_\xi^T D_x
% -D_x^T D_xi - D_xi^T A^T D_xi
\[ b_2(\kappa(x_0),\lambda \eta) \sim \sum_{2 \nu \geq 3 \mu} \frac{1}{(-2)^\nu} \frac{1}{\mu! \nu!} \frac{1}{(2\pi i \lambda)^{\nu - \mu}} \langle (A + B)^{-1} \nabla, \nabla \rangle^\nu \{ r^\mu a_\lambda \}(x, D\kappa(x)^T). \]
%
The next Lemma implies that we can write this asymptotic development as
%
\[ b_2(\kappa(x_0),\lambda \eta) \sim \sum_{\nu = 0}^\infty \langle i \cdot \nabla_x / 2\pi, \nabla_\xi / \lambda \rangle^\nu \left\{ e^{2 \pi i \lambda r(x)} a_\lambda(0,\xi) \right\}_{x = 0, \xi = B^T \eta} \]
%
where if we sum over $\nu \leq N$, then the error term will be $O(\lambda^{(d-N)/2})$.

\begin{lemma}
    Let $A$ be a symmetric, invertible matrix, let $B$ be a symmetric matrix, and suppose $\det(A + tB)$ is independent of $t$. Then there exists $k$ such that $(A^{-1}B)^k = 0$ for some $k$. For this $k$, and for any $N > 0$,
    %
    \[ \sum_{j < N} \frac{1}{j!} (2 i \lambda)^{-j} \langle (A + B)^{-1} \nabla, \nabla \rangle^j u(0) = \sum_{j < N} \frac{1}{j!} (2 i \lambda)^{-j} \langle A^{-1}\nabla, \nabla \rangle^j \left\{ e^{2 \pi i \lambda x^T B x / 2} u \right\}(0) + O(\lambda^{-N/k}). \]
\end{lemma}
\begin{proof}
    See Hormander, 7.7.
\end{proof}

Thus we have proved the following representation formula for pseudodifferential operators under changes of coordinates.

\begin{theorem}
    Let $\kappa: U \to V$ be a diffeomorphism, where $U$ and $V$ are open sets in $\RR^n$, and suppose $T = a(x,D)$ is a pseudodifferential operator of order $t$ on $U$. If $S$ is an operator defined on $V$ defined such that $\kappa_* \circ S = T \circ \kappa_*$, then $S$ is a pseudodifferential operator of order $t$, whose symbol $a_\kappa$ has the asymptotic development
    %
    \[ a_\kappa(\kappa(x),\eta) \sim \sum \frac{1}{(2 \pi i)^\alpha} \frac{1}{\alpha!} (\partial^\alpha_\xi a) (x,D\kappa(x)^T \eta) \left. \left\{ \partial^\alpha_y e^{2\pi i \eta \cdot r_x(y)} \right\} \right|_{y = x}, \]
    %
    where $r_x(y) = \kappa(y) - \kappa(x) - D\kappa(x)(y-x)$. The term in the sum corresponding to a multi-index $\alpha$ is a symbol of order $t - \lceil |\alpha| / 2 \rceil$, and thus this really is an asymptotic expansion.
\end{theorem}

Pseudodifferential operators defined by a local symbol of order $t$ are therefore invariant under a change of coordinates, modulo smoothing operators. Moreover, if $\kappa_* \circ S = T \circ \kappa_*$, where $T$ is a $\Psi DO$ with symbol $a \in \loc{S}^t(U)$, and $S$ is a $\Psi DO$ with symbol $b \in \loc{S}^t(V)$, then
%
\[ b(y,\eta) - a(\kappa^{-1}(y), D\kappa(x)^{-T} \cdot \eta) \]
%
is a pseudodifferential operator of order $t - 1$.

%\begin{theorem}
%    Let $U$ and $V$ be open subsets of Euclidean space together with a diffeomorphism $\kappa: U \to V$ be a diffeomorphism. If $a(x,\xi)$ is a symbol of order $m$, and $\text{supp}_x(a)$ forms a compact subset of $U$, then
    %
%    \[ a_\kappa(y,\eta) = e^{-2 \pi i \kappa^{-1}(y) \cdot \eta} a(x,D) e^{2 \pi i y \cdot \eta}. \]
    %
%    TODO (H\"{o}rmander's book seems to have the most readable discussion)
%\end{theorem}

Given a manifold $M$, a continuous operator $T: \DD(M) \to \EC(M)$ is called a \emph{pseudodifferential operator of order $t$} if whenever $(x,U)$ is a coordinate chart on $M$, the operator $T_x: \DD(x(U)) \to \EC(x(U))$ given by $T$ in coordinates is a pseudodifferential operator of order $t$ on $x(U)$. We let $\loc{\Psi^t}(M)$ denote the family of operators of this form. The next Lemma shows that when $M = U$ is an open subset of $\RR^n$, this really is the family of pseudodifferential operators defined by symbols in $\loc{\mathcal{S}^t}(T^* U)$.

\begin{lemma}
    Suppose $T: \DD(U) \to \EC(U)$ is a pseudodifferential operator on $U$ of order $t$, where $U$ is viewed as a manifold as above. Then we can find a symbol $a(x,\xi) \in \loc{\mathcal{S}^t}(U \times \RR^d)$ such that $T - a(x,D)$ is a smoothing operator. The symbol $a$ is uniquely determined up to a symbol in $\mathcal{S}^{-\infty}(\RR^d \times \RR^d)$.
\end{lemma}
\begin{proof}
    The idea is to work on a partition of unity, which we can sum up appropriately to get a sum over local estimates. The complete proof is supplied in Hormander, Proposition 18.1.19.
\end{proof}

\begin{remark}
    Under the assumption that the operator has a kernel which is smooth away from the diagonal, an operator $T: \DD(M) \to \EC(M)$ is a pseudodifferential operator if and only if it is pseudodifferential operator when transferred in coordinates on a family of coordinates charts that form an atlas for $M$. The assumption on the kernel is needed, for instance, since if we take $M = \RR$, we consider an atlas of the form $\{ (n, n+1) \}$, and we consider $Tf(x) = f(x - 2)$, then $T$ vanishes on each of these coordinates charts, and so looks to be a pseudodifferential operator in these charts, whereas $T$ clearly is not a pseudodifferential operator on $\RR$ since it is not pseudolocal.
\end{remark}

It is often to discuss operators that can be asymptotically expanded in terms of homogeneous symbols. That is, a symbol $a \in \dot{\mathcal{S}}^t$ of order $t$ is classical if there exist a sequence of symbols $\{ a_k \}$, where $a_k(x,\xi)$ is smooth away from $\xi = 0$, homogeneous of order $t - k$, and
%
\[ a \sim \sum_{k = 0}^\infty a_k. \]
%
Such symbols are called \emph{classical}, and denoted $\mathcal{S}^t_{\text{cl}}$, since this was the family of pseudodifferential operators initially studied by Kohn and Nirenberg. Given a manifold $M$, we write $\Psi_{\text{cl}}^t(M) = \text{Op}(\mathcal{S}^t_{\text{cl}})(M)$ for the class of all pseudodifferential operators $T$ such that $T_x$ is classical for any coordinate system $(x,U)$ on $M$. It suffices to check this on a cover because of the asymptotic expansion for the change of variables formula. The homogeneous function which agrees with the leading term in the expansion is invariant under coordinate changes if we interpret it as a function on $T^* M - 0_M$, so for a classical pseudodifferential operator $T$ in $\Psi_{\text{cl}}^t(M)$, we can define the \emph{principal symbol} $a \in C^\infty(T^* M - 0_M)$ to be that homogeneous function of order $t$ which agrees with the leading term in the asymptotic expansions above. For nonclassical operators, there is no canonical choice of a principal symbol, though one can consider the principal symbol as an element of $\loc{\mathcal{S}^t}(M) / \mathcal{S}^{t-1}_{\text{loc}}(M)$. A pseudodifferential operator is then called \emph{elliptic} if it's principal symbol is nonvanishing at a suitably distance away from the origin, or equivalently, if it is elliptic in coordinates ranging over the entirety of $M$.

\begin{example}
    Let $\TT = \RR / \ZZ$. For any `symbol of order $t$' on $\TT^d$, i.e. any function $a: \TT^d_x \times \ZZ^d_\xi \to \CC$ with extends to an element of $\mathcal{S}^t(\TT^d_\xi \times \RR^d_\xi)$, we can consider the operator $T: \EC(\TT^d) \to \EC(\TT^d)$ defined by setting
    %
    \[ T\phi(x) = \sum_{\xi \in \ZZ^d} a(x,\xi) \widehat{\phi}(\xi) e^{2 \pi i \xi \cdot x}, \]
    %
    where $\widehat{\phi}: \ZZ^d \to \CC$ is the Fourier transform of $\phi$ on $\TT^d$. We claim that $T$ is a pseudodifferential operator. By translation invariance, it will suffice to find a neighborhood $\Omega$ of the origin upon and a coordinate system on $\Omega$ upon which the transfer of $T$ is a pseudodifferential operator. Write
    %
    \begin{align*}
        T\phi(x) &= \sum_\xi a(x,\xi) \widehat{\phi}(\xi) e^{2 \pi i \xi \cdot x}\\
        &= \int_{\TT^d} \left( \sum_\xi a(x,\xi) e^{2 \pi i \xi \cdot (x - y)} \right) \phi(y)\; dy\\
        &= \int_{\TT^d} K(x,y) \phi(y)\; dz,
    \end{align*}
    %
    where we treat
    %
    \[ K(x,y) = \sum_\xi a(x,\xi) e^{2 \pi i \xi \cdot (x - y)}. \]
    %
    as the distribution on $\DD(\TT^d \times \TT^d)^*$ obtained as the distributional limit of the partial sums. Consider the periodic kernel $\tilde{K} \in \DD(\RR^d \times \RR^d)^*$ induced by $K$. Then the Poisson summation formula implies that
    %
    \[ \tilde{K}(x,y) = \sum_\xi a(x,\xi) e^{2 \pi i \xi \cdot (x - y)} = \sum_n (\mathcal{F}_\xi^{-1} a)(x,(x-y) + n) = \sum_n K_a(x,y+n), \]
    %
    where $K_a$ is the kernel of the psedodifferential operator $a(x,D)$. If $\widetilde{\Omega} = (-1/4,1/4)^d$, then for $x,y \in \widetilde{\Omega}$, and $n \neq 0$,
    %
    \[ |\partial^\alpha_x \partial^\beta_y K_a(x,y+n)| \lesssim_N |n|^{-N}. \]
    %
    This implies that
    %
    \[ (x,y) \mapsto \sum_{n \neq 0} K_a(x,y+n) \]
    %
    lies in $C^\infty(\widetilde{\Omega} \times \widetilde{\Omega})$, and so induces a smoothing operator on $\widetilde{\Omega}$. Thus $\tilde{T}$, restricted to $\tilde{\Omega}$, differs by a smoothing operator from the pseudodifferential operator $a(x,D)$. But this means that $\tilde{T} \in \dot{\Psi}^t(\widetilde{\Omega}$. If $\Omega$ is the subset of $\TT^d$ corresponding to $\widetilde{\Omega}$, then this means $T$ behaves like a pseudodifferential operator on $\widetilde{\Omega}$. Thus $T$ is actually a pseudodifferential operator of order $t$ on $\TT^d$. Modulo smoothing operators, working backwards through this argument also clearly shows that \emph{all pseudodifferential operators} of order $t$ can be written in the form introduced at the beginning of this example.

    As a particular example of this kind of construction, consider the operator $T: \EC(\TT) \to \EC(\TT)$ given by
    %
    \[ P_+ \phi(t) = \sum_{n > 0} \widehat{\phi}(n) e^{2 \pi nit}, \]
    %
    then $P_+$ is a classical pseudodifferential operator on $\TT$ of order zero, and it's principal symbol is $\mathbf{I}(\xi > 0)$. Similarily, if
    %
    \[ P_- \phi(t) = \sum_{n < 0} \widehat{\phi}(n) e^{2 \pi nit}, \]
    %
    then $P_-$ is a classical pseudodifferential operator on $\TT$ of order zero with principal symbol $\mathbf{I}(\xi < 0)$. For any non-zero complex values $a_+,a_- \in \CC - \{ 0 \}$, $a_+ P_+ + a_- P_-$ is an elliptic classical pseudodifferential operator on $\TT$ of order zero.
\end{example}

\begin{example}
    On a compact Riemannian manifold $M$, the operator $- \Delta$ is a positive-semidefinite differential operator of order two with symbol given in local coordinates as
    %
    \[ a(x,\xi) = \sum g^{ij}(x) \xi_i \xi_j. \]
    %
    TODO: Change argument by arguing it is elliptic, and so we can take fractional powers.

    The spectral theory of this operator decomposes it as a sum $\sum \lambda_i^2 E_i$, where $E_i$ is an orthogonal projection onto a one dimensional eigenspace, and $\lambda_0 \leq \lambda_1 \leq \dots$ are ordered by multiplicity. Then we can define the operator $\sqrt{-\Delta}$ to be the operator $\sum \lambda_i E_i$. Let us argue that $\sqrt{-\Delta}$ is a classical pseudodifferential operator of order one on $M$. More precisely, we will construct a positive definite pseudodifferential operator $Q$ of order one such that $Q - \sqrt{-\Delta}$ is smoothing.

    To construct $Q$, we first consider the modified Laplacian $Lu = E_0u + \sum_{i \geq 1} \lambda_i^2 E_i$. Then $L$ is invertible since it is positive definite, and differs from $- \Delta$ by the smoothing operator
    %
    \[ E_0 u = \frac{1}{\text{Vol}(M)} \int_M u(x) dV(x). \]
    %
    We will not exploit the fact that $L$ is invertible until the end of the argument, but keep this invertibility in mind for later.

    To begin with, we construct a positive definite $Q \in \Psi_{\text{cl}}^1(M)$ such that $L - Q^2$ is a smoothing operator. To do this we may assume without loss of generality that we are working in a small coordinate patch in local coordinates, and that we are patching these locally defined parts of the operator globally using a partition of unity. We then define
    %
    \[ \tilde{b}_1(x,\xi) = \chi_1(\xi) \left( \sum g^{i,j}(x) \xi_i \xi_j \right)^{1/2} \]
    %
    where $\chi_1$ is smooth and vanishes for $|\xi| \leq 1$. Now define $Q_1 = b_1(x,D)$ to be equal to the pseudodifferential operator $\text{Re}(\tilde{b}_1(x,D)) = (1/2)(\tilde{b}_1(x,D) + \tilde{b}_1(x,D)^*)$. The calculus of pseudodifferential operators tells us that $b_1$ is a classical symbol of order one, and the principal part of $b_1$ agrees with the principal part of $\tilde{b}_1$. We therefore conclude that $R_1 = L - Q_1^2$ is a classical pseudodifferential operator of order one with principal symbol $r_1(x,\xi)$.

    We now proceed inductively, assuming that we have found classical symbols $b_i$ of order $2 - i$ for each $1 \leq i \leq n$, such that if $Q_i = b_i(x,D)$, then $Q_i$ is self-adjoint, and $R_n = L - (Q_1 + \dots + Q_n)^2$ is a classical pseudodifferential operator of order $2 - n$ with some principal symbol $r_n(x,\xi)$ of order $2 - n$. Our goal is now to find a non-negative symbol $\tilde{b}_{n+1}(x,\xi)$, homogeneous of order $1 - n$ for large $\xi$, defining a self-adjoint pseudodifferential operator $Q_{n+1}$ such that $L - (Q_1 + \dots + Q_{n+1})^2$ is a classical pseudodifferential operator of order $1 - n$. Expanding out the order $2 - n$ part of $L - (Q_1 + \dots + Q_{n+1})^2$, this is possible provided that for large $\xi$,
    %
    \[ r_n(x,\xi) = 2 b_1(x,\xi) b_{n+1}(x,\xi) \]
    %
    Thus we define
    %
    \[ \tilde{b}_{n+1}(x,\xi) = \chi_{n+1}(\xi) \left( \frac{r_n(x,\xi)}{2 b_1(x,\xi)} \right) \]
    %
    where $\chi_{n+1}$ is smooth, equal to one for large $\xi$, and vanishes for values of $\xi$ such that $b_1(x,\xi) = 0$, and then set $b_{n+1}$ to be the classical symbol of order $1 - n$ such that $Q_{n+1} b_{n+1}(x,D) = \text{Re}(\tilde{b}_{n+1}(x,D))$. Then $R_{n+1} = L - (Q_1 + \dots + Q_{n+1})^2$ is a classical pseudodifferential operator of order $1 - n$, allowing us to continue our induction.

    Thus we find now that we have constructed an infinite sequence of classical pseudodifferential operators $b_1, b_2, \dots$, where $b_i$ is order $2 - i$. Using symbol asymptotics, we can therefore pick any representative symbol $\tilde{b}$, such that
    %
    \[ \tilde{b} \sim \sum_{i = 1}^\infty b_i, \]
    %
    and then $\tilde{b}$ will be a classical symbol of order 1. The fact that each of the operators $Q_i$ is self-adjoint implies that the difference between $\tilde{b}(x,D)$ and $\text{Re}(\tilde{b}(x,D))$ is a smoothing operator. Thus if we let $b$ be the symbol of $\text{Re}(\tilde{b}(x,D))$, then we find
    %
    \[ b \sim \sum_{i = 1}^\infty b_i, \]
    %
    the operator $Q = b(x,D)$ is self-adjoint, and $L - Q^2$ is a smoothing operator.

    We claim that $\sqrt{L} - Q$ is also a smoothing operator. To see this, we employ heuristics from the holomorphic functional calculus. Since the spectrum of $L$ is contained in $(\varepsilon,\infty)$ for some $\varepsilon > 0$, and the function $\sqrt{z}$ is holomorphic in a neighborhood of this region, if we consider the rectifiable curve $\gamma(t) = |t| + i \text{sgn}(t) \cdot t$, then we should expect to have
    %
    \[ L^{-1/2} = \frac{1}{2 \pi i} \int_\gamma z^{-1/2} (z - L)^{-1}\; dz. \]
    %
    The operator $(z - L)^{-1}$ is bounded, with operator norm $O(1/z)$ for large $z$, and operator norm $O(1)$ for small $z$, which implies that the operator-valued integral is absolutely convergent. Thus the left hand and right hand side of this equation both define bounded operators on $L^2(M)$. And for each of the eigenvectors $e_i$ for $L^{-1}$,
    %
    \[ \frac{1}{2 \pi i} \int_\gamma z^{-1/2} (z - L)^{-1} e_i\; dz = \left( \frac{1}{2 \pi i} \int_\gamma z^{-1/2} (z - \lambda_i)^{-1} \right) e_i = \lambda_i^{-1/2} e_i, \]
    %
    so $L^{-1/2}$ and the integral formula have the same eigenvectors and eigenvalues, and are therefore equal. Similarily,
    %
    \[ Q^{-1} = \frac{1}{2 \pi i} \int_\gamma z^{-1/2} (z - Q^2)^{-1}, \]
    %
    since $Q^{-2}$ is bounded and positive definite, hence diagonalizable, and so one can apply analogous arguments to those above. If we set $R = L - Q^2$, then
    %
    \begin{align*}
        L^{-1/2} - Q^{-1} &= \frac{1}{2 \pi i} \int_\gamma z^{-1/2} \left( (z - L)^{-1} - (z - Q^2)^{-1} \right)\\
        &= \frac{1}{2 \pi i} \int_\gamma z^{-1/2} \left( (z - L)^{-1} - (z - L + R))^{-1} \right)\\
        &= \frac{1}{2 \pi i} \int_\gamma z^{-1/2} (z - (L + R))^{-1} R (z - L)^{-1}.
    \end{align*}
    %
    Because the operator norm of $(z - (L + R))^{-1}$ and $(z - L)^{-1}$ are each individually $O(1/z)$, and $R$ is a smoothing operator, if we let $T_z$ be the integrand of this operator, then for any $k > 0$,
    %
    \[ \| T_z u \|_{L^2_k(M)} \lesssim_k \frac{1}{1 + z^2} \cdot \| u \|_{L^2(M)}, \]
    %
    and from this we see that $L^{-1/2} - Q^{-1}$ is a smoothing operator. But now we can write
    %
    \[ Q - L^{1/2} = Q(L^{-1/2} - Q^{-1}) L^{-1/2}, \]
    %
    and so we see that $Q - L^{1/2}$ is a smoothing operator. But this means that $Q - \sqrt{-\Delta}$ is smoothing.
\end{example}











\section{$\Psi$DOs and Microsupport}

Fix an open set $\Omega \subset \RR^d$. The theory of microlocal analysis asks us to think of the singularities of a distribution $u \in \DD(\Omega)^*$ as having both position \emph{and} direction, i.e. existing on the wavefront set $\text{WF}(u) \subset T^* M$, and indicated by the directions that the Fourier transform of $u$ does not decay rapidly, once sufficiently localized. The heuristics of pseudodifferential operators also ask us to view a distribution $u$ as living on $T^*M$, such that a pseudodifferential operator $S$ given by a symbol $a$ acts on $u$ by multiplying that part of $u$ that `lives at' the point $(x,\xi)$ by the quantity $a(x,\xi)$. In particular, if $a$ is made to decay rapidly in the directions that define the wavefront set $u$, then we should expect $Su$ to be smooth, i.e. application of $S$ annihilates the singularities of $a$. In this section, we elaborate on these ideas, as well as introducing further notions of the microlocal properties of pseudodifferential operators.

If $T: \DD(\Omega) \to \EC(\Omega)$ is a pseudodifferential operator with symbol $a$, then we define the microsupport $\msupp(T)$ of $T$ to be equal to the microsupport $\msupp(a)$ of it's symbol, defined in the parts of these notes on oscillatory integral distributions.

\begin{lemma}
    Let $T$ be a pseudodifferential operator on $\Omega$. Then it's canonical relation is equal to
    %
    \[ \mathcal{C}_T = \{ (x,x;\xi,\xi): (x,\xi) \in \msupp(T) \}. \]
    %
    This follows because an open conic set $\Gamma \subset \Omega \times \RR^d$ is disjoint from $\msupp(T)$ if and only if $\Gamma$ is disjoint from $\text{WF}(Tu)$ for any $u \in \EC(\Omega)^*$.
\end{lemma}
\begin{proof}
    Let $T = a(x,D)$ for some symbol $a(x,\xi)$. It follows from the general theory of oscillatory integral distributions that the canonical relation of $T$ is equal to
    %
    \[ \mathcal{C}_T \subset \{ (x,\xi;x,\xi): (x,\xi) \in \msupp(T) \}. \]
    %
    Thus if $\Gamma$ is disjoint from $\msupp(T)$, then it follows from the general theory that $\Gamma$ is disjoint from $\text{WF}(Tu)$ for any $u \in \EC(\Omega)^*$. If we assume the converse, then for any $(x_0,\xi_0) \in \Gamma$, we can find a pseudodifferential operator $S: \DD(\Omega) \to \EC(\Omega)$ with a symbol $b(x,\xi)$ supported on $\Gamma$ and equal to one in a conic neighborhood of $(x_0,\xi_0)$. It follows that $ST$ is a smoothing operator. It follows from the composition calculus that $a \cdot b \in \mathcal{S}^{-\infty}(\Omega \times \RR^d)$. Thus $(x_0,\xi^0) \not \in \msupp(a) = \msupp(T)$.
\end{proof}

Similar to the argument above, the composition calculus implies that for any two properly supported pseudodifferential operators $T$ and $S$,
%
\[ \msupp(TS) \subset \msupp(T) \cap \msupp(S), \]
%
and for any pseudodifferential operator $T$,
%
\[ \msupp(T^t) = \{ (x,-\xi): (x,\xi) \in \msupp(T) \} \quad \msupp(T^*) = \msupp(T). \]
%
One can prove these either using the microsupport of the symbols defining $T$ and $S$, or the properties of the canonical relation of $T$ and $S$.

\begin{theorem}
    Let $\Gamma$ be a closed conic set, and suppose $T$ is a properly pseudodifferential operator with $\msupp(T) \cap \Gamma = \emptyset$, then $T$ maps $\DD^*_\Gamma(\Omega)$ continuously into $\EC(\Omega)$.
\end{theorem}
\begin{proof}
    TODO: For any open set $U \subset \Omega$, control the derivatives of $T \phi$ on $U$ by decomposing $T$ into inputs outside of $U$, a $\Psi$DO with a symbol in $\mathcal{S}^{-\infty}$, and a $\Psi$DO localized near $\msupp(T)$, and so on.
\end{proof}

In fact, a sequence $\{ u_n \}$ converges in $\DD^*_\Gamma(\Omega)$ to some $u \in \DD^*_\Gamma(\Omega)$ if and only if it converges distributionally to $u$, and $Tu_n \to Tu$, where $T$ is an arbitrary properly supported $\Psi DO$ with $\Gamma \cap \msupp(T) = \emptyset$. The proof is left to the reader.

\begin{theorem}
    Fix $u \in \DD(\Omega)^*$. Then $(x_0,\xi^0) \not \in \text{WF}(u)$ if and only if there exists a conic neighborhood $\Gamma$ of $(x_0,\xi^0)$ such that for any properly supported pseudodifferential operator $T$ on $\Omega$ with $(x_0,\xi^0) \in \msupp(T)$, $Tu \in \EC(\Omega)$.
\end{theorem}

\begin{remark}
    It follows from this theorem, and the fact that the class of pseudodifferential operators and the microsupport of such operators are invariant under diffeomorphism, that the wavefront set of a distribution is invariant under diffeomorphisms.
\end{remark}

We have already seen the regularity theory of pseudodifferential operators. Recall that if $T$ is a properly supported pseudodifferential operator of order $t$, then $T$ maps $\loc{H^s}(\Omega) \to \loc{H^{s-t}}(\Omega)$. By virtue of the fact that this is an isomorphism if $T$ is elliptic, one can \emph{define} $\loc{H^s}(\Omega)$ to be the space of all distributions $u \in \DD(\Omega)^*$ such that $Tu \in L^2_{\text{loc}}(\Omega)$, where $T$ can be an arbitrary properly supported $\Psi DO$ of order $m$.



Let us also now look at the \emph{microlocal regularity} of $T$, i.e. the regularity of $T$ where we only care about regularity in certain conical subsets of $T^* \Omega$. For a conic open set $\Gamma \subset \Omega \times \RR^d$, we define $H^s_{\Gamma,\text{loc}}(\Omega)$ to be the family of all distributions $u \in \DD(\Omega)^*$ such that for any properly supported pseudodifferential operator $T$ of order zero with conically compact microsupport $\msupp(T) \subset \Gamma$, $Tu \in \loc{H^s}(\Omega)$. By the results above, and the Sobolev embedding theorem, $\lim_{s \to \infty} H^s_{\Gamma,\text{loc}} (\Omega) = \DD^*_\Gamma(\Omega)$.

\begin{theorem}
    If $T$ is a properly supported pseudodifferential operator of order $t$ on $\Omega$, then $T$ maps $H^s_{\Gamma,\text{loc}}(\Omega)$ into $H^{s-t}_{\Gamma,\text{loc}}(\Omega)$.
\end{theorem}
\begin{proof}
    Fix a conically compact set $\Gamma_1 \subset \Gamma$. If $S$ is a properly supported pseudodifferential operator of order zero with $\msupp(S) \subset \Gamma$, it suffices to show that $ST$ maps $H^s_{\Gamma,\text{loc}}(\Omega)$ into $\loc{H^{s-t}}(\Omega)$. But $ST = STU_1 + STU_2$, where $U_1$ and $U_2$ are properly supported pseudodifferential operators of order zero, such that $\msupp(U_1)$ is conically compact and contained in $\Gamma$, and $\msupp(U_2)$ is disjoint from $\Gamma_1$. But this means that if $v \in H^s_{\Gamma,\text{loc}}(\Omega)$, then $U_1 v \in \loc{H^s}(\Omega)$, and since $ST$ is a properly support pseudodifferential operator of order $t$, $STU_1 v \in \loc{H^{s-t}}(\Omega)$. On the other hand, $\text{WF}(U_2 v)$, and thus $\text{WF}(TU_2 v)$, is disjoint from $\Gamma_1$, which implies that $STU_2$ is smooth, and thus lies in $\loc{H^{s-t}}(\Omega)$ trivially.
\end{proof}

Ellipticity can also be microlocalized. If $T$ was an elliptic pseudodifferential operator, we found a pseudodifferential parametrix $S$ for $T$, i.e. such that $S \circ T$ and $T \circ S$ are smoothing operators. We say a symbol $a \in \dot{S}^t(\Omega)$ is \emph{elliptic} on a conic open set $\Gamma \subset \Omega \times \RR^d$ if for $(x,\xi) \in \Gamma$,
%
\[ |a(x,\xi)| \sim \langle \xi \rangle^t \]
%
with an implicit constant uniform in $\xi$, and locally uniform in $x$. In this case, we can find a `parametrix' on \emph{conically compact} subcones of $\Gamma$.

\begin{theorem}
    If $T$ is a pseudodifferential operator with a symbol $a \in \dot{S}^t(\Omega)$ which is elliptic on a conic set $\Gamma$, then there exists a pseudodifferential operator $S \in \dot{S}^{-t}(\Omega)$ such that $T \circ S$ and $S \circ T$ are regularizing on $\Gamma$.
\end{theorem}
\begin{proof}
    If $\Gamma_1$ is a conic open subset of $\Gamma$, we can find a pseudodifferential operator $S$ with a symbol $b$ by a recursive formula similar to that of the construction of a parametrix of an elliptic symbol, though applying cutoffs outside of $\Gamma_1$ so that the symbol vanishes outside of $\Gamma$ and remains smooth and well defined.
\end{proof}

As a consequence, if a $\Psi$DO $T$ is elliptic on $\Gamma$, then for any compactly supported distribution $u$, it follows that
%
\[ \text{WF}(Tu) \cap \Gamma = \text{WF}(u) \cap \Gamma, \]
%
i.e. the wavefront set is preserved on $\Gamma$.

If $T$ is a properly supported classical pseudodifferential operator of order $t$ with principal symbol $p(x,\xi)$, then the \emph{characteristic set} of $T$ is
%
\[ \Char(T) = \{ (x,\xi): p(x,\xi) = 0 \}. \]
%
The characteristic set is closed and conic, and $T$ is elliptic on the complement of $\Char(T)$. Thus it follows that for $u \in \DD(\Omega)^*$, $\text{WF}(u) \subset \text{WF}(Tu) \cup \Char(T)$. Therefore, if $Tu \in \EC(\Omega)$, then $\text{WF}(u) \subset \Char(T)$.





\section{Vector-Valued Pseudodifferential Operators}

For certain applications of pseudodifferential operators, e.g. to systems of partial differential equations, it is useful to have a theory of vector-valued pseudodifferentail operators acting on systems of distributions. The theory is almost entirely analogous to the scalar-valued theory, except that products of pseudodifferential operators may not commute.

We consider only finite dimensional vector-valued quantities here, though the generalization to Banach spaces is not too difficult to imagine. If $V$ is a finite dimensional vector space, and $\Omega \subset \RR^d$, we can define a family $\DD(\Omega;V)$ of $V$-valued test functions, and thus obtain a family $\DD(\Omega,V)^*$ of $V$-valued distributions on $\Omega$. These spaces are really just $\DD(\Omega) \CT V$ and $\DD(\Omega)^* \widehat{\otimes} V$. Thus if $V$ has a basis $\{ e_1, \dots, e_n \}$, then elements of $\DD(\Omega,V)$ can be uniquely expanded as $\phi_1 e_1 + \dots + \phi_n e_n$ for test functions $\{ \phi_i \}$ in $\DD(\Omega)$, and elements of $\DD(\Omega,V)^*$ can be uniquely expanded as $u_1 e_1 + \dots + u_n e_n$ for distributions $\{ u_i \}$ in $\DD(\Omega)^*$. Similarily, we can define $\EC(\Omega,V)$, $\EC(\Omega,V)^*$, $H^s(\Omega,V)$, and virtually all of the other function spaces of interest to us in analysis, and these also just turn out to be tensor products.

If $V$ and $W$ are both finite dimensional linear spaces, then we have a natural isomorphism
%
\[ L(X \otimes V, Y \otimes W) \cong L(X,Y) \otimes L(V,W). \]
%
In particular, after fixing bases $\{ e_1, \dots, e_n \}$ and $\{ e_1', \dots, e_n' \}$ for $V$ and $W$, an arbitrary operator $T$ in $L(X \otimes V, Y \otimes W)$ can be written uniquely as
%
\[ T \left( \sum_{i = 1}^n x_i \otimes e_i \right) = \sum_{i = 1}^n \sum_{j = 1}^m T_{ij}(x_i) e_j \]
%
where $T_{ij}$ lie in $L(X,Y)$. Thus every vector-valued continuous linear map is given by an $n \times m$ matrix of scalar-valued continuous linear maps.

It is trivial that we have a Schwartz kernel for such operators, i.e. for any $T: \DD(\Omega_1,V) \to \DD(\Omega_2,W)^*$, there exists a matrix-valued distribution $K \in \DD(\Omega_2 \times \Omega_1, V \otimes W)^*$ such that
%
\[ \langle T \phi, \psi \rangle = \int K(x,y) (\phi(x) \otimes \psi(y))\; dx\; dy. \]
%
Let us now specialize to the study of vector valued pseudodifferential operators

We define a pseudodifferential operator $T$ of order $t$ on $\Omega$, \emph{valued in $L(V,W)$}, to be an operator from $\DD(\Omega,V)$ to $\EC(\Omega,W)^*$ induced by an element of $\loc{\Psi^t}(\Omega) \widehat{\otimes} L(V,W)$. With any such operator we can associate $a: \Omega \times \RR^d \to L(V,W)$, a \emph{vector-valued symbol} of order $t$, such that
%
\[ T\phi(x) = \int a(x,\xi) \phi(y) e^{2 \pi i \xi \cdot (x-y)}\; dy. \]
%
The symbolic calculus remains unpeturbed, except for a few modifications:
%
\begin{itemize}
    \item The operators $T^t$ and $T^*$ are $L(W^*,V^*)$-valued.

    \item If $T$ is an $L(V,W)$ valued proper pseudodifferential operator of order $t$, and $S$ an $L(W,U)$ valued operator of order $s$, then the operator $T \circ S$ is an $L(V,U)$ valued pseudodifferential operator of order $t + s$ with an analogous asymptotic expansion to the scalar-valued case, but the non-commutativity implies that the commutator $[T,S] = T \circ S - S \circ T$ is \emph{not necessarily} a pseudodifferential operator of $t + s - 1$.

    \item If $T$ is a classical $L(V,W)$ valued pseudodifferentail operator with principal symbol $p(x,\xi)$, then the characteristic set of $T$ is
    %
    \[ \Char(T) = \{ (x,\xi): p(x,\xi): V \to W\ \text{is not injective} \}. \]
    %
    Then $T$ is elliptic if $\Char(T) = \emptyset$. One can consider the analogous microlocal variants.
\end{itemize}

We can also consider pseudodifferential operators valued in linear maps between two vector bundles $E$ and $F$ on a space $\Omega$, i.e. an operator $T: C^\infty(\Omega;E) \to C^\infty(\Omega;F)$ which, when taken in coordinates given by trivializations of $E$ and $F$, look like matrix valued pseudodifferential operators. We can then associate such an operator $T$ with a symbol, i.e. a family of sections $\RR^n \to \Gamma(\Omega, L(E,F))$ with appropriate smoothness and decay. One can then define $\Char(T)$ as above, and the notion of an elliptic operator.

\begin{example}
    If $M$ is a smooth manifold, then the exterior derivative $\Omega^n(M) \to \Omega^{n+1}(M)$ is a pseudodifferential operator. Let us focus on the case where $M$ is an open submanifold of $\RR^d$ so that it has a natural coordinate system. Since
    %
    \[ d(f dx^S) = \sum_{j \not \in S} \frac{\partial f}{\partial x^j} dx^j \wedge dx^S = \sum_{j \not \in S} (2 \pi i) (D_x^j f) dx^j \wedge dx^S, \]
    %
    it follows that the symbol of the exterior derivative is
    %
    \[ a(\xi) = \sum_{\substack{S \subset \{ 1, \dots, d \}\\\#(S) = n}} \sum_{j \not \in S} \sigma(j,S) \cdot 2 \pi i \xi_j \cdot E_{S,S \cup \{ j \}}, \]
    %
    where $E_{S,T}: L(\Lambda^n, \Lambda^m)$ is simply the bundle map that maps $dx^S$ to $dx^T$, and everything else to zero, and where $\sigma(j,S) \in \{ -1, 1 \}$ is chosen such that $dx^j \wedge dx^S = \sigma(j,S) dx^{S \cup \{ j \}}$. For $n = 0$ in particular, we have
    %
    \[ a(\xi) = \sum_{j = 1}^d 2 \pi i \xi_j E_j, \]
    %
    thus we see that the exterior differential in this case is always injective, and so $d: \Omega^0(M) \to \Omega^1(M)$ is elliptic. For $n > 0$ on the other hand, is \emph{not} elliptic. For instance, in the case $n = 1$, we have
    %
    \[ a(\xi) = \sum_{i < j} 2 \pi i (\xi_j E_{i, \{ i, j \}} - \xi_i E_{j, i \cup \{ i,j \}} ). \]
    %
    But then we notice that for each fixed $\xi^0 \in \RR^d$, $a(\xi^0)$ maps $\sum \xi^0_i dx^i$, and thus $\Char(T) = \Omega \times \RR^d_\xi$.
\end{example}

\begin{example}
    We recall that the principal symbol of a classical pseudodifferential operator on a manifold $M$ is invariantly defined on $T^*M$. Given a manifold $M$, we can consider the line bundle $\text{Vol}^{1/2}(M)$ of half scalar densities. We claim there is a more rich invariant that can be associated with pseudodifferential operators from $\text{Vol}^{1/2}(M)$ to itself, namely, the \emph{subprincipal symbol}, given in coordinates such that if $T$ has symbol $a \sim \sum_{i = 0}^\infty a_{m-i}$, then
    %
    \[ a^{\text{sp}} = a_{m-1} + C \cdot \partial_j^x \partial_j^\xi a_m \]
    %
    TODO: GET COEFFICIENT $C$ CORRECT. is invariantly defined as a $\text{Hom}(\text{Vol}^{1/2}(M))$ valued symbol of order $m-1$ on $T^*M$.
    %Indeed, if we switch to some other coordinates via some diffeomorphism $\kappa$, giving a new classical family of symbols $b \sim \sum_{i = 0}^\infty b_{m-i}$, then
    %
    %\[ b_0(\kappa(x),\eta) = a_0(x, D\kappa(x)^T \eta) \]
    %\[ b_1(\kappa(x),\eta) = \sum_{i = 1}^n \frac{1}{2\pi i} (\partial_\xi^i a_0)(x, D\kappa(x)^T \eta) + \frac{1}{4 \pi i} \sum_{j \leq k} (\partial_\xi^{jk} a_0)(x, D\kappa(x)^T \eta) \{ 2 \pi i \eta \cdot \partial_y^{jk} r_x(y) \}|_{y = x} \]
    In particular, if $a_0$ vanishes up to second order, the $a_{m-1}$ is invariantly defined on $T^*M$.
\end{example}





\section{The Index Theorem}

Let $T \in \loc{\Psi^0}(\Omega)$. Then $T$ is a continuous operator from $L^2_c(\Omega)$ to $L^2_{\text{loc}}(\Omega)$. We begin this section by determining what conditions ensure that this operator is compact, i.e. what conditions ensure that $T$ maps bounded subsets of $L^2_c(\Omega)$ to bounded subsets of $L^2_{\text{loc}}(\Omega)$. This is equivalent to the induced maps $L^2(K) \to L^2_{\text{loc}}(\Omega)$ being compact for any compact set $K \subset \Omega$, and our job is simplified considerably by the following lemma.

\begin{lemma}
    Let $X$ be a Banach space, and let $Y$ be a Fr\'{e}chet space. Then a bounded linear operator $T: X \to Y$ is compact if and only if for any sequence $\{ x_i \}$ converging weakly to zero in $X$, $\{ Tx_i \}$ converges in the standard topology of $Y$.
\end{lemma}
\begin{proof}
    TODO: Maybe move to functional analysis notes?

    Since $Y$ is a Fr\'{e}chet space, $T$ is compact if and only if for any bounded sequence $\{ x_i \}$ in $X$, $\{ Tx_i \}$ has a convergent subsequence in $Y$.

    If $T$ is compact, and a sequence $\{ x_i \}$ converges weakly to zero, then that sequence is bounded, hence $\{ Tx_i \}$ is precompact. But $\{ Tx_i \}$ converges weakly to zero, since $T$ is continuous from the weak topology on $X$ to the weak topology on $Y$. But this means $\{ Tx_i \}$ converges in norm to zero, since any subsequence has a further subsequence converging in norm to zero.

    Conversely, suppose $T$ maps a sequence converging weakly to zero to a sequence converging in norm. If $\{ x_i \}$ is a bounded sequence in $X$, then by Banach-Alaoglu, there exists a subsequence of $\{ x_i \}$ which is Cauchy in the weak topology, which without loss of generality we will assume to be $\{ x_i \}$ itself. To show this implies $\{ Tx_i \}$ converges in $Y$, we note that if $d_Y$ is a translation invariant metric defining $Y$, and $\{ Tx_i \}$ did not converge, then there would be $\varepsilon > 0$ and a strictly increasing pair of sequences $\{ i_j \}$ and $\{ i'_j \}$ such that for all $j$, $d_Y(Tx_{i_j}, Tx_{i'_j}) \geq \varepsilon$. But $x_{i_j} - x_{i'_j}$ converges to zero in the weak topology, which gives a contradiction.
\end{proof}

Thus an operator $T: L^2_c(\Omega) \to L^2_{\text{loc}}(\Omega)$ is compact if and only if, for any compact set $K \subset \Omega$, and any sequence $\{ f_i \}$ in $L^2(K)$ converging weakly to zero, $\{ Tf_i \}$ converges to zero in $L^2_{\text{loc}}(\Omega)$, which means that for any other compact set $K_1 \subset \Omega$,
%
\[ \lim_{i \to \infty} \int_{K_1} |Tf_i(x)|^2\; dx = 0. \]
%
Any operator TODO-

Any smoothing operator $T: L^2_c(\Omega) \to L^2_{\text{loc}}(\Omega)$ is compact, since if $K \in C^\infty(\Omega \times \Omega)$, then it is certainly true that $K \in $

The dual of $L^2_c(\Omega)$ can be identified with $L^2_{\text{loc}}(\Omega)$. Thus a family $\{ f_i \}$ converges to zero weakly in $L^2_c(\Omega)$ if and only if
%
\[ \int f_i(x) g(x)\; dx \to 0 \]

A family $\{ f_i \}$ in $L^2_c(\Omega)$ converges weakly to zero precisely when it converges in $L^2_{\text{loc}}(\Omega)$ to zero.

Thus an operator $T: L^2_c(\Omega) \to L^2_{\text{loc}}(\Omega)$ is compact if and only if 







\section{Self-Adjoint Elliptic Pseudo-Differential Operators on a Compact Manifold}

Given a scalar density $d\omega$ on a manifold $M$, we can consider $L^2(M)$ as a Hilbert space induced by the inner product
%
\[ \langle u, v \rangle = \int u(x) \overline{v(x)}\; d\omega. \]
%
Thus given a scalar density we can consider the adjoint of a pseudodifferential operator on $M$ by considering the operator as an unbounded operator on $L^2(M)$ with domain $H^k_c(M)$, and in particular, we can consider \emph{self-adjoint unbounded} operators on $M$; we are being somewhat lax with terminology here, by a self-adjoint operator we mean one which is self adjoint on $\DD(\RR^n)$ with the natural pairing, i.e. a formally self-adjoint operator. A self-adjoint unbounded operator is then an unbounded operator self-adjoint as defined in the theory of such operators. If $T$ is a classical self-adjoint pseudodifferential operator of order $t$, then it follows from the calculus of pseudodifferential operators that it's principal symbol is real-valued. If $T$ is self-adjoint and also \emph{elliptic}, and $M$ is connected and has dimension bigger than one, it also follows that the principal symbol is either positive everywhere in $T^* \RR^n - 0$, or negative everywhere for large $\xi$. Thus we might as well assume in what follows that the principal symbol is positive. We begin with a calculation which tells us we are safe to use the spectral calculus for unbounded operators for the class of operators considered here provided we are working on a compact manifold, so that $H^k_c(M) = H^k(M)$.

\begin{theorem}
    Suppose $T$ is a self-adjoint pseudodifferential operator of order $t$ on a compact manifold $M$. Then $T$ is a closed, self-adjoint unbounded operator.
\end{theorem}
\begin{proof}
    Let us begin by showing $T$ is closed. We begin by noting that $T$ is a continuous operator from $\DD(M)^*$ to itself. We desire to show
    %
    \[ \{ (u,v) \in H^t(M) \times L^2(M) : Tu = v \} \]
    %
    is closed in $L^2(M) \times L^2(M)$. If $\{ u_i \}$ is a sequence in $H^t(M)$ converging in the $L^2(M)$ norm to some $u \in L^2(M)$, and $\{ Tu_i \}$ converges in the $L^2(M)$ norm to some $v \in L^2(M)$, then $\{ u_i \}$ converges distributionally to $u$, and so $Tu = v$. Applying elliptic regularity, since $v \in L^2(M)$, we conclude that $u \in H^t(M)$. Thus we have shown $T$ is closed.

    Next, we show that $T$ is a self-adjoint unbounded operator. To do this, since $T$ is already self-adjoint, it suffices to show that if $v \in L^2(M)$ induces an inequality of the form
    %
    \[ |\langle Tu, v \rangle| \lesssim \| u \|_{L^2(M)} \]
    %
    for all $u \in H^t(M)$, then $v \in H^t(M)$. But this again follows by elliptic regularity. Indeed, the inequality then holds for $u \in C^\infty(M)$, and the distributional theory thus implies that for such $u$,
    %
    \[ \langle Tu, v \rangle = \langle u, Tv \rangle. \]
    %
    Thus we have an inequality of the form
    %
    \[ |\langle u, Tv \rangle| \lesssim \| u \|_{L^2(M)} \]
    %
    for all $u \in C^\infty(M)$, and a density argument thus shows that $Tv \in L^2(M)$, so that $v \in H^t(M)$.
\end{proof}

We begin our analysis by focusing on the case where $T$ has \emph{order one}, though we will later find many question about elliptic self-adjoint operators of other orders reduce to this question. If the principal symbol of this elliptic, self-adjoint classical operator $T$ is positive away from the origin, one might expect $T$ to be almost positive semidefinite, in some sense.

\begin{lemma}
    Let $T$ be a classical self-adjoint elliptic pseudodifferential operator of order $1$. Then there exists a classical self-adjoint elliptic pseudodifferential operator $S$ of order $1/2$ such that $T - S^* S$ is a smoothing operator.
\end{lemma}
\begin{proof}
    We proceed as in the proof that $\sqrt{-\Delta}$ is a classical pseudodifferential operator. Let $S_0$ be a classical self-adjoint pseudodifferential operator of order $1/2$ with principal symbol $b_0(x,\xi) = a(x,\xi)^{1/2}$, then $S_0^* S_0$ is positive-semidefinite, and $T - S_0^* S_0$ is a classical pseudodifferential operator of order zero. Let $r_1$ denote the principal symbol of this operator. More generally, let us suppose that $T - (S_0 + \dots + S_n)^* (S_0 + \dots + S_n)$ is a classical pseudodifferential operator of order $-n$, with principal symbol $r_n$. If we consider a classical psuedodifferential operator $S_{n+1}$ of order $1/2 - (n+1)$ with principal symbol $b_{n+1}(x,\xi) = r_n(x,\xi) / 2 b_0(x,\xi)$, then $T - (S_0 + \dots + S_{n+1})^* (S_0 + \dots + S_{n+1})$ is a classical pseudodifferential operator of order $-(n+1)$. Thus if we choose a self-adjoint operator $S$ of order $1/2$ such that $S \sim \sum_{i = 0}^\infty S_i$, then $T - S^* S$ is smoothing. 
\end{proof}

\begin{lemma}
    Suppose $T$ is a classical self-adjoint elliptic pseudodifferential operator of order $1$ with a non-negative principal symbol. Then for sufficiently large $\lambda_0 > 0$, we have that for $u \in H^1_c(M)$,
    %
    \[ \langle (T + \lambda_0) u, u \rangle \sim \| u \|_{H^{1/2}(M)}^2, \]
    %
    where the implicit constants are locally uniform over $u$ supported on a common compact subset of $M$.
\end{lemma}

\begin{proof}
    Without loss of generality, we may assume that the support of the kernel of $T$ is compact. Choose a classical, self-adjoint, elliptic pseudodifferential operator $S$ of order $1/2$ such that $T - S^* S$ is smoothing. Then for $u \in H^1_c(M)$, there exists $\gamma_0 > 0$ such that
    %
    \begin{align*}
        \left| \langle Tu, u \rangle - \| Su \|_{L^2(M)}^2 \right| &= \left| \int_M (T - S^* S) u(x) \overline{u}(x)\; dx \right| \leq \gamma_0 \| u \|_{L^2(M)}^2.
    \end{align*}
    %
    Thus if $\gamma > \gamma_0$,
    %
    \[ \langle (T + \gamma) u, u \rangle = \langle Tu, u \rangle + \gamma \| u \|_{L^2(M)} \geq \| Su \|_{L^2(M)}^2 + (\gamma - \gamma_0) \| u \|_{L^2(M)}. \]
    %
    Thus we conclude that $T + \gamma$ is positive definite. Since $S$ has principal symbol $a(x,\xi)^{1/2}$, it is elliptic, and so for $u \in H^{1/2}(M)$, if $\gamma > \gamma_0$.
    %
    \begin{align*}
        \| u \|_{H^{1/2}(M)}^2 &\lesssim \| Su \|_{L^2(M)}^2 + \| u \|_{L^2(M)}^2\\
        &\lesssim \left| \langle Tu, u \rangle \right| + \| u \|_{L^2(M)}^2\\
        &= |\langle (T + \gamma) u, u \rangle - \langle \gamma u, u \rangle | + \| u \|_{L^2(M)}^2\\
        &\leq \langle (T + \gamma) u, u \rangle + \langle \gamma u, u \rangle + \| u \|_{L^2(M)}^2\\
        &\leq \langle (T + 2\gamma + 1) u , u \rangle.
    \end{align*}
    %
    Thus if $\lambda_0 > 2 \gamma_0 + 1$, then
    %
    \[ \langle (T + \lambda_0) u, u \rangle \lesssim \| Su \|_{L^2(M)} + \| u \|_{L^2(M)} \lesssim \| u \|_{H^{1/2}(M)}. \qedhere \]
\end{proof}

\begin{theorem}
    Suppose $T$ is a classical, self-adjoint pseudodifferential operator of order one on a compact manifold $M$, with a non-negative principal symbol. Then for suitably large $\lambda_0 > 0$,
    %
    \begin{itemize}
        \item The operators $(T + \lambda_0)^k$ are isomorphisms from $H^k(M)$ to $L^2(M)$ for all $k > 0$.

        \item There exists an increasing family of real numbers $\{ \lambda_i \}$ with $\lambda_i \to \infty$, and an orthogonal basis $\{ e_i \}$ for $L^2(M)$ consisting solely of elements of $C^\infty(M)$, such that $Te_i = \lambda_i e_i$ for each $i > 0$. The functions $\{ e_i \}$ satisfy
        %
        \[ \| e_i \|_{H^k(M)} \sim_k (\lambda_0 + |\lambda_i|)^k \]

        \item If $u \in H^k(M)$, then $\langle u, e_i \rangle \lesssim_k \| u \|_{H^k(M)} (\lambda_0 + \lambda_i)^{-k}$.
    \end{itemize}
\end{theorem}
\begin{proof}
    The last result implies that $T + \lambda_0$ is an injective operator from $H^1(M)$ to $L^2(M)$ for sufficiently large $\lambda_0 > 0$. We have already seen that, since $M$ is compact, $T + \lambda_0$ is a closed, self-adjoint unbounded operator on $L^2(M)$. But the theory of unbounded operators thus implies that because $T + \lambda_0$ is injective, it has dense image in $L^2(M)$. Thus we can define it's inverse $R = (T + \lambda_0)^{-1}$, which is a densely defined continuous operator in $L^2(M)$, and thus extends to a continuous operator on all of $L^2(M)$. Because $T + \lambda_0$ is positive-definite, i.e. $\langle (T + \lambda_0) u, u \rangle > 0$ for all non-zero $u \in H^1(M)$, density implies that $\langle Rv, v \rangle > 0$ for all $v \neq 0$. Thus $R$ is positive definite. If $u \in H^1(M)$, and $(T + \lambda_0) u = v$, i.e. $u = Rv$, then ellipticity implies that
    %
    \[ \| Rv \|_{H^1(M)} \lesssim \| v \|_{L^2(M)} + \| Rv \|_{L^2(M)} \lesssim \| v \|_{L^2(M)}. \]
    %
    Conversely, continuity of pseudodifferential operators implies that
    %
    \[ \| v \|_{L^2(M)} = \| (T + \lambda_0) u \|_{L^2(M)} \lesssim \| u \|_{H^1(M)}. \]
    %
    Thus density implies that $R$ is an \emph{isomorphism} from $L^2(M)$ to $H^1(M)$. In particular we note that that $T + \lambda_0$ is an isomorphism from $H^1(M)$ to $L^2(M)$, i.e. it was surjective in the first place rather than merely having dense image. Moving on, The Rellich-Kondrachov theorem thus implies that $R$ is a \emph{compact}, positive definite operator on $L^2(M)$. Thus the spectral analysis of such operators implies that there exists an orthogonal basis $\{ e_i \}$ and a family of positive, decreasing eigenvalues $\{ \gamma_i \}$ such that if $E_i$ is orthogonal projection onto the span of $e_i$, then $(T + \lambda_0)^{-1} = \sum_i \gamma_i^{-1} E_i$. If $\lambda_i = \gamma_i - \lambda_0$, then this means that $T = \sum_i \lambda_i E_i$. 

    Applying elliptic regularity, since $(T + \lambda_0) e_i = \gamma_i e_i$ lies in $L^2(M)$, we conclude that $e_i \in H^1(M)$. But iterating this argument gives that $e_i \in H^N(M)$ for any $N > 0$. Applying the Sobolev embedding theorem thus shows that $e_i \in C^\infty(M)$ for each $i$.

    A simple generalization of the argument above (the spectral theory, ellipticity, and so on) shows that $(T + \lambda_0)^k$ is actually an isomorphism from $H^k(M)$ to $L^2(M)$ for each $k > 0$. We thus find that for any $u \in H^k(M)$,
    %
    \[ \| u \|_{H^k(M)}^2 \sim_k \| (T + \lambda_0)^k u \|_{L^2(M)}^2 = \sum (\lambda_i + \lambda_0)^{2k} |\langle f, e_i \rangle|^2. \]
    %
    Thus we find that
    %
    \[ |\langle u, e_i \rangle| \lesssim_k \| u \|_{H^k(M)} |\lambda_i + \lambda_0|^{-k} \lesssim_k \| u \|_{H^k(M)} |1 + \lambda_0|^{-k}. \qedhere \] 
    %which we briefly describe the details to. It is verified to be an injective, closed, self-adjoint map from $H^k(M)$ to $L^2(M)$, and thus has dense image, and a unique continuous left inverse $R_k: L^2(M) \to L^2(M)$. Ellipticity implies that if $u \in H^k(M)$, and $(T + \lambda_0)^k u = v$, with $v \in L^2(M)$, then $u = R_k v$, and
    %
    %\[ \| R_k v \|_{H^k(M)} \lesssim \| v \|_{L^2(M)} + \| R_k v \|_{L^2(M)} \lesssim \| v \|_{L^2(M)} \]
    %
    %and the continuity of pseudodifferential operators imples
    %
    %\[ \| v \|_{L^2(M)} \lesssim \| u \|_{H^k(M)}. \]
    %
    %Thus we conclude that
\end{proof}

The following, known as \emph{H\"{o}rmander's square root trick}, allows us to extend the theory above to elliptic, self-adjoint operators of arbitrary order.

\begin{theorem}
    Fix integers $p$ and $q$ with $q \neq 0$, and set $t = p/q$. Suppose $T$ is a classical, positive-definite, elliptic pseudodifferential operator of order $t$ on a compact manifold $M$. Then the operator $T^{q/p}$, defined using the spectral functional calculus, is a classical, positive-definite, elliptic pseudodifferential operator of order $1$, with principal symbol $a(x,\xi)^{q/p}$.
\end{theorem}
\begin{proof}
    The case $t = -1$ is easily seen to be true, since the assumptions imply $T$ is invertible, and it's inverse is a pseudodifferential operator modulo smoothing. Thus, without loss of generality, we may assume that $q = 1$, and that $p > 0$. We can certainly \emph{construct} an operator $S \in \Psi^1_{\text{cl}}(M)$ which is self-adjoint, with principal symbol $a(x,\xi)^{1/p}$, such that $S^p - T$ is a smoothing operator. Applying the theory above, we see that $S$ has an eigenvalue decomposition, with eigenvectors lying in $C^\infty(M)$, and only finitely many eigenvalues being negative. Thus, modulo a smoothing operator, i.e. a finite sum of the projection operators $\{ E_i \}$, we may actually replace $S$ with a positive-definite operator. We claim that $S - T^{1/p}$ is also a smoothing operator, which would complete the proof since the principal symbol of $S$ is equal to $a(x,\xi)^{q/p}$. Since $S$ and $T$ are both positive-definite, they are injective, and so by the spectral calculus of unbounded operators, $T^{-1}$ and $S^{-1}$ is a well-defined bounded, positive-definite operator on $L^2(M)$. If $\gamma$ is a contour in the complex plane which winds around each element of $\sigma(T^{-1})$ exactly once, then we find using the holomorphic functional calculus that
    %
    \[ T^{1/p} = \frac{1}{2\pi i} \int_\gamma z^{-1/p} (z - T^{-1})^{-1}\; dz \]
    %
    and
    %
    \[ S = \frac{1}{2 \pi i} \int_\gamma z^{-1/p} (z - S^{-p})^{-1}\; dz \]
    %
    Now $S^{-p}$ is a pseudodifferential operator of order $-p$. Since $S^p - T$ is a smoothing operator, the composition calculus tells us that so is $T^{-1} (S^p - T) S^{-p} = T^{-1} - S^{-p}$. Write $S^{-p} = T^{-1} - R$. Then
    %
    \begin{align*}
        S - T^{1/p} &= \frac{1}{2 \pi i} \int_\gamma z^{-1/p} \left( (z - T^{-1} + R)^{-1} - (z - T^{-1})^{-1} \right)\\
        &= \frac{1}{2 \pi i} \int_\gamma z^{-1/p} \left( (z - T^{-1} + R)^{-1} R (z - T^{-1})^{-1} \right).
    \end{align*}
    %
    This is a integral over a compact curve over a continuous family of smoothing operators, which is therefore a smoothing operator.
\end{proof}

\begin{remark}
    As above, finding an operator $S$ of order $t/2$ such that $T - S^*S$ is smoothing allows one to show that if $T$ is a classical, elliptic, self-adjoint pseudodifferential operator of order $t$, then $T + \lambda_0$ is positive definite for suitably large $\lambda_0 > 0$, so that we may apply the theorem above to a wider situation.
\end{remark}

Using the trick, we quickly extend the theory above to general operators of this form.

\begin{theorem}
    Suppose $T$ is a classical, self-adjoint, elliptic pseudodifferential operator of order $t = p/q$ on a compact manifold $M$. Then for sufficiently large $\lambda_0 > 0$,
    %
    \begin{itemize}
        \item $(T + \lambda_0)^k$ is an isomorphism from $H^{tk}(M)$ to $L^2(M)$ for all $k > 0$.
        \item There is a basis $\{ e_i \}$ for $L^2(M)$ consisting of elements of $C^\infty(M)$ and an increasing sequence of positive real numbers $\{ \lambda_i \}$ with $\lambda_i \to \infty$, such that $Te_i = (\lambda_0 + \lambda_i)^t e_i$. The functions $\{ e_i \}$ satisfy
        %
        \[ \| e_i \|_{H^k(M)} \sim_k (\lambda_0 + |\lambda_i|)^k \]
        \item If $u \in H^k(M)$, then $|\langle u, e_i \rangle| \lesssim_k \| u \|_{H^k(M)} (\lambda_0 + |\lambda_i|)^{-k}$.
    \end{itemize}
\end{theorem}

\begin{example}
    Let us explicitly check the result of this theorem in a well studied example. The operator $-\Delta$ is a classical positive \emph{semi} definite elliptic pseudodifferential operator of order 2 on $\TT^n = \RR^n / \ZZ^n$. The theorem above, applied to $\alpha - \Delta$ for each $\alpha > 0$, shows that there is a basis $\{ e_i \}$ for $L^2(\TT^n)$, and a sequence of non-negative real numbers $\{ \lambda_i \}$, such that $\Delta e_i = - \lambda_i^2 e_i$, and that if $u \in H^k(\TT^n)$, then
    %
    \[ |\langle u, e_i \rangle| \lesssim_k \| u \|_{H^k(M)} (1 + \lambda_i)^{-k}. \]
    %
    In this case, we can explicitly compute the basis $\{ e_i \}$. It is \emph{precisely} the functions $e_\xi(x) = e^{2 \pi i \xi \cdot x}$, where $\xi \in \ZZ^n$, with corresponding eigenvalue $\lambda_\xi = 2 \pi |\xi|$ since $\Delta e_\xi = -4 \pi^2 |\xi|^2$. In this case the Fourier transform explicitly tells us the slightly stronger identity
    %
    \[ |\langle u, e_\xi \rangle| = |\widehat{u}(\xi)| \lesssim_k \min \left( \| u \|_{L^1(\TT^n)}, \| D^k u \|_{L^1(\TT^n)} |\xi|^{-k} \right). \]
\end{example}

\section{Distribution of Eigenvalues and the Half Wave Operator}

Consider the setup we began our study of in the last section. We have a classical, self-adjoint, elliptic pseudodifferential operator $T$ of order one on a compact manifold $M$ with principal symbol $p(x,\xi) \geq 0$, and we have associated with $T$ a sequence of orthonormal smooth functions $\{ e_i \}$ in $C^\infty(M)$ which form a basis for $L^2(M)$, and an increasing sequence $\{ \lambda_i \}$ tending to $\infty$, such that $Te_i = \lambda_i e_i$. Our goal now is to further the study of the behaviour of the eigenfunctions $\{ e_i \}$, and the distributions of the eigenvalues $\{ \lambda_i \}$. A key object in the study of the distribution of the eigenvalues is the function $N(\lambda)$, which gives the number of eigenvalues of $T$ less than or equal to $\lambda$. Our results already give a very basic estimate for $N$, which we will considerably sharpen over the course of this section.

\begin{theorem}
    The function $N$ is tempered in $\lambda$. More precisely, for $\lambda > 0$,
    %
    \[ N(\lambda) \lesssim_\varepsilon \lambda^{n + \varepsilon}. \]
\end{theorem}
\begin{proof}
    Define the operators $S_\lambda$, which are the projections in $L^2(M)$ onto the eigenspaces corresponding to eigenvalues with value $\leq \lambda$. Then $S_\lambda$ clearly has kernel
    %
    \[ K_\lambda(x,y) = \sum_{\lambda_i \leq \lambda} e_i(x) \overline{e_i(y)}. \]
    %
    This function is smooth, and
    %
    \[ \int K_\lambda(x,x)\; dx = \sum_{\lambda_i \leq \lambda} |e_i(x)|^2 = N(\lambda). \]    
    %
    Thus $N(\lambda)$ is the trace of the operator $S_\lambda$. Because $\| e_i \|_{H^k(M)} \lesssim_k (1 + |\lambda|)^k$ if $\lambda_i \leq \lambda$, we have the elementary estimate
    %
    \[ \| S_\lambda f \|_{H^k(M)} \lesssim_k \lambda^k \| f \|_{L^2(M)}. \]
    %
    The Sobolev embedding theorem thus implies that if $\dim(M) = d$, and if $\varepsilon > 0$, then
    %
    \[ \| S_\lambda f \|_{L^\infty(M)} \lesssim_\varepsilon \| S_\lambda f \|_{H^{d/2 + \varepsilon}(M)} \lesssim \lambda^{d/2 + \varepsilon} \| f \|_{L^2(M)}. \]
    %
    Thus we conclude that, for any $f \in C^\infty(M)$, and any $x \in M$,
    %
    \[ \left| \int K_\lambda(x,y) f(y)\; dy \right| \lesssim_\varepsilon \lambda^{d/2 + \varepsilon} \| f \|_{L^2(M)}. \]
    %
    By density, this implies that if $g_{\lambda,x}(y) = K_\lambda(x,y)$, then
    %
    \[ \| g_{\lambda,x} \|_{L^2(M)} \lesssim_\varepsilon \lambda^{d/2 + \varepsilon}. \]
    %
    But $g_{\lambda,x} \in \text{span} \{ e_i: \lambda_i \leq \lambda \}$, so this means that
    %
    \[ \| g_{\lambda,x} \|_{L^\infty(M)} = \| S_\lambda g_{\lambda,x} \|_{L^\infty(M)} \lesssim_\varepsilon \lambda^{d/2 + \varepsilon} \| g_{\lambda,x} \|_{L^2(M)} \lesssim_\varepsilon \lambda^{d + 2 \varepsilon}. \]
    %
    But this means we conclude that $\| K_\lambda \|_{L^\infty(M \times M)} \lesssim_\varepsilon \lambda^{d + \varepsilon}$, and thus that
    %
    \[ N(\lambda) = \int K_\lambda(x,x)\; dx \lesssim_\varepsilon \lambda^{d + \varepsilon}. \qedhere \]
\end{proof}

Our next goal is now to establish the \emph{Sharp Weyl formula}. If
%
\[ c = \int_{\{ (x,\xi) \in T^*M : a(x,\xi) \leq 1 \}} d\xi\; dx, \]
%
then the result states that
%
\[ N(\lambda) = c \lambda^n + O(\lambda^{n-1}). \]
%
We will continue to use the representation $N(\lambda) = \text{Tr}(S_\lambda)$. However, a key step here is to recognize that $S_\lambda$ is a \emph{function} of $T$, namely $S_\lambda = \chi_\lambda(T)$, where $\chi_\lambda(\xi) = \mathbf{I}(\xi \leq \lambda)$. The next trick is to use the Fourier multiplication formula to obtain a useful representation of the structure of $S_\lambda$. For $f \in C^\infty(M)$, we can write
%
\begin{align*}
    S_\lambda f &= \int_{-\infty}^\infty \chi_\lambda(\tau) \left( \sum_i \delta_{\lambda_i}(\tau) E_i f \right)\; d\tau\\
    &= \int_{-\infty}^\infty \widehat{\chi}_\lambda(t) \left( \sum_i e^{2 \pi i t \lambda_i} E_i f \right)\; dt\\
    &= \int_{-\infty}^\infty \widehat{\chi}_\lambda(t) e^{2 \pi i t T}\; dt\\
    &= \frac{1}{2 \pi i} \int_{-\infty}^\infty \frac{e^{2 \pi i t (T - \lambda)} f}{t + i0}\; dt.
\end{align*}
%
Thus we are lead to study the \emph{half wave operator} $\partial_t - 2 \pi i T$, since one sees that $(\partial_t - 2 \pi i T) \{ e^{2 \pi i t T} f \} = 0$ for all $f \in L^2(M)$; it is easily verified for the eigenfunctions of $T$, and then the result follows by the distributional continuity of all operations involved.

We will understand the operators $e^{2 \pi i t T}$ by constructing a \emph{parametrix} for the equation $\partial_t - 2 \pi i T$ over small times, i.e. for $|t| \lesssim 1$. This equation is a pseudodifferential variant of a hyperbolic partial differential equation, so based on the parametrix construction in this scenario, we might hope to find a parametrix defined by an oscillatory integral of the form
%
\[ Sf(x,t) = S(t)f(x) = \int s(t,x,y,\xi) e^{2 \pi i \Phi(t,x,y,\xi)} f(y)\; dy\; d\xi, \]
%
such that:
%
\begin{itemize}
    \item $\Phi(t,x,y,\xi) = \phi(x,y,\xi) + t p(y,\xi)$, where $\phi$ is smooth away from $\xi = 0$, homogeneous of degree one, and $\phi(x,y,\xi) \approx (x - y) \cdot \xi$ in the sense that on the support of $s$,
    %
    \[ \partial^\beta_\xi \{ \phi(x,y,\xi) - (x - y) \cdot \xi \} \lesssim_\beta |x - y|^2 |\xi|^{1 - \beta}. \]
    %
    In particular, this implies that $\phi(x,y,\xi) = 0$ when $(x - y) \cdot \xi = 0$.

    \item $s$ is a symbol of order zero, supported on $|x - y| \lesssim 1$ and on $|\xi| \geq 1$, in such a way that
    %
    \[ |\nabla_\xi \phi(x,y,\xi)| \gtrsim |x - y| \]
    %
    and
    %
    \[ |\nabla_x \phi(x,y,\xi)| \gtrsim |\xi| \]
    %
    for $(x,y) \in \text{supp}_x(s) \times \text{supp}_y(s)$.
\end{itemize}
%
If $(\partial_t - 2 \pi i T) \circ S = 0$ is a smoothing operator on $(-\varepsilon,\varepsilon) \times M$ and $S(0)$ differs from the identity operator by a smoothing operator, then $S - e^{2 \pi i t T}$ is smoothing for $|t| \leq \varepsilon$. This will therefore give us much more information about the behaviour of the propogators $e^{2 \pi i t T}$ over small times.

To find a choice of $\phi$ and $s$ which gives us this parametrix, let us start by determining what properties these functions should satisfy. Let us fix a coordinate system $(x,U)$, where $x(U)$ is a precompact subset of $\RR^n$. Let us assume that in these coordinates, $T$ has symbol $a(x,\xi)$. Then the kernel of $(\partial_t - 2 \pi i T) \circ S$ in this coordinate system is
%
\[ \int (\partial_t + 2 \pi i T(x,D)) \left\{ s(t,\cdot,y,\xi) e^{2 \pi i \Phi(t,\cdot,y,\xi)} \right\}\; d\xi. \]
%
If we set
%
\[ s'(t,x,y,\xi) = e^{- 2 \pi i \Phi(t,x,y,\xi)} (\partial_t - 2 \pi i T(x,D)) \left\{ s(t,\cdot,y,\xi) e^{2 \pi i \Phi(t,\cdot,y,\xi)} \right\} \]
%
then the kernel is
%
\[ \int s'(t,x,y,\xi) e^{2 \pi i \Phi(t,x,y,\xi)}\; d\xi. \]
%
Provided that $s'$ is a symbol of order $-\infty$ for $0 < |t| \leq \varepsilon$, integration by parts shows that $(\partial_t + 2 \pi i T) \circ S$ is smoothing, and so we will try to choose $\phi$ and $s$ so as to obtain such a result.

In our discussion of pseudodifferential operators, we have already discussed an asymptotic formula for $s'$, namely, if
% Phi(t,x,y,xi) = phi(x,y,xi) - t p(y,xi)
% phi(x,y,xi) = (x - y) * xi + O(|x-y|^2 |xi|)
% nabla_x phi = xi + nabla_x O( |x - y|^2 |xi| )
%
\begin{align*}
    r_{x,y}(z) &= \nabla_x \phi(x,z,\xi) \cdot (x - z) - \{ \phi(x,y,\xi) - \phi(z,y,\xi) \}.
\end{align*}
%
then for any $N > 0$, if $a \sim \sum_{k = -\infty}^1 a_k$, where $a_k$ is homogeneous of degree $k$, and if $\xi_\phi = \nabla_x \Phi(t,x,y,\xi) = \nabla_x \phi(x,y,\xi)$,
% Symbol of (partial_t + 2 pi i T) is
% 2 pi i (tau + a(x,xi))
\begin{align*}
    s' & (t,x,y,\xi)\\
    &= \underbrace{ \left( p(y,\xi) - a(x, \xi_\phi) \right) \cdot s(t,x,y,\xi) }_{\text{symbols of order 1}}\\
    &\quad + \underbrace{\partial_t s(t,x,y,\xi)}_{\text{symbol of order $0$}}  \\
    &\quad - \sum_{1 \leq |\beta| < N} \underbrace{\frac{2 \pi i}{\beta! \cdot (2 \pi i)^\beta} \cdot \partial_\xi^\beta a(x, \xi_\phi) \partial_z^\beta \{ e^{2 \pi i r_{x,y}(z)} s(t,z,y,\xi) \} |_{z = y}}_{\text{symbols of order $1 - \lceil |\beta| / 2 \rceil$}}\\
    &\quad + R_N(t,x,y,\xi).
\end{align*}
%
where, because $|\nabla_x \Phi(t,x,y,\xi)| \gtrsim |\xi|$ on the support of $s$,
%
\[ \langle \xi \rangle^{t - \lceil N/2 \rceil} R_N \in L^\infty((-\varepsilon,\varepsilon) \times U \times U \times \RR^d). \]
%
It is simple to  establish estimates of the form
%
\[ |\partial_x^\alpha \partial_y^\beta \partial_\xi^\lambda s'(t,x,y,\xi)| \lesssim \langle \xi \rangle^{N_{\alpha \beta \lambda}}. \]
%
Thus if we can justify that $|s'(t,x,y,\xi)| \lesssim_N \langle \xi \rangle^{-N}$ for all $N > 0$, then it will follow that $s'$ is a symbol of order $-\infty$. We now determine the propoerties of the symbol $s$ and the symbol $\phi$ which will give us these estimates.

To begin with, let us specify the function $\phi$. In order to guarantee that $s'$ is a symbol of order zero, the expansion above shows that $(p(y,\xi) - p(x,\xi_\phi)) \cdot s(t,x,y,\xi)$ must be a symbol of order zero. This will be true if we can pick $\phi$ such that, on the support of $s$, and for $|\xi| \gtrsim 1$,
%
\[ p(x, \nabla_x \phi(x,y,\xi)) = p(y,\xi). \]
%
This is an example of an \emph{Eikonal equation}, e.g. an equation of the form
%
\[ q(z,\nabla_z \psi(z)) = 0 \]
%
for some function $q(z,\zeta)$. In our case, $z = (x,y,\xi)$, so $\zeta = (\zeta_x, \zeta_y,\zeta_\xi)$, and so
%
\[ q(z,\zeta) = p(x,\zeta_x) - p(y,\xi). \]
%
Let us make some further remarks we desire about our choice of function $\phi$:
%
\begin{itemize}
    \item We want $\phi$ to be homogeneous and smooth away from the origin. If we solve the equation for all $|\xi| = 1$, and then extend $\phi$ such that for $\lambda > 0$ and $|\xi| = 1$,
    %
    \[ \phi(x,y,\lambda \xi) = \lambda \phi(x,y,\xi) \psi(\lambda), \]
    %
    where $\psi$ is smooth, equal to one for $|\lambda| \geq 3/4$, and vanishing for $|\lambda| \leq 1/2$, then $\phi$ will satisfy the equation for all $|\xi| \gtrsim 1$. This means that
    %
    \[ p(x,\nabla_x \phi(x,y,\xi)) - p(y,\xi) \]
    %
    is smooth and supported on $|\xi| \lesssim 1$, which implies it is a symbol of order $-\infty$, which suffices for our construction. Thus it suffices to solve the equation for $|\xi| = 1$.

    \item Since $\phi$ is smooth away from the origin and homogeneous, the equation
    %
    \[ |\partial_\xi^\beta \left\{ \phi(x,y,\xi) - (x - y) \cdot \xi \right\}| \lesssim_\beta |x - y|^2 \langle \xi \rangle^{1 - \beta} \]
    %
    holds if, for $|\xi| = 1$, we have $\phi(x,y,\xi) = 0$ whenever $(x - y) \cdot \xi = 0$, and $\nabla_x \phi(x,y,\xi) = \xi$ whenever $x = y$. Thus we have some \emph{initial conditions} for our Eikonal equation.
\end{itemize}
%
The second condition constitutes a type of initial condition for $\phi$, since it specifies it's behaviour on a hypersurface, a kind of Cauchy condition, and thus we should expect these are close to the conditions that give unique solutions to the equation. And the following Lemma indeed shows that there is a unique function $\phi$, defined for $|x - y| \lesssim 1$ and $|\xi| = 1$ with these properties.

\begin{lemma}
    Let $Z$ be a smooth manifold, and let $q(z,\zeta)$ be a real-valued, smooth function defined locally around a point $(z_0,\zeta_0) \in T^*Z$. Let $S$ be a smooth hypersurface in $Z$ passing through $z_0$ with conormal vector $\zeta_S$ at $z_0$, such that
    %
    \[ \frac{\partial q}{\partial \zeta_S}(z_0,\zeta_0) = \lim_{t \to 0} \frac{q(z_0,\zeta_0 + t \zeta_S) - q(z_0,\zeta_0)}{t} \]
    %
    is nonzero. Suppose that $\psi$ is any smooth function defined on $S$ locally about $z_0$, such that $d \psi(z_0)$ agrees with the action of $\zeta_0$ on $T_{x_0} S$. Then there exists a unique smooth function $\phi$ defined in a neighborhood of $z_0$, which agrees with $\psi$ on $S$, satisfies the Eikonal equation $q(z,\nabla_z \phi(z)) = 0$, and has $\nabla_z \phi(z_0) = \zeta_0$.
\end{lemma}
\begin{proof}
    TODO: See Sogge, Theorem 4.1.1.
\end{proof}

In our case,
%
\[ Z = \{ (x,y,\xi) : |\xi| = 1 \}. \]
%
We have $z_0 = (x_0,x_0,\xi_0)$, $\zeta_0 = (\xi,\xi,0)$, and
%
\[ S = \{ (x,y,\xi): |\xi| = 1 \quad\text{and}\quad (x - y) \cdot \xi = 0 \}. \]
%
The conormal vector $\xi_S$ of $S$ at $z_0$ is a multiple of $(\xi_0,-\xi_0,0)$, and so by homogeneity,
%
\[ \frac{\partial q}{\partial \xi_S} = \lim_{t \to 0} \frac{p(x_0,(1 + t)\xi_0) - p(x_0,\xi_0)}{t} = p(x_0,\xi_0), \]
%
which is nonvanishing because $T$ is elliptic. If we define $\psi$ equal to zero on $S$, then $d \psi = 0$, which agrees with the action of $\zeta_0$ on $S$. Thus the theorem applies local uniqueness and existence to solutions to the Eikonal equation, and by compactness of $Z$ we can patch such solutions together to find a solution defined for all $|x - y| \lesssim 1$.

We therefore conclude that there exists a unique choice of $\phi$ such that, if $s$ has small enough support, $s'(t,x,y,\xi)$ is a symbol of order zero. Next, let us see what constraints are forced on us in order to ensure that $S(0)$ differs from the identity by a smoothing operator. The kernel of $U$ is precisely
%
\[ \int s(0,x,y,\xi) e^{2 \pi i \phi(x,y,\xi)}\; d\xi. \]
%
We now show that this operator is actually a \emph{pseudodifferential operator} of order zero, and determine it's symbol up to first order.

To do this, we write $\phi_\alpha(x,y,\xi) = (1 - \alpha) \phi(x,y,\xi) + \alpha (x - y) \cdot \xi$. Let $U_\alpha$ be the operator with kernel
%
\[ \int s(0,x,y,\xi) e^{2 \pi i \phi_\alpha(x,y,\xi)}\; d\xi. \]
%
Assume the support of $s$ is close enough to the diagonal such that
%
\[ |\nabla_\xi \phi_\alpha(x,y,\xi)| \gtrsim |x - y| \]
%
on the support of $s$. Then $\partial_\alpha^n U_t$ has kernel
%
\[ \int (2 \pi i)^n ( \phi_1 - \phi_0 )^n s(0,x,y,\xi) e^{2 \pi i \phi_\alpha(x,y,\xi)}\; d\xi. \]
%
This is an oscillatory integral defined by a symbol of order $n$. However, when $t = 1$, the fact that $\phi(x,y,\xi) \approx (x - y) \cdot \xi$, together with the formula for converting pseudodifferential operators with compound symbols into standard Kohn-Nirenberg type symbols shows that $\partial_\alpha^n U_1$ is actually a pseudodifferential operator of order $-n$. Integration by parts, similarily, shows that $\partial_\alpha^n U_t$ is defined by an oscillator integral against a symbol of order $-n$. But this means that if we define a pseudodifferential operator by the asymptotic formula
%
\[ V \sim \sum \frac{(-1)^n}{n!} \partial_\alpha^n U_1, \]
%
then $U - V$ is smoothing. Indeed, for any $n$, by Taylor's formula we have
%
\[ U = \sum_{k = 0}^{n-1} \frac{(-1)^n}{n!} \partial_\alpha^n U_1 + \frac{(-1)^n}{n!} \int_0^1 \alpha^{n-1} \partial_\alpha^n U_\alpha\; d\alpha \]
%
The integral here is an oscillatory integral defined against a symbol of order $-n$, and thus taking $n \to \infty$ verifies the claim.

It is an important remark that reversing this argument shows that \emph{any} pseudodifferential operator can be written in the form above for the particular choice of $\phi$ we have given. This is a special case of the \emph{equivalence of phase functions} theorem. This in particular guarantees that we can choose a symbol $I(x,y,\xi)$ of order zero such that $U - 1$ is smoothing if and only if $s(0,x,y,\xi) - I(x,y,\xi)$ is a symbol of order $-\infty$. The symbol $I$ can be chosen to be vanishing for $|x - y| \gtrsim 1$, since the difference will be a smoothing pseudodifferential operator.

Next, the quantity
%
\begin{align*}
    &\partial_t s(t,x,y,\xi)\\
    &\quad\quad + \sum_{k = 1}^d \partial_\xi^k a(x,\xi_\phi) \partial_x^k s(t,x,y,\xi)\\
    &\quad\quad + \left( a_0(x,\xi_\phi) + \frac{1}{2 \pi} \sum_{|\beta| = 2} \partial_\xi^\beta p(x,\xi_\phi) \partial_x^\beta \phi(x,y,\xi) \right) s(t,x,y,\xi).
\end{align*}
%
must be a symbol of order $-1$. But because the coefficients of this equation are smooth, and all derivatives are bounded, it follows from the general theory of transport equations that there exists a unique smooth, function $s_0$ defined for $|t| \leq \varepsilon$, which is a symbol of order zero, such that $s_0(0,x,y,\xi) = I(x,y,\xi)$, $s_0$ vanishes for $|x - y| \gtrsim 1$, and satisfies the transport equation
%
\begin{align*}
    &\partial_t s_0(t,x,y,\xi)\\
    &\quad\quad + \sum_{k = 1}^d \partial_\xi^k a(x,\xi_\phi) \partial_x^k s_0(t,x,y,\xi)\\
    &\quad\quad + \left( a_0(x,\xi_\phi) + \frac{1}{2 \pi} \sum_{|\beta| = 2} \partial_\xi^\beta p(x,\xi_\phi) \partial_x^\beta \phi(x,y,\xi) \right) s_0(t,x,y,\xi) = 0.
\end{align*}
%
We have thus justified that the quantity
%
\[ R_0(t,x,y,\xi) = e^{-2 \pi i \Phi(t,x,y,\xi)} (\partial_t - 2 \pi i T)(s_0(t,\cdot,y,\xi) e^{2 \pi i \Phi(t,x,y,\xi)}) \]
%
is a symbol of order $-1$. Now we come to a quirk of this parametrix, which does not occur in the study of hyperbolic partial differential equations. Since the operator $P(x,D)$ is only \emph{pseudolocal} rather than completely local, the remainder term $R_0$ is \emph{not} necessarily supported on a neighborhood of the origin. To fix this, we now successively define the terms $\{ s_k \}$ for $k < 0$, which are symbols of order $-k$, such that $s_k(0,x,y,\xi) = 0$, and
%
\[ TODO: SPECIFY REQUIRED EQUATION. \]
%
Again, solutions exist for small time periods. And this implies that $e^{-2 \pi i \Phi(t,x,y,\xi)} (\partial_t + 2 \pi i T)((s_0 + \dots + s_{-k}) e^{2 \pi i \Phi(t,x,y,\xi)})$ is a symbol of order $-k$ (TODO: Is It), and we can continue the calculation to complete the argument.

% PROBLEMS WITH EXTENDING TO ALL TIMES
%       CANNOT ASSUME |x - y| << 1
%           - SO this means that oscillatory integrals become bad.

\begin{comment}
\begin{lemma}
    Consider an operator of the form
    %
    \[ Sf(x) = \int a(x,y,\xi) e^{2 \pi i \phi(x,y,\xi)} f(y)\; dy\; d\xi, \]
    %
    where $a \in S^r$ and vanishes for $|x - y| \gtrsim 1$, $\phi \in S^1$, and is homogeneous of degree one in $\xi$, $|\nabla_\xi \phi(x,y,\xi)| \gtrsim |x - y|$ on the support of $a$, and for all $r > 0$,
    %
    \[ \partial^\beta_\xi \{ \phi(x,y,\xi) - (x - y) \cdot \xi \} \lesssim_\beta |x - y|^2 |\xi|^{1-|\beta|}. \]
    %
    Then $S$ is well defined, and is actually a pseudodifferential operator of order $r$. If $T$ is the pseudodifferential operator with symbol $(x,\xi) \mapsto a(x,x,\xi)$, then $T - S$ is a pseudodifferential operator of order $r-1$.

    Conversely, for \emph{any} $\Psi DO$ $T$ of order $r$, there exists a symbol $a$ of order $r$ such that for the resulting operator $S$ of order $r$, $S - T$ is a smoothing operator.
\end{lemma}
\begin{proof}
    Let us first define the operator $S$. Let $\phi_0(x,y,\xi) = (x - y) \cdot \xi$, $\phi_1(x,y,\xi) = \phi(x,y,\xi)$, $\phi_t = t \phi_1 + (1 - t) \phi_0$, and define $S_t$ with the phase function $\phi_t$ and symbol $a$. Note that $|\nabla_\xi \phi_t| \gtrsim |x - y|$, uniformly in $t$. This enables us to compute the kernel $K_t(x,y)$ for $0 \leq t \leq 1$. For $t = 0$ we have a pseudodifferential operator, and for $t = 1$, we get the kernel $K(x,y)$ we get to compute. It is also simple to see that, since $|\nabla_\xi \phi_t| \gtrsim |x - y|$, that for large $N$,
    %
    \[ K(x,y)| \lesssim_N \frac{1}{|x - y|^N}, \]
    %
    so we already see that $S$ is somewhat pseudolocal.

    We have
    %
    \[ \frac{\partial^N K_t(x,y)}{\partial t^N} = (2 \pi i)^N \int (\phi_1(x,y,\xi) - \phi_0(x,y,\xi))^N a(x,y,\xi) e^{2 \pi i \phi_t(x,y,\xi)}\; d\xi. \]
    %
    Now $(\phi_1 - \phi_0)^N \cdot a$ is a symbol of order $r + N$. But on the other hand, using the fact that $(\phi_1 - \phi_0)^N \lesssim |x - y|^{2N} |\xi|^N$, and thus vanishes to order $2N$ on the diagonal, then combined with the fact that $|\nabla_\xi \phi(x,y,\xi)| \gtrsim |x - y|$, we actually see via an integration by parts $2N$ times in $\xi$ that we can rewrite the integral in terms of a symbol of order $r - N$ and the same phase $\phi_t$. Applying Taylor's theorem, we write
    %
    \[ K(x,y) = K_1(x,y) = \sum_{k = 0}^{N-1} \frac{1}{k!} \left. \frac{\partial^k K_t(x,y)}{\partial t^k} \right|_{t = 1} + \frac{1}{N!} \int_0^1 t^{N-1} \frac{d^NK_t(x,y)}{dt^N}\; dt. \]
    %
    This integral gives an arbitrarily smooth kernel as $N \to \infty$. Thus if we let $T$ be a pseudodifferential operator of order $r$ such that
    %
    \[ T \sim \sum_{k = 0}^\infty \frac{1}{k!} \left. \frac{\partial^k K_t(x,y)}{\partial t^k} \right|_{t = 1}, \]
    %
    then $T - S$ is a smoothing operator. Now if $\tilde{T}$ is the pseudodifferential operator corresponding to the symbol $a(x,x,\xi)$, then $T - \tilde{T}$, and thus $S - \tilde{T}$, is a pseudodifferential operator of order $r-1$. The converse is similar, working in the opposite direction, i.e. from $t = 1$ to $t = 0$, and is left as an exercise.
\end{proof}
\end{comment}

%Since $I$ is a $\Psi DO$ of order zero, we can find a symbol $a$ of order zero such that if $T$ is the operator with kernel
%
%\[ \int a(x,y,\xi) e^{2 \pi i \phi(x,y,\xi)}\; d\xi, \]
%
%then $T - I$ is a smoothing operator. To ensure that $S(0) - I$ is a smoothing operator, it is natural to insist that $s(0,x,y,\xi) = a(x,y,\xi)$. We note in particular that since $I$ is a $\Psi DO$ with $1$ as a symbol, this implies $s(0,x,x,\xi) - 1$ is a symbol of order $-1$.

Let us now use this equation to prove the sharp Weyl formula, which immediately is given by the following, more precise estimate.

\begin{theorem}
    Let $M$ be a compact manifold, and let $T$ be a classical, self-adjoint, elliptic pseudodifferential operator of order one on $M$, with principal symbol $p(x,\xi)$. If $S_\lambda$ is the projection operator onto the span of eigenfunctions with eigenvalue $\leq \lambda$, and $K_\lambda$ is the kernel of the operator $S_\lambda$, then
    %
    \[ K_\lambda(x,x) = c(x) \lambda^n + O(\lambda^{n-1}), \]
    %
    where
    %
    \[  c(x) = \int_{p(x,\xi) \leq 1}\; d\xi. \]
\end{theorem}
\begin{proof}
    Recall that we have
    %
    \[ S_\lambda f(x) = \frac{1}{2 \pi i} \int_{-\infty}^\infty \frac{e^{2 \pi i t (T - \lambda)} f}{t + i 0}\; dt. \]
    %
    This integral only has a singularity when $t = 0$, so it makes sense to decompose the integral into two parts using an even function $\rho \in C_c^\infty(\RR)$ supported on $|t| \leq \varepsilon / 2$ and equal to one for $|t| \leq \varepsilon / 4$, i.e. writing
    %
    \[ S_\lambda f = S_{\lambda, \text{low}} f + S_{\lambda,\text{high}} f, \]
    %
    where
    %
    \[ S_{\lambda, \text{low}} f = \frac{1}{2 \pi i} \int_{-\infty}^\infty \frac{\rho(t)}{t + i0} e^{2 \pi i t (T - \lambda)} f\; dt \]
    %
    and
    %
    \[ S_{\lambda,\text{high}} f = \frac{1}{2 \pi i} \int_{-\infty}^\infty \frac{1 - \rho(t)}{t} e^{2 \pi i t (T - \lambda)} f\; dt. \]
    %
    To understand $S_{\lambda, \text{low}} f$, we use the parametrix for the wave equation. If we switch to coordinates around a particular point, then we find that the kernel of $S_{\lambda, \text{low}}$ is equal to $\tilde{K}_\lambda + R_\lambda$, where $\tilde{K}_\lambda$ is supported on a neighborhood of the diagonal, and locally in coordinates we can write
    %
    \[ \tilde{K}_\lambda(x,y) = \frac{1}{2 \pi i} \int_{-\infty}^\infty \int_{\RR^n} \frac{\rho(t)}{t + i0} s(t,x,y,\xi) e^{2 \pi i (\phi(x,y,\xi) + t(p(y,\xi) - \lambda))}\; d\xi\; dt. \]
    %
    and where
    %
    \[ R_\lambda(x,y) = \frac{1}{2 \pi i} \int_{-\infty}^\infty \frac{\rho(t)}{t + i 0} A(t,x,y) e^{2 \pi i t \lambda}\; dt, \]
    %
    where $A$ is smooth for $|t| \leq \varepsilon$ and for $x,y \in M$. But since $\rho \cdot A$ is smooth, with compact support in the $t$ variable taking inverse Fourier transforms in the $t$ variable shows that
    %
    \[ |R_\lambda(x,y)| = \left| \int_{-\infty}^\lambda \mathcal{F}_t^{-1}\{\rho A\}(\lambda - \tau,x,y)\; d\tau \right| \lesssim 1, \]
    %
    By Taylor's formula, we can write $s(t,x,x,\xi) = s(0,x,x,\xi) + t \cdot r(t,x,\xi)$, where $r$ is a symbol of order zero. But if $N$ is sufficiently large, we have
    %
    \begin{align*}
        \frac{1}{2 \pi i} &\int_{-\infty}^\infty \int_{\RR^n} \frac{\rho(t)}{t + i0} (t \cdot r(t,x,\xi)) e^{2 \pi i t(p(x,\xi) - \lambda)}\; d\xi\; dt\\
        &= \frac{1}{2 \pi} \int_{-\infty}^\infty \int_{\RR^n} \rho(t) r(t,x,\xi) e^{2 \pi i t(p(x,\xi) - \lambda)}\; d\xi\; dt\\
        &= \frac{1}{2 \pi} \int_{\RR^n} \mathcal{F}^{-1}_t \{ \rho \cdot r \}( p(x,\xi) - \lambda ,x,\xi)\; d\xi\\
        &= \int O_N \left( 1 + |p(x,\xi) - \lambda| \right)^{-N}\; d\xi = O(\lambda^{n-1}).
    \end{align*}
    %
    We also have $s(0,x,x,\xi) = 1 + s_{-1}(x,\xi)$, where $s_{-1}$ is a symbol of order $-1$. Now if $\tilde{\chi}_\lambda = \chi_\lambda * \mathcal{F}_t^{-1} \{ \rho \}$, then
    %
    \[ \frac{1}{2 \pi i} \int_{-\infty}^\infty \int_{\RR^n} \frac{\rho(t)}{t + i0} s_{-1}(x,\xi) e^{2 \pi i t (p(x,\xi) - \lambda)}\; d\xi\; dt = \int_{\RR^n} \tilde{\chi}_\lambda(p(x,\xi)) s_{-1}(x,\xi)\; d\xi. \]
    %
    Then $\tilde{\chi}_\lambda(p(x,\xi)) \sim 1$ for $|\xi| \lesssim \lambda$, and for $|\xi| \gtrsim \lambda$,
    %
    \[ \tilde{\chi}_\lambda(p(x,\xi)) \lesssim_N \langle |\xi| - \lambda \rangle^{-N} \]
    %
    This implies that
    %
    \[ \left| \int_{\RR^n} \tilde{\chi}_\lambda(p(x,\xi)) s_{-1}(x,\xi)\; d\xi \right| \lesssim \int_0^{\lambda + O(1)} r^{n-2} + \int_{\lambda + O(1)}^\infty r^{n-1} (r - \lambda)^{-N} \lesssim \lambda^{n-1}. \]
    %
    Thus we have
    %
    \begin{align*}
        \tilde{K}_\lambda(x,x) &= \frac{1}{2 \pi i} \int_{-\infty}^\infty \int_{\RR^n} \frac{\rho(t)}{t + i0} e^{2 \pi i t (p(x,\xi) - \lambda)} + O(\lambda^{n-1})\\
        &= \int_{\RR^n} \tilde{\chi}_\lambda(p(x,\xi))\; d\xi + O(\lambda^{n-1})\\
        &= \int_{\RR^n} \tilde{\chi}(p(x,\xi) - \lambda)\; d\xi + O(\lambda^{n-1})\\
        &= c(x) \lambda^n + \int_{\RR^n} (\tilde{\chi} - \chi)(p(x,\xi) - \lambda)\; d\xi + O(\lambda^{n-1}).
    \end{align*}
    %
    Now $\tilde{\chi} - \chi = \chi * u$, where $u$ is the inverse Fourier transform of $\rho(t) - 1$. But then
    %
    \[ (\chi * u)(\tau) = \int_{-\infty}^0 \widehat{\rho}(\tau - \omega)\; d\omega - \mathbf{I}(\tau \geq 0), \]
    %
    from which we verify that
    %
    \[ |(\chi * u)(\tau)| \lesssim_N (1 + |\tau|)^{-N}. \]
    %
    But we can use this to again show that
    %
    \begin{align*}
        \left| \int_{\RR^n} (\tilde{\chi} - \chi)(p(x,\xi) - \lambda)\; d\xi \right| &\lesssim \int_0^{\lambda - O(1)} r^{n-1} |\xi|^{-N}\; dr\\
        &\quad + \int_{\lambda - O(1)}^{\lambda + O(1)} (\lambda - r)^{-N} r^{n-1}\; dr\\
        &\quad + \int_{\lambda + O(1)}^\infty r^{n-1} (r - \lambda)^{-N}\\
        &\lesssim \lambda^{n-1}.
    \end{align*}
    %
    Thus we have established the result for the kernel $K_{\lambda,\text{low}}$ of the operator $S_{\lambda,\text{low}}$. The proof of this theorem would be finished if we were able to justify that
    %
    \[ |K_\lambda(x,x) - K_{\lambda,\text{low}}(x,x)| \lesssim \lambda^{n-1}. \]
    %
    Define the function
    %
    \begin{align*}
        g(\lambda,x) &= K_\lambda(x,x) - K_{\lambda,\text{low}}(x,x)\\
        &= \frac{1}{2\pi i} \int_{-\infty}^\infty (1 - \rho(t)) \{ e^{-2 \pi i t (T - \lambda)}(x,x) \}\; dt.
    \end{align*}
    %
    Then the Fourier transform in the $\lambda$ variable is
    %
    \[ \widehat{g}(t,x) = \frac{1}{2 \pi i} (1 - \rho(t)) e^{-2 \pi i t T}. \]
    %
    Thus the support of $\widehat{g}$ is supported away from $|t| \leq \varepsilon / 2$. A Lemma of Tauberian type following this proof shows that it suffices to show that for $\lambda > 0$, and $0 < \tau \leq 1$,
    %
    \[ |g(\lambda + \tau,x) - g(\lambda + \tau,x)| \lesssim (1 + \lambda)^{n-1} \]
    %
    But this follows if we can show that
    %
    \[ |K_{\lambda + \tau}(x,x) - K_\lambda(x,x)| \lesssim (1 + \lambda)^{n-1} \]
    %
    and that
    %
    \[ |K_{\lambda + \tau, \text{low}}(x,x) - K_{\lambda, \text{low}}(x,x)| \lesssim (1 + \lambda)^{n-1}. \]
    %
    These inequalities are equivalent to the $(L^1,L^2)$ \emph{discrete restriction theorem} for eigenfunctions of the Laplacian, which we will analyze in the next section. Indeed, if we let $\chi_{\lambda,\tau}$ be the operator with kernel $K_{\lambda + \tau} - K_\lambda$, then the discrete restriction theorem tells us that
    %
    \[ \| \chi_{\lambda,\tau} f \|_{L^2(M)} \lesssim (1 + \lambda)^{(n-1)/2} \| f \|_{L^1(M)} \]
    %
    which by Schur's test, is equivalent to us having
    %
    \[ \sup_{x \in M} \int_M |K_{\lambda + \tau}(x,y) - K_\lambda(x,y)|^2\; dy \lesssim (1 + \lambda)^{n-1} \]
    %
    But the left hand side, by orthgonality, is equal to
    %
    \[ \sup_{x \in M} \sum_j \mathbf{I}(\lambda \leq \lambda_j \leq \lambda + \tau) |e_j(x)|^2 = \sup_{x \in M} |K_{\lambda + \tau}(x,x) - K_{\lambda}(x,x)|, \]
    %
    which gives the required inequality. The analysis of $K_{\lambda, \text{low}}$ follows because we can write
    %
    \[ K_{\lambda + \tau, \text{low}}(x,y) - K_{\lambda, \text{low}}(x,y) = \frac{1}{2 \pi i} \int \frac{\rho(t)}{t + i0} (e^{-2 \pi i \tau t} - 1) e^{2 \pi i (T - \lambda)}\; dt. \]
    %
    The $e^{-2 \pi i t \tau t} - 1$ term annihilates the singularity at the origin, and one can now argue as above to show this quantity is $O(\lambda^{n-1})$.
\end{proof}

\begin{lemma}
    Let $g(\lambda)$ be a piecewise continuous function on $\RR$, such that for $\lambda > 0$, and $0 < \tau \leq 1$,
    %
    \[ |g(\lambda + \tau) - g(\lambda)| \lesssim (1 + \lambda)^a. \]
    %
    If $\widehat{g}(t)$ vanishes for $|t| \leq 1$, then
    %
    \[ |g(\lambda)| \lesssim (1 + \lambda)^a. \]
\end{lemma}
\begin{proof}
    If
    %
    \[ G(\lambda) = \int_\lambda^{\lambda + 1} g(\tau)\; d\tau, \]
    %
    then $G$ is absolutely continuous, and for almost all $\lambda$, $G$ is differentiable with
    %
    \[ |G'(\lambda)| = |g(\lambda + 1) - g(\lambda)| \lesssim (1 + \lambda)^a. \]
    %
    The Fourier transform of $G$ also vanishes for $|t| \leq 1$. But
    %
    \[ |g(\lambda)| \leq |G(\lambda)| + O((1 + \lambda)^a) \]
    %
    so it suffices to prove the estimates for $G$. If $\eta(t) = 1$ for $|t| > 1$, and $\eta(t) = 0$ for $|t| \leq 1/2$, and if $\psi$ has Fourier transform $\eta(t) / 2 \pi i t$, then $\psi$ is bounded and rapidly decreasing at infinity. But $G' * \psi = G$, which gives
    %
    \[ |G(\lambda)| \lesssim (1 + \lambda)^a \int |\psi(s)| (1 + |s|)^a\; ds \lesssim (1 + \lambda)^a. \qedhere \]
\end{proof}









\section{Discrete Restriction Theory}

Let $T$ be a classical, self-adjoint, elliptic pseudodifferential operator of order one on a compact manifold $M$, and consider an eigenfunction decomposition $\{ e_j \}$ for $T$, corresponding to an increasing set of eigenvalues $\{ \lambda_i \}$. We are concerned with the $(L^p,L^q)$ bounds for the \emph{spectral band projection operators}
%
\[ \chi_{\lambda,\varepsilon} = \sum_{\lambda \leq \lambda_j \leq \lambda + \varepsilon} E_j. \]
%
These operators are closely related to restriction theory. The Stein-Tomas theorem studies the spherical projection operator, defined for functions on $\RR^n$ by setting
%
\[ \chi f(x) = \int_{|\xi| = 1} \widehat{f}(\xi) e^{2 \pi i \xi \cdot x}\; d\xi. \]
%
We can also consider the \emph{spherical projection operators}
%
\[ \chi_\lambda f(x) = \int_{|\xi| = \lambda} \widehat{f}(\xi) e^{2 \pi i \xi \cdot x}\; d\xi = \lambda^{-1} \text{Dil}_{1/\lambda} \chi \{ \text{Dil}_\lambda f \}. \]
%
The Stein-Tomas theorem characterizes the mapping properties of the operator $S$, showing that the only inequality the operator has are of the form
%
\[ \| \chi f \|_{L^{p^*}(\RR^n)} \lesssim \| f \|_{L^p(\RR^n)} \]
%
for $1 \leq p \leq 2(n+1)/(n+3)$. Rescaled, one concludes that we have
%
\[ \| \chi_\lambda f \|_{L^{p^*}(\RR^n)} \lesssim \lambda^{n(1/p - 1/p^*)-1} \| f \|_{L^p(\RR^n)}. \]
%
This inequality implies that if we define the spectral band projection operators
%
\[ \chi_{\lambda,\varepsilon} f(x) = \int_{\lambda \leq |\xi| \leq \lambda + \varepsilon} \widehat{f}(\xi) e^{2 \pi i \xi \cdot x}\; d\xi = \int_\lambda^{\lambda + \varepsilon} S_\tau f(x)\; d\tau, \]
%
then,
%
\begin{align*}
    \| \chi_{\lambda,\varepsilon} f \|_{L^{p^*}(\RR^n)} &\lesssim \int_\lambda^{\lambda + \varepsilon} \tau^{n(1/p - 1/p^*) - 1} \| f \|_{L^p(\RR^n)}\; d\tau\\
    &\lesssim \varepsilon (1 + \lambda)^{n(1/p - 1/p^*) - 1} \| f \|_{L^p(\RR^n)}.
\end{align*}
%
It is also clear that a proof of this inequality would yield a proof of bounds for $\chi_\lambda$, since $\chi_\lambda = \lim_{\varepsilon \to 0} \varepsilon^{-1} \chi_{\lambda, \varepsilon}$. Thus the theory of boundedness of spherical projections is closely related to the theory of boundedness for spectral band projections.

The original context for the spherical projection operators is to apply a $TT^*$ argument, i.e. proving the $(L^p,L^2)$ and $(L^2,L^{p^*})$ boundedness of the \emph{restriction} and \emph{extension} operators, which map functions on $\RR^n$ to functions on $\lambda S^{n-1}$, and vice-versa, by setting
%
\[ R_\lambda f(\xi) = \widehat{f}(\xi) \quad\text{and}\quad E_\lambda f(x) = \int_{|\xi| = \lambda} f(\xi) e^{2 \pi i \xi \cdot x}\; d\xi. \]
%
These operators are adjoints of one another, and $S_\lambda = E_\lambda \circ R_\lambda$. Thus a $T^*T$ argument justifies that for $1 \leq p \leq 2(n+1)/(n+3)$,
%
\[ \| R_\lambda f \|_{L^2(\lambda S^{n-1})} \lesssim \lambda^{n(1/p - 1/2) - 1/2} \| f \|_{L^p(\RR^n)} \]
%
and
%
\[ \| E_\lambda f \|_{L^{p^*}(\RR^n)} \lesssim \lambda^{n(1/p - 1/2) - 1/2} \| f \|_{L^2(\lambda S^{n-1})}. \]
%
It is difficult to find analogous operators to $R_\lambda$ and $E_\lambda$ on a manifold, just as we cannot directly find an analogue of $S_\lambda$, because our eigenfunction is discrete. But we \emph{can} consider analogues obtained by thickening the restriction set, i.e. considering the operators $R_{\lambda,\varepsilon}$ and $E_{\lambda,\varepsilon}$, mapping functions on $\RR^n$ to functions on the annulus $A_{\lambda,\varepsilon} = \{ \xi \in \RR^n: \lambda \leq |\xi| \leq \lambda + \varepsilon \}$ and vice versa by setting
%
\[ R_{\lambda,\varepsilon} f(\xi) = \widehat{f}(\xi) \quad\text{and}\quad E_{\lambda,\varepsilon} f(x) = \int_{\lambda \leq |\xi| \leq \lambda + \varepsilon} f(\xi) e^{2 \pi i \xi \cdot x}\; d\xi. \]
%
We have $\chi_{\lambda,\varepsilon} = E_{\lambda,\varepsilon} \circ R_{\lambda,\varepsilon}$, and the Stein-Tomas theorem is then verified to be equivalent to bounds of the form
%
\[ \| R_{\lambda,\varepsilon} f \|_{L^2(A_{\lambda,\varepsilon})} \lesssim \varepsilon \lambda^{n(1/p - 1/2) - 1/2} \| f \|_{L^p(\RR^n)} \]
%
and
%
\[ \| E_{\lambda,\varepsilon} f \|_{L^{p^*}(\RR^n)} \lesssim \varepsilon \lambda^{n(1/p - 1/2) - 1/2} \| f \|_{L^2(A_{\lambda,\varepsilon})}. \]
%
But the orthogonality of the Fourier transform, shows that the restriction bound is equivalent to the fact that
%
\[ \| \chi_{\lambda,\varepsilon} f \|_{L^2(\RR^n)} \lesssim \varepsilon \lambda^{n|1/p - 1/2| - 1/2} \| f \|_{L^p(\RR^n)} \]
%
and
%
\[ \| \chi_{\lambda,\varepsilon} f \|_{L^{p^*}(\RR^n)} \lesssim \varepsilon \lambda^{n|1/p - 1/2| - 1/2} \| f \|_{L^2(\RR^n)}. \]
%
This bound is the one we will begin by focus on generalizing to arbitrary manifolds, and we will start with case $p = 1$. Since there is no ambiguity, in the sequel, we let $\chi_\lambda = \chi_{\lambda,1}$.

We note that in $\RR^n$, one can prove the same result for any positive homogeneous function $p(\xi)$ of degree one such that the \emph{cosphere} $\Sigma = \{ \xi : p(\xi) = 1 \}$ has non-vanishing curvature, Stein-Tomas gives results about
%
\[ \chi_\lambda f(x) = \int_\Sigma \widehat{f}(\xi) e^{2 \pi i \xi \cdot x}\; d \xi. \]
%
Returning to the analysis of the operator $T$ with principal symbol $p(x,\xi)$. We will obtain good results about the operator $\chi_\lambda$ provided that the cospheres $\Sigma(x) = \{ \xi : p(x,\xi) = 1 \}$ has non-vanishing curvature for all $x \in M$.

The easiest bound in the Stein-Tomas theorem are the equivalent bounds
%
\[ \| \chi_\lambda f \|_{L^2(\RR^n)} \lesssim \lambda^{(n-1)/2} \| f \|_{L^1(\RR^n)} \]
%
and
%
\[ \| \chi_\lambda f \|_{L^\infty(\RR^n)} \lesssim \lambda)^{(n-1)/2} \| f \|_{L^2(\RR^n)}, \]
%
which does not even utilize the curvature of the domain of projection whatsoever, solely using the fact that the sphere of radius $\lambda$ has surface measure $O(\lambda^{n-1})$. Let us see if we can justify the analogous result on a compact manifold. We will find we do not need to assume curvature here either.

\begin{lemma}
    Let $T$ be a classical, elliptic, self-adjoint pseudodifferential operator of order one on a compact manifold $M$ of dimension $n$, and consider the association spectral band projection operators $\{ \chi_\lambda \}$. Then
    %
    \[ \| \chi_\lambda f \|_{L^2(M)} \lesssim \lambda^{(n-1)/2} \| f \|_{L^1(M)}, \]
    %
    and thus by self-adjointness, that
    %
    \[ \| \chi_\lambda f \|_{L^\infty(M)} \lesssim \lambda^{(n-1)/2} \| f \|_{L^2(M)}. \]
\end{lemma}
\begin{proof}
    We want to exploit the parametrix for the wave equation. Thus we start by replacing $\chi_\lambda$ with an operator $\tilde{\chi}_\lambda$ given by
    %
    \[ \tilde{\chi}_\lambda f(x) = \sum_j \chi(\lambda_j - \lambda) E_j, \]
    %
    where $\chi$ is a non-negative Schwartz function with $\chi(0) > 0$, and such that $\widehat{\chi}(t)$ has support on $|t| \leq \varepsilon/2$. Because $\mathbf{I}(\lambda \leq \lambda_j \leq \lambda + \varepsilon) \lesssim \chi(\lambda_j - \lambda)$, orthogonality shows that
    %
    \[ \| \chi_\lambda f \|_{L^2(M)} \lesssim \| \tilde{\chi}_\lambda f \|_{L^2(M)} \]
    %
    and so it suffices to prove the result for $\tilde{\chi}_\lambda$. But if $\tilde{\chi}_\lambda$ has kernel
    %
    \[ \tilde{K}_\lambda(x,y) = \sum_j \chi(\lambda_j - \lambda) e_j(x) \overline{e_j(y)}. \]
    %
    This is equivalent to showing that
    %
    \[ \| \tilde{K}_\lambda \|_{L^\infty_y L^2_x} \lesssim \lambda^{(n-1)/2}. \]
    %
    But orthogonality shows that
    %
    \begin{align*}
        \| \tilde{K}_\lambda \|_{L^\infty_y L^2_x} &= \left( \sup_{y \in M} \sum_j \chi(\lambda_j - \lambda)^2 |e_j(y)|^2 \right)^{1/2}\\
        &\leq \| \chi \|_{L^\infty}^{1/2} \left( \sup_{y \in M} \sum_j \chi(\lambda_j - \lambda) |e_j(y)|^2 \right)^{1/2}\\
        &\leq \| \chi \|_{L^\infty}^{1/2} \left( \sup_{y \in M} \tilde{\chi}_\lambda(y,y) \right)^{1/2}.
    \end{align*}
    %
    Thus we need to show that for all $y \in M$,
    %
    \[ \tilde{\chi}_\lambda(y,y) \lesssim \lambda^{n-1}. \]
    %
    This follows by an argument similar to the proof of the Weyl law, e.g. applying the parametrix for the wave equation.
\end{proof}

Now we move onto results that require a curvature assumption.

\begin{theorem}
    Let $T$ be a classical, elliptic, self-adjoint pseudodifferential operator of order one on a compact manifold $M$ of dimension $n$ and with principal symbol $p(x,\xi)$, such that the cospheres
    %
    \[ \Sigma(x) = \{ \xi \in T^*M : p(x,\xi) = 1 \} \]
    %
    have non-vanishing curvature for each $x \in M$. If we consider the association spectral band projection operators $\{ \chi_\lambda \}$, then for $p_c = 2(n+1)/(n+3)$, we have
    %
    \[ \| \chi_\lambda f \|_{L^2(M)} \lesssim (1 + \lambda)^{(n - 1)/2(n+1)} \| f \|_{L^{p_c}(M)}. \]
\end{theorem}

\begin{remark}
    Combined with the $(L^1,L^2)$ boundedness of the equation, and the fact that
    %
    \[ \| \chi_\lambda f \|_{L^2(M)} \lesssim \| f \|_{L^2(M)} \]
    %
    we conclude using interpolation that for $1 \leq p \leq 2(n+1)/(n+3)$,
    %
    \[ \| \chi_\lambda f \|_{L^2(M)} \lesssim (1 + \lambda)^{n|1/p - 1/2| - 1/2} \| f \|_{L^p(M)} \]
    %
    and for $2(n+1)/(n+3) \leq p \leq 2$,
    %
    \[ \| \chi_\lambda f \|_{L^2(M)} \lesssim (1 + \lambda)^{(\frac{n-1}{2})(1/p-1/2)} \| f \|_{L^p(M)}. \]
    %
    All these inequalities are sharp, as we will discuss after the proof.
\end{remark}

\begin{proof}
    We start by using a strategy we began the $(L^1,L^2)$ boundedness result with, namely, to swap $\chi_\lambda$ out with a more well behaved operator. The operators $\tilde{\chi}_\lambda$ used in that proof are not well behaved near the diagonal (TODO: Why?). Thus we modify their definition slightly. We fix a small quantity $\varepsilon_0 \leq \varepsilon/2$, and define a Schwartz function $\chi$ (not necessarily positive this time), with $\chi(0) = 1$ and with Fourier support on $\varepsilon_0 / 2 \leq t \leq \varepsilon_0$. Nonetheless, it remains true that $\chi(x) \gtrsim 1$ for $|x| \leq \varepsilon$, so that
    %
    \[ \| \chi_\lambda f \|_{L^2(M)} \lesssim \| \tilde{\chi}_\lambda f \|_{L^2(M)}. \]
    %
    It thus suffices to obtain a bound for $\tilde{\chi}_\lambda$ at the critical exponent. Applying the parametrix, and seeing that the remainder term has operator norm $O_N((1 + \lambda)^{-N})$ for all $N > 0$, it suffices to show that the operator
    %
    \[ \tilde{\chi}_\lambda f = \frac{1}{2 \pi i} \int S(t) f e^{-2 \pi i t \lambda} \widehat{\chi}(t)\; dt \]
    %
    is well behaved 
\end{proof}









\chapter{Fourier Integral Operators}

Pseudodifferential operators formalize the family of all operators that modulate the amplitude of wave packets. The theory of Fourier integral operators extends this theory by not only modulating the amplitude of wave packets, but also moving them around in phase space in a \emph{symplectic manner}, i.e. in a way which, roughly speaking, obeys the uncertainty principle, in the sense that the area where a function is supported in phase space is preserved. Basic examples of Fourier integral operators will include the translation operators
%
\[ \text{Trans}_{x_0} f(x) = f(x - x_0), \]
%
modulation operators
%
\[ \text{Mod}_{\xi_0} f(x) = e^{2 \pi i \xi_0 \cdot x} f(x), \]
%
the change of variables operator $T_A$ associated with an invertible linear transformation $A: \RR^n \to \RR^n$, i.e. such that
%
\[ T_A f(x) =  |\det(A)|^{-1/2} f(A^{-1} x), \]
%
and the Fourier transform
%
\[ \mathcal{F}f(\xi) = \widehat{f}(\xi). \]
%
These four operators are all unitary, so might be thought of as preserving the amplitude of wave packets:
%
\begin{itemize}
    \item The translation operators move wave packets in phase space according to the diffeomorphism $\Phi(x,\xi) = (x + x_0, \xi)$.

    \item The modulation operators move packets in phase space according to the diffeomorphism $\Phi(x,\xi) = (x,\xi + \xi_0)$.

    \item The change of variables move packets in space according to the diffeomorphism $\Phi(x,\xi) = (Ax, (A^T)^{-1} \xi)$.

    \item The Fourier transform move packets according to the diffeomorphism $\Phi(x,\xi) = (\xi,-x)$.
\end{itemize}
%
All four of the diffeomorphisms $\Phi$ are \emph{symplectomorphisms}, a fact very important to the tractability of the study of the associated operators, because of the uncertainty principle. Let us assume that with a diffeomorphism $\Phi: T^* \RR^n \to T^* \RR^n$, we can associate a resulting unitary operator $T_\Phi$ from $L^2(\RR^n)$ to itself, which roughly speaking, has the property that it maps a wave packet localized at a point $(x_0,\xi_0)$ to a wave packet localized near $\Phi(x_0,\xi_0)$. If we consider an pseudodifferential operator $S_a$ which maps amplifies a wave packet localized at $(x_0,\xi_0)$ by a quantity $a(x,\xi)$, then morally speaking, we should expect the operator $T_\Phi^{-1} \circ S \circ T_\Phi$ to fix the location of wave packets, and amplify wave packets localized at $(x_0,\xi_0)$ by the quantity $a \circ \Phi$. We hope this at least holds up to first order. If we have another operator $S_b$, then the symbolic calculus would say that the principal parts of  $[(S_a \circ \Phi), (S_b \circ \Phi)]$ and $T_\Phi^{-1} \circ [S_a, S_b] \circ T_\Phi$ agree with one another, which means that, in terms of the Poisson bracket, we must have $\{ a \circ \Phi, b \circ \Phi \} = \{ a, b \} \circ \Phi$. Since $\{ f,g \} = \omega( \nabla f, \nabla g )$, where $\omega = dx \wedge d\xi - d\xi \wedge dx$ is the standard symplectic form, this can only be true if $\Phi^* \omega = \omega$, i.e. $\Phi$ is a symplectomorphism. In this case, the graph of $\Phi$, namely the set
%
\[ \Lambda_\Phi = \{ (x,y;\xi,\eta) : (x,\xi) = \Phi(y,\eta) \} \subset T^*(\RR^n \times \RR^n) \]
%
will be a \emph{Lagrangian submanifold} of $T^* \RR^n \times T^* \RR^n$ with respect to the symplectic form $\omega \oplus (- \omega)$.
%since, once we identify each of the tangent spaces $T_p(T^* \RR^n \times T^* \RR^n)$ with $\RR^n \times \RR^n \times \RR^n \times \RR^n$ via the coordinate system $(dx, d\xi, dy, d\eta)$, then the tangent space to $\Lambda_\Phi$ at each point $(p,q) \in T^* \RR^n \times T^* \RR^n$ is the set of all pairs $(v_x, v_\xi, w_y, w_\eta)$ such that $(w_y, w_\eta) = D\Phi(q) (v_x,v_\xi)$, and then
%
%\[ \omega(v_x,v_\xi) - \omega(w_y,w_\eta) = \omega(v_x, v_\xi) - \omega(D\Phi(q)(v_x, v_\xi)) \]
%since the tangent space at each point is $(dx, d\xi) = D \Phi \cdot (dy, d\eta)$
% Phi^* omega (v,w) = omega( DPhi(v), DPhi(w)  ) = v^T DPhi^T M DPhi w = v^T M w
% so omega oplus -omega

The fact that we are reducing ourselves to the study of symplectomorphisms $\Phi$ gives us a hint as to how to define the resulting operator $T_\Phi$, at least microlocally. Intuitively speaking, our discussion above shows that the wave front set of the kernel of $\Phi$ must be contained in $\Lambda_\Phi$, because spectral singularities at a point $(y,\eta)$ will be moved to singularities at a point $(x,\xi)$, so we should expect that $\text{WF}(T_\Phi f) = \Lambda_\Phi \circ \text{WF}(f)$. We know a family of operators of this property: in a previous section we discussed the oscillatory integral distributions
%
\[ \int a(x,\theta) e^{2 \pi i \phi(x,\theta)}\; d\theta. \]
%
The wave-front set contained in $\Lambda_\phi = \{ (x,\nabla_x \phi(x,\theta)) : \nabla_\xi \phi(x) = 0 \}$, and, provided $\phi$ is non-degenerate, $\Lambda_\phi$ is a Lagrangian manifold. Thus, given a symplectomorphism $\Phi: T^* \RR^n \to T^* \RR^n$, if we can find a phase $\phi: \RR^n \times \RR^n \to \RR^n$ such that $\Lambda_\Phi = \Lambda_\phi$, then, modulo $C^\infty$ kernels, we might expect to find a symbol $a$ such that
%
\[ T_\Phi f(x) = \int a(x,y,\theta) e^{2 \pi i \phi(x,\theta)} f(y)\; d\theta\; dy. \]
%
More generally, we might not be able to find a phase $\phi$ that works globally for $\Phi$, but we can localize, and such $\phi$ will \emph{always} exist locally by virtue of the symplectic structure of $\Lambda_\Phi$. Thus we might wish to consider the families of operators formed from finite sums of integral operators given by the right hand side of this equation. We have now reached the study of general Fourier integral operators.

\section{Hyperbolic Equations}

Fourier integral operators were initially introduced to obtain parametrices for hyperbolic equations. To see how these arise, let us begin with a constant coefficient linear differential operator on $\RR_t \times \RR^n_x$, given by $P(\partial_t, D_x)$, where $P(\tau, \xi)$ is a polynomial, which we will assume can be written in the form $\tau^m + \tau^{m-1} Q_1(\xi) + \dots + Q_m(\xi)$, where $Q_i(\xi)$ is a polynomial of degree at most $i$. Then the hyperbolic equation is
%
\[ L = \partial_t^m + \partial_t^{m-1} Q_1(D_x) + \dots + \partial_t Q_{m-1}(D_x) + Q_m(D_x). \]
%
We note that here $\partial_t$ is the standard derivative operator, whereas $D_x^\alpha$ is the derivative operator, normalized by dividing by an appropriate power of $2 \pi i$ so that it is the Fourier multiplier of $\xi^\alpha$. We recall the Cauchy-Kovalevskaya theorem, which gives unique analytic solutions to the Cauchy problem $Lu = f$ given initial conditions $u_0,\partial_t u_0, \dots, \partial_t^{m-1} u_0$, given that $f$, and the initial conditions are analytic functions on $\RR^n$. But we are interested in more general existence results.

The surprising feature of this problem is that these operators need not even have solutions if we switch from studying analytical initial conditions to say, compactly supported smooth initial conditions. For instance, suppose there exists a distribution $u$ on $\RR_t \times \RR_x$, tempered in the $x$-variable, such that $Lu = 0$, where $L = \partial_t - D$, and we let $u_0 \in \SW(\RR_x)^*$ be the initial value of the distribution. Then, taking Fourier transforms in the $x$ variable, we conclude that $\partial_t \widehat{u}(t,\xi) = \xi \widehat{u}(t,\xi)$, which implies that $\widehat{u}(t,\xi) = \widehat{u_0}(\xi) e^{\xi t}$. But this distribution is \emph{never} tempered in the $\xi$ variable; if it was tempered for one positive value of $t$, and one negative value of $t$, then we could conclude that
%
\[ |\widehat{u_0}(\xi)| \lesssim e^{- \varepsilon |\xi|} \]
%
for some $\varepsilon > 0$. But the Paley-Wiener theorem and it's variants therefore imply that $u_0$ is analytic, and actually extends to a holomorphic function on a small strip containing the real line. Thus the existence of solutions to the Cauchy problem $\partial_t u - Du = 0$ is very delicate; in particular, there are no solutions with initial conditions in $\DD(\RR^d_x)$. This hints at the fact that to make the solution to the Cauchy problem tractable, we must ensure that the polynomial $P(\tau,\xi) = 0$ have \emph{imaginary roots}. A desire to find a more powerful existence statement for solutions to such equations, tempered in the $x$-variable, will force us to choose polynomials $P(\tau,\xi)$ that have purely imaginary roots in the $\tau$ variable.

%Another kind of problem occurs if the polynomial $P(\tau,\xi) = 0$ has \emph{repeated roots}. For instance, if we consider the operator $L = (\partial_t - 2 \pi i D)^2$. If $u$ is tempered in the $x$-variable and solves the equation $Lu$ with initial conditions $u_0 \in \SW(\RR_x)^*$, then taking the Fourier transform leads to an expression of $u$ in the form
%
%\[ \widehat{u}(\xi,t) = \widehat{u_0}(\xi) e^{i \xi t} + \left( \partial_t \widehat{u}_0(\xi) - i \xi \widehat{u}_0(\xi) \right) t e^{i \xi t}. \]
%
%Taking inverse Fourier transforms implies that
%
%\[ u(x,t) = u_0(x + t/2\pi) + t \cdot \partial_t u_0(x + t/2\pi) - i t \cdot Du_0(x + t/2\pi). \]
%
%TODO

To continue a discussion of this problem, let us set $\tilde{E}_0, \dots, \tilde{E}_{m-1}: \RR^n_\xi \times \RR_t \to \CC$ be the solutions to the Cauchy problem $P(\partial_t,\xi) = 0$ with initial conditions
%
\[ \partial_t^i \tilde{E}_j(0,\xi) = \delta_{ij} \]
%
for $0 \leq i \leq m-1$. In fact, we can calculate the functions $\{ \tilde{E}_i \}$ explicitly. If we fix $\xi_0$, and assume for simplicity that the roots of $P(\tau,\xi_0)$ in the $\tau$ variable are distinct. Then the roots are distinct locally around $\xi_0$. If we let $\tau_1(\xi),\dots,\tau_m(\xi)$ are the roots of the equation, then these are analytic functions in the $\xi$ variable locally around $\xi_0$. The functions $h_i(t,\xi) = e^{i \tau_i(\xi) t}$ then satisfy the Cauchy problem $P(\partial_t,\xi) = 0$ with initial conditions
%
\[ \partial_t^i h_j(0,\xi) = \tau_j(\xi)^i \]
%
for $0 \leq i \leq m-1$. The uniqueness of analytic solutions to this equation implies that
%
\[ h_i(t,\xi) = \tilde{E}_0(t,\xi) + \tilde{E}_1(t,\xi) \tau_i(\xi) + \dots + \tilde{E}_{m-1}(t,\xi) \tau_i^{m-1}(\xi). \]
%
If $\tilde{E} = (\tilde{E}_0,\dots,\tilde{E}_{m-1})$ and $h = (h_0,\dots,h_{m-1})$, then
%
\[ h = \begin{pmatrix} 1 & \tau_1 & \dots & \tau_1^{m-1} \\ 1 & \tau_2 & \dots & \tau_2^{m-1} \\ \vdots & \ddots & \dots & \vdots \\ 1 & \tau_m & \dots & \tau_m^{m-1} \end{pmatrix} \tilde{E}. \]
%
Provided that the roots $\{ \tau_i \}$ are distinct, we can solve this equation to find the functions $\{ \tilde{E}_i \}$ in terms of the functions $\{ h_i \}$ using Cramer's rule. Namely, if $V(\tau_1,\dots,\tau_m)$ is the Vandermonde determinant, i.e. the determinant of the matrix $\{ \tau_i^j \}$, and if $V_i(\tau_1,\dots,\tau_m;t)$ is the determinant of the matrix obtained by replacing the $i$th column with the vector $\{ e^{t \tau_j(\xi)} \}$, then
%
\[ \tilde{E}_i(t,\xi) = \frac{V_i(\tau_1(\xi),\dots,\tau_m(\xi);t)}{V(\tau_1(\xi),\dots,\tau_m(\xi))}. \]
%
For instance, if $m = 2$, then
%
\[ \tilde{E}_1(t,\xi) = \frac{e^{t \tau_1(\xi)} \tau_2(\xi) - e^{t \tau_2(\xi)} \tau_1(\xi)}{\tau_2(\xi) - \tau_1(\xi)} \quad\text{and}\quad \tilde{E}_2(t,\xi) = \frac{e^{t \tau_2(\xi)} - e^{t \tau_1(\xi)}}{\tau_2(\xi) - \tau_1(\xi)} \]
%
Note that the functions
%
\[ A_i(\tau_1,\dots,\tau_m,t) = \frac{V_i(\tau_1,\dots,\tau_m;t)}{V(\tau_1,\dots,\tau_m)} \]
%
are analytic in $t$, symmetric in the variables $\{ \tau_i \}$, and \emph{entire} in the variables $(\tau_1,\dots,\tau_m) \in \CC^n$. For instance, in the case $m = 2$, when $\tau_1(\xi_0) = \tau_2(\xi_0) = \tau$ we have
%
\[ \tilde{E}_1(t,\xi_0) = e^{t \tau} (1 - t \tau) \quad\text{and}\quad \tilde{E}_2(t,\xi_0) = t e^{t \tau}. \]
%
Thus we have an expression for these important solutions to the problem.

By virtue of the roots switching around, one cannot necessarily define the functions $h_1,\dots,h_n$ globally for all $\xi \in \RR^n$. But quantities symmetric in the variables $\{ h_i \}$ are well defined, for instance, the function
%
\[ s(t,\xi) = |\text{Re}(h_1(t,\xi))| + \dots + |\text{Re}(h_n(t,\xi))|. \]
%
If $\tilde{E}_0,\dots,\tilde{E}_{m-1}$ are tempered in the $x$ variable, then it follows that the function $s$ is tempered in the $\xi$ variable, and thus satisfies some equation of the form
%
\[ |s(t,\xi)| \lesssim_t \langle \xi \rangle^{N_t} \]
%
for some $N > 0$. But this means that
%
\[ |\text{Re}(\tau_1(\xi))| + \dots + |\text{Re}(\tau_m(\xi))| \lesssim 1 + \log \langle \xi \rangle. \]
%
But the theory of semialgebraic sets (which applies because $(\tau_1,\dots,\tau_m)$ are the projections onto the $\tau$ variable of solutions to the polynomial equation $P(\tau,\xi) = 0$) implies that this can only be possible if
%
\[ |\text{Re}(\tau_1(\xi))| + \dots + |\text{Re}(\tau_m(\xi))| \lesssim 1, \]
%
i.e. because we can only have polynomial growth on semialgebraic sets. This is actually a necessary and sufficient condition for the Cauchy problem to be solvable (a result of Garding). An operator with this property will be called \emph{hyperbolic}.

There is an equivalent specification of being hyperbolic which is very useful to the study of such equations. If $L = P(\partial_t, D_x)$ is a hyperbolic constant-coefficient partial differential equation of order $m$, then we can consider the degree $m$ polynomial $P_m(\tau,\xi)$. We claim the roots of $P_m$ in the $\tau$ variable differ from the roots of $P$ by at most $O(1)$. Indeed, if $P(\tau,\xi) = \tau^m + a_{m-1}(\xi) \tau^{m-1} + \dots + a_0(\xi)$, then
%
\[ \tau^m = - a_{m-1} \]

if $P(\tau,\xi) = 0$, then
%
\[ \tau^m = a_m(\xi) \tau^{m-1} + \dots + a \]

$P_m(\tau,\xi)$


if we label them as $i \lambda_1(\xi), \dots, i \lambda_(\xi)$, up to multiplicity, it suffices to show that the values $\{ \lambda_i(\xi) \}$ are all real-valued. Now if $P(\tau,\xi) = 0$, 

$P_{\leq m}(\tau,\xi)$

Since $P(\tau,\xi) - P_m(\tau,\xi)$

We say a differential operator $P(\partial_t, D_x)$ is \emph{strongly}, or \emph{strictly hyperbolic} if, for each $\xi \in \RR^d$, the roots of the homogeneous leading terms $P_m(\tau, \xi)$ of the operator are distinct and purely imaginary. If we label the roots as $i \lambda_1(\xi), \dots, i \lambda_m(\xi)$, then the functions $\{ \lambda_i \}$ are analytic function in $\RR^n - \{ 0 \}$, and homogeneous of degree one. Thus $|\lambda_i(\xi) - \lambda_j(\xi)| \gtrsim |\xi|$ for $i \neq j$. This actually means we are in the previous 

Because $P(\tau,\xi) - P_m(\tau,\xi)$

Because The roots of $P$ will also be distinct for large enough $\xi$, 

\section{Local Theory}

Let us recall the theory of oscillatory integral distributions. Let $U \subset \RR^n$ be open. Consider a real-valued phase function $\phi \in C^\infty(U_x \times \RR^p_\theta)$, homogeneous in the $\theta$ variable, and such that $\nabla_{x,\theta} \phi$ is non-vanishing on the support of a symbol $a \in S^m(U \times \RR^p)$. Then we can define a distribution $u$, formally speaking, by the equation
%
\[ u(x) = \int_{\RR^p} a(x,\theta) e^{2 \pi i \phi(x,\theta)}\; d\theta. \]
%
If we consider the conic set
%
\[ \Sigma_\theta = \{ (x,\theta) \in U \times \RR^p : \nabla_\theta \phi(x,\theta) = 0 \}, \]
%
then $\text{WF}(u) \subset \Sigma_\theta$. Let us now assume the phase is \emph{nondegenerate}, in the sense that whenever $\nabla_\theta \phi = 0$, $D_{x,\theta} (\nabla_\theta \phi)$ is an invertible matrix. Then $\Sigma_\theta$ will be an $n$ dimensional manifold in $U \times \RR^p$, and the map $\Sigma_\theta \to T^* U$ given by $(x,\theta) \mapsto \nabla_x \phi(x,\theta)$ will be an immersion, which we will denote by $\Lambda_\theta$. This immersed manifold will be a \emph{Lagrangian submanifold} of $T^* U$, in the sense that the Tangent spaces $W$ at each point of $\Lambda_\theta$ will satisfy $W^\perp = W$ with respect to the symplectic form $d\xi \wedge dx$.

Let us now get to the new theory. A distribution $u$ of the form above will be called a \emph{Lagrangian distribution} of order $m + p/2 - n/4$ associated to the Lagrangian manifold $\Lambda_\theta$. This definition of order seems odd, but we note that if we consider the Lagrangian distribution
%
\[ K(x,y) = \int_{\RR^n} a(x,\xi) e^{2 \pi i \xi \cdot (x - y)}\; d\xi, \]
%
which is the kernel of a pseudodifferential operator, then this distribution will have the same order as the associated pseudodifferential operator.














