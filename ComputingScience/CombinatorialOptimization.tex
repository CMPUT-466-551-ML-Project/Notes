\documentclass[12pt, dvipsnames]{report}

\usepackage{amsmath}
\usepackage{algorithm}
%\usepackage{algorithmic}
\usepackage[noend]{algpseudocode}

\usepackage{amsmath}
\usepackage{amssymb}
\usepackage{amsthm}
\usepackage{amsopn}

\usepackage{kpfonts}

\usepackage{graphicx}

% Probably don't need this on notes anymore
%\usepackage{kbordermatrix}

% Standard tool for drawing diagrams.
\usepackage{tikz}
\usepackage{tkz-berge}
\usepackage{tikz-cd}
\usepackage{tkz-graph}

\usepackage{comment}

%
\usepackage{multicol}

%
\usepackage{framed}

%
\usepackage{mathtools}

%
\usepackage{float}

%
\usepackage{subfig}

%
\usepackage{wrapfig}

%
\let\savewideparen\wideparen
\let\wideparen\relax
\usepackage{mathabx}
\let\wideparen\savewideparen

% Used for generating `enlightening quotes'
\usepackage{epigraph}

% Forget what this is used for :P
\usepackage[utf8]{inputenc}

% Used for generating quotes.
\usepackage{csquotes}

% Allows what to generate links inside
% generated pdf files
\usepackage{hyperref}

% Allows one to customize theorem
% environments in mathematical proofs.
\usepackage{thmtools}

% Gives access to a proof
\usepackage{lplfitch}

% I forget what this is for.
\usepackage{accents}

% A package for drawing simple trees,
% as a substitute for unnesacary TIKZ code
\usepackage{qtree}

% Enables sequent calculus proofs
\usepackage{ebproof}

% For braket notation
\usepackage{braket}

% To change line spacing when using mathematical notations which require some height!
\usepackage{setspace}

%\usepackage[dvipsnames]{xcolor}

\usepackage{float}

% For block commenting
\usepackage{comment}




\setlength\epigraphwidth{8cm}

\usetikzlibrary{arrows, petri, topaths, decorations.markings}

% So you can do calculations in coordinate specifications
\usetikzlibrary{calc}
\usetikzlibrary{angles}

\theoremstyle{plain}
\newtheorem{theorem}{Theorem}[chapter]
\newtheorem{axiom}{Axiom}
\newtheorem{lemma}[theorem]{Lemma}
\newtheorem{corollary}[theorem]{Corollary}
\newtheorem{prop}[theorem]{Proposition}
\newtheorem{exercise}{Exercise}[chapter]
\newtheorem{fact}{Fact}[chapter]

\newtheorem*{example}{Example}
\newtheorem*{proof*}{Proof}

\theoremstyle{remark}
\newtheorem*{exposition}{Exposition}
\newtheorem*{remark}{Remark}
\newtheorem*{remarks}{Remarks}

\theoremstyle{definition}
\newtheorem*{defi}{Definition}

\usepackage{hyperref}
\hypersetup{
    colorlinks = true,
    linkcolor = black,
}

\usepackage{textgreek}

\makeatletter
\renewcommand*\env@matrix[1][*\c@MaxMatrixCols c]{%
  \hskip -\arraycolsep
  \let\@ifnextchar\new@ifnextchar
  \array{#1}}
\makeatother

\renewcommand*\contentsname{\hfill Table Of Contents \hfill}

\newcommand{\optionalsection}[1]{\section[* #1]{(Important) #1}}
\newcommand{\deriv}[3]{\left. \frac{\partial #1}{\partial #2} \right|_{#3}} % partial derivative involving numerator and denominator.
\newcommand{\lcm}{\operatorname{lcm}}
\newcommand{\im}{\operatorname{im}}
\newcommand{\bint}{\mathbf{Z}}
\newcommand{\gen}[1]{\langle #1 \rangle}

\newcommand{\End}{\operatorname{End}}
\newcommand{\Mor}{\operatorname{Mor}}
\newcommand{\Id}{\operatorname{id}}
\newcommand{\visspace}{\text{\textvisiblespace}}
\newcommand{\Gal}{\text{Gal}}

\newcommand{\xor}{\oplus}
\newcommand{\ft}{\wedge}
\newcommand{\ift}{\vee}

\newcommand{\prob}{\mathbf{P}}
\newcommand{\expect}{\mathbf{E}}
\DeclareMathOperator{\Var}{\mathbf{V}}
\newcommand{\Ber}{\text{Ber}}
\newcommand{\Bin}{\text{Bin}}

%\newcommand{\widecheck}[1]{{#1}^{\ft}}

\DeclareMathOperator{\diam}{\text{diam}}

\DeclareMathOperator{\QQ}{\mathbf{Q}}
\DeclareMathOperator{\ZZ}{\mathbf{Z}}
\DeclareMathOperator{\RR}{\mathbf{R}}
\DeclareMathOperator{\HH}{\mathbf{H}}
\DeclareMathOperator{\CC}{\mathbf{C}}
\DeclareMathOperator{\AB}{\mathbf{A}}
\DeclareMathOperator{\PP}{\mathbf{P}}
\DeclareMathOperator{\MM}{\mathbf{M}}
\DeclareMathOperator{\VV}{\mathbf{V}}
\DeclareMathOperator{\TT}{\mathbf{T}}
\DeclareMathOperator{\LL}{\mathcal{L}}
\DeclareMathOperator{\EE}{\mathbf{E}}
\DeclareMathOperator{\NN}{\mathbf{N}}
\DeclareMathOperator{\DQ}{\mathcal{Q}}
\DeclareMathOperator{\IA}{\mathfrak{a}}
\DeclareMathOperator{\IB}{\mathfrak{b}}
\DeclareMathOperator{\IC}{\mathfrak{c}}
\DeclareMathOperator{\IP}{\mathfrak{p}}
\DeclareMathOperator{\IQ}{\mathfrak{q}}
\DeclareMathOperator{\IM}{\mathfrak{m}}
\DeclareMathOperator{\IN}{\mathfrak{n}}
\DeclareMathOperator{\IK}{\mathfrak{k}}
\DeclareMathOperator{\ord}{\text{ord}}
\DeclareMathOperator{\Ker}{\textsf{Ker}}
\DeclareMathOperator{\Coker}{\textsf{Coker}}
\DeclareMathOperator{\emphcoker}{\emph{coker}}
\DeclareMathOperator{\pp}{\partial}
\DeclareMathOperator{\tr}{\text{tr}}

\DeclareMathOperator{\supp}{\text{supp}}

\DeclareMathOperator{\codim}{\text{codim}}

\DeclareMathOperator{\minkdim}{\dim_{\mathbf{M}}}
\DeclareMathOperator{\hausdim}{\dim_{\mathbf{H}}}
\DeclareMathOperator{\lowminkdim}{\underline{\dim}_{\mathbf{M}}}
\DeclareMathOperator{\upminkdim}{\overline{\dim}_{\mathbf{M}}}
\DeclareMathOperator{\lhdim}{\underline{\dim}_{\mathbf{M}}}
\DeclareMathOperator{\lmbdim}{\underline{\dim}_{\mathbf{MB}}}
\DeclareMathOperator{\packdim}{\text{dim}_{\mathbf{P}}}
\DeclareMathOperator{\fordim}{\dim_{\mathbf{F}}}

\DeclareMathOperator*{\argmax}{arg\,max}
\DeclareMathOperator*{\argmin}{arg\,min}

\DeclareMathOperator{\ssm}{\smallsetminus}

\usepackage{amsmath}
\usepackage{algorithm}
\usepackage[noend]{algpseudocode}

\DeclareMathOperator{\Dom}{Dom}
\DeclareMathOperator{\delin}{\delta_{\text{in}}}
\DeclareMathOperator{\delout}{\delta_{\text{out}}}

\def\@fnsymbol#1{\ensuremath{\ifcase#1\or *\or \dagger\or \ddagger\or
   \mathsection\or \mathparagraph\or \|\or **\or \dagger\dagger
   \or \ddagger\ddagger \else\@ctrerr\fi}}
\renewcommand{\thefootnote}{\fnsymbol{footnote}}

\title{Combinatorial Optimization}
\author{Jacob Denson}

\begin{document}

\pagenumbering{gobble}

\maketitle

\tableofcontents

\pagenumbering{arabic}

\chapter{Matchings and Flows}

\section{Bipartite Matching}

Let $G = (V,E)$ be a graph. A {\bf matching} is $M \subset E$ such that no vertex in $V$ is the endpoint of more than one edge in $M$. $M$ is maximal if $|M| \geq |M'|$ for any other matching $M'$. The maximal matching problem asks us to find a fast algorithm to find a maximum matching in any graph.

There is a polynomial time algorithm which can find matchings on any graph, but the problem is much more simple if $G$ is bipartite -- that is, if we may partition $V$ into the disjoint union $W \amalg U$ of two sets of vertices, such that every edge in $E$ contains a point in $W$ and a point in $U$. The bipartite matching asks us to find a maximal matching in a bipartite graph.

\begin{figure}
\begin{center}
\begin{tikzpicture}[scale=0.5,transform shape]
    \GraphInit[vstyle=Classic]

    \Vertex[NoLabel,x=1,y=0]{A1}
    \Vertex[NoLabel,x=1,y=-1.5]{B1}
    \Vertex[NoLabel,x=1,y=-3]{C1}
    \Vertex[NoLabel,x=1,y=-4.5]{D1}
    \Vertex[NoLabel,x=1,y=-6]{E1}

    \Vertex[NoLabel,x=6,y=-0.75]{A2}
    \Vertex[NoLabel,x=6,y=-2.25]{B2}
    \Vertex[NoLabel,x=6,y=-3.75]{C2}
    \Vertex[NoLabel,x=6,y=-5.25]{D2}

    \Edge[style={dashed}](A1)(A2)
    \Edge(B1)(A2)
    \Edge[style={dashed}](B1)(D2)
    \Edge(C1)(B2)
    \Edge[style={dashed}](D1)(C2)
    \Edge(E1)(C2)

    \Vertex[x=9,y=-3]{S}
    \Vertex[x=18,y=-3]{T}

    \Vertex[NoLabel,x=11,y=0]{A1'}
    \Vertex[NoLabel,x=11,y=-1.5]{B1'}
    \Vertex[NoLabel,x=11,y=-3]{C1'}
    \Vertex[NoLabel,x=11,y=-4.5]{D1'}
    \Vertex[NoLabel,x=11,y=-6]{E1'}

    \Vertex[NoLabel,x=16,y=-0.75]{A2'}
    \Vertex[NoLabel,x=16,y=-2.25]{B2'}
    \Vertex[NoLabel,x=16,y=-3.75]{C2'}
    \Vertex[NoLabel,x=16,y=-5.25]{D2'}

    \draw[->, thick](S)--(A1');
    \draw[->, dashed, thick](S)--(D1');
    \draw[->, thick](D2')--(T);
    \draw[->, thick](A1')--(A2');
    \draw[->, thick](B1')--(D2');
    \draw[->, dashed, thick](D1')--(C2');
    \draw[->, thick](A2')--(B1');
    \draw[->, dashed, thick](B2')--(C1');
    \draw[->, dashed, thick](C2')--(E1');
\end{tikzpicture}
\end{center}
\caption{The left graph shows a matching in a graph of cardinality 3. The solid lines correspond to the edges chosen. On the right is the augmenting path graph corresponding to the matching. It contains a path from $s$ to $t$, hence there is a way to improve the matching to a match of cardinality 4.}
\end{figure}

Suppose we are given a particular matching in a bipartite graph. Is there a reliable procedure to improve the matching? We could proceed by a guess and check method -- we remove an edge in our matching, then try and add an edge using one of the vertices which has been freed up. If this edge cannot be taken because the end point is attached to an edge still in the matching, we remove that edge, freeing up more vertices. If we ever end up adding more edges than we started with (which occurs when we add an edge not attached to any points in the current matching), then we find a matching with an extra edge than before. This process is formalized by the `augmenting paths' construction.

Given a particular matching $M$, construct a directed graph $G_M = (V \cup \{ s, t \},E_M)$, where $s$ and $t$ are new vertices. Let $w \in W$, $u \in U$. Construct the edges $E_M$ such that
%
\begin{itemize}
    \item $wu \in E_M$, if $wu \in M^c$.
    \item $uw \in E_M$, if $uw \in M$.
    \item $sw \in E_M$, if $w$ is not the endpoint of any edge in $M$.
    \item $ut \in E_M$, if $u$ is not the endpoint of any edge in $M$.
\end{itemize}
%
Let $(s,w_1,u_1,\dots,w_n,u_n,t)$ be a path in $G_M$ from $s$ to $t$, with $w_i \in W$, $u_i \in U$. Then $w_i u_i \in M^c$ and $u_iw_{i+1} \in M$, and $w$ and $u$ are both unused in $M$. Let $M'$ be obtained from $M$ by removing all edges of the form $u_iw_{i+1}$, and adding all edges of the form $w_i u_i$. It is easy to convince yourself that $M'$ is a matching, with one more edge than $M$.

Conversely, suppose that there is no path from $s$ to $t$ in $G_M$. We claim $M$ is then a maximal matching on $G$. Otherwise, we would have a matching $M^*$ with $|M^*| > |M|$. Consider the multigraph $H = (V, M \amalg M^*)$. Every vertex in $V$ has at most degree two (for it can only be the endpoint of a single edge in $M$, and a single endpoint in $M^*$), and thus $V$ breaks into cycles and paths. For any cycle $C$, we find $|M \cap C| = |M^* \cap C|$, for the edges in the cycle must alternate being being in $M$ and $M^*$. Since $|M^*| > |M|$, there must be a path $P$ in $H$ containing more edges in $M^*$ than in $M$. $P$ must therefore begin and end with a path in $M^*$, and we see that this path is exactly an augmenting path for $M$.

This argument justifies the correctness of the following matching algorithm, a variant of the `Ford Fulkerson Algorithm'. We take any initial matching $M$. Then, we construct the augmenting path graph $G_M$, and use any of the standard algorithms to find a path from $s$ to $t$. If we have $2n$ nodes, and $m$ edges, then we can find a path in $O(m)$ time, and we can bound the number of augmentations we require to $O(n)$, so the algorithm runs in $O(nm)$ time.

\section{Vertex Covers}

A vertex cover $C \subset V$ is a set of points containing at least one endpoint of every edge in $E$. Clearly if $M$ is a maximal matching in $G$, then $|M| \leq |C|$, because there is an injective function $f:M \to C$, obtained by mapping an edge to one of its endpoints which lies in $C$.

\begin{theorem}[K\"{o}nig - Egerv\'{a}ry]
    Let $G = (W \amalg U, E)$ be bipartite, and let $M$ be a maximal matching in $G$. Let $Z \subset V \cup \{ s, t \}$ be the nodes reachable from $s$ in the augmenting path graph $G_M$. Then $C = (W - Z) \cup (U \cap Z)$ is a cover with $|C| = |M|$.
\end{theorem}
\begin{proof}
    First we prove $C$ is a cover. Let $wu \in E$ be arbitrary.
    %
    \begin{itemize}
        \item Suppose that $wu \in M$. If $u \in Z$, then $w \in Z$, because $uw$ is an edge in $G_M$, so that if $w \not \in C$, $u \in C$. \item Suppose that $wu \not \in M$. Then $wu$ is an edge in $G_M$, so that if $w \in Z$, $u \in Z$, and therfore if $w \not \in C$, $u \in C$.
    \end{itemize}
    %
    Now we show $|U| = |M|$. First, note that $C$ contains {\it exactly one} of each of the endpoints of $M$, so $|U| \geq |M|$. Conversely, let $wu \in E - M$ be arbitrary. If $u \in C$, then $u$ is reachable from $s$, and hence $u$ is the endpoint of an edge in $M$, for otherwise $t$ is reachable from $s$ through $u$. Conversely, if $w \in C$, then $w$ is not reachable from $s$, hence $w$ is on an edge in $M$ (for otherwise there is an edge directly to $w$ from $s$ in $G_M$). This shows that every vertex in $C$ is on one of the edge in $M$, and hence $|U| = |M|$.
\end{proof}

This gives us a method of finding a min cover in a graph. One equivalent form of this theorem is very useful. Given an undirected graph $G = (V,E)$, let $N$ be the neighbour function $N(S) = \{ v \in V : sv \in E, s \in S \}$.

\begin{theorem}[Hall]
    Let $G = (W \amalg U, E)$ be a bipartite graph. Then $G$ has a matching $M$ which covers $W$ if and only if $|N(S)| \geq |S|$ for all $S \subset W$.
\end{theorem}
\begin{proof}
    Certainly if a matching $M$ covers $W$, then it induces an injective function from $S$ to $N(S)$ for each $S \subset W$. Conversely, assume no matching of $M$ covers $W$. Then K\"{o}nig's theorem tells us there is a cover $C$ of $W \amalg U$ containing less than $|W|$ nodes. Then write $A = C \cap W$, $B = C \cap U$. We find $N(W - A) \subset B$, so $|N(W-A)| \leq |B| < |W| - |A|$.
\end{proof}

\section{The Hopcroft-Karp Matching Algorithm}

The augmenting path approach to finding a maximal matching in a Bipartite graph improves the length of a candidate matching by one for each cycle of the algorithm. If we were able to consistently find augmentations which improved matchings by more than one edge, we could find a faster algorithm for maximal matching.

If $M$ is a matching, and $N_M$ be the length of a shortest alternating path in $G_M$, we shall let a collection of vertex disjoint alternating paths $P_1, \dots, P_n$ be called {\bf $\mathbf{M}$-blocking} if $|P_i| = N_M$, and every alternating path of length $N_M$ shares a vertex with one of the $P_i$. The next lemma is a simple generalization of the augmenting paths approach.

\begin{theorem}
    If $P_1, \dots, P_n$ are vertex disjoint $M$-alternating paths, then
    %
    \[ M' = (M - \cup P_i) \cup (\cup P_i - M) \]
    %
    is a matching with $|M'| = |M| + n$.
\end{theorem}

The Hopcroft-Karp algorithm proceeds as in Ford-Fulkerson, albeit improving matchings by taking $M$-blocking paths as augmentations, improving matchings in chunks. We shall discover a method to find an $M$-blocking in linear time, and also show the chunk method yields an asymptotic speedup in the number of cycles to find a maximal matching, so that Hopcroft-Karp is objectively faster than Ford-Fulkerson.

\begin{lemma}
    If $M$ is a matching, and $P_1, \dots, P_n$ an $M$-blocking set of paths. Then $M' = (M - \cup P_i) \cup (\cup P_i - M)$, and $N_{M'} \geq N_M + 1$.
\end{lemma}
\begin{proof}
    Let $Q = (v_1, \dots, v_k)$ be an alternating path in $G_{M'}$ with $|Q| \leq |N_M|$, where $v_1$ and $v_k$ are $M'$ exposed. Then $v_1$ and $v_k$ are also $M$ exposed, because switching from $M$ to $M'$ only removes exposed vertices, not adds them.

    Let $H$ be the directed graph with edges taken from all the edge lying on the $P_i$, and the directed edges of $Q$, except that we remove both copies of duplicate edges. The reverse $vu$ of any edge $uv$ in $Q$ that is not an edge on some $P_i$ must be on some $P_j$
\end{proof}

\section{Network Flow}

Let $G = (V,E)$ be a directed graph with two identified vertices $s \neq t$.  Let $\mu: E \to \mathbf{R}^+$ be a function measuring the `capacity' of each edge in the graph (Thinking of the edges as if they were `pipes' which can only carry a certain throughput). Define, for $U \subset V$,
%
\[ \delout(U) = \{ w \in V - U : vw \in E \}\ \ \ \ \ \ \ \ \ \ \delin(v) = \{ w \in V - U : wv \in E \} \]
%
A flow is a mapping $f: E \to \mathbf{R}^+$ such that $0 \leq f \leq \mu$, and for any $v \neq s,t$,
%
\[ f(\delout(v)) = f(\delin(v))\ \ \footnote{We extend functions $g: X \to \mathbf{R}$ to $g: 2^X \to \mathbf{R}$ by defining $g(A) = \sum_{x \in A} g(x)$} \]
%
a relation known as the flow conservation law. The aim of the maximum flow problem is to find $f$ such that the value function
%
\[ \text{val}(f) = f(\delout(s)) - f(\delin(s)) \]
%
is maximized. This is essentially the amount of flow which is created at $f$. It is also the amount of flow which is `absorbed' at $t$, because
%
\begin{align*}
    f(\delout(t)) - f(\delin(t)) &= f(\delout(t)) - f(\delin(t)) + \sum_{v \neq s,t} f(\delout(v)) - f(\delin(v))\\
    &= \sum_{v \neq s} \left( \sum_{vw \in E} f(vw) - \sum_{wv \in E} f(wv) \right)\\
    &= \left( \sum_{\substack{v \neq s\\vs \in E}} f(vs) + f(ss) \right) - \left( \sum_{\substack{v \neq s\\sv \in E}} f(sv) + f(ss) \right)\\
    &= -[\delout(f,s) - \delin(f,s)]
\end{align*}
%
Thus the maximum flow problem tells us a method of maximizing the amount of flow which gets to $t$.

There is an interesting relation between flows on graphs, and another structure on these graphs known as a cut, which we will take full advantage of in finding solutions to the max flow problem. Define a $(s,t)$ cut on a directed graph $G = (V,E)$ with vertices $s$ and $t$, to be a partition of $V$ into two sets, one containing $s$, and one containing $t$. It is however more simple to consider a cut to be a subset $C$ of $V$ containing $s$, but not $t$. If we have a capacity function $\mu$, then we define the value of the cut to be $\text{val}(C) = \mu(\delout(C))$. The min cut problem is to find a cut of smallest value.

\begin{lemma}
    If $f$ is a flow, and $C$ is an $(s,t)$ cut, then $\text{val}(f) \leq \mu(\delout(C))$.
\end{lemma}
\begin{proof}
    Since $t \not \in C$, similar manipulations to the ones above show that
    %
    \begin{align*}
        \text{val}(f) &= \sum_{v \in C} f(\delout(v)) - f(\delin(v))\\
        &= f(\delout(C)) - f(\delin(C))\\
        &\leq \mu(\delout(C))
    \end{align*}
    %
    Note that we obtain equality if $f(\delout(C)) = \mu(\delout(C))$ and $f(\delin(C)) = 0$, in which case $f$ must be a maximal flow, and $C$ a min cut.
\end{proof}

As with maximal matchings, we attempt to find a maximal flow by finding ways of augmenting suboptimal flows to a maximal solution. Given a flow $f$, we construct the residual graph $G_f$, which has the same vertices as $G$, but whose edges are defined to be the set
%
\[ E_f = \{ uv \in E : f(uv) < \mu(uv) \} \coprod \{ vu : uv \in E, f(uv) > 0 \} \]
%
We shall denote the element $vu$ in the second set which makes up $E_f$ by $\overleftarrow{uv}$, since we obtained in from an edge $uv$ in $G$. Also define a capacity function
%
\[ \mu_f(uv) = \mu(uv) - f(uv)\ \ \ \ \ \ \ \ \ \mu_f(\overleftarrow{uv}) = f(uv) \]
%
Suppose we can find a simple path $P$ from $s$ to $t$ in $G_f$, and define $\alpha = \min_{e \in P} \mu_f(e)$. Consider a new flow
%
\[ f'(uv) = \begin{cases} f(uv) + \alpha : uv \in P \\ f(uv) - \alpha : \overleftarrow{uv} \in P \end{cases} \]
%
First we show that $f'$ is a flow. By the choice of $\alpha$, $0 \leq f' \leq \mu$. For each vertex $v \neq s,t$, we must show that flow conservation still holds. Let $e_1, e_1', \dots, e_n, e_n'$ be the edges in $G$ containing $v$ as an endpoint obtained from $P$, by reversing edges of the form $\overleftarrow{uv}$, considered in the order they appear in $P$. We may write pair up the edges in this manner because, for every edge that enters $v$ in $P$, there must be an edge that leaves $v$ in $P$. Let
%
\[ S_i = \begin{cases} f(e_i) - f'(e_i) & e_i = uv \\ f'(e_i) - f(e_i) & e_i = vu \end{cases} \]
%
Define $S_i'$ similarily for $e_i'$. Then, since no edges are repeated,
%
\begin{align*}
    f'(\delout(v)) - f'(\delin(v)) &= f(\delout(v)) - f(\delin(v))\\
    &\ \ \ \ \ + \sum_{i = 1}^n (S_i + S_i')
\end{align*}
%
We now show $S_i + S_i' = 0$ for each $i$. This breaks into several cases.
%
\begin{itemize}
    \item If $e_i = uv$, $e_i' = vw$, in which case $f'(e_i) = f(e_i) + \alpha$, $f'(e_i') = f(e_i') + \alpha$, and $S_i + S_i' = \alpha - \alpha = 0$.
    \item If $e_i = uv$, $e_i' = wv$, then $e_i'$ was obtained from an edge of the form $\overleftarrow{vw}$ in $P$, in which case $f'(e_i') = f(e_i') - \alpha$, and so $S_i + S_i' = -\alpha -(-\alpha) = 0$.
    \item If $e_i = vu$, $e_i' = wv$, then $e_i$ and $e_i'$ were obtained from reversed edges in $P$, and so $S_i + S_i' = (-\alpha) - (-\alpha) = 0$.
    \item If $e_i = vu$, $e_i' = vw$, then $e_i$ was obtained from reversing edges, and so $S_i + S_i' = (-\alpha) + \alpha = 0$.
\end{itemize}
%
And it follows, because flow conservation holds at $v$ for $f$, that it holds at $v$ for $f'$ as well. Finally, we find that $\text{val}(f') = \text{val}(f) + \alpha$, which can be shown by performing an analysis, similar to $v$ above, for $s$, noting that $s$ will have an extra edge at the beginning of the path, which causes the extra $\alpha$. All that remains is to show this augmentation method yields a maximum flow in all cases, after enough iterations.

Let $f$ be a flow for which $G_f$ has no augmenting path. Let $U$ be the nodes reachable from $s$ in $G_f$, easily verified to be a cut.
%
\[ \text{val}(f) = \mu(\delout(U)) \]
%
proving maximality of the flow, and minimality of the cut $U$. To see this, let $uv \in \delout(U)$. Then $f(uv) = \mu(uv)$, for otherwise $v$ would be reachable from $s$ in $G_f$. Similarily, we must have $f(\delin(U)) = 0$, for if $vu \in \delin(U)$, and $f(vu) > 0$, then $\overleftarrow{vu} \in G_f$, and so $v$ is reachable from $s$, a contradiction. We conclude that
%
\[ \text{val}(f) = f(\delout(U)) - f(\delin(U)) = \mu(\delout(U)) \]
%
which shows that our augmenting paths approach works. Notice that if all edge weights are integral, then the augmenting flows are always integral, and thus there exists a maximal flow with interal weights.

The Ford Fulkerson algorithm solves the max flow problem by repeatedly augmenting an initial flow. In principle, this approach is correct, but if you do not find a residual path in a smart way, this algorithm does not terminate in polynomial time for all inputs. If all edge weights are integers $n_1, \dots, n_m$, then each augmentation increases the value of the flow by at least the greatest common factor of the $n_i$. Since there is a bound on the size of a maximal flow, obtained by taking the trivial cut $\{ s \}$, the algorithm will eventually terminate in time proportional to
%
\[ O \left(\frac{n_1 + \dots + n_m}{\lcm(n_1, \dots, n_m)} \right) \]
%
which is exponential in the bit complexity of the representation. We cannot do any better than this, as the graph in the figure above provides an example. If the edge weights are rational, similar results can be obtained by multiplying out denominators.

\begin{figure}
\begin{center}
\begin{tikzpicture}[transform shape]
    \GraphInit[vstyle=Classic]

    \Vertex[x=-3,y=0]{S}
    \Vertex[NoLabel,x=0,y=1.5]{A1}
    \Vertex[NoLabel,x=0,y=-1.5]{A2}
    \Vertex[x=3,y=0]{T}

    \draw[label={$2^n$}, ->, thick](S)--(A1) node[draw=none,fill=none,midway,above] {$2^n$};
    \draw[label={$1$}, ->, thick](A1)--(A2) node[draw=none,fill=none,midway,left] {1};
    \draw[label={$2^n$}, ->, thick](S)--(A2) node[draw=none,fill=none,midway,below] {$2^n$};
    \draw[label={$2^n$}, ->, thick](A1)--(T) node[draw=none,fill=none,midway,above] {$2^n$};
    \draw[->, thick](A2)--(T) node[draw=none,fill=none,midway,below]  {$2^n$};
\end{tikzpicture}
\end{center}
\caption{If our algorithm continuously switches between adding weights from left to right via the central edge, then we will need to compute $O(2^n)$ residual graphs before termination, even though the graph can be represented in $O(n)$ bits.}
\end{figure}

We achieve much better estimates if we use breadth first search to find our residual path -- that is, we always take the shortest path (in length, not in weight) from $s$ to $t$. But this, of course, requires a careful analysis.

\begin{theorem}[Mader]
    Let $G$ be an undirected graph, and $s,t$ vertices not directly connected by an edge. Then the maximal number of vertex-disjoint $(s,t)$ paths is equal to the minimum cardinality of a set $B \subset V - \{ s, t \}$ which contains a vertex in any path from $s$ to $t$.
\end{theorem}
\begin{proof}
    Form a directed bipartite graph $W$, which consists of two copies of the vertices in $G$. If $v$ is a vertex, then we shall separately denote its copies by $v^1$ and $v^2$. If $uv$ is an edge in $G$, attach edges $u^2v^1$ and $v^2u^1$ to $W$, and attach an edge $v^1v^2$ for each vertex $v$. Define a capacity function on $w$ by
    %
    \[ \mu(u^2v^1) = \mu(v^2u^1) = \infty\ \ \ \ \ \mu(v^1v^2) = 1 \]
    %
    Note that any $(s,t)$ path in $G$ leads to an alternating path in $W$ from $s^2$ to $t^1$, which must cross at least one 1-capacity edge because there is no edge from $s$ to $t$.

    Let $f$ be an integer-valued flow in $W$. Then $f \leq 1$, because an infinite capacity edge $u^2 v^1$ satisfies $f(u^2 v^1) \leq f(\delin(u^2)), f(\delout(v^1))$
\end{proof}

\section{Goldberg-Tarjan Push Relabel Algorithm}

Our previous algorithm found max flows in a graph by starting with a flow, then slowly augmenting it, improving the flow value until it is maximum. An alternative idea would be to take a flow-like function, which would be a maximum flow if it obeyed flow conservation, and slowly correcting the flow until we have a maximum flow. Define a {\bf preflow} on a graph with capacity function $\mu$ to be a positive real-valued function $f$, defined on the edges, such that $f \leq \mu$, and $f(\delout(v)) - f(\delin(v)) \geq 0$ for all $v \neq s$. A preflow defines a residual graph $G_f$ in the same way that a flow does. A non-negative integer valued function $\psi$ on vertices is a {\bf distance label} for a particular preflow $f$ if $\psi(s) = |V|$, $\psi(t) = 0$, and $\psi(v) \leq \psi(w) + 1$ for $vw \in G_f$. Notice that if there is a path from $s$ to $t$ in $G_f$, then a distance label cannot exist, because if it has length $k$, then the inequality property gives
%
\[ \psi(s) \leq (k-1) + \psi(t) \leq (|V|-1) + \psi(t) < |V| \]
%
Thus if we continuously keep track of a preflow and a distance label, and keep adjusting the preflow and distance label to obtain a flow with a distance label, we have obtained a maximal flow. In order to describe a method achieving this, we call a vertex $v \neq s,t$ {\bf active} if $f(\delout(v)) - f(\delin(v)) > 0$, and an edge $vw \in G_f$ {\bf admissable} if $\psi(v) = \psi(w) + 1$. The goal is to find a preflow with no active vertices, in which case we have a flow.

We now describe the push-relabel algorithm for finding the maximal flow in a graph. Begin by setting
%
\[ f(e) = \begin{cases} \mu(e) & e = sv \\ 0 & \text{otherwise} \end{cases}\ \ \ \ \ \ \ \ \ \ \psi(v) = \begin{cases} |V| & v = s \\ 0 & v \neq s \end{cases} \]
%
Then $\psi$ is a distance label for the preflow $f$, since $G_f$ has no edges leaving $s$.

Suppose there is an active vertex $v$ relative to $f$. If there are no admissable edges leaving $v$, we may introduce an admissable edge by setting $\psi(v) = \min_{vw \in G_f} \psi(w) + 1$. This leaves the $f$ a preflow, and $\psi$ a distance label. If there is an admissable edge $vw$ leaving $v$, we set $f(vw)$ to be
%
\[ \min[ \mu_f(vw), f(\delout(v)) - f(\delin(v))] \]
%
Then $f \leq \mu$ still holds, and $f(\delout(v)) - f(\delin(v)) \geq 0$, since we can only have let at most $f(\delout(v)) - f(\delin(v))$ flow out. Also $f(\delout(w)) - f(\delin(w)) \geq 0$, since we have only let more flow in. The distance labeling is still a distance labelling, because we introduce an edge $wv$ in $G_f$, which requires that $\psi(w) \leq \psi(v) + 1 = \psi(w) + 2$, and an edge from $vw$, so that $\psi(v) = \psi(w) + 1 \leq \psi(w) + 1$. Thus the fact that $\psi$ is a distance labelling of the preflow $f$ is an invariant of the algorithm.

\begin{lemma}
    If $f$ is a preflow and $\psi$ a distance label for $f$, then if $f(\delout(v)) - f(\delin(v)) > 0$, there is a $(v,s)$ path in $G_f$.
\end{lemma}
\begin{proof}
    s
\end{proof}

\section{Cuts in Undirected Graphs}

Let $G$ be an undirected graph with capacities $\mu \geq 0$. For a subset of vertices $U$ let $\delta(U)$ denote all edges with a single endpoint in $U$. A {\bf cut} in such a graph is $\emptyset \subsetneq U \subsetneq V$, with value $\mu(\delta(U))$. The global minimum cut problem is to find the cut with $\mu(\delta(U))$ minimized, without regard to where any particular pair $(s,t)$ resides. A naive way to calculate the global minimum cut is to calculate a minimum $(s,t)$ cut, for all pairs $(s,t)$. This requires $n^2$ calls to the maximum flow problem.

And while we discuss this problem, we briefly sketch out how to compute a max flow in an undirected graph. Given such a graph $G$ with capacity $\mu$, we define $H$ to be the directed graph with the same vertices as $G$, and for each edge $uv \in G$, two edges $uv$ and $vu$ in $H$, with a capacity function $\psi$ such that $\psi(uv) = \psi(vu) = \mu(uv)$. Given any flow $f$ in the directed graph, we obtain a flow $f'$ in the undirected graph of the same value by subtracting $\min(f(uv), f(vu))$ from each edge. Thus we can compute a max flow in $O(V^2E)$ time in an undirected graph as well. The value of min cuts in the directed graph is the same as the value of min cuts in the undirected graph, so the max flow min cut theorem holds here as well.

More smartly, we notice that if $s$ is fixed, $s$ lies on either side of the cut, and since $\mu(\delta(U)) = \mu(\delta(U^c))$, we need only try each $(s,t)$ cut as $t$ ranges over all $t \neq s$, which only requires $n-1$ calls to the max flow function. In a directed call, we need try all $(s,t)$ cuts and $(t,s)$ cuts, which requires $2(n-1)$ calls to max flow.

We can do much better than just computing a min-cut in $O(n)$ times. It turns out that there is a way to compactly represent all minimum cuts via $O(n)$ calls to a min-cut algorithm. The first step to the representation is to find a useful relationship between the values of the numbers $\lambda_{u,v}$, the capacities of a min $(u,v)$ cut.

\begin{lemma}
    Let $v_1, \dots, v_k$ be vertices. Then $\lambda_{v_1, v_k} \geq \min(\lambda_{v_1,v_2}, \dots, \lambda_{v_{k-1},v_k})$.
\end{lemma}
\begin{proof}
    Assume that $k \geq 3$, since the proof is trivial for $k = 2$. Let $C$ be a $(v_1,v_k)$ cut with $\mu(\delta(C)) < \min(\lambda_{v_1,v_2}, \dots, \lambda_{v_{k-1},v_k})$. Inductively, we conclude that $v_2, \dots, v_k \in C$. But then by definition $C$ cannot be a $(v_1, v_k)$ cut.
\end{proof}

Let $T = (V,E)$ be a tree, and $e \in E$ an edge in the tree. Then the set of nodes in some connected components of the graph $T' = (V, E - \{ e \})$ is known as the {\bf fundamental cut}. There are only two connected components, one for each vertex on the ends of $e$. A {\bf Gomory-Hu Tree} for an undirected graph $G$ is a tree $T$ with the same vertices as $G$, such that the fundamental cut for every edge $uv \in T$ is a minimal $uv$ cut. It turns out that Gomory-Hu trees always exist, can be computed in $O(n)$ calls to a min $(s,t)$ cut function, and compactly represent every min cut in a graph, for if $(v_1, \dots, v_n)$ is the unique simple path in $T$, then
%
\[ \lambda_{v_1,v_n} = \min(\lambda_{v_1,v_2}, \dots, \lambda_{v_{n-1},v_n}) \]
%
Certainly the left side is $\geq$ to the right side. On the other hand, by construction of the tree, the fundamental cut of $(v_i,v_{i+1})$ is also a $(v_1,v_n)$, hence the left side is $\leq$ to the right side.

\begin{theorem}[Cut Submodularity]
    For $A,B \subset V$, we have
    %
    \[ \mu(\delta(A)) + \mu(\delta(B)) \geq \mu(\delta(A \cap B)) + \mu(\delta(A \cup B)) \]
\end{theorem}
\begin{proof}
    Fix an edge $uv$. We shall show that every edge counted on the right side occurs at least as many times on the left side.
    %
    \begin{itemize}
        \item Suppose $uv \in \delta(A \cap B)$, $uv \not \in \delta(A \cup B)$. Then we may assume $u \in A \cap B$, $v \not \in A \cap B$, $v \in A \cup B$. Then $\mu(uv)$ is counted once on the right hand side, and once on the left hand side.
        \item Suppose $uv \not \in \delta(A \cap B)$, $uv \in \delta(A \cup B)$. Thus we may assume $u \in A \cup B$, $v \not \in A \cup B$, $u \not \in A \cap B$. Then $\mu(uv)$ is counted once on the right hand side, and once on the right hand side as well.
        \item Suppose $uv \in \delta(A \cap B)$, $uv \in \delta(A \cup B)$. Then we may assume $u \in A \cup B$, $v \not \in A \cup B$, and so $u \in A \cap B$. Then $\mu(uv)$ is counted once on the right hand side, and twice on the left hand side.
    \end{itemize}
    %
    Equality holds when $\delta(A \cap B)$ is disjoint from $\delta(A \cup B)$.
\end{proof}

\begin{lemma}
    Let $s \neq t$ be two vertices in a graph $G$, and let $C$ be a min $(s,t)$ cut. If $u \neq v$ are also distinct vertices, then there is a min $(u,v)$ cut $D$ with $C \subset D$ or $D \cap C = \emptyset$.
\end{lemma}
\begin{proof}
    Note that the statement is really redundant, because $C \subset D$ occurs exactly when $D^c \cap C = \emptyset$. Let $B$ be any minimum $(u,v)$ cut. We will modify $B$ to another min cut with the desired properties. Note that
    %
    \[ \mu(\delta(C)) + \mu(\delta(B)) \geq \mu(\delta(C \cap B)) + \mu(\delta(C \cup B)) \]
    %
    We may assume $s,u \in B$. Then $t \not \in B$, $C \cap B$ is an $(s,t)$ cut, hence $\mu(\delta(C \cap B)) \geq \mu(\delta(C))$, and so
    %
    \[ \mu(\delta(B)) \geq \mu(\delta(C \cup B)) \]
    %
    Now $u \in C \cup B$, and $v \not \in C \cup B$ (WHY?), so $D = C \cup B$ satisfies the properties of the lemma.
\end{proof}

\section{Potentials}

Fix a directed graph $G$ with edge costs $c$. A {\bf potential} for $G$ is a mapping $\phi: V \to \mathbf{R}$ such that
%
\[ c_\phi(uv) = c(uv) - (\phi(v) - \phi(u)) \geq 0 \]
%
for al edges $uv$ in $G$. The costs $c_\phi$ are known as the potential costs.

\begin{lemma}
    For any $s-t$ path $P$ in $G$,
    %
    \[ c_\phi(P) = c(P) - (\phi(t) - \phi(s)) \]
\end{lemma}
\begin{proof}
    If $P = (v_1, \dots, v_n)$, then we have an alternating sum
    %
    \[ c_\phi(P) = \sum_{k = 1}^{n-1} c(v_k v_{k+1}) + \phi(v_k) - \phi(v_{k+1}) = \sum_{k = 1}^{n-1} c(v_k v_{k+1}) + \phi(v_1) - \phi(v_n) \]
    %
    and the right side is exactly the form considered in the statement of the theorem.
\end{proof}

This is a discrete version of the fact that the energy change of a particle under a conservative force is invariant of the path of the particle, but only the change of the particles position in space. A simple corollary to the lemma is that the minimal cost paths relative to the metric $c$ are the same as the minimal cost paths relative to $c_\phi$.

If a graph has negative edge weights, but possesses some potential $\phi$, then we can compute minimal paths under the cost function $c_\phi$ (which is now has positive edge weights everywhere) using Dijkstra's algorithm. There is a simple method to finding a potential, which we now detail.

If $P$ is a cycle in a graph $G$ with potential $\phi$, we calculate
%
\[ c_\phi(P) = c(P) \geq 0 \]
%
so that $G$ has no negative weight cycles. Conversely, if $G$ has no negative weight cycles, then it turns out that $G$ has a potential, and this potential can be computed in polyomial time.

\begin{theorem}
    A potential can be computed for a graph $G$ can be computed in $O(nm)$ time.
\end{theorem}
\begin{proof}
    Given $G$, compute a graph $H$ by adding a new vertex $r$ to $G$, and adding edges from $r$ to every vertex in $G$. Set $c(rv) = 0$ for all vertices $v \in V$, and let $\phi(v)$ be the cost of the cheapest walk from $r$ to $v$. This is well defined since $G$ has no negative weight cycles (there is a cheapest walk with no repeated vertices, for if a path has a repeated vertex, it contains a cycle, which must have non-negative weight, and this cycle can be removed with no penalty to the length of the walk -- we then need only consider the finite set of paths with no repeated vertices to obtain an answer). Note that $c(uv) + \phi(u) \geq \phi(v)$, so $\phi$ is a potential for $G$. The Bellman-Ford can compute $\phi$ in $O(nm)$ time.
\end{proof}

Let $\omega$ be a differential 1-form on a manifold $M$ such that
%
\[ \int_\gamma \omega \geq 0 \]
%
for any cycle $\gamma$. Then if $\gamma$ and $\delta$ are two paths from a point $x$ to a point $y$, we find
%
\[ \int_\gamma \omega - \int_\delta \omega, \int_\delta \omega - \int_\gamma \omega \geq 0 \]
%
so $\int_\gamma \omega = \int_\delta \omega$, and given a fixed $x$, we can define
%
\[ F(y) = \int_\gamma \omega \]
%
where $\gamma$ is a path from $x$ to $y$, and one verifies that $dF = \omega$, so that
%
\[ \int_\gamma \omega - (F(y) - F(x)) = 0 \]
%
a stronger fact which is analogous to the result above, although our result is weaker because in directed graphs we cannot perform the swapping trick to obtain equality of all paths.

\section{Minimum Cost Flows}

Throughout this section, we fix a directed graph with edge costs $c$, edge capacities $\mu$, fixed vertices $s,t$, and a target flow value $\gamma$. The minimum cost flow problem tries to find a flow $f$ with value $\text{val}(f) \geq \gamma$, which minimizes the cost $\text{cost}(f) = \sum c(e) f(e)$, or to determine if no such flow exists.

We can certainly compute in polynomial time if a flow exists satisfying the constraints, because we can find a maximal flow in the graph. It is not so clear that we can find such a flow with minimal cost. It shall always be the case that the minimal flow satisfies $\text{val}(f) = \gamma$, because given any flow $f$, the value of the flow $g = \text{val}(f)/\gamma f$ satisfies $\text{val}(g) = \gamma$, and $\text{cost}(g) = \text{val}(f)/\gamma \sum c(e) f(e)$, which will be smaller than $\text{cost}(f)$ if $\text{val}(f) \neq \gamma$.

\section{Minimal Cost Bipartite Matching}

We now take a brief aside to discuss how minimal cost flows can be used to solve the analogous problem in bipartite graphs. Given the name, the minimal cost bipartite matching problem should be fairly self explanatory. We take a particular Bipartite graph $G = (L \cup R, E)$ with edge costs $c$, and we must find a minimum cost perfect matching on $G$ in polynomial time. That is, our goal is a perfect matching $M$ which minimizes
%
\[ \text{cost}(M) = \sum_{e \in M} c(e) \]
%
The similarity to the minimal cost flow problem immediately suggests a connection between the two problems.

We can formulate the perfect matching problem as a minimal cost max flow problem, by modifying slightly the standard reformulation of the maximal matching problem as a max flow problem. The original process is to construct a directed bipartite graph $H$, whose vertices are obtained from $G$ by adding two new vertices $s$ and $t$, orienting all edges in $E$ so that they point from $L$ to $R$, and adding new edges $sv$, for $v \in L$, and $wt$, for $w \in R$. If we define a capacity function $\mu = 1$, then there is a one-to-one correspondence between integral valued $s-t$ flows and matchings in the graph. To see this, note that an integral valued flow $f$ gives rise to a characteristic function $\chi_{A_f}$ on the edges of $G$, and the flow conservation law shows that if $e$ and $e'$ are adjoint, then $\chi_{A_f}(e) + \chi_{A_f}(e') \leq 1$, so that $A_f$ forms a matching on $G$. Correspondingly, the reverse process takes a matching to a characteristic function on the edges of $G$, which can easily be completed into a flow on $H$.

Now we define a cost function $c'$ on $H$ in a way which mirrors the cost function $c$ on $G$. We let
%
\[ c'(sv) = c'(wt) = 0\ \ \ \ \ c'(vw) = c(vw) \]
%
Then we see that the correspondence described above preserves the cost function. That is, if a matching $M$ corresponds to a flow $f$, then $\text{cost}(M) = \text{cost}(f)$, because $f$ is obtained from $M$ by adding edges of the form $sv$ and $wt$, which add no value to the cost of the flow. Thus if we find a minimal cost maximal flow in $H$, we can find a minimal cost perfect matching in $G$, and we know we can find such a maximal flow using the successive shortest paths algorithm, and we need only compute $|L| = |R|$ successive paths, since each iteration of the algorith improves the length of the candidate path, hence now have an algorithm for the minimal cost perfect matching problem which runs in $O(n(m + n \log n))$ time.

\section{Minimum Mean Cycle Cancelling}

We already know that a flow $f$ has minimal cost if $G_f$ contains no negative-cost cycles. This suggests that a strategy for finding minimal cost flows is obtained from successively pruning negative-cost cycles from $G_f$ until there are none left. As with Ford Fulkerson, we must be careful which cycles we prune in order to guarantee a good asymptotic speed to our algorithm. It turns out that fast asymptotic results are available if we prune negative cycles with `minimum mean'.

\begin{algorithm}
\caption{The Minimum Mean Cycle Canceling Algorithm}
\begin{algorithmic}[1]
\State $f \gets$ any $(s,t)$ flow of value $\gamma$ (found in $O(n^3 \sqrt{m})$ time)
\While {$G_f$ contains negative weight cycles}
\State $C \gets$ a cycle in $G$ which minimizes its mean cost $c(C)/|C|$.
\State Augment $f$ along $C$.
\EndWhile
\State \Return $f$
\end{algorithmic}
\end{algorithm}

The algorithm certainly terminates, since the cost of $f$ decreases for every iteration of the algorithm, at a rate bounded by the rational values of the edge costs. It remains to be seen that the algorithm terminates in a polynomial amount of time (and this is not immediately obvious, and is not true if we do not take cycles which minimize the mean cost).

First, we argue why we can find a minimum ratio cycle $C$ in polynomial time.

\begin{lemma}
    Any Eulerian multigraph $H = (W,F)$ with edge costs $c$ contains a cycle $C$ such that
    %
    \[ \frac{c(C)}{|C|} \leq \frac{c(F)}{|F|} \]
\end{lemma}
\begin{proof}
    Perform a cycle decomposition on $H$, writing $F = C_1 \amalg \dots \amalg C_n$, we find
    %
    \[ \frac{c(F)}{|F|} = \frac{c(C_1) + \dots + c(C_n)}{|C_1| + \dots + |C_n|} \]
    %
    If $c(C_i)/|C_i| > c(F)/|F|$ for each $i$, then $c(C_i) |F| > c(F) |C_i|$ also holds for each $i$, and so, summing up we find
    %
    \[ c(F) |F| = \sum_i c(C_i) |F| > \sum_i c(F) |C_i| = |F| c(F) \]
    %
    a clear contradiction which proves the existence of a $C_i$ satisfying the constraints of the lemma.
\end{proof}

Given any cycle $C$, the corresponding multigraph (which contains duplicates of repeated edges in the cycle) is Eulerian, and can thus be broken up into disjoint cycles. The lemma above shows that if $C$ is not a simple cycle, we can always find a strictly smaller cycle $C'$ without increasing the mean value of the cycle, so that we need only search simple cycles to find a minimum mean.

For $k \leq n$, define $\delta_k(v)$ to be the cheapest closed walk from $v$ to itself using exactly $k$ (not necessarily distinct) edges. If we let $\delta_k(v,w)$ be the cheapest path from $v$ to $w$ using $k$ edges, then we find
%
\[ \delta_k(v,w) = \min_{u \in V}\ c(v,u) + \delta_{k-1}(u,w) \]
%
Each value of $\delta_k(v,w)$ can be computed in $O(m)$ time, and so the set of all $\delta_k(v,w)$ can be computed in $O(mn^3)$ time. Let $v^*$ and $k^*$ minimize $\delta_{k^*}(v^*)/k^*$, and pick $k^*$ to be smallest to break ties. Then the walk from $v^*$ in $k^*$ stops is verified to be a cycle, from the above remark, and thus is a minimum ratio cycle, since it is a minimum ratio walk in general.

\begin{theorem}
    The number of iterations of the minimum mean cycle cancelling algorithm is $O(m^2 n \log n)$.
\end{theorem}
\begin{proof}
    Consider the sequence of flows
    %
    \[ f_1, f_2, \dots, f_N \]
    %
    which are found in the MMCC algorithm, together with the cycles
    %
    \[ C_1, \dots, C_{N-1} \]
    %
    which are augmented to produce the sequence of flows. For $i < j$, define
    %
    \[ \alpha(i,j) = \{ e, \overleftarrow{e} : e, \overleftarrow{e} \in C_i \cup C_j \} \]
    %

    \begin{lemma}
        If $\alpha(i,k) = \emptyset$ for $i < k < j$, then we find
        %
        \[ \frac{c(C_i)}{|C_i|} \leq \frac{n}{n-|\alpha(i,j)|}\ \frac{c(C_j)}{|C_j|} \]
    \end{lemma}
    \begin{proof}
    %
    Let $H = C_i \coprod C_j - \alpha(i,j)$ (essentially, $H$ is obtained from the cycles $C_i$ and $C_j$ by keeping duplicate edges, and `cancelling out' edges with their reverse). Then $H$ is Eulerian, for $C_i \coprod C_j$ is Eulerian, and removing edges along with their reverse changes both the in degree and out degree of a vertex by the same amount. Furthermore, each edge in $H$ is in $E_{f_i}$ because $\alpha(i,k) = \emptyset$, so
    %
    \begin{align*}
        \frac{c(C_i)}{|C_i|} (|C_i| + |C_j|) &= \frac{c(C_i)}{|C_i|} (|H| + |\alpha(i,j)|)\\
        &\leq c(H) + |\alpha(i,j)| \frac{c(C_i)}{|C_i|}\\
        &= c(C_i) + c(C_j) + |\alpha(i,j)| \frac{c(C_i)}{|C_i|}
    \end{align*}
    %
    where we have used the fact that $c(H) = c(C_i) + c(C_j)$ since the cost of reverse edges cancels out. This can be rephrased as
    %
    \[ \frac{c(C_i)}{|C_i|} \left( |C_j| - |\alpha(i,j)| \right) \leq \frac{c(C_j)}{|C_j|} |C_j| \]
    %
    And dividing out, we find
    %
    \[ \frac{c(C_i)}{|C_i|} \leq \frac{|C_j|}{|C_j| - |\alpha(i,j)|} \frac{c(C_j)}{|C_j|} \]
    %
    and this inequality can be weakened to the form above, because $x/(x-y)$ is a decreasing function of $x$ for $y > 0$, and $n/(n-|\alpha(i,j)|) \geq 0$.
    \end{proof}

    Returning to our original proof, we see that $j = i+1$ satisfies the theorem, and proves that the minimum mean ratios are non-decreasing. If $i < j$ is the smallest value such that $\alpha(k,i) \neq \emptyset$, for some $i \leq k < j$, then we find
    %
    \[ \frac{c(C_i)}{|C_i|} \leq \frac{n}{n - 2} \frac{c(C_j)}{|C_j|} = \left(1 - \frac{n}{2} \right) \frac{c(C_j)}{|C_j|} \]
    %
    and we have a bound $j \leq i + m$. To see this, note that at least one edge is removed in the residual graph from each of the cycles $C_i, C_{i+1}, \dots, C_{i+m}$ during the algorithm. Thus we conclude
    %
    \[ \frac{c(C_i)}{|C_i|} \leq \left( 1 - \frac{n}{2} \right)^n \frac{c(C_{i+nm})}{|C_{i+nm}|} \leq e^{-1/2} \frac{c(C_{i+nm})}{|C_{i+nm}|} \leq 2 \frac{c(C_{i+nm})}{|C_{i+nm}|} \]
    %
    If the edge weights are integral (and all rational cost problems can be reduced to this form), this already gives us enough information to prove a polynomial time bound. Note that since the minimum mean cycles are simple, we must have
    %
    \[ \frac{|c(C_i)|}{|C_i|} \geq \frac{1}{n} \]
    %
    and
    %
    \[ \frac{|c(C_1)|}{|C_1|} \leq \min_{e \in E} |c(e)| \]
    %
    so if $N$ is large enough that $2^{-N} \min_{e \in E} |c(e)| \leq n^{-1}$, then
    %
    \[ \frac{c(C_{i+Nnm})}{|C_{i+Nnm}|} \geq \frac{1}{2^N} \frac{c(C_1)}{|C_1|} \geq - \frac{\min_{e \in E} |c(e)|}{2^N} \geq -\frac{1}{n} \]
    %
    which forces the mean cycle ratio to be positive, and hence the algorithm will terminate. We can choose any $\lg(n \min_{e \in E} |c(E)|) < N$, so the algorithm terminates in at most
    %
    \[ nm  \lceil \lg n \min_{e \in E} |c(E)| \rceil \]
    %
    iterations, which is polynomial in the bit complexity of the algorithm, since if the costs are given with $l$ bits, then the total bit complexity is about $\Theta(n + ml)$, and $|c(E)| \leq 2^l$, so
    %
    \[ nm \lceil \lg n \min_{e \in E} |c(E)| \rceil \leq nm \lg(n) + lnm \]
    %
    This finishes our initial analysis of the mean cycle canceling algorithm.
\end{proof}

The polynomial bound for this algorithm is absolutely awful, and we can do much much better. In the next lecture we will begin the arguments that give us a much better bound on the runtime on this algorithm.

\section{Minimum Spanning Trees and Arboresences}

Let $G$ be an undirected graph, together with a (possibly-negative) edge cost function $c$. A {\bf spanning tree} $T$ on $G$ is a connected subset of edges containing no cycles (a tree) such that every vertex is covered by an edge in the tree. Note that this definition is equivalent to $T$ having no cycles, and $n-1$ edges (a fact proven most easily by induction, by removing vertexes and edges from $T$ which form the leaves of $T$). There is another equivalent definition which is more important to the spanning tree problems we'll discuss. $T$ is a spanning tree if it has $n-1$ edges, and $|E(S)| \leq |S| - 1$ for all vertex sets $S$ (where $E(S) = \{ uv : u,v \in S, uv \in T \}$).

\[ TODO: Draw a picture of S bounding the number of edges in the set S \]

The minimum spanning tree problem is self explanatory -- given $G$, find a spanning tree containing the fewest edges. We can formulate the problem as an $LP$ in exponentially many constraints in functions $x: E \to \mathbf{R}$,
%
\begin{gather*}
\min \quad \sum_e c(e) x(e) \\
\begin{aligned}
\textup{s.t.}\quad x(E(S)) &\geq |S| - 1\ \ \ \forall S \subsetneq V\\
                   x(E(V)) &= n - 1 \\
                   0 &\leq x
\end{aligned}
\end{gather*}
%
Note that the constraint $x \leq 1$ is already encoded in the problem by taking $S$ to be a pair of vertices corresponding to an edge. Our discussion above entails that any integral solution to this algorithm corresponds to a spanning tree in $G$. It can be shown that all extreme points are integral, but we shall only show that we can choose an optimal extreme point which is integral.

In it's current form, the minimal spanning tree linear program is unfeasible, having far too many constraints. We obtain a much better optimization problem if we switch to the dual of this program. This can be motivated by the fact that $x$ behaves more like a linear functional than a vector in this linear program (the constraints of the program are expressed as $x$'s operations on the vector space $\mathbf{R} \cdot E$ generated by the edges $E$), so that the program is the dual of some other, more easily understood linear program. The dual program finds functions $y: 2^V - \{ \emptyset \} \to \mathbf{R}$ which satisfy
%
\begin{gather*}
\max \quad\ \  -\sum (1 - |S|) y(S) \\
\begin{aligned}
\ \ \ \ \ \ \ \ \ \ \ \ \ \ \ \ \ \ \ \ \textup{s.t.} \quad \sum_{e \in E(S)} y(S) &\geq -c(e)\ \ \ \ \forall e \in E(S) \\
                   y(S) &\geq 0\ \ \ \ \ \ \ \ \ \ \ \ \forall S \neq V
\end{aligned}
\end{gather*}
%
Kruskal's algorithm, a method for finding the minimal spanning tree, can be viewed as a combinatorial method to solving the dual LP of the spanning tree problem. We first recall the simple, greedy method to form a spanning tree.
%
\begin{algorithm}
\caption{Kruskal's Algorithm}
\begin{algorithmic}[1]
\State Set $T = \emptyset$, $K = E$.
\While {$T$ is not a spanning tree}
\State Remove $e \in K$ with minimal weight.
\State Append $e$ to $T$ if it connects two connected components of $T$.
\EndWhile
\State \Return $T$
\end{algorithmic}
\end{algorithm}
%
We can verify this algorithm's correctness using the duality of linear programming. Suppose that $T$ contains the edges $e_1, \dots, e_n$, which are placed in the order they were added to $T$. Let $S_k$ be the component containing $e_k$ in the graph consisting only of the edges $e_1, \dots, e_k$ (it is the component that was freshly merged together in the $k$'th iteration of the algorithm). Let $x: E \to \{ 0, 1 \}$ be the indicator function of $T$, and $y: 2^V - \{ \emptyset \} \to \mathbf{R}$ be defined by letting $y(S_i) = c(e_j) - c(e_i)$, where $j$ is the smallest index greater than $i$ such that one of the endpoints of $e_j$ contains points in $X_i$, and define $y(S_n) = y(V) = - c(e_n)$. Define $y(S) = 0$ otherwise. For any edge $e$, by the telescoping sum property of our definition we have
%
\[ \sum_{e \subset S} y(S) = -c(e_i) \]
%
where $i$ is the smallest index such that $X_i$ contains both endpoints of $e$. The way we selected edges guarantees that $c(e_i) \leq c(e)$, so that our constraints are satisfied, and our tight for $e_1, \dots, e_n$.

We now verify complementary slackness, so that $x$ is verified optimal. If $x(e) > 0$, then our calculation above shows
%
\[ - \sum_{e \subset S} y(S) = c(e) \]
%
If $y(S) > 0$, then $S = S_i$ for some $i$, hence $S_i$ is a connected tree, and contains $n-1$ vertices, so $x(E(S)) = |S| - 1$. Complementary slackness guarantees that $x$ and $y$ are optimal solutions to their corresponding algorithms, verifying correctness of Kruskal's algorithm.

\section{Arboresences}

An (out) {\bf arboresence} in a directed graph with root $s$ is a subset $T$ of $n-1$ edges, such that there is a unique directed path from $s$ to any other vertex. Given a cost function $c \geq 0$ and vertex $s$, we want to try and find the min-cost arboresence at $s$. Like the minimal cost spanning tree problem, we can express this problem as an LP,
%
\begin{gather*}
\min \quad\ \ \sum_e c(e) x(e) \\
\begin{aligned}
\ \ \ \ \ \ \ \ \ \ \ \ \ \ \ \ \ \ \ \ \textup{s.t.} \quad x(\delta^{\text{in}}(U)) &\geq 1\ \ \ \ \forall U \subset V - \{ s \} \\
                  x &\geq 0
\end{aligned}
\end{gather*}
%
If $T$ is an arbitrary arboresence at $v$, then the corresponding characteristic function certainly is a feasible solution to the problem. The problem with this LP is that there are solutions (possibly even optimal ones) which don't look like arboresences. However, we shall not use the LP to solve linear programs, but instead use properties of linear programs to guarantee optimality. We can form a simple argument to show the existence of optimal solutions which are arboresence. First, note that we only every have upper bounds to the algorithm, and there is no need to have $x(e) > 1$ to satisfy any of the constraints, due to the positivity of $x$, so we may always assume that an optimal integral solution takes value in $\{ 0, 1 \}$, and thus correspond to characteristic functions of edges. Second, we see that the subgraph formed contains paths from $v$ to any other vertex. Since $x(\delta^{\text{in}}(V - s) \geq 1$, there is an edge connecting $s$ to some over vertex $v_1$. Then $x(\delta^{\text{in}}(V - \{ s, v_1 \}) \geq 1$, so there is some edge from $s$ to $v_1$, or $v_1$ to $v_2$. In either case, there is an edge to $v_2$. Continuing this process gives you a path to any vertex in the graph. We may now assume our subgraph is a tree, because we can always remove edges to obtain an algorithm that is at least as optimal. Thus we have argued that there is an optimal solution which is an arboresence, like we are looking for.

Let us form the dual LP, so we can use complementary slackness to verify optimality.
%
\begin{gather*}
\max \quad\ \ \sum_S y(S) \\
\begin{aligned}
\ \ \ \ \ \ \ \ \ \ \ \ \ \ \ \ \ \ \ \ \textup{s.t.} \quad \sum_{e \in \delta^{\text{in}}(S)} y(S) &\leq c(e)\\
                  y &\geq 0
\end{aligned}
\end{gather*}
%
We now describe the algorithm which gives us minimal cost arboresences. Note that $y$ is only included here for purposes of algorithmic correctness, and needn't be coded up in real algorithms.

\begin{algorithm}
\caption{Edmond's Algorithm}
\begin{algorithmic}[1]
\State $y_{\{ v \}} = \min c(uv)$, for $v \neq r$.
\State Set $F = \{ e_w : w \in V - \{ r \} \}$, where $e_w$ is the cheapest edge entering $w \neq r$.
\While {$F$ has a cycle}
\State Let $G'$ be the graph obtained by contracting each cycle in $(G,F)$.
\State Define $c'(uv) = c(uv) - y(\{v\})$ if $v$ is on a cycle, else $c'(uv) = c(uv)$.
\State Let $(T',y')$ be a min-cost arboresence in $G'$.
\State Define $T = (F \cup T') - \{ uv : u\ \text{on a cycle} wv \in T' \}$.
\State Set $y(S) = 0$ if $S$ cuts a cycle, and $y'(S)$ if $S = [S]$ after contraction.
\EndWhile
\State \Return $F$
\end{algorithmic}
\end{algorithm}

* DRAW A PICTURE OF CONTRACTING CYCLES

\chapter{Matroids}

If you have a problem, and a matroid is involved, you can probably use a greedy algorithm to solve the problem. This is very useful, because greedy solutions are normally far faster than conservative methods. A matroid is a set of sets which can be `greedily expanded' without losing the properties defining a matroid. Formally, a matroid is a collection of subsets of some set $X$ containing the emptyset, closed under the subset relation, and such that if two sets $A$ and $B$ are in the matroid with $|A| < |B|$, there is $z \in B - A$ such that $A \cup \{ z \}$ is in the matroid.

\begin{example}
    If $G$ is a graph, the set of edge subsets which contain no cycle forms a matroid. Certainly it contains the emptyset, and certainly a subset of edges with no cycle contains no cycle.
\end{example}

\begin{example}
    If $V$ is a vector space, the collection of linearly independant vector sets forms a matroid. Essentially, you can greedily expand vectors to form a basis.
\end{example}

\begin{example}
    If $X$ is a finite set, and $L$ is a laminar family over $X$, together with a function $f \geq 0$ defined on the Laminar family, then the same of subsets $A$ such that $|A \cap B| \leq f(B)$ for each $B \in L$, forms a matroid known as a laminar family.
\end{example}

\begin{example}
    If $G$ is a directed graph, and $A, B \subset V$ are given, the set of subsets of $A$ which contains vertex disjoint paths into $B$ is a matroid, known as a Gammoid.
\end{example}

\begin{example}
    Given a connected graph $G$, the set of bond matroid.
\end{example}

\begin{example}
    Given a field extension $F \subset K$, the set of finite subsets $Y$ of $F$ with trancendence degree $|Y|$ forms a matroid, known as the algebraic matroid.
\end{example}

An independence oracle is an algorithm to decide whether an arbitrary subset is an element of a given matroid in polynomial time. This is key to having efficient algorithms for solving matroids.

A base of a matroid is a maximal set -- an element $B$ such that if $B \subset B'$, and $B'$ is in the matroid, then $B = B'$. If our matroid consists of finite sets, then two bases have the same cardinality. Indeed, if $|B| < |B'|$, there is $z \in B' - B$ such that $B + z$ is in the matroid, which cannot occur since this is an expansion of $B$.

The first algorithm we consider is to efficiently find a min weight base for a matroid over a set $X$, for some function $c: X \to \mathbf{R}$. This is a generalization of the minimum spanning tree problem. The algorithm to do this is to just greedily add the minimum element of $X$ to a set until we cannot anymore. The set $B$ we end up with is a base by construction, because the order we add elements is immaterial. To prove that $B$ is minimal, let $C$ be a min-weight base. Order the items in $B$ by weight, $B = \{ x_1, \dots, x_n \}$, $C = \{ y_1, \dots, y_n \}$. Suppose that $x_1 = y_1, \dots, x_k = y_k$, for $k < n$, but $x_{k+1} \neq y_{k+1}$. Consider $C - \{ y_n \}$. Since $|C - \{ y_n \}| < |B|$, there is a minimal $i > k$ such that $C' = C - \{ y_n \} + \{ x_i \}$ is in the matroid, and therefore a base. But by minimality, $c(y_n) \leq c(x_i)$.

\end{document}