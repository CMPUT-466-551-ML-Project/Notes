\documentclass[dvipsnames,letterpaper,12pt]{article}

\usepackage[margin = 1.0in]{geometry}
\usepackage{amsmath,amssymb,graphicx,mathabx,accents}
\usepackage{enumerate,mdwlist}

%\setlist[enumerate]{label*={\normalfont(\Alph*)},ref=(\Alph*)}

\numberwithin{equation}{section}

\usepackage{tikz, tkz-berge, tkz-graph}
\usetikzlibrary{patterns,arrows,decorations.pathreplacing}

\usepackage{color,xcolor}
\definecolor{crimsonred}{RGB}{132,22,23}
\definecolor{darkblue}{RGB}{72,61,139}

\usepackage{amsthm}
\theoremstyle{plain}
\newtheorem{theorem}{Theorem}
\newtheorem{lemma}{Lemma}
\newtheorem{corollary}{Corollary}
\newtheorem*{example}{Example}
\theoremstyle{remark}
\newtheorem*{remark}{Remark}
\newtheorem*{remarks}{Remarks}

\usepackage{nag}

\DeclareMathOperator{\minkdim}{\dim_{\mathbf{M}}}
\DeclareMathOperator{\hausdim}{\dim_{\mathbf{H}}}
\DeclareMathOperator{\lowminkdim}{\underline{\dim}_{\mathbf{M}}}
\DeclareMathOperator{\upminkdim}{\overline{\dim}_{\mathbf{M}}}

\DeclareMathOperator{\lhdim}{\underline{\dim}_{\mathbf{M}}}
\DeclareMathOperator{\lmbdim}{\underline{\dim}_{\mathbf{MB}}}

\DeclareMathOperator{\RR}{\mathbf{R}}
\DeclareMathOperator{\ZZ}{\mathbf{Z}}
\DeclareMathOperator{\QQ}{\mathbf{Q}}
\DeclareMathOperator{\Prob}{\mathbf{P}}
\DeclareMathOperator{\Expect}{\mathbf{E}}

\DeclareMathOperator{\B}{\mathcal{B}}









\title{Large Sets Avoiding Rough Patterns}
\author{Jacob Denson\thanks{University of British Columbia, Vancouver BC, \{denson, malabika, jzahl\}@math.ubc.ca.} \and Malabika Pramanik\footnotemark[1] \and Joshua Zahl\footnotemark[1]}

\begin{document}

\maketitle

\begin{abstract}
	The pattern avoidance problem seeks to construct a set $X\subset \RR^d$ with large dimension that avoids a prescribed pattern. Examples of such patterns include three-term arithmetic progressions (solutions to $x_1 - 2x_2 + x_3 = 0$), or more general patterns of the form $f(x_1, \dots, x_n) = 0$. Previous work on the subject has considered patterns described by polynomials, or by functions $f$ satisfying certain regularity conditions. We consider the case of `rough' patterns, not prescribed by functional zeros.

	There are several problems that fit into the framework of rough pattern avoidance. As a first application, if $Y \subset \RR^d$ is a set with Minkowski dimension $\alpha$, we construct a set $X$ with Hausdorff dimension $1-\alpha$ such that $X+X$ is disjoint from $Y$. As a second application, if $C$ is a Lipschitz curve, we construct a set $X \subset C$ of dimension $1/2$ that does not contain the vertices of an isosceles triangle.
\end{abstract}



A major question in modern geometric measure theory is whether sufficiently large sets are forced to contain copies of certain patterns. Intuitively, one expects the answer to be yes, and many results in the literature support this intuition. For example, the Lebesgue density theorem implies that a set of positive Lebesgue measure contains an affine copy of every finite set.
%
% CHANGE: Why not make this more precise, at the cost of making it less general. Since this is only an introduction we don't need a complete discussion of all results. Furthermore, discussing colinearity more precisely links to the example in the next paragraph.
%
% ORIGINAL: If $X \subset \RR^d$ has large Hausdorff dimension, then it must contain many points that lie in a lower dimensional plane section (see e.g. \cite[Thm 6.8]{Matilla}).
%
And any set $X \subset \RR^2$ with Hausdorff dimension exceeding one must contain three collinear points.
%
On the other hand, there is a distinct genre of results that challenges this intuition. Keleti \cite{KeletiDimOneSet} constructs a set $X \subset \RR$ that avoids all solutions of the equation $x_2 - x_1 = x_4 - x_3$ with $x_1 < x_2 \leq x_3 < x_4$, and which consequently does not contain any nontrivial arithmetic progression. Maga \cite{Maga} constructs a set $X \subset \RR^2$ of full Hausdorff dimension such that no four points in $X$ form the vertices of a parallelogram. The pattern avoidance problem (informally stated) asks: for a given pattern, how large can the dimension of a set $X \subset \RR^d$ be before it is forced to contain a copy of this pattern? 

One way to formalize the notion of a pattern is as follows. If $d \geq 1$ and $n \geq 2$ are integers, we define a pattern to be a set $Z \subset \RR^{dn}$. We say that a set $X \subset \RR^d$ avoids the pattern $Z$ if for every $n$-tuple of distinct points $x_1, \ldots, x_n\in X$, we have $(x_1,\ldots,x_n) \not \in Z$.
%
% CHANGE: Remove wedge product, while still keeping the equation perspective clear. Again, since we are in the introduction, there's no real reason for ultimate generality at the cost of brevity. Furthermore, the problem we discussed in our meeting about the other definition of the parallelogram configuration avoiding nondegenerate parallelograms is not avoided by your equation. I've substituted x_2 - x_1 = x_4 - x_3 for the other definition, since this is much simpler. It seems like it would be difficult to specify the family of non-degnerate parallelograms as the common zeroes of functions, since the non-degenerate paralellograms form an open subset of the set of all paralellograms in the Zariski topology, and thus don't form a variety. I'm not sure whether we need to explicitly say that the parallelograms are (possible degenerate).
%
% ORIGINAL: For example, a set $X \subset \RR^d$ avoids the pattern 
%$$
%Z = \{ (x_1,x_2,x_3) \in \RR^{3d} \colon |(x_1-x_2)\wedge (x_1-x_3)|=0\}
%$$ 
%if and only if it does not contain three collinear points. Here $u \wedge v$ denotes the wedge product of $u$ and $v$; its length specifies the area of the parallelogram with sides $u$ and $v$. This length vanishes if and only if $u$ and $v$ are parallel. Similarly, a set $X\subset\RR^2$ avoids the pattern 
%$$
%Z=\{(x_1,x_2,x_3,x_4)\in\RR^{8}\colon |(x_1-x_2)\wedge (x_3-x_4)|=0, \quad |x_1 - x_2| = |x_3 - x_4| \}
%$$ 
%if and only if no four points in $X$ form the vertices of a parallelogram. 
%
For example, a set $X \subset \RR^2$ does not contain three collinear points if and only if it avoids the pattern
%
\[ Z = \{ (x_0,x_1,x_2) \in \RR^6 : \det(x_1 - x_0, x_2 - x_0) = 0 \}. \]
%
Similarly, a set $X \subset \RR^2$ avoids the pattern
%
\[ Z = \{ (x_1, x_2,x_3,x_4) \in \RR^8 : x_1 + x_4 = x_2 + x_3 \} \]
%
if and only if no four points in $X$ form the vertices of a (possibly degenerate) parallelogram.

A number of recent articles have established pattern avoidance results for increasingly general patterns. In \cite{Mathe}, M\'{a}th\'{e} constructs a set $X\subset\RR^d$ that avoids a pattern specified by a countable union of algebraic varieties of controlled degree. In \cite{MalabikaRob}, Fraser and the second author consider the pattern avoidance problem for countable unions of $C^1$ manifolds.
%
% CHANGE: I really like the flow of this paragraph, how it describes the general development of past research and how it relates to our current result. I feel things would flow even nicer if we moved the next sentence to this paragraph, and connected it to what is being said here.
%
% ORIGINAL: In this paper, we consider the pattern avoidance problem for `rough' patterns $Z\subset\RR^{dn}$ that are the countable union of sets with controlled lower Minkowski dimension.
%
In this paper, we consider the pattern avoidance problem for an even more general class of `rough' patterns $Z \subset \RR^{dn}$, that are the countable union of sets with controlled lower Minkowski dimension.
%

\begin{theorem}\label{mainTheorem}
	%
	% CHANGE: Switch from an inequality in alpha to a specification of the interval alpha lies to make things more gramatically clear, and emphasize alpha is the parameter. Also used the avoidance definition we have now introduced in the last paragraph to simplify the statement of the proof.
	%
	% ORIGINAL: Let $d \leq \alpha < dn$ and let $Z \subset \RR^{dn}$ be a countable union of compact sets, each with lower Minkowski dimension at most $\alpha$. Then there exists a set $X \subset [0,1)^d$ with Hausdorff dimension at least $(nd - \alpha)/(n-1)$ such that whenever $x_1, \dots, x_n \in X$ are distinct, we have $(x_1, \dots, x_n) \not \in Z$.
	%
	Suppose $\alpha \geq d$, and let $Z \subset \RR^{dn}$ be a countable union of compact sets, each with lower Minkowski dimension at most $\alpha$. Then there exists a set $X \subset [0,1)^d$ with Hausdorff dimension at least $(nd - \alpha)/(n-1)$ such that whenever $x_1, \dots, x_n \in X$ are distinct, we have $(x_1, \dots, x_n) \not \in Z$.
\end{theorem}

% CHANGE: Made Remarks section an AMSTHM class so that things are properly spaced / the code is neater. Also split up remarks that addressed two different points into two separate remarks. Also moved the proof discussion from the remarks since it is a bit too long for a remark.
\begin{remarks}
	\
	\begin{enumerate}[1.]
		\item When $\alpha < d$, the pattern avoidance problem is trivial, since $X = [0,1)^d - \pi(Z)$ is full dimensional and solves the pattern avoidance problem, where $\pi(x_1, \dots, x_n) = x_1$ is a projection map from $\RR^{dn}$ to $\RR^d$. Obtaining a full dimensional set in the case $\alpha = d$, however, is still interesting.

		\item Theorem \ref{mainTheorem} is trivial when $\alpha = dn$, since we can set $X = \emptyset$. We will therefore assume that $\alpha < dn$ in our proof of the theorem, without loss of generality.

		\item When $Z$ is a countable union of smooth manifolds in $\RR^{nd}$ of co-dimension $m$, we have $\alpha = nd - m$. In this case Theorem \ref{mainTheorem} yields a set in $\RR^d$ with Hausdorff dimension at least $(nd - \alpha)/(n-1) = m/(n-1)$. This recovers Theorem 1.1 and 1.2 from \cite{MalabikaRob}, making Theorem \ref{mainTheorem} a generalization of these results.

		\item Since Theorem \ref{mainTheorem} does not require any regularity assumptions on the set $Z$, it can be applied in contexts that cannot be addressed using previous methods. Two such applications, new to the best of our knowledge, have been recorded in Section \ref{applications}; see Theorems \ref{sumset-application} and \ref{C1IsoscelesThm} there.
	\end{enumerate}
\end{remarks}

The set $X$ in Theorem \ref{mainTheorem} is obtained by constructing a sequence of approximations to $X$, each of which avoids the pattern $Z$ at different scales. For a sequence of lengths $l_k \searrow 0$, we construct a nested family of sets $\{X_k\}$, where $X_k$ is a union of cubes of sidelength $l_k$ that avoids $Z$ at scales close to $l_n$. The set $X=\bigcap X_k$ avoids $Z$ at all scales. While this proof strategy is not new, our method for constructing the sets $\{X_k\}$ has several innovations that simplify the analysis of the resulting set $X=\bigcap X_k$. In particular, through a probabilistic selection process we are able to avoid the complicated queuing techniques used in \cite{KeletiDimOneSet} and \cite{MalabikaRob}, that required storage of data from each step of the iterated construction, to be retrieved at a much later stage of the construction process.

		At the same time, our construction continues to share certain features with \cite{MalabikaRob}. For example, between each pair of scales $l_{k-1}$ and $l_{k}$, we carefully select an intermediate scale $r_{k}$. The set $X_{k}\subset X_{k-1}$ avoids $Z$ at scale $l_{k}$, and it is `evenly distributed' at scale $r_k$: the set $X_{k}$ is a union of intervals of length $l_{k}$ whose midpoints resemble (a large subset of) an arithmetic progression of step size $r_k$. The details of a single step of this construction are described in Section \ref{discretesection}. In Section \ref{discretizationsection}, we explain how the length scales $l_k$ and $r_k$ for $X$ are chosen, and prove its avoidance property. In Section \ref{dimensionsection} we analyze the size of $X$ and show that it satisfies the conclusions of Theorem \ref{mainTheorem}.





\section{Frequently Used Notation and Terminology}\label{notationSection}

% DISCUSS: DOES THE SECTION PARTITIONING HELP?

\begin{enumerate}%[label=\Alph*]
	\item\label{defDyadicLength} A {\it dyadic length} is a number $l$ of the form $2^{-k}$ for some non-negative integer $k$.

	\item\label{defDyadicGrid} Given a length $l > 0$, we let $\B^d_l$ denote the set of all half open cubes in $\RR^d$ with sidelength $l$ and corners on the lattice $(l \cdot \ZZ)^d$, i.e.
	%
	\[ \B^d_l = \{ [a_1, a_1 + l) \times \cdots \times [a_d, a_d+l) : a_k \in l \cdot \ZZ \}. \]
	%
	If $E \subset \RR^d$, we let $\B^d_l(E)$ denote the set of cubes in $\B^d_l$ intersecting $E$, i.e.
	%
	\[ \B^d_l(E) = \{ I \in \B^d_l : I \cap E \neq \emptyset \}. \]

	% CHANGE: Removed upper Minkowski dimension, since it is never used in the paper.
	\item\label{defnMinkowskiDim} The {\it lower Minkowski dimension} of a bounded set $Z \subset \RR^d$ are defined as
	%
	\[ \lowminkdim(Z) = \liminf_{l \to 0} \frac{\log(\# \B^d_l(Z))}{\log(1/l)}. \]

	\item\label{defHausdorffDim} If $0 \leq \alpha$ and $\delta > 0$, we define the dyadic Hausdorff content of a set $E\subset\RR^d$ as 
		%
	\[ H^\alpha_\delta(E) = \inf \left\{ \sum_{k = 1}^\infty l_k^\alpha : E \subset \bigcup_{k = 1}^\infty I_k \right\}, \]
	%
	where the infinum is taken over all families of intervals $\{ I_k \}$ such that for all $k$, $I_k \in \B^d_{l_k}$, and $l_k$ is a dyadic length with $l_k \leq \delta$. The $\alpha$-dimensional dyadic Hausdorff measure $H^\alpha$ on $\RR^d$ is $H^\alpha(E) = \lim_{\delta \to 0} H_\delta^\alpha(E)$, and the {\it Hausdorff dimension} of a set $E$ is $\hausdim(E) = \inf \{ \alpha \geq 0 : H^\alpha(E) = 0 \}$.

	\item\label{defStronglyNonDiagonal} Given $I \in \B^{dn}_l$, we can decompose $I$ as $I_1 \times \cdots \times I_n$ for unique cubes $I_1, \dots, I_n \in \B_l^d$. We say $I$ is {\it strongly non-diagonal} if the cubes $I_1, \dots, I_n$ are distinct. Strongly non-diagonal cubes will play an important role in Section \ref{discretesection}, when we solve a discrete version of Theorem \ref{mainTheorem}.

	\item\label{defStrongCover} Adopting the terminology of \cite{KatzTao}, we say a collection of sets $\{ U_k \}$ is a {\it strong cover} of a set $E$ if $E \subset \limsup U_k$, which means every element of $E$ is contained in infinitely many of the sets $U_k$. This idea will be useful in Section \ref{discretizationsection}.  

	\item\label{defFrostmanItem} A {\it Frostman measure} of dimension $\alpha$ is a non-zero compactly supported probability measure $\mu$ on $\RR^d$ such that for every cube $I$ of sidelength $l$, $\mu(I) \lesssim l^\alpha$. Note that a measure $\mu$ satisfies this inequality for every cube $I$ if and only if it satisfies the inequality for cubes whose sidelengths are dyadic lengths. {\it Frostman's lemma} says that
	%
	% CHANGE: The aligned environment makes the spacing between lines in the set definition quite awkward.
	% ORIGINAL: \begin{equation}  \hausdim(E) = \sup \left\{ \alpha: 
	% \begin{aligned} 
	% &\text{there is a Frostman measure of}\\
	% &\text{dimension $\alpha$ supported on $E$} 
	% \end{aligned} 
	% \right\}.  \label{Hdim-defn} \end{equation} 
	\[ \hausdim(E) = \sup \left\{ \alpha:
		\begin{array}{c}
			\text{there is a Frostman measure of}\\
			\text{dimension $\alpha$ supported on $E$}
		\end{array} \right\}. \]
\end{enumerate}









\section{Avoidance at Discrete Scales}\label{discretesection}

In this section we describe a method for avoiding $Z$ at a single scale. We apply this technique in Section \ref{discretizationsection} at many scales to construct a set $X$ avoiding $Z$ at all scales. This single scale avoidance technique is the core building block of our construction, and the efficiency with which we can avoid $Z$ at a single scale has direct consequences on the Hausdorff dimension of the set $X$ obtained in Theorem \ref{mainTheorem}.

% Change: Using B_s^{dn}(Z) is not correct here, i.e if Z is a dense set. Also reference definition when using strongly non diagonal cubes for the first time.
% Original: At a single scale, we solve a discretized version of the problem, where all sets are unions of cubes at two dyadic lengths $l \geq s$ (later, we will choose $l=l_n$ and $s=l_{n+1}$). Given a set $E \subseteq [0,1)^d$ that is a union of cubes in $\B_l^d$, our goal is to construct a set $F\subset E$ that is a union of cubes in $\B_s^d$ such that $F^n$ is disjoint from the strongly non-diagonal cubes of $\B_{s}^{dn}(Z)$.
At a single scale, we solve a discretized version of the problem, where all sets are unions of cubes at two dyadic lengths $l > s$. In this discrete setting, $Z$ is replaced by a discretized version of itself, a union of cubes in $\B^{dn}_s$ denoted by $G$. Given a set $E$, which is a union of cubes in $\B_l^d$, our goal is to construct a set $F \subset E$ that is a union of cubes in $\B_s^d$, such that $F^n$ is disjoint from strongly non-diagonal cubes (see Definition \ref{defStronglyNonDiagonal}) in $\B^{dn}_s(G)$. Using the setup introduced at the end of the introduction, we will later choose $l = l_k$, $s = l_{k+1}$, and $E = X_k$. The set $X_{k+1}$ will be defined as the set $F$ constructed.
%(see Definition \ref{defStronglyNonDiagonal}).
% DISCUSS: Don't we want to reference definitions when we first use them?

In order to ensure the final set $X$ obtained in Theorem \ref{mainTheorem} has large Hausdorff dimension regardless of the rapid decay of scales used in the construction of $X$, it is crucial that $F$ is uniformly distributed at intermediate scales between $l$ and $s$.
%
% CHANGE: This is a bad choice of language to use, because it is really a combination of non-concentration and large size which gives the uniform distribution (taking F to be empty would satisfy non concentration).
% ORIGINAL: This is the `non-concentration' property discussed below. The next lemma constructs a set $F$ with these properties. 
We achieve this by decomposing $E$ into sub-cubes in $\B^d_r$ for some intermediate scale $r \in [s,l]$, and distributing $F$ as evenly among these intermediate sub-cubes as possible. This is possible assuming a mild regularity condition on the number of cubes in $G$, i.e. Equation \eqref{ZsLarge}.
%

\begin{lemma} \label{discretelemma}
	Fix two distinct dyadic lengths $l$ and $s$, with $l > s$. Let $E \subseteq [0,1)^d$ be a nonempty union of cubes in $\B^d_l$, and let $G\subset\RR^{dn}$ be a nonempty union of cubes in $\B_s^{dn}$ such that
	%
	\begin{equation}\label{ZsLarge}
		(l/s)^d \leq \# \B^{dn}_s(G)  \leq \frac{1}{2}(l/s)^{dn}.
	\end{equation} 
	%
	Then there exists a dyadic length $r \in [s,l]$ of size
	%
	\begin{equation} \label{rBound}
	 	r \sim \left( l^{-d}s^{dn} \# \B^{dn}_s(G) \right)^{\frac{1}{d(n-1)}},
	\end{equation}
	%
	and a set $F \subset E$ that is a nonempty union of cubes in $\B^d_s(E)$ satisfying the following three properties:
	%
	\begin{enumerate}
		\item\label{avoidanceItem} \emph{Avoidance}: For any choice of distinct cubes $J_1, \dots, J_n \in \B^d_s(F)$, $J_1 \times \dots \times J_n \not \in \B_s^{dn}(G)$.

		\item\label{nonConcentrationItem} \emph{Non-Concentration}: For every $I' \in \B_r^d(E)$, there is at most one $J \in \B_s^d(F)$ with $J \subset I'$.

		\item\label{largeSizeItem} \emph{Large Size}: For every $I \in \B^d_l(E)$, $\# \B^d_s(F \cap I) \geq \# \B^d_r(I) / 2 = (l/r)^d / 2$.
	\end{enumerate}
\end{lemma}

\begin{remark}
	Property \ref{avoidanceItem} says that $F$ avoids strongly non-diagonal cubes in $\B^{dn}_s(G)$. Properties \ref{nonConcentrationItem} and \ref{largeSizeItem} together imply that for every $I \in \B^d_l(E)$, at least half the cubes $I'\in \B_r^d(I)$ contribute a single sub-cube of sidelength $s$ to $F$; the rest contribute none. 
	%contains a single cube of sizelength $s$ inside of $I$. 
\end{remark}

\begin{proof}
	Let $r$ be the smallest dyadic length at least as large as $R$, where
	%
	\begin{equation} \label{What-is-r}
		R = \big(2 l^{-d}s^{dn}\# \B^{dn}_s(G)\big)^{\frac{1}{d(n-1)}}.
		%r\geq\max\Big(s,\ \big(l^{-d}s^{dn}\# \B^{dn}_s(G)\big)^{\frac{1}{d(n-1)}}\Big).
	\end{equation} 
	%
	%By the first inequality in \eqref{ZsLarge}, 
	This choice of $r$ satisfies \eqref{rBound}. 
	%Define $A_l = (2^{1/d}/l)^{1/(n-1)}.$ By the second inequality in \eqref{ZsLarge}, we have
	%
	%	\[ A_l (s^{dn}\#\B^{dn}_s(G))^{1/d(n-1)} \leq A_l l^{n/(n-1)} / 2^{1/d(n-1)} = l. \]
	%Since $l$ is a dyadic length, we conclude that $r\leq l$ and thus 
	The inequalities in \eqref{ZsLarge} ensure that $r \in [s,l]$; more precisely, the left inequality in \eqref{ZsLarge} implies $R$ is bounded from below by $s$, and the right inequality implies $R$ is bounded from above by $l$. The minimality of $r$ ensures $s \leq r \leq l$.

	For each $I' \in \B_r^d(E)$, let $J_{I'}$ be an element of $\B^d_s(I)$ chosen uniformly at random; these choices are independent as $I'$ ranges over the elements of $\B_r^d(E)$. Define
	%
	\[ 	U = \bigcup \left\{ J_{I'} : I' \in \B_r^d(E) \right\}, \]
	%
	and
	%
	\[ \mathcal{K}(U) = \{ K \in \B^{dn}_s(G) : K \in U^n, \text{$K$ strongly non-diagonal} \}. \]
	%
	Note that the sets $U$ and $\mathcal{K}(U)$ are random sets, in the sense that they depend on the random variables $\{ J_{I'} \}$. Define
	%
	\begin{equation} \label{defnOfF}
		F(U) = U - \{ \pi(K) : K \in \mathcal{K}(U) \},
	\end{equation}
	%
	where $\pi: \RR^{dn} \to \RR^d$ is the projection map $(x_1, \dots, x_n) \mapsto x_1$, for $x_i \in \RR^d$. Thus $\pi$ sends the cube $J_1 \times \dots \times J_n\in \B^{dn}_s$ to the cube $J_1 \in \B^d_s$. Given any strongly non-diagonal cube $K = J_1 \times \cdots \times J_n \in \B_s^{dn}(G)$, either $K \not \in \B_s^{dn}(U^n)$, or $K \in \B_s^{dn}(U^n)$. If the former occurs then $K \not \in \B_s^{dn}(F(U)^n)$ since $F(U) \subset U$, so $\B_s^{dn}(F(U)^n) \subset \B_s^{dn}(U^n)$. If the latter occurs then $K \in \mathcal{K}(U)$, and since $\pi(K) = J_1$, $J_1 \not \in \B_s^d(F(U))$. In either case, $K \not \in \B_s^{dn}(F(U)^n)$, so $F(U)$ satisfies Property \ref{avoidanceItem}. By construction, $U$ contains at most one subcube $J \in \B^{dn}_s$ for each $I \in \B^{dn}_l(E)$. Since $F(U) \subset U$, $F(U)$ satisfies Property \ref{nonConcentrationItem}. Thus the set $F(U)$ satisfies Properties \ref{avoidanceItem} and \ref{nonConcentrationItem} regardless of which values are assumed by the random variables $\{ J_{I'} \}$. Next we will show that with non-zero probability, the set $F(U)$ satisfies Property \ref{largeSizeItem}. 

	For each cube $J \in \B_s^d(E)$, there is a unique `parent' cube $I' \in \B_r^d(E)$ such that $J \subset I'$. Since $I'$ contains $(r/s)^d$ elements of $\B^d_s(E)$, and $J_{I'}$ is chosen uniformly at random from $\B^d_s(I)$,
	%
%	\begin{equation} \label{singleCubeProb}
	\[ \Prob(J \subset U) = \Prob(J_{I'} = J) = (s/r)^d. \]
%	\end{equation}
	%
	%Here the probability measure $\Prob(\cdot)$ is taken with respect to the randomly chosen set $U$ defined in \eqref{Udefinition}.
	The cubes $J_{I'}$ are chosen independently, so if $J_1, \dots, J_k$ are distinct cubes in $\B^d_s(E)$, then %the last calculation combined with Property \ref{nonConcentrationItem} shows that
	%
	\begin{equation}\label{jointprob}
	\Prob(J_1, \dots, J_k \in U) = \begin{cases} (s/r)^{dk} & \text{if $J_1, \dots, J_k$ have distinct parents,} \\ 0 & \text{otherwise}. \end{cases} 
	\end{equation}
	%
	Let $K = J_1 \times \dots \times J_n \in \B^{dn}_s(G)$. If the cubes $J_1, \dots, J_n$ are distinct, we deduce from \eqref{jointprob} that
	%
	\begin{equation}\label{probaKSubsetUn}
		\Prob(K \subset U^n) = \Prob(J_1, \dots, J_n \in U) \leq (s/r)^{dn}.
	\end{equation}
	%
	By \eqref{probaKSubsetUn} linearity of expectation, and \eqref{What-is-r},
	%
	\begin{align*}
		\Expect(\# \mathcal{K}(U)) &= \sum_{K \in \B^{dn}_s(G)} \Prob(K \subset U^n) \leq \# \B_s^{dn}(G) \cdot (s/r)^{dn}
		% (l/r)^{dn}/2
	%	&= \left[ s^{dn}\#\B^{dn}_s(G) r^{-d(n-1)} \right] r^{-d} \\
	%	& \leq \left[ s^{dn}\#\B^{dn}_s(G) (A_l (s^{dn}\#\B^{dn}_s(G))^{1/d(n-1)})^{-d(n-1)} \right] r^{-d} \\
		\leq 0.5 \cdot (l/r)^d.
		%& = (l/r)^d /2.
	\end{align*}
	%
	In particular, there exists at least one (non-random) set $U_0$ such that
	%
	\begin{equation}\label{KU0Small}
		\# \mathcal{K}(U_0) \leq \Expect(\# \mathcal{K}(U)) \leq 0.5 \cdot (l/r)^d.
	\end{equation}
	%
	 In other words, $F(U_0) \subset U_0$ is obtained by removing at most $0.5 \cdot (l/r)^d$ cubes in $\B^d_s$ from $U_0$. For each $I \in \B_l^d(E)$, we know that $\# \B_{s}^d(I \cap U_0) = (l/r)^d$. Combining this with \eqref{KU0Small}, we arrive at the estimate 
	%
	% Change: Some of the manipulations of the old version of this inequality are not technically true using the notation provided.
	\begin{align*}
		\# \B_s^d(I \cap F(U_0)) &= \B_s^d(I \cap U_0) - \# \{ \pi(K) : K \in \mathcal{K}(U_0), \pi(K) \in F(U_0) \}\\
		&\geq \B_s^d(I \cap U_0) - \#(\mathcal{K}(U_0))\\
		&\geq (l/r)^d - 0.5 \cdot (l/r)^d \geq 0.5 \cdot (l/r)^d
%		\# \B_{s}^d(I \cap F_{U_0}) &= \# \B_{s}^d(I \cap F_{U_0}) - \# \B_{s}^d \bigl[ I \cap \pi(\mathcal K(U_0)) \bigr] \\  
%		&\geq \# \B_{s}^d(I \cap F_{U_0}) - \# \B_{s}^d (\pi(\mathcal K(U_0))) \\ 
%		&\geq \# \B_{s}^d(I \cap F_{U_0}) - \# \B_{s}^d (\mathcal K(U_0))\\
%		&\geq (l/r)^d - 0.5 \cdot (l/r)^d \geq 0.5 \cdot (l/r)^d.  
	\end{align*}  
	%
	In other words, $F(U_0)$ satisfies Property \ref{largeSizeItem}. Setting $F = F(U_0)$ completes the proof.
\end{proof}

\begin{remarks}
	\
	\begin{enumerate}[1.]
		\item While Lemma \ref{discretelemma} uses probabilistic arguments, the conclusion of the lemma is not a probabilistic statement. In particular, one can find a suitable $F$ constructively by checking every possible choice of $U$ (there are finitely many) to find one particular choice $U_0$ which satisfies \eqref{KU0Small}, and then defining $F$ by \eqref{defnOfF}. Thus the set we obtain in Theorem \ref{mainTheorem} exists by purely constructive means.
		
		\item At this point, it is possible to motivate the numerology behind the dimension bound $\dim(X) \geq (dn-\alpha)/(n-1)$ from Theorem \ref{mainTheorem}, albeit in the context of Minkowski dimension. We will pause to do so here before returning to the proof of Theorem \ref{mainTheorem}. For simplicity, let $\alpha > d$, and suppose that $Z \subset \RR^{dn}$ satisfies 
		%
		\begin{equation}\label{specialCase}
			\#\B_{s}^{dn}(Z) \sim s^{-\alpha} \quad \textrm{for every}\ s \in (0,1].
		\end{equation}
		%
		Let $l = 1$, $E = [0,1)^d$, and let $s > 0$ be a small parameter. If $s$ is chosen sufficiently small compared to $d,n$, and $\alpha$, then \eqref{ZsLarge} is satisfied with $G = \bigcup \B^{dn}_s(Z)$. We can then apply Lemma \ref{discretelemma} to find a dyadic scale $r \sim s^{(dn-\alpha)/d(n-1)}$ and a set $F$ that avoids the strongly non-diagonal cubes of $\B_{s}^{dn}(Z)$. The set $F$ is a union of approximately $r^{-d} \sim s^{-(dn-\alpha)/(n-1)}$ cubes of sidelength $s$. Thus informally, the set $F$ resembles a set with Minkowski dimension $\alpha$ when viewed at scale $s$. 

		The set $X$ constructed in Theorem \ref{mainTheorem} will be obtained by applying Lemma \ref{discretelemma} iteratively at many scales. At each of these scales, $X$ will resemble a set of Minkowski dimension $(dn - \alpha)/(n-1)$. A careful analysis of the construction (performed in Section \ref{dimensionsection}) shows that $X$ actually has Hausdorff dimension at least $(dn - \alpha)/(n-1)$.

		\item Lemma \ref{discretelemma} is the core method in our avoidance technique. The remaining argument is fairly modular. If, for a special case of $Z$, one can improve the result of Lemma \ref{discretelemma} so that $r$ is chosen on the order of $s^{\beta/d}$, then the remaining parts of our paper can be applied near verbatim to yield a set $X$ with Hausdorff dimension $\beta$, as in Theorem \ref{mainTheorem}. 
	\end{enumerate} 
\end{remarks}









\section{Fractal Discretization}\label{discretizationsection}

% CHANGE: Now this section has been organized, you talk about the sets Z_k before you even introduce them. This needs to be reworded.
% ORIGINAL: In this section we will construct the set $X$ from Theorem \ref{mainTheorem} by applying Lemma \ref{discretelemma} at many scales. The goal is to find a nested decreasing family of discretized sets $\{ X_k \}$ and to set $X = \bigcap X_k$. One condition guaranteeing that $X$ avoids $Z$ is that $X_k^n$ is disjoint from {\it strongly non-diagonal} cubes in $Z_k$.
In this section we construct the set $X$ from Theorem \ref{mainTheorem} by applying Lemma \ref{discretelemma} at many scales. Since $Z$ is a countable union of bounded sets with Minkowski dimension at most $\alpha$, there exists a strong cover (see Definition \ref{defStrongCover}) of $Z$ by cubes restricted to a sequence of dyadic lengths $\{ l_k \}$, with a quantitative bound on the number of cubes at each scale. We fix a cover so that the scales $l_k$ converge to $0$ very quickly.

\begin{lemma}\label{coveringLemma}
	Let $Z \subset \RR^{dn}$ be a countable union of bounded sets with Minkowski dimension at most $\alpha$, and let $\epsilon_k \searrow 0$ with $2\epsilon_k < dn - \alpha$ for all $k$. Then there exists a sequence of lengths $\{ l_k \}$ and a strong cover of $Z$ by a sequence of sets $\{ Z_k \}$, such that
	%
	\begin{enumerate}
		\item\label{DiscretenessProperty} \emph{Discreteness}: For all $k \geq 0$, $Z_k$ is a union of cubes in $\B^{dn}_{l_k}$.

		\item\label{SparsityProperty} \emph{Sparsity}: For all $k \geq 0$, $l_k^{-d} \leq \B^{dn}_{l_k}(Z_k) \leq l_k^{-\alpha-\epsilon_k}$.

		\item\label{RapidDecayProperty} \emph{Rapid Decay}: For all $k > 1$,
			\begin{align}
				l_k^{dn-\alpha-\varepsilon_k} & \leq 0.5 \cdot l_{k-1}^{dn} \label{coverBoundRequirement} \quad \text{and}\ \\
				l_k^{\epsilon_k} & \leq l_{k-1}^{2d}\label{quadDecayRequirement}.
			\end{align}
	\end{enumerate}
\end{lemma}
\begin{proof}
	We can write
	%
	\[ Z = \bigcup_{i = 1}^\infty Y_i \quad \text{with}\ \lowminkdim(Y_i) \leq \alpha\ \text{for each $i$}. \]
	%
	Consider the $d$ dimensional hyperplane
	%
	\[ H = \{ (x_1,\dots, x_1) : x_1 \in [0,1)^d \} \]
	%
	We then set $Y_i' = Y_i \cup H$. In particular, this means for any $l$,
	%
	\begin{equation}\label{YPrimeLowerBound}
		\# \B^{nd}_l(Y_i') \geq \# \B^{nd}_l(H) = l^{-d},
	\end{equation}
	%
	yet we still know $\lowminkdim(Y_i') \leq \alpha$ for each index $i$. Let $\{ i_k \}$ be a sequence of integers that repeats each integer infinitely often.

	The lengths $\{ l_k \}$ and sets $\{ Z_k \}$ are defined inductively. As a base case, set $l_0 = 1$ and $Z_0 = [0,1)^d$. Suppose that the lengths $l_0, \ldots, l_{k-1}$ have been chosen. Since $\lowminkdim(Y_{i_k}) \leq \alpha$, Definition \ref{defnMinkowskiDim} implies that there exists arbitrarily small lengths $l$ which saitsfy
	%
	\begin{equation}\label{coveringOfBdnlZk}
		\# \B^{dn}_l(Y_{i_k}') \leq l^{-\alpha - \frac{\varepsilon_k}{2}}.
	\end{equation}
	%
	In particular, we can choose $l = l_k$ small enough to satisfy \eqref{coverBoundRequirement}, \eqref{quadDecayRequirement}, and \eqref{coveringOfBdnlZk}, so Property \ref{RapidDecayProperty} is satisfied. We then set $Z_k = \bigcup \B^{dn}_{l_k}(Y_{i_k}')$. For any $z \in Z$, there is some $i$ such that $z \in Y_i$, and there is a subsequence of integers $k_1, k_2, \dots$ such that $i_{k_j} = i$ for all $j$. But then $z \in Y_i \subset Y_i' \subset Z_{i_{k_j}}$, so $z$ is contained in the infinite sequence of sets $Z_{i_{k_j}}$. In particular, $z \in \limsup Z_i$, and since $z$ was arbitrary, $Z \subset \limsup Z_i$, so $\{ Z_i \}$ is a strong cover of $Z$. Now Property \ref{DiscretenessProperty} is satisfied by construction of the $\{ Z_i \}$. And Property \ref{SparsityProperty} is implied by \eqref{YPrimeLowerBound} and \eqref{coveringOfBdnlZk}.
\end{proof}

To construct $X$, we consider a nested, decreasing family of discretized sets $\{ X_k \}$, where $X_k$ is a union of cubes in $\B^d_{l_k}(X_k)$. We then set $X = \bigcap X_k$. The goal is to choose $X_k$ such that $X_k^n$ is disjoint from {\it strongly non diagonal} cubes in $Z_k$.

\begin{lemma} \label{stronglydiagonal}
	Let $Z \subset \RR^{dn}$, let $\{Z_k\}$ be a sequence of sets that strongly cover $Z$, and let $\{ l_k \}$ be a sequence of lengths converging to zero. For each index $k$, let $X_k$ be a union of cubes in $\B^d_{l_k}$. Suppose that for each $k$, $X_k^n$ avoids strongly non-diagonal cubes in $\B^{dn}_{l_k}(Z_k)$. If $X = \bigcap X_k$, then for any distinct $x_1, \dots, x_n \in X$, we have $(x_1, \dots, x_n) \not \in Z$.
\end{lemma}
\begin{proof}
	Let $z \in Z$ be a point with distinct coordinates $z_1, \dots, z_n$. Define
	%
	\[ \Delta = \{ (w_1, \dots, w_n) \in \RR^{dn}: \text{there exists $i \neq j$ such that $w_i = w_j$} \}. \]
	%
	Then $d(\Delta,z) > 0$, where $d$ is the Hausdorff distance between $\Delta$ and $z$. Since $\{ Z_k \}$ strongly covers $Z$, there is a subsequence $\{ k_m \}$ such that $z \in Z_{k_m}$ for every index $m$. Since $l_k \searrow 0$ and thus $l_{k_m}\searrow 0$, if $m$ is sufficiently large then $\sqrt{dn} \cdot l_{k_m} < d(\Delta,z)$. Note that $\sqrt{dn} \cdot l_{km}$ is the diameter of a cube in $\B_{l_{k_m}}^{dn}$. For such a choice of $m$, if $I\in \B_{l_{k_m}}^{dn}(Z_{k_m})$ is the (unique) cube in $\B_{l_{k_m}}^{dn}$ containing $z$, then $I \cap \Delta = \emptyset$. But this means $I$ is strongly non-diagonal. Since $X_{k_m}$ avoids the strongly non-diagonal cubes of $Z_{k_m}$, we conclude that $z \not \in X_{k_m}^n$. In particular, this means $z \not\in X^n$.
\end{proof}

All that remains is to apply the discrete lemma to choose the sets $X_k$.

\begin{lemma} 
	Given a sequence of dyadic length scales $l_k$ obeying, \eqref{coverBoundRequirement}, \eqref{quadDecayRequirement}, and \eqref{coveringOfBdnlZk} as above, there exists a sequence of sets $\{X_k\}$ and a sequence of dyadic intermediate scales $\{ r_k \}$ with $l_k \leq r_k \leq l_{k-1}$ for each $k \geq 1$, such that each set $X_k$ is a union of cubes in $\B_{l_k}^d(X_{k-1})$ that avoids the strongly non-diagonal cubes of $\mathcal B_{l_k}^{dn}(Z_k)$. Furthermore, for each index $k\geq 1$ we have
	%
	\begin{align}
		& r_k \lesssim l_k^{(dn-\alpha -\epsilon_k)/d(n-1)},\label{rkSizeBound}\\
		& \# \B^d_{l_k}(X_k \cap I) \geq 0.5 \cdot (l_{k-1}/r_k)^d \quad \text{for each}\ I\in \B_{l_{k-1}}^d(X_{k-1}), \label{manyIkInIkm1}\\
		&\# \B^d_{l_k}(X_k \cap I') \leq 1 \quad \text{ for each}\ I' \in \B_{r_{k}}^d(X_{k-1}).\label{XkWellDistributed}
	\end{align}
\end{lemma}
\begin{proof}
	We construct $X_k$ by induction, using Lemma \ref{discretelemma} as building block. Set $X_0=[0,1)^d$. Next, suppose that the sets $X_0, \ldots, X_{k-1}$ have been defined. Our goal is to apply Lemma \ref{discretelemma} to $E = X_{k-1}$ and $G = Z_k$ with $l = l_{k-1}$ and $s = l_k$. This will be possible once we verify the hypothesis \eqref{ZsLarge}, which in this case takes the form
	%
	\begin{equation}
		(l_{k-1}/l_k)^d \leq \#\B_{l_k}^{dn}(Z_k) \leq 0.5 \cdot (l_{k-1}/l_k)^{dn}. \label{need-to-check}
	\end{equation}
	%
	The right hand side follows from Property \ref{SparsityProperty} of Lemma \ref{coveringLemma} and \eqref{quadDecayRequirement}. 
%imply that 
%$$
%\#\B_{l_k}^{dn}(Z_k)\leq\frac{1}{2}(l_{k-1}/l_k)^{dn}.
%$$
	On the other hand, Property \ref{SparsityProperty} and the fact that $l_{k-1} \leq 1$ implies that
	%
	\[ (l_{k-1}/l_k)^d\leq l_{k}^{-d}\leq \#\B_{l_k}^{dn}(Z_k), \]
	%
	establishing the left inequality in \eqref{need-to-check}. Applying Lemma \ref{discretelemma} as described above now produces a dyadic length
	%
	\begin{equation}\label{definOfr}
		r \sim \big(l_{k-1}^{-d}l_k^{dn} \# \B^{dn}_{l_k}(Z_k)\big)^{\frac{1}{d(n-1)}} 
	\end{equation}
	%
	and a set $F\subset X_{k-1}$ that is a union of cubes in $\B_{l_k}^{d}$. The set $F$ satisfies Properties \ref{avoidanceItem}, \ref{nonConcentrationItem}, and \ref{largeSizeItem} from the statement of Lemma \ref{discretelemma}. Define $r_k=r$ and $X_k=F$. The estimate  \eqref{rkSizeBound} on $r_k$ follows from \eqref{definOfr} using the known bounds \eqref{quadDecayRequirement} and \eqref{coveringOfBdnlZk}:
	%
	\[ r_k \lesssim \bigl( l_{k-1}^{-d}  l_k^{dn -\alpha - 0.5 \epsilon_k} \bigr)^{\frac{1}{d(n-1)}} = \bigl( l_{k-1}^{-d} l_k^{0.5 \epsilon_k} l_k^{dn -\alpha - \epsilon_k} \bigr)^{\frac{1}{d(n-1)}} = \bigl( l_{k-1}^{-2d} l_k^{\epsilon_k}\bigr)^{\frac{1}{2d(n-1)}} l_{k}^{\frac{dn-\alpha -\epsilon_k}{d(n-1)}} \lesssim l_{k}^{\frac{dn-\alpha -\epsilon_k}{d(n-1)}}. \]
	%
	The requirements \eqref{manyIkInIkm1} and \eqref{XkWellDistributed} follow from Properties \ref{nonConcentrationItem} and \ref{largeSizeItem} of Lemma \ref{discretelemma} respectively.
\end{proof} 

Now we have defined the sets $\{ X_k \}$, we set $X = \bigcap X_k$. Since $X_k$ avoids strongly non-diagonal cubes in $Z_k$, Lemma \ref{stronglydiagonal} implies that if $x_1, \dots, x_n \in X$ are distinct, then $(x_1, \dots, x_n) \not \in Z$. To finish the proof of Theorem \ref{mainTheorem}, we must show that $\hausdim(X) \geq (dn - \alpha)/(n - 1)$. This will be done in the next section. 







\section{Dimension Bounds}\label{dimensionsection}

To complete the proof of Theorem \ref{mainTheorem}, we must show that $\hausdim(X) \geq (dn - \alpha)/(n - 1)$.  %, where
%
% \[ \beta = \frac{dn - \alpha}{n - 1}. \]
In view of Definition \ref{defFrostmanItem}, we will do this by constructing a Frostman measure of appropriate dimension supported on $X$. 
%
% We begin with a rough outline of our proof strategy. Recall that from the previous section, we have a decreasing sequence of lengths $\{ l_k \}$. The most convenient way to examine the dimension of $X$ at various scales is to use Frostman's lemma (see Definition \ref{frostmanItem}). We construct a probability measure $\mu$ supported on $X$ such that for all $\varepsilon > 0$, for all dyadic lengths $l$, and for all $I \in \B^d_l$, $\mu(I) \lesssim_\varepsilon l^{\beta - \varepsilon}$. We begin by showing that for each $k\geq 1$,
% \begin{equation}\label{muIScalek}
% \mu(I) \lesssim l_k^{\beta - O(1/k)}\quad\textrm{for all}\ I \in \B^d_{l_k}.
% \end{equation}
% Heuristically, this inequality stays that $X$ looks like a set with dimension $\beta - O(1/k)$ at the scale $l_k$. Our next task will be understand the behavior of $\mu$ (and thus $X$) at scales between $l_{k-1}$ and $l_k$. This task is complicated by the fact that $l_{k}$ might be much smaller than $l_{k-1}$ (indeed, we have no effective control on how quickly the length scales $\{l_k\}$ converge to 0). Thankfully, however, the sets $X_k$ defined in the previous section are unions of cubes of sidelength $I_{l_k}$ that are somewhat uniformly distributed at scales larger than $l_k$ (this Property \ref{nonConcentrationItem} in Lemma \ref{discretelemma}); this fact will allow us to establish an analogue of \eqref{muIScalek} at intermediate scales between $l_k$ and $l_{k+1}$. 
%

We start by defining a premeasure on $\bigcup_{i = 1}^\infty \B^d_{l_i}[0,1)^d$. Set $\mu([0,1)^d) = 1$. Suppose now that $\mu(I)$ has been defined for all cubes in $\B^d_{l_{k-1}}[0,1)^d$, and let $J \in \B^d_{l_k}$. Consider the unique `parent cube' $I \in \B^d_{l_{k-1}}$ for which $J \subset I$. Define
%
\begin{equation} \label{muRecurse} 
	\mu(J) = \begin{cases} {\mu(I)}/{\# \B^d_{l_k}(X_k \cap I)} & \textrm{if}\ J \subset X_k,\\
0 & \textrm{otherwise}.
\end{cases}
\end{equation}
Observe that for each index $k\geq 1$ and each $I \in \B_{l_{k-1}}^d$, 
%
\begin{equation}\label{muBreakDown}
	\sum_{J \in \B_{l_k}^d(I)} \mu(J) = \sum_{J \in \B_{l_k}^d(X_k\cap I)} \mu(J) = \mu(I).
\end{equation}
In particular, for each index $k$ we have
%
\[ \sum_{I\in\B_{l_k}}\mu(I)=1. \]
%
By a standard argument involving the Caratheodory extension theorem \cite[Proposition 1.7]{Falconer}, the premeasure $\mu$ extends to a measure on the Borel subsets of $[0,1)^d$. Note that for each $k \geq 1$, $\mu$ is supported on $X_k$. Thus $\mu$ is supported on $\bigcap X_k = X$. To complete the proof of Theorem \ref{mainTheorem} we will show that $\mu$ is a Frostman measure of dimension $(dn - \alpha)/(n - 1)-\epsilon$ for every $\epsilon>0$. 



\begin{lemma}\label{massSomeScales}
	For each $k \geq 1$ and each $J \in \B^d_{l_k}(X)$, 
	%
	\[ \mu(J) \lesssim l_k^{\frac{dn-\alpha}{n-1}- \eta_k}, \quad \text{ where } \quad \eta_k = \frac{n+1}{2(n-1)} \cdot \epsilon_k \searrow 0 \text{ as } k \rightarrow \infty. \]
\end{lemma}
\begin{proof}
	Let $J \in \B^d_{l_k}$ and let $I \in \B^d_{l_{k-1}}$ be the parent cube of $J$. Since $\mu$ is a probability measure, we have $\mu(I) \leq 1$. Combining \eqref{muRecurse}, \eqref{manyIkInIkm1}, \eqref{rkSizeBound}, and \eqref{quadDecayRequirement} we obtain
	%
	\[ \mu(J)\leq \frac{2r_k^d}{l_{k-1}^d}\mu(I)\leq \frac{2r_k^d}{l_{k-1}^d}\lesssim \frac{l_{k}^{\frac{dn-\alpha - \epsilon_k}{n-1}}}{l_{k-1}^d}=l_k^{\frac{dn-\alpha}{n-1}-\eta_k}\big(l_k^{0.5 \epsilon_k}/l_{k-1}^d\big)\leq l_k^{\frac{dn-\alpha}{n-1}-\eta_k}.\qedhere \]
\end{proof}

\begin{corollary}\label{muAtScaleRk}
	For each $k\geq 1$ and each $I' \in \B^d_{r_k} (X_{k-1})$, 
	%
	\begin{equation} 
	\mu(I') \lesssim (r_k/l_{k-1})^d \cdot l_{k-1}^{\frac{dn-\alpha}{n-1}-\eta_{k-1}}. \label{mu-Rk}
	\end{equation} 
\end{corollary}
\begin{proof}
%Lemma \ref{massSomeScales} allows us to control $\mu$ at the scales $\{l_k\}$. 
Let us fix a cube $I' \in \B^d_{r_k}(X_{k-1})$, and let $I$ denote its unique parent cube in $\B_{l_{k-1}}^d (X_{k-1})$. According to \eqref{XkWellDistributed}, $I'$ contains at most one cube in $\B_{l_k}^d(I)$; let us denote this cube by $J$ if it exists. Then the mass distribution rule given by \eqref{muRecurse} dictates that 
\begin{align*}
\mu(I') = \mu(X_k \cap I') = \begin{cases} \mu(J) = {\mu(I)}/{\# \B_{l_k}^d(X_k \cap I)}  &\text{ if } \# \B_{l_k}^d(X_k \cap I') = 1, \\ 0 &\text{ if } \# \B_{l_k}^d(X_k \cap I') = 0. \end{cases} 
\end{align*} 
Using the estimate \eqref{manyIkInIkm1} and applying Lemma \ref{massSomeScales} to $I \in \mathcal B_{l_{k-1}}^d(X)$, we arrive at the claimed bound \eqref{mu-Rk}. 
\end{proof}
Lemma \ref{massSomeScales} and Corollary \ref{muAtScaleRk} allow us to control the behavior of $\mu$ at all scales. %To understand the behavior of $\mu$ at other scales, we will obtain a Frostman measure bound at {\it all} scales, we need to apply a covering argument. This is where the uniform mass assignment technique comes into play. Because $\mu$ behaves like a full dimensional set between the scales $l_k$ and $r_{k+1}$, we won't be penalized for making the gap between $l_k$ and $r_{k+1}$ arbitrarily large. This is essential to our argument, because $l_k$ decays faster than $2^{-k^m}$ for any $m > 0$.

\begin{lemma} \label{frostmanBound}
For every $\alpha \in [d, dn)$, and for each $\epsilon>0$, there is a constant $C_\epsilon$ so that for all dyadic lengths $l\in (0,1]$ and all $I \in \B_l^d$, we have
	\begin{equation} 
	\mu(I) \leq C_{\epsilon} l^{\frac{dn - \alpha}{n - 1} - \epsilon}. \label{mu-ball-condition} 
	\end{equation} 
\end{lemma}
\begin{proof}
	Fix $\epsilon > 0$. Since $\eta_k \searrow 0$ as $k\to\infty$, there is a constant $C_{\epsilon}$ so that $l_k^{-\eta_k}\leq C_{\epsilon}l_k^{-\epsilon}$ for each $k \geq 1$. For instance, if $\varepsilon_k$ is decreasing, we could choose $C_{\epsilon}=l_{k_0}^{-\eta_{k_0}}$, where $k_0$ is the largest integer for which $\eta_{k_0} \geq \epsilon$. Next, let $k$ be the (unique) index so that $l_{k+1} \leq l < l_{k}$. We will split the proof of \eqref{mu-ball-condition} into two cases, depending on the position of  $l$ within $[l_{k+1}, l_k]$. 
	%We now consider several cases. 
	%\begin{itemize}
	%\item If $k<k_0$, then $l\geq l_{k_0}$ and thus 
	%$$
	%\mu(I)\leq 1 = \big(l^{\frac{dn - \alpha}{n - 1} - \epsilon}\big)^{-1}\big(l^{\frac{dn - \alpha}{n - 1} - \epsilon}\big)\leq C_{\epsilon}\big(l^{\frac{dn - \alpha}{n - 1} - \epsilon}\big).
	%$$

	{\em{Case 1: }} If $r_{k+1} \leq l < l_k$, 
	we can cover $I$ by $(l/r_{k+1})^d$ cubes in $\B^d_{r_{k+1}}$. By Corollary \ref{muAtScaleRk},
	\begin{equation}
	\begin{split}
	\mu(I) & \lesssim (l/r_{k+1})^d (r_{k+1}/l_k)^d l_k^{\frac{dn-\alpha}{n-1}-\eta_k} \\
	& = (l/l_k)^d l_k^{\frac{dn-\alpha}{n-1}-\eta_{k}}\\
	& = l^{\frac{dn-\alpha}{n-1}} (l/l_k)^{\frac{\alpha - d}{n-1}} l_k^{-\eta_k}\\
	& \leq l^{\frac{dn-\alpha}{n-1} - \eta_k}  \\
	& \leq C_{\epsilon}l^{\frac{dn-\alpha}{n-1}-\epsilon}.
	\end{split}
	\end{equation}
The penultimate inequality is a consequence of our assumption $\alpha \geq d$. 

	{\em{Case 2: }} If $l_{k+1} \leq l \leq r_{k+1},$ we can cover $I$ by a single cube in $\B^d_{r_{k+1}}$. By \eqref{XkWellDistributed}, each cube in $\B^d_{r_{k+1}}$ contains at most one cube $I_0 \in \B^d_{l_{k+1}}(X_{k+1})$, so by Lemma \ref{massSomeScales},
	%
	\[ 
		\mu(I) \leq \mu(I_0) \lesssim l_{k+1}^{\frac{dn - \alpha}{n - 1} - \eta_{k+1}} 
		% \lesssim l_{k+1}^{\frac{dn - \alpha}{n - 1}}r_{k+1}^{-\eta_{k+1}\frac{d(n-1)}{dn-\alpha-\epsilon_{k+1}}}
		% \leq C_{\epsilon}l_{k+1}^{\frac{dn - \alpha}{n - 1}}r_{k+1}^{-\epsilon}
		\leq C_{\epsilon}l_{k+1}^{\frac{dn - \alpha}{n - 1} - \epsilon}
		\leq C_{\epsilon}l^{\frac{dn - \alpha}{n - 1} - \epsilon}\qedhere
	\]
	%\end{itemize}

\end{proof}

Applying Frostman's lemma to Lemma \ref{frostmanBound} gives $\hausdim(X) \geq \frac{dn - \alpha}{n - 1} - \epsilon$ for every $\epsilon>0$, which concludes the proof of Theorem \ref{mainTheorem}.









\section{Applications}\label{applications}

As discussed in the introduction, Theorem \ref{mainTheorem} generalizes Theorems 1.1 and 1.2 from \cite{MalabikaRob}. In this section, we present two applications of Theorem \ref{mainTheorem} in settings where previous methods do not yield any results.

\subsection{Sum-sets avoiding specified sets}

\begin{theorem} \label{sumset-application} 
	Let $Y \subset \RR^d$ be a countable union of sets of Minkowski dimension at most $\beta < d$. Then there exists a set $X \subset \RR^d$ with Hausdorff dimension at least $d - \beta$ such that $X + X$ is disjoint from $Y$.
\end{theorem}
\begin{proof}
	Define $Z = Z_1 \cup Z_2$, where
	%
	\[ Z_1 = \{ (x,y) : x + y \in Y \} \quad \text{and} \quad Z_2 = \{ (x,y) : y \in Y/2 \}. \]
	%
	Since $Y$ is a countable union of sets of Minkowski dimension at most $\beta$, $Z$ is a countable union of sets with lower Minkowski dimension at most $d + \beta$. Applying Theorem \ref{mainTheorem} with $n = 2$ and $\alpha = d + \beta$ produces a set $X \subset \RR^d$ with Hausdorff dimension $2d  - (d + \beta) = d - \beta$ such that $(x,y) \not \in Z$ for any $x,y \in X$ with $x \neq y$. We claim that $X+ X$ is disjoint from $Y$. To see this, first suppose $x, y \in X$, $x \ne y$. Since $X$ avoids $Z_1$, we conclude that $x + y \not \in Y$. Suppose now that $x = y \in X$. Since $(x,y) \not \in Z_2$ for any $x,y \in X$, $x \not \in Y/2$ for any $x \in X$. Thus for any $x \in X$, $x + x = 2x \not \in Y$. This completes the proof. 
\end{proof}


\subsection{Subsets of Lipschitz curves avoiding isosceles triangles}

In \cite{MalabikaRob}, Fraser and the second author prove that if $\gamma\subset\RR^n$ is a simple $C^2$ curve with non-vanishing curvature, then there exists a set $S\subset\gamma$ of Hausdorff dimension $1/2$ that does not contain the vertices of an isosceles triangle. Using Theorem \ref{mainTheorem}, we generalize this result to Lipschitz curves. 

\begin{theorem}\label{C1IsoscelesThm}
	Let $f\colon [0,1) \to \RR^{n-1}$ be Lipschitz. Then there is a set $X \subset [0,1)$ of Hausdorff dimension $1/2$ so that the set $\{(t,f(t)) : t\in X\}$ does not contain the vertices of an isosceles triangle.
\end{theorem}
\begin{proof}
	Set
	%
	\[ Z = \left\{ (x_1,x_2,x_3) \in [0,1]^3: \begin{array}{c} (x_1,f(x_1)), (x_2,f(x_2)), (x_3,f(x_3))\\
		\textrm{form the vertices of an isosceles triangle} \end{array} \right\}. \]
	%
	In the next lemma, we show $Z$ has lower Minkowski dimension at most two. By Theorem \ref{mainTheorem}, there is a set $X_1\subset[0,1]$ of Hausdorff dimension $1/2$ so that for each distinct $x_1,x_2,x_3\in X_1$, we have $(x_1,x_2,x_3)\not\in Z$. This is precisely the statement that for each $x_1,x_2,x_3\in X$, the points $(x_1,f(x_1)),\ (x_2,f(x_2))$, and $(x_3,f(x_3))$ do not form the vertices of an isosceles triangle. To complete the proof, let $X = \{ (x,f(x)) : x \in X_1 \}$.
\end{proof}

\begin{lemma}
	Let $f\colon [0,1) \to \RR^{n-1}$ be Lipschitz. Then the set
	%
	\[ Z = \left\{ (x_1,x_2,x_3) \in [0,1]^3 : \begin{array}{c} (x_1,f(x_1)), (x_2,f(x_2)), (x_3,f(x_3))\\
		\textrm{form the vertices of an isosceles triangle} \end{array} \right\}. \]
	%
	has Minkowski dimension at most two.
\end{lemma}
\begin{proof}
By translating and rescaling the range of $f$, which does not change the Minkowski dimension of the graph, we may assume withous loss of generality that $f$ is $1/10$ Lipschitz, and that it's graph is contained in $[0,1]^n$. Fix $0<\delta<1$. It suffices to show that
	%
	\begin{equation}\label{deltaCoveringZ}
		\# \mathcal{B}_{\delta}^{3}(Z)\lesssim\delta^{-2}\log(1/\delta).
	\end{equation}
	%
	We have
	%
	\begin{align*}
		|\mathcal{B}_{\delta}^{3}(Z)| &= \sum_{I_1 \in \mathcal{B}_{\delta}^1([0,1])} \sum_{k=0}^{\log_2(1/\delta)} \sum_{\substack{I_2 \in \mathcal{B}_{\delta}^1([0,1]) \\ \operatorname{dist}(I_1,I_2)\sim \delta 2^k }} \#\{ I_3\in \mathcal{B}_{\delta}^1([0,1]) : I_1\times I_2\times I_3\in \mathcal{B}_{\delta}^{3}(Z) \}.
	\end{align*}
	%
	In the above expression we abuse notation slightly and say that $\operatorname{dist}(I_1,I_2)\sim \delta$ if $I_1=I_2$; this will not affect our estimates.

Note that for each $I_1 \in \mathcal{B}_{\delta}^1([0,1])$, there are roughly $(\delta 2^k)/\delta=2^k$ intervals $I_2\in  \mathcal{B}_{\delta}^1([0,1])$ with $\operatorname{dist}(I_1,I_2)\sim \delta 2^k$. Thus to establish \eqref{deltaCoveringZ}, it suffices to prove that for each $I_1 \in \mathcal{B}_{\delta}^1([0,1])$ and each $I_2\in  \mathcal{B}_{\delta}^1([0,1])$ with $\operatorname{dist}(I_1,I_2)\sim \delta 2^k$, we have
\begin{equation}\label{numberOfI3}
\#\{ I_3\in \mathcal{B}_{\delta}^1([0,1]) : I_1\times I_2\times I_3\in \mathcal{B}_{\delta}^{3}(Z) \}\lesssim 2^{-k}/\delta.
\end{equation}

For each distinct $p,q\in [0,1]^n$, define 
%
\[ H_{p,q}= \left\{z\in \RR^n : \left(z-\frac{p+q}{2}\right)\cdot (p-q)=0  \right\}. \]
%
This is the hyperplane passing through the midpoint of $p$ and $q$ that is perpendicular to the line passing through $p$ and $q$. We will call $H_{p,q}$ the perpendicular bisector of $p$ and $q$.

Fix a choice of intervals $I_1$ and $I_2$ with $\operatorname{dist}(I_1,I_2)\sim \delta 2^k$. Let $\tilde I_1$ and $\tilde I_2$ denote the twofold dilates of $I_1$ and $I_2$, respectively. Note that if $I_3\in\mathcal{B}_\delta^1([0,1])$ with $I_1\times I_2\times I_3\in \mathcal{B}_{\delta}^{3}(Z)$, then there are points $x_j\in \tilde I_j,\ i=1,2,3$ so that 
%
\[ (x_3,f(x_3))\in H_{(x_1,f(x_1)), (x_2,f(x_2))}. \]
%
Consider the set
%
\[ S_{I_1,I_2}=[0,1]^n \cap \bigcup_{\substack{x_1\in \tilde I_1\\ x_2\in \tilde I_2}}H_{(x_1,f(x_1)), (x_2,f(x_2))}. \]
%
For each $x_1\in \tilde I_1$ and $x_2\in \tilde I_2$, the line passing through $(x_1,f(x_1))$ and $(x_2,f(x_2))$ makes an angle $\leq 1/10$ with the $e_1$ direction. Thus the hyperplane $H_{(x_1,f(x_1)), (x_2,f(x_2))}$ makes an angle $\leq 1/10$ with the hyperplane spanned by the $e_2,\ldots,e_n$ directions. Since $\tilde I_1$ and $\tilde I_2$ are intervals of length $\leq 3\delta$ that are $\sim \delta 2^k$ separated, $S_{I_1,I_2}$ is contained in the $\sim 2^{-k}$ neighborhood of a hyperplane that makes an angle $\leq 1/10$ with the $e_2,\ldots,e_n$ directions.  

Suppose that $x_3,x_3^\prime\in [0,1]$ satisfy
%
\begin{equation}\label{x3x3pContainedR}
(x_3,f(x_3))\in S_{I_1,I_2}\quad\textrm{and}\quad(x_3^\prime,f(x_3^\prime))\in S_{I_1,I_2}.
\end{equation}
%
Since $f$ is $1/10$-Lipschitz, we must have 
%
\[ |f(x_3)-f(x_3^\prime)|\leq \frac{1}{10}|x_3-x_3^\prime|. \]
%
On the other hand, by \eqref{x3x3pContainedR} and the fact that $S_{I_1,I_2}$ is contained in the $\sim 2^{-k}$ neighborhood of a hyperplane that makes an angle $\leq 1/10$ with the $e_2,\ldots,e_n$ directions, we have
%
\[ |f(x_3)-f(x_3^\prime)|\geq 10|x_3-x_3^\prime|-O(2^{-k}). \]
%
We conclude that $|x_3-x_3^\prime|\lesssim 2^{-k}$. This establishes \eqref{numberOfI3}. We conclude that \eqref{deltaCoveringZ} holds, so $Z$ has lower Minkowski dimension at most 2. 
\end{proof}

\bibliographystyle{amsplain}
\bibliography{FractalsAvoidingFractalSetsPaper}

\end{document}