\documentclass{article}

\usepackage{amsthm}
\usepackage{amsmath}
\usepackage{multicol}
\usepackage{amssymb}
\usepackage{mathabx}
\usepackage{accents}
%\usepackage[margin=0.7in]{geometry}
\usepackage[english]{babel}
\usepackage{blindtext}

\theoremstyle{plain}
\newtheorem{lemma}{Lemma}
\newtheorem{prop}{Proposition}
\newtheorem*{example}{Example}
\newtheorem*{fact}{Fact}
\newtheorem*{corollary}{Corollary}

\usepackage{algorithm}
\usepackage[noend]{algpseudocode}

\theoremstyle{plain}
\newtheorem{theorem}{Theorem}
\newtheorem{proposition}[theorem]{Proposition}
\newtheorem*{remark}{Remark}

\def\changemargin#1#2{\list{}{\rightmargin#2\leftmargin#1}\item[]}
\let\endchangemargin=\endlist 

\DeclareMathOperator{\codim}{codim}

\title{Fractals Avoiding Fractal Sets}

\author{Jacob Denson\\ \and Malabika Pramanik\\ \and Josh Zahl}

\begin{document}

\maketitle

%\begin{multicols}{2}

\begin{abstract}
	\blindtext[1]
\end{abstract}

% Discrete vs Continuous analogues - how do they differ.
% Ergodic Theoretic Connections: Furstenberg Katznelson and Weiss. Ziegler sets of positive density.

Suppose you have to determine the stability of an operator on function spaces. Thinking geometrically, one can study how the operator acts on indicator functions. The existence of interesting characteristics in sets often gives insight into how the operator behaves on the corresponding indicator function. The problem then reduces to studying fractals with certain characteristics.
%
%\begin{itemize}
%	\item Classically, the tubular maximal operator
	%
%	\[ f_\delta^*(e) = \sup_{a \in \mathbf{R}^n} \frac{1}{|T_e^\delta|} \int_{a + T_e^\delta} |f(y)|\; dy \]
	%
%	where $T_e^\delta$ is a $\delta$ thickened line in the direct $e \in S^{n-1}$, satisfies a bound $\| f^*_\delta \|_{L^n(S^{n-1})} \lesssim_\varepsilon \delta^{-\varepsilon} \| f \|_{L^n(\mathbf{R}^n)}$ for all $\varepsilon$ if and only if every set in $\mathbf{R}^n$ containing a unit line segment in every direction has Hausdorff dimension $n$.

%	\item For a given Radon measure $\mu$ and kernel $K$, we can consider the singular integral operator
	%
%	\[ T(f)(x) = \int K(x,y) f(y)\; d\mu(x) dy \]
	%
%	The nonexistence of principal values for the operator depends on the geometric structure of the support of $\mu$. For instance, using the symmetry of the fractals given, in \cite{David} it is show that there is $\mu$ supported on a one-dimensional four corners Cantor set in the plane for which all principal values do not exist, and in \cite{Vasilis} the same is shown for low Hausdorff dimension variants of the planar Sierpinski triangle.

%	\item For a compactly supported Borel measure $\mu$ on $\mathbf{R}^n$, the authors of \cite{Bennett} consider the operator
	%
%	\[ T(\mu) = \int \prod_{i = 1}^k (\sigma_{t_i} * \rho_\varepsilon)(x_i - x_{i-1})\;  d\mu^{k+1}(x_0, \dots, x_k) \]
	%
%	where $\rho$ is a smooth cutoff function and $\sigma_r$ is the Lebesgue measure on the sphere of radius $r$ is $\mathbf{R}^n$. The existence of a uniform estimate $T(\mu) \leq C^k$ independent of $\varepsilon$ and $t_1, \dots, t_k$ implies that if a set $E$ has Hausdorff dimension greater than $(d+1)/2$, then the `$k$ distance set'
	%
%	\[ \Big\{ \big(|x_0 - x_1|, |x_1 - x_2|, \dots, |x_{k-1} - x_k|\big) : x_i \in E \Big\} \]
	%
%	has positive Lebesgue measure.
%\end{itemize}

In this paper, we do not discuss particular operators, but instead describe methods to find large sets avoiding fine-scale patterns. This provides users studying operators to have general tools to construct fractals for their needs. Important examples of fine-scale patterns include affine configurations, such as sets not containing the vertices of isosceles triangles, sets not containing three term arithmetic progressions, sets not generating particular families of angles, and sets not containing points in a common hyperplane. For these examples, the Lebesgue density theorem implies any set of positive measure contains these patterns. Thus we quantify the size of sets by their Hausdorff dimension.

There are two approaches to the pattern avoidance problem. We get upper bounds by proving sets with large Hausdorff dimension contains patterns. Constructing sets avoiding patterns with large Hausdorff dimension give lower bounds. In this paper, we focus on the {\it construction problem} for lower bounding pattern avoidance problems.

There are already generic pattern avoidance methods in the literature. We compare our method to them in detail in section 6. But these rely on the non-singularity of the patterns. The novel feature of our method is we can avoid points with patterns lying on an {\it arbitrary} fractal set, and the Hausdorff dimension of our constructions is still comparable to previous methods.

The {\it key idea} of our method is the introduction of a new geometric framework for pattern avoidance problems, described in section 1. A simple combinatorial argument, described in section 2, exploited repeatedly in section 3 via a queueing process leads directly to a pattern avoiding set. We believe this new geometric framework should help find further methods in the field. We show this by proving another pattern avoidance result in section 5, assuming extra geometric information on the patterns.

\section{A Fractal Avoidance Framework}

The common framework used to think about pattern avoidance problems is to specify the pattern as the zero set of some function.
%
\begin{itemize}
	\item A set $X \subset \mathbf{R}$ contains no three term arithmetic progressions if and only if for any distinct $x,y,z \in X$,
	%
	\[ f(x,y,z) = x + z - 2y \neq 0  \]
	%
	This function vanishes if and only if there exists $b$ and $t$ such that $x = b$, $y = b+t$, and $z = b+2t$.

	\item A set $X \subset \mathbf{R}^d$ contains the vertices of no isosceles triangles if and only if for any three distinct $x,y,z \in X$,
	%
	\[ f(x,y,z) = d(x,y) - d(y,z) \neq 0 \]
	%
	The problem of avoiding this function is similar to the last example, since isosceles triangles can be considered as planar variants of three term arithmetic progressions.

	\item A set $X \subset \mathbf{R}^d$ does not contain a family of angles $\{ \alpha_i \}$ if and only if for any distinct $x,y,z \in X$, and $i$,
	%
	\[ f(x,y,z) = \frac{(x - z) \cdot (y - z)}{|x - z||y - z|} \neq \cos(\alpha_i) \]
	%
	where the cosine formula for the dot product is used.

	\item A set $X \subset \mathbf{R}^d$ does not contain $d+1$ points in a lower dimensional hyperplane if and only if for any distinct $x_0, \dots, x_d \in X$,
	%
	\[ f(x_0, \dots, x_d) = \det(x_1 - x_0, \dots, x_d - x_0) \neq 0 \]
	%
	since $\det(x_1 - x_0, \dots, x_d - x_0) = 0$ only when the vectors $x_1 - x_0, \dots, x_d - x_0$ do not form a basis, and thus span a plane of dimension smaller than $d$.
\end{itemize}
%
This leads naturally to a generic formulation of pattern avoidance problem used commonly in the literature:

\begin{changemargin}{0.5em}{0em}
{\bf The Configuration Avoidance Problem:} Given a function $f: (\mathbf{R}^d)^n \to \mathbf{R}$ as input, find $X \subset \mathbf{R}^d$ such that for any {\it distinct} $x_1, \dots, x_n \in X$, $f(x_1, \dots, x_n) \neq 0$, with as high a Hausdorff dimension as possible.
\end{changemargin}

This is the viewpoint for the methods of \cite{MalabikaRob} and \cite{Mathe}, who give results assuming various regularity conditions on the function $f$. It is the viewpoint of this paper that the function $f$ contains extraneous information which is irrelevant to the problem. The only important information we need to extract from the function $f$ is the geometric structure of it's zero set. If we denote the zero set of $f$ by $Z$, the configuration avoidance problem becomes equivalent to a more flexible framework:

\begin{changemargin}{0.5em}{0em}
	{\bf The Fractal Avoidance Problem:} Given $Z \subset (\mathbf{R}^d)^n$, find a set $X \subset \mathbf{R}^d$ such that $X^d \cap Z \subset \Delta$, where $\Delta = \{ x \in (\mathbf{R}^d)^n : x_i = x_j\; \text{for some $i$ and $j$} \}$, with as high a Hausdorff dimension as possible.
\end{changemargin}

The set $\Delta$ is not an artificial part of the framework. It fulfills the same role as the condition in the configuration avoidance problem that we only require {\it distinct} variables $x_1, \dots, x_n$ to satisfy $f(x_1, \dots, x_n) \neq 0$. The interesting configuration avoidance problems we consider occur at affine scales of all orders. In particular, this means that if $f$ specifies the configuration and is continuous, then we should have $f(x,\dots,x) = 0$ for all $x$. Fortunately, as in the examples considered at the beginning of the section, in most configuration avoidance problems encountered, the distinctness of the variables is implicit in the problem.

Because we are the first to introduce the fractal avoidance problem, a natural first goal is to solve the generic problem with minimal assumptions on $Z$. Here we let $Z$ take the form of an arbitrary fractal, and the only assumptions we place on $Z$ are it's fractal dimension.

\begin{theorem}
	If $Z$ has Minkowski dimension $\alpha \geq d$, then there exists $X$ solving the fractal avoidance problem for $Z$ with
	%
	\[ \dim_{\mathbf{H}}(X) = \frac{nd - \alpha}{n - 1} = \frac{\codim_{\mathbf{H}}(Z)}{n - 1} \]
	%
	which is a solution 
\end{theorem}

\begin{remark}
	If $Z$ has dimension $\alpha < d$, then the set obtained from $\mathbf{R}^n$ by removing the projections of $Z$ onto each coordinate has full Hausdorff dimension and trivially solves the fractal avoidance problem. Thus the problem is trivial for these parameters.
\end{remark}

A second goal of our paper is to show an example where assuming extra geometric conditions on $Z$ leads to constructions with a higher Hausdorff dimension. Thus the framework naturally incorporates further methods. We consider a condition where $Z$ is efficiently coverable by parallel hyperplanes of a fixed dimension.

\begin{theorem}
	If there is $k \geq 2$ and a linear $T: \mathbf{R}^{nd} \to \mathbf{R}^{kd}$ such that $T(Z)$ is $\alpha$ dimensional, with $\alpha \leq (d-1)k$, then there exists $X$ with
	%
	\[ \dim_{\mathbf{H}}(X) = \frac{dk - \alpha}{2k-1} \]
	%
	solving the fractal avoidance problem for $Z$.
\end{theorem}

Because of the lack of any {\it rigid} geometric information about the set $Z$, such as smoothness, the only techniques we can use to avoid $Z$ are discretizing the problem using covering arguments. This can be neatly summarized as a combinatorial argument on graphs, which we detail in the next section and exploit repeatedly in the proofs of both results.



\section{Avoidance at a Single Scale}

We now develop a discrete technique used to construct solutions to the fractal avoidance problem. Once we have discretized the problem, the Euclidean structure of the setting becomes irrelevant. As such, we rephrase our single scale avoidance method as a combinatorial problem on graphs. Here we prove a form of Tur\'{a}n's theorem, which quantifies the idea that graphs with few edges contain large independant sets. In the next section, we apply this theorem to find large discretized sets avoiding a discretization of the fractal we are required to avoid. Exploiting this repeatedly at an infinite series of discretizations then gives a set completely avoiding the required fractal.

Recalling notation, we consider a fixed set of $V$ elements, whose points we call {\it vertices}. A set of $n$ {\it distinct} vertices will be called an {\it $n$ vertex hyperedge}. The structure consisting of a set of vertices, and a family of $n$ vertex hyperedges on those vertices, will be called an {\it $n$ uniform hypergraph}. We let $E$ denote the number of edges in such a graph. A subset of vertices which does not contain any hyperedge as a subset is known as an {\it independent set}. We will consider partitions of the vertex set, and we say a partition is {\it $K$ spread} if there are at least $K$ vertices in each equivalence class.

\begin{lemma}
	Every $n$ uniform hypergraph with a $K$ spread partition contains an independent set with elements selected from all but at most $E/K^n$ equivalence classes.
\end{lemma}
\begin{proof}
	Let $X$ be a random vertex set chosen by selecting a representative vertex from every equivalence class uniformly at random. Then every vertex occurs in $X$ with probability at most $1/K$. If an edge $e = \{ v_1, \dots, v_n \}$ satisfies $\mathbf{P}(v_1, \dots, v_n \in X) > 0$, then the $v_1, \dots, v_n$ are distinct, and lie in separate equivalence classes of the partition. This implies that $v_1, \dots, v_n$ each occur in $X$ with independant likelihood, so a bound on the probability that all vertices in $e$ lie in $X$ is
	%
	\begin{align*}
		\mathbf{P}(v_1, \dots, v_n \in X) = \mathbf{P}(v_1 \in V) \dots \mathbf{P}(v_n \in X) \leq 1/K^n
	\end{align*}
	%
	If $E_0$ denotes the number of edges $e = \{ v_1, \dots, v_n \}$ with $v_1, \dots, v_n \in X$, then
	%
	\begin{align*}
		\mathbf{E}(E_0) = \sum_{e \in E} \mathbf{P}(e \in E_0) \leq \sum_{e \in E} 1/K^n = E/K^n
	\end{align*}
	%
	This means we may choose a particular, {\it nonrandom} $X$ for which $E_0 \leq E/K^n$. If we form a vertex set $W \subset V$ by removing, for each $e \in E_0$, a vertex in $X$ adjacent to $e$, then $W$ is an independent set containing representatives from all but $E_0 \leq E/K^n$ equivalence classes of the partition.
\end{proof}

%\begin{corollary}
%	If $|V| \gtrsim N^a$, $|E| \lesssim N^b$, and $K \gtrsim N^c$, where $b < a + c(n-1)$, then as $N \to \infty$ we can find an independent set containing all but a fraction $o(1)$ of the colors.
%\end{corollary}
%\begin{proof}
%	A simple calculation on the quantities of the previous lemma yields
	%
%	\begin{align*}
%		\frac{\# ( \text{colors removed} )}{\# ( \text{all colors} )} = \frac{|E|/K^n}{|V|/K} = \frac{|E|}{|V|K^{n-1}} \lesssim \frac{N^b}{N^{a + c(n-1)}}
%	\end{align*}
	%
%	This is $o(1)$ if $b < a + c(n-1)$.
%\end{proof}

%To apply the theorem to fractals, we will take cubes in the Euclidean plane as our vertices, and connect an edge between cubes if their cartesian product intersects a portion of $Y$. For technical reasons, we require that the cubes are essentially uniformly chosen across the candidate set of choices, and this is the reason for the introduction of partitions of vertices in the graph theory result above.

An important feature of our bound on the independant set selected is the ratio between the number of equivalence classes we choose versus the total number of equivalence classes. Since each class contains $K$ elements, there are $V/K$ equivalence classes. We calculate
%
\[ \frac{\# (\text{classes not chosen})}{\# (\text{total classes})} \leq \frac{E/K^n}{V/K} \leq \frac{E/V}{K^{n-1}} \]
%
To make this ratio small, we upper bound $E$, and lower bound $V$ and $K$.

%In our case, $E$ will be dependant on the Hausdorff dimension of $Y$, $V$ will depend on the sidelength

%In our case, we get few edges if our set $Y$ has low Hausdorff dimension, and we have lots of vertices if we choose a proportionally small sidelengths for our cubes.

%We now apply these constructions to a problem clearly related to the fractal avoidance problem. It will form our key method to construct fractal avoidance solutions. Given a real number $L$, we subdivide $\mathbf{R}^d$ into a lattice of side length $L$ cubes with corners on $L \cdot \mathbf{Z}^d$, the collection of such cubes we will denote by $\mathcal{B}(L,d)$. This grid is used to granularize configuration avoidance.

%\begin{theorem}
%	Suppose $\mathcal{I}_1, \dots, \mathcal{I}_n$ are disjoint collections of cubes in $\mathcal{B}(L,d)$, with $|\mathcal{I}_i| \gtrsim (1/L)^d$. If $\alpha$ strictly bounds the lower Minkowski dimension of $Z$ from above, and a rational parameter
	%
%	\[ \beta > \max \left(1, d \cdot \frac{n-1}{nd-\alpha} \right) \]
	%
%	is fixed, then there exists collections of cubes $\mathcal{J}_1, \dots, \mathcal{J}_n \in \mathcal{B}(L^\beta,d)$ with each cube in $\mathcal{J}_1 \times \dots \times \mathcal{J}_n$ disjoint from $Z$, and each $\mathcal{J}_i$ containing cubes in all but a fraction $o(1)$ of cubes in $\mathcal{I}_i$ as $L \to 0$.
%\end{theorem}
%\begin{proof}
%	We reduce this problem to the independent set problem we just proved a result about. We define the discretization of $Z$ to be
	%
%	\[ \mathcal{Z} = \{ I \in \mathcal{B}(L^\beta,nd) : I \cap Z \neq \emptyset \} \]
	%
%	For each $i$, we set
	%
%	\[ \mathcal{I}_i' = \{ I \in \mathcal{B}(L^\beta,d) : \text{there is}\ J \in \mathcal{I}_i\ \text{s.t.}\ I \subset J \} \]
	%
%	We obtain an $n$ uniform hypergraph $G$ by taking the cubes in $\mathcal{I}_1', \dots, \mathcal{I}_n'$ as vertices, with a hyperedge between $I_1 \in \mathcal{I}'_1, \dots, I_n \in \mathcal{I}'_n$ if $I_1 \times \dots \times I_n \in \mathcal{Z}$. We then take a coloring on $G$ by declaring two cubes in $\mathcal{I}_i'$ to be the same color if they are contained in a common cube in $\mathcal{I}_i$. In particular, this gives us a true coloring because every color occurs solely in $\mathcal{I}_i$ for some index $i$, and every edge in the hypergraph contains exactly one vertex from each index. An independent set in this graph can be written as $\mathcal{J}_1, \dots, \mathcal{J}_n$, where the $\mathcal{J}_i$ are collections of cubes which are sub-cubes of cubes in $\mathcal{I}_i$ and no cube in $\mathcal{J}_1 \times \dots \times \mathcal{J}_n \subset \mathcal{B}(1/N^\beta,nd)$ intersects $Z$. Thus to prove the theorem it suffices to find a large independent set in this graph.

%	We employ the corollary we just proved to do this, by bounding the number of vertices and edges in $G$. Since a cube in $\mathcal{B}(L,d)$ contains $\Omega(1/L^{d(\beta-1)})$ cubes in $\mathcal{B}(L^\beta,d)$, we conclude that $|\mathcal{I}'_i| \gtrsim (1/L)^{d(\beta - 1)} |\mathcal{I}_i| \gtrsim (1/L)^{d \beta}$. But then the number of vertices of $G$ is $\sum |\mathcal{I}'_i| \gtrsim (1/L)^{d \beta}$. The Minkowski dimension bound on $Z$ implies that the number of edges in $G$ is bounded above by $|\mathcal{Z}| \lesssim (1/L)^{\alpha \beta}$,Finally, the coloring is $N^{d(\beta - 1)}$ uniform. Thus in the terminology of the previous corollary, $a = d \beta$, $b = \alpha \beta$, and $c = d(\beta - 1)$. The inequality $\beta > d(n-1)/(nd - \alpha)$, is equivalent to the inequality $b < a + c(n-1)$, and so the corollary applies to give the required result.
%\end{proof}

%The value $d \cdot (n-1)/(n-\alpha)$ in the theorem is directly related to the dimension $(n-\alpha)/(n-1)$ we get in our main result. If, for a specific $Z$, we can prove a variant of this lemma with $d \cdot (n-1)/(n-\alpha)$ replaced with $d/\lambda$, then going through the rest of our proof will immediately yield $X$ with Hausdorff dimension $\lambda$. In particular, to prove our second result it will suffice to prove the theorem above given that $Z$ has a low rank assumption and $d \cdot (n-1)/(n-\alpha)$ is replaced with $d \cdot(2k-1)/(dk - k - \alpha)$.

% TODO: Include Tightness Calculation?

\section{A Fractal Avoiding Set}

We now describe an avoidance technique applied at a single scale. The technique is repeated for an infinite sequence of scales to then get the general result. Given a dyadic length $L$, we let $\mathcal{B}(L,d)$ denote the partition of $\mathbf{R}^d$ into a family of half open cubes with corners lying on the lattice $(\mathbf{Z}/L)^d$. If the dimension is clear, we simplify denote $\mathcal{B}(L,d)$ as $\mathcal{B}(L)$.

\begin{lemma}
	Consider three dyadic scales $L \gg R \gg S$. If $I \subset \mathbf{R}^d$ is a union of $\mathcal{B}(L)$ cubes and $K \subset \mathbf{R}^{nd}$ a union of $\mathcal{B}(S)$ cubes, then we can find $J$ such that for any distinct $\mathcal{B}(S)$ subcubes $J_1, \dots, J_n$ of $J$, $J_1 \times \dots \times J_n$ is disjoint from $K$, and $J$ contains a $\mathcal{B}(S)$ subcube from all but $|K|R^{-nd}$ of the $\mathcal{B}(R)$ subcubes of $I$. 
\end{lemma}
\begin{proof}
	Form a random set $U$ by selecting uniformly randomly, from each $\mathcal{B}(R)$ subcube of $I$, a single subcube in $\mathcal{B}(S)$. Thus the probability that any subcube is selected is $(S/R)^d$. Since any two $\mathcal{B}(S)$ subcubes of $U$ lie in distinct elements of $\mathcal{B}(R)$, the only chance that a $\mathcal{B}(S)$ subcube $I$ of $K$ with distinct sides intersects $U^n$ is if $I_1, \dots, I_n$ all lie in separate cubes in $\mathcal{B}(R)$. Then the chance that each occurs is independant of one another, and so
	%
	\begin{align*}
		\mathbf{P}(I \in U^n) &= \mathbf{P}(I_1 \in U) \dots \mathbf{P}(I_n \in U) = (S/R)^{nd}
	\end{align*}
	%
	If $E$ denotes the number of $\mathcal{B}(S)$ subcubes $I$ of $K$ contained in $U^n$,
	%
	\[ \mathbf{E}(E) = \sum_{\substack{I \subset K}} \mathbf{P}(I \in U^n) = [|K| S^{-nd}] [(S/R)^{nd}] = |K| R^{-nd} \]
	%
	If, for each $\mathcal{B}(S)$ subcube $I$ of $U^n$, we remove $I_i$ from $U$, for any index $i$, we obtain a set $J$ with $J_1 \times \dots \times J_n$ disjoint from $K$ for any distinct $\mathcal{B}(R)$ subcubes $J_i$ of $J$. The interval $J$ contains an interval from all but $E$ sidelength $R$ cubes. In particular, we can select some nonrandom choice of $U$ such that $E \leq |K| R^{-nd}$, which gives the required $J$.
\end{proof}

We will be most interested in applying this lemma when $L$ and $S$ are consecutive hyperdyadic numbers. We fix a small $\delta$, and define $H_N = 2^{-\lfloor (1 + \delta)^N \rfloor}$ to be the $N$'th hyperdyadic number. Hyperdyadic numbers are useful scales for discussing Hausdorff dimension, because a weak-type bound allows us understand coverings of the set. The idea to use such scales was inspired by Katz and Tao's 2001 paper on Falconer's distance problem. We say a sequence $Y_1, Y_2, \dots$ {\it strongly} covers $Y$ if $Y \subseteq \limsup Y_N$, so each $y \in Y$ is in infinitely many of the sets $Y_N$.

\begin{lemma}
	If $Y$ has Hausdorff dimension $\alpha$, then for each $\varepsilon > \delta$ we can find a strong cover of $Y$ by sets $Y_N$, where $Y_N$ is a union of $O_\varepsilon(N/H_N^{\alpha + \varepsilon})$ hyperdyadic cubes with sidelength $H_N$.
\end{lemma}
\begin{proof}
	Since $Y$ is $\alpha$ dimensional, then for each $\varepsilon_0$ and $N$ we can find a covering of $Y$ by dyadic cubes $I_i$, where $I_i$ has sidelength $L_i \leq H_N$, and $\sum L_i^{\alpha + \varepsilon_0} \lesssim_{\varepsilon_0} 1$. But we can now apply a weak type bound to conclude that the number of $L_i$ between $H_{N+1}$ and $H_N$ is $O_{\varepsilon_0}(1/H_{N+1}^{\alpha + \varepsilon_0})$. The calculation
	%
	\[ 1/H_{N+1}^{\alpha + \varepsilon_0} = (H_N/H_{N+1})^{\alpha + \varepsilon_0} (1/H_N^{\alpha + \varepsilon_0}) \lesssim 1/H_N^{\alpha + \varepsilon_0 + \delta} \]
	%
	implies we have used $O_{\varepsilon_0}(1/H_N^{\alpha + \delta + \varepsilon_0})$ sidelength $H_N$ cubes to cover $Y$. If we now perform this process for each index $N$, then collect all cubes together from all the covers to obtain a strong cover, and set $\varepsilon = \delta + \varepsilon_0$, we obtain the required result.
\end{proof}

\begin{corollary}
	Suppose $L = H_N$, $S = H_{N+1}$, and $R$ is the closest dyadic number to $S^\beta$. Furthermore, suppose $|K| \lesssim S^{nd - \gamma}$. Then provided
	%
	\[ nd - \gamma - \beta(n-1)d = \Omega \left( \frac{ 1 + \log |I|^{-1}}{(1 + \delta)^{N+1}} \right) = \log_S |I| + \Omega(1/\log(1/S)) \]
	%
	the set $J$ obtained in the discrete avoidance lemma contains a portion of at least half of all the sidelength $L_1$ intervals contined in $I$.
\end{corollary}

\begin{proof}
	We simply calculate
	%
	\begin{align*}
		&\frac{\# (\text{$\mathcal{B}(R)$ subcubes not selected from})}{\# (\text{all $\mathcal{B}(R)$ subcubes})} = \frac{|K| R^{-nd}}{|I|R^{-d}}\\
		&\ \ \ \ \ \lesssim \frac{[S^{nd - \gamma}][S^{- \beta n d}]}{|I| S^{-\beta d}} = |I|^{-1} S^{nd - \gamma - \beta (n-1)d}
	\end{align*}
	%
	The inequality above guarantees that the quantity above is bounded by $1/2$, so less than half of the intervals are missed in the selection process.
\end{proof}

%\begin{theorem}[Katz-Tao, 2001]
%	If $Y \subset \mathbf{R}^{nd}$ is an $\alpha$ dimensional set, and a small $\varepsilon > 0$ is given, then we can find an efficient {\it strong cover} of $Y$ by a sequence of discretized sets $Y_N$. By a strong cover, we mean $Y \subseteq \limsup Y_N$. By discretized, we mean $Y_N$ is a union of cubes with sidelength's between $C_\varepsilon^{-1} H_N^{1 + C\varepsilon}$ and $C_\varepsilon H_N^{1 - C\varepsilon}$. By an efficient cover, we mean that $|Y_N| \leq C_\varepsilon H_N^{nd - \alpha - C\varepsilon}$.
%\end{theorem}

%If we replace $Y_N$ with a set obtained from uniform sidelength cubes, each with sidelength $C_\varepsilon H_N^{1 - C\varepsilon}$, then we obtain a new strong cover with total mass $O_\varepsilon(H_N^{nd - \alpha - C\varepsilon(2nd + 1)})$

%\begin{align*}
%	2 \sum C_\varepsilon^{nd} H_N^{nd(1 - C\varepsilon)} &= 2 C_\varepsilon^{2nd} H_N^{-2 C\varepsilon nd} \sum C_\varepsilon^{-nd} H_N^{nd(1 + C \varepsilon)}\\
%	&\leq 2 C_\varepsilon^{2nd} H_N^{-2C\varepsilon nd} C_\varepsilon H_N^{nd - \alpha - C\varepsilon}\\
%	&= 2C_\varepsilon^{2nd + 1} H_N^{nd - \alpha - C \varepsilon (2nd + 1)}
%\end{align*}

%If we replace each cube in $Y_N$ by an expanded cube with sidelength $C_\varepsilon H_N^{1 - C\varepsilon}$, we get a set constructed from cubes with a uniform sidelength, 

%$(C_\varepsilon^{-1} H_N^{1 + C\varepsilon})^{nd} \leq |B| \leq (C_\varepsilon H_N^{1 - C\varepsilon})^{nd}$

%If $L$ and $S$ are the closest dyadic numbers to $C_\varepsilon H_N^{1 - C\varepsilon}$ and $C_\varepsilon H_{N+1}^{1 - C\varepsilon}$ respectively, and $|K| \leq H_{N+1}^{(1 - C\varepsilon)(n - \alpha)}$

We now describe a general method for finding solutions to fractal avoidance problems by breaking the problem down into a sequence of discrete configuration problems. To achive this, we rely on a hyperdyadic discretization of the set $Y$. If we fix a small constant $\delta$, we obtain a sequence of dyadic scales $\smash{L_N = 2^{- \lfloor (1 + \delta)^k \rfloor}}$. These lengths are known as {\it hyperdyadic}. Since $2^{-(1 + \delta)^k} \leq L_N \leq 2^{1- (1 + \delta)^k}$, the floor operation doesn't get in our way that much.

For any length $L$, we let $\mathcal{B}(L,d)$ denote the family of all cubes with sidelength $L$, and with corners on the lattice $(\mathbf{Z}/L)^d$. If $d$ is clear, we abbreviate the notation as $\mathcal{B}(L)$. It is a result of Katz and Tao that if $Y \subset \mathbf{R}^{nd}$ is a set with Hausdorff dimension $0 < \alpha < nd$, and $\varepsilon$ is a small constant, then we can find an efficient {\it strong cover} of $Y$ by a union of sets $Y_N \subset \mathbf{R}^{nd}$, for $N \geq 1$, with $Y_N$ a union of cubes in $\mathcal{B}(C_\varepsilon L_N^{1 - \varepsilon})$. By a strong cover, we mean $Y \subseteq \limsup Y_N$, and by efficient, we mean that $Y_N$ is the union of at most $\smash{C_\varepsilon L_N^{-(\alpha + \varepsilon)}}$ cubes with sidelength $L_N$. For the purposes of our calculations, we fix a constant $A$ such that $Y_N$ contains at most $A/L_N^\alpha$ sidelength $L_N$ cubes\footnote{As Josh indicated, this bound might not be true, but can hopefully be adjusted as required for the true result which Katz and Tao prove}.

We now construct $X$ as a limit of discrete nested sets $X_N$, where $X_N$ is a union of cubes in $\mathcal{B}(L_N)$, and $X_N^n$ is disjoint from all {\it non-diagonal} cubes with length $L_N$ in $Y_N$. By a non-diagonal cube, we mean $I \subset \mathcal{B}(L_N)$ such that $\pi_i(I) \neq \pi_j(I)$, where $\pi_i, \pi_j: \mathbf{R}^{nd} \to \mathbf{R}^d$ are projections onto the $i$'th and $j$'th coordinate vector. In particular, this means that $X^n$ avoids all cubes at all lengths in the cover of $Y$, and in particular, avoids non-diagonal elements of $Y$. As an initial set, for lack of an interesting choice, we let $X_0 = [0,1]^d$.

\begin{lemma}
	If for all $N$, $X_N^n$ avoids non-diagonal cubes in $Y_N$, $X^n \cap Y \subset \Delta$.
\end{lemma}
\begin{proof}
	Let $y \in Y$ be a point not contained in $\Delta$. Because $Y_N$ forms a strong cover, we can find an infinite sequence of indices $N_k$ with $y \in Y_{N_k}$. For a suitably large choice of $K$, $\sqrt{n} \cdot L_{N_k} < d(y,\Delta)$ for $k \geq K$. But this means that the cube of $\mathcal{B}(L_{N_k})$ containing $y$ is disjoint from $\Delta$, and therefore cannot be a non-diagonal cube. This means that $X_{N_k}^n$ is disjoint from this cube, and therefore does not contain $y$. But then $X^n = \lim X_{N_k}^n$ cannot contain $y$. Taking contrapositives to our argument, we have shown every point in $X^n \cap Y$ must lie in $\Delta$.
\end{proof}

The recursive definition of $X_{N+1}$ from $X_N$ relies on the discrete result just proved in the last section. We set $I = X_N$, and $X_{N+1}$ equal to the resultant $J$, where $K$ is formed from the union of cubes in the hyperdyadic cover of $K$.

Thus we must formulate the problem of avoiding cubes as a problem of finding independant sets in hypergraphs. Given $X_N$, which is a union of cubes in $\mathcal{B}(L_N)$, we divide the cubes into a fine scale set of cubes in $\mathcal{B}(L_{N+1})$. These cubes will form the vertices of our hypergraph. Given a set of $n$ {\it distinct} cubes $I_1, \dots, I_n$ at the fine scale, we add an edge between them if $I_1 \times \dots \times I_n$ is contained in $Y_{N+1}$. An independant set of vertices then corresponds exactly to a selection of cubes whose cartesian product does not contain any non-diagonal elements of the cubes in $Y_{N+1}$. We will find a large such independant set in this graph, using the technique of the last section. The union of the corresponding cubes will form $X_{N+1}$.

For technical purposes, we require that the intervals in $X_{N+1}$ are uniformly chosen over intervals at a slightly coarser scale that $X_N$. This is the reason for the introduction of the partition in the graph theory result of the last section. We fix a parameter $(1 + \delta)^{-1} < \beta < 1$, and define $R_N = 2^{-\lfloor \beta(1 + \delta)^k \rfloor}$. Thus $L_{N+1} \leq R_{N+1} \leq L_N$. We can then partition the cubes considered as vertices in the graph above into equivalence classes consisting of cubes which are contained in a common element of $\mathcal{B}(R_{N+1})$. Given all this information, we now use the theorem of the last section to find a family of cubes containing a single interval in $\mathcal{B}(L_{N+1})$ from a large ratio of the intervals in $\mathcal{B}(R_{N+1})$. To find the ratio that is kept, we must bound the values of $V$, $E$, and $K$ for the graph we have constructed.

If $X_N$ contains $M_N$ sidelength $L_N$ cubes, then our graph has $M_N (L_N/L_{N+1})^d$ vertices. There are also at most $A/L_{N+1}^\alpha$ edges, since $Y_{N+1}$ contains at most this many cubes. Finally, each equivalence class of the partition contains $(R_{N+1} / L_{N+1})^d$ elements, since there are this many sidelength $L_{N+1}$ intervals in a single sidelength $R_{N+1}$ intervals. The discrete result of the last section enables us to find an independent set containing one cube from all but at most
%
\begin{align*}
	E/K^n &\leq [A/L_{N+1}^\alpha] [(R_{N+1} / L_{N+1})^d]^{-n}\\
		&= A L_{N+1}^{dn-\alpha} R_{N+1}^{-dn} \leq (4 A) 2^{(\alpha - (1 - \beta) d n) (1 + \delta)^{N+1}}
\end{align*}
%
sidelength $R_{N+1}$ cubes. The total number of such sidelength $R_{N+1}$ intervals is
%
\[ M_N (L_N/R_{N+1})^d \geq (M_N/4) 2^{\beta d(1 + \delta)^{N+1} - d (1 + \delta)^N} \]
%
This means that in order for our procedure to keep half of the intervals at each stage, we need
%
\[ (4A) 2^{[\alpha - (1 - \beta) dn] (1 + \delta)^{N+1}} \leq \frac{(M_N/4) 2^{\beta d(1 + \delta)^{N+1} - d (1 + \delta)^N}}{2} \]
%
We can rearrange this inequality to read
%
\[ (\alpha - (1 - \beta) dn - \beta d) (1 + \delta)^{N+1} \leq \log_2 (M_N/32A) \]

\newpage

We prove the estimate by induction. For simplicity, we assume the base case is proven. To obtain the estimate, if $M_N \geq B/L_N^d$, we have $M_{N+1} \geq (1 - AB) M_N (L_N/L_{N+1}^\beta)^d$. If we assume the inductive hypothesis for all $K \leq N$, then we can calculate that
%
\[ M_N \geq M_0 (1 - AB)^N L_0^d L_1^{d(1 - \beta)} \dots L_{N-1}^{d(1 - \beta)} L_N^{-\beta d} \]
%
Since $M_0 = L_0^{-d}$, and $L_{K+1} = L_K^{1 + \delta}$ for all indices $K$, this simplifies to an inequality
%
\begin{align*}
	M_N &\geq (1 - AB)^N L_N^{d(1 - \beta) \sum_{k = 1}^{N-1} (1 + \delta)^{-k} -\beta d}
\end{align*}
%
By our previous calculation, to continue the inductive proof that $M_{N+1} \geq B/L_{N+1}^d$, we must show
%
\[ (1 - AB)^{N+1} L_N^{d(1 - \beta) \sum_{k = 1}^{N-1} (1 + \delta)^{-k} - \beta d} (L_N/L_{N+1}^\beta)^d \geq B/L_{N+1}^d \]
%
This simplifies to proving
%
\[ (1 - AB)^{N+1} L_N^{d(1 - \beta)(1/\delta + 1 + \delta)} \geq B \]
%


\section{Dimension Bounds}

To complete the proof, it suffices to choose the parameters $R_N$ which lead to the correct Hausdorff dimension bound on $X$. To do this, we also fix a decreasing sequence $\lambda_N$ such that $\lambda_N \beta_N > d$, used later on in our argument. Since $\beta_N$ converges to $\beta$ from above, we can let $\lambda_N$ tend to $\lambda = (dn - \alpha)/(n - 1)$ from below. The fact that the dissection of $X_{N+1}$ for $X_N$ occurs uniformly over the will aid us in annihilating the super-exponentially increasing constants which inherently occur from the exponentially decreasing values of $L_N$ forced upon us.

We rely on the mass distribution principle to construct a probability measure $\mu$ supported on $X$. This enables us to calculate the Hausdorff dimension of $X$ using Frostman's lemma. We begin by putting the uniform probability measure $\mu_0$ on $X_0 = [0,1]^d$. Then, at each stage of the construction, we construct $\mu_{N+1}$ from $\mu_N$ by taking the mass on a certain side length $L_N$ cube in $X_N$, and uniformly distributing it's mass over the side length $L_{N+1}$ cubes in $I \cap X_{N+1}$. Using the weak compactness of the unit ball in $L^1(\mathbf{R}^d)^*$, we get a weak limit $\mu = \lim \mu_n$. The fact that $X_n$ contains the support of $\mu_n$ implies $X$ supports $\mu$.

It is intuitive that the mass on $\mu$ will distribute more thinly at each stage the fatter the cubes we keep on each iteration of the discrete scale argument. Quantifying this leads directly to our result. More precisely, we will prove that for each length $L$ interval $I$, $\mu(I) \lesssim_N L^{\lambda_N}$. Thus Frostman's lemma guarantees that $\dim_{\mathbf{H}}(X) \geq \lambda_N$, and taking $\lambda_N \to \lambda$ will complete the proof.

\begin{lemma}
	For $R_N \gg 0$, if $I \in \mathcal{B}(L_{N+1},d)$ and $J \in \mathcal{B}(L_N,d)$,
	%
	\[ \mu(I) \leq 2 (R_N/L_N)^d \mu(J)\ \ \ \mu(I) \leq 2^N R_0^{d - \beta_0} \dots R_N^{d - \beta_N} R_N^d \]
\end{lemma}
\begin{proof}
	If $I$ is not a cube in $X_{N+1}$, then
	%
	\[ \mu(I) = \mu_{N+1}(I) = 0 \]
	%
	so the inequality is obviously true. Otherwise, we can find a cube $J \in \mathcal{B}(L_N,d)$ in $I \cap X_N$. $J$ contains $(L_N/R_N)^d$ side length $R_N$ cubes. Our main discrete result implies that $X_{N+1}$ contains a side length $L_{N+1}$ cube in all but a fraction $o(1)$ of these cubes,. In particular, if we choose $R_N$ sufficiently large, then we know that we keep a side length $L_N$ portion of at least half of these cubes. Thus
	%
	\begin{align*}
		\mu(I) &= \mu_{N+1}(I) \leq \frac{\mu_N(J)}{(L_N/R_N)^d/2}\\
		&= 2 \mu_N(J) (R_N/L_N)^d = 2 \mu(J) (R_N/L_N)^d
	\end{align*}
	%
	completing the calculation. Applying this calculation iteratively, we conclude
	%
	\begin{align*}
		\mu(I) &\leq 2^N (R_0/L_0)^d (R_1/L_1)^d \dots (R_N/L_N)^d\\
		&= 2^N R_0^{d - \beta_0} \dots R_{N-1}^{d - \beta_{N-1}} R_N^d
	\end{align*}
	%
	completing the calculation.
\end{proof}

\begin{corollary}
	If $R_N \gg 0$, $\mu(I) \leq L_N^{\lambda_N}$ for $I \in \mathcal{B}(L_N,d)$.
\end{corollary}
\begin{proof}
	We write the inequality in the last problem as
	%
	\[ \mu(I) \leq [2^N R_0^{d - \beta_0} \dots R_{N-1}^{d - \beta_{N-1}} R_N^{d - \lambda_N \beta_N}] L_{N+1}^{\lambda_N} \]
	%
	Since $\lambda_N \beta_N > d$, the quantity in the square brackets is $o(1)$ as $R_N \to \infty$. Thus for sufficiently large $R_N$, we conclude that $\mu(I) \leq L_N^{\lambda_N}$.
\end{proof}

This is almost the required inequality, except we have only proven it for intervals at particular scales. To get a general inequality, we use the fact that our construction distributes uniformly across all intervals.

\begin{theorem}
	If $R_N \gg 0$, then we have $\mu(I) \leq 2^{d+1} L^{\lambda_N}$ for all intervals $I$ with side length $L \leq L_N$.
\end{theorem}
\begin{proof}
	We break our analysis into three cases, depending on the size of $L$:
	%
	\begin{itemize}
		\item If $R_N \leq L \leq L_N$, we can cover $I$ by at most $2^d(L/R_N)^d$ cubes in $\mathcal{B}(R_N,d)$. For each such cube, we know that the mass on each side length $R_N$ cube is at most $2(R_N/L_N)^d$ times the mass on an element of $\mathcal{B}(L_N,d)$. Thus
		%
		\begin{align*}
			\mu(I) &\leq [2^d(L/R_N)^d] [2(R_N/L_N)^d] [L_N^{\lambda_N}]\\
			&\leq \frac{2^{d+1} L^d}{L_N^{d - \lambda_N}} \leq 2^{d+1} L^{\lambda_N}
		\end{align*}
		%
		which gives the required result.

		\item If $L_{N+1} \leq L \leq R_N$, we can cover $L$ by at most $2^d$ cubes in $\mathcal{B}(R_N,d)$. Each cube in $\mathcal{B}(R_N,d)$ contains at most one cube in $\mathcal{B}(L_{N+1},d)$ which is also contained in $X_{N+1}$, so the bound in the last corollary gives that $\mu(I) \leq 2^d L_{N+1}^{\lambda_N} \leq 2^d L^{\lambda_N}$.

		\item If $L \leq L_{N+1}$, there certainly exists $M$ such that $L_{M+1} \leq L \leq L_M$, and one of the previous cases yields that $\mu(I) \leq 2^{d+1} L^{\lambda_M} \leq 2^{d+1} L^{\lambda_N}$.
	\end{itemize}
	%
	This covers all possible situations.
\end{proof}

To use Frostman's lemma, we need the result $\mu(I) \lesssim L^{\lambda_N}$ for an {\it arbitrary} interval, not just one with $L \leq L_N$. But this is no trouble; it is only the behavior of the measure on arbitrarily small scales that matters. This is because if $L \geq L_N$, then $\mu(I)/L^{\lambda_N} \leq 1/L_N^{\lambda_N} \lesssim_N 1$, so $\mu(I) \lesssim_N L^{\lambda_N}$ holds automatically for all sufficiently large intervals. Thus all problems with the Hausdorff dimension argument are complete, and we have proven that there is a choice of parameters which constructs a set $X$ with Hausdorff dimension no less than $(nd - \alpha)/(n-1)$. By looking at the way we dissect our intervals at each scale, it is easy to see that $X$ cannot have dimension any higher than this quantity, so it has {\it precisely} this dimension.

\section{Low Rank Projection Method}

We now introduce a method which enables us to find a higher dimensional subset $X$, under some `low rank' assumptions about the set $Z$. The theorem below should be compared to Theorem 3. We will substitute the theorem below in the construction of our solutions to the fractal avoidance problem to improve the Hausdorff dimension.

\begin{theorem}
	Let $\mathcal{I}_1, \dots, \mathcal{I}_n$ be disjoint collections of cubes in $\mathcal{B}(1/N,d)$, with $|\mathcal{I}_i| \gtrsim N^d$. If there is a linear transformation $T: \mathbf{R}^{nd} \to \mathbf{R}^{nk}$ with rational coefficients such that the Minkowski dimension of $T(Z)$ is bounded above by $\alpha$, and $\beta$ is a rational parameter such that $\beta > d(k-1)/(dk - \alpha - k)$, then for arbitrarily large $N$ such that $N^\beta$ is an integer, there exists collections of cubes $\mathcal{J}_1, \dots, \mathcal{J}_n \in \mathcal{B}(1/N^\beta,d)$ with each cube in $\mathcal{J}_1 \times \dots \times \mathcal{J}_n \subset \mathcal{B}(1/N^\beta,nd)$ disjoint from $Y$, and as $N \to \infty$, each $\mathcal{J}_i$ contains cubes in all but a fraction $o(1)$ of cubes in $\mathcal{I}_i$.
\end{theorem}
\begin{proof}
	Since for any integer $M$, $M \cdot T(Z)$ has the same Minkowski dimension as $T(Z)$, we may without loss of generality assume by multiplying by a large enough $M$ that $T$ has integer coefficients. Write $T(x) = S(x_1) + U(x_2)$, where $x_1$ is a subset of $kd$ coordinates of $x$, $x_2$ are the remaining $(n-k)d$ coordinates, and $S$ is invertible. Let $\mathcal{J}_{k+1}, \dots, \mathcal{J}_n \subset \mathcal{B}(1/N^\beta,d)$ be the set of cubes contained in a cube in $\mathcal{I}_{k+1}, \dots, \mathcal{I}_n$ and also containing a point in the lattice $(\mathbf{Z}/N)^{d(n-k)}$. We then consider the set $\mathcal{K}$ of cubes in $\mathcal{B}(1/N^\beta,kd)$ intersecting the image of some cube in $U(\mathcal{J}_{k+1}, \dots, \mathcal{J}_n)$. Because $U$ is integral, if $x \in (\mathbf{Z}/N)^{d(n-k)}$, then $U(x) \in (\mathbf{Z}/N)^{dk}$. Since all points in cubes in $\mathcal{J}_{k+1}, \dots, \mathcal{J}_n$ are contained in a $1/N^\beta$ thickening of the lattice $(\mathbf{Z}/N)^{d(n-k)}$, their image under $U$ is contained in a $\lesssim 1/N^\beta$ thickening of the lattice $(\mathbf{Z}/N)^{dk}$. Since the image of the cubes $U(\mathcal{J}_{k+1}, \dots, \mathcal{J}_n)$ is a bounded set, this implies $|\mathcal{K}| \lesssim N^{dk}$. If we let
	%
	\begin{align*}
		\mathcal{Z} = &\{ I \in \mathcal{B}(1/N^\beta,dk):\\
		&\ \ \ \text{there is}\ J \in \mathcal{K}\ \text{s.t.}\ (S(I) + J) \cap T(Z) \neq \emptyset \}
	\end{align*}
	%
	and we form a graph $G$ on the side length $1/N^\beta$ cubes in $\mathcal{J}_1, \dots, \mathcal{J}_k$ where there is an edge between $I_1, \dots, I_k$ if their Cartesian product lies in $\mathcal{Z}$, then an independent set $\mathcal{J}_1, \dots, \mathcal{J}_k$ gives a candidate solution $\mathcal{J}_1, \dots, \mathcal{J}_n$ for the entire theorem. Since $T(Z)$ intersects $O(N^{\alpha \beta})$ cubes in $\mathcal{B}(1/N^\beta,dk)$, $|\mathcal{K}| \lesssim N^{dk}$, and $S$ is invertible, $\mathcal{Z}$ contains $O(N^{dk + \alpha \beta})$ cubes. Then $G$ has $\Omega(N^{d \beta})$ vertices, $O(N^{dk + \alpha \beta})$ edges, and if we consider the coloring as in the previous problem, an $\Omega(N^{d(\beta - 1)})$ uniform coloring. Thus when applying the discrete corollary, we have $a = d \beta$, $b = dk + \alpha \beta$, and $c = d(\beta - 1)$. The inequality
	%
	\[ \beta > d \cdot \frac{2k - 1}{dk - \alpha} \]
	%
	is equivalent to the equality $b < a + c(k-1)$, which allows us to use the corollary to obtain the required result.
\end{proof}

Following our proof constructing $X$, as well as proving it's Hausdorff dimension, but using the lemma above instead of the standard lemma, and replacing the $\beta$ in this proof with the $\beta$ in the lemma above, we obtain $X$ solving the fractal avoidance problem for $Z$ with
%
\[ \dim_{\mathbf{H}}(X) = \frac{dk - \alpha}{2k - 1} \]
%
The fact that the dimension is independent of $n$ has some interesting consequences. In particular, it can be used to solve problems involving infinitely many variables.

\begin{example}
	Consider $Z = \{ (x,y): \mathbf{R}^{m+1}: y = f(Sx) \}$, where $S: \mathbf{R}^m \to \mathbf{R}^l$. If we consider the map $T(x,y) = (Sx,y)$, then $T(Z) = \{ (a,b) \in \mathbf{R}^{l+1}: b = f(a) \}$. This is a hypersurface of dimension $l$. Applying the result above with $k = l+1$, $d = 1$, and $\alpha = l$, we find a set $X$ with Hausdorff dimension $1/(2l + 1)$. This isn't exactly what we want, so maybe the proof can be fine tuned.
\end{example}

%\begin{theorem}
%	Let $Z$ be a zero set, and let $f: \mathbf{R}^{nd} \to \mathbf{R}^{kd}$ be a polynomial map with degree at most $m$ such that $f(Z)$ is $\alpha$ dimensional. Then can we improve our result?
%\end{theorem}

\section{Comparison with Other Generic Avoidance Schemes}

In the past few years in the discrete setting it has been noticed that rephrasing particular questions in terms of abstract problems on hypergraphs allows one to extend various results into sparse analogues \cite{BaloghMorrisSamotij}. In this paper we consider a continuous analogue, where sparsity is represented in terms of the dimension of the set $Z$ we are trying to avoid.

\section{Concluding Remarks}

\begin{thebibliography}{9}

\bibitem{KeletiDimOneSet}
Tam\'{a}s Keleti
\textit{A 1-Dimensional Subset of the Reals that Intersects Each of its Translates in at Most a Single Point}

\bibitem{MalabikaRob}
Robert Fraser, Malabika Pramanik
\textit{Large Sets Avoiding Patterns}

\bibitem{Mathe}
A. Ma\'{t}h\'{e}
\textit{Sets of Large Dimension Not Containing Polynomial Configurations}

\bibitem{BaloghMorrisSamotij}
J\'{o}zsef Balogh, Robert Morris, Wojceich Samotij
\textit{Independent Sets in Hypergraphs}

\bibitem{David}
Guy David
\textit{Bounded singular integrals on a Cantor set}

\bibitem{Vasilis}
Vasilis Chousionis
\textit{Singular Integrals On Sierpinski Gaskets}

\bibitem{Bennett}
Michael Bennett, Alex Iosevich, Krystal Taylor
\textit{Finite Chains Inside Thin Subsets of $\mathbf{R}^d$}

\end{thebibliography}

%\end{multicols}

\end{document}