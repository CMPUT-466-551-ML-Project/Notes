\documentclass{article}

\usepackage{amsthm}
\newtheorem{theorem}{Theorem}
\newtheorem{lemma}{Lemma}

\title{The Jordan Separation Theorem}
\author{Jacob Denson}

\begin{document}

\maketitle

The story of the Jordan Curve theorem begins with the words of Bernhard Bolzano, an italian priest turned mathematician:
%
\begin{quote}
    If a closed line lies in a plane and if by means of a connected line one joins a point of the plane which is enclosed within the closed line with a point which is not enclosed within it, then the connected line must cut the closed line.
\end{quote}
%
Classical language allows us to speak of things in a beautiful and profound way -- the only problem is it is impossible to comprehend what the person is actually saying. In fact, the main reason Bolzano challenged mathematicians with the theorem is that, at the time, no-one had sat out and worked out the details; what exactly was a curve, and was does it mean for a line to be connected? Perhaps the reason topology is so foundatational to modern mathematics is that it provides a language of talking about the geometrical in a precise, rigorous manner. The purpose of this article is to elaborate on how the language of topology enables us to reduce the theorem, elaborated by the mathematician Camille Jordan, down to something we can break apart, and even then in a non-trivial manner.

We have to be very careful in how we analyse this situation, because intuitive ideas break open easily when subjected to torrential logic. As a case example, consider another conjecture of intution.
%
\begin{theorem}
    If $D$ is a segment of the plane that is the boundary of every single one of the components it splits the remainder of the plane into, and one such component is bounded, then $D$ is a closed line.
\end{theorem}
%
Obviously, $D$ must be infinitisimally tiny, since it covers every one of the components, and since it it squeezed together so tinily, it must be some curve in the plane, right? Alas, this is not so, as the Japanese mathematician Kunizō Yoneyama constructs in his `Lakes of Wada' counterexample: We begin with a two holed disc (`two lakes on an island') in the plane (the ocean'), and allow water to flow from the ocean into the island. For each $d = 1/2, 1/3, \dots$, the lake dissolves the island -- we construct canals such that every point on the island comes within $d$ units of the water. In the limit, we obtain a land mass whose boundary is the two lakes and the ocean, and such that the two lakes are bounded, hence $D$ cannot be a closed line.

Logic is a capricious partner to modern mathematicians. On one hand, it gives us absolute certainty over the realm of theorems we seek to govern. At any moment, it can throw some monstrous creature which seeks to break down any intution you may hold dear. We have encountered the Cantor set and topologist's sine curve in our journey through topology, and the lakes of Wada provide another example of the perils of mathematics.

So what do we mean by a line? A first glance would suggest that it is simply the image of a continuous map from some segment $[a,b]$ to $\mathbf{R}^2$. Unfortunately, this does not work. The peano curve offers a nasty counterexample to why the Jordan Curve theorem isn't true. The true definition is fortunately even simpler to state after all the topology we've learned. A closed curve is simply a subset of the plane homeomorphic to the circle $S^1$. A more proper definition of the Jordan Curve theorem is the following:
%
\begin{theorem}[The Actual Jordan Curve Theorem]
The complement of any closed curve consists of two components, one of which is bounded.
\end{theorem}
%
Even this article has fallen into implicitly using the Jordan curve theorem to establish a statement! We deduced that the Lakes of Wada is not a closed curve {\bf because it separates the plane into three components}. If the Jordan Curve theorem was false, this could very naturally be the case. It is ridiculously easy to fall into the trap of assuming this obvious statement is true.

Once properly stated a theorem of this form naturally has a proof (intermediate value theorem, extreme value theorem, uniform convergence theorems, etc) -- the Jordan Curve theorem is definitely not easy to prove even when stated. It will therefore suffice for us to prove a small step in the entire proof, since it will give us more than enough insight into the problems of the theorem.
%
\begin{theorem}[The Jordan Separation Theorem]
    The complement of a closed curve in $S^2$ is not connected.
\end{theorem}
%
To prove that a closed curve splits the plane into two pieces, we must first show that it splits the plane into any number. The sphere, as the one point compactification of the plane, naturally connected to the plane in a nice enough way that the theorem is proved in tandem for the two spaces. Of course, to show this, we must first consider some lemmas.
%
\begin{lemma}
    Let $C$ be a compact subspace of $S^2$, Take some point not in $C$, and remove, it, unravelling $S^2$ into the plane in the process as a homeomorpism $f:S^2 - \{b\} \to \mathbf{R}^2$. If $U$ is a component of $S^2 - \{b\}$, then $f(U)$ is a component of $\mathbf{R}^2$, and is unbounded if and only if it contains $b$. If $S^2 - C$ has $n$ components, then so does $f(S^2 - C)$.
\end{lemma}
%
This is a technical lemma, which does not involve many interesting ideas. Perhaps something of interest is the use of the one point compactification trick established in the midterm. If a shape in $\mathbf{R}^2$ is unbounded, $b$ is a limit point of that shape when $\mathbf{R}^2$ is compactified.
%
\begin{lemma}
    If $f:A \to S^2 - a - b$ is continuous, and $a$ and $b$ lie in the same component as $S^2 - f(A)$, then $f$ is nullhomotopic.
\end{lemma}
%
To obtain the last lemma, we converted $\mathbf{S}^2$ to $\mathbf{R}^2$, and then back again. In the proof of this lemma, the roles are reversed. The elegance of Alexandroff compactification is brought into the forefront here. The fact that any two points are separated, but connected by a path, allows us to slice the sphere enough that we can peel the function off the sphere.

This is all we need to establish the theorem in full. We do this, like many difficult proofs, by contradiction. Suppose, if $C$ is a closed curve, that $X = S^2 - C$ is connected. First, we split $C$ into two lines $A$ and $B$ that connect together at two points. Consider $U = S^2 - A$ and $V = S^2 - B$. Then $U \cap V = X$ is the shape we want to consider. When we map loops from $\pi(U,x_0)$ to $\pi(X, x_0)$ by the inclusion map, we may collapse them by the previous lemma, since these loops to do pass through the connected segment above. Therefore all these loops are trivial. If we take the space $Y = S^2 - a - b$, where $a$ and $b$ are the two points shared by $A$ and $B$, we see that by Van Kampen's theorem that the space is trivial. But this cannot be so, since taking one point makes $Y$ homeomorphic to the plane, and another jab makes the plane a punctured plane, whose fundamental group is the integers. The argument is complete. Q.E.D.

It cannot be dismissed how incredibly useful the techniques of algebraic topology are in this proof. Nevertheless, the use of such tools obscures the reasoning behind the proof. Beside the obvious intuition for why the theorem is true, the above proof gives us no further insight. Perhaps some depth can be seen in the fact that we need not assume that $A$ and $B$ are lines, but only connected subsets of the plane intersecting at single points, but this can hardly be seen as too much of an advantage. Many more theorem can be proven using the techniques, but similarily, they offer no insight into the situtation as a whole aside from establishing truth -- they are sledgehammers destroying the door, rather than lockpicks enabling us to cleverly discover the intricacies by which the problem can be deconstructed.

Perhaps the ease of use of algebraic techniques in comparison to geometric techniques relates to the higher standards of rigour that we have encumbered ourselves with since the Jordan Curve theorem was postulated. As the great intuitionist Poincare once said, mathematics has been `arithmetized'. It is obvious from this point of view why algebra is a more fluid aid in the rigorous mathematics we now encounter. Maybe, if we can come up with a more precise method of geometrization than the loose handwaving of pre 20th century mathematics, we will discover much more intuitive reasonings for statements such as Camille's famous theorem. For now, the arithmetized algebraic topology will have to suffice...

\newpage

\begin{thebibliography}{10}
    \bibitem{main} James R. Munkres,
    \emph{Topology}
    \bibitem{graph} Richard J. Trudeau,
    \emph{Introduction to Graph Theory}
    \bibitem{basic} Fred H. Croom,
    \emph{Basic Concepts of Algebraic Topology}
    \bibitem{poin} Henri Poincare,
    \emph{Intution and Logic in Mathematics}
\end{thebibliography}

\end{document}