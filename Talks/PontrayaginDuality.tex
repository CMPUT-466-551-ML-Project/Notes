\documentclass{article}

\usepackage{amsmath}
\usepackage{amssymb}
\usepackage{amsthm}

\theoremstyle{plain}
\newtheorem{theorem}{Theorem}
\newtheorem{lemma}[theorem]{Lemma}
\newtheorem{corollary}[theorem]{Corollary}
\newtheorem{prop}[theorem]{Proposition}

%\theoremstyle{remark}
\newtheorem*{example}{Example}
\newtheorem*{proof*}{Proof}

\theoremstyle{definition}
\newtheorem*{defi}{Definition}
\newenvironment{definition}
    {\begin{samepage}\begin{framed}\begin{defi}}
    {\end{defi}\end{framed}\end{samepage}}

\title{Pontryagin Duality}
\author{Jacob Denson}

\begin{document}

\maketitle

If $G$ is a locally compact abelian group, then its character group $\Gamma(G)$ is also a locally compact abelian group. One can show that
%
\[ \Gamma(\mathbf{Z}_n) \cong \mathbf{Z}_n\ \ \ \ \ \Gamma(\mathbf{T}) \cong \mathbf{Z}\ \ \ \ \ \Gamma(\mathbf{Z}) \cong \mathbf{T}\ \ \ \ \ \Gamma(\mathbf{R}) \cong \mathbf{R} \]
%
In each of these groups, we have some sort of duality transform between the structure of $G$, and the structure of $\Gamma(G)$. Information about $G$ gives us information about $\Gamma(G)$, and vice versa.  That is, we have the Fourier transforms
%
\[ f(m) = \sum_{k = 1}^n \widehat{f}(k) e^{2 \pi i k m/n}\ \ \ \ \ f(z) = \sum_{k \in \mathbf{Z}} \widehat{f}(k) z^k\ \ \ \ \ f(t) = \int_{x \in \mathbf{R}} \widehat{f}(t) e^{2 \pi i t} \]
%
We would hope that we have an analogous inversion theorem for all locally compact groups. Unfortunately, things aren't so simple in the nonabelian case, but for abelian groups the theory is still pleasant. The problem here is that the characters of the group do not contain enough non-commutative information, for any such character $\chi: G \to \mathbf{T}$ must factor through the commutator of the group, and it is entirely possible for $[G,G]$ to be equal to $G$, when $G$ is nonabelian. In this case there are no non-trivial characters, and the Fourier transform is trivial, giving us no information about the space. The abelian principle here is that of Pontryagin duality, which expresses itself in different ways, depending on how you look at the algebraic structure of the space.
%
\begin{itemize}
    \item If we define the dual map $\widehat{x} \in \Gamma^2(G)$ defined by $\widehat{x}(\phi) = \phi(x)$ (this is entirely analogous to the double dual map from $G$ to $\Gamma^2(G)$).
    \item The Fourier transform (Stieltjes and non-Stieltjes) is injective.
\end{itemize}

The second manifestation of duality follows from the first.

\begin{lemma}
    If $\mu \in M(\Gamma(G))$ satisfies $\int_{\Gamma(G)} \phi(x) d\mu(\phi) = 0$ for any $x$, then $\mu = 0$.
\end{lemma}
\begin{proof}
    If $f \in L^1(G)$ is given, then
    %
    \[ \int_{\Gamma(G)} \widehat{f}(\phi) d\mu(\phi) = \int_G f(x) \int_{\Gamma(G)} \frac{d\mu(\phi)}{\phi(x)} dx = \int 0 = 0 \]
    %
    so $\mu$ annihilates all of $A(G)$, which is dense in $L^1(\Gamma(G))$, so $\mu$ must annihilate all functions.
\end{proof}

If the map $x \mapsto \widehat{x}$ is an isomorphism, then in the above theorem, we can let $G = \Gamma(G)$, $x = \phi$, $\phi = \widehat{x}$, and $\mu \in M(G)$, to find that if $\widehat{\mu} = 0$, then $\mu = 0$. Thus the Gelfand transform of $L^1(G)$ is injective, so $L^1(G)$ is semisimple. One can show that $L^1(G)$ is semisimple even if $G$ is nonabelian, but the proof is much more difficult. This also shows that $M(G)$ is semisimple, since the Fourier-Stieltjes transform is essentially a simplification of the Gelfand transform. Indeed, we have verified the characters on $G$ form a system of characters for $M(G)$ that separate points.

So let's begin the dirty work of proving the Pontryagin theorem, which surprisingly isn't that difficult to reason about. Fix a compact $C \in \Gamma(G)$ and $r > 0$, and define
%
\[ V = \{ x \in G: |1 - \phi(x)| < r, \phi \in C \}\ \ \ \ \ W = \{ y \in \Gamma^2(G) : |1 - y(\phi)| < r, \phi \in C \} \]
%
By the topological construction of the dual space, sitting inside the weak $*$ compact unit sphere of $L^1(G)^*$, we obtain a neighbourhood basis of $G$ and $\Gamma(G)$ as $C$ ranges over all compact sets. Then we see that $V$ maps into $W$, and covers all points in $W$ which are in the image of the dual map. This shows the dual map is open onto it's range, and hence a homeomorphism.

Now we show that the image of $G$ is closed in $\Gamma^2(G)$, which follows from the general theorem that a locally compact subgroup of a locally compact group is closed. This doesn't seem particularly intuitive, until we note that local compactness over groups is essentially a topological generalization of completeness, and swapping locally compact for complete is certainly obvious to those experience in metric space theory (this also explains somewhat why locally compact groups are tractable, for they behave analogously to complete metric spaces). Indeed, if two metric spaces are locally compact, they are certainly complete, for the Bolzano Weirstra\ss\ holds. The problem with generalizing the closure condition for local compactness is that topologically, we cannot infer the topology of the extension space in terms of the base space (unlike in metric spaces, where we can identify points that once existed by Cauchy sequences in the extracted space). However, for subgroups of a locally compact space, we can identify such boundary values, since we can identify points which were never in the group. Unfortunately the actual proof is more austere.

Certainly proving that closed subgroups are locally compact is easy, but the other is not so obvious to prove formally. If $H \subset G$ is a relation between locally compact groups, fix $x \in \overline{H}$, and let $x = \lim x_\alpha$. Fix a neighborhood $U$ of the origin in $G$ such that $U \cap H$ is precompact in $H$, whose closure is also therefore closed in $G$. Since $x^{-1} \in \overline{H}$, $Vx^{-1}$ contains some $y \in H$. Since $x_\alpha$ is eventually in $xV$, $y x_\alpha$ is eventually in $Vx^{-1}xV = V^2 \subset U$. Also $yx_\alpha \in H$ and $yx_\alpha \to yx$ and so $yx \in H$. But then $x$ is also in $H$.

If we could conclude the image of $G$ was dense in $\Gamma^2(G)$, the Pontryagin duality would be proved in full. Provided the image of $G$ was dense in $\Gamma^2(G)$, there would be a function $F \in A(\Gamma^2(G))$ which is zero on the image of $G$ but not zero everywhere. For some $f \in L^1(\Gamma(G))$, for $y \in \Gamma^2(G)$,
%
\[ F(y) = \int \frac{f(\phi)}{\phi(x)} \]
%
Since $F(\hat{x}) = 0$, we find that the Fourier coefficients of $f$ vanish, so by the uniqueness theorem we find $f = 0$, a contradiction, so that $G$ is dense in $\Gamma^2(G)$, and since $G$ is closed we actually have equality here.

The Pontryagin duality gives us an easy generalization of the inversion theorem to $M(G)$. Attempting to imitate the inversion theorem for functions, if $\widehat{\mu} \in L^1(\Gamma(G))$, we consider the function
%
\[ g: x \mapsto \int \widehat{\mu}(\phi) \phi(x) d\phi = \int \int \frac{d\mu(y)}{\phi(x^{-1}y)} dy d\phi \]
%
where $\phi$ is particularly chosen so that the inversion theorem holds. We assume the function obtained is in $L^1(G)$ (If there's any justice in the world, this should be true, and you can find the proof in Rudin), and the $\hat{g} = \hat{\mu}$. Applying what we know, we conclude that $g = \mu$ in $L^1(G)$. This is really not really much of a `generalization' of the inversion theorem, for it says the inversion theorem only holds in $L^1(G)$.

An interesting application of Pontrayagin duality allows us to see a duality between subgroups and quotient groups of a LCA group $G$. If $H$ is a closed subgroup of $G$, let $H^\perp$ be the set of characters whose kernel contains $H$. This is clearly a closed group of characters in $\Gamma(G)$.

\begin{theorem}
    $(H^\perp)^\perp = H$.
\end{theorem}
\begin{proof}
    Certainly $H \subset (H^\perp)^\perp$. Conversely, consider the quotient map $\pi: G \to G/H$. If $x \not \in H$, we may apply the Gelfand Raikov theorem (and the fact that abelian irreducible representations are all one dimensional) to conclude there is $\chi \in \Gamma(G/H)$ such that $\chi(\pi(x)) \neq 1$. But then $\chi \circ \pi \in H^\perp$ and so $x \not \in (H^\perp)^\perp$.
\end{proof}

\begin{theorem}
    If $H$ is a closed subgroup of $G$, and let $\Phi: \Gamma(G/H) \to H^\perp$ by $\Phi(\chi) = \chi \circ \pi$, and $\Psi:\Gamma(G)/H^\perp \to H$ be defined by $\Psi(\chi H^\perp) = \chi|_H$, then $\Phi$ and $\Psi$ are topological isomorphisms.
\end{theorem}
\begin{proof}
    It is easy to see that $\Phi$ and $\Psi$ are algebraic isomorphisms. That they are homeomorphisms is obtained by taking uniform convergence on compact subsets to be the notion of convergence in the dual space. If $\chi_\alpha \to \chi$ in $\Gamma(G/H)$, and $K \subset G$ is compact, then $\chi_\alpha \to \chi$ uniformly on $\pi(K)$, so $\chi_\alpha \circ \pi \to \chi \circ \pi$ in $\Gamma(G)$, and so the inverse of $\Phi$ is continuous. Conversely, if $\chi_\alpha \to \chi$ and $F \subset G/H$ is compact, there is a compact $K$ such that $\pi(K) = F$, and $\chi_\alpha \to \chi$ uniformly on $K$, hence $\chi_\alpha \circ \pi \to \chi \circ \pi$ on $F$, and so $\chi_\alpha \to \chi$ in $\Gamma(G/H)$. Thus $\Phi$ is a homeomorphism.

    If we replace $G$ with $\Gamma(G)$, and $H$ with $H^\perp$, we have a homeomorphism
    %
    \[ \Gamma(\Gamma(G)/H^\perp) \cong (H^\perp)^\perp \cong H \]
    %
    If $x \in H$, the correspondence gives us $\eta \in \Gamma(\Gamma(G)/H^\perp)$ for which $\eta(\chi H^\perp) = \chi(x)$, so $\Psi$ is obtained from $\Psi$ and the dual map to obtain a homeomorphism.
\end{proof}

We obtain a corollary to this proof which is essentially a `Hahn-Banach extension' type theorem for abelian groups.

\begin{theorem}
    Any character on a closed subgroup of an abelian group can be extended to the entire group.
\end{theorem}

Finally, we discuss how the Poisson summation formula generalizes via Pontryagin duality. It turns out this statement is really about the quotient of a group by, so we need to talk about the relation of the Haar measure on $G$ and $G/H$, where $H$ is a closed subgroup of a locally compact group $G$. It turns out a very natural choice is one such that for any integrable function $f$ defined on $G$,
%
\[ \int_G f(x) dx = \int_{G/H} \left( \int_H f(xy) dy \right) d(xH) \]

\begin{theorem}
    Suppose $H$ is a closed subgroup of $G$.If we normalize the measures on $H$ and $H^\perp$ appropriately, then for any $f \in C_c(G)$,
    %
    \[ \int_H f(xy) dy = \int_{H^\perp} \hat{f}(\phi) \phi(x) d\phi \]
    %
    The regular Poisson summation formula is obtained by taking $H = \mathbf{Z}$, $G = \mathbf{R}$, in which case $H^\perp$ is the set of exponentials with integer power.
\end{theorem}
\begin{proof}
    If $\phi \in H^\perp$ then $\phi(xy) = \phi(x)$ for $y \in H$, so if we define $F \in C_c(G/H)$ by the formula
    %
    \[ F(xH) = \int_H f(xy) dy \]
    %
    Then
    %
    \[ \hat{F}(\phi) = \int_{G/H} \frac{F(xH)}{\phi(xH)} d(xH) = \int_{G/H} \int_H \frac{f(xy)}{\phi(xy)} dy d(xH) = \int_G \frac{f(x)}{\phi(x)} = \hat{f}(\phi) \]
    %
    Where we are treating characters on $G/H$ as elements of $H^\perp$ due to the identification above. Applying the inversion theorem, we obtain the theorem above.
\end{proof}

\begin{thebibliography}{9}

\bibitem{folland}
    Gerald Folland,
    \emph{A Course in Abstract Harmonic Analysis},
    CRC Press,
    2015.

\bibitem{rudin}
    Walter Rudin,
    \emph{Fourier Analysis on Groups},
    Wiley,
    1990.

\end{thebibliography}

\end{document}