\documentclass[12pt, dvipsnames]{report}

\usepackage{amsmath}
\usepackage{algorithm}
%\usepackage{algorithmic}
\usepackage[noend]{algpseudocode}

\usepackage{amsmath}
\usepackage{amssymb}
\usepackage{amsthm}
\usepackage{amsopn}

\usepackage{kpfonts}

\usepackage{graphicx}

% Probably don't need this on notes anymore
%\usepackage{kbordermatrix}

% Standard tool for drawing diagrams.
\usepackage{tikz}
\usepackage{tkz-berge}
\usepackage{tikz-cd}
\usepackage{tkz-graph}

\usepackage{comment}

%
\usepackage{multicol}

%
\usepackage{framed}

%
\usepackage{mathtools}

%
\usepackage{float}

%
\usepackage{subfig}

%
\usepackage{wrapfig}

%
\let\savewideparen\wideparen
\let\wideparen\relax
\usepackage{mathabx}
\let\wideparen\savewideparen

% Used for generating `enlightening quotes'
\usepackage{epigraph}

% Forget what this is used for :P
\usepackage[utf8]{inputenc}

% Used for generating quotes.
\usepackage{csquotes}

% Allows what to generate links inside
% generated pdf files
\usepackage{hyperref}

% Allows one to customize theorem
% environments in mathematical proofs.
\usepackage{thmtools}

% Gives access to a proof
\usepackage{lplfitch}

% I forget what this is for.
\usepackage{accents}

% A package for drawing simple trees,
% as a substitute for unnesacary TIKZ code
\usepackage{qtree}

% Enables sequent calculus proofs
\usepackage{ebproof}

% For braket notation
\usepackage{braket}

% To change line spacing when using mathematical notations which require some height!
\usepackage{setspace}

%\usepackage[dvipsnames]{xcolor}

\usepackage{float}

% For block commenting
\usepackage{comment}




\setlength\epigraphwidth{8cm}

\usetikzlibrary{arrows, petri, topaths, decorations.markings}

% So you can do calculations in coordinate specifications
\usetikzlibrary{calc}
\usetikzlibrary{angles}

\theoremstyle{plain}
\newtheorem{theorem}{Theorem}[chapter]
\newtheorem{axiom}{Axiom}
\newtheorem{lemma}[theorem]{Lemma}
\newtheorem{corollary}[theorem]{Corollary}
\newtheorem{prop}[theorem]{Proposition}
\newtheorem{exercise}{Exercise}[chapter]
\newtheorem{fact}{Fact}[chapter]

\newtheorem*{example}{Example}
\newtheorem*{proof*}{Proof}

\theoremstyle{remark}
\newtheorem*{exposition}{Exposition}
\newtheorem*{remark}{Remark}
\newtheorem*{remarks}{Remarks}

\theoremstyle{definition}
\newtheorem*{defi}{Definition}

\usepackage{hyperref}
\hypersetup{
    colorlinks = true,
    linkcolor = black,
}

\usepackage{textgreek}

\makeatletter
\renewcommand*\env@matrix[1][*\c@MaxMatrixCols c]{%
  \hskip -\arraycolsep
  \let\@ifnextchar\new@ifnextchar
  \array{#1}}
\makeatother

\renewcommand*\contentsname{\hfill Table Of Contents \hfill}

\newcommand{\optionalsection}[1]{\section[* #1]{(Important) #1}}
\newcommand{\deriv}[3]{\left. \frac{\partial #1}{\partial #2} \right|_{#3}} % partial derivative involving numerator and denominator.
\newcommand{\lcm}{\operatorname{lcm}}
\newcommand{\im}{\operatorname{im}}
\newcommand{\bint}{\mathbf{Z}}
\newcommand{\gen}[1]{\langle #1 \rangle}

\newcommand{\End}{\operatorname{End}}
\newcommand{\Mor}{\operatorname{Mor}}
\newcommand{\Id}{\operatorname{id}}
\newcommand{\visspace}{\text{\textvisiblespace}}
\newcommand{\Gal}{\text{Gal}}

\newcommand{\xor}{\oplus}
\newcommand{\ft}{\wedge}
\newcommand{\ift}{\vee}

\newcommand{\prob}{\mathbf{P}}
\newcommand{\expect}{\mathbf{E}}
\DeclareMathOperator{\Var}{\mathbf{V}}
\newcommand{\Ber}{\text{Ber}}
\newcommand{\Bin}{\text{Bin}}

%\newcommand{\widecheck}[1]{{#1}^{\ft}}

\DeclareMathOperator{\diam}{\text{diam}}

\DeclareMathOperator{\QQ}{\mathbf{Q}}
\DeclareMathOperator{\ZZ}{\mathbf{Z}}
\DeclareMathOperator{\RR}{\mathbf{R}}
\DeclareMathOperator{\HH}{\mathbf{H}}
\DeclareMathOperator{\CC}{\mathbf{C}}
\DeclareMathOperator{\AB}{\mathbf{A}}
\DeclareMathOperator{\PP}{\mathbf{P}}
\DeclareMathOperator{\MM}{\mathbf{M}}
\DeclareMathOperator{\VV}{\mathbf{V}}
\DeclareMathOperator{\TT}{\mathbf{T}}
\DeclareMathOperator{\LL}{\mathcal{L}}
\DeclareMathOperator{\EE}{\mathbf{E}}
\DeclareMathOperator{\NN}{\mathbf{N}}
\DeclareMathOperator{\DQ}{\mathcal{Q}}
\DeclareMathOperator{\IA}{\mathfrak{a}}
\DeclareMathOperator{\IB}{\mathfrak{b}}
\DeclareMathOperator{\IC}{\mathfrak{c}}
\DeclareMathOperator{\IP}{\mathfrak{p}}
\DeclareMathOperator{\IQ}{\mathfrak{q}}
\DeclareMathOperator{\IM}{\mathfrak{m}}
\DeclareMathOperator{\IN}{\mathfrak{n}}
\DeclareMathOperator{\IK}{\mathfrak{k}}
\DeclareMathOperator{\ord}{\text{ord}}
\DeclareMathOperator{\Ker}{\textsf{Ker}}
\DeclareMathOperator{\Coker}{\textsf{Coker}}
\DeclareMathOperator{\emphcoker}{\emph{coker}}
\DeclareMathOperator{\pp}{\partial}
\DeclareMathOperator{\tr}{\text{tr}}

\DeclareMathOperator{\supp}{\text{supp}}

\DeclareMathOperator{\codim}{\text{codim}}

\DeclareMathOperator{\minkdim}{\dim_{\mathbf{M}}}
\DeclareMathOperator{\hausdim}{\dim_{\mathbf{H}}}
\DeclareMathOperator{\lowminkdim}{\underline{\dim}_{\mathbf{M}}}
\DeclareMathOperator{\upminkdim}{\overline{\dim}_{\mathbf{M}}}
\DeclareMathOperator{\lhdim}{\underline{\dim}_{\mathbf{M}}}
\DeclareMathOperator{\lmbdim}{\underline{\dim}_{\mathbf{MB}}}
\DeclareMathOperator{\packdim}{\text{dim}_{\mathbf{P}}}
\DeclareMathOperator{\fordim}{\dim_{\mathbf{F}}}

\DeclareMathOperator*{\argmax}{arg\,max}
\DeclareMathOperator*{\argmin}{arg\,min}

\DeclareMathOperator{\ssm}{\smallsetminus}

\title{Ordinary Differential Equations}
\author{Jacob Denson}

\begin{document}

\pagenumbering{gobble}

\maketitle

\tableofcontents

\pagenumbering{arabic}

\chapter{Vector Fields on $\mathbf{RP}^n$}

On $\mathbf{R}^n$, local flows related to a smooth vector field $v$ may fail to extend to global flows because solutions approach $\infty$ in finite time. However, often we may be able to embed $\mathbf{R}^n$ in a compact manifold $K$, and extend $v$ to a smooth vector field on $K$. Since $K$ is compact, all local flows extend to global flows, and thus we can consider a global flow on $\mathbf{R}^n$ which `passes through $\infty$' in finite time.

For instance, recall that the space $\mathbf{RP}^n$ is the compact quotient space of $\mathbf{R}^{n+1} - \{ 0 \}$ by the group action of $\mathbf{R}^\times$ by scaling, so that $x$ is identified with $\lambda x$ for any $\lambda \neq 0$. The quotient structure gives it a natural topological structure, which can also be identified with the topology which makes the projection maps on each of the coordinate systems
%
\[ x_i: [x] \mapsto (x^1/x^i, \dots, \widehat{x^i/x^i}, \dots, x^{n+1}/x^i) \]
%
defined on $U_i = \{ [x] : x_i \neq 0 \}$, homeomorphisms. It is a smooth manifold if we consider the $x_i$ as diffeomorphisms.

\begin{example}
    The classic example of a vector field which cannot be extended to a global flow is $v(x) = x^2$ on $\mathbf{R}$, which has a flow
    %
    \[ \varphi_t(x) = \frac{x}{1 - xt} \]
    %
    Which has a singularity at $t = x^{-1}$. Note, however, that if we write this map in projective coordinates, then we find $\varphi_t[x:y] = [x:y - tx]$. In this formulation, it is easy to see that each map $\varphi_t$ can be extended uniquely to a smooth map from $\mathbf{RP}^1$ to $\mathbf{RP}^1$, and the group equation still holds.
    %
    \begin{align*}
        \varphi_{t+s}[x:y] &= [x:y - x(t+s)] = [x:(y - sx) - tx] = \varphi_t(\varphi_s[x:y])
    \end{align*}
    %
    An alternate way to see this is to let $y = 1/x$ denote the inverse coordinate system on projective space. We then calculate that for $y \neq 0, \infty$, that
    %
    \[ v(y) = x^2 \partial_x(y) = - x^2 y^2 = - \partial_y \]
    %
    and $v$ can be uniquely extended to a smooth vector field on $\mathbf{RP}^1$ by defining $v(\infty) = \partial_y$, and therefore generates a global flow on $\mathbf{RP}^1$ because $\mathbf{RP}^1$ is compact. This technique is not general, however. If we consider the vector field $v(x) = x^3 \partial_x$, then we find that $v(y) = -y^{-1} \partial_y$, which cannot be extended to a smooth vector field at $y = 0$. This is because solutions approach infinity `too fast' -- we find the flows take the form
    %
    \[ \varphi_t(y) = \sqrt{y^2 - 2t} \]
    %
    And these solutions approach $y = 0$ with infinite slope.
\end{example}

Sometimes the geometry of projective space provides an enlightening viewpoint on a particular differential equation.

\begin{example}
    Consider the differential equation $\ddot{u} + \alpha u = 0$, as $\alpha$ ranges over $\mathbf{R}$. This corresponds to the two dimensional first order system specified by the vector field $v(u,w) = (w, -\alpha u)$. This means that on the integral curves defined by this vector field,
    %
    \[ - \alpha u du = w dw \]
    %
    so the integral curves lie on the level curves to $w^2 + \alpha u^2$. For $\alpha > 0$, this value is always positive, and defines an ellipse. Since $v$ does not vanish on any ellipse of a positive radius, we see these ellipses must describe the integral curves. For $\alpha < 0$, the level curves of $w^2 + \alpha u^2$ describe hyperbolas not passing through the origin, so these hyperbolas are the integral curves. For $\alpha = 0$, the integral curves are easily seen to be the lines parallel to the $x$ axis. Switching to the coordinates $x = u/w$, $y = 1/w$, we find that for $y \neq 0$,
    %
    \begin{align*}
        v(x,y) &= (w \partial_u(x) - \alpha u \partial_w(x), w \partial_u(y) - \alpha u \partial_w(y))\\
        &= (1 + \alpha u^2/w^2, \alpha u/w^2) = (1 + \alpha x^2, \alpha xy)
    \end{align*}
    %
    This function can be extended to a smooth vector field on the whole of $\mathbf{RP}^2$ by defining $v(x,0) = (1 + \alpha x^2,0)$.
\end{example}

\begin{thebibliography}{9}

%\bibitem{evans}
%Lawrence C. Evans
%\textit{Partial Differential Equations}

\end{thebibliography}

\end{document}