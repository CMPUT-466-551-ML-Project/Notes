\documentclass[dvipsnames,letterpaper,12pt]{article}

\usepackage[margin = 1.0in]{geometry}
\usepackage{amsmath,amssymb,graphicx,mathabx,accents}
\usepackage{enumerate,mdwlist}

\usepackage{tikz}

%\setlist[enumerate]{label*={\normalfont(\Alph*)},ref=(\Alph*)}

\numberwithin{equation}{section}

\usepackage{amsthm}

\usepackage{hyperref}

\usepackage{verbatim}

\usepackage{nag}

\DeclareMathOperator{\minkdim}{\dim_{\mathbb{M}}}
\DeclareMathOperator{\hausdim}{\dim_{\mathbb{H}}}
\DeclareMathOperator{\lowminkdim}{\underline{\dim}_{\mathbb{M}}}
\DeclareMathOperator{\upminkdim}{\overline{\dim}_{\mathbb{M}}}
\DeclareMathOperator{\fordim}{\dim_{\mathbb{F}}}

\DeclareMathOperator{\lhdim}{\underline{\dim}_{\mathbb{M}}}
\DeclareMathOperator{\lmbdim}{\underline{\dim}_{\mathbb{MB}}}

\DeclareMathOperator{\RR}{\mathbb{R}}
\DeclareMathOperator{\ZZ}{\mathbb{Z}}
\DeclareMathOperator{\QQ}{\mathbb{Q}}
\DeclareMathOperator{\TT}{\mathbb{T}}
\DeclareMathOperator{\CC}{\mathbb{C}}

\DeclareMathOperator{\B}{\mathcal{B}}

\newtheorem{theorem}{Theorem}
%\newtheorem{lemma}{Lemma}
%\newtheorem{corollary}{Corollary}
\newtheorem{lemma}[theorem]{Lemma}
\newtheorem{corollary}[theorem]{Corollary}
%\newtheorem{prop}[theorem]{Proposition}
\newtheorem{remark}[theorem]{Remark}
\newtheorem{remarks}[theorem]{Remarks}
%\newtheorem*{concludingremarks}{Concluding Remarks}
\numberwithin{theorem}{section}

\DeclareMathOperator{\EE}{\mathbb{E}}
\DeclareMathOperator{\PP}{\mathbb{P}}

\DeclareMathOperator{\DQ}{\mathcal{Q}}
\DeclareMathOperator{\DR}{\mathcal{R}}

\newcommand{\psitwo}[1]{\| {#1} \|_{\psi_2(L)}}
\newcommand{\TV}[2]{\| {#1} \|_{\text{TV}({#2})}}








\title{Large Salem Sets Avoiding Nonlinear Configurations 2}
\author{Jacob Denson\footnote{University of Madison Wisconsin, Madison, WI, jcdenson@wisc.edu}}

\begin{document}

\maketitle

Our goal is to improve the result of the last paper. Most of the work has been done in that setting in the precise range needed, so we can skip all but the final analysis of an expectation in this paper. Let us reintroduce notation:
%
\begin{itemize}
    \item We consider a family of axis-aligned cubes $Q_1,\dots,Q_n \subset [0,1]^d$, each with common sidelength $s > 0$, such that $d(Q_i,Q_j) \geq 10s$ for $i \neq j$.

    \item We consider a family of density functions $\psi_1,\dots,\psi_n \in C^\infty(\TT^d)$ supported on $2Q_i$. We fix a large integer $M > 0$, and consider a family of independent random variables
    %
    \[ \{ X_i(k): 1 \leq i \leq n, 1 \leq k \leq M \} \]
    %
    where $X_i(k)$ is chosen with respect to the probability density function $\psi_i$.

    \item Let $r = M^{-1/\lambda}$, and consider the random set $I$ of all indices $k_n \in \{ 1, \dots, M \}$ such that there exists indices $k_1,\dots,k_{n-1} \in \{ 1, \dots, M \}$ such that
    %
    \[ |X_n(k_n) - f(X_1(k_1),\dots,X_{n-1}(k_{n-1}))| \leq r. \]

    \item Finally, define
    %
    \[ H(\xi) = \sum_{k \in I} e^{2 \pi i \xi \cdot X_n(k)}. \]
\end{itemize}
%
Our goal is to show that for any $\delta > 0$, there is $M_0 > 0$ such that for $M \geq M_0$, and any $\xi \in \ZZ^d$,
%
\[ \EE[H(\xi)] \leq \delta M |\xi|^{-\lambda/2} + O_\delta(M^{1/2}). \]
%
In the previous paper, we were able to establish this result with $\lambda = d/(n-3/4)$, but we expect that one can establish this result with $\lambda = d/(n-1)$. This was established in the case $n = 2$, where $\lambda = d$. So the case of interest is where $n \geq 3$.

\section{Review of Last Paper Proof}

We start by writing
%
\[ \EE[H(\xi)] = M \int \psi_n(x_n) p_M(x_n) e^{2 \pi i \xi \cdot x_n}\; dx_n, \]
%
where $p_M(x_n)$ denotes the probability that there exists $k_1,\dots,k_{n-1} \in \{ 1, \dots, M \}$ such that
%
\[ |x_n - f(X_1(k_1),\dots, X_{n-1}(k_{n-1}))| \leq r. \]
%
In that paper, it was shown using an inclusion exclusion argument that if $E_{x_n} = f^{-1}(x_n)$, then
%
\[ P_M(x_n) = M^{n-1} \int_{E_{x_n}} \psi_1(x_1) \cdots \psi_{n-1}(x_{n-1})\; dx_1 \dots\; dx_{n-1} + O(M^{2(n-1) - 2d/\lambda}). \]
%
The error here is $O(M^{-1/2})$ provided that $\lambda \leq d / (n - 3/4)$. Thus in this situation if we write $\psi = \psi_1 \otimes \dots \otimes \psi_n$ then
%
\[ \EE[H(\xi)] = \left( M \int \int_{E_{x_n}} \psi(x) e^{2 \pi i \xi \cdot x_n} \right) + O(M^{1/2}). \]
%
The integral here can be converted using the coarea formula into an oscillatory integral, which yields the $\delta |\xi|^{-\lambda/2}$ term required.

\section{Does Smoothness Help?}

The function $\psi_n$ is smooth. Thus if we could show that
%
\[ \| p_M \|_{W^{s/2,1}(\TT^d)} \lesssim M^{-1/2} \]
%
then the fact that $\EE[H(\xi)]$ is $M$ times the  Fourier transform of $\psi_n \cdot P_M$ would give the required result. But heuristically, it doesn't seem like the function should be that smooth, though I should redo the calculation just to be sure.

\section{Counting Solutions}

The reason the argument in the first section worked was that with high probability, the number of solutions to
%
\[ |x_n - f(X_1(k_1), \dots, X_{n-1}(k_{n-1}))| \leq r \]
%
was equal to $M^{n-1}$ times the surface area of the hyperplane $E_{x_n}$. Could we do something more robust to get around the inclusion-exclusion argument?

One potential idea, instead of drawing $X_1,\dots,X_{n-1}$ from a continuous distribution, is to draw the points from a discrete distribution, e.g. uniformly distributed on some rational points with a fixed denominator in the cube $Q_i$? If $f$ is an integer valued polynomial, then perhaps the circle method might then be of some use since then we might get much better bounds on the number of solutions to the equation? We could also use probabilistic decoupling to remove the cubes $\{ Q_i \}$ from the equation if needed, provided the concentration argument goes through.

\begin{comment}

\section{Techniques for Avoiding Hyperplanes}

Let $y = f(x)$ be a curve in $\TT^2$ defining a curve $S$, where $f$ is an analytic function (except perhaps at finitely many points?). Given $\varepsilon > 0$, we want to determine the differentiability of the map
%
\[ A(x) = H^1(S_\varepsilon \cap \{ x \times \TT \}). \]
%
We wish to show $A$ is a smooth function. The tangent to $S$ at a point $(x,f(x))$ is given by $(1,f'(x))$, and so the unit normal vector is
%
\[ N(x) = \frac{(f'(x),-1)}{\sqrt{1 + f'(x)^2}}. \]
%
Suppose that $x_0 \in \TT$ is fixed, and let $x \in \TT$ and $|\delta| \leq \varepsilon$ be given such that
%
\[ (x_0,y_0) = (x,f(x)) + \delta N(x) = \left( x + \frac{\delta f'(x)}{\sqrt{1+ f'(x)^2}}, f(x) - \frac{\delta}{\sqrt{1 + f'(x)^2}} \right) \]
%
Thus
%
\[ x_0 = x + \frac{\delta f'(x)}{\sqrt{1 + f'(x)^2}} \]
%
and
%
\[ y_0 = f(x) - \frac{\delta}{\sqrt{1 + f'(x)^2}}. \]
%
Now the first equation tells us that
%
\[ \delta = - (x - x_0) \frac{\sqrt{1 + f'(x)^2}}{f'(x)}. \]
%
Thus if we define
%
\[ g(x,x_0) = \begin{cases} f(x) + \frac{x - x_0}{f'(x)} &: f'(x) \neq 0 \\ BLAH &: BLAH, \end{cases} \]
%
then $y_0 = g(x,x_0)$. Thus we ask ourselves what is the value of
%
\[ A(x_0) = \max \left\{ f(x) + \frac{x - x_0}{f'(x)} : |x - x_0| \leq \frac{\varepsilon |f'(x)|}{\sqrt{1 + f'(x)^2}} \right\}. \]
%
Now $g(x,x_0)$ is a smooth function except where $f'(x) = 0$. In particular, if $f'(x_0) \neq 0$, then the constraint region defining $A(x_0)$ is a finite union of closed intervals. And $f'(x_0) = 0$ only at finitely many points, and if we make $\varepsilon$ small enough we can make $A(x_0) = f(x_0) + \varepsilon$ at these points.

so this causes us no problems since we only care about whether $A$ is differentiable except at finitely many points. To analyze $A(x_0)$ when $f'(x_0) = 0$, we note that a solution that gives the maximum either satisfies
%
\[ f'(x)(f'(x)^2 + 1) + (x - x_0) f''(x) = 0 \]
%
or
%
\[ x - x_0 = \frac{\varepsilon f'(x)}{\sqrt{1 + f'(x)^2}} \]
%
or
%
\[ x - x_0 = \frac{-\varepsilon f'(x)}{\sqrt{1 + f'(x)^2}}. \]
%
If $\varepsilon$ is small enough, then the implicit function theorem implies that the second and third equations have finitely many solutions for each $x_0$, which are locally smoothly parameterized. Since $f'(x_0) \neq 0$, the first equation does not even have any solutions if $\varepsilon \lesssim 1$. Thus we conclude that if $\varepsilon$ is small enough, there exists a function $x(x_0)$ which is smooth, except at finitely many points, such that
%
\[ g(x_0) = f(x) + \frac{x - x_0}{f'(x)}. \]
%
Thus at any $x_0$ where $x$ is smooth, we conclude
%
\[ g'(x_0) = f'(x) \cdot x' + \frac{x' - 1}{f'(x)} - \frac{x - x_0}{f'(x)^2} f''(x) x'. \]

\end{comment}

\bibliographystyle{amsplain}
\bibliography{FourierDimensionNonlinearPatterns}

% c = f(x_2,...,x_n)
% Tangent space equals common null space of df^1,...,df^n, and this null space is d(n-2) dimensional.
% so df^1,..., df^n spans a d dimensional subspace of linear functionals.
% If we add in dx^j_{i_1},...,dx^j_{i_m}, then this spans a dm space of functions and the nullspace is d(n-m-1) dimensional.
% Hopefully the intersection is d(n-m-2) dimensional, so the set
%{ df^1,...,df^n,dx^j_{i_1},..,dx^j_{i_m} }
% should have dimension d(m+1)

% This holds if the map x -> f(x_2,...,x_{i-1},x,x_{i+1},...,x_n) is a diffeomorphism.

% Is there a determinant condition for this? If we remove the m minors corresponds to dx^j_{i_1}, the remaining d(n-1) -> d
% d(n-m-1) -> d

% So if we add in dx_{i_1},...,dx_{i_{n-m}}
% y = (x_1 + x_2 - 2x_3)^2
% df = 2(x_1 + x_2 - 2x_3) (dx_1 + dx_2 - 2dx_3)
% m = 1: should be 2 dimensional CHECK
% m = 2: should be 3 dimensional CHECK

\end{document}
