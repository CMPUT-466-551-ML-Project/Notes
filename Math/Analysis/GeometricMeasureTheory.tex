\documentclass[12pt, dvipsnames]{report}

\usepackage{amsmath}
\usepackage{algorithm}
%\usepackage{algorithmic}
\usepackage[noend]{algpseudocode}

\usepackage{amsmath}
\usepackage{amssymb}
\usepackage{amsthm}
\usepackage{amsopn}

\usepackage{kpfonts}

\usepackage{graphicx}

% Probably don't need this on notes anymore
%\usepackage{kbordermatrix}

% Standard tool for drawing diagrams.
\usepackage{tikz}
\usepackage{tkz-berge}
\usepackage{tikz-cd}
\usepackage{tkz-graph}

\usepackage{comment}

%
\usepackage{multicol}

%
\usepackage{framed}

%
\usepackage{mathtools}

%
\usepackage{float}

%
\usepackage{subfig}

%
\usepackage{wrapfig}

%
\let\savewideparen\wideparen
\let\wideparen\relax
\usepackage{mathabx}
\let\wideparen\savewideparen

% Used for generating `enlightening quotes'
\usepackage{epigraph}

% Forget what this is used for :P
\usepackage[utf8]{inputenc}

% Used for generating quotes.
\usepackage{csquotes}

% Allows what to generate links inside
% generated pdf files
\usepackage{hyperref}

% Allows one to customize theorem
% environments in mathematical proofs.
\usepackage{thmtools}

% Gives access to a proof
\usepackage{lplfitch}

% I forget what this is for.
\usepackage{accents}

% A package for drawing simple trees,
% as a substitute for unnesacary TIKZ code
\usepackage{qtree}

% Enables sequent calculus proofs
\usepackage{ebproof}

% For braket notation
\usepackage{braket}

% To change line spacing when using mathematical notations which require some height!
\usepackage{setspace}

%\usepackage[dvipsnames]{xcolor}

\usepackage{float}

% For block commenting
\usepackage{comment}




\setlength\epigraphwidth{8cm}

\usetikzlibrary{arrows, petri, topaths, decorations.markings}

% So you can do calculations in coordinate specifications
\usetikzlibrary{calc}
\usetikzlibrary{angles}

\theoremstyle{plain}
\newtheorem{theorem}{Theorem}[chapter]
\newtheorem{axiom}{Axiom}
\newtheorem{lemma}[theorem]{Lemma}
\newtheorem{corollary}[theorem]{Corollary}
\newtheorem{prop}[theorem]{Proposition}
\newtheorem{exercise}{Exercise}[chapter]
\newtheorem{fact}{Fact}[chapter]

\newtheorem*{example}{Example}
\newtheorem*{proof*}{Proof}

\theoremstyle{remark}
\newtheorem*{exposition}{Exposition}
\newtheorem*{remark}{Remark}
\newtheorem*{remarks}{Remarks}

\theoremstyle{definition}
\newtheorem*{defi}{Definition}

\usepackage{hyperref}
\hypersetup{
    colorlinks = true,
    linkcolor = black,
}

\usepackage{textgreek}

\makeatletter
\renewcommand*\env@matrix[1][*\c@MaxMatrixCols c]{%
  \hskip -\arraycolsep
  \let\@ifnextchar\new@ifnextchar
  \array{#1}}
\makeatother

\renewcommand*\contentsname{\hfill Table Of Contents \hfill}

\newcommand{\optionalsection}[1]{\section[* #1]{(Important) #1}}
\newcommand{\deriv}[3]{\left. \frac{\partial #1}{\partial #2} \right|_{#3}} % partial derivative involving numerator and denominator.
\newcommand{\lcm}{\operatorname{lcm}}
\newcommand{\im}{\operatorname{im}}
\newcommand{\bint}{\mathbf{Z}}
\newcommand{\gen}[1]{\langle #1 \rangle}

\newcommand{\End}{\operatorname{End}}
\newcommand{\Mor}{\operatorname{Mor}}
\newcommand{\Id}{\operatorname{id}}
\newcommand{\visspace}{\text{\textvisiblespace}}
\newcommand{\Gal}{\text{Gal}}

\newcommand{\xor}{\oplus}
\newcommand{\ft}{\wedge}
\newcommand{\ift}{\vee}

\newcommand{\prob}{\mathbf{P}}
\newcommand{\expect}{\mathbf{E}}
\DeclareMathOperator{\Var}{\mathbf{V}}
\newcommand{\Ber}{\text{Ber}}
\newcommand{\Bin}{\text{Bin}}

%\newcommand{\widecheck}[1]{{#1}^{\ft}}

\DeclareMathOperator{\diam}{\text{diam}}

\DeclareMathOperator{\QQ}{\mathbf{Q}}
\DeclareMathOperator{\ZZ}{\mathbf{Z}}
\DeclareMathOperator{\RR}{\mathbf{R}}
\DeclareMathOperator{\HH}{\mathbf{H}}
\DeclareMathOperator{\CC}{\mathbf{C}}
\DeclareMathOperator{\AB}{\mathbf{A}}
\DeclareMathOperator{\PP}{\mathbf{P}}
\DeclareMathOperator{\MM}{\mathbf{M}}
\DeclareMathOperator{\VV}{\mathbf{V}}
\DeclareMathOperator{\TT}{\mathbf{T}}
\DeclareMathOperator{\LL}{\mathcal{L}}
\DeclareMathOperator{\EE}{\mathbf{E}}
\DeclareMathOperator{\NN}{\mathbf{N}}
\DeclareMathOperator{\DQ}{\mathcal{Q}}
\DeclareMathOperator{\IA}{\mathfrak{a}}
\DeclareMathOperator{\IB}{\mathfrak{b}}
\DeclareMathOperator{\IC}{\mathfrak{c}}
\DeclareMathOperator{\IP}{\mathfrak{p}}
\DeclareMathOperator{\IQ}{\mathfrak{q}}
\DeclareMathOperator{\IM}{\mathfrak{m}}
\DeclareMathOperator{\IN}{\mathfrak{n}}
\DeclareMathOperator{\IK}{\mathfrak{k}}
\DeclareMathOperator{\ord}{\text{ord}}
\DeclareMathOperator{\Ker}{\textsf{Ker}}
\DeclareMathOperator{\Coker}{\textsf{Coker}}
\DeclareMathOperator{\emphcoker}{\emph{coker}}
\DeclareMathOperator{\pp}{\partial}
\DeclareMathOperator{\tr}{\text{tr}}

\DeclareMathOperator{\supp}{\text{supp}}

\DeclareMathOperator{\codim}{\text{codim}}

\DeclareMathOperator{\minkdim}{\dim_{\mathbf{M}}}
\DeclareMathOperator{\hausdim}{\dim_{\mathbf{H}}}
\DeclareMathOperator{\lowminkdim}{\underline{\dim}_{\mathbf{M}}}
\DeclareMathOperator{\upminkdim}{\overline{\dim}_{\mathbf{M}}}
\DeclareMathOperator{\lhdim}{\underline{\dim}_{\mathbf{M}}}
\DeclareMathOperator{\lmbdim}{\underline{\dim}_{\mathbf{MB}}}
\DeclareMathOperator{\packdim}{\text{dim}_{\mathbf{P}}}
\DeclareMathOperator{\fordim}{\dim_{\mathbf{F}}}

\DeclareMathOperator*{\argmax}{arg\,max}
\DeclareMathOperator*{\argmin}{arg\,min}

\DeclareMathOperator{\ssm}{\smallsetminus}

\title{Geometric Measure Theory}
\author{Jacob Denson}

\begin{document}

\pagenumbering{gobble}

\maketitle

\tableofcontents

\pagenumbering{arabic}

\chapter{The Hausdorff Measure}

The expression of geometric properties of subsets of $\mathbf{R}^d$ requires more than can be expressed using the Lebesgue measure. For instance, curves and surfaces all have measure zero in two and three dimensions respectively, and thus we cannot distinguish them by the Lebesgue measure from any of the other nasty Lebesgue measurable subsets of measure zero. Hausdorff showed that there is a notion of `dimension' of measure zero subsets of $\mathbf{R}^d$ which matches the dimension of corresponding curves and surfaces. Even more interestingly, Hausdorff's theory of dimension gives certain fractal subsets non-integer dimension.

Here is the general idea. If $X = [0,1)$ is a unit interval, then $nX = [0,n)$ is the union of $n$ disjoint translates of $[0,1)$. If we instead consider the unit square $X = [0,1) \times [0,1)$, then $nX = [0,n) \times [0,n)$ is the union of $n^2$ disjoint translates of $[0,1)$. If $X$ is a unit cube, the $nX$ is the union of $n^3$ disjoint translates of $[0,1)$, and so on and so forth. Thus, it makes sense to define the dimension of $X$ to be the value $\alpha$ such that $nX$ is the union of $n^\alpha$ disjoint copies of $X$. Note that if $X$ is the Cantor set, then $3X$ is the union of two translates of $X$, so we would have to be willing to say that the Cantor set `has dimension $\log_3 2 = 0.6309\dots$.

By definition, for a subset $E$ of Euclidean space, we define
%
\[ H_\delta^s(E) = \inf \left\{ \sum_{n = 1}^\infty \text{diam}(B_n)^s : E \subset \bigcup_{n = 1}^\infty B_n, \text{diam}(B_n) \leq \delta \right\} \]
%
We then define $H^s(E) = \lim_{\delta \to 0} H_\delta^s(E)$. Then $H^s$ is an exterior measure, which satisfies $H^s(E \cup F) = H^s(E) + H^s(F)$ if $d(E,F) > 0$. Thus all Borel sets are measurable.

TODO: Which sets are measurable with respect to $H_\delta^s$?

\section{Energy Integrals and Frostman's Lemma}

Although upper bounding Hausdorff measures and dimensions is often quite easy to do, because the definition is defined in terms of outer covers. But finding lower bounds is often much harder. Frostman's lemma reduces this lower bound to finding the existence of measures on the set with good upper bounds for measures on balls. For a set $E$, we let $M(E)$ denote the set of finite measures with compact support which is a subset of $E$.

\begin{theorem}[Frostman's Lemma]
	Let $0 \leq s \leq n$. For a Borel set $E$, the Hausdorff dimension $H^\alpha(E)  > 0$ if and only if there is a finite Borel measure $\mu$ with compact support on $E$ and $\mu(B_r(x)) \lesssim r^\alpha$. In particular,
	%
	\[ \dim_{\mathbf{H}}(E) = \sup \{ \alpha \geq 0: \text{there is}\ \mu \in M(E)\ \text{such that}\ \mu(B_r(x)) \lesssim r^\alpha \} \]
\end{theorem}
\begin{proof}
	TODO
\end{proof}

A fundamental concept in the lower bounding of dimensions is the $\alpha$ energy of a Borel measure $\mu$, which is
%
\[ I_\alpha(\mu) = \int \int |x-y|^{-\alpha}\; d\mu(x) d\mu(y) = \int k_s * \mu\; d\mu \]
%
where $k_s(x) = |x|^{-\alpha}$, for $x \in \mathbf{R}^d$. If $0 < \beta < \alpha$, and $\mu$ has compact support. Integrating Frostman's lemma gives that for a Borel $E$,
%
\[ \text{dim}_{\mathbf{H}}(E) = \sup \{ \alpha: \text{there is}\ \mu \in M(A)\ \text{such that}\ I_\alpha(\mu) < \infty \} \]
%
The $\alpha$ dimensional energy then
%
\[ I_\alpha(\mu) \propto_{n,\alpha} \int_{\mathbf{R}^d} |\widehat{\mu}(\xi)|^2 |\xi|^{\alpha-d}\; d\xi \]
%
Thus
%
\[ \text{dim}_{\mathbf{H}}(E) = \sup \left\{ \text{there is}\ \mu \in M(A)\ \text{such that}\ \int |\widehat{\mu}(x)|^2 |x|^{\alpha-d}\; dx < \infty \right\} \]

\section{Projection Theorems}

Recall that the Grassmanian manifold $G(n,m)$ is a space parameterizing the family of $m$ dimensional hyperplanes in $\mathbf{R}^n$. The orthogonal group $O(n)$ acts on $G(n,m)$, and we let $\gamma_{nm}$ denote the resultant Borel probablity measure. We then have Marstrand's projection theorem

\begin{theorem}[Marstrand]
	s
\end{theorem}

\chapter{Fourier Dimension}

The {\bf Fourier dimension} of a Borel set $E$ is
%
\[ \dim_{\mathbf{F}}(E) = \sup \{s : \text{there is}\ \mu \in M(E)\ \text{s.t.}\ |\widehat{\mu}(\xi)| \leq |\xi|^{-s/2}  \} \]
%
This implies the energy integrals of the right dimension to converge, implying $\dim_{\mathbf{F}}(E) \leq \dim_{\mathbf{H}}(E)$. A set is {\bf Salem} if $\dim_{\mathbf{H}}(E) = \dim_{\mathbf{F}}(E)$.

\section{Dimensions of Brownian Motion}

Consider a one dimensional Brownian motion $W$. Then almost surely, for each $0 < \alpha < 1/2$, $W$ is locally $\alpha$ H\"{o}lder continuous. For a fixed Borel set $E$, The bound
%
\[ \dim_{\mathbf{H}}(W(E)) \leq \frac{1}{\alpha} \dim_{\mathbf{H}}(E) \]
%
then holds for almost every path of the motion. Taking $\alpha \uparrow 1/2$, we find the $\dim_{\mathbf{H}}(W(E)) \leq 2 \dim_{\mathbf{H}}(E)$. In this lecture we focus on a converse.

\begin{theorem}[Mckean, 1955]
	Let $A \subset [0,\infty)$ be Borel. Then $\dim_{\mathbf{H}}(W(E)) = 2\dim_{\mathbf{H}}(E) \wedge 1$ almost surely.
\end{theorem}

More generally,

\begin{theorem}[Kaufman's Dimension Doubling Theorem]
	Let $B$ be a Brownian motion in $\mathbf{R}^d$, for $d \geq 2$, then almost surely, for every Borel set $E$,
	%
	\[\dim_{\mathbf{H}}(B(E)) = 2\dim_{\mathbf{H}}(E) \]
\end{theorem}

Note that the almost surely condition is now independent of $E$, so we can apply this theorem to random sets. If $Z = \{ t \geq 0: B_t = 0 \}$ is the random zero set of a path of Brownian motion, and $d \geq 2$, then almost surely we find $\dim_{\mathbf{H}}(E) = 0$. For $d = 1$, the zero may not even be zero dimensional, so we know that Mckean's theorem cannot take out the almost surely over all subsets. We will follow Kahane's 1966 proof of Mckean's result. Consider the following lemma.

\begin{lemma}
	If $\mu$ is an $s$ dimensional measure supported on $[0,\infty)$, then almost surely, for all $|\xi| > 2$,
	%
	\[ |\widehat{\mu_W}(\xi)| \lesssim \frac{(\log |\xi|)^{1/2}}{|\xi|^{-s}} \]
	%
	where $\mu_W$ is the random pushforward measure of $\mu$ by the random path $W$, and the constant in the inequality is random.
\end{lemma}
\begin{proof}
	
\end{proof}

If $E$ was $s$ dimensional, we could find an $s$ dimensional probability measure on $E$. Then by Kahane's lemma, we compute, almost surely, that
%
\[ I_s(\mu_W) \lesssim O(1) + \int_{|\xi| > 2} |\widehat{\mu_W}(\xi)|^2 |\xi|^{s-1} \lesssim O(1) + \int_{-\infty}^\infty \frac{\log |x|}{|x|^{-1-s}} < \infty \]
%
so $\mu_W$ is $s$ dimensional, and so $\dim_{\mathbf{H}}(W(E)) \geq s$. Note that we actually proved something even stronger. The inequality above implies that $\dim_{\mathbf{F}}(W(E)) \geq s$ almost surely, so $W(E)$ is a Salem set almost surely. In particular, Salem sets exist. An alternate proof is to calculate, using Fubini's theorem, we calculate that if $Z \sim N(0,1)$, then provided $r < 2s < 1$,
	%
	\[ \mathbf{E}[I_r(\mu_W)] = \int_{-\infty}^\infty \mathbf{E} \left( \frac{1}{|W_t - W_s|^r} \right)\; dr = \left( \int_{-\infty}^\infty \frac{dt\; ds}{|t-s|^{r/2}} \right) \mathbf{E} \left( \frac{1}{|Z|^r} \right) < \infty \]
	%
	This doesn't require the $K$ lemma at all.

\end{document}