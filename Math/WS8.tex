\documentclass[12pt,a4paper]{article}

\usepackage{amsmath,amsthm,amsfonts,amssymb,amscd}
\usepackage{exsheets}
\usepackage{paralist}
\usepackage{fancyhdr}
\usepackage[top=2cm, bottom=4.5cm, left=2.5cm, right=2.5cm]{geometry}
\usepackage{enumitem}
\usepackage{multicol} %Allows to distribute \enumerates in multiple columns
\usepackage{multirow}

\theoremstyle{definition}
\newtheorem*{ssolution}{Solution} % This is implemented to not overlap with the exsheets package's solution environment.

\SetupExSheets{solution/print=false} %Set print=false to not print out solutions.
%\SetupExSheets{question/type=exam}
%\SetupExSheets[points]{name=point,name-plural=points}

\setlength{\parindent}{0.0in}
\setlength{\parskip}{0.05in}


\pagestyle{fancyplain}
\headheight 35pt
\lhead{\NetIDa}
\chead{\textbf{\Large Math 340 Worksheet 8}}
\lhead{Sections: 3.1 and 3.2}
\rhead{Date: 9.7.2022 \hspace{0.5in}}
\lfoot{}
\cfoot{}
\rfoot{\small\thepage}
\headsep 1.5em


\begin{document}
%% Written by: Jacob Denson
\PrintSolutionsTF{
    \textbf{The following are select solutions from the worksheet}.
}{
    \textbf{It may be useful for you to have this worksheet for future discussion sections.} \newline
    \textbf{It may be in your interest to solve the questions not in the order listed, but according to which questions you need practice with.}
    \textbf{Your TA may or may not give you specific advice or directions on which questions to try first.}
}

\begin{question}
Compute the following determinants
\begin{enumerate}
    \item $\begin{vmatrix}  4 & -3 & 5 \\ 5 & 2 & 0 \\ 2 & 0 & 4 \end{vmatrix}$.

    \item $\begin{vmatrix}  4 & 1 & 3 \\ 2 & 3 & 0 \\ 1 & 3 & 2 \end{vmatrix}$.
\end{enumerate}
\end{question}
\begin{solution}
	There are several methods we can use to compute the determinants. The first is to use the formula for the determinant directly, i.e. expanding a sum in permutations. This is feasible for the first two equations, but will lead to a lot of computation in the third example. An alternate method is to use Gaussian elimination, i.e. performing elementary row operations to obtain an upper triangular matrix. Each elementary row operation scales the determinant of the matrix, and if we keep track of all these multipliers, we can obtain the determinant of the original matrix. For $1.$ we get 
\begin{align*}
    \begin{vmatrix}  4 & -3 & 5 \\ 5 & 2 & 0 \\ 2 & 0 & 4 \end{vmatrix} &= -\begin{vmatrix} 2 & 0 & 4 \\ 5 & 2 & 0 \\ 4 & -3 & 5  \end{vmatrix}\\
    &= -2\begin{vmatrix} 1 & 0 & 2 \\ 5 & 2 & 0 \\ 4 & -3 & 5  \end{vmatrix}\\
    &= -2\begin{vmatrix} 1 & 0 & 2 \\ 0 & 2 & -10 \\ 4 & -3 & 5  \end{vmatrix}\\
    &= -2\begin{vmatrix} 1 & 0 & 2 \\ 0 & 2 & -10 \\ 0 & -3 & -3  \end{vmatrix}\\
    &= -4\begin{vmatrix} 1 & 0 & 2 \\ 0 & 1 & -5 \\ 0 & -3 & -3  \end{vmatrix}\\
    &= -4\begin{vmatrix} 1 & 0 & 2 \\ 0 & 1 & -5 \\ 0 & 0 & -18  \end{vmatrix}=(-4)\cdot (-18)=72\\
\end{align*}
%
 For $2.$ a similar method yields that
 %
 \[ \begin{vmatrix}  4 & 1 & 3 \\ 2 & 3 & 0 \\ 1 & 3 & 2 \end{vmatrix}=29. \]
 %
 One could use this method on $3.$ but one can save time by a simple trick. Notice that the third row of the matrix in the problem is twice the second row of the matrix. Thus the rows of the matrix are \emph{not} linearly independent, which means the determinant of the matrix is equal to zero.
\end{solution}

\begin{question}
	Compute the determinant
	%
	\[ \begin{vmatrix}
    	1 & 3 & 7 & 9 & 5 \\
    	2 & 0 & 3 & 0 & 1 \\
    	4 & 0 & 6 & 0 & 2 \\
    	0 & 1 & 2 & 8 & 3 \\
    	2 & 0 & 2 & 2 & 0 
    	\end{vmatrix}. \]
\end{question}

\begin{question}
For which values of $t$ is the following determinant nonzero?
$$\begin{vmatrix} t+1 & 4 \\ 2 & t-3\end{vmatrix}$$

\end{question}

\begin{solution}
We start by calculating the determinant. 
\begin{align*}
    \begin{vmatrix} t+1 & 4 \\ 2 & t-3\end{vmatrix} &= -\begin{vmatrix}  2 & t-3\\ t+1 & 4 \end{vmatrix}\\
    &= -2\begin{vmatrix}  1 & (t-3)/2\\ t+1 & 4 \end{vmatrix}\\
    &=-2\begin{vmatrix}  1 & (t-3)/2\\ 0 & 4-(t+1)(t-3)/2 \end{vmatrix}\\
    &=(t+1)(t-3)-8=t^2-2t-11
\end{align*}
The determinant is zero only when $t^2-2t-11=0$, so only when $t=1-2\sqrt{3}$ or $t=1+2\sqrt{3}$.
\end{solution}


\newpage

\begin{question}
	Let's compute a formula for the determinant of the $3 \times 3$ matrix
	%
	\[ A = \begin{pmatrix}
		1 & x & x^2\\
		1 & y & y^2\\
		1 & z & z^2
	\end{pmatrix} \]
	%
	where $x_1,x_2,x_3$ are arbitrary numbers. $A$ is called the $3 \times 3$ \emph{Vandermonde matrix}. To do this, let's use the definition of the determinant given by sums over permutations, i.e. for the sum
	%
	\[ \det(A) = \sum_j \text{sgn}(j) a_{1 j_1} a_{2 j_2} a_{3 j_3}. \]
	%
	where $j$ ranges over all permutations of $\{ 1, 2, 3 \}$.
	\begin{enumerate}
		\item Start by listing out all permutations of $\{ 1, 2, 3 \}$. There should be $3! = 6$ of them. Write them in the first column of the grid below.

		\item For each permutation $j$, compute the \emph{sign} $\text{sgn}(j)$ of the permutation. Write these values in the second column of the grid below.

		\item Finally, for each permutation $j$, compute $a_{1 j_1} a_{2 j_2} a_{3 j_3}$. Write these values in the third column of the grid below.

		\item One can now calculate the determinant by summing up the values in the third column multiplied by the signs in the second column. By doing some algebra, show that
		%
		\[ \det(A) = (x-y)(y-z)(z-x). \]
		%
		Challenge: Can one find a $4 \times 4$ matrix $A$ for which a similar result holds?
	\end{enumerate}
	%
	\begin{center}
	\begin{tabular}{ |c|c|c| } 
	\hline
	Permutations & Signs & Products \\
	\hline
	& &\\
	\hline 
	& &\\ 
	\hline
	& &\\ 
	\hline
	& &\\ 
	\hline
	& &\\ 
	\hline
	& &\\ 
	\hline
	\end{tabular}
	\end{center}
\end{question}

\begin{solution}
	\begin{center}
	\begin{tabular}{ |c|c|c| } 
	\hline
	Permutations & Signs & Products \\
	\hline
	$123$ & $+1$ & $yz^2$ \\
	\hline 
	$132$ & $-1$ & $y^2z$ \\ 
	\hline
	$213$ & $-1$ & $xz^2$ \\ 
	\hline
	$231$ & $+1$ & $xy^2$ \\ 
	\hline
	$312$ & $+1$ & $x^2z$ \\ 
	\hline
	$321$ & $-1$ & $x^2 y$ \\ 
	\hline
	\end{tabular}
	\end{center}
	%
	Thus the determinant is
	%
	\[ yz^2 - y^2 z - xz^2 + xy^2 + x^2z - x^2y. \]
	%
	We now expand out and rearrange terms to find that
	%
	\begin{align*}
		(x-y)(y-z)(z-x) &= (xy - xz - y^2 + yz)(z - x)\\
		&= xyz - xz^2 - y^2z + yz^2 - x^2y + x^2z + xy^2 - xyz\\
		&= yz^2 - y^2 z - xz^2 + xy^2 + x^2z - x^2y\\
		&= \det(A).
	\end{align*}
\end{solution}





\end{document}