\documentclass[12pt, dvipsnames]{report}

\usepackage{amsmath}
\usepackage{algorithm}
%\usepackage{algorithmic}
\usepackage[noend]{algpseudocode}

\usepackage{amsmath}
\usepackage{amssymb}
\usepackage{amsthm}
\usepackage{amsopn}

\usepackage{kpfonts}

\usepackage{graphicx}

% Probably don't need this on notes anymore
%\usepackage{kbordermatrix}

% Standard tool for drawing diagrams.
\usepackage{tikz}
\usepackage{tkz-berge}
\usepackage{tikz-cd}
\usepackage{tkz-graph}

\usepackage{comment}

%
\usepackage{multicol}

%
\usepackage{framed}

%
\usepackage{mathtools}

%
\usepackage{float}

%
\usepackage{subfig}

%
\usepackage{wrapfig}

%
\let\savewideparen\wideparen
\let\wideparen\relax
\usepackage{mathabx}
\let\wideparen\savewideparen

% Used for generating `enlightening quotes'
\usepackage{epigraph}

% Forget what this is used for :P
\usepackage[utf8]{inputenc}

% Used for generating quotes.
\usepackage{csquotes}

% Allows what to generate links inside
% generated pdf files
\usepackage{hyperref}

% Allows one to customize theorem
% environments in mathematical proofs.
\usepackage{thmtools}

% Gives access to a proof
\usepackage{lplfitch}

% I forget what this is for.
\usepackage{accents}

% A package for drawing simple trees,
% as a substitute for unnesacary TIKZ code
\usepackage{qtree}

% Enables sequent calculus proofs
\usepackage{ebproof}

% For braket notation
\usepackage{braket}

% To change line spacing when using mathematical notations which require some height!
\usepackage{setspace}

%\usepackage[dvipsnames]{xcolor}

\usepackage{float}

% For block commenting
\usepackage{comment}




\setlength\epigraphwidth{8cm}

\usetikzlibrary{arrows, petri, topaths, decorations.markings}

% So you can do calculations in coordinate specifications
\usetikzlibrary{calc}
\usetikzlibrary{angles}

\theoremstyle{plain}
\newtheorem{theorem}{Theorem}[chapter]
\newtheorem{axiom}{Axiom}
\newtheorem{lemma}[theorem]{Lemma}
\newtheorem{corollary}[theorem]{Corollary}
\newtheorem{prop}[theorem]{Proposition}
\newtheorem{exercise}{Exercise}[chapter]
\newtheorem{fact}{Fact}[chapter]

\newtheorem*{example}{Example}
\newtheorem*{proof*}{Proof}

\theoremstyle{remark}
\newtheorem*{exposition}{Exposition}
\newtheorem*{remark}{Remark}
\newtheorem*{remarks}{Remarks}

\theoremstyle{definition}
\newtheorem*{defi}{Definition}

\usepackage{hyperref}
\hypersetup{
    colorlinks = true,
    linkcolor = black,
}

\usepackage{textgreek}

\makeatletter
\renewcommand*\env@matrix[1][*\c@MaxMatrixCols c]{%
  \hskip -\arraycolsep
  \let\@ifnextchar\new@ifnextchar
  \array{#1}}
\makeatother

\renewcommand*\contentsname{\hfill Table Of Contents \hfill}

\newcommand{\optionalsection}[1]{\section[* #1]{(Important) #1}}
\newcommand{\deriv}[3]{\left. \frac{\partial #1}{\partial #2} \right|_{#3}} % partial derivative involving numerator and denominator.
\newcommand{\lcm}{\operatorname{lcm}}
\newcommand{\im}{\operatorname{im}}
\newcommand{\bint}{\mathbf{Z}}
\newcommand{\gen}[1]{\langle #1 \rangle}

\newcommand{\End}{\operatorname{End}}
\newcommand{\Mor}{\operatorname{Mor}}
\newcommand{\Id}{\operatorname{id}}
\newcommand{\visspace}{\text{\textvisiblespace}}
\newcommand{\Gal}{\text{Gal}}

\newcommand{\xor}{\oplus}
\newcommand{\ft}{\wedge}
\newcommand{\ift}{\vee}

\newcommand{\prob}{\mathbf{P}}
\newcommand{\expect}{\mathbf{E}}
\DeclareMathOperator{\Var}{\mathbf{V}}
\newcommand{\Ber}{\text{Ber}}
\newcommand{\Bin}{\text{Bin}}

%\newcommand{\widecheck}[1]{{#1}^{\ft}}

\DeclareMathOperator{\diam}{\text{diam}}

\DeclareMathOperator{\QQ}{\mathbf{Q}}
\DeclareMathOperator{\ZZ}{\mathbf{Z}}
\DeclareMathOperator{\RR}{\mathbf{R}}
\DeclareMathOperator{\HH}{\mathbf{H}}
\DeclareMathOperator{\CC}{\mathbf{C}}
\DeclareMathOperator{\AB}{\mathbf{A}}
\DeclareMathOperator{\PP}{\mathbf{P}}
\DeclareMathOperator{\MM}{\mathbf{M}}
\DeclareMathOperator{\VV}{\mathbf{V}}
\DeclareMathOperator{\TT}{\mathbf{T}}
\DeclareMathOperator{\LL}{\mathcal{L}}
\DeclareMathOperator{\EE}{\mathbf{E}}
\DeclareMathOperator{\NN}{\mathbf{N}}
\DeclareMathOperator{\DQ}{\mathcal{Q}}
\DeclareMathOperator{\IA}{\mathfrak{a}}
\DeclareMathOperator{\IB}{\mathfrak{b}}
\DeclareMathOperator{\IC}{\mathfrak{c}}
\DeclareMathOperator{\IP}{\mathfrak{p}}
\DeclareMathOperator{\IQ}{\mathfrak{q}}
\DeclareMathOperator{\IM}{\mathfrak{m}}
\DeclareMathOperator{\IN}{\mathfrak{n}}
\DeclareMathOperator{\IK}{\mathfrak{k}}
\DeclareMathOperator{\ord}{\text{ord}}
\DeclareMathOperator{\Ker}{\textsf{Ker}}
\DeclareMathOperator{\Coker}{\textsf{Coker}}
\DeclareMathOperator{\emphcoker}{\emph{coker}}
\DeclareMathOperator{\pp}{\partial}
\DeclareMathOperator{\tr}{\text{tr}}

\DeclareMathOperator{\supp}{\text{supp}}

\DeclareMathOperator{\codim}{\text{codim}}

\DeclareMathOperator{\minkdim}{\dim_{\mathbf{M}}}
\DeclareMathOperator{\hausdim}{\dim_{\mathbf{H}}}
\DeclareMathOperator{\lowminkdim}{\underline{\dim}_{\mathbf{M}}}
\DeclareMathOperator{\upminkdim}{\overline{\dim}_{\mathbf{M}}}
\DeclareMathOperator{\lhdim}{\underline{\dim}_{\mathbf{M}}}
\DeclareMathOperator{\lmbdim}{\underline{\dim}_{\mathbf{MB}}}
\DeclareMathOperator{\packdim}{\text{dim}_{\mathbf{P}}}
\DeclareMathOperator{\fordim}{\dim_{\mathbf{F}}}

\DeclareMathOperator*{\argmax}{arg\,max}
\DeclareMathOperator*{\argmin}{arg\,min}

\DeclareMathOperator{\ssm}{\smallsetminus}

%\newcommand{\vvvert}[1]{{\left\vert\kern-0.25ex\left\vert\kern-0.25ex\left\vert #1 
%    \right\vert\kern-0.25ex\right\vert\kern-0.25ex\right\vert}}

\title{Functional Analysis}
\author{Jacob Denson}

\begin{document}

\pagenumbering{gobble}
\maketitle
\tableofcontents
\pagenumbering{arabic}

\part{Vector Spaces with a Metric}

Functional analysis is the interlace of algebra and analysis, in which algebraic structures are endowed with topological structure. The approach's utility counts for the rapid growth of applications over the past century, be it in quantum mechanics, statistics, or computing science. Rarely are we concerned with a single object, like a function, a random variable, or a measure, but instead consider large classes of such objects. One technique for handling these classes is to add algebraic structure which elaborates on the natural relation between these objects. Functional analysis adds in analytical structure to the class, so we may combine the algebraic and analytical structure to obtain powerful results about the objects involved.

\begin{example}
    We rarely analyze a measurable function $f$ in isolation. Instead, we prove theorems about a class of measurable functions defined on the same measurable space. If $f$ and $g$ are measurable, then we may consider their addition $f + g$, their multiplication $fg$, and scaling $\lambda f$ (for $\lambda \in \mathbf{R}$), which are all measurable. Thus the space of measurable functions on a set is a vector space. Similarily, the sum and product of two continuous functions on a topological space is continuous, so, we may consider $C[0,1]$, 2the space of all continuous, real-valued functions on the unit interval $[0,1]$, as a vector space.
\end{example}

The most common analytic operations are scaling and rotation. As a result, the most common vector spaces which occur in functional analysis are defined over the real and complex numbers. More specifically, real vector spaces $X$ are those equipped with natural scaling operations $x \mapsto \lambda x$ for $\lambda > 0$, combined with `reflection' operators $x \mapsto -x$. If $X$ is a real vector space with a specified `counterclockwise twisting', i.e. a linear map $J: X \to X$ with $J^2(x) = -x$ for all $x \in X$, then there is a natural way to turn $X$ into a complex vector space by defining $(a + ib)x = ax + b(Jx)$. The geometry of the complex numbers then enables us to define a counterclockwise twist by any angle $\theta$, by the operation $x \mapsto e^{i\theta} x$. Thus we have a natural operation of the complex numbers over the space.





\chapter{Banach Spaces}

The most natural way to associate a space with topological structure is by associating a metric on the space. There is an even more natural vector space operation on a vector space. We can think of elements of a vector space as `arrows' extending from the origin. Thus a natural way to add a topology is to give these arrows a `length', and the function associating a vector with a length is konwn as a {\it norm}. More precisely, a {\bf pre-norm} on a vector space $X$ is a map $\| \cdot \|: X \to [0,\infty)$ which is {\it homogenous}, in the sense that for each $\alpha \in \mathbf{K}$ and $x \in X$, $\| \alpha x \| = |\alpha| \| x \|$, and satisfies the {\it triangle inequality} $\| x + y \| \leq \| x \| + \| y \|$ for each $x,y \in X$. If in addition, we have $\| x \| = 0$ if and only if $x = 0$, then we say $\| \cdot \|$ is a {\it norm}. A vector space $X$ equipped with a norm is called a {\it norm space}.

On finite dimensional spaces, there is essentially a unique way to define a norm on the space. We say two norms $\| \cdot \|_1$ and $\| \cdot \|_2$ on a vector space $X$ are {\it comparable} if for each $x \in X$, $\| x \|_1 \sim \| x \|_2$. It is a simple consequence of the Heine-Borel theorem that all norms on a finite dimensional space are normable. Thus there is only a single notion of `size' on such spaces. On the other hand, there are many ways to measure the size of a function, or more general vectors in infinite dimensional norm spaces.

\begin{example}
    Let $X$ be a measure space. For each $0 < p < \infty$, let $\mathcal{L}^p(X)$ denote the space of all measurable functions $f: X \to \mathbf{K}$ such that
    %
    \[ \| f \|_p = \int |f(x)|^p\; dx < \infty. \]
    %
    Minkowski's inequality tells us that $\| \cdot \|_p$ is a prenorm on $\mathcal{L}^p(X)$. Notice that if $u$ is a function such that $u(x) = 0$ for almost every $x$, then $\| f \|_p = \| f + u \|_p$. In particular, this means that $\| \cdot \|_p$ is a well defined operation on the quotient space of $\mathcal{L}^p(X)$ by the vector space of functions which are equal to zero almost everywhere. We denote this quotient space by $L^p(X)$. Since $\| f \|_p = 0$ if and only if $f$ is equal to zero almost everywhere, this implies that $L^p(X)$, equipped with the induced norm $\| \cdot \|$, is a norm space.
\end{example}

\begin{example}
    Given a measurable function $f$ on a measure space $X$, we can define the $L^\infty$ pre-norm
    %
    \[ \| f \|_\infty = \inf \{ t > 0: |f(x)| \leq t\ \text{for almost every $x$} \}. \]
    %
    The space of functions with finite $L^\infty$ prenorm is denoted by $\mathcal{L}^\infty(X)$, and the quotient space by the class of functions equal to zero is denoted $L^\infty(X)$.
\end{example}

\begin{example}
    On a compact topological space $K$, we can consider the family of all bounded continous functions from $K$ to $\mathbf{K}$, denoted $C(K)$. The quantity
    %
    \[ \| f \|_\infty = \sup \{ |f(x)|: x \in K \}, \]
    %
    gives $C(K)$ the structure of a norm space. More generally, if $X$ is any norm space, we can consider the family of all bounded continuous functions from $K$ to $X$, with associated norm
    %
    \[ \| f \|_{C(K,X)} = \sup \{ \| f(x) \|: x \in K \}. \]
    %
    If $X$ is a Banach space, then $C(K,X)$ is a Banach space.
\end{example}

For the purpose of the general theory, one need not know too much measure theory. The most important norm spaces for intuition are the simplest infinite dimensional examples of the spaces $L^p(X)$, where $X$ is the collection of all natural numbers, with the associated counting measure. This space is often denoted $l^p$, and can be practically defined as the collection of all sequences $\{ a_n \}$ such that
%
\[ \| a \|_p = \left( \sum_{n = 1}^\infty |a_n|^p \right)^{1/p} < \infty. \]
%
For $p = \infty$, we let $l_\infty$ denote the collection of all bounded sequences, and let $\| a \|_\infty = \sup |a_n|$.

\begin{remark}
    We will focus on norms in these notes, but prenorms are really no more general. If $X$ is a space with a prenorm $\| \cdot \|$, then the space $X_0 = \{ x \in X: \| x \| = 0 \}$ forms a subspace of $X$. We have $\| x \| = \| x + x_0 \|$ for each $x_0 \in X_0$, so if we form the quotient space $X' = X/X_0$, then the quantity $\| x + X_0 \|$ is well defined, and this is a {\it norm} on $X'$. So results about norms automatically extend to results about prenorms where we identify vectors which differ by an element of $X_0$ in a single equivalence class. This is precisely what we did to obtain the norm spaces $L^p(X)$ from the prenorm spaces $\mathcal{L}^p(X)$.
\end{remark}

For each norm on a space $X$, we can associate a metric given by $d(x,y) = \| x - y \|$. The most powerful theorems of functional analysis deal with the study of norm spaces where the induced metric on a norm space is {\it complete}. We say such a space is a {\it Banach space}. It is a basic fact of measure theory that the spaces $L^p(X)$ are complete, and it is also easy to see that the space $C(K)$ is complete, as is proved in a basic class in real analysis.

\begin{theorem}
    $L^p(X)$ is a Banach space.
\end{theorem}
\begin{proof}
    We leave the $p = \infty$ case to the reader, since the argument is elementary. Let $f_1, f_2, \dots$ be a Cauchy sequence in $L^p(X)$. Then we can find a subsequence $f_{k_i}$ such that for each $m \geq 1$,
    %
    \[ \| f_{k_{i+m}} - f_{k_i} \|_p \leq 1/4^k. \]
    %
    In particular, this means that the infinite sum
    %
    \[ \sum_{i = 1}^\infty |f_{k_{i+1}}(x) - f_{k_i}(x)| \]
    %
    converges absolutely for almost all $x$. This is a consequence of the Borel-Cantelli lemma, since if 
    %
    \[ E_k = \{ x : |f_{k_{i+1}}(x) - f_{k_i}(x)| \geq 1/2^{pk} \}, \]
    %
    then Markov's inequality implies $|E_i| \leq 1/2^{ip}$, and so $\sum_{i = 1}^\infty |E_i| < \infty$. Thus we can define a measurable function $f$ almost everywhere by the infinite series
    %
    \[ f(x) = f_{k_1}(x) + \sum_{i = 1}^\infty f_{k_{i+1}}(x) - f_{k_i}(x) = \lim_{n \to \infty} f_{k_n}(x). \]
    %
    Fatou's lemma and Minkowski's inequality imply that
    %
    \begin{align*}
        \| f - f_{k_i} \|_p &= \left( \int \left| \lim_{n \to \infty} f_{k_n}(x) - f_{k_i}(x) \right|^p\; dx \right)^{1/p}\\
        &\leq \liminf_{n \to \infty} \left( \int \left| f_{k_n}(x) - f_{k_i}(x) \right|^p \right)^{1/p}\\
        &\leq \liminf_{n \to \infty} \| f_{k_n} - f_{k_i} \|_p \leq 1/4^k.
    \end{align*}
    %
    Thus $f_{k_i} \to f$ in the $L^p$ norm. Since $\{ f_k \}$ is a Cauchy sequence, it is easy to justify that it is also true that $f_k \to f$ in the $L^p$ norm. Thus Cauchy sequences converge in $L^p(X)$.
\end{proof}

\begin{remark}
    Studying Banach spaces is not too much more general than studying norm spaces. Given any norm space $X$, we can always find a unique Banach space $X'$ containing $X$, such that $X$ is dense in $X'$. The space $X'$ is known as the {\it completion} of $X$. One can view the elements of $X' - X$ as `artifical' elements of $X$. But, nonetheless are key to understanding the geometric structure of $X$. For instance, if we equip $C[0,1]$ with the $L^1$ norm induced by the Lebesgue measure, then $C[0,1]$ is an incomplete norm space. The completion of $C[0,1]$ with respect to this norm is the space $L^1[0,1]$ of integrable functions. One perspective of the class of measurable functions is that they are an artifice which provides us with stronger tools to analyze the class of the more basic functions which we grew familiar with in our first years of university.
\end{remark}

The examples above provide most of the intuition we will need for the basic Banach space theory. But it is useful to know other examples for applications in other areas of analysis.

\begin{example}
    Let $K$ be a compact metric space. For each $0 < p \leq 1$, we say a function $f: K \to \mathbf{K}$ is {\it Lipschitz of order $p$} if $|f(x) - f(y)| \lesssim d(x,y)^p$ for each $x,y \in K$. The set of such functions forms a vector space $\text{Lip}_p(K)$, which is a subspace of $C[0,1]$, but not a closed subspace. For each $f \in \text{Lip}_p [0,1]$, we define the Lipschitz norm
    %
    \[ \| f \|_{\text{Lip}_p(K)} = \| f \|_\infty + \sup \frac{|f(x) - f(y)|}{d(x,y)^p}. \]
    %
    This norm gives $\text{Lip}_p(K)$ the structure of a complete metric space. If $\{ f_n \}$ is a Cauchy sequence in $\text{Lip}_p(K)$, the completeness of $L^\infty(K)$ shows there exists $f \in L^\infty(K)$ such that $f_n \to f$ in the $L^\infty$ norm. Since $\{ f_n \}$ is Cauchy,  there exists a constant $M$ such that for all $n$, $|f_n(x) - f_n(y)| \leq M |x - y|^p$. In particular, the uniform convergence of $f_n$ to $f$ implies that $|f(x) - f(y)| \leq M |x - y|^p$, so $f \in \text{Lip}_p(K)$. Now
    %
    \begin{align*}
        \lim_{n \to \infty} &\sup_{x,y \in K} \frac{|[f(x) - f_n(x)] - [f(y) - f_n(y)]|}{d(x,y)^p}\\
        &= \lim_{n \to \infty} \sup_{x,y \in K} \lim_{m \to \infty} \frac{[f_m(x) - f_n(x)] - [f_m(y) - f_n(y)]}{d(x,y)^p}\\
        &\leq \lim_{n \to \infty} \limsup_{m \to \infty} \sup_{x,y \in K} \frac{[f_m(x) - f_n(x)] - [f_m(y) - f_n(y)]}{d(x,y)^p}\\
        &\leq \lim_{n \to \infty} \limsup_{m \to \infty} \| f_m - f_n \|_{\text{Lip}_p(K)} = 0.
    \end{align*}
    %
    The last equality is equivalent to the fact that $\{ f_n \}$ is a Cauchy sequence in $\text{Lip}_p(K)$. But this means that $f_n$ converges to $f$ in the $\text{Lip}_p(K)$ metric. Thus $\text{Lip}_p(K)$ is a complete metric space.
\end{example}

We can develop the theory of series in a norm space. Given a sequence $\{ x_n \}$. We say the sequence is absolutely summable if $\sum \| x_n \| < \infty$. The same proof as for $\RR$ shows that such a sequence is {\it unconditionally summable}, i.e. that for each permutation $\pi: \mathbf{N} \to \mathbf{N}$,
%
\[ \sum_{n = 1}^\infty x_{\pi(n)} = \sum_{n = 1}^\infty x_n. \]
%
A space $X$ is a Banach space if and only if every absolutely summable series converges, which was a fact we implicitly used in our proof that $L^p(X)$ was a Banach space. However, in infinite dimensional spaces, there are always examples of unconditionally summable sequences which are not absolutely summable. This is a result of A. Dvoretzky and C.A. Rogers.

\begin{example}
    Consider the space $c_0$. For each $n$, let $e_n$ denote the sequence which is equal to zero except at the $n$'th position, which takes the value 1. If we set $x_n = e_n / n$, then the sequence $\{ x_n \}$ is not absolutely summable, since
    %
    \[ \sum_{n = 1}^\infty \| x_n \|_\infty = \sum_{n = 1}^\infty \frac{1}{n} = \infty. \]
    %
    Nonetheless, the sequence is unconditionally summable to the sequence $\sum_{n = 1}^\infty e_n/n$.
\end{example}

\section{Subspaces}

If $Y$ is a subspace of a norm space $X$, then $Y$ obviously has the natural structure of a norm space. We note that if $X$ is a Banach space, then $Y$ is a Banach space under this induced norm if and only if $Y$ is a {\it closed} subspace of $X$. Here are some examples.

\begin{example}
    Let $c_0 \subset l_\infty$ be the family of all sequences $\{ x_n \}$ such that $\lim_{n \to \infty} x_n = 0$. Then $c_0$ is a closed subspace of $l_\infty$, and is thus a Banach space. The space $c_{00}$ of all sequences $\{ x_n \}$ which are only non-zero for finitely many values is a subspace of $c_0$, but it is not closed, and therefore not a Banach space. $c_{00}$ is dense in $l_p$ for all $1 \leq p < \infty$, and dense in $c_0$, but is not dense in $l_\infty$.
\end{example}

\begin{example}
    Let $\mathbf{T}$ be the interval $[-\pi,\pi)$ equipped with the Lebesgue measure. For each $f \in L^p(\mathbf{T})$, and $n \in \mathbf{Z}$, we define
    %
    \[ \widehat{f}(n) = \frac{1}{2\pi} \int_{-\pi}^\pi f(x) e^{-nix}\; dx. \]
    %
    H\"{o}lder's inequality shows that $|\widehat{f}(n)| \leq \| f \|_p$ for each $n \in \mathbf{Z}$. In particular, this means the set
    %
    \[ H^p(\mathbf{T}) = \{ f \in L^p(\mathbf{T}) : \widehat{f}(n) = 0\ \text{if $n < 0$} \} \]
    %
    is closed, and is therefore a Banach space. These are examples of {\it Hardy spaces} on $\mathbf{T}$. They can be identified with holomorphic functions on the unit disk under the identification of $f$ with the analytic function
    %
    \[ u(z) = \sum_{n = 0}^\infty \widehat{f}(n) z^n, \]
    %
    which converges for $|z| < 1$ since $\{ \widehat{f}(n) \}$ is a bounded sequence. The $L^p$ norm also implies $u$ satisfies a suitable growth condition as $z$ approaches the boundary of the unit disk, but we do not dwell on this fact.
\end{example}

Before we move on, let us introduce some notation. For any norm space $X$, we let $B_X$ denote the closed unit ball in $X$, and $S_X$ the unit sphere, i.e.
%
\[ B_X = \{ x \in X: \| x \| \leq 1 \}\quad\text{and}\quad S_X = \{ x \in X: \| x \| = 1 \}. \]
%
Given two sets $A$ and $B$, we let $A + B = \{ a + b : a \in A, b \in B \}$. Then for each $a \in X$, $a + B_X$ is the closed unit ball centered at $a$. Similarily, we can define $A - B = \{ a - b : a \in A, b \in B \}$, and for each $\alpha \in \mathbf{K}$, $\alpha A = \{ \alpha a : a \in A \}$.

\section{Convexity}

A critical notion in functional analysis is {\it convexity}. A set $E \subset X$ is {\it convex} if for each $x,y \in E$ and $\lambda \in (0,1)$, $\lambda x + (1 - \lambda) y \in E$. A set $E$ is {\it balanced} if $\alpha E \subset E$ for each $|\alpha| \leq 1$, and {\it symmetric} if $-E = E$. And a set $E$ is {\it absorbing} if, for each $x \in X$, there exists $t$ such that for $|\alpha| \geq t$, $x \in \alpha E$. As an example, the unit ball $B_X$ is closed, convex, and absorbing. In the future, we will make heavy use of the following result.

\begin{theorem}
    Every closed, convex, absorbing set in a Banach space contains a neighbourhood of the origin.
\end{theorem}
\begin{proof}
    Let $E \subset X$ be a closed, convex, absorbing set. Then $E \cap (-E)$ is closed, convex, and absorbing, so we may assume without loss of generality that $E$ is symmetric. It suffices to show $E^\circ$ is nonempty, for then $E^\circ/2 + (-E^\circ)/2$ is a neighbourhood of the origin contained in $E$. Now assume $E^\circ = \emptyset$. This means that for each $n$, $F_n = (nE)^c$ is an open, dense set. Thus $\bigcap F_n$ is dense. But this is impossible, since $\bigcup nE = X$. Thus we conclude $E^\circ \neq \emptyset$.
\end{proof}

For each set $E \subset X$, we let $\text{co}(E)$ denote the smallest convex subset of $X$ containing $E$. This is the {\it convex hull} of $E$. For a set $E$, we let $\langle E \rangle$ denote the smallest subspace of $X$ containing $E$, and $[E]$ denote the smallest {\it closed} subspace.

\section{Bounded Linear Operators}

A natural object of study in linear algebra is to understand the family of linear maps between two vector spaces $X$ and $Y$. In the theory of norm spaces, we want to understand the family of {\it continuous} linear maps. As the next theorem shows, these linear maps are also called bounded.

\begin{theorem}
    Let $X$ and $Y$ be norm spaces, and $T: X \to Y$ a linear map. Then the following are equivalent:
    %
    \begin{itemize}
        \item $T$ is continuous.
        \item $T$ is continuous at $0 \in X$.
        \item $T$ is uniformly continuous.
        \item There exists $M \geq 0$ such that $\| Tx \| \leq M \| x \|$ for all $x \in X$.
    \end{itemize}
\end{theorem}
\begin{proof}
    We leave the proof of the equivalence of (1), (2), and (3) to the reader. If $T$ is continuous at the origin, then in particular, since $T(0) = 0$, for each $\varepsilon > 0$, there exists $\delta > 0$ such that if $\| x \| \leq \delta$, $\| Tx \| \leq \varepsilon$. But this means that $\| Tx \| \leq (\varepsilon / \delta) \| x \|$ if $\| x \| \leq \delta$, and by scaling an arbitrary $x$, one sees that this inequality actually holds for all $x \in X$. Conversely, if $M$ exists such that $\| Tx \| \leq M \| x \|$, and $\varepsilon > 0$, and we set $\delta = \varepsilon / M$, then if $\| x \| \leq \delta$, $\| Tx \| \leq \varepsilon$, so $T$ is continuous at the origin.
\end{proof}

We let $B(X,Y)$ denote the class of bounded linear operators from $X$ to $Y$. For each such operator, we define the {\it operator norm}
%
\[ \| T \| = \sup_{x \neq 0} \frac{\| Tx \|}{\| x \|}. \]
%
This quantity is finite if and only if the operator is bounded by the equivalence above. One can verify that this definition gives $B(X,Y)$ the structure of a norm space, and moreover, if $Y$ is a Banach space, then $B(X,Y)$ is a Banach space. Moreover, given $T \in B(X,Y)$ and $S \in B(Y,Z)$, we can form the composition operator $S \circ T \in B(X,Z)$, and one verifies that
%
\[ \| S \circ T \| \leq \| S \| \| T \|. \]
%
Thus spaces of operators have even more algebraic and analytic structure than norm spaces. This structure is heavilty exploited in the theory of {\it operator algebras}, covered later in these notes.

\begin{example}
    If $Y$ is not a Banach space, then $B(X,Y)$ can fail to be a Banach space. For instance, if $Y$ is $c_{00}$ equipped with the same norm as $l_1$, then $B(l_1,Y)$ is not a Banach space. For instance, if
    %
    \[ T_n(a) = \sum_{k = 1}^n \frac{1}{k^2} \left( \sum_{i = k}^\infty a_i \right) e_k, \]
    %
    then $\| T_n - T_{n + m} \| \leq \sum_{k = n}^\infty (1/k)^2$, so $\{ T_n \}$ is Cauchy. However, the limit of these sequence of operators is the map
    %
    \[ T(a) = \sum_{k = 1}^\infty \frac{1}{k^2} \left( \sum_{i = k}^\infty a_i \right) e_k \]
    %
    which is an element of $B(l_1,l_1)$, but not an element of $B(l_1,Y)$.
\end{example}

If $\{ T_n \}$ is a sequence of operators in $B(X,Y)$ which coverges in the operator norm to an operator $T$, then it is obvious that $Tx = \lim_{n \to \infty} T_nx$ for each $x \in X$. Pointwise convergence does not imply convergence in the operator norm, as the next example shows.

\begin{example}
    Recall the Banach space $c_0$, and define a sequence of operators $T_n: c_0 \to \mathbf{K}$ by letting $T_n(a) = a_n$. Then each $T_n$ is a bounded operator, with $\| T_n \| = 1$. Thus the sequence $\{ T_n \}$ does not converge to zero in the operator norm. However, for each $a \in c_0$, $\lim_{n \to \infty} T_n(a) = 0$, so the sequence $\{ T_n \}$ does converge pointwise to zero.
\end{example}

The theory of bounded operators for finite dimensional norm spaces is effectively trivial.

\begin{theorem}
    Every linear map from a finite dimensional norm space $X$ to a norm space $Y$ is bounded.
\end{theorem}
\begin{proof}
    Let $\{ e_1, \dots, e_d \}$ be a basis for $X$, for which, without loss of generality, we may assume $\| e_i \| = 1$ for each $i$. Define, for each $x \in X$ with $x = a_1e_1 + \dots + a_d e_d$,
    %
    \[ |x| = (|a_1|^2 + \dots + |a_d|^2)^{1/2}. \]
    %
    Notice that the triangle inequality implies that
    %
    \[ \| x \| \leq |a_1| + \dots + |a_d| \leq \sqrt{d} \left( |a_1|^2 + \dots + |a_d|^2 \right)^{1/2}. \]
    %
    Thus $\| \cdot \|$ is continuous with respect, to the topology induced by $|\cdot|$. If $S$ is the unit sphere with respect to the norm $|\cdot|$, then the Heine-Borel theorem implies that $S$ is compact, and in particular, $\| \cdot \|$ attains a maximum and a minimum on $S$. In particular, since $\| x \| \neq 0$ for each $x \in S$, this implies that there is $\varepsilon > 0$ such that $\| x \| \geq \varepsilon$ for each $x \in S$. And rescaling thus shows that $\| x \| \geq \varepsilon |x|$ for each $x \in X$. In particular, this means that $|\cdot|$ and $\| \cdot \|$ are comparable. In particular, $T$ is continuous with respect to $\| \cdot \|$ if and only if it is continuous with respect to $| \cdot |$. And we then find that for any $x = a_1e_1 + \dots + a_de_d$, the triangle inequality implies
    %
    \[ \| Tx \| \leq |a_1| \| Te_1 \| + \dots + |a_d| \| Te_d \| \leq (|a_1| + \dots + |a_d|) \max \| Te_i \| \leq \sqrt{d} |x| \| Te_i \|. \]
    %
    Thus $T$ is a continuous linear map.
\end{proof}

Conversely, if $X$ is infinite dimensional, and $Y \neq 0$, then there exist discontinous linear maps from $X$ to $Y$. For instance, if we consider a basis $\mathcal{B}$ for $X$ containing at least countably many elements $\{ e_1, e_2, \dots \}$, such that $\| e_i \| = 1$ for each $i$, and we find $y \neq 0$ is in $Y$, then we can define a linear map such that $T(e_i) = i \cdot y$, and $T(f) = 0$ for each $f \in \mathcal{B}$ with $f \neq e_i$ for each $i$. Then $T$ is unbounded.

\begin{remark}
    A key component of the proof above was the fact that $X$ possessed a norm whose respective topology satisfied the Heine-Borel property. If $X$ is infinite dimensional, no such norm can exist. This can be shown by showing that for any norm $\| \cdot \|$ on an infinite dimensional space $X$, and any $n$, there is a countable collection $\{ x_1, x_2, \dots \} \subset S_X$ such that $\| x_i - x_j \| \geq 1$ for $i \neq j$. Clearly this sequence can have no accumulation points. We can define such a sequence inductively. Given $\{ x_1, \dots, x_n \}$, let $X_n = \langle x_1, \dots, x_n \rangle$. Then $X_n$ is a finite dimensional norm space, hence a Banach space, hence closed in $X$. In particular, if $y \not \in X_n$, then $d(X_n,y) > 0$. If $\alpha = d(X_n,y)$, then $d(X_n,y/\alpha) = 1$. But this means that $\| x_i - y/\alpha \| \geq 1$ for each $i \in \{ 1, \dots, n \}$, so we can set $x_{n+1} = y/\alpha$.
\end{remark}

We say a map $T: X \to Y$ between two norm spaces is an {\it isomorphism} if for each $x \in X$, $\| Tx \| \sim \| x \|$. In particular, this implies $T$ is continuous and injective, but not necessarily surjective. We say $X$ is {\it embedded} in $Y$. If $T$ is surjective, we can define an inverse map $T^{-1}: Y \to X$, which is continuous, and we then say $X$ and $Y$ are isomorphic. Note that if $X$ is a Banach space, then $T(X)$ is also a Banach space. In most other areas of math, an isomorphism has to be surjective, but unfortunately, the definition of isomorphism above has become standardized. A stronger notion than an isomorphism is an {\it isometry}, which is a map $T: X \to Y$ such that $\| Tx \| = \| x \|$ for all $x \in X$. Again, $T$ need not be surjective, and we say $X$ has been {\it embedded} in $Y$. If $T$ is surjective, $T^{-1}$ is also an isometry, and we say $X$ and $Y$ are isometrically isomorphic. Isomorphic spaces have many of the same properties, but in order to transfer all properties about norm spaces, one really needs an isometry.

\begin{example}
    Let $c \subset l_\infty$ be the space of all convergent sequences $\{ x_n \}$. Then $c$ is a closed subspace, and thus a Banach space. In fact, $c$ is isomorphic to $c_0$, since we can define an operator $T: c_0 \to c$ by setting
    %
    \[ T(a) = (a_0 + a_1,a_0 + a_2,\dots). \]
    %
    We have $\| T(a) \|_\infty \leq 2 \| a \|_\infty$, so $T$ is continuous. Furthermore, $T$ is invertible, with
    %
    \[ T^{-1}(b) = \left( \lim_{n \to \infty} b_n, b_0 - \lim_{n \to \infty} b_n, b_1 - \lim_{n \to \infty} b_n, \dots \right), \]
    %
    and $\| T^{-1}(b) \|_\infty \leq 2 \| b \|_\infty$. Thus we conclude
    %
    \[ (1/2) \| a \|_\infty \leq \| T(a) \|_\infty \leq 2 \| a \|_\infty, \]
    %
    and so $c$ and $c_0$ are isomorphic. Nonetheless, these spaces are {\it not} isometric. One property that $c_0$ has, but $c$ does not have, is that if $a \in S_{c_0}$, then there are $b, c \in S_{c_0}$ with $b \neq c$ such that $a = (b + c)/2$. In the language of convexity, we would say that $S_{c_0}$ has {\it no extreme points}. To see why, we define
    %
    \[ b_n = \begin{cases} a_n &: |a_n| \geq 1/2 \\ a_n/2 &: |a_n| < 1/2 \end{cases}\quad\text{and}\quad c_n = \begin{cases} a_n &: |a_n| \geq 1/2 \\ 3a_n/2 &: |a_n| < 1/2 \end{cases}. \]
    %
    Clearly $a = (b+c)/2$, and $\| b \|_\infty = \| c \|_\infty = 1$. Since $\lim_{n \to \infty} a_n = 0$, eventually $|a_n| \leq 1/2$, and so $b_n \neq c_n$. On the other hand, the sequence $a = (1,1,\dots)$ lies in $S_c$, yet it $b,c \in S_c$ satisfy $a = (b + c)/2$, then $b = c = a$.
\end{example}

A special case of linear operators is where $Y = \mathbf{K}$. The space $B(X,\mathbf{K})$ is then known as the {\it dual space} of $X$, denoted by $X^*$. This is one of the simplest operator spaces to analyze, and thus is a useful tool for analyzing the space $X^*$.

\section{Three Fundamental Theorems}

\begin{theorem}
    No Banach space $X$ has a countably infinite basis.
\end{theorem}
\begin{proof}
    Suppose $X$ has a countable infinite basic $\{ e_1, e_2, \dots \}$. For each $n$, let $X_n = \langle e_1, \dots, e_n \rangle$. Then $X_n$ is closed and has non-empty interior (if it's interior was non-empty, it would be absorbing, and thus $X_n = X$. But then the Baire category theorem says that $\bigcup X_n$ is nowhere dense, which is impossible since $X = \bigcup X_n$.
\end{proof}

\begin{theorem}
    Suppose $T: L^p(X) \to L^q(X)$ is an operator such that if $\{ f_n \}$ is a sequence converging pointwise almost everywhere to a function $f$, then $Tf_n$ converges almost every to $Tf$. Then $T$ is a bounded operator.
\end{theorem}
\begin{proof}
    We apply the closed graph theorem. Suppose $\{ f_n \}$ is a sequence converging to some function $f$ in the $L^p$ norm, and $Tf_n$ converges to $g$ in the $L^q$ norm. Then $\{ f_n \}$ has a subsequence $\{ f_{n_k} \}$ converging almost everywhere to $f$, so $Tf_{n_k}$ converges almost everywhere to $Tf$. But this implies that $Tf = g$, since $\{ Tf_{n_k} \}$ must have a subsequence converging almost everywhere to $g$.
\end{proof}

\section{Quotient Spaces}

Let $X$ be a norm space, and $Y$ a closed subspace. We can then associate a natural norm on the quotient space $X/Y$. We define
%
\[ \| x + Y \| = \inf_{y \in Y} \| x + y \|. \]
%
If $X$ is a Banach space, then $X/Y$ is a Banach space. More generally, being a Banach space is a {\it three space property}, in the sense that if any two of $\{ X, Y, X/Y \}$ are Banach spaces, then the third space is also a Banach space. The quotient map $\pi: X \to X/Y$ is continuous and surjective, since $\| x + Y \| \leq \| x \|$ for each $x \in X$. The open mapping theorem thus implies that $\pi$ is an open map.

\begin{example}
    The space $c_0$ is a closed subspace of $l_\infty$, so we can consider the quotient space $l_\infty / c_0$. For any sequence $a \in l_\infty$, and any integer $n$, we have $a + c_0 = a^n + c_0$, where $a^n_k = 0$ for $k < n$, and $a^n_k = a_k$ for $k \geq n$. Thus we have
    %
    \[ \| a + c_0 \| \leq \inf \| a^n \|_\infty = \limsup_{n \to \infty} |a_n|. \]
    %
    Conversely, for any sequence $a$, and for any sequence $b \in c_0$, we have
    %
    \[ \| a + b \| = \sup |a_n + b_n| \geq \limsup |a_n + b_n| = \limsup |a_n|. \]
    %
    Thus we have $\| a + c_0 \| = \limsup |a_n|$.
\end{example}

\begin{theorem}
    Let $T: X \to Y$ be a (not necessarily continuous) linear map, and let $Z$ be a closed subspace of the kernel of $T$. Then there is a unique map $S: X/Z \to Y$ such that $T = S \circ \pi$, $T$ is bounded if and only if $S$ is bounded, and $\| T \| = \| S \|$.
\end{theorem}
\begin{proof}
    The existence and uniqueness of $S$ is provided by linear algebra. If $T$ is bounded, then
    %
    \[ \| S(x + Z) \| = \| Tx \| \lesssim \| x \|, \]
    %
    Thus
    %
    \[ \| S(x + Z) \| \lesssim \inf_{z \in Z} \| x + z \| = \| x + Z \|, \]
    %
    which shows $S$ is bounded and $\| S \| \leq \| T \|$. Conversely, if $S$ is bounded, then $T = S \circ \pi$ is bounded, and $\| T \| \leq \| S \| \| \pi \| = \| S \|$.
\end{proof}

This theorem leads to a `first isomorphism theorem' for Banach spaces. If $T: X \to Y$ is a bounded linear map, and $K$ is the kernel then the induced map $S: X/K \to Y$ is an isomorphism. We conclude with an application to the theory of finite rank operators.

\begin{theorem}
    Let $T: X \to Y$ be a finite rank linear map from $X$ to $Y$. Then $T$ is bounded if and only if it's kernel is closed in $X$.
\end{theorem}
\begin{proof}
    Suppose the kernel $K$ of $T$ is closed. Then $X/K$ is a finite dimensional norm space, and so $S: X/K \to Y$ is automatically continuous, hence $T$ is continuous.
\end{proof}

If $T$ is a linear functional, a much stronger statement can be made.

\begin{theorem}
    Let $T: X \to \mathbf{K}$ be a non-zero linear functional. Then $T$ is bounded if and only if it's kernel $K$ is not dense in $X$.
\end{theorem}
\begin{proof}
    The kernel $K$ of $T$ has codimension one in $X$. If $K$ is not dense in $X$, then $\overline{K}$ is a closed, proper subspace of $X$, which implies $K = \overline{K}$, and thus the last theorem implies that $T$ is continuous.
\end{proof}

Another application deals with sums of closed subspaces. If $M$ and $N$ are closed subspaces of a norm space $X$, then $M + N$ need not be closed. But in one special case, the sum is closed.

\begin{theorem}
    Let $M$ be a finite dimensional subspace of $X$, and let $N$ be a closed subspace. Then $M + N$ is closed.
\end{theorem}
\begin{proof}
    Since $N$ is closed, we can consider the quotient space $X/N$. For the resultant projection map $\pi: X \to X/N$, $\pi(M + N) = \pi(M)$ is a finite dimensional subspace of $X/N$, hence closed. By continuity, this implies that $\pi^{-1}(\pi(M)) = M + N$ is closed.
\end{proof}

\chapter{Hilbert Spaces}

\section{Real Inner Product Spaces}

Multivariate calculus tells us how to perform analysis on $\mathbf{R}^n$. Hilbert spaces are those spaces closest to $\mathbf{R}^n$. It is here that we may measure angles and distance, which give rise to the topological properties of the space. Recall that in $\mathbf{R}^2$, the angle between two vectors $v$ and $w$ in $\mathbf{R}^2$ is connected by the scalar product identity
%
\[ v_1w_1 + v_2w_2 = |v||w| \cos \theta \]
%
Where $|v|^2 = v_1^2 + v_2^2$ and $|w|^2 = w_1^2 + w_2^2$ can be measured by Pythagoras' identity. One way to see why this equation is true is to note that the projection
%
\[ P_w(v) = \frac{v_1w_1 + v_2w_2}{|w|^2} w \]
%
which is the projection of $v$ onto the line spanned by $w$, forms a right angled triangle with $v$ as the hypotenuse, and the equation is just given by the trigonometric identities since the length of $P_w(v)$ is $|v| \cos \theta$. Because of how useful this operation is, in higher dimensions, we define the {\it scalar product} in $\RR^n$ by the equation $v \cdot w = v_1w_1 + \dots + v_nw_n$. The length of a vector $v$ is then defined by $|v|^2 = v \cdot v$, and the angle $\theta$ between the vectors by the equation $v \cdot w = |v| |w| \cos \theta$, where the equation makes sense because of the Cauchy Schwarz inequality, which says $|v \cdot w| \leq |v||w|$.

Inner product spaces generalize the scalar product on $\mathbf{R}^n$ to more general vector spaces. An inner product associates to any two vectors $v$ and $w$ a real number $\langle v, w \rangle$, which is {\it bilinear}, in the sense that it is linear in each variable once the other vector is fixed, {\it symmetric}, in the sense that $\langle v, w \rangle = \langle w, v \rangle$, and is {\it positive definite}, in the sense that $\langle v, v \rangle > 0$ for $v \neq 0$.

\begin{example}
    The spaces $\mathbf{R}^n$ are inner product spaces under the scalar product
    %
    \[ v \cdot w = v_1w_1 + \dots + v_nw_n \]
\end{example}

\begin{example}
    The space $L^2(\mathbf{R})$ of square summable real-valued functions on the real line is an inner product space under the inner product
    %
    \[ \langle f, g \rangle_{L^2(\mathbf{R})} = \int_{-\infty}^\infty f(x) g(x)\ dx \]
    %
    The H\"{o}lder inequality guarantees that this definition is finite for all $f,g \in L^2(\mathbf{R})$. More generally, the space $L^2(X)$ of square summable functions on any measure space $X$ is an inner product space under the inner product
    %
    \[ \langle f, g \rangle_{L^2(X)} = \int_X f(x) g(x)\ dx \]
    %
    The most useful applications are where we let $X$ denote a subregion of $\mathbf{R}^n$.
\end{example}

\begin{example}
    The space $l^2(\mathbf{Z})$ of sequences $f: \mathbf{Z} \to \mathbf{R}$ such that
    %
    \[ \sum_{n \in \mathbf{Z}} f(n)^2 < \infty \]
    %
    forms an inner product space under the inner product
    %
    \[ \langle f, g \rangle_{l^2(\mathbf{Z})} = \sum_{n \in \mathbf{Z}} f(n)g(n) \]
    %
    where the discrete version of the H\"{o}lder inequality guarantees that the sum converges absolutely.
\end{example}

In any inner product space, we can measure the length of a vector $v$, generalizing the length in Euclidean space, by the equation $\| v \|^2 = \langle v, v \rangle$ (we note this definition is linear in the sense that $\| \lambda v \| = |\lambda| \| v \|$. This gives the inner product space a metric space structure, if we define the distance between two vectors $v$ and $w$ to be the length of the vector $w - v$, once we prove that the triangle inequality holds. First, we must prove the most fundamental theorem in analysis, the Cauchy Schwartz inequality. This is just a generalization of the Pythagorean theorem, which we now state in the appropriate terms of inner product space theory. We say two vectors $v$ and $w$ are {\bf orthogonal}, or {\bf perpendicular}, if $\langle v, w \rangle = 0$. Note that in the case of two dimensions, if $v$ and $w$ are both nonzero, this means $\cos \theta = 0$, so $\theta$ is a quarter angle clockwise or anticlockwise.

\begin{theorem}[Pythagorean Theorem]
    If $v$ and $w$ are orthogonal,
    %
    \[ \| v + w \|^2 = \| v \|^2 + \| w \|^2 \]
\end{theorem}
\begin{proof}
    We just calculate that
    %
    \[ \| v + w \|^2 = \langle v + w, v + w \rangle = \| v \|^2 + 2 \langle v, w \rangle + \| w \|^2 \]
    %
    and we then just note that $\langle v, w \rangle = 0$.
\end{proof}

\begin{corollary}
    The projection
    %
    \[ P_w(v) = \frac{\langle v, w \rangle}{|w|^2} w \]
    %
    is the closest vector in the line spanned by $w$ to the vector $v$.
\end{corollary}
\begin{proof}
    The vector $v - P_w(v)$ is orthogonal to any scalar multiple of $w$, and so the Pythagorean theorem guarantees that for any $\lambda \in \mathbf{R}$,
    %
    \[ \| v - \lambda w \|^2 = \| v - P_w(v) + P_w(v) - \lambda w \|^2 = \| v - P_w(v) \|^2 + \| P_w(v) - \lambda w \|^2 \geq \| v - P_w(v) \|^2 \]
    %
    with equality if and only if $P_w(v) - \lambda w = 0$.
\end{proof}

\begin{theorem}[Cauchy Schwartz Inequality]
    For any vectors $v$ and $w$ in an inner product space,
    %
    \[ |\langle v, w \rangle| \leq |v||w| \]
\end{theorem}
\begin{proof}
    The idea of this proof, in two dimensions, is just a trivial consequence of the Pythagorean theorem: the hypotenuse is the longest side of a right angled triangle. We just rephrase this in terms of an inner product space. In two dimensions, we saw that the quantity
    %
    \[ P_w(v) = \frac{\langle v, w \rangle}{|w|^2} w \]
    %
    gives the projection of $v$ onto $w$. In two dimensions, we saw this formed a right angled triangle with $v$ as the hypotenuse, and similarly, in a general inner product space, we find that $w$ is orthogonal to the vector $v - P_w(v)$. The Pythagorean theorem guarantees that
    %
    \[ |v|^2 = \| P_w(v) \|^2 + \| v - P_w(v) \|^2 \geq \| P_w(v) \|^2 = \frac{\langle v, w \rangle^2}{|w|^2} \]
    %
    Rearranging the equation and taking square roots gives the inequality.
\end{proof}

Note that this theorem justifies the fact that we can define the angle between two vectors by the equation
%
\[ \langle v, w \rangle = \|v\| \|w\| \cos \theta \]
%
The triangle inequality, giving the inner product space a metric space structure, is an easy consequence of the Cauchy Schwartz inequality.

\begin{theorem}[Triangle Inequality]
    For any vectors $v$ and $w$,
    %
    \[ \| v + w \| \leq \|v\| + \|w\| \]
\end{theorem}
\begin{proof}
    We just calculate
    %
    \[ \| v + w \|^2 = \| v \|^2 + 2 \langle v, w \rangle + \| w \|^2 \leq \| v \|^2 + 2 \|v\| \|w\| + \|w\|^2 = \left(\|v \| + \|w\| \right)^2 \]
    %
    and then we take square roots.
\end{proof}

We can now think of an inner product space as a kind of `infinite dimensional' Euclidean space, with a topology given by the length function, and induced by the triangle inequality. If the metric structure is {\it complete}, then we say that the inner product space is a {\bf Hilbert space}.

\section{Complex Inner Product Spaces}

To generalize inner product spaces to the complex domain, we turn to our most basic example, $\mathbf{R}^2$, and reinterpret it as the complex plane $\mathbf{C}$. Here, complex conjugation plays a role in defining the distance function, removing the rotation properties of a complex number. The length of a particular complex number $z$ is given by the equation $|z|^2 = z \overline{z}$. Over $\mathbf{C}^n$, we therefore introduce the scalar {\it hermitian product}
%
\[ v \cdot w = v_1 \overline{w_1} + \dots + v_n \overline{w_n} \]
%
this equation agrees with the standard scalar product when $v$ and $w$ have real coordinates, and also means that the length defined by the equation $|v|^2 = v \cdot v$ agrees with the length given in $\mathbf{R}^{2n}$. Unlike the scalar product over the real numbers, this equation is not bilinear, but instead {\it sesquilinear} (a word meaning `one and a half' in Greek), which is linear in the first argument, but {\it antilinear} in the second, so that $(v \cdot (\lambda w + \gamma u) = \overline{\lambda} (v \cdot w) + \overline{\gamma} (v \cdot u)$. In general, we define a {\bf Hermitian inner product} $(\cdot, \cdot)$ over a complex vector space to be a sesquilinear symmetric map which is positive definite. A Hermitian inner product space is just a space with a Hermitian inner product. A complex Hilbert space is just a Hermitian inner product space whose norm gives a complete metric space structure.

\begin{example}
    If we allow for the existence of complex valued functions and sequences, then $L^2(X)$ and $l^2(\mathbf{Z})$ are Hermitian inner product spaces under the inner product
    %
    \[ (f,g) = \int_X f(x) \overline{g}(x)\ dx \]
    %
    \[ (f,g) = \sum_{n \in \mathbf{Z}} f(n) \overline{g(n)} \]
    %
    The conjugation is all that is needed to generalize these spaces to complex functions.
\end{example}

\begin{example}
    We define the {\bf Hardy space} $H^2(\mathbf{D}^\circ)$ to be all holomorphic functions $f: \mathbf{D}^\circ \to \mathbf{C}$ with
    %
    \[ \sup_{0 \leq r < 1} \left( \frac{1}{2\pi} \int_{-\pi}^\pi f(re^{it})^2 \right)^{1/2} \ dt < \infty \]
    %
    We define this to be the norm on the space, induced from the inner product
    %
    \[ (f,g) = \sup_{0 \leq r < 1} \frac{1}{2\pi} \int_{-\pi}^\pi f(re^{it}) \overline{g(re^{it})} \]
    %
    which is finite by the Cauchy Schwartz inequality on $L^2[-\pi,\pi]$, and easily verified to by sesquilinear and positive definite. In this definition, it is difficult to write down what a real version of this space would look like. All these spaces are Hilbert spaces.
\end{example}

If $W$ is a closed subspace of a Hilbert space $H$, in the sense that it is a subspace, and it is a closed set under the topology induced by the metric structure of the hermitian product, then $W$ is also a Hilbert space, because a closed subset of a complete metric space is also complete. Conversely, a complete subspace is also closed.

\begin{example}
    If $D$ is an open subset of the complex plane, the space $L_2^a(D)$ of holomorphic square-integrable functions is a Hilbert space. If $B_r(x)$ is a ball of radius $r$ centered at $x$ whose closure is contained in $D$, and $f$ is analytic in $D$, then the mean value property guarantees that
    %
    \[ f(x) = \fint_{B_r(x)} f(y)\ dy \]
    %
    It follows from Cauchy Schwarz that if $0 < r < \text{dist}(x,\partial D)$, then
    %
    \[ |f(x)| \leq \frac{1}{r\sqrt{\pi}} \|f \|_2 \]
    %
    In particular, this implies that if a sequence of analytic functions in $L_2^a(D)$ converges in $L^2$ norm to some function, then they must also converge locally uniformly, so the resultant function is analytic. It follows that $L_2^a(D)$ is closed in $L^2(D)$, and so $L_2^a(D)$ is a Hilbert space.
\end{example}

The definition of a Hermitian inner product appears to avoid geometric intuition, but I will now attempt to explain why this definition is geometrically intuitive. As we have seen in real spaces, the normalized inner product
%
\[ \frac{\langle v, w \rangle}{|w|^2} \]
%
is the number we must scale $w$ by to obtain the projection of $v$ onto the line spanned by the vector $w$. In complex vector spaces, the space spanned by a single vector is a {\it complex line}, or a one dimensional plane. If we set
%
\[ \frac{\langle v, w \rangle}{|w|^2} \]
%
to be the value we must scale $w$ by to obtain the projection of $v$ onto the plane spanned by $w$, then we might have to rotate $w$, hence $\langle v, w \rangle$ might have a rotational part. Indeed, this is exactly the correct geometrical definition of the inner product, if we already have a well defined projection operation. The Pythagorean theorem, Cauchy Schwartz inequality,and triangle inequality goes through for Hermitian product spaces almost unperturbed, except that
%
\[ \| v + w \|^2 = \| v \|^2 + 2 \Re(\langle v, w \rangle) + \| w \|^2 \]
%
The reason that everything for real inner product spaces `seems' to work for Hermitian products, except for more advanced theorems which require we restrict ourselves only to Hermitian products, is one of the main reasons we push through the initial confusion.

Another approach to understanding Hermitian inner products is to use the fact that we can construct a Hermitian inner product from a real inner product. Recall that a complex vector space $V$ is just a real vector space with a fixed twisting linear map $J: V \to V$ with $J^2 = -1$. If we already have a {\it real} inner product $\langle \cdot, \cdot \rangle$ on $V$, then our hope is to extend the real inner product to a complex inner product $(\cdot,\cdot)$, where the real part of the Hermitian inner product is our original real product, so that
%
\[ \Re((v,w)) = \langle v, w \rangle \]
%
By geometric intuition, it is reasonable to assume that $J$ is a twist by a right angle, so that $\langle v, Jv \rangle = 0$ for all vectors $v$, and that $J$ is an isometry with respect to the inner product. It then follows that
%
\[ \langle v, Jw \rangle = \langle Jv, J^2w \rangle = \langle -Jv, w \rangle \]
%
so we cannot hope for the extension to be complex linear, only sesquilinear. If $(\cdot, \cdot)$ is a sesquilinear form, then we calculate
%
\[ \Im (v,w) = \Re(-i (v,w)) = \Re (v,iw) = \langle v, iw \rangle \]
%
so the {\it only} way to define $(\cdot, \cdot)$ properly is as
%
\[ (v,w) = \langle v, w \rangle + i \langle v, iw \rangle \]
%
The fact that $(v,w)$ measures the projection of $v$ onto the complex line containing $w$ appears again, because over the real numbers, the complex line is spanned by $w$ and $Jw$, and the value above, once normalized, gives the real projection onto this two dimensional plane.

If $V$ is any real inner product space, then we can always complexify it by embedding $V$ in $V \oplus V$ by the trivial map $v \mapsto (v,0)$, and defining a real inner product $\langle (v_1,v_2), (w_1,w_2) \rangle = \langle v_1,w_1 \rangle + \langle v_2,w_2 \rangle$. We can then add an artifical twisting map $J(v,w) = (-w,v)$ gives $V \oplus V$ a complex structure, and since it satisfies the properties of the discussion above, the real inner product extends to give a complex inner product. The space with all these gadgets attached is called the {\bf complexification} of $V$, denoted $V^\mathbf{C}$. If $V$ is a Hilbert space, then so $V^\mathbf{C}$ will also be a Hilbert space. Because of this fact, we can prove all the theorems in the sequel for complex inner spaces, and then use the embedding to obtain the result for real inner product spaces. Unless otherwise mentioned, all inner product spaces in the sequel will be over the complex numbers.

\section{Orthogonality}

Because we can measure angles in a Hilbert space, we can also measure right angles. Two vectors $v$ and $w$ are {\bf orthogonal}, denoted $v \perp w$, if $(v,w) = 0$. This is a symmetric relation, since if $(v,w) = 0$,
%
\[ (w,v) = \overline{(v,w)} = \overline{0} = 0 \]
%
The theory of right triangles essentially manifests in the theory of equations involving two to three vectors, as we have already seen in the case of the Pythagorean identity.

\begin{theorem}[Parallelogram Law]
    If $x,y$ lie in a Hilbert space, then
    %
    \[ \| x + y \|^2 + \| x - y \|^2 = 2 \| x \|^2 + 2 \| y \|^2 \]
    %
    which is obvious viewing $0, x, y$, and $x + y$ as vertices of a parallelogram.
\end{theorem}
\begin{proof}
    We simply calculate
    %
    \begin{align*}
        \| x + y \|^2 + \| x - y \|^2 &= \langle x + y, x + y \rangle + \langle x - y, x - y \rangle\\
        &= 2 \| x \|^2 + 2 \| y \|^2 + \langle x, y \rangle + \langle y, x \rangle - \langle x, y \rangle - \langle y, x \rangle\\
        &= 2 \| x \|^2 + 2 \| y \|^2
    \end{align*}
    %
    and this shows the theorem is true.
\end{proof}

A simple extension of this identity implies that if $e_1, \dots, e_n$ are an {\it orthonormal} family of vectors in an inner product space, in the sense that they are pairwise orthogonal, and $\| e_i \| = 1$ for all $i$, then
%
\[ \| \sum a_ie_i \|^2 = \sum |a_i|^2 \]
%
Using slightly more analysis, and using the fact that we are working in a Hilbert space, we can extend this to an infinite sequence of vectors $e_1, e_2, \dots$, provided that $\sum |a_i|^2 < \infty$, because then the partial sums of $\sum_{k = 1}^\infty a_ie_i$ are Cauchy, hence the partial sums converge in $L^2$ to some vector, and it is easy to see the norm must be the infinite sum $\sum |a_i|^2$. In particular, for any vector $v$, $v - \sum (v,e_i) e_i$ is orthogonal to the $e_i$, and so
%
\[ \| v \|^2 = \| v - \sum (v,e_i) e_i \|^2 +\sum |(v,e_i)|^2 \geq \sum |(v,e_i)|^2 \]
%
This is known as {\it Bessel's inequality}. We may also conclude that
%
\[ \left\| \sum_{i = 1}^\infty (v,e_i) e_i \right\|^2 = \sum_{i = 1}^\infty |(v,e_i)|^2 \]
%
If the left sum converges to $v$ in the $L^2$ norm, then we obtain
%
\[ \| v \|^2 = \sum_{i = 1}^\infty |(v,e_i)|^2 \]
%
this equation is known as {\it Parsevel's identity}. Regardless, we can always take $n \to \infty$ in Bessel's inequality to obtain an infinite dimensional version of the inequality. We now give a series of conditions guaranteeing that the left sum does actually converge to $v$ in the $L^2$ norm.

\begin{theorem}
    The following properties of an orthonormal sequence $e_1, e_2, \dots$ are equivalent:
    %
    \begin{itemize}
        \item[(a)] For any vector $v \in H$, $v = \sum_{i = 1}^\infty (v,e_i) e_i$.
        \item[(b)] The finite linear combinations of the $e_i$ are dense in $H$.
        \item[(c)] If $(v,e_i) = 0$ for all $i$, then $v = 0$.
        \item[(d)] Parseval's identity holds.
    \end{itemize}
    %
    and then we say the sequence is an {\bf Orthonormal basis} for the space.
\end{theorem}
\begin{proof}
    (a) clearly implies (b). Assuming (b), note that if $(v,e_i) = 0$ for all $i$, then in particular, $(v,w) = 0$ for all $w$ which can be expressed as finite linear combinations of the $e_i$. But we may choose $w_n$ with $w_n \to v$ in $L^2$, and then since $|(v,w_n - v)| \leq \|v\|\|w_n - v\| \to 0$, we conclude that $0 = (v,w_n) = (v,v) + (v,w_n-v)$ converges to $(v,v)$, so $(v,v) = 0$, and hence $v = 0$. If $v$ is given, then Bessel's inequality ensures that
    %
    \[ \sum_{i = 1}^\infty |(v,e_i)|^2 \leq \| v \| < \infty \]
    %
    and so the partial sums of the sequence $\sum_{i = 1}^\infty |(v,e_i)|^2$ are Cauchy, and because we are working in a Hilbert space, converge to some vector $w$. Now $v - w$ is orthogonal to $e_i$ for each $i$, since
    %
    \[ (w,e_i) = \lim \sum_{j = 1}^n (v,e_j)(e_j,e_i) = (v,e_i) \]
    %
    and therefore $v = w$, proving (a). (a) implies (d) is trivial, and if (d) holds, then Pythagoras' identity implies that
    %
    \[ \| v - \sum_{i = 1}^n (v,e_i) e_i \|^2 = \sum_{i = n+1}^\infty |(v,e_i)|^2 \to 0 \]
    %
    so (a) holds.
\end{proof}

A topological space is {\it separable} if it has a countable dense subset. If $H$ is a separable Hilbert space, then by performing the Gram Schmidt process to a countable dense subset, one finds an orthogonal sequence, which is easily verified to be an orthonormal basis for the Hilbert space. Thus we have proved that {\it every separable Hilbert space has a orthonormal basis}. With some more work on the geometry of Hilbert spaces, and the axiom of choice, we can show that every separable Hilbert space has an orthonormal basis, if we allow the bases to be uncountable. Before we do this, however, we consider some applications.

\section{Fourier Series in $L^2$}

First, let's consider the theory of Fourier series, which is the scenario where the abstract theory of Hilbert spaces found it's place. Since $[-\pi,\pi]$ is a finite measure space, the space $L^2[-\pi,\pi]$ is contained within $L^1[-\pi,\pi]$, and so if $f \in L^2[-\pi,\pi]$, we can consider it's Fourier series
%
\[ \widehat{f}(n) = \fint_{-\pi}^\pi f(x) e^{-nix}\ dx \]
%
Note that if we define $e_n(x) = e^{nix}$, then $\widehat{f}(n) = (f,e_n)$, and
%
\[ (e_n,e_m) = \fint_{-\pi}^\pi e^{(n-m)ix} = \begin{cases} 1 & n = m \\ 0 & n \neq m \end{cases} \]
%
It follows that $e_n$ is an orthogonal sequence in $L^2[-\pi,\pi]$. Using the basic fact from Fourier analysis that for any continuous function $f \in L^2[-\pi,\pi]$ with $f(-\pi) = f(\pi)$, the sum
%
\[ f(x) = \sum \widehat{f}(n) e^{nix} \]
%
converges uniformly, and the fact that continuous functions are dense in $L^2[-\pi,\pi]$, we conclude that the exponentials $e_n$ are an orthonormal basis for $L^2[-\pi,\pi]$. As a result, we obtain the classic Parseval's inequality
%
\[ \fint_{-\pi}^\pi |f(x)|^2\ dx = \sum |\widehat{f}(n)|^2 \]
%
and the $L^2$ convergence of the partial sums $\sum_{k = -N}^N \widehat{f}(k) e^{kix}$ to $f(x)$.

\section{Optimizing over Convex Sets}

One of the biggest advantages of Hilbert spaces is that we have powerful theorems about the structure of convex sets in the space. A {\bf convex set} in a vector space is a set $C$ such that the line between any two points is contained in $C$. That is, for any $x, y \in C$, and $\lambda \in [0,1]$, $\lambda x + (1 - \lambda) y \in C$.

\begin{theorem}
    If $C$ is any closed convex subset of a Hilbert space, and $x \in H$ is arbitrary, then there is a unique point $y \in C$ which minimizes $\| x - y \|$, over all choices of $y$ in the convex set.
\end{theorem}
\begin{proof}
    Since $C - x$ is a convex set, we may assume without loss of generality that $x = 0$, and so we are trying to minimize the norm over $C$. If $x,y \in C$ minimize this norm, with $\| x \| = \| y \| = d$, then
    %
    \[ d \leq \left\| \frac{x + y}{2} \right\| \leq \frac{1}{2}(d + d) = \frac{d}{2} \]
    %
    Hence the parallogram law implies
    %
    \[ d^2 = \left\| \frac{x + y}{2} \right\|^2 = d^2 - \left\| \frac{x - y}{2} \right\|^2 \]
    %
    which implies $x = y$. We know that there must a sequence $x_i \in C$ with $\| x_i \| \to d$ monotonically, and then the parallelogram law implies that if $\| x_n \|, \| x_m \| \leq d + \varepsilon$, then since $(x_n + x_m)/2 \in C$, then
    %
    \[ \left\| \frac{x_n - x_m}{2} \right\|^2 = \frac{1}{2} \| x_n \|^2 + \frac{1}{2} \| x_m \|^2 - \left\| \frac{x_n + x_m}{2} \right\|^2 \leq 2 d \varepsilon + \varepsilon^2 \to 0 \]
    %
    hence $\{ x_n \}$ is a Cauchy sequence, and therefore converges to a point $x$, which satisfies $\| x \| = d$.
\end{proof}

Linear subspaces of a vector spaces are trivially verified to be convex sets, and we find that

\begin{theorem}
    If $V$ is a closed linear subspace of a Hilbert space $H$, and $v \in V$ is the closest point to a fixed point $x$, then $v - x \perp V$, in the sense that $v - x \perp w$ for any $w \in V$. Conversely, if $v - x \perp V$, then $v$ is the closest point to $x$.
\end{theorem}
\begin{proof}
    If $\langle v - x, w \rangle \neq 0$, then by multiplying $w$ by a scalar value, we may assume that $\langle v - x, w \rangle < 0$, and $\| w \| = 1$, and then for $\alpha > 0$,
    %
    \[ \| \alpha w + v - x \|^2 = \| v - x \|^2 + 2 \alpha \Re \langle w, v - x \rangle + |\alpha|^2 \| w \|^2  \]
    %
    and so as $\alpha \to 0$, we find that eventually $\| \alpha w + v - x \|^2 < \| v - x \|^2$. Conversely, if $\langle v - x, w \rangle = 0$ for all $w$, then for any $w$,
    %
    \[ \| w - x \|^2 = \| (w - v) + (v - x) \|^2 = \| w - v \|^2 + \| v - x \|^2 \geq \| v - x \|^2 \]
    %
    so $v$ minimizes the distance to $x$ among all vectors in $V$.
\end{proof}

For any subset $X$ of an inner product space $V$, define $X^\perp$ to be the set of vectors $v$ such that $\langle x, v \rangle = 0$ for all $x \in X$. It is clearly a closed linear subspace of $V$. If $V$ is a closed linear subspaces of a Hilbert space $H$, then for any $x \in H$, there is a unique $y \in V$ with $y - x \perp V$, and we may define a map $P: H \to V$ such that $Px = y$, which we call the {\bf orthogonal projection} of $H$ onto $V$.

\begin{theorem}
    If $P: H \to V$ is defined as above, then
    %
    \begin{enumerate}
        \item $P$ is a linear map.
        \item $\| Px \| \leq \| x \|$.
        \item $P^2 = P$.
        \item $\ker P = V^\perp$, and $\im P = V$.
    \end{enumerate}
\end{theorem}
\begin{proof}
    If $x_0 - y_0 \perp V$ and $x_1 - y_1 \perp V$, then $(\alpha x_0 - \alpha y_0) + (\beta x_1 - \beta y_1) \perp V$, hence $P(\alpha x_0 + \beta x_1) = \alpha y_0 + \beta y_1$, and so $P$ is a linear map. It is obvious that $P$ is a contraction, because $\| x \|^2 = \| Px \|^2 + \| x - Px \|^2$, and $\| x - Px \|^2 \geq 0$. For any $y \in V$, $y - y \perp V$, hence $Py = y$, and so $P^2 = P$. Finally, $Py = 0$ if and only if $y \perp V$, in which case we find $y \in V^\perp$.
\end{proof}

\begin{corollary}
    For any closed subspace $V$ of a Hilbert space $H$, $(V^\perp)^\perp = V$. If $X$ is an arbitrary subset of a Hilbert space $H$, then $(X^\perp)^\perp = \overline{\text{span}}(X)$ is the closure of the linear span of the elements of $X$.
\end{corollary}
\begin{proof}
    If $P$ is the orthogonal projection of $H$ onto $V$, and $Q$ is the orthogonal projection of $H$ onto $V^\perp$, then $Px + Qx = x$ for all $x \in H$, and so the kernel of $Q$ is exactly the set of vectors $x$ for which $Px = x$, from which we conclude that $x \in V$. If $X$ is arbitrary, then $\overline{\text{span}}(X)$ is contained in $(X^\perp)^\perp$, because $(X^\perp)^\perp$ is a closed subspace of $H$ containing all elements in $X$. Conversely, we conclude that $X^\perp = \overline{\text{span}}(X)^\perp = X^\perp$, because if $\langle y, x \rangle = 0$ for all $x \in X$, then surely $\langle y, \sum \alpha_i x_i \rangle = \sum \alpha_i \langle y, x_i \rangle = 0$, so $y \perp \text{span}(X)$, and if $v_i \to v$, with $v_i \in \text{span}(X)$, then by continuity of the inner product we find that $0 = \langle y, v_i \rangle \to \langle y, v \rangle$, hence $\langle y, v \rangle = 0$, and so $y \perp \overline{\text{span}}(X)$.
\end{proof}

\begin{corollary}
    A subspace $V$ is dense in $H$ if and only if $V^\perp = 0$.
\end{corollary}
\begin{proof}
    For then the closure of $V$ is $(V^\perp)^\perp = 0^\perp = H$.
\end{proof}

Given two Hilbert spaces $H_0$ and $H_1$, we can define a canonical Hilbert space structure on $H_0 \oplus H_1$ by letting $\langle x_0 + x_1, y_0 + y_1 \rangle = \langle x_0, y_0 \rangle + \langle x_1, y_1 \rangle$. This is exactly the product topology, because $x_i + y_i \to x + y$ if and only if $x_i \to x$, and $y_i \to y$. The orthogonal projection of $H$ onto $V$ is incredibly important to the theory of Hilbert spaces, because it gives us a decomposition of $H$ into the direct sum of $V$ and $V^\perp$. The Pythagorean theorem tells us that for any $v \in V$ and $w \in V^\perp$,
%
\[ \| v + w \|^2 = \| v \|^2 + \| w \|^2 \]
%
and so the map $x \mapsto Px + (1 - P)x$ is actually an {\bf isometry} between vector spaces (it preserves the inner product). This is the basic notion of equivalence between various Hilbert spaces, and so we really can think of $H$ as being composed of the two closed subspaces $V$ and $V^\perp$.

\section{The Riesz Representation Theorem}

The Riesz representation theorem shows that the isometry class of Hilbert spaces is uniquely identified by the dimension of a {\bf Schauder basis} of the underlying Hilbert space. It is incredibly important, because it shows that to analyze a Hilbert space $H$ with a Schauder basis of cardinality $\mathfrak{a}$, we need only analyze the Hilbert space $l_2(\mathfrak{a})$.

Many problems in functional analysis can be reduced to the analysis of certain linear functions between vector spaces. The easiest linear functions to analyze are the {\bf linear functionals}, which are linear maps $f: V \to K$, where $K$ is the base field of the vector space $V$. We will say a linear functions $f: V \to K$ is {\bf bounded} if there exists $M > 0$ such that $|f(x)| \leq M \| x \|$ for all $x \in V$. The bounded linear maps are exactly the continuous linear maps (and to verify continuity, we need only verify continuity at zero, so bounded linear maps are essentially `uniformly continuous'). If $f$ and $g$ are continous, then $\lambda f + \gamma g$ is continuous, and so the set $V^*$ of bounded linear functionals on $V$ is a vector space, called the {\bf dual space} of $V$. On the dual space of a norm space, we may define a dual norm by letting $\| f \|$ be the smallest number $M$ such that $|f(x)| \leq M \| x \|$ holds for all $x$. Since $|f(x) + g(x)| \leq |f(x)| + |g(x)|$, we find $\| f + g \| \leq \| f \| + \| g \|$, and it is easy to verify that $\| \lambda f \| = |\lambda| \| f \|$.

Given a Hilbert space $H$, we can use the inner product structure on $H$ to transport vectors in $H$ to functionals in $H^*$. That is, for a given $x \in H$, we define $x^* \in H^*$ to be the functional $x^*(y) = \langle x, y \rangle$. The Cauchy-Schwarz inequality tells us that $\| x^* \| = \| x \|$, so the dual map is an isometry. The Riesz representation theorem tells us that this isometry is actually invertible.

\begin{theorem}[Riesz]
    For any bounded linear function $f \in H^*$, there exists a unique vector $x \in H$ such that $f(y) = \langle x, y \rangle$ for all $y \in H$.
\end{theorem}
\begin{proof}
    Assume without loss of generality that $f \neq 0$. Let $V$ be the kernel of $f$. Then $V$ is a closed linear subspace of $H$, and $V^\perp$ is one dimensional by the first isomorphism theorem. Consider the orthogonal projection $P$ onto $V^\perp$. If $x$ is a nonzero vector in $V^\perp$, with $\| x \| = 1$ and $f(x) = \gamma$, then for each $y$ there is $\lambda(y)$ for which $f(y) = f(Py) = \lambda(y)$, and what's more, $\langle y, x \rangle = \langle Py, x \rangle = \lambda(y)$, and so we have found a unique $y$ as specified, and $y$ is unique because the dual map is an isometry.
\end{proof}

\begin{example}
    For any measure space $X$, $L_2(X)$ is a Hilbert space, and so for any bounded linear function $F$ on $L_2(X)$, there is a unique $g \in X$ such
    %
    \[ F(f) = \int f \overline{g} \]
    %
    so studying bounded linear functionals reduces to integration theory.
\end{example}

\section{Constructing an Orthonormal Basis}

In the theory of infinite dimension vector spaces, the standard definition of a basis does not lead to a natural theory,but we can obtain a much more elegant decomposition of a space by instead talking about the vector space generated by infinite linear combinations of elements of some infinite set, rather than linear combinations of a finite subset of the set. A {\bf Schauder basis} for a norm space $X$ is a set $Y$ such that any $x \in X$ has a unique expansion as $x = \sum_{y \in Y} \alpha_y y$, where the sum is allowed to be infinite, and converges to $x$ in the limit (if $Y$ is uncountable, this sum is defined to be the limit of the net of all finite sums, which is essentially the integral with respect to the counting measure. If the sum exists, its support must be countable, and therefore this kind of sum isn't really that much more general than absolute convergence). The finite dimensional notion of a basis will be called a {\bf Hamel basis}, and when we refer to `a basis' is a norm space, we will be referring to a Schauder basis.

Given an inner product space $V$, an {\bf orthonormal set} in $V$ is a set $X$ such that $\langle x, y \rangle = \mathbf{I}(x = y)$ for all $x,y \in X$. An {\bf orthonormal basis} for an inner product space $V$ is a Schauder basis for $X$ which forms an orthonormal set. It is clear that an orthonormal set $Y$ is automatically linearly independent, for if $\sum a_y y = 0$, then $0 = \langle \sum a_y y, \sum a_y y \rangle = \sum |a_y|^2$, so $a_y = 0$ for all $y \in Y$.

\begin{lemma}[Bessel's Inequality]
    If $Y$ is an orthonormal set in an inner product space $V$, then for any $x \in V$,
    %
    \[ \sum |\langle x, y \rangle|^2 \leq \| x \|^2 \]
    %
    hence in particular, $\sum |\langle x, y \rangle|^2 < \infty$.
\end{lemma}
\begin{proof}
    For any $x \in V$, and any finite subset $X$ of $Y$, $x - \sum_{y \in X} \langle x, y \rangle y$ is perpendicular to the span of the $X$, because for any $z \in X$,
    %
    \[ \langle x - \sum_{y \in X} \langle x, y \rangle y, z \rangle = \langle x, z \rangle - \sum_{y \in X} \langle x, y \rangle \langle y, z \rangle = \langle x, z \rangle - \langle x, z \rangle = 0 \]
    %
    Hence we may apply Pythagoras' theorem to conclude that
    %
    \[ \| x \|^2 = \left\| \sum_{y \in X} \langle x, y \rangle y \right\|^2 + \left\| x - \sum_{x \in X} \langle x, y \rangle y \right\|^2 \geq \sum_{x \in X} |\langle x, y\rangle|^2 \]
    %
    and the general inequality holds by taking limits of the finite sums.
\end{proof}

Bessel's inequality is an equality only where $x = \sum \langle x, y \rangle y$, a fact that will soon become important. In particular, we shall find that Bessel's inequality is always an equality when $Y$ is an orthonormal basis, in which case we call the equation {\bf Parseval's equality}.

\begin{lemma}
    If $Y$ is an orthonormal set in a Hilbert space $H$, then for any $x$, $\sum \langle x, y \rangle y$ converges in $H$.
\end{lemma}
\begin{proof}
    First, note that for any $x$, since $\sum |\langle x, y \rangle|^2 < \infty$, $\langle x, y \rangle \neq 0$ for only countably many $y \in Y$, so we may assume that $Y$ is countable from the getgo. If we fix some ordering $Y = \{ y_0, y_1, \dots \}$, then the sum
    %
    \[ \sum_{k = 0}^\infty \langle x, y_k \rangle y_k \]
    %
    converges absolutely, in the sense that
    %
    \[ \sum_{k = 0}^\infty \| \langle x, y_k \rangle y_k \|^2 = \sum_{k = 0}^\infty |\langle x, y_k \rangle|^2 < \infty \]
    %
    and it therefore follows from basic estimates analogous to the theory of absolute convergence over the real numbers, that the entire sum converges, to the same value irrespective to the ordering we placed on $Y$.
\end{proof}

The set of all `possibly infinite' linear combinations of an orthonormal set $Y$ forms a closed subspace $V$ of an inner product space (it is the closure of the set of finite sums). For any $x \in H$, we find that the projection of $x$ onto $V$ is $Px = \sum \langle x, y \rangle y$, because for any $z \in Y$,
%
\[ \langle x - \sum \langle x, y \rangle y, z \rangle = \langle x, z \rangle - \sum \langle x, y \rangle \langle y,z \rangle = \langle x, z \rangle - \langle x, z \rangle = 0 \]
%
Now suppose that $Y$ is an orthonormal basis for $H$, the closed span of $Y$ is $H$ itself, and therefore the projection onto the closed span is just the identity map, and so we obtain Parseval's equality as we stated before, because then $x = Px = \sum \langle x, y \rangle y$.

\begin{lemma}
    An orthonormal set of linearly independent elements of a Hilbert space $H$ is an orthonormal basis if and only if the set cannot be expanded to a larger set of linearly independent orthonormal elements.
\end{lemma}
\begin{proof}
    First, assume that $Y$ is an orthonormal basis. If $x \in H$ satisfies $\langle x, y \rangle = 0$ for all $y \in Y$, and $x = \sum a_y y$, then for each $y \in Y$, $0 = \langle x, y \rangle = a_y$, hence $x = 0$. Conversely, suppose that $Y$ is an orthonormal set, and $x \in H$ is linearly independent to $Y$. Then $x - \sum \langle x, y \rangle y \perp Y$, is linearly independent to $Y$, and if we normalize so that $\| x - \sum \langle x, y \rangle y \|^2 = 1$, we find that $Y \cup \{ x - \sum \langle x, y \rangle y \}$ is an orthonormal set larger than $Y$.
\end{proof}

On a finite dimensional Hilbert space $H$, we may repeat this process to construct an orthonormal basis on $H$, and the constructive process above is called {\bf Gram-Schmidt orthogonalization}. If $H$ is not finite dimensional, we have to rely on non-constructive measures. If $\{ Y_\alpha \}$ is a chain of orthonormal sets, then $\bigcup Y_\alpha$ is orthonormal, so we may apply Zorn's lemma to conclude that a maximal orthonormal set exists, and this must be an orthonormal basis. Thus orthonormal bases exist on any Hilbert space, and we can always extend any orthonormal set ot a basis.

\begin{theorem}
    Any two orthonormal bases of a Hilbert space $H$ have the same cardinality.
\end{theorem}
\begin{proof}
    Let $X$ and $Y$ be two bases of a Hilbert space $H$. If $H$ is finite dimensional, this is just standard linear algebra. If $X$ has infinite cardinality $\mathfrak{a}$, and $Y$ has cardinality $\mathfrak{b}$, then for each $x \in X$, the set $Y_x$ of $y \in Y$ with $\langle x, y \rangle \neq 0$ is at most countable. But then $Y' = \bigcup Y_x$ has cardinality at most $\mathfrak{a}$. If $y \in Y' - Y$, then $y = \sum \langle y, x \rangle y = 0$, so $Y' = Y$, and therefore $\mathfrak{a} \geq \mathfrak{b}$. By symmetry, we conclude that $\mathfrak{a} = \mathfrak{b}$.
\end{proof}

We may therefore define the {\bf dimension} of a Hilbert space to be the cardinality of any orthonormal basis. This actually uniquely defines the Hilbert space, for if $H_0$ and $H_1$ are Hilbert spaces with orthonormal bases $X$ and $Y$, then a bijection $f: X \to Y$ induces a map $T: H_0 \to H_1$ defined by $T \left( \sum a_x x \right) = \sum a_x f(x)$. This is well defined, for the sum $\sum a_x x$ converges if and only if $\sum |a_x|^2 < \infty$, and the same criterion shows $\sum a_x f(x)$ converges. By linear independence, and the invertibility of $f$, we find $T$ is injective and surjective, hence an isomorphism. It is actually an isometry, because
%
\[ \left\| \sum a_x x \right\|^2 = \sum |a_x|^2 = \left\| \sum a_x f(x) \right\|^2 \]
%
hence $T$ is an isometry, and so the Hilbert spaces are isomorphic.

\begin{example}
    On $l_2(X)$, we have an orthonormal basis $e_x(y) = \mathbf{I}(x = y)$, because certainly $\langle e_x, e_y \rangle = \mathbf{I}(x = y)$, and for any $f \in l_2(X)$, $f(x) = 0$ for all but countably many $x \in X$, because $\sum \| f(x) \|^2 < \infty$, and so we may assume that $X$ is countable from the getgo, writing $X = \{ x_0, x_1, \dots \}$, thinking of $f$ as a sequence of numbers as in $l_2$, and then
    %
    \[ f - \sum_{k = 0}^\infty \langle f, e_{x_k} \rangle e_{x_k} = \lim_{n \to \infty} (0,0,\dots,x_n, x_{n+1}, \dots) \]
    %
    and since $\sum \| f(x) \|^2 < \infty$, $\lim_{n \to \infty} \sum_{k = n}^\infty \| f(x_k) \|^2 \to 0$. Thus the family $\{ e_x \}$ really is an orthonormal basis for $l_2(X)$. If $X$ has cardinality $\mathfrak{a}$, then $l_2(X)$ has dimension $\mathfrak{a}$, and so we find from the previous discussion that every Hilbert space is isometric to $l_2(X)$ for some set $X$.
\end{example}

\begin{example}
    The space $L_2(\mathbf{T})$ of complex-valued square summable functions with respect to the Haar measure on $\mathbf{T}$ has an orthonormal basis consisting of the functions $z \mapsto z^n$, for $n \in \mathbf{Z}$. This follows because
    %
    \[ \langle z^n, z^m \rangle = \int z^{n-m} = \frac{1}{2 \pi} \int_0^{2\pi} e^{i (n-m)t} = \frac{1}{2\pi} \int_0^{2\pi} \cos((n-m)t) + i \sin((n - m)t) \]
    %
    and so $\langle z^n, z^m \rangle = \delta_{nm}$. It requires a little bit of work to prove that this is actually an orthonormal basis for $L_2(\mathbf{T})$ (see my notes on elementary Harmonic analysis), so that every square summable function $f: \mathbf{T} \to \mathbf{C}$ has a unique Fourier expansion
    %
    \[ f(z) = \sum_{n = 0}^\infty \widehat{f}(n) z^n \]
    %
    where $\widehat{f}(n)$ is the $n$'th Fourier coefficient of $f$, and the convergence is in $L_2(\mathbf{T})$. The fact that $\sum |\widehat{f}(n)|^2 < \infty$ is obvious form our Hilbert space theory, and this shows that the Riemann Lebesgue lemma holds for square summable functions. That is,
    %
    \[ \lim_{n \to \infty} \int f(z) z^{-n} \to 0 \]
    %
    The map $f \mapsto \widehat{f}$ is an isometry between $L_2(\mathbf{T})$ and $l_2$. In the case that $L_2(\mathbf{T})$ is the space of real-valued functions, which is isometric to $L_2[0,1]$ under the association $t \mapsto e^{it}$, and has an orthonormal basis consisting of the maps $\sin(nt)$ and $\cos(mt)$, for $n,m \in \mathbf{Z}$.
\end{example}

\section{Convexity}

A function $f:(a,b) \to \mathbf{R}$ is convex when the line segment between $(a,f(a))$ and $(b,f(b))$ lies above the graph of $f$. The line segment connecting these two points is described by
%
\[ \{ (\lambda a + (1 - \lambda) b, \lambda f(a) + (1 - \lambda) f(b)) : 0 \leq \lambda \leq 1 \} \]
%
and we require that $(\lambda a + (1 - \lambda)b, f(\lambda a + (1 - \lambda)b)$ lies below the corresponding point on the line defined above. This is satisfied exactly when we have a certain inequality:

A function $f:U \to \mathbf{R}$ is {\bf convex} on $(a,b)$ if, for any $a \leq x < y \leq b$, and $0 \leq \lambda \leq 1$, we have
    %
    \begin{equation} \label{convex1} f(\lambda x + (1 - \lambda) y) \leq \lambda f(x) + (1 - \lambda) f(y) \end{equation}
    %
    By rewording the definition, convexity is also satisfied when, for $a \leq x < y < z \leq b$,
    %
    \begin{equation} \label{convex2} \frac{f(y) - f(x)}{y - x} \leq \frac{f(z) - f(y)}{z - y} \end{equation}
    %
    Geometrically, this equation says that the slope of the tangent line from $(x, f(x))$ to $(y, f(y))$ is smaller than the tangent line from $(y, f(y))$ to $(z, f(z))$.

\begin{lemma}
    A $C^1$ convex function's derivative is non-decreasing on $(a,b)$.
\end{lemma}
\begin{proof}
    Suppose $f$ is a $C^1$ function, and $f'$ is non-decreasing, then consider any $a \leq x < y < z \leq b$. Then $f'(x) \leq f'(y) \leq f'(z)$. Applying the mean value theorem, we conclude there is $t \in (x,y)$, and $u \in (y,z)$, for which
    %
    \begin{equation} \label{convexderivative} f'(t) = \frac{f(y) - f(x)}{y - x}\ \ \ \ \ f'(u) = \frac{f(z) - f(y)}{z - y} \end{equation}
    %
    And since $t < u$, (\ref{convexderivative}) implies $f'(t) \leq f'(u)$, i.e. (\ref{convex2}) is satisfied.

    If $f$ is $C^1$ and convex, then surely $f'$ is non-decreasing. Fix $a \leq x < y \leq b$. In (\ref{convex2})), letting $y$ converge to $x$, we obtain
    %
    \begin{equation} \label{yconvergeleft} f'(x) = \lim_{y \to x^+} \frac{f(y) - f(x)}{y - x} \leq \lim_{y \to x} \frac{f(z) - f(y)}{z - y} = \frac{f(z) - f(x)}{z - x} \end{equation}
    %
    Conversely, letting $y \to z$, we obtain
    %
    \begin{equation} \label{yconvergeright} f'(y) = \lim_{y \to z^-} \frac{f(z) - f(y)}{z - y} \geq \lim_{y \to z} \frac{f(y) - f(x)}{y - x} = \frac{f(z) - f(x)}{z - x} \end{equation}
    %
    And in tandem, (\ref{yconvergeleft}), (\ref{yconvergeright}) and (\ref{convex2}) imply that $f'(x) \leq f'(y)$.
\end{proof}

\begin{example}
    $\exp: \mathbf{R} \to \mathbf{R}$ is convex on $(-\infty, \infty)$, since $\exp'' = \exp > 0$.
\end{example}

\begin{lemma}
    A function is continuous on the open segments where it is convex.
\end{lemma}

The most important inequality in analysis is the triangle inequality, undershadowed by the Schwarz inequality. Almost as important is Jensen's inequality. Despite its importance, the proof is fairly simple and intuitive. Consider the center of mass of an object.

\begin{theorem}[Jensen's Inequality]
    Let $(\Omega, \mathbf{P})$ be a probability space. If $X \in L^1(\mathbf{P})$, where $a < X< b$, and if $f$ is a real function, convex on $(a,b)$, then
    %
    \begin{equation} \label{jensen} f(\mathbf{E}[X]) \leq \mathbf{E}[f \circ X] \end{equation}
\end{theorem}
\begin{proof}
    Since $a < X < b$, $a < \mathbf{E}[X] < b$. Let
    %
    \[ \beta = \sup_{a < x < \mathbf{E}[X]} \frac{g(\mathbf{E}[X]) - g(x)}{\mathbf{E}[X] - x} \]
    %
    For any $a < x < y$, $g(x) + \beta(\mathbf{E}[X] - x) \geq g(y)$. But also, by the right side of $(\ref{convex2})$, for any $z > y$, $g(z) - \beta(z - \mathbf{E}[X]) \geq g(\mathbf{E}[X])$. For any $\omega \in \Omega$, we may restate these equations as as
    %
    \[ g(f(\omega)) - \beta(f(\omega) - y) \geq g(y) \]
    %
    But then taking expectations, we obtain (\ref{jensen}).
\end{proof}

Jensen's inequality is incredibly useful. To see this, consider some examples.

\begin{example}
    We have seen the exponential function is convex. Hence for any $L_1(\mathbf{P})$, where $\mathbf{P}$ is a probability measure, we have
    %
    \begin{equation} \label{harmonic} exp\left(\int f d\mathbf{P} \right) \leq \int e^f d\mathbf{P} \end{equation}
    %
    Let $\mathbf{P}$ be the uniform measure over a finite set $\{ x_1, x_2, \dots, x_n \}$. Then (\ref{harmonic})) tells us that
    %
    \[ e^{f(x_1)/n}e^{f(x_2)/n} \dots e^{f(x_n)/n} \leq \frac{\sum e^{f(x_i)}}{n} \]
    %
    If we let $x_i$ be real numbers, and $f(x_i) = \log(x_i)$, we conclude
    %
    \[ (x_1x_2 \dots x_n)^{1/n} \leq \frac{\sum x_i}{n} \]
    %
    In other words, the geometric mean is always smaller than the arithmetic mean.
\end{example}

Jensen's inequality implies so many other inequalities in analysis.

    Let $1 \leq p \leq \infty$. We say $1 \leq q \leq \infty$ is the {\bf conjugate} of $p$ if $p^{-1} + q^{-1} = 1$, and write $q = p'$.

\begin{theorem}[H\"{o}lder]
    If $p' = q$, $(\Omega,\mu)$ is a measure space, and $f,g$ are positive measurable functions on $\Omega$, then
    %
    \begin{equation} \label{holder} \| fg \|_1 \leq \| f \|_p \| g \|_q \end{equation}
\end{theorem}
\begin{proof}
    Let $A,B$ be the values on the right hand side of (\ref{holder})). If $A = 0$, then $f = 0$ almost everywhere, so that the theorem is trivial. Symmetry shows that the same is true if $B = 0$, so assume $A, B \neq 0$. Define $F = f/A$, and $G = g/B$. Then
    %
    \[ \left( \int F^p d\mu \right)^{1/p} = A^{-1} \left( \int f^p d\mu \right)^{1/p} = 1\ \ \ \ \ \left( \int G^q d\mu \right)^{1/q} = B^{-1} \left( \int g^q d\mu \right)^{1/q} = 1 \]
    %
    For any $x$, there is a number $s$ such that $e^{s/p} = F(x)$, and $t$ such that $e^{t/q} = G(x)$. By the convexity of the exponential function, $e^{s/p + t/q} \leq e^s p^{-1} + e^t q^{-1}$. Thus $FG \leq p^{-1} F^p + q^{-1} G^q$, and thus by integrating,
    %
    \[ \int FG\ d\mu \leq p^{-1} + q^{-1} = 1 \]
\end{proof}

\begin{corollary}[Minkowski]
    For
    \[ \| f + g \|_p \leq \| f \|_p + \| g \|_p \]
\end{corollary}






\section{The Dual of a Hilbert Space}

\begin{lemma}
    Fix $x,z \in H$. If $\langle x, y \rangle = \langle z, y \rangle$ for all $y \in H$, then $x = z$.
\end{lemma}
\begin{proof}
    For then $\langle x - z, y \rangle = 0$ for all $y$, and in particular, $\langle x - z, x - z \rangle = 0$.
\end{proof}





\chapter{Operator Theory}

\section{Operators on a Hilbert Space}

If $H$ is a Hilbert space, then the characterization of $H^*$ is very easy - it is naturally isomorphic to $H$ itself. And Hilbert spaces form very neat isomorphism classes, so the theory of Hilbert spaces on their own is quite trivial. When one says they study Hilbert spaces, they really study the bounded operators from one space to another. This is the task we begin in this section. Given two Hilbert spaces, we let $B(H,K)$ denote the continuous linear maps from $H$ to $K$.

\begin{lemma}
    If $( \cdot, \cdot )$ is a continuous sesquilinear form on a Hilbert space, then there is a unique $M \in B(H)$ such that $(x,y) = \langle x, My \rangle$.
\end{lemma}
\begin{proof}
    For each $x$, $(\cdot, x)$ is a bounded linear functional, so there is a unique $Mx \in H$ such that $(y,x) = \langle y, Mx \rangle$. Since
    %
    \[ (z, \lambda x + \gamma y) = \overline{\lambda}(z, x) + \overline{\gamma}(z,y) = \langle z, \lambda Mx + \gamma My \rangle \]
    %
    $M$ is linear, because $M(\lambda x + \gamma y) = \lambda Mx + \gamma My$. $M$ is certainly uniquely defined, and
    %
    \[ \langle x, My \rangle = (x,y) \leq \|(\cdot, \cdot)\| \| x \| \| y \| \]
    %
    So $\| M \| \leq \|(\cdot,\cdot)\|$, but also
    %
    \[ (x,y) = \langle x, My \rangle \leq \| M \| \| x \| \| y \| \]
    %
    so $\|(\cdot, \cdot)\| \leq \| M \|$, and we have equality of norms.
\end{proof}

\begin{theorem}
    Let $H$ and $K$ be Hilbert spaces. If $M \in B(H,K)$, there is a unique map $M^* \in B(K,H)$ such that
    %
    \[ \langle Mx, y \rangle = \langle x, M^*y \rangle \]
    %
    $M^*$ is known as the {\bf adjoint} of $M$. We also have $\| M^* \| = \| M \|$.
\end{theorem}
\begin{proof}
    The map $(x,y) = \langle Mx, y \rangle$ is a sesquilinear form, so there is a unique $M^*$ such that $(x,y) = \langle x, M^*y \rangle$.
\end{proof}

\begin{example}
    If we are working in $\mathbf{R}^n$, then every operator $B(\mathbf{R}^n, \mathbf{R}^m)$ can be identified with a unique matrix in $M_{n,m}(\mathbf{R})$. If $(a_{ij})$ is that operator, then the adjoint is the transpose $(a_{ji}) \in M_{m,n}(\mathbf{R})$. In $\mathbf{C}^n$, the adjoint is the conjugate transpose $(\overline{a_{ji}})$.
\end{example}

\begin{example}
    Given $\phi \in L^\infty(X)$, where $X$ is a $\sigma$-finite measure space, we have an operator $M_\phi \in B(L^2(X))$ defined by $M_\phi(f) = \phi f$. Then
    %
    \[ \langle M_\phi f, g \rangle = \int \phi f \overline{g} = \int f \overline{g \overline{\phi}} = \langle f, \overline{\phi} g \rangle \]
    %
    Thus $M_\phi^* = M_{\overline{\phi}}$.
\end{example}

The adjoint operation $*$ is an antilinear operator from $B(H,K)$ to $B(K,H)$, such that $(ST)^* = T^*S^*$. This is verified by calculation, since
%
\[ \langle ST x, y \rangle = \langle Tx, S^* y \rangle = \langle x, T^*S^* y \rangle \]
%
We also have $\| T^*T \| = \| T \|^2$, since
%
\[ \langle T^*T x, x \rangle = \langle Tx, Tx \rangle = \| T \|^2 \| x \| \]
%
so $\| T^*T \| \leq \| T \|^2$, and this is inequality is attained in the suprema, since if $x_i \in B_H$ is such that $\| T x_i \| \to \| T \|$, then
%
\[ \langle T^*T x_i, x_i \rangle = \| T x_i \|^2 \to \| T \|^2 \]
%
Thus $*$ is what is known as an \emph{involution}.

\begin{theorem}
    If $T \in B(H)$, then $H$ is the orthogonal sum of the kernel of $T^*$ and the image of $T$, or symmetrically, the kernel of $T$ and the image of $T^*$.
\end{theorem}
\begin{proof}
    The kernel of $T$ is closed. Certainly any $Tx$ in the image of $T$ is perpendicular to some $y$ in the kernel of $T^*$, because then
    %
    \[ \langle Tx, y \rangle = \langle x, T^* y \rangle = \langle x, 0 \rangle = 0. \]
    %
    Conversely, the condition that $\langle Tx, y \rangle = 0$ for all $x \in H$ is equivalent to saying that $\langle x, T^* y \rangle = 0$ for all $x \in H$, which implies $T^* y = 0$.
\end{proof}

We can described some operators from a Hilbert space to itself by their action under the involution. $T$ is a {\bf self adjoint operator} if $T^* = T$. This is something like $T$ being symmetric, but in the infinite dimensional case. $T$ is unitary if $T^* = T^{-1}$. These are analogous to rotations, but in infinite dimensions. $T$ is {\bf normal} if $T^*T = TT^*$. Any operator $M$ can be expressed as $M = T + iS$, where $T$ and $S$ are self adjoint operators, by defining
%
\[ T = \frac{1}{2}(M + M^*)\ \ \ \ \ S = \frac{1}{2i}(M - M^*) \]
%
We shall show this is a unique expression. This is essentially like beginning with $\mathbf{C}$, and identifying $\mathbf{R}$ as those elements invariant under complex conjugation.

\begin{theorem}
    If $T$ is a self-adjoint operator on a complex Hilbert space, then
    %
    \[ \| T \| = \sup \{ \langle Tx, x \rangle: \| x \| \leq 1 \} \]
\end{theorem}
\begin{proof}
    Let $M$ be the supremum. Clearly $M \leq \| T \|$. If $\| x \|, \| y \| \leq 1$,
    %
    \begin{align*}
        \langle T(x \pm y), x \pm y \rangle &= \langle Tx, x \rangle \pm \langle Ty, x \rangle \pm \langle Tx, y \rangle + \langle Ty, y \rangle\\
        &= \langle Tx, x \rangle \pm \overline{\langle Tx, y \rangle} \pm \langle Tx, y \rangle + \langle Ty, y \rangle\\
        &= \langle Tx, x \rangle \pm 2 \Re \langle Tx, y \rangle + \langle Ty, y \rangle
    \end{align*}
    %
    Then, subtracting, we find
    %
    \[ 4\Re \langle Tx, y \rangle = \langle T(x+y),x+y \rangle - \langle T(x - y), x - y \rangle \]
    %
    Thus
    %
    \[ 4 \Re \langle Tx, y \rangle \leq M(\|x + y\|^2 + \|x - y\|^2) = 2M(\|x\|^2 + \|y\|^2) \leq 4M \]
    %
    Let $\lambda$ be such that $\langle Tx, y \rangle = \lambda |\langle Tx, y \rangle|$. If we replace $x$ by $\lambda x$, we find
    %
    \[ |\langle Tx, y \rangle| = \overline{\lambda} \langle Tx, y \rangle = \langle T(\overline{\lambda}x), y \rangle \leq M \]
    %
    Hence
    %
    \[ \| Tx \| = \sup \{ |\langle Tx, y \rangle| : y \in B_H \} \leq M \]
    %
    and therefore $\|T\| \leq M$.
\end{proof}

\begin{corollary}
    $\langle Tx, x \rangle = 0$ if and only if $T = 0$.
\end{corollary}
\begin{corollary}
    If $\langle Tx, x \rangle = \langle Sx, x \rangle$ for each $x \in H$, then $T = S$.
\end{corollary}

This theorem is false for real spaces, for instance, if we take a rotation operator $R: \RR^2 \to \RR^2$, then $\langle Rx, x \rangle = 0$ for all $x$, but $R \neq 0$.

\begin{theorem}
    $T$ is normal if and only if $\| Tx \| = \| T^* x \|$ for each $x$.
\end{theorem}
\begin{proof}
    For then, for each $x$,
    %
    \[ \| Tx \|^2 = \langle Tx, Tx \rangle = \langle T^*Tx, x \rangle \]
    \[ \| T^*x \|^2 = \langle T^*x, T^*x \rangle = \langle TT^*x, x \rangle \]
    %
    which implies $T^*T = TT^*$. The converse repeats backwards through the proof, since the norm of $\| T \|$ is specified by its action through the inner product.
\end{proof}

\begin{theorem}
    Any normal operator $T$ has the following properties.
    %
    \begin{enumerate}
        \item[(a)] $\text{ker}(T) = \text{ker}(T^*)$.
        \item[(b)] $T(H)$ is dense in $H$ if and only if $T$ is injective.
        \item[(c)] $T$ is a surjective isomorphism if and only if $\| Tx \| \geq C \| x \|$ for some $C$.
        \item[(d)] If $Tx = \lambda x$, then $T^*x = \overline{\lambda} x$.
        \item[(e)] If $\lambda$ and $\gamma$ are distinct eigenvalues, then the eigenspaces are orthogonal to one another.
    \end{enumerate}
\end{theorem}
\begin{proof}
    To prove $(a)$ notice that $\| Tx \| = 0$ if and only if $\| T^*x \| = 0$. $(b)$ follows since
    %
    \[ T(H)^\perp = \text{ker}(T^*) = \text{ker}(T) \]
    %
    If $\| Tx \| \geq C \| x \|$, then $T(H)$ is closed, and dense, therefore $T(H) = H$. The converse follows by definition. If $T$ is normal, then $\lambda - T$ is also normal, since $(\lambda - T)^* = \overline{\lambda} - T^*$, so $\text{ker}(\lambda - T) = \text{ker}(\lambda - \overline{\lambda})$. Finally, if $Tx = \lambda x$, and $Ty = \gamma y$, then
    %
    \[ \lambda \langle x, y \rangle = \langle Tx, y \rangle = \langle x, T^*y \rangle = \langle x, \overline{\gamma} y \rangle = \gamma \langle x, y \rangle \]
    %
    so if $\lambda \neq \gamma$, then $\langle x,y \rangle = 0$.
\end{proof}

If $M$ and $N$ are operators such that $MN = NM$, then certainly $M^*N^* = N^*M^*$, but it is not necessarily true that $MN^* = N^* M$ or $M^* N = N M^*$. For instance, if $M$ is not a normal operator, and we take $M = N$. But if $M$ and $N$ are normal, then the result is true. In fact, much more is true.

\begin{theorem}[Fuglede-Putnam-Rosenblum]
    If $M,N \in B(H)$ are normal, and $T \in B(H)$ satisfies $MT = TN$, then $M^*T = TN^*$.
\end{theorem}
\begin{proof}
    If $S \in B(H)$, and $T = S - S^*$, then $T^* = -T$, and so it follows that
    %
    \[ e^{S - S^*} = \sum_{k = 0}^\infty \frac{i^k}{k!} S_2^k \]
    %
    is a unitary operator. Thus $\| e^{i(S - S^*)} \| = 1$ for every $S \in B(H)$. If $MT = TN$, then $M^k T = T N^k$ for all $k$, and so $e^M T = T e^N$, or $T = e^{-M} T e^N$. Thus
    %
    \[ e^{M^*} T e^{-N^*} = e^{M^* - M} T e^{N-N^*}. \]
    %
    Since $e^{M^* - M}$ and $e^{N - N^*}$ are unitary, it follows that
    %
    \[ \| e^{M^*} T e^{-N^*} \| \leq \| T \|. \]
    %
    More generally, if $f(\lambda) = e^{\lambda M^*} T e^{- \lambda N^*}$, then $f$ is a bounded entire function, and is therefore constant. Taking the first order term in the power series shows that $M^* T = TN^*$.
\end{proof}

\begin{theorem}
    If $U \in B(H)$, then the following are equivalent.
    %
    \begin{enumerate}
        \item[(a)] $U$ is unitary.
        \item[(b)] $GL(H) = H$ and $\langle Ux, Uy \rangle = \langle x, y \rangle$.
        \item[(c)] $GL(H) = H$ and $\| Ux \| = \| x \|$, so $U$ is an isometry from $H$ to itself.
    \end{enumerate}
\end{theorem}
\begin{proof}
    If $U$ is unitary, then $U$ is invertible, so $GL(H) = H$, and
    %
    \[ \langle Ux, Uy \rangle = \langle U^*Ux, y \rangle = \langle x, y \rangle \]
    %
    If $(b)$ holds, then
    %
    \[ \| Ux \|^2 = \langle Ux, Ux \rangle = \langle x, x \rangle = \| x \|^2 \]
    %
    If $(c)$ holds, then
    %
    \[ \langle U^*Ux, x \rangle = \langle Ux, Ux \rangle = \|Ux\|^2 = \|x\|^2 = \langle x, x \rangle \]
    %
    Thus $U^*U$ is the identity. Since $U$ is injective and surjective, the open mapping theorem tells us $U$ is invertible, so $U^* = U^{-1}$.
\end{proof}

This shows that unitary operators are the natural automorphisms in the category of Hilbert spaces.

\begin{lemma}
    An operator $T \in B(H)$ in a complex Hilbert space is self adjoint if and only if $\langle Tx, x \rangle \in \mathbf{R}$ for each $x \in H$.
\end{lemma}
\begin{proof}
    If $T$ is self adjoint, then
    %
    \[ \langle Tx, x \rangle = \langle x, Tx \rangle = \overline{\langle Tx, x \rangle} \]
    %
    Conversely, if $\langle Tx, x \rangle \in \mathbf{R}$ for each $x \in H$, then
    %
    \[ \langle Tx, x \rangle = \overline{\langle x, Tx \rangle} = \langle x, Tx \rangle = \langle T^*x, x \rangle \]
    %
    for each $x$, from which we conclude $T = T^*$.
\end{proof}

\begin{theorem}
    If $P$ is a projection, then the following are equivalent:
    %
    \begin{enumerate}
        \item[(a)] $P$ is self-adjoint.
        \item[(b)] $P$ is normal.
        \item[(c)] $P(H) = \text{ker}(P)^\perp$.
        \item[(d)] $\langle Px, x \rangle = \| Px \|^2$.
    \end{enumerate}
\end{theorem}
\begin{proof}
    If $P$ is self-adjoint, then $P$ is trivially normal. If $P$ is normal, then
    %
    \[ P(H) = \text{ker}(P^*)^\perp = \text{ker}(P)^\perp \]
    %
    If (c) holds, then every $x = y + Px$, where $Py = 0$. Then
    %
    \[ \langle P(y + Px), y + Px \rangle = \langle Px, y + Px \rangle = \langle Px, Px \rangle = \| Px \|^2 \]
    %
    If (d) holds, then $\langle Px, x \rangle \in \mathbf{R}$ for each $x$, so $P$ is self-adjoint. If (c) holds, then $\langle Px, x \rangle$
\end{proof}






\section{Compact Operators}

We already know operators on an infinite dimensional Banach space are more tricky than there finite dimensional counterparts. Thus we restrict our attention to operators which do not `spread themselves out' too much. An operator is {\bf compact} if the image of any bounded set is precompact. It suffices to verify that the image of the unit ball is precompact. Given two Banach spaces $X$ and $Y$, we shall let $K(X,Y)$ denote the space of all compact operators from $X$ to $Y$.

\begin{theorem}
    $K(X,Y)$ is a closed subspace of $B(X,Y)$, such that for any $M \in K(X,Y)$, $N \in B(Y,Z)$, and $L \in B(Z',X)$, $NML \in K(Z',Z)$.
\end{theorem}
\begin{proof}
    Suppose $M$ and $N$ are compact. Let $U$ be a bounded susbet of $X$. Then
    %
    \[ (\lambda M + \gamma N)(U) = \lambda M(U) + \gamma N(U) \]
    %
    And the sum of two precompact sets is precompact. Thus $\lambda M + \gamma N$ is precompact. If $U$ is a bounded subset of $Z'$, then $L(U)$ is a bounded subset of $X$, so $ML(U)$ is a precompact subset of $Y$, implying $NML(U)$ is a precompact subset of $Z$, for if every sequence $y_i$ has a convergent subsequence, so must $N(y_i)$.

    To prove $K(X,Y)$ is closed we rely on the fact that a subset of a complete metric space is precompact if and only if, for any $\varepsilon > 0$, the subset can be covered by finitely many balls of radius $\varepsilon$. Now suppose we have $M_1, M_2, \dots \in K(X,Y)$, and $M_i \to N$. We claim $N$ is compact. Let $U$ be a bounded subset of $X$, such that $\| x \| < K$ for each $x \in U$. Then for each $\varepsilon > 0$, there are finitely many balls of radius $\varepsilon$ which cover $M_i(U)$, for each $i$. If $\| M_i - N \| < \varepsilon / K$, then $\| (M_i - N)(x) \| < \varepsilon$ for each $x \in U$, so that if $M_i x$ is contained in some ball with center $y$ of radius $\varepsilon$, then $Nx$ is contained in the ball with center $y$ and radius $2 \varepsilon$. Thus $N(U)$ can be covered with finitely many balls of radius $2\varepsilon$, and is therefore precompact.
\end{proof}
\begin{corollary}
    $K(X)$ is a closed, two-sided ideal of $B(X)$.
\end{corollary}

The best examples of compact operators were found in function spaces, and the best way to verify compactness in these spaces is to use the Arzela-Ascoli criterion. Say a family of functions $\{ g_\alpha : X \to \mathbf{C} \}$ is {\bf equicontinuous} at $x \in X$ if, for any $\varepsilon > 0$, there is an open set $U$ containing $x$ such that $\| g(a) - g(b) \| < \varepsilon$ for $a,b \in U$.

\begin{theorem}[Arzela-Ascoli]
    The precompact subsets of $l^\infty(X)$ are precisely those subsets which are equicontinuous, when $X$ is compact.
\end{theorem}
\begin{proof}
    Suppose $C$ is a precompact subset of $l^\infty(X)$.
\end{proof}

\begin{example}
    The best examples of compact operators, and the operators for which most of the theory was developed, are integral equations. Given two Hausdorff spaces $A$ and $B$, equipped with respective measures $\mu$ and $\nu$, consider some measurable function $K: A \times B \to \mathbf{C}$, and consider the integral operator $\mathbf{K}$ mapping a measurable function $f$ to a function on $A$ defined by
    %
    \[ \mathbf{K}f(x) = \int_A K(x,y) f(y) d \mu(y) \]
    %
    $K$ is known as the {\bf kernel} of $\mathbf{K}$. We assume that $K$ has bee chosen such that $\mathbf{K}f$ is measurable for each measurable $f$. If $A$ and $B$ are compact, and $K$ is continuous, then $\mathbf{K}$ is a compact operator from $L^1(A)$ to $L^1(B)$.
\end{example}

It is a theorem of Calkin that $K(X)$ is the only closed, two-sided ideal of $B(X)$, when $X$ is a Hilbert space. Thus we are lead to consider the {\bf Calkin Algebra} $B(X)/K(X)$, which is simple, and quite useful in $K$-theory.

A {\bf finite-rank} operator is an operator whose range is finite-dimensional. It is clear that every finite-rank operator is compact. We shall eventually show that the set $F(X)$ of all finite-rank operators is dense in $K(X)$, when $X$ is a Hilbert space.



\newpage

In Linear Algebra, we know we may diagonalize any symmetric matrix. Here we apply our knowledge of Hilbert spaces to extend this to infinite dimensional spaces, for `symmetric' operators which are `almost finite dimensional'.

\begin{theorem}
    If $X$ is a Banach space, and $Y$ is a closed subspace, $M$ is a compact operator on $X$, and $\lambda \neq 0$ is such that
    %
    \[ \inf \{ \| (\lambda - M)y : y \in B_Y \| \} = 0 \]
    %
    then $Y \cap \ker(\lambda - M) \neq 0$.
\end{theorem}
\begin{proof}
    Let $y_i$ be a sequence in $B_Y$ such that $(\lambda - M)y_i \to 0$. Since $M$ is a compact operator, we may assume that $My_i$ converges to some $z$. But then, by continuity,
    %
    \[ \lim_i \lambda y_i = \lim_i \lambda y_i - M y_i + M y_i = \lim_i y_i - M y_i + \lim_i M y_i = 0 + z \]
    %
    Since $\lambda \neq 0$, $y_i \to \lambda^{-1} z$. But then $z \in Y$, and $z = Mz$.
\end{proof}

\begin{corollary}
    Let $M$ be a compact operator. If $\lambda \in \sigma(M)$ is non-zero, then $\dim(\ker(\lambda - M)) < \infty$, $(\lambda - M)X$ is closed and $\text{codim}(\lambda - M) < \infty$, and there is $n$ such that
    %
    \[ \ker[(\lambda - M)^n] = \ker[(\lambda - M)^{n+1}] \]
    %
    \[ (\lambda - M)^n X = (\lambda - M)^{n+1} X \]
\end{corollary}
\begin{proof}
    First note that $\lambda = M$ on $\ker(\lambda - M)$, implying the identity is compact on $\ker(\lambda - M)$, and therefore that $\dim(\ker(\lambda - M)) < \infty$. It follows that $\ker(\lambda - M)$ is closed, and thus there is a subspace $Y$ of $X$ such that $X = \ker(\lambda - M) \oplus Y$. $M$ restricted to $Y$ is injective, and the last theorem implies, since $Y \cap \ker(\lambda - M) = (0)$, that
    %
    \[ C = \inf \{ \| (\lambda - M)y \| : y \in B_Y \} > 0 \]
    %
    Hence $\| (\lambda - M)y \| \geq C \|y\|$ on $Y$, and so $\lambda - M$ is an isomorphism from $Y$ to a subspace of $X$, which is therefore complete and thus closed. Note that
    %
    \[ [(\lambda - M)X]^\perp = \{ \phi \in X^*: \phi \circ (\lambda - M) = 0 \} = \ker( \lambda - M^*) \]
    %
    Since $T$ is compact, so is $T^*$, and because $\lambda$ must also be an eigenvalue of $T^*$, $\text{dim}[((\lambda - M)X)^\perp] < \infty$, so that $(\lambda - M)X$ has finite codimension.
\end{proof}























\chapter{Vector Valued Integration}

Ubiquitous to analysis is the integral, defined to be $\int f(x)\; dx$, defined to be an average of the infinitisimal values of the function $f$ across a certain domain. In this context, $f$ is almost always real or complex valued. But there are many spaces in which an average can be taken. Indeed, in order to define the average of a set of values $a_1, \dots, a_N$, given by
%
\[ \frac{a_1 + \dots + a_N}{N} \]
%
we need only that we can add the $a_n$, and multiply them by a scalar. Thus the natural place to generalize the integration of functions is to those which take their values in a topological vector space. In this chapter we analyze such a theory.

The weakest definition of the integral exploits the fact that the integral is linear. Because integration is the limit of finite sums, and linear functions are interchangable with summation, for any continuous linear operator $\Lambda: X \to Y$, and function $f: \Omega \to X$, we ought to have
%
\[ \Lambda \left( \int_\Omega f \right) = \int_\Omega (\Lambda \circ f) \]
%
In particular, if $\Lambda$ is a linear functional, the right hand side is just a scalar-valued function, and we know how to integrate those. In fact, we can {\it define} the {\bf weak integral} $\int_\Omega f$ such that this equation holds for any continuous linear functional $\Lambda$. In particular, if $X^*$ separates $X$, then such an integral is unique, and the only question of applicability is when such an integral exists. The conditions of the next theorem, which show that such an integral exists, are automatically satisfied if $X$ is a Fr\'{e}chet space.

\begin{theorem}
    If $X$ is a topological vector space such that $X^*$ separates points, and $\mu$ is a Borel probability measure on a compact Hausdorff space $\Omega$, and $f: \Omega \to X$ is a continuous function such that the closed convex hull of $f(\Omega)$ is compact, then the weak integral exists.
\end{theorem}
\begin{proof}
    Let $H$ be the closed convex hull of $f(\Omega)$. Given a finite family of linear functionals $L = \{ \Lambda_1, \dots, \Lambda_N \}$, we consider the family $E_L$ of all points $x \in H$ such that
    %
    \[ \Lambda_n(x) = \int_\Omega (\Lambda_n \circ f)(x)\; d\mu \]
    %
    If we can prove that each $E_L$ is non-empty, then by the compactness of $H$, the intersection of all such $E_L$ will be nonempty, and this will be the required weak integral. If we define a vector $y$ by
    %
    \[ y_n = \int_\Omega (\Lambda_n \circ f)\; d\mu \]
    %
    then it suffices to show that the image of $H$ under the map $L(x) = (\Lambda_1 x, \dots, \Lambda_N x)$ contains $y$. Because $L$ is a linear map into $\mathbf{C}^n$, $L(H)$ is compact and convex. Thus if $y$ was not an element of this set, we can separate $y$ from $H$ by a linear functional. In other words, there are constants $c_1, \dots, c_N$ such that for any $x \in H$,
    %
    \[ \sum c_n y_n < \sum c_n \Lambda_n x \]
    %
    In particular,
    %
    \[ \sum c_n y_n < \sum c_n (\Lambda_n \circ f) \]
    %
    Now integrating the right hand side gives
    %
    \[ \sum c_n y_n < \sum c_n y_n \]
    %
    which contradicts the fact that $y$ was not in the convex hull.
\end{proof}

Thus the integral of a continuous function is defined in a great many circumstances. TODO: ESTABLISH PROPERTIES ALA RUDIN.

\section{Convolution}

Convolution of functions can be realized as a vector valued integral. The normal definition
%
\[ (f * g)(x) = \int f(x - y) g(y)\; dy \]
%
By removing the value $x$, can be realized as the function-valued integral
%
\[ f * g = \int f_y g(y)\; dy \]
%
If $g$ is integrable, then we can consider this as integration of the function $f_y$ against the finite measure $g\; dy$, so provided that the map $y \mapsto f_y$ is continuous, then the weak integral exists. The weak convolution is the normal convolution, because if $h$ is a bounded function, then since $|f| * |g|$ is integrable, $(|f| * |g|)h$ is also integrable, so
%
\[ \int |(f * g)(x)| |h(x)|\; dx \leq \int (|f| * |g|)(x) h(x)\; dx < \infty \]
%
and so we may apply Fubini's theorem to conclude that
%
\[ \int \left( \int f(x-y) g(y)\; dy \right) h(x)\; dx = \int \left( \int f(x-y) h(x)\; dx \right) g(y)\; dy \]
%











\part{Linear Analysis}

\chapter{Topological Vector Spaces}

Basic functional analysis deals with norm spaces. But in many problems, more qualitative notions of convergence are encountered that cannot be induced by a norm structure. Here are some examples:
%
\begin{itemize}
    \item We say a sequence of functions $\{ f_i: X \to Y \}$ converges \emph{pointwise} to a function $f$ if $f_i(x) \to f(x)$ for each $x \in X$.

    \item If $X$ is a locally compact, but non-compact topological space, then we say a sequence of functions $\{ f_i: X \to \CC \}$ converges \emph{locally uniformly} to $f: X \to \CC$ if for any compact set $K \subset X$, $f_i$ converges to $f$ uniformly on $K$.

    \item Given a norm space $X$, we say a sequence $\{ x_i \}$ in $X$ converges \emph{weakly} to some $x$ if for each linear functional $\phi \in X^*$, $\phi(x_i) \to \phi(x)$. Similarily, we say $\{ \phi_i \}$ in $X^*$ converges in the \emph{weak $*$ topology} to some $\phi$ if for any $x \in X$, $\phi_i(x) \to \phi(x)$.
\end{itemize}
%
The notions of convergence considered in these examples cannot be induced by a norm, but they can be induced by a more qualitative `topological structure'. Our goal in this part of notes is to study vector spaces equipped with this more general topological structure.

In order to be completely general, we begin by making a framework to study function spaces with a topology under the weakest assumptions. A \emph{topological vector space} is a vector space equipped with a topology making both addition and scalar multiplication continuous. In terms of nets, this continuity means exactly that if $x_\alpha \to x$, $y_\beta \to y$, and $\lambda_\gamma \to \lambda$, then $\lambda_\gamma (x_\alpha + y_\beta) \to \lambda (x + y)$. In terms of neighbourhoods, the continuity says that if $V$ is an open subset containing $\lambda(x + y)$, then there exists $\varepsilon > 0$, and neighbourhoods $W$ of $x$ and $U$ of $y$ such that if $|\lambda - \gamma| < \varepsilon$, then $\gamma(W + U) \subset V$.

For each $x' \in X$, and $\lambda \neq 0$, the maps $T_{x'}(x) = x + x'$ and $M_\lambda(x) = \lambda x$ are homeomorphisms of $X$, as a consequence of the continuity of addition and multiplication. As a consequence, every neighbourhood of a point $x$ can be written as $x + W$, where $W$ is a neighbourhood of the origin. Thus most of the topological structure of a space can be understood by looking at the structure around the origin. Another special case of the continuity of addition and multiplication is that, since $1 \cdot (0 + 0) = 0$, for each neighbourhood $V$ of the origin, we can find a subneighbourhood $W$ with $W + W \subset V$. We can repeat this process to find neighbourhoods $W$ such that $W + W + \dots + W \subset V$ for any fixed, finite number of additions.

\begin{remark}
    A topological vector space is, of course, an abelian topological group. However, unlike in the case of topological groups, all but the finite dimensional topological vector spaces fail to be locally compact. This is why these two theories proceed down very different paths.
\end{remark}

%Algebraically, a vector space arised from a representation of a field $\mathbf{F}$ over the automorphisms of an abelian group. The topological vector spaces can also be characterized by such a structure. If $G$ is an abelian topological group, and $\rho: \mathbf{F} \to \text{Aut}(G)$ is a map into the continuous automorphisms of $G$, then the conditions which guarantee that this representation gives a topological vector space structure to $G$ is that if $g = \rho(\lambda) h$, then for any neighbourhood of the origin $W$ there is $\varepsilon > 0$ and $U$ such that if $|\lambda - \gamma| < \varepsilon$ then $\rho(\gamma) U \subset gW$. This is sufficiently satisfied if the representation is continuous with respect to the topology on $\text{Aut}(G)$ generated by neighbourhoods of the origin of the form
%
%\[ W_{UV} = \{ \varphi \in \text{Aut}(G) : \varphi(U) \subset V  \} \]
%
%where $U$ and $V$ are neighbourhoods of the origin, which can be seen as a topology analogous to the topology of functions converging locally uniformly.

\begin{example}
    If $X$ is a set, consider the space $\mathbf{F}^X$ of functions from $X$ to $\mathbf{F}$, under the product topology. Then $\mathbf{F}^X$ is a vector space, which we can make topological if we equip the space with the topology of pointwise convergence. The continuity of the operations follows because if $f_\alpha \to f$, $g_\beta \to g$, and $\lambda_\gamma \to \lambda$, then for each $x \in X$,
    %
    \[ \lambda_\alpha (f_\alpha(x) + g_\beta(x)) \to \lambda( f(x) + g(x)) \]
    %
    because addition and multiplication is continuous on $\mathbf{F}$, and this shows that the functions $\lambda_\gamma (f_\alpha + g_\beta)$ converges pointwise to $\lambda(f + g)$. The natural neighbourhood base around the origin are the sets
    %
    \[ \varepsilon B_{x_1, \dots, x_n} = \{ f \in \mathbf{F}^X : |f(x_1)|, \dots, |f(x_n)| < \varepsilon \}\ \]
    %
    for $x_1, \dots, x_n \in X$, and $\varepsilon > 0$. This space is only first countable if $X$ is countable.
\end{example}

\begin{example}
    Consider the space $L_0(X)$ of {\it all} measurable functions on a measure space $X$ with the topology induced by convergence of measure. That is, a net $f_\alpha$ converges to $f$ if, for any $\varepsilon > 0$, $\mu ( |f_\alpha - f| \geq \varepsilon ) \to 0$. Under a probabilistic interpretation, the probability that the two functions differ by more than $\varepsilon$ converges to zero for every $\varepsilon$. The neighbourhoods of the origin with respect to this topology are of the form
    %
    \[ \varepsilon U_\delta = \{ f \in L_0(X): \mu(|f| < \varepsilon) \leq \delta \} \]
    %
    The space is therefore first countable.
\end{example}

\begin{example}
    We would expect the space $L_0(X)$ to have an additional topological structure given by pointwise convergence almost everywhere. That is, we would like to have a topology such that $f_\alpha \to f$ if the set of points $x$ where $f_\alpha(x)$ does not converge to $f(x)$ has zero measure. Tempting a notion as it is, there is no topological structure giving this notion of convergence on $L_0(X)$. To see why, consider the space $L_0[0,1]$, induced by the Lebesgue measure on $[0,1]$. Consider the typewrite sequence of characteristic functions $\chi_n$, defined in terms of the sets $[H_n, H_{n+1}]$ modulo 1, where $H_n$ is the $n$th Harmonic number. Since $H_n \to \infty$, every point intersects infinitely many of the intervals, so that $\chi_n$ does not converge anywhere to zero. But $\chi_n$ does converge in measure to 0, so that for every subsequence of $\chi_n$, there is a further subsequence consisting of functions which converge pointwise almost everywhere to 0. But this contradicts the topological fact that a net $x_\alpha$ converges to a point $x$ if and only if every subnet of $x_\alpha$ has a further subnet which converges to $x$. Thus there can be no topology whose convergence agrees with almost everywhere convergence.
\end{example}

\begin{example}
    Consider the space $\mathbf{F}[[X]]$ of formal power series in a single variable. Define a topology on $\mathbf{F}[[X]]$ by letting $f_\alpha \to f$ if, eventually, the lower order coefficients in the power series expansion of $f_\alpha - f$ vanish. Thus a neighbourhood basis about the origin consists of the decreasing family $\mathfrak{m}$, $\mathfrak{m}^2$, and so on, where $\mathfrak{m}$ is the ideal consisting of power series with vanishing constant coefficient. It is easy to see that addition is continuous. Multiplication of two power series is also continuous, for if $f_\alpha \to f$ and $g_\beta \to g$, then
    %
    \[ f_\alpha g_\beta - fg = f_\alpha (g_\beta - g) + (f_\alpha - f) g \]
    %
    and because each $\mathfrak{m}^n$ is an ideal, if $g_\beta - g$ and $f_\alpha - f$ are in $\mathfrak{m}^n$, then $f_\alpha g_\beta$ is also in $\mathfrak{m}^n$. However, scalar multiplication is {\it not} continuous, for the convergence properties of this space are `too discrete'. Indeed, if we consider $\lambda_n \neq 0$ with $\lambda_n \neq 0$, then $\lambda_n X$ does {\it not} converge to 0, because $\lambda_n X \not \in \mathfrak{m}^2$. Thus $\mathbf{F}[[X]]$ is a topological ring (over any field), but is not a topological vector space. It is a useful space in algebra, which favours the discrete properties of the topological structure, but it is not so useful in analysis. We can weaken the topology on $\mathbf{C}[[X]]$ by taking the topology of pointwise convergence of the coefficients, and this gives $\mathbf{C}[[X]]$ a vector space structure, but it no longer reflects the algebraic structure of the space as strongly.
\end{example}

\section{Basic Theorems}

We now introduce some terminology from analytical linear algebra, which will become even more prevalent in the theory of topological vector spaces.
%
\begin{itemize}
    \item A {\bf convex} subset $C$ of a vector space $X$ is a set such that if $x, y \in C$ imply $tx + (1 - t) y \in C$ for $0 \leq t \leq 1$, so $C$ contains line segments between each of its points, or in set notation, $tC + (1 - t)C \subset C$ for all $0 \leq t \leq 1$. For any $t, u \in \mathbf{R}$, we have $tC + uC \subset (t + u) C$ which reduces to the ordinary convex equation by dividing both sides by $t + u$.

    \item A subset $B$ is {\bf balanced} if $\lambda B \subset B$ for $|\lambda| \leq 1$. For any scalar $\lambda$, $\lambda B$ is balanced, and if $|\lambda| = 1$, then $\lambda B = B$, since $B = \lambda^{-1}(\lambda B) \subset \lambda B \subset B$
    %
    which implies $B = \lambda B$.

    \item A subset $A$ is {\bf absorbing} if
    %
    \[ X = \bigcup_{t > 0} tA \]
    %
    so every $x \in X$ is equal to $ty$ for some scalar $t > 0$ and $y \in A$.

    \item A set $K$ is {\bf (Von Neumann) bounded} if, for every neighbourhood $U$ of the origin, there is $u > 0$ such that if $t > u$, $tU \supset K$.

    % As an example, compact sets are bounded. If $x \neq 0$, then the set $\{ x, 2x, \dots \}$ is {\it never} boundedness, because otherwise if $W$ is a balanced neighbourhood, with $nx \in tW$ for all $n$, then $x \in (t/n)W$ for all $n$, which for $t < n$ implies $x \in W$. But this means $x$ is in every neighbourhood of the origin
\end{itemize}
%
One of the cornerstones of the basis theory of topological vector spaces is that analytic properties of a space can be established from the existence of open sets with the geometric properties above.

\begin{theorem}
    Every neighbourhood of the origin is absorbing.
\end{theorem}
\begin{proof}
    Given a topological vector space $X$, and a neighbourhood $U$ of the origin. For any point $x$, by continuity,
    %
    \[ \lim_{\lambda \downarrow 0} \lambda x = \left( \lim_{\lambda \downarrow 0} \lambda \right) x = 0 \cdot x = 0 \]
    %
    Thus, eventually, $\lambda x \in U$, which implies $x \in \lambda^{-1} U$.
\end{proof}

\begin{theorem}
    Every topological vector space has a basis of balanced sets.
\end{theorem}
\begin{proof}
    Given a neighbourhood $U$ of the origin in a topological vector space $X$, note that by continuity of multiplication there is $\varepsilon > 0$ and $V \subset U$ such that if $|\lambda| < \varepsilon$, $\lambda V \subset U$. Then $\bigcup_{|\lambda| < \varepsilon} \lambda V \subset U$ is a balanced open neighbourhood of the origin contained in $U$. But this implies that the balanced open neighbourhoods form a basis of the entire topology at the origin.
\end{proof}

Note that now if $0 < t_1 < t_2 < \dots \to \infty$, and $V$ is a balanced neighbourhood, then $t_n V$ is balanced, hence contains $uV$ for all $|\lambda| < t_n$ and therefore, since $V$ is absorbing,
%
\[ \bigcup t_n V = \bigcup_{u > 0} uV = X \]
%
The previous theorem says that all balanced neighbourhoods contain absorbing neighbourhoods, so this statements remains true for any neighbourhood $V$, not just the balanced ones.

\begin{prop}
    If a space has a bounded open set, it is first countable.
\end{prop}
\begin{proof}
    Assume the bounded open set $V$ lies at the origin. Then if $t_n$ are a sequence of scalars tending to zero, then $t_n V$ is a neighbourhood basis of the origin, bcause if $W$ is an arbitrary neighbourhood, then $V \subset t_n^{-1} W$ for some $t_n$, so $t_n V \subset W$. Thus the $t_n V$ provide a countable neighbourhood base around the origin.
\end{proof}

\begin{lemma}
    If $U$ is a neighbourhood of the origin such that $x + U$ does not contain $y$, then there is a neighbourhood $V$ of the origin such that $x + V$ does not intersect $y + V$.
\end{lemma}
\begin{proof}
    Find a balanced subset $U_0$ of $U$ such that $U_0 + U_0 \subset U$. If there is $u,w \in U_0$ such that $x + u = y + w$, then
    %
    \[ y = x + u - w \in x + U_0 - U_0 = x + U_0 + U_0 \subset x + U \]
    %
    hence $x + U_0$ is disjoint from $y + U_0$.
\end{proof}

\begin{corollary}
    $T_1$ topological spaces are Hausdorff.
\end{corollary}

From now on, we will assume all the vector spaces we consider are Hausdorff. To see that a particular vector space is Hausdorff, it suffices to show that the intersection of a neighbourhood basis of the origin consists of only the origin, so that $\overline{0} = 0$. In most cases, any topological vector space can be made Hausdorff by considering an appropriate quotient space, ala identifying functions almost everywhere in the theory of $L^p$ spaces.

\begin{corollary}
    If $K$ is a compact set disjoint from $C$, then there is a neighbourhood $W$ such that $K + W$ is disjoint from $C + W$.
\end{corollary}
\begin{proof}
    For each $x \in K$, we may find a balanced set $U_x$ such that $x + U_x + U_x$ is disjoint from $C + U_x$. By compactness, finitely many $U_{x_1}, \dots, U_{x_n}$ cover $K$, and then if $W = U_{x_1} \cap \dots \cap U_{x_n}$, we find $K + W$ is disjoint from $C + W$.
\end{proof}

A corollary of this is that Hasudorff topological spaces are automatically regular. Since $C + W$ is open, we find that, in fact, $\overline{K + W}$ does not intersect $C + W$. If we choose $K = \{ 0 \}$, and $C = V^c$ for $V$ a neighbourhood of the origin, we find a neighbourhood $W$ with $\smash{\overline{W}} \subset V$. Thus for any neighborhood $V$ we can find a subneighbourhood whose closure is contained in $V$.

\begin{theorem}
    Let $X$ be a topological vector space.
    %
    \begin{enumerate}
        \item[(i)] If $V$ is a neighbourhood of the origin, and $A$ is any set, then $\overline{A} \subset A + V$, and in fact $\overline{A} = \bigcap_V A + V$.
        \item[(ii)] If $A,B \subset X$, then $\overline{A} + \overline{B} \subset \overline{A + B}$.
        \item[(iii)] The closure of a subspace is a subspace.
        \item[(iv)] If $C$ is a convex, then $C^\circ$ and $\overline{C}$ are convex subsets.
        \item[(v)] If $B$ is balanced, then so is $\overline{B}$, and if $0 \in B^\circ$, then $B^\circ$ is balanced.
        \item[(vi)] If $E$ is a bounded subset of $X$, then so is $\overline{E}$.
        \item[(vii)] If $A$ and $B$ are bounded, then so is $A + B$.
        \item[(viii)] If $A$ and $B$ are compact, so is $A + B$.
        \item[(ix)] If $A$ is compact, and $B$ is closed, $A + B$ is closed.
    \end{enumerate}
\end{theorem}
\begin{proof}
    $x \in \overline{A}$ if and only if $x + V$ intersects $A$ for every neighbourhood $V$ of the origin, hence $x \in A - V$, and $-V$ is a neighbourhood of the origin if and only if $V$ is. This proves (i). The continuity of addition essentially proves (ii), for if $x \in \overline{A}$, and $y \in \overline{B}$, then we find $x_\alpha \to x$, $y_\alpha \to y$ with $x_\alpha, y_\alpha \in X$, then $x_\alpha + y_\alpha \to x + y$, and $x_\alpha + y_\alpha \in A + B$. If $Y$ is a subspace, then (ii) implies $\overline{Y} + \overline{Y} \subset \overline{Y + Y} = \overline{Y}$, and $\lambda \overline{Y} = \overline{\lambda Y} = \overline{Y}$, so $\overline{Y}$ is closed under addition and multiplication, and is therefore a subspace. Similarily, if $C$ is convex, then
    %
    \[ \lambda \overline{C} + (1 - \lambda) \overline{C} = \overline{\lambda C + (1 - \lambda) C} \subset \overline{C} \]
    %
    and if $x,y \in C^\circ$, we can find a neighbourhood $U$ such that $x + U, y + U \subset C$, and then for any $\lambda$, by convexity
    %
    \[ (\lambda x + (1 - \lambda) y) + (\lambda U + (1 - \lambda) U) = \lambda(x + U) + (1 - \lambda)(y + U) \subset C \]
    %
    which implies $\lambda x + (1 - \lambda) y \in C^\circ$, hence (iv) is shown. If $\lambda B \subset B$ for $|\lambda| < 1$, then $\lambda \overline{B} = \overline{\lambda B} \subset \overline{B}$, so $\overline{B}$ is balanced. Conversely, $\lambda B^\circ$ is open for each $\lambda \neq 0$, so for nonzero $|\lambda| < 1$, $\lambda B^\circ \subset B^\circ$, and provided $0 \in B^\circ$, we find that $\lambda B^\circ \subset B^\circ$ for all $|\lambda| < 1$. If $E$ is a bounded subset, and $V$ is some neighbourhood, then there is some $W$ such that $\overline{W} \subset V$, and if we find $t$ such that $E \subset uW$ for all $u > t$, then $\overline{E} \subset u\overline{W} \subset uV$, hence $\overline{E}$ is bounded. If $A$ and $B$ are bounded, then $A + B$ is bounded, because if $W$ is fixed, we can find $U$ with $U + U \subset W$, and if $A,B \subset \lambda U$, then
%
\[ A + B \subset \lambda(U + U) \subset \lambda W \]
%
If $A$ and $B$ are compact, then $A + B$ is compact, as it is the image of $A \times B$ under the continuous addition map. If $A$ is compact, and $B$ is closed, then $A + B$ is closed, because if $x_\alpha + y_\alpha \to z$ in $A + B$, then $x_\alpha$ has a subnet converging to some $x \in A$, and then $y_\alpha \to z - x$, hence $z - x \in B$, and therefore $z = x + (z - x) \in A + B$.
\end{proof}




\section{Convexity and Metrizability}

We say a space is {\bf locally convex} if it has a locally convex base.

\begin{theorem}
    Every convex neighbourhood contains a subneighbourhood which is convex and balanced.
\end{theorem}
\begin{proof}
    If $U$ is a convex neighbourhood, consider the set $A = \bigcap_{|\lambda| = 1} \lambda U$, which is, convex, and balanced because if $|\gamma| < 1$,
    %
    \[ \gamma A = \gamma \bigcap_{|\lambda| = 1} \lambda U = \bigcap_{|\lambda| = 1} \gamma \lambda U = \bigcap_{|\lambda| < |\gamma|} \lambda U \subset A \]
    %
    If $V \subset U$ is balanced, then $V \subset A$, so $0 \in A^\circ$, and in particular, $A^\circ$ is a balanced, convex neighbourhood of the origin.
\end{proof}

Given a set $A$, we let $\text{Conv}(A)$ denote the {\bf convex hull} of $A$, the smallest convex set containing $A$. It can be written as
%
\[ \text{Conv}(A) = \bigcup_{n = 1}^\infty \bigcup_{t_1 + \dots + t_n = 1} (t_1A + \dots + t_nA) \]
%
It follows that the convex hull of an open set is open.

\begin{theorem}
    In a locally convex space, the convex hull of a bounded set is bounded.
\end{theorem}
\begin{proof}
    If $W$ is a convex neighbourhood, and $A$ is bounded, then $A \subset tW$ for some $t$, and so $\text{Conv}(A) \subset tW$.
\end{proof}

We now move on to a very important point of the basis theory of topological vector spaces: When can we metrize the topology of a vector space. The most useful metrics on a vector space are the {\bf translation invariant} ones, which satisfy $d(a + x, a + y) = d(x,y)$ for all points $a,x,y$ in the vector space. It turns out that every first countable vector space has such a metrization.

\begin{theorem}
    If $X$ is a first countable vector space with a countable base, then $X$ is metrizable by an invariant metric $d$ whose open balls are balanced. If $X$ is locally convex, the metric can be chosen so the open balls are convex.
\end{theorem}
\begin{proof}
    We consider a balanced neighbourhood base of the origin $V_1, V_2, \dots$ with $V_{n+1} + V_{n+1} + V_{n+1} \subset V_n$. If $X$ is locally convex, choose the $V_n$ to be convex as well. With each Dyadic rational number $r = \sum c_n 2^{-n}$ in $[0,1]$, we can associate the interval $V_r = \sum c_n V_n$, with $V_1 = X$. These neighbourhoods are monotonic because of the inclusion properties of these neighbourhoods, and satisfy $V_r + V_s \subset V_{r+s}$ (this is most easily proved when $r = s$, in which case the general case is proved). Thus we cam define
    %
    \[ f(x) = \inf \{ r \in [0,1] : x \in V_r \} \]
    %
    and $d(x,y) = f(x-y)$. It is clear $d$ is invariant, and satisfies the triangle inequality, because if $x - y \in V_r$ and $y - z \in V_s$, then $x - z \in V_r + V_s \subset V_{r + s}$. Since the $V_r$ is balanced if the $V_n$ are, the open balls of $d$ are balanced, and if the $V_r$ are convex, the open balls of $d$ are convex.
\end{proof}

\begin{remark}
    The metric $d$ obtained in the proof above is always bounded, i.e. $0 \leq d \leq 1$, and so we cannot use the same techniques to show that certain topological vector spaces are normable. We will describe the technique to obtain a norm very shortly.
\end{remark}

The notion of Cauchy sequences can be used to determine if the metric $d$ is complete. But even if we don't have a metric, we can still define the notion of a Cauchy sequence. We say $x_\alpha$ is a Cauchy net in a topological vector space if, for every neighbourhood $U$ of the origin, eventually we have $x_\alpha - x_\beta \in U$. If $x_\alpha$ converges, then it is obviously Cauchy. Of course, if $d$ is an invariant metric for the topological vector space, these notions agree exactly, which gives the surprising fact that if a single metric for a metrizable topological vector space is complete, then {\it any} metric for the vector space is complete.

\begin{theorem}
    Cauchy sequences are bounded.
\end{theorem}
\begin{proof}
    Let $x_1, x_2, \dots$ be Cauchy. If $U$ is a balanced neighbourhood of the origin, then eventually for $n \geq N$, $x_n - X_N \in U$. We know that for some $\lambda$, $x_1, \dots, X_N \in \lambda U$, hence all of the $x_n$ are in $(\lambda + 1)U$.
\end{proof}

\begin{theorem}
    A complete metrizable subspace of a topological vector space is closed.
\end{theorem}
\begin{proof}
    If $Y$ is complete and metrizable, and $y_\alpha \to x$, then $y_\alpha$ is Cauchy, hence converges in $Y$, and since $X$ is Hausdorff, $Y$ is closed.
\end{proof}

A {\bf seminorm} on a space $X$ is a positive function $\| \cdot \|$ on $X$ such that $\| x + y \| \leq \| x \| + \| y \|$, and $\| \alpha x \| \leq |\alpha| \| x \|$. These seminorms would give $X$ a metric by setting $d(x,y) = \| x - y \|$, except that we might have $d(x,y) = 0$ with $x \neq y$. One way to fix this is to consider the set $Y$ of all $x$ with $\| x \| = 0$. Then $Y$ forms a subspace of $X$, and if $y \in Y$, then $\| x + y \| \leq \| x \|$, and also $\| x \| = \| x + y - y \| \leq \| x + y \|$, hence $\| x + Y \| = \| x \|$, and so the seminorm descends to $X/Y$, and is now a norm. This is the perspective taken in the $L^p$ spaces, where $X$ is the space of all $p$ integrable functions, and $Y$ is the space of $p$ integrable functions equal to zero almost everywhere.

We shall now describe another way to give $X$ a Hausdorff topological structure if, instead of a single seminorm, we have a {\it family} of seminorms. Given such a family, we can give $X$ a topological structure by letting $x_\alpha \to 0$ if $\| x_\alpha \| \to 0$ for every seminorm $\| \cdot \|$ in the family. A family of seminorms is called {\bf separating} if for every $x \neq 0$, there is a seminorm $\| \cdot \|$ with $\| x \| \neq 0$. This implies that the topological structure is Hausdorff, for if $x_\alpha$ converges to two points $x$ and $x'$, then for any seminorm $\| \cdot \|$
%
\[ \| x - x' \| \leq \| x - x_\alpha \| + \| x_\alpha - x' \| \to 0 \]
%
hence $\| x - x' \| = 0$, and since this is true for every seminorm, $x = x'$. Every space of this form is locally convex, because a neighbourhood basis is given by the family of convex sets
%
\[ \varepsilon B_{\rho_1 \dots \rho_n} = \{ x: \rho_1(x), \dots, \rho_n(x) < \varepsilon \} = \varepsilon(B_{\rho_1} \cap \dots \cap B_{\rho_n}) \]
%
and each of these neighbourhoods is convex, since they are the intersections of convex balls with respect to an individual seminorm. In other words, the family of balls $\varepsilon B_\rho = \{ x : \rho(x) < \varepsilon \}$ form a subbase for a topology. If the family of seminorms defining the topology is closed under taking the maximum operation, then these balls actually form a base for the topology.

If $\| \cdot \|$ is a seminorm, then the `unit ball' with respect to this seminorm is convex, balanced, and absorbing. Conversely, suppose $A$ is a convex subset of a topological vector space containing the origin. Then we can define the {\bf Minkowski functional}
%   
\[ \mu_A(x) = \inf \{ t \geq 0: x \in tA \} \]
%
If $x \in \alpha A$ and $y \in \beta A$, then by convexity,
%
\[ \lambda x + \gamma y \in \lambda \alpha A + \gamma \beta A \subset (\lambda \alpha + \gamma \beta) A \]
%
Taking infima, we find $\mu_A(\lambda x + \gamma y) \leq \lambda \mu_A(x) + \gamma \mu_A(y)$, and so, as a special case, we find $\mu_A(x + y) \leq \mu_A(x) + \mu_A(y)$. If $A$ is an absorbing set, $\mu_A$ is always finite, and if, in addition, $A$ is balanced, then $\mu_A(\lambda x) = |\lambda| \mu_A(x)$, so $\mu_A$ is a seminorm. If $A_0 = \mu_A^{-1}([0,1))$ and $A_1 = \mu_A^{-1}([0,1])$, then $A_0 \subset A \subset A_1$, and $\mu_{A_0} = \mu_A = \mu_{A_1}$. Thus every {\it continuous} Minkowski functional can be given in terms of an open or a closed set.

The connection between convex sets and Minkowski functionals gives a close correspondence between locally convex spaces and spaces with metrics given by seminorms. If $\| \cdot \|$ is any continuous seminorm on $X$, we let $A$ consist of the points $x$ with $\| \cdot \| \leq 1$. Then $A$ is convex, because if $\| x \|, \| y \| \leq 1$, then
%
\[ \| \lambda x + (1 - \lambda) y \| \leq \lambda \| x \| + (1 - \lambda) \| y \| \leq \lambda + (1 - \lambda) = 1 \]
%
It is absorbing, since $x \in \| x \| A$, and balanced, since $\| \lambda x \| = |\lambda| \| x \|$. We find
%
\[ \mu_A(x) = \inf \{ \lambda \geq 0: x \in \lambda A \} = \inf \{ \lambda \geq 0: \| x \| \leq \lambda \} = \| x \| \]
%
so every seminorm is given by a Minkowski functional.

The previous paragraphs hint that the topology of {\it every} locally convex space is given by a family of seminorms. Indeed, suppose that a space has a convex balanced neighbourhood basis $U_\alpha$. If $\rho_\alpha = \mu_{U_\alpha}$, then each $\rho_\alpha$ is continuous, because if $x_\beta \to x$, then eventually $x_\beta - x \in \varepsilon U_\alpha$ for every $\varepsilon > 0$, hence eventually $\rho_\alpha(x_\beta - x) < \varepsilon$. This implies
%
\[ \rho_\alpha(x_\beta) \leq \rho_\alpha(x_\beta - x) + \rho_\alpha(x) \leq \rho_\alpha(x) + \varepsilon \]
\[ \rho_\alpha(x_\beta) \geq \rho_\alpha(x) - \rho_\alpha(x - x_\beta) \geq \rho_\alpha(x) - \varepsilon \]
%
So $\rho_\alpha(x_\beta) \to \rho_\alpha(x)$. Conversely, if $\rho_\alpha(x_\beta) \to 0$ for every fixed $\alpha$, then eventually $\rho_\alpha(x_\beta) \leq 1$, so $x_\beta \in U_\alpha$, so $x_\beta$ is eventually in $U_\alpha$ for every $\alpha$, so $x_\beta \to 0$. This shows that the topology of the vector space is exactly the same as the one formed from the family of seminorms $\rho_\alpha$.

If the topology of a space $X$ is specified by a countable family of seminorms, then the space is first countable, and therefore metrizable. However, in this situation we can come up with a natural metric to use, rather than appealing to the theorem we just came up with. If $\rho_1, \rho_2, \dots$ is an ordering of the seminorms, then the functions
%
\[ d_n(x,y) = \frac{\rho_n(x - y)}{1 + \rho_n(x-y)} \]
%
are bounded and satisfy the triangle inequality, and the metric for the space can be given by
%
\[ d(x,y) = \sum_{n = 1}^\infty \frac{d_n(x,y)}{2^n} = \sum_{n = 1}^\infty \frac{\rho_n(x-y)}{2^n (1 + \rho_n(x-y))} \]
%
Since {\it any} invariant metric on a complete, metrizable space is complete, there is normally not a need to switch to a different metric when doing general calculations on metric spaces, unless it is possible to switch to a single norm for the entire space.

\begin{theorem}
    Let $X$ be a locally convex, specified by a family of seminorms $\rho_\alpha$. Then a set $E$ is bounded if and only if $\rho_\alpha$ is bounded on $E$ for each $\alpha$.
\end{theorem}
\begin{proof}
    Suppose $E$ is bounded. If $U_\alpha$ is the unit ball with respect to $\rho_\alpha$, then for some $t > 0$, $E \in t U_\alpha$, so $\rho_\alpha(E) \leq t$. Conversely, suppose that $\rho_\alpha(E) \leq t_\alpha$ for each $\alpha$. Then we consider a neighbourhood basis of the elements $\varepsilon U_{\rho_1 \dots \rho_N}$, as introduced above, and for each such element we find
    %
    \[ E \in \max(t_1, \dots, t_N) \varepsilon U_{\rho_1 \dots \rho_N \varepsilon} \]
    %
    so $E$ is bounded.
\end{proof}

To finish our understanding of locally convex spaces, we ask when the topology of a locally convex space can be given by a single norm. If $\| \cdot \|$ is a norm for the space, then the unit ball with respect to this norm is bounded by the theorem above, and convex. It turns out this is sufficient for a locally convex space to be normable.

\begin{theorem}
    A space is normable if and only if it has a bounded, convex neighbourhood.
\end{theorem}
\begin{proof}
    If $U$ is a bounded, convex neighbourhood about the origin, then the $\varepsilon U$ form a neighbourhood basis around the origin. It follows that the topology of the space is specified by the single Minkowski functional $\mu_U$. The fact that $\mu_U$ is actually a norm, rather than a seminorm, follows from the fact that the space is Hausdorff.
\end{proof}

\begin{example}
    Consider the class $C(X)$ of continuous functions on a locally compact space $X$, under the topology of locally uniform convergence (uniform convergence on compact sets). Then $C(X)$ is a topological vector space. For each compact set $K$, we have the seminorms $\| \cdot \|_{L^\infty(K)}$, which define a seminorms which show $C(X)$ and $H(X)$ have the topology of a locally convex space. If $X$ is {\it $\sigma$ compact}, then the topology can be given by a countable family of seminorms, and so $C(X)$ is metrizable. If we write $X = \lim_{n \to \infty} K_m$, and $f_n$ is Cauchy, then the $f_n$ converge uniformly to some continuous function $g_m$ on $K_m$, and $g_{m+1}$ agrees with $g_m$ on $K_m$, hence we can consider the function $g$ defined everywhere by letting $g(x) = g_m(x)$ for all $x \in K_m$, and then $f_n \to g$. Thus $C(X)$ is Fr\'{e}chet. However, $C(X)$ is not normable if $X$ is not compact. We know a set $E \subset C(X)$ is bounded if and only if there is $M_K$ such that $\| f \|_{L^\infty(K)} \leq M_K$ for each $f \in E$. But we have a neighbourhood consisting of $U_{K_1 \dots K_N \varepsilon_1 \dots \varepsilon_N}$. And since $X$ is not compact, there is some $x \not \in K_1 \cup \dots \cup K_N$, and since $X$ is completely regular, we can find a continuous function $f$ with $f(x) = 1$, but $f$ vanishing on $K_1 \cup \dots \cup K_N$. It follows that $nf \in U_{K_1 \dots K_N \varepsilon_1 \dots \varepsilon_N}$ for all integers $n$, and the are not bounded $nf$. Thus $C(X)$ is not normable unless $X$ is compact.
\end{example}

\begin{example}
    If $X$ is an open subset of $\mathbf{C}$, then the space $H(X)$ of holomorphic functions on $X$ is also a topological vector space under locally uniform convergence. Since the local uniform limit of a holomorphic function is a holomorphic function, $H(X)$ is a closed subspace of $C(X)$, and is therefore a Fr\'{e}chet space. Montel's theorem in the theory of complex variables says that $H(X)$ has the Heine-Borel property: Every closed, bounded subset of $H(X)$ is compact. But since $H(X)$ is never finite dimensional, $H(X)$ cannot be locally compact, hence $H(X)$ cannot be locally bounded. It follows that $H(X)$ is also not normable.
\end{example}

\begin{example}
    Let $X$ be an open subset of $\mathbf{R}^n$. Then we can give $C^\infty(X)$ the structure of a topological vector space by letting $f_\alpha \to f$ if all the partial derivatives of $f_\alpha$ converge to the partial derivatives of $f$ locally uniformly. Arguing essentially the same as with $C(X)$ and $H(X)$, this shows $C^\infty(X)$ is a metrizable locally convex space. If $f_\alpha$ is a Cauchy sequence in this space, then $f_\alpha$ is Cauchy on each compact set $K$, hence $f_\alpha$ and all it's partial derivatives converges locally uniformly on $K$ to some $C^\infty$ function $g_K$. Putting these things together gives that $C^\infty(X)$ is a Fr\'{e}chet space. Surprisingly $C^\infty(X)$ also satisfies the Heine-Borel property. This is easily proven using the Arzela-Ascoli theorem. Since $C^\infty(X)$ is never finite dimensional, this space cannot be locally bounded, hence $C^\infty(X)$ isn't normable.
\end{example}




\section{Linear Maps}

As in the functional analysis of Banach spaces, linear maps play an important role in the theory. In particular, we are interested in those linear maps between topological vector spaces which are continuous. Continuous maps are exactly those which are uniformly continuous, because if $T: X \to Y$ is continuous, then for every neighbourhood $V$ of the origin in $Y$, there is a neighbourhood $U$ of the origin in $X$ such that if $x - y \in U$, then $Tx - TY \in V$.

\begin{lemma}
    If $T$ is continuous at $0$, then $T$ is continuous everywhere.
\end{lemma}
\begin{proof}
    Suppose $T$ is continuous at the origin. If $x_\alpha \to x$, then $x_\alpha - x \to 0$, hence $T(x_\alpha - x) = T(x_\alpha) - T(x) \to 0$, so $T(x_\alpha) \to T(x)$. Thus $T$ is continuous everywhere.
\end{proof}

The theory of linear functions is difficult to classify in the general case, but linear {\it functionals} are easier to understand. Given $X$, we let $X^\sharp$ denote the class of linear functionals, and $X^*$ the class of {\it continuous} linear functionals.

\begin{theorem}
    If $f \in X^\sharp$ is a non-zero linear functional, then the following are equivalent:
    %
    \begin{enumerate}
        \item $f$ is continuous.
        \item The nullspace of $f$ is closed.
        \item The nullspace of $f$ is not dense in $X$.
        \item $f$ is bounded in some neighbourhood of the origin.
    \end{enumerate}
\end{theorem}
\begin{proof}
    (1) implies (2) implies (3) is obvious. If (3) holds, so the nullspace of $f$ is not dense in $X$, we can find $x \in X$ and a balanced set $U$ such that $0 \not \in f(x + U) = f(x) + f(U)$, hence $f(u) \neq - f(x)$ for any $u \in U$. Since $U$ is balanced, this must imply that $f(u) \neq \lambda f(x)$ for any $|\lambda| \geq 1$, so $|f(u)| \leq |f(x)|$ for all $u \in U$, so $f$ is bounded on $U$, implying (4). If $f$ is bounded in some neighbourhood of the origin, then $|f(u)| \leq M$ for all $u \in U$, and if $x_\alpha \to 0$, then eventually for all $\varepsilon > 0$, $x_\alpha \in \varepsilon U$, hence $|f(x_\alpha)| \leq \varepsilon M$, implying $f(x_\alpha) \to 0$. This means $f$ is continuous at the origin. The general statement now follows from the fact that a linear function is continuous if and only if it is continuous at the origin.
\end{proof}

\begin{example}
    Suppose $X$ is a locally convex space, with the topology specified by a family of seminorms closed under the maximum operation. Then the balls $\varepsilon U_\rho$ form a basis for the space. It follows that if $f \in X^*$ is a continuous linear functional, then $f$ is bounded on a neighbourhood on the origin, so there exists $\rho, \varepsilon$, and $M$ such that if $\rho(x) \leq \varepsilon$, then $|f(x)| \leq M$. It then follows that for any $x \in X$,
    %
    \[ |f(x)| = \frac{\rho(x)}{\varepsilon} \left| f \left( \frac{\varepsilon x}{\rho(x)} \right) \right| \leq (M/\varepsilon) \rho(x) \]
    %
    so the functional is directly bounded above by a seminorm.
\end{example}

For finite dimensional topological vector spaces, the continuous linear maps are particularly easy to consider.

\begin{lemma}
    Every linear map $T: \mathbf{F}^n \to X$ is continuous.
\end{lemma}
\begin{proof}
    If we write $x_i = Te_i$, where $e_i$ is the $i$'th element of canonical basis of $\mathbf{F}^n$, then $T(a) = \sum a_i x_i$, and the continuity follows because addition and scalar multiplication is continuous.
\end{proof}

\begin{corollary}
    Every finite dimensional vector space has a unique topology making it into a topological vector space.
\end{corollary}
\begin{proof}
    if $\tau_0$ and $\tau_1$ are two topologies on $\mathbf{F}^n$ turning it into a topological vector space, then the identity map on $\mathbf{F}^n$ is a continuous map from $(\mathbf{F}^n, \tau_0)$ to $(\mathbf{F}^n, \tau_1)$, and from $(\mathbf{F}^n, \tau_1)$ to $(\mathbf{F}^n, \tau_0)$, hence we conclude the identity map is a homeomorphism.
\end{proof}

\begin{corollary}
    Every finite dimensional subspace of a Hausdorff topological vector space is closed.
\end{corollary}
\begin{proof}
    If $Y \subset X$ is a subspace of $X$, let $T: \mathbf{F}^n \to Y$ be an isomorphism. Then we know $S = T(S_{\mathbf{F}^n})$ is compact in $X$, and $B = T(B_{\mathbf{F}^n})$ is open. By compactness, there is an open neighbourhood $V$ of the origin in $X$ which does not intersect $S$. Then $W = T^{-1}(V \cap Y)$ is an open neighbourhood of the origin in $\mathbf{F}^n$ disjoint from $S_{\mathbf{F}^n}$, so that it only contains elements in $B$, and so $\overline{W}$ is compact, hence $V \cap Y \subset T(\overline{W}) \subset Y$, and so the closure of each $V \cap Y$ lies in $Y$. Now if $z \in \overline{Y}$ is arbitrary, then $z = tx$ for $x \in V$, and since $\overline{Y}$ is a subspace we find $x$ lies in $\overline{Y}$. But if $y_\alpha \to x$, with $y_\alpha \in Y$, then $y_\alpha$ is eventually in $Y \cap V$, which is a closed subset of $Y$, hence we find $x \in Y$, hence $z \in Y$.
\end{proof}

\begin{theorem}
    Every locally compact vector space $X$ has finite dimension.
\end{theorem}
\begin{proof}
    If a vector space has a precompact neighbourhood $V$, then that neighbourhood is bounded, so $V/2^n$ form a local base for $X$. Now there are $x_1, \dots, x_n \in V$ such that
    %
    \[ \overline{V} \subset (x_1 + V/2) \cup \dots \cup (x_n + V/2) \]
    %
    Then the span of the $x_i$ is a closed subspace $Y$ of $X$, and $\overline{V} \subset Y + V/2$. Then, of course, $V \subset Y + V/2$, hence, dividing by two, we find $V/2 \subset Y + V/4$, and so $\overline{V} \subset Y + V/4$. Continuing this process, we find $\overline{V} \subset \bigcap (Y + V/2^n)$, and since $V/2^n$ form a local base, we find $\overline{V} \subset \overline{Y} = Y$. We conclude that $Y$ contains an absorbing set for $X$, hence $Y = X$.
\end{proof}

Since we have a definition of boundedness in any topological vector space $X$, we can ask if a version of the Heine Borel theorem holds in $X$. That is, are closed, bounded subsets of $X$ compact.

\begin{corollary}
    No infinite dimensional space can be locally bounded and satisfy the Heine-Borel theorem.
\end{corollary}
\begin{proof}
    If $V$ is a bounded neighbourhood, then it is precompact, hence the space is locally compact, so finite dimensional.
\end{proof}

In the context of norm spaces, we found a linear map was continuous if and only if it was bounded. An analogous definition of a bounded operator $T: X \to Y$ between topological vector spaces is an operator that maps bounded sets to bounded sets. It is easy to prove that a continuous map is bounded, but the converse is not always true. However, in a metrizable space, it is true.

\begin{lemma}
    if $X$ is a metrizable space, and $x_n \to 0$, then there are constants $\lambda_n \to \infty$ with $\lambda_n x_n \to 0$.
\end{lemma}
\begin{proof}
    If $d$ is translation invariant, it is easy to prove that for any $x$, $d(nx,0) \leq n d(x,0)$. If $d(x_n,0) \to 0$, there are an increasing sequence of positive integers $N_m$ with $d(x_n,0) \leq 1/m^2$ for $n \geq N_m$. Then $d(mx_{n_m},0) \leq 1/m$, so if $\lambda_n$ is the largest $m$ with $n_m \leq n$, then $\lambda_n x_{n_m} \to 0$.
\end{proof}

\begin{example}
    If $X$ is non-metrizable, this need not be true. Consider the space $\mathbf{C}^{[0,1]}$ of all complex functions on $[0,1]$, under the topology of pointwise convergence. Since $[0,1]$ has the same cardinality as the space $\mathbf{C}^{\mathbf{N}}$ of all complex-valued sequences converging to zero pointwise, we can define a map $F: [0,1] \to \mathbf{C}^{\mathbf{N}}$, and then define a sequence $f_n$ of functions converging pointwise to zero by letting $f_n(x) = F(x)_n$. For any sequence of constants $\lambda_1, \lambda_2, \dots$ converging to $\infty$, the sequence $1/\lambda_1, 1/\lambda_2, \dots$ converges to zero, hence there is some $x \in [0,1]$ with $F(x)_n = 1/\lambda_n$. Then $\lambda_n f_n(x) = \lambda_n/\lambda_n = 1$, so $\lambda_n f_n$ does not converge pointwise to zero.
\end{example}

\begin{theorem}
    If $X$ is a metrizable space, every bounded operator with $X$ as a domain is continuous.
\end{theorem}
\begin{proof}
    Since $X$ is first countable, it suffices to prove that if $x_n$ is a sequence converging to zero, and $T: X \to Y$ is bounded, then $T$ is continuous. Certainly $x_n$ is bounded, so $Tx_n$ is bounded. If we pick $\lambda_n$ as in the last lemma, so $\lambda_n x_n \to 0$, then $\lambda_n x_n$ is bounded, so $\lambda_n Tx_n$ is bounded, and since $\lambda_n^{-1} \to 0$, $Tx_n = \lambda_n^{-1} (\lambda_n Tx_n) \to 0$.
\end{proof}

\begin{example}
    A quasinorm space is a vector space $X$ equipped with a non-negative real-valued function $\| \cdot \|$ such that $\| x \| = 0$ if and only if $x = 0$, $\| \lambda x \| = |\lambda| \| x \|$, and $\| x + y \| \lesssim \|x\| + \|y\|$, rather than the stronger inequality $\| x + y \| \leq \| x \| + \| y \|$. The most precient examples of quasinorm spaces are the $L^p(X)$ spaces over some measure $\mu$, where $p < 1$. Since the unit ball in these spaces are not convex, we know that $\| \cdot \|_p$ is not a norm. However, the norms corresponding to these spaces {\it are} quasinorms. To see this, we use the fairly trivial inequality $(a + b)^p \leq 2^p(a^p + b^p)$, for $a,b,p > 0$, which follows because
    %
    \[ (a + b)^p \leq (2 \max(a,b))^p = 2^p \max(a,b)^p \leq 2^p(a^p + b^p) \]
    %
    so that $\| f + g \|_p^p \leq 2^p ( \| f \|_p^p + \| g \|_p^p )$, and therefore
    %
    \[ \| f + g \|_p \leq 2 ( \| f \|_p^p + \| g \|_p^p )^{1/p} \leq 2^{1 + 1/p} ( \| f \|_p + \| g \|_p ) \]
    %
    So the `$L^p$ norm' is a quasinorm. This quasinorm gives $L^p(X)$ the structure of a complete metric space, the proof of completeness being exactly the same as in the case where $p \geq 1$. The fact that only a single quasinorm is defined in the definition of the topology shows the space is locally bounded.  However, the space is not locally convex, hence not a Fr\'{e}chet space. In fact, as an example, $L^p[0,1]$ contains no convex open sets other than $\emptyset$ and $L^p[0,1]$. To see this, let $W$ be an non-empty, open, convex set, and without loss of generality, let $0 \in W$, and suppose $W$ contains all functions $f$ with $\| f \|_p < \varepsilon$. Given any $f \in L^p[0,1]$, and any integer $N$, we can divide $[0,1]$ into disjoint intervals $I_1, I_2, \dots, I_N$ with
    %
    \[ \| f \|_{L^p(I_n)} = \frac{\| f \|_{L^p[0,1]}}{N^{1/p}} \]
    %
    If $f_n = N \chi_{I_n} f$, then $\| f_n \|_{L^p[0,1]} = N^{1-1/p} \| f \|_{L^p[0,1]}$. Since $p < 1$, we can choose $N$ large enough that $\| f_n \|_{L^p[0,1]} < \varepsilon$, hence $f_n \in W$, and then by convexity,
    %
    \[ f = \frac{f_1 + \dots + f_N}{N} \in W \]
    %
    Thus $W = L^p[0,1]$. Now a strange result of this is that if $T: L^p[0,1] \to X$ is a continuous linear transformation into a locally convex space $X$, then $T = 0$. This follows because if $W$ is an open, convex neighbourhood, then $T^{-1}(W)$ is open and convex, hence $T^{-1}(W) = L^p[0,1]$. But then taking $W$ to be an arbitrary convex neighbourhood of the origin,
    %
    \[ T^{-1}(0) = T^{-1} \left( \bigcap W \right) = \bigcap T^{-1}(W) = L^p[0,1] \]
    %
    Thus there are {\it no} nonzero continuous linear functions on $L^p[0,1]$.
\end{example}

For non-metrizable locally convex spaces, we still have a nice analytical criterion for continuity in terms of seminorms.

\begin{theorem}
    Let $T: X \to Y$ be a map between convex spaces. Then $T$ is continuous if and only if for any continuous seminorm $\| \cdot \|_Y$ on $Y$, there is a continuous seminorm $\| \cdot \|_X$ on $X$ such that $\| Tx \|_Y \leq \| x \|_X$.
\end{theorem}
\begin{proof}
    Suppose $T$ is continuous. Then the unit ball $B$ in the $Y$ norm is open, so $T^{-1}(B)$ is an open neighbourhood of the origin in $X$, and there therefore contains a convex, balanced subset about the origin which is the unit ball with respect to some continuous seminorm $\| \cdot \|_X$. Thus $\| Tx \|_Y \leq 1$ if $\| x \|_X \leq 1$, which implies that
    %
    \[ \| Tx \|_Y = \| x \|_X \left\| T \left( x/\|x \|_X \right) \right\|_Y \leq \| x \|_X \]
    %
    Conversely, if the other condition is satisfied and $U$ is an open convex set about the origin in $Y$, then it is the unit ball with respect to some norm $\| \cdot \|_Y$, and so there is $\| \cdot \|_X$ such that $\| Tx \|_Y \leq \| x \|_X$, hence the unit ball with respect to the $X$ norm is an open neighbourhood which maps into $U$, so $T$ is continuous at the origin, hence continuous everywhere.
\end{proof}

If $X$ and $Y$ are generated by a family of seminorms closed under maxima, we can restrict the conditions of the theorem to only the seminorms contained in this group, if we weaken the condition $\| Tx \|_Y \leq \| x \|_Y$ to $\| Tx \|_Y \lesssim \| x \|_X$.

\section{Quotient Spaces}

If $Y$ is a subspace of a topological vector space $X$, then $X/Y$ can be given the quotient topology, and one easily verifies that addition and multiplication is continuous, so the quotient of a topological vector space is a topological vector space. Since the neighbourhood of $0 + Y$ in $X/Y$ is an open set containing $Y$,
%
\[ \overline{0 + Y} = \bigcap_{U + Y} (0 + Y) + (U + Y) = \bigcap_{U} (U + Y) = \overline{Y} + Y \]
%
So $X/Y$ is Hausdorff if and only if $Y$ is closed.

We can now argue why assuming a space is Hausdorff isn't too strong a statement. Let $X$ be a (not necessarily Hausdorff) topological vector space. Then $0$ is a subspace of $X$, and so the closure of $\overline{0}$ is also a closed subspace. But now if we consider the space $X/\overline{0}$, we obtain a Hausdorff topological space. Thus the techniques of Hausdorff topological spaces can normally be used to obtain results about non Hausdorff spaces by means of quotienting.

\begin{theorem}
    The quotient map $\pi: X \to X/Y$ is open, and a neighbourhood base for $X/Y$ is given by taking $\pi(U)$, where $U$ ranges over a neighbourhood base.
\end{theorem}
\begin{proof}
    The fact that $\pi(U)$ is open when $U$ is open follows because
    %
    \[ \pi^{-1}(\pi(U)) = U + Y \]
    %
    is obviously a union of open neighbourhoods. If $V$ is an open neighbourhood of the origin in $X/Y$, then $\pi^{-1}(V)$ is a neighbourhood of the origin, hence contains some $U$ in a neighbourhood base, and then $\pi(U) \subset V$.
\end{proof}

\begin{theorem}
    The quotient of a locally convex / locally bounded / metrizable / normable / complete metrizable space also has these properties.
\end{theorem}
\begin{proof}
    The image of a convex / balanced / absorbing / bounded set under a quotient map is also convex / balanced / absorbing / bounded. This shows that the quotient of a locally convex / bounded space is locally convex / bounded. Since the quotient of a first countable space is first countable, this shows that the quotient of a metrizable space is metrizable. The fact that the quotient of a normable space is normable follows because a normable is precisely a locally convex and locally bounded space. The only remaining fact is to show that the quotient of a complete metrizable space is complete. If $d$ is a complete invariant metric for the space, with $f(x) = d(0,x)$, then the metric for the quotient is given by
    %
    \[ d(x_0 + Y, x_1 + Y) = \inf \{ f((x_1 - x_0) + y) : y \in Y \} \]
    %
    The triangle inequality follows because if $f((x_1 - x_0) + y_0) \leq \alpha$ and $f((x_2 - x_1) + y_1) \leq \beta$, then $d(x_1 + y_0, x_0) \leq \alpha$, and $d(x_2 + y_1 + y_0, x_1 + y_0) \leq \beta$, and so the triangle inequality on $X$ gives
    %
    \[ d(x_0, x_2 + y_1 + y_0) \leq \alpha + \beta \]
    %
    so $d(x_0 + Y, x_2 + Y) \leq \alpha + \beta$, and taking infima gives the triangle inequality. If $x_\alpha + Y \to 0$ in $X/Y$, then we can pick $y_\alpha$ with $x_\alpha + y_\alpha \to 0$ in $X$, hence $d(0, x_\alpha + y_\alpha) \to 0$, so $d(0,x_\alpha + Y) \to 0$. Conversely, if $d(0,x_\alpha + Y) \to 0$, then there are $y_\alpha$ with $d(0, x_\alpha + y_\alpha) \to 0$, so $x_\alpha + y_\alpha \to 0$, hence $x_\alpha + Y \to 0$. Thus $d$ is a metric for the topological structure of $X/Y$. Now if $d$ is complete on $X$, it remains to show $d$ is complete on $X/Y$. Suppose $x_n + Y$ is Cauchy. Then we can inductively, for each $n$, pick $k_n$ and $y_n$ such that $d(x_{k_{n+1}} + y_{n+1}, x_{k_n} + y_n) \leq 1/2^n$. This means $x_{k_n} + y_n$ is Cauchy, hence converges, hence $x_{k_n} + Y$ converges, and because of the Cauchy property it follows that $x_n + Y$ converges to the same value.
\end{proof}

\begin{corollary}
    If $Y$ is a closed subspace of $X$, and $Z$ is finite dimensional, then $Y + Z$ is closed.
\end{corollary}
\begin{proof}
    $Z/Y$ is finite dimension in $X/Y$, hence is closed. But $Y + Z$ is the inverse image of $Z/Y$ under the quotient map, which is continuous, so $Y + Z$ is closed.
\end{proof}






\section{Inductive Limits}

We have a theory of topological vector spaces, and maps between them, so we have a category. Let us study inductive limits in this space. The canonical example of this is the test function space $\DD(\Omega)$ on an open subset $\Omega$ of $\RR^d$, which is the inductive limit of the spaces $C_c^\infty(K)$ of smooth functions with support on a compact set $K \subset \Omega$. For general (not necessarily Hausdorff) topological spaces, inductive limits exist. Given a system $\{ f_{ij}: X_i \to X_j \}$ of linear maps, we can form the topological inductive limit $X$, together with the family of maps $f_i: X_i \to X$ such that $U \subset X$ is open if and only if $f_i^{-1}(U)$ is open in $X_i$ for each $i$. The set $X$ is defined such that $X = \bigcup f_i(X_i)$, and $f_i(x) = f_j(x)$ if and only if there exists $k \succ i,j$ such that $f_{ik}(x) = f_{jk}(y)$. This is precisely the topology with the most open sets, such that each of the maps $f_i$ is continuous. The algebraic operations on each of the spaces $X_i$ induces algebraic operations on $X$, and it is simple to see that these operations are continuous. Thus $X$ is a topological vector space. There are several problems with this construction that make this not so nice, however, because topological properties do not play quite so nicely with direct limits (the direct limit of Hausdorff topological vector spaces, for instance, need not be Hausdorff).

Our construction can be made a bit nicer if we work in the subcategory of \emph{locally convex topological spaces}. Working in this subcategory, we require less functions to be continuous, and so the direct limit can be defined with less open sets. In particular, if each of the spaces $\{ X_i \}$ is locally convex, then to form an direct limit in this subcategory, then we obtain an direct limit in the category of locally convex topological vector spaces by declaring a \emph{convex} set $U \subset X$ to be an open neighborhood of the origin if and only if $f_i^{-1}(U)$ is open for each $i$, and then using these sets as a basis for the overall topology. It is easy to see this is a basis, because the intersection of convex sets is convex, and thus we get a topology. Even in this subcategory, the direct limit of Hausdorff locally convex spaces need not be Hausdorff.

\begin{lemma}
    Consider a countable direct system $X_1 \to X_2 \to \dots$, where each $X_i$ is a locally convex $T1$ Hausdorff space, and such that each map $f_i: X_i \to X_{i+1}$ is injective. Then the direct limit $X$ is $T1$ (and thus Hausdorff).
\end{lemma}
\begin{proof}
    Consider $z,w \in X$. Suppose $i$ is the smallest index such that $z,w \in X_i$ and set $U_{i-1} = \emptyset$. Starting with $k = i$, define a convex set $U_k$, which is open in $X_k$, such that $U_{k-1} \subset U_k$, $z \in U_k$, but $w$ is not an element of $\overline{U_k}$. To do this, we start by finding a convex, balanced set $V$ such that $\overline{w + V}$ is disjoint from $\overline{U_{k-1}}$ by assumption. Then $U_k = U_{k-1} + V$ is convex, open, contains $z$, and it's closure is disjoint from $w$. Then the set $\bigcup_k U_k$ is a convex, open neighborhood of $z$ in $X$, not containing $w$, so $X$ is $T1$.
\end{proof}

A similar method to the proof above, combined with the following lemma, shows that if each of the maps $X_i \to X_{i+1}$ is am embedding, then the maps $X_i \to X$ will also be embeddings. A space obtained from the limit of countably many Fre\'{e}chet spaces embedded in one another is called an \emph{LF Space}.

\begin{lemma}
    Suppose $X$ and $Y$ are locally convex vector spaces, with $Y$ a subspace of $X$. Then for any convex open set $V \subset Y$, there exists a convex open set $U \subset X$ such that $U \cap Y = V$.
\end{lemma}
\begin{proof}
    By the definition of the subspace topology, we can write $V = V' \cap Y$ for some open set $V'$ in $X$. Let $U$ be the convex hull of $V'$. Because $V$ is convex, $U \cap Y = V$. But $U$ is also open in $X$.
\end{proof}

A direct limit of a countable family of Fr\'{e}chet spaces as in the Lemma above is called an \emph{LF space}. Most interesting results about direct limits are reserved for LF spaces, like the following result. For instance, that every LF space is complete.

These notes are probably worth looking into if I ever want to learn more about limits of convex spaces: Bierstedt, Klaus-Dieter (1988). An Introduction to Locally Convex Inductive Limits. Functional Analysis and Applications. Singapore-New Jersey-Hong Kong: Universitätsbibliothek. pp. 35–133. MR 0046004. Retrieved 20 September 2020.








\chapter{Weak Topologies}

\begin{example}
    In Banach theory, the most important examples of topological spaces are those obtain by placing other topologies on a norm space than the original norm topology. Given a norm space $X$, we have a space $X^*$ consisting of continuous linear functionals on $X$. We say that a net $x_\alpha$ converges weakly to $x$ if $f(x_\alpha) \to f(x)$ for every $f \in X^*$. This defines the {\bf weak topology} on $X$. Similarily, on $X^*$ we may define the {\bf weak $*$ topology}, by letting $f_\alpha \to f$ if $f_\alpha(x) \to f(x)$ for every $x \in X$. Note that we may also put a weak topology on $X^*$, since it is also a norm space, but the weak and weak $*$ topologies on $X^*$ rarely correspond (the weak topology has more stringent notions of continuity). These topologies are effectively topologies induced by pointwise convergence, because the weak topology is obtained by embedding $X$ in $\mathbf{F}^{X^*}$, and the weak $*$ topology is obtained by considering $X^*$ as a subspace of $\mathbf{F}^X$. The weak topology on $X$ is rarely equal to the original topology on $X$, except in degenerate circumstances. For instance, if we consider $e_n \in l_p$, for $1 < p < \infty$, then $\| e_n - e_m \| = 1$, so $e_n$ does not converge, yet if we consider any element $f \in l_q$ of the dual space, then $\langle f, e_n \rangle = f(n)$, which converges to zero as $n \to \infty$. Thus $e_n$ converges to zero in the weak topology. In $l_1$, $e_n$ does not converge at all, because the function $f(n) = (-1)^n$ is in $l_\infty$, and $\langle f, e_n \rangle = (-1)^n$ does not converge. It is important to note that the weak $*$ topology depends on the predual we are using. If we consider $l_1 = c_0^*$, then the $e_n$ converges to zero in the weak $*$ topology, yet if we consider $l_1$ as $c^*$, then the $e_n$ do not converge in the induced weak $*$ topology.

    The weak topology on a space is rarely first countable. If $A = \{ \sqrt{n} e_n : n \in \mathbf{N} \}$ is viewed as a subset of $l_2$, then 0 is contained in the weak closure of $A$, because if $\varepsilon > 0$ and we are given a finite set of sequences $\{ a_n^k \}_{n \in \mathbf{N}}$, for $1 \leq k \leq K$ and, for any $n$, there is $k_n$ such that $|a_n^{k_n}| \sqrt{n} > \varepsilon$, and if we consider the sum $\sum_k |a_n^k| \in l^2$, then
    %
    \[ \sum_n \left|\sum_k |a_n^k| \right|^2 \geq \sum_n |a_n^{k_n}|^2 \geq \varepsilon \sum \frac{1}{n} = \infty \]
    %
    so there must be an element of $A$ in each $U_{\varepsilon, \{ a_n^1 \}, \dots, \{ a_n^K \}}$, and hence $0$ is in the weak closure of $A$. However, 0 is not the weak limit of any sequence of elements in $A$, for
    %
    \[ \sum_{n = 0}^\infty \frac{e_n}{n} \in l_2 \]
    %
    and if we consider any sequence $\sqrt{n_i} e_{n_i}$ that converges weakly to zero, then we find that $(n_i)^{-1/2}$ converges to zero, hence $n_i \to \infty$. But then the sequence is not norm bounded, and hence cannot converge!
\end{example}








\part{Operator Theory}





\chapter{Banach Algebras}

In this chapter, we develop the machinery to study the various classes of bounded operators which occur when classifying transformations on a space. To do this, we employ tools from abstract algebra. Recall that an algebra over a field $\mathbf{F}$ is a (not necessarily unital) ring $A$ together with a fixed embedding of $\mathbf{F}$ into $A$, which gives $A$ a vector space structure. A consistant norm attached to an algebra is a nice situation to study operators over a Banach space. What's more, the general theory gives light to many other circumstance  which would not have been apparent were we just analyzing concrete operators. We shall use capital letters, like $M$ and $N$, to denote abstract elements of these algebras, and we denote algebras by capital letters near the beginning of the alphabet, such as $A$ or $B$. A {\bf Banach Algebra} is a Banach space which is also an algebra, and satisfies
%
\begin{equation} \label{algebranorm} \| MN \| \leq \| M \| \| N \|\ \ \ \ \ \| 1 \| = 1 \end{equation}
%
Thus multiplication is a continuous operation. We shall only consider Banach algebras over the complex numbers, because here we can apply the unique properties of holomorphicity and conjugation. Why this is so important will become clear over time.

\begin{example}
    Let $K$ be a compact space. Then the space $C(K)$ of complex-valued continuous functions is a Banach algebra under the uniform convergence norm $\| \cdot \|_\infty$ and with pointwise multiplication. If $K$ consists of $n$ points, then $C(K) \cong \mathbf{C}^n$, which is perhaps the most basic commutative Banach algebra.
\end{example}

\begin{example}
    The space $L^\infty(X)$ of essentially bounded functions on a measure space $X$ is a Banach algebra. More generally, the space $C(X)$ of continuous, bounded complex-valued functions on any topological space $X$ is a Banach algebra. If $\mu$ is the measure of $X$, then provided $\mu(X) > 0$, $L^\infty(X)$ is unital.
\end{example}

\begin{example}
    If $X$ is locally compact, then the space $C_0(X)$ of functions which vanish at infinity forms a Banach algebra, except that the space does not always contain an identity. To add an identity, we enlarge the space of functions of the form $\{ \lambda + f : \lambda \in \CC, f \in C_0(X) \}$, which can be viewed as the space of all functions such that the functions which converge to a common value at $\infty$. This space can be naturally identified with $C(X \cup \{ \infty \})$, where $X \cup \{ \infty \}$ is the one point compactification of $X$.
\end{example}

\begin{example}
    If $K$ is a compact neighbourhood in $\mathbf{C}$, then we define $A(K)$ to be the set of continuous functions on $K$ which are analytic in $K^\circ$. Since uniform convergence preserves holomorphicity, $A(K)$ is a closed subalgebra of $C(K)$, and is therefore a Banach algebra. $A(\mathbf{D})$ is known as the disk algebra.
\end{example}

\begin{example}
    Let $G$ be a locally compact group, with Haar measure $\mu$, and consider the space of functions $L^1(G)$, where multiplication is convolution,
    %
    \[ (f * g)(x) = \int f(y) g(y^{-1}x) d\mu(y) \]
    %
    One can verify using Tonelli's theorem on each component of $G$, which is $\sigma$-compact, and the translation invariance of the Haar measure, we find $\| f * g \|_1 \leq \| f \|_1 \| g \|_1$. This algebra is commutative when $G$ is a commutative. It does not always possess a unit ($L^1(\RR)$ does not, $L^1(\ZZ)$ does), but we can always enlarge the space so it does. Let $M(G)$ be the space of complex-valued Radon measures on $G$, and define convolution on $M(G)$ by letting
    %
    \[ \int f d (\eta * \nu) = \int \int f(xy) d\eta(x) d\nu(y), \]
    %
    then $M(G)$ forms a unital Banach algebra, and the Dirac delta function at the identity element of $G$ is then a unit for $M(G)$.

    The space $L^1(\ZZ)$ forms a concrete example from this class, which can be taken to be the space of integer valued sequences $c = \{ c_n \}$ (where $\sum |c_n| < \infty$), with convolution
    %
    \[ (a * b)_n = \sum_{k \in \mathbf{Z}} a_k b_{n-k} \]
    %
    The dirac delta function here is just the sequence $\delta$ such that
    %
    \[ \delta_k = \begin{cases} 1 & k = 0 \\ 0 & \text{elsewise} \end{cases} \]
    %
    If we define $\delta^1$ to be the sequence valued at one, and zero elsewhere, then we see that $\delta^1$ and its inverse, for any finitely non-zero sequence can be written as the sum of convolutions of $\delta^1$ and $(\delta^1)^{-1}$, so that $L^1(\mathbf{Z})$ is generated by $\delta^1$ and its inverse (a Banach algebra $A$ is generated by $M_1, \dots, M_n$ if $\mathbf{C}[M_1, \dots, M_n]$ is dense in $A$).
\end{example}

Except for $L^1(G)$ where $G$ is non-commutative, almost all the examples above are abelian Banach algebras. One of the prime reasons to study Banach algebras is to study operators on a Banach space, which are almost always non-commutative. In fact, some folks call the study of Banach algebras `non-commutative analysis'.

\begin{example}
    Let $E$ be a Banach space. The space $B(E)$ of all bounded linear operators from $E$ to itself is a Banach algebra with respect to the operator norm. It is a unital algebra, since it possesses the identity operator. The subset $K(E)$ of compact linear operators is a closed (double-sided) ideal of $B(E)$, and so is also a Banach algebra. $K(E)$ is unital if and only if $E$ is finite dimensional.
\end{example}

Really, all that distinguishes a Banach space from a Banach algebra is a continuous multiplication structure, modulo the norm we use to define the topology of the space.

\begin{prop}
    Let $X$ be a Banach space upon with a continuous multiplication structure. Then there is an equivalent norm on $X$ which makes the space into a Banach algebra.
\end{prop}
\begin{proof}
    Embed $X$ in $B(X)$ by defining $\Lambda_M(N) = MN$ (since multiplication on the right is continuous, $\Lambda_M$ truly is in $B(X)$ rather than just being a linear map). It is trivial to verify this is an algebra morphism, and we have the inequality
    %
    \begin{equation} \label{embedinequality} \| M \| = \| \Lambda_M (1) \| \leq \| \Lambda_M \| \| 1 \| \end{equation}
    %
    which implies the map is injective, and its inverse continuous. The closed graph theorem implies the embedding is continuous, provided we can show that if $M_i \to M$, and $\Lambda_{M_i} \to \Lambda$, then $\Lambda = \Lambda_M$. Since multiplication is continuous,
    %
    \[ \Lambda(N) = \lim \Lambda_{M_i}(N) = \lim M_iN = MN \]
    %
    So $\Lambda = \Lambda_M$, and we have continuity. The proof is complete, since in $B(X)$, the inequalities ($\ref{algebranorm}$) are known to hold, except in one special case: if $X = (0)$, then $\| \text{id}_X \| = 0$, but in this case, it is impossible to define an algebra structure on $X$.
\end{proof}

\section{Invertibility in Banach Algebras}

We shall begin Banach algebra theory by analyzing criteria for invertibility, which coincides with the spectral theory of these algebras. For obvious reasons, we restrict our attention to unital algebras. The set of all units in a unital algebra $A$ will be denoted $GL(A)$, and called the general linear group of $A$. The {\bf spectrum} and {\bf resolvent} of an element $M$ of a Banach algebra $A$ are defined respectively as
%
\[ \sigma_{A}(M) = \{ \lambda \in \mathbf{C} : \lambda - M \not \in GL(A) \} \]
%
\[ \rho_{A}(M) = \{ \lambda \in \mathbf{C} : \lambda - M \in GL(A) \} \]
%
The resolvent is the complement of the spectrum. When the underlying algebra is canonical, we just use $\sigma(M)$ and $\rho(M)$ to denote the spectrum. %One way to view the addition of a complex number to an operator as an `infinitisimal shift' in the effects of the operator. The spectrum tells us in which directions the operator goes bad under infinitisimal changes, and we shall find that knowledge of the spectrum will allow us to perturb operators more finely. A large number of theorems are based on relating these perturbations to the original operator.

\begin{example}
    Let $X$ be a space, and consider $f \in C(X)$. Then $\sigma(f) = \overline{f(X)}$. If $\lambda \in \rho(f)$, then
    %
    \[ (\lambda - f)^{-1}(x) = \frac{1}{\lambda - f(x)} \]
    %
    This function is bounded if and only if $\lambda \not \in \overline{f(X)}$. If $X$ is compact, and $f \in C(X)$, $\sigma(f) = f(X)$.
\end{example}

\begin{example}
    Consider a Banach space $E$. The inverse mapping theorem tells us that $M \in B(E)$ is invertible if and only if it is bijective. If $\dim(E) < \infty$, this is simply the set of injective operators; the spectrum is then exactly the set of eigenvalues of the operator. One can consider eigenvalues in the infinite dimensional case, yet they are almost always a proper subset of the spectra. The collection of eigenvalues is known as the {\bf point spectra}, denoted $\sigma_p(M)$.
\end{example}

The next lemma is incredibly important, and is an extension of the power series formula
%
\[ \frac{1}{1 - z} = \sum_{k = 0}^\infty z^k \]
%
whose conclusion relies solely on the product and addition structure on the complex numbers, which all Banach algebras possess.

\begin{lemma}[Neumann Series]
    If $\|M\| \leq 1$, then $1 - M \in GL(A)$, and
    %
    \[ (1 - M)^{-1} = \sum_{k = 0}^\infty M^k \]
    %
    in the sense that the right hand side also converges to a well-defined value.
\end{lemma}
\begin{proof}
    The right side converges absolutely by the comparison test, since $\| M^k \| \leq \| M \|^k$. Because $A$ is Banach, absolute convergence implies convernences, and so we are justified in the manipulation
    %
    \[ (1 - M) \sum_{k = 0}^\infty M^k = \sum_{k = 0}^\infty (1 - M)M^k = \sum_{k = 0}^\infty M^k - M^{k+1} = \lim_{n \to \infty} 1 - M^{n+1} \]
    %
    As $n \to \infty$, $M^{n+1} \to 0$, so the limit above tends to one. To obtain a right side inverse, note $(1 - M)$ commutes with $\sum_{k = 0}^n M^k$ for each $k$, so that, by continuity, $(1 - M)$ commutes with $\sum_{k = 0}^\infty M^k$, and thus $\sum_{k = 0}^\infty M^k$ is also a right sided inverse.
\end{proof}

\begin{corollary}
    If $\| 1 - M \| < 1$, then $M \in GL(A)$, and
    %
    \[ M^{-1} = \sum_{k = 0}^\infty (1 - M)^k \]
\end{corollary}

\begin{corollary}
    $GL(A)$ is an open subset of $A$.
\end{corollary}
\begin{proof}
    If $M \in GL(A)$, and if $\| M - N \| < 1/\| M^{-1} \|$, then
    %
    \[ \| 1 - M^{-1}N \| \leq \| M^{-1} \| \| M - N \|  < 1 \]
    %
    so $M^{-1}N \in GL(A)$, and thus $N \in GL(A)$.
\end{proof}

\begin{corollary}
    $\sigma(M)$ is a closed and bounded subset of $\mathbf{C}$, and $\rho(M)$ is open.
\end{corollary}
\begin{proof}
    The map $f: \lambda \mapsto \lambda - M$ is a continuous operation, for
    %
    \[ \| (\lambda - M) - (\mu - M) \| = \| \lambda - \mu \| = | \lambda - \mu | \]
    %
    Since $GL(A)$ is open, $f^{-1}(GL(A)) = \rho(M)$ is open, hence $\sigma(M)$ is closed. If $|\lambda| > \|M\|$, then $\| M/\lambda \| < 1$, so $(1 - M/\lambda) \in GL(A)$, which means $\lambda - M$ is also invertible. Thus $\sigma(A)$ is closed and bounded, hence compact.
\end{proof}

We shall see that the spectra of complex algebras are never empty. This is why we mainly study complex algebras, rather than real algebras; there are even finite dimensional real operators with empty spectra.

\begin{example}
    Consider the matrix
    %
    \[ M = \begin{pmatrix} 0 & 1 \\ -1 & 0 \end{pmatrix} \]
    %
    as an element of the Banach algebra $M_2(\mathbf{R})$. The characteristic polynomial is calculated to be $\lambda^2 + 1$, so the matrix has no eigenvalues in the real numbers, and correspondingly, $\sigma(M) = \emptyset$. Over $M_2(\mathbf{C})$, we find $\sigma(M) = \{ i, -i \}$, so the spectrum is non-empty in the complex extension of the Banach algebra.
\end{example}

\begin{lemma}
    Inversion in an operator algebra $A$ is continuous.
\end{lemma}
\begin{proof}
    Let $(M_n)$ be a sequence in $GL(A)$ converging to an invertible element $M$. Then, by continuity, $M_nM^{-1} \to 1$. If $\| 1 - M_n M^{-1} \| < 1/2$, then
    %
    \[ M_n^{-1} M = (M^{-1}M_n)^{-1} = \sum_{k = 0}^\infty (1 - M^{-1}M_n)^k \]
    %
    It follows that
    %
    \begin{align*}
        \| M_n^{-1} \| &\leq \| M^{-1} \| \| M_n^{-1} M \| \leq \| M^{-1} \| \sum_{k = 0}^\infty \| (1 - M^{-1} M_n)^k \|\\
        &\leq \| M^{-1} \| \sum 2^{-k} = 2 \| M^{-1} \|
    \end{align*}
    %
    Finally, we obtain convergence of inverses,
    %
    \[ \| M_n^{-1} - M^{-1} \| = \| M_n^{-1} (M - M_n) M^{-1} \| \leq \| M_n^{-1} \| \| M - M_n \| \| M^{-1} \| \]
    %
    which tends to zero, since the first and third values are bounded, and the second converges to zero. Thus inversion is continuous.
\end{proof}

The invertible elements $GL(A)$ cannot form a closed set of $A$, for $A$ is path-connected, so there is a sequence of invertible elements $x_i$ which converge to a non-invertible element. Fortunately, Banach algebras have a way of telling us when things are about to go wrong; the norm of the operator blows up. If $A$ was an algebra of operators on a Banach space, this result would follow from the uniform boundedness principle.

\begin{lemma}
    If $M_i \to M$, each $M_i \in GL(A)$, and $M \not \in GL(A)$, then $\| M_i^{-1} \| \to \infty$.
\end{lemma}
\begin{proof}
    If $\| M_n^{-1} \| \leq M$ for all $n$, then
    %
    \[ \| 1 - M_n^{-1} M \| \leq \| M_n^{-1} \| \| M_n - M \| \leq M \| M_n - M \| \]
    %
    If we choose $n$ large enough that $\| M_n - M \| < 1/M$, then $M_n^{-1} M$ is invertible, implying $M$ is invertible.
\end{proof}

\begin{theorem}
    If $B$ is a closed subalgebra of $A$, and $M \in GL(B)$, then the entire connected component of $M$ in $B \cap GL(A)$ is contained in $GL(B)$.
\end{theorem}
\begin{proof}
    $GL(A)$ is open, so $B \cap GL(A)$ is open in $B$. If $M_n \to M$, with $M_n \in GL(B)$, and $M \in GL(A)$, then $\| M_n^{-1} \| \to \| M^{-1} \| < \infty$, so by the last lemma, $M$ must be invertible in $B$.
\end{proof}

\begin{corollary}
    If $B$ is a closed subalgebra of $A$, then $\sigma_B(M)$ is obtained from $\sigma_A(M)$ by adding certain components of $\rho_A(M)$.
\end{corollary}
\begin{proof}
    $\sigma_B(M)$ is homeomorphic to the set $GL(B) \cap \{ \lambda - M : \lambda \in \mathbf{C} \}$, and $\sigma_A(M)$ homeomorphic to $GL(A) \cap \{ \lambda - M : \lambda \in \mathbf{C} \}$. Any component in $GL(B) \cap \{ \lambda - M : \lambda \in \mathbf{C}$ can be enlarged to a component of $GL(B)$, so if $\lambda \in \sigma_A(M)$, consider the component of $\lambda - M$ in $GL(B)$.
\end{proof}

\begin{corollary}
    If $A \subset B$ are algebras, and $\sigma_B(M) \subset \mathbf{R}$, then $\sigma_A(M) = \sigma_A(M)$.
\end{corollary}
\begin{proof}
    $\sigma_B(M)$ is a bounded subset of $\mathbf{R}$, so $\rho_B(M)$ is connected in $\mathbf{C}$. Hence $\sigma_A(M) = \sigma_B(M)$, since $\sigma_A(M) \neq \mathbf{C}$, and if we added any more points to the spectrum we would have to add the entire component, which would have to be all of $\rho_B(M)$.
\end{proof}

\subsection{Holomorphicity and Resolvent Formalism.}

The fundamental theorem of spectral theory relies on heavy complex analysis, hence its restricted application to Banach algebras over the complex numbers. Define the resolvent of $M$, defined on $\rho(M)$ by
%
\[ R(z; M) = (M - z)^{-1} \]
%
Fixing $M$, we obtain an analytic function into $A$.

\begin{lemma}
    $R$ is analytic, in the sense that
    %
    \[ \lim_{w \to 0} \frac{R(z + w, M) - R(z,M)}{w} \]
    %
    converges to a well defined value. for all $z \in \rho(M)$
\end{lemma}
\begin{proof}
    The proof is a pure computation.
    %
    \begin{align*}
        \lim_{w \to 0} &\frac{R(z + w, M) - R(z,M)}{w}\\
        &= \lim_{w \to 0} \frac{(z + w - M)^{-1} - (z - M)^{-1}}{w}\\
        &= \lim_{w \to 0} (z + w - M)^{-1} \frac{(z - M) - (z + w - M)}{w} (z - M)^{-1}\\
        &= \lim_{w \to 0} -(z + w - M)^{-1} (z - M)^{-1}\\
        &= -(z - M)^{-2}
    \end{align*}
    %
    We apply the continuity of multiplication and inversion.
\end{proof}

In Banach theory, we call such a mapping {\bf strongly analytic}. A {\bf weakly analytic function} is $f: \mathbf{C} \to A$ for which $\langle \phi, f \rangle$ is analytic for any choice of $\phi \in A^*$. It is clear that all strongly analytic functions are weakly analytic, which follows because
%
\[ \lim_{w \to 0} \phi \left( \frac{f(z + w) - f(z)}{w} \right) = \phi \left( \lim_{w \to 0} \frac{f(z + w) - f(z)}{w} \right) \]
%
For fun, we shall also prove that all weakly analytic functions are strongly analytic.

\begin{theorem}
    Every weakly analytic function $f: D \to X$ is strongly analytic.
\end{theorem}
\begin{proof}
    Fix $\phi \in X^*$. Consider a particular contour winding counterclockwise around a point $w$ in the domain, which is at least a unit distance away from $w$ at any point on the contour. If $h,k \in \mathbf{C}$ are small enough that $w + h$ and $w + k$ are contained within the contour, then by the Cauchy integral theorem,
    %
    \begin{align*}
        &\phi\left( \frac{1}{h-k} \left[ \frac{f(w + h) - f(w)}{h} - \frac{f(w + k) - f(w)}{k} \right] \right)\\
        &\ \ \ \ \ =  \phi \left( \frac{f(w)}{hk} + \frac{f(w + h)}{(h - k)h} - \frac{f(w + k)]}{(h - k)k} \right)\\
        &\ \ \ \ \ = \frac{1}{2\pi i} \int_C \frac{\phi[f(z)]\ dz}{[z - (w + h)][z - (w + k)][z - w]}
    \end{align*}
    %
    Find $\delta$ such that if $\| h \| < \delta$, the distance between any point on $C$ and $w + h$ is greater than $1/2$. Then, if $M$ is the length of $C$, and $K$ is the supremum of $f$ on $C$, then
    %
    \[ \left| \frac{1}{2\pi i} \int_C \frac{\phi[f(z)]\ dz}{[z - (w + h)][z - (w + k)][z - w]}\right| \leq \frac{4MK}{2 \pi} \| \phi \| = \frac{2MK}{\pi} \| \phi \| \]
    %
    Applying the Banach Steinhaus theorem (on $X^*$, viewing elements of $X$ as elements of $X^{**}$), we conclude that for all $h,k$ sufficiently small, there exists $K$ such that
    %
    \[ \left| \frac{f(w + h) - f(w)}{h} - \frac{f(w + k) - f(k)}{k} \right| \leq K |h - k| \]
    %
    By the completeness of $X$, the quotients of $h$ and $k$ converge to a well defined quantity as $h - k$ converges to zero.
\end{proof}

The non-emptiness of the spectrum relies on Louiville's theorem, in Banach space form. The reason for this is that the resolvent is an entire function if the spectrum exists.

\begin{theorem}
    A bounded weakly analytic function $f: \mathbf{C} \to X$ is constant.
\end{theorem}
\begin{proof}
    Let $\phi \in X^*$. Then $\phi \circ f$ is an analytic function, and
    %
    \[ | (\phi \circ f)(z) | \leq \| \phi \| \| f \|_\infty \]
    %
    Louiville tells us there is $w \in \mathbf{C}$ such that $\phi \circ f = w$. Fix $x,y \in \mathbf{C}$. Then $\phi[f(x) - f(y)] = 0$ for all $\phi \in X^*$. By the Hahn Banach theorem, since $\phi$ was arbitrary, we must have $f(x) - f(y) = 0$, so $f(x) = f(y)$.
\end{proof}

There is a deep relationship between complex analysis and Banach algebras. We shall return to `holomorphic functional analysis' later.

\begin{theorem}
    The spectrum of a complex Banach algebra is non-empty.
\end{theorem}
\begin{proof}
    Assume $\sigma(M)$ is empty. Then $R(\cdot, M)$ is an analytic, entire function, and tends to zero at infinity, since if $|\lambda| > N \| M \|$,
    %
    \begin{align*}
        \| R(\lambda, M) \| &= \| (\lambda - M)^{-1} \| = \frac{1}{|\lambda|} \left\| \sum_{k = 0}^\infty \frac{M^k}{\lambda^k} \right\| \leq \sum_{k = 0}^\infty \frac{\| M \|^k}{|\lambda|^{k+1}}\\
        &\leq \frac{1}{|\lambda|} \sum_{k = 0}^\infty \frac{1}{N^{k+1}} = \frac{1}{N(N - 1)\|M\|}
    \end{align*}
    %
    This implies $\| R(\cdot, M) \|$ is therefore constant. Since it tends to zero at infinity, $R(z, M) = 0$ for all $z \in \mathbf{C}$. But then
    %
    \[ R(z, M) = (z - M)^{-1} = 0 \]
    %
    and this is clearly impossible for any particular $\lambda$.
\end{proof}

A cute little theorem arises from this property of Banach algebras that will surprisingly find great use in the analysis of abelian Banach algebras.

\begin{corollary}
    Every complex Banach division algebra is isometric to $\mathbf{C}$.
\end{corollary}
\begin{proof}
    In any unital Banach algebra $A$, $\mathbf{C} \cdot 1$ is isometric to $\mathbf{C}$, for
    %
    \[ \| \lambda \cdot 1 \| = |\lambda| \| 1 \| = |\lambda| \]
    %
    Let $A$ be a complex division algebra, and fix $M \in A$. Pick some $\lambda \in \sigma(A)$. Then $\lambda - M \not \in GL(A)$, hence $\lambda - M = 0$, i.e. $M = \lambda$. Thus $A = \mathbf{C} \cdot 1 \cong \mathbf{C}$.
\end{proof}

The real case is much more complicated. There are three real division algebras: $\mathbf{R}$, $\mathbf{C}$, and $\mathbf{Q}$ (the quaternions), and it is much more difficult to show that these are the only three.

\begin{theorem}
    If $A$ is a Banach algebra for which $M$ exists such that
    %
    \[ \| x \| \| y \| \leq M \| x y \| \]
    %
    for all $x,y \in \mathbf{C}$, then $A$ is isometric to $\mathbf{C}$.
\end{theorem}
\begin{proof}
    Let $y \in \partial GL(A)$. Then there are $y_i \to y$ with $y_i \in GL(A)$. Thus $\| y_i^{-1} \| \to \infty$. But since
    %
    \[ \| y_i \| \| y_i^{-1} \| \leq M \]
    %
    we must have $\| y_i \| \to 0$, so $y = 0$. Suppose $z \not \in GL(A)$. Consider the line between $z$ and 1. Consider
    %
    \[ \lambda = \inf \{ \gamma \in [0,1] : \gamma + (1 - \gamma) z \in GL(A) \} \]
    %
    then $\gamma + (1 - \gamma) z \in \partial GL(A)$, so $\gamma + (1 - \gamma) z = 0$, and since $\gamma \neq 1$ (since $GL(A)$ is open), we conclude $z = \gamma/(1-\gamma)$, and since $z$ is not invertible, we must have $z = 0$. Thus $A$ is a division algebra, so Gelfand implies that $A$ is isometric to $\mathbf{C}$.
\end{proof}

\begin{theorem}
    $\sigma$ is `continuous', in the sense that for any open set $U \subset \mathbf{C}$ containing $\sigma(M)$, there is an open neighbourhood $V$ of $M$ such that if $N \in V$, $\sigma(N) \subset U$.
\end{theorem}
\begin{proof}
    The map $\lambda \mapsto \| (\lambda - M)^{-1} \|$ is a bounded continous function in $U^c$, with some bound $K$. If $\| N \| < 1/K$, and $\lambda \in U^c$, then
    %
    \[ \lambda - (M + N) = (\lambda - M)(1 - (\lambda - M)^{-1}N) \]
    %
    is invertible, since $(\lambda - M)$ is invertible, since $\| (\lambda - M)^{-1} N \| < 1$. It follows that $\sigma(M + N) \subset U$.
\end{proof}

\subsection{Spectral Radii}

The {\bf spectral radius} of an element $M \in A$ is defined to be
%
\[ r(M) = \sup \{ |\lambda| : \lambda \in \sigma(M) \} \]
%
It bounds where to look for the spectrum of $M$ occurs. What is amazing is that we can define the spectral radius without any algebraic reference; this is crazy, since if we enlarge our Banach algebra, more elements become invertible, and thus the spectrum shrinks in size. The radius formula says that there still exists points on the boundary of the spectrum.

\begin{lemma}
    Let $M \in A$, $n \in \mathbf{N}$. Then, if $\lambda \in \sigma(M)$, $\lambda^n \in \sigma(M^n)$.
\end{lemma}
\begin{proof}
    Suppose $\lambda \in \sigma(M)$. Then
    %
    \[ \lambda^n - M^n = (\lambda - M) \left(\sum \lambda^{n-1-k} M^k \right) = \left(\sum \lambda^{n-1-k} M^k \right) (\lambda - M) \]
    %
    If $\lambda - M$ was invertible, then $\lambda^n - M^n$ would also be invertible.
\end{proof}

\begin{theorem}[Spectral Radius Theorem]
    \[ r(M) = \lim_{n \to \infty} \| M^n \|^{1/n} \]
\end{theorem}
\begin{proof}
    If $\lambda \in \sigma(M)$, then $\lambda^n \in \sigma(M^n)$, and therefore by Neumann's lemma, $|\lambda|^n \leq \| M^n \|$. We conclude
    %
    \[ |\lambda| \leq \| M^n \|^{1/n} \]
    %
    taking extrema,
    %
    \[ r(M) = \sup_{\lambda \in \sigma(M)} |\lambda| \leq \liminf_{n \to \infty} \|M^n\|^{1/n} \]
    %
    Set $R = r(M)^{-1}$ (which can be $\infty$, if $r(M) = 0$), and $r = \|M\|^{-1}$. Let $\lambda$ be a complex number with modulus less than $R$. Then $1/|\lambda| > r(M)$, so $1 - \lambda M \in GL(A)$. If $\phi \in A^*$, define
    %
    \[ f: \lambda \mapsto \langle \phi, (1 - \lambda M)^{-1} \rangle \]
    %
    Then $f$ is holomorphic in the disk of radius $R$. If $|\lambda| < r$, then $\| \lambda M \| < 1$, $1 - \lambda M \in GL(A)$, and
    %
    \[ \phi \left( (1 - \lambda M)^{-1} \right) = \sum_{k = 0}^\infty \phi( \lambda^k M^k) \]
    %
    power series expansions are unique, hence this expansion should work in the whole disk of radius $R$. But $\phi$ was arbitrary, so the sequence $\lambda^k M^k$ must be bounded, by Banach Steinhaus. If $\lambda$ is fixed, then there is $C$ such that
    %
    \[ |\lambda^n| \|M^n\| \leq C \]
    %
    for all $n$, so
    %
    \[ \|M^n\|^{1/n} \leq \frac{C^{1/n}}{|\lambda|} \]
    %
    Hence
    %
    \[ \limsup_{n \to \infty} \|M^n\|^{1/n} \leq \lim_{n \to \infty} \frac{C^{1/n}}{\lambda} = \frac{1}{\lambda} \]
    %
    Letting $\lambda \to R$, we obtain that $\limsup \|M^k\|^{1/k} \leq r(a)$. We have shown
    %
    \[ \liminf \|M^k\|^{1/k} \geq r(M) \geq \limsup \|M^k\|^{1/k} \]
    %
    from which the theorem follows.
\end{proof}

\begin{corollary}
    The spectral radius of a Banach algebra element is invariant of which Banach algebra the element is in. If $B$ is a Banach subalgebra of $A$, with $M \in B$, then $r_{A}(M) = r_B(a)$.
\end{corollary}

\begin{example}
    We can isometrically embed $A(\mathbf{D})$ in $C(\mathbf{T})$ via the map $f \mapsto f|_\mathbf{T}$, so that we may view $A(\mathbf{D})$ as a closed subspace of $C(\mathbf{T})$. These properties follows simply from the maximum modulus principle. Let $z: \mathbf{T} \to \mathbf{C}$ be the identity map. Then
    %
    \[ \sigma_{A(\mathbf{D})}(z) = \mathbf{D} \supsetneqq \mathbf{T} = \sigma_{C_0(\mathbf{T})}(z|_\mathbf{T}) \]
    %
    while the spectrum are different, the spectral radius is the same.
\end{example}

In our last example, enlarging the spectrum of $f$ from $\mathbf{T}$ to $\mathbf{D}$ ws the only thing that could occur from enlarging the algebra, for we cannot add any more points outside of the unit disk, and adding a single point in the unit disk will add the entire component to the spectrum.

\subsection{Non-Unital Spectrum \& Non-Commutative Topology}

We would like to extend these tools to non-unital algebras, but there is no way to talk about invertibility since there is no unit to invert to! We still need to analyze these algebras, since they naturally occur in analysis. If $X$ is not compact, then $C_0(X)$ does not contain an identity. Though we have noted that there is usually a natural trick for adding an identity, this modifies the structure of the algebra (for instance, the ideals change). Nonetheless, it is psychologically relieving to find that all non-unital Banach algebras can be isometrically embedded into Banach algebras with unit. Given a non-unital algebra $A$, consider
%
\[ A \ltimes \mathbf{C} = \{ (a,\lambda) : a \in A, \lambda \in \mathbf{C} \} \]
%
with a multiplicative structure
%
\[ (M + \lambda)(N + \gamma) = MN + \lambda N + \gamma M + \lambda \gamma \]
%
We have an identity $0 + 1$, and the space considered still satisfies the properties that make it a Banach algebra if we give it the norm
%
\[ \| M + \lambda \| = \| M \| + |\lambda| \]
%
If $M_i + \lambda_i$ is a cauchy sequence, then $M_i$ and $\lambda_i$ are both separately cauchy sequences, and therefore converge to a well defined quantity, which also converges in the abstract norm. We may then define $\sigma_A(M) = \sigma_{A \ltimes \mathbf{C}}(M)$. The fundamental theorem of spectral theory still applies, as does the spectral radius formula.

Given $X$, we may embed $C_0(X)$ in $C(X^*)$, the continuous functions on the one-point compactification of $X$. We then find that $\mathbf{C} \ltimes C_0(X)$ is isometric to $C(X^*)$, so unitization on an arbitrary Banach algebra corresponds to the one-point compactification of the `underlying space', even if there is no underlying space. This is the premise behind the school of non-commutative topology. We shall soon see that all commutative Banach algebras can be seen as subalgebras of continuous functions on some compact space. The school of non-commutative topology contends that the intuition gained from the algebra of continuous functions gives us intuitions about arbitrary Banach algebras, which are seen as spaces of continuous functions over `noncommutative space'.

\subsection{Bounded Approximate Identities}

We may not have an identity in a non-unital Banach algebra, but we may be able to `approximate' an identity in some sense. An {\bf approximate identity} (or {\bf AI}) for an algebra $A$ is a net $\{ M_\alpha \}$ such that for any $N \in A$,
%
\[ \lim M_\alpha N = N\ \ \ \ \ \ \ \ \lim N M_\alpha = N \]
%
Nets for which one of these equations hold are known as {\bf left} or {\bf right} approximate identities. A {\bf bounded approximate identity} ({\bf BLAI} for short) is an approximate identity for which $\sup \| M_\alpha \| < \infty$.

\begin{example}
    Let $X$ be locally compact and Hausdorff. The set of all compact subsets is a directed, exhausting set. Using Urysohn's lemma, find $f_K$, for each compact $K$, such that $f_K |_K = 1$, and $\| f_K \|_\infty \leq 1$. Fix $g \in C_0(X)$. Pick $K$ such that $|g| \leq \varepsilon$ outside of $K$. Then, in $K$, $|f_K g - g| = 0$, and outside of $K$, $|f_K g - g| < 2 \varepsilon$. Thus $\| f_K g - g \|_\infty < 2 \varepsilon$. $\{ f_K \}$ is not cauchy, for if $|f_K| < \varepsilon$ outside of $K'$, then $\| f_K - f_{K'} \|_\infty > 1 - \varepsilon$.
\end{example}

\begin{example}
    Consider the net $P_r(z) = \sum_\mathbf{Z} r^{|n|} z^n$ in the group algebra $L^1(S^1)$, for $0 < r < 1$. Given a continuous function $g$, it is well known that $(P_r * g) \to g$ uniformly. Given $g \in L^1(S^1)$, pick $h \in C(S^1)$ with $\| g - h \|_1 < \varepsilon$. Then if $\| h * P_r - h \|_\infty < \varepsilon$ as well, then
    %
    \begin{align*}
        \| g * P_r - g \|_1 &\leq \|(g - h) * P_r \|_1 + \| h * P_r - h \|_1 + \| h - g \|_1\\
        &\leq \|g - h\|_1 \| P_r \|_1 + \| h * P_r - h \|_\infty + \| h - g \|_1\\
        &\leq 3 \varepsilon
    \end{align*}
    %
    And we see that $g * P_r \to g$ in $L^1$.
\end{example}

\begin{example}
    If $H$ is a Hilbert space, $B(H)$ always has the bounded approximation property, as do the classical sequence spaces $c_0$ and $l_p$. Finding a space without the approximation property was an open problem for more than 40 years. Proposed as a challenge by Stanislaw Mazur in 1936, the solver of this problem was promised a live goose. In 1972, the goose was granted to the swedish mathematician Per Enflo, who solved the problem.
\end{example}

Advanced Banach space theory makes large uses of bounded left approximate identities. A Banach space $E$ has the {\bf approximation property} if there is a net $\{ T_\alpha \}$ of finite rank operators which tend to $\text{id}_E$ uniformly on compact subsets. The space has the {\bf bounded approximation property} if $\| T_\alpha \|$ is bounded. Uniform convergence on compact sets is equivalent to pointwise convergence for bounded operators. Choose some compact $K$, fix $\varepsilon > 0$, and pick $x_1, \dots, x_n$ for which $\{ B_\varepsilon(x_i) \}$ is a cover of $K$. Then, if $x \in K$, there is $x_j$ with $\| x - x_j \| < \varepsilon$, and then
%
\begin{align*}
    \| T_\alpha x - x \| &\leq \| T_\alpha x - T_\alpha x_j \| + \| T_\alpha x_j - x_j \| + \| x_j - x \|\\
    &\leq \varepsilon \sup \|T_\alpha\| + \| T_\alpha x_j - x_j \| + \varepsilon
\end{align*}
%
If we choose $\alpha$ big enough, then this is guaranteed to be less than $3 \varepsilon$. For any Banach space $X$, we denote by $A(X)$ the closure of all finite rank operators in $B(X)$. It is clear than $A(X) \subset K(X)$. The problem of whether $A(X) = K(X)$ is much more subtle.

\begin{example}
    Let $X$ be a Banach space with the bounded approximation property. Let $\{ T_\alpha \}$ be a bounded net. If $S$ is a compact operator, then $S(\overline{B_X})$ is precompact, and
    %
    \[ \| T_\alpha S - S \| = \sup \{ \| T_\alpha Sx - Sx \| : x \in B_X \} \leq \sup \{ \| T_\alpha y - y \| : y \in \overline{S(B_X)} \} \]
    %
    The right side is the supremum over a compact set, and since $\{ T_\alpha \}$ uniformly tends to the identity on compact sets, $\| T_\alpha S - S \| \to 0$. This shows $\{ T_\alpha \}$ is a bounded left approximate identity, and that $A(X) = K(X)$.

    Now suppose $A(X)$ has a bounded left approximate identity $\{ T_\alpha \}$. Without loss of generality, we may assume each $T_\alpha$ is of finite rank, because if we choose, for each $\alpha$, a bounded net $W_{\alpha, \beta}$ of finite rank operators such that $\lim_\beta W_{\alpha, \beta} = T_\alpha$, then $\{ W_{\alpha, \beta} \}$ is a bounded left approximate identity. For $x \in X$, $\phi \in X^*$, let $x \otimes \phi : X \to X$ be defined by $(x \otimes \phi)(y) = \phi(y) x$. Then $x \otimes \phi$ obviously has finite rank. Pick $y$ for which $\langle \phi, y \rangle = 1$. Then
    %
    \[ \| T_\alpha x - x \| = \| T_\alpha (x \otimes \phi)(y) - (x \otimes \phi)(y) \| \leq \| y \| \| T_\alpha (x \otimes \phi) - (x \otimes \phi) \| \to 0 \]
    %
    So the nets converge pointwise, which implies convergence on compact sets. To summarize, $X$ has the bounded approximxation property if and only if $A(X)$ has a bounded left approximate identity if and only if $K(X)$ has a bounded left approximate identity belonging to $A(X)$.
\end{example}

\begin{lemma}
    If a space has the bounded left and right approximation properties then it has the two sided approximation property.
\end{lemma}
\begin{proof}
    Let $\{ M_\alpha \}$ be a left approximation identity, and $\{ N_\beta \}$ a right approximation identity. We contend $\{ M_\alpha + N_\beta - M_\alpha N_\beta \}$ is a two sided approximator. The limits below certainly converge, and the iterated limits must therefore equal the convergent factor, which is
    %
    \[ \lim_\alpha \lim_\beta L(M_\alpha + N_\beta - M_\alpha N_\beta) = \lim_\alpha LM_\alpha + L - LM_\alpha = L \]
    %
    \[ \lim_\beta \lim_\alpha (M_\alpha + N_\beta - M_\alpha N_\beta)L = \lim_\beta L + N_\beta L - N_\beta L = L \]
    %
    So twe have a two sided approximator.
\end{proof}

Before the logician Paul Cohen got into logic, he was a functional analyst who contributed to the theory of approximation identities. We shall prove the theorem he contributed to the field, generalized to work over arbitrary modules. Let $A$ be a Banach algebra. A {\bf left banach $A$-module} is a Banach space $X$, which is a module over $A$, for which for any $M \in A$, $x \in X$,
%
\[ \| Mx \| \leq \| M \| \| x \| \]
%
This makes the module operations continuous. As with Banach algebras, if module operations are continuous, then a Banach space can be made into a Banch $A$-module.

Now if $A$ is a unital Banach algebra, then every $M \in A$ can be trivially written as $NL$, with $NL \in A$ (just let $N = 1$). If we have an approximate identity, then we may write $M$ approximately as $NL$. Suprisingly, Cohen found that we need not approximate this factorization.

\begin{theorem}[Cohen's Factorization Theorem]
    Let $A$ be a Banach algebra with BLAI $\{ E_\alpha \}$, bounded by $K$. If $X$ is a Banach $A$-module, let $x \in \overline{AX}$, and let $\varepsilon > 0$. Then there is $M \in A$ with $\| M \| \leq K$, $y \in \overline{AE}$ with $\| y - x \| < \varepsilon$, for which $x = My$.
\end{theorem}
\begin{proof}
    If $A$ has an identity, then the proof is trivial. Since $\overline{AX}$ is a closed subset of $X$, we might as well assume $\overline{AX} = X$. We may extend $X$ to be an module over $A \ltimes \mathbf{C}$, by defining
    %
    \[ (M + \lambda) x = Mx + \lambda x \]
    %
    Pick $\lambda > 0$ small enough such that
    %
    \[ 0 < \frac{\lambda}{1 - \lambda} K < 1 \]
    %
    Define a net $\{ L_\alpha \}$ in $A \ltimes \mathbf{C}$ by letting
    %
    \[ L_\alpha = \lambda E_\alpha + (1 - \lambda) = (1 - \lambda) \left( 1 + \frac{\lambda}{1 - \lambda} E_\alpha \right) \]
    %
    Then $L_\alpha N \to N$ for all $N \in A$, so $\{ L_\alpha \}$ is a BLAI, and each $L_\alpha$ is invertible in $A \ltimes \mathbf{C}$ by the choice of $\lambda$, so
    %
    \[ L_\alpha^{-1} = \frac{1}{1 - \lambda} \sum_{k = 0}^\infty \left(- \frac{\lambda}{1 - \lambda} L_\alpha \right)^k \]
    %
    which implies
    %
    \begin{align*}
        \| L_\alpha^{-1} N - N \| &= \| L_\alpha^{-1} N - L_\alpha^{-1} L_\alpha N \| \leq \| L_\alpha^{-1} \| \| N - L_\alpha N \|\\
        &\leq \left( \frac{1}{1 - \lambda} \sum_{k = 0}^\infty \left( \frac{\lambda K}{1 - \lambda} \right)^k \right) \| N - L_\alpha N \| \to 0
    \end{align*}
    %
    Therefore $L_\alpha^{-1}$ is also a BLAI. Fix $x \in X$, and let
    %
    \[ \delta < 1, \frac{\varepsilon}{2 + \|x\|} \]
    %
    We will inductively construct a sequence $\alpha_n$ of indices such that the sequence $\{ L_{\alpha_n} \dots L_{\alpha_1} \}$ and $\{ L_{\alpha_1}^{-1} \dots L_{\alpha_n}^{-1} \}$ are convergent subnets. Choose $\alpha_1$ such that
    %
    \[ \| L_{\alpha_1}^{-1} x - x \| < \delta/2 \]
    %
    If $\alpha_1, \dots, \alpha_n$ has been chosen. There is a unique element $T_n \in A$ such that
    %
    \[  L_{\alpha_n} \dots L_{\alpha_1} = (1 - \lambda)^n + T_n \]
    %
    Pick $\alpha_{n+1}$ such that
    %
    \[ \| L_{\alpha_{n+1}} x - x \| < \frac{\delta}{\| L_{\alpha_1}^{-1} \dots L_{\alpha_n}^{-1} \| 2^{n+2}}\ \ \ \ \ \ \ \ \ \ \| L_{\alpha_{n+1}} T_n - T_n \| < 1/2^n \]
    %
    Then
    %
    \[ \| L_{\alpha_1}^{-1} \dots L_{\alpha_n}^{-1} L_{\alpha_{n+1}}^{-1} x - L_{\alpha_1}^{-1} \dots L_{\alpha_n}^{-1} x \| \leq \| L_{\alpha_1}^{-1} \dots L_{\alpha_n}^{-1} \| \| x - L_{\alpha_{n+1}}^{-1} x \| < \frac{\delta}{2^{n+2}} \]
    %
    This implies $\lim_{n \to \infty} L_{\alpha_1}^{-1} \dots L_{\alpha_n}^{-1} x = y$ exists, and
    %
    \begin{align*}
    \| y - x \| &= \left\| \sum_{n = 0}^\infty (L_{\alpha_1}^{-1} \dots L_{\alpha_{n+1}^{-1}} x - L_{\alpha_1}^{-1} \dots L_{\alpha_n}^{-1} x) \right\|\\
    &\leq \sum_{n = 0}^\infty \| L_{\alpha_1}^{-1} \dots L_{\alpha_{n+1}^{-1}} x - L_{\alpha_1}^{-1} \dots L_{\alpha_n}^{-1} x \| < \sum_{n = 0}^\infty \frac{\delta}{2^{n+2}} < \delta/2
    \end{align*}
    %
    By construction,
    %
    \[ L_{\alpha_{n+1}} (L_{\alpha_n} \dots L_{\alpha_1}) = (1 - \lambda)^n L_{\alpha_{n+1}} + L_{\alpha_{n+1}} T_n \]
    %
    Thus
    %
    \begin{align*}
        \| L_{\alpha_{n+1}} L_{\alpha_n} \dots L_{\alpha_1} - L_{\alpha_n} \dots L_{\alpha_1} \| &= \| (1 - \lambda)^n L_{\alpha_{n+1}} + L_{\alpha_{n+1}} T_n - T_n - (1-  \lambda)^n \|\\
        &\leq (1 - \lambda)^n \| L_{\alpha_{n+1}} - 1 \| + \frac{1}{2^n}
    \end{align*}
    %
    Thus $L_{\alpha_n} \dots L_{\alpha_1}$ converge to some element $M$. Since $A$ is closed in $A \ltimes \mathbf{C}$, $M \in A$, and moreover,
    %
    \[ x = \lim_{n \to \infty} (L_{\alpha_1} \dots L_{\alpha_n}) (L_{\alpha_1}^{-1} \dots L_{\alpha_n}^{-1} x) = My \]
    %
    Also, observe that
    %
    \begin{align*}
        T_{n+1} &= L_{\alpha_{n+1}} \dots L_{\alpha_1} - (1 - \lambda)^{n+1}\\
        &= L_{\alpha_{n+1}} T_n + L_{\alpha_{n+1}} (1 - \lambda)^n - (1 - \lambda)^{n+1}\\
        &= L_{\alpha_{n+1}} T_n + (1 - \lambda)^n \lambda e_{\alpha_{n+1}}
    \end{align*}
    %
    Thus
    %
    \[ \| T_{n+1} - T_n \| = \| (L_{\alpha_{n+1}} T_n - T_n + (1 - \lambda)^n \lambda e_{\alpha_{n+1}} \| \leq 1/2^n + (1 - \lambda)^n \lambda K \]
    %
    Since $M = \lim T_n$,
    %
    \[ \| M \| \leq \| T_1 \| + \left(\sum_{n = 1}^\infty (1 - \lambda)^n \lambda K \right) + \delta \leq \lambda K + (1 - \lambda) K + \delta = K + \delta \]
    %
    Let $M' = \frac{K}{K + \delta} M$, $y' = \frac{K + \delta}{K} y$. Then $M'y' = My = x$, and by the choice of $\delta$ (kept hidden all this time),
    %
    \[ \| x - y' \| \leq \frac{\| Kx - Ky \| + \| \delta y \|}{K} < \delta + \delta \| x \| + \delta^2 \leq \delta(2 + \|x\|) = \varepsilon \]
    %
    And we have verified what was needed.
\end{proof}

\begin{corollary}
    If $A$ is a non-unital Banach algebra with a BLAI, then each $M \in A$ may be written $M = NL$, with $N,L \in A$.
\end{corollary}

\begin{corollary}
    Let $A$ be a Banach algebra with a BLAI, and let $\{ M_n \}$ be a sequence in $A$ converging to 0. Then there is $M \in A$ and $\{ N_m \}$ in $A$ tending to zero such that $M_k = M N_k$.
\end{corollary}
\begin{proof}
    Let $X$ be the set of all sequences in $A$ that converge to zero, with the $\| \cdot \|_\infty$ norm. Then $X$ is a left $A$ module, and $X = \overline{AX}$, since $A$ has a BLAI. Applying Cohen's theorem, we find that we may write
    %
    \[ (M_1, M_2, \dots) = M (N_1, N_2, \dots) \]
    %
    hence $M_i = M N_i$ for each $i$.
\end{proof}





\section{Gelfand Theory}

Gelfand realized that all commutative Banach algebras were really just manifestations of continuous functions over a compact set. The way to find a space representing a given algebra. This has incredibly important repurcussions for all of Banach algebra theory, since all algebras contain commutative subalgebras. The ingenious trick to Gelfand theory is how to find a compact space on which to represent this algebra. We shall first analyze $C(K)$, for compact $K$, and find that points in $K$ can be uniquely defined with maximal ideals in $C(K)$. This means that, in general, if a given algebra is to be represented by continuous functions, the only space we could represent it on would be homeomorphic to the space of maximal ideals, with some topology. In any algebra, we may consider this space, but it is only for abelian algebras where we obtain a full representation.

\subsection{Algebra Homomorphisms and Ideals}

Algebraically, if $A$ is an algebra, the ideals are the objects we may quotient $A$ by. Since $A$ is both a ring and a vector space, an ideal therefore must be a ring ideal and a subspace. This continues to hold in the non-unital cases. We can consider left, right, and two-sided ideals, and we'll denote them by gaudy letters such as $\mathfrak{a}$ and $\mathfrak{b}$. It is an easy consequence of ring theory that a proper left ideal cannot contain any left invertible elements, and a right ideal cannot contain any right invertible elements. It is also easy to verify that the closure of any ideal is an ideal. In the commutative case, the quotient of an algebra by a maximal ideal is a field. Furthermore, every left, right, or double sided ideal can be extended to a maximal ideal of the same type.

For completeness, we include a discussion of quotients of algebras by ideals, which will come very handy in the near future. Given an ideal $\mathfrak{a}$ in an algebra $A$, we can surely form a subalgebra $A/\mathfrak{a}$, which consists of equivalence classes of the form
%
\[ [M] = \{ N : M - N \in \mathfrak{a} \} \]
%
where multiplication is obtained by performing the opeartions in the equivalence classes. This also has a natural Banach space structure, under the quotient norm.
%
\[ \| X \| = \inf \{ \| N \| : N \in X \} \]
%
It remains to be seen that this is a Banach algebra. By definition, for each $M$ and $\varepsilon > 0$, there is $N \in X$ such that
%
\[ \| N \| \leq \| X \| + \varepsilon \]
%
Similarily, for another equivalence class $Y$, there is $L$ such that
%
\[ \| L \| \leq \| Y \| + \varepsilon \]
%
From which it follows that
%
\[ \| XY \| \leq \| NL \| \leq \| N \| \| L \| \leq \| X \| \| Y \| + [\| X \| + \| Y \|] \varepsilon + \varepsilon^2 \]
%
Letting $\varepsilon$ tend to zero, we find
%
\[ \| XY \| \leq \| X \| \| Y \| \]
%
so the quotient structure really is a Banach algebra.

Ideals are naturally connected with homomorphisms $f: A \to B$ between algebras, maps which are both ring homomorphisms and linear maps. Every ideal is the kernel of some homomorphism, and the kernel of every homomorphism is an ideal. Correspondingly, {\it closed ideals} in a Banach algebra correspond to continuous homomorphisms. The most tractable homomorphisms to study are the {\bf characters}, homomorphisms from a ring $A$ to $\mathbf{C}$. In this section, we will deduce their structure.

\begin{lemma}
    If $\| M \| \leq 1$, then $|\phi(M)| \leq 1$ for a character $\phi$. Thus $\| \phi \| \leq 1$.
\end{lemma}
\begin{proof}
    The kernel of every homomorphism cannot be all of $A$, and it is certainly an ideal of $A$. Therefore the kernel cannot contain any invertible elements. If $f(M) = \lambda$, then $\lambda - M \in \ker(M)$, so that $\lambda - M$ is not invertible. This implies, from our discussion of $\sigma(M)$, that $|\lambda| \leq 1$.
\end{proof}

\begin{corollary}
    Every character is a continuous linear map.
\end{corollary}

The next lemma follows because a proper ideal of an algebra cannot contain invertible elements.

\begin{lemma}
    Every maximal ideal of a Banach algebra is closed.
\end{lemma}
\begin{proof}
    Let $\mathfrak{a}$ be a maximal ideal of an algebra $A$. It is easy to show the closure of any ideal is an ideal. It follows that either $\overline{\mathfrak{a}} = \mathfrak{a}$ (so that $\mathfrak{a}$ is closed), or $\mathfrak{a}$ is dense in $A$. Suppose the second option holds. Let $M \in GL(A)$ be chosen. Then there is $M_i \in \mathfrak{a}$ converging to $M$. But then the $M_i$ are eventually invertible, since $GL(A)$ is open, from which we conclude $\mathfrak{a} = A$, a contradiction.
\end{proof}

\begin{corollary}
    If $\mathfrak{a}$ is a maximal ideal in an abelian Banach algebra, then $A/\mathfrak{a}$ is isometric to $\mathbf{C}$.
\end{corollary}
\begin{proof}
    For then $A/\mathfrak{a}$ is a Banach division algebra.
\end{proof}

\begin{lemma}
    The kernel of every character is a maximal ideal. In an abelian algebra, every maximal ideal corresponds to the kernel of some character.
\end{lemma}
\begin{proof}
    We apply the first isomorphism theorem. Let $\phi$ be a character. Then $\phi$ is surjective, for $\phi(\mathbf{C} \cdot 1) = \mathbf{C}$. Thus if $\mathfrak{a} = \ker(f)$, $A/\mathfrak{a} \cong \mathbf{C}$, implying $\mathfrak{a}$ is maximal. If $A$ is commutative, and $\mathfrak{a}$ is a maximal ideal, then $A/\mathfrak{a}$ is a division ring, so we may project $A$ down to $A/\mathfrak{a}$, and then apply an isometry to $\mathbf{C}$.
\end{proof}

\begin{lemma}
    If the kernels of two characters correspond, then the two characters are equal.
\end{lemma}
\begin{proof}
    Suppose $\ker(\phi) = \ker(\psi)$. Fix $x \in A$. Then
    %
    \[ \phi(x - \phi(x)) = \phi(x) - \phi(x) = 0 \]
    %
    so
    %
    \[ \psi(x - \phi(x)) = \psi(x) - \phi(x) = 0 \]
    %
    Thus $\psi(x) = \phi(x)$.
\end{proof}

\begin{corollary}
    In a commutative algebra, every maximal ideal of $A$ is the kernel of a unique complex homomorphism, so there is a one-to-one correspondence between maximal ideals and characters.
\end{corollary}

We denote the set of characters of a Banach algebra by $A$ by $\Phi_A$, known as the {\bf character space} of $A$. Gelfand's theory is the study of $\Phi_A$, and its relation to $A$.

\begin{example}
    Let $K$ be a compact space, and consider the algebra $C(K)$ of continuous functions. Given a closed subset $C$ of $K$, the set
    %
    \[ \{ f \in C(K) : f|_C = 0 \} \]
    %
    is an ideal of $C(K)$. We contend these are all such closed ideals. Let $\mathfrak{a}$ be an arbitrary closed ideal of $C(K)$. It is easy to see that the set
    %
    \[ C = \{ x \in K : (\forall f \in \mathfrak{a}: f(x) = 0) \} \]
    %
    is closed. Let $x_i \to x$. If $f(x_i) = 0$ for all $x_i$, then $f(x_i) \to f(x)$, so $f(x) = 0$. Let $g$ be an arbitrary continuous function with $g|_C = 0$. Fix $\varepsilon > 0$. Let $V = g^{-1}(B_\varepsilon(0))$. For each $y \in K - C$, there is a function $f_y \in \mathfrak{a}$ with $f_y(y) \neq 0$. We may assume $f_y(y) = g(y)$, since $\mathfrak{a}$ is closed under multiplication by a constant. Let $U_y = (f_y - g)^{-1}(B_\varepsilon(0))$. Then $\{ V \} \cup \{ U_y \}$ is an open cover of $K$, and thus has a finite subcover $V, U_{y_1}, \dots, U_{y_n}$. Let $h_V, h_{y_1}, \dots, h_{y_n}$ be a partition of unity subordinate to the cover. Consider the function $g' = \sum h_{y_i} f_{y_i} \in \mathfrak{a}$. For $x \in K - V$,
    %
    \[ |g'(x) - g(x)| = | \sum h_{y_i}(x) (f(x) - f_{y_i}(x)) | \leq \sum h_{y_i}(x) |f(x) - f_{y_i}(x)| < \varepsilon \]
    %
    For $x \in V$,
    %
    \[ |g'(x) - g(x)| = | \sum h_{y_i}(x) (f(x) - f_{y_i}(x)) - h_V(x) f(x) | < \varepsilon + h_V(x) f(x) \leq 2 \varepsilon \]
    %
    Thus $\| g' - g \|_\infty \leq 2 \varepsilon$. It follows that $g$ can be approximated to arbitrary precision by elements of $\mathfrak{a}$, so $g \in \mathfrak{a}$, since $\mathfrak{a}$ is closed. We conclude $\mathfrak{a}$ consists of all functions which vanish on $C$. Thus closed ideals of $C(K)$ are in one-to-one correspondence with closed sets of $K$. From this, it is fairly easy to identify the maximal ideals. If $C \subset D$, then the ideal corresponding to $D$ is a subset of the ideal corresponding to $C$. The closed ideal corresponding to $\emptyset$ is $C(K)$, so the maximal ideals correspond to the smallest, non-empty compact sets; these are just the points. Thus the character space of $C(K)$ is in one-to-one correspondence with $K$.
\end{example}

\begin{lemma}
    If $M \in GL(A)$ then $M \not \in \text{ker}(\phi)$ for all $\phi \in \Phi_A$. The converse holds in commutative algebras.
\end{lemma}
\begin{proof}
    If $M \in GL(A)$, we know $\phi(M) \neq 0$ for all $\phi \in \Phi_A$, for otherwise $\ker(\phi)$ would contain an invertible element, implying $\ker(\phi) = A$. Conversely, in a commutative ring, if $M \not \in GL(A)$, then $(M) \neq A$, so $(M)$ extends to a maximal ideal $\mathfrak{a}$, and the projection $\phi: A \to A/\mathfrak{a}$ contains $M$ in its kernel, so we have an algebra isomorphism from $A/\mathfrak{a}$ to $\mathbf{C}$ by Gelfand-Mazur, which does not enlarge the kernel.
\end{proof}

\begin{corollary} \label{spectralhomomorphism}
    If $\phi(M) = \lambda$ for some character $\phi$, then $\lambda \in \sigma(M)$. The converse holds for commutative algebras.
\end{corollary}

\begin{example}
    Things can go very wrong for noncommutative algebras. Let us classify the ideals of $M_n(\mathbf{C})$. Let $\mathfrak{a}$ be a left ideal of $M_n(\mathbf{C})$, and suppose that $\mathfrak{a}$ contains a non-zero element $M$. It follows from the elementary theory of matrices that there are matrices $N$ and $L$ such that $NML \in \mathfrak{a}$ is a diagonal matrix non-zero somewhere. By multiplying on the left by a basis matrix $e_{ij}$, and normalizing, we find that $\mathfrak{a}$ contains an element with a single non-zero entry on the diagonal. Multiplying by permutation matrices, we find that $\mathfrak{a}$ contains all matrices with a one in a certain position, and zeroes everywhere else. Finally, by summing up, we find $\mathfrak{a}$ contains all matrices in $M_n(\mathbf{C})$. Thus $M_n(\mathbf{C})$ is a {\it simple} algebra, it contains no non-trivial ideals. Consequently, we find every algebra homomorphism from $M_n(\mathbf{C})$ to another algebra must be injective, for the kernel is a proper ideal. Then, if $n > 1$, the character space of $M_n(\mathbf{C})$ is $\emptyset$; there is no injective homomorphism from $M_n(\mathbf{C})$ to $\mathbf{C}$.
\end{example}

There is a fairly natural topology on $\Phi_A$. We assign to each $M \in A$ the {\bf Gelfand transform} $\widehat{M}: \Phi_A \to \mathbf{C}$, which is defined by $\widehat{M}(\phi) = \phi(M)$. The {\bf Gelfand topology} is the weakest topology on $\Phi_A$ which makes each $\widehat{M}$ continuous. We shall denote the Gelfand transform itself as the map $\Gamma: A \to C(\Phi_A)$.

\begin{lemma}
    The map $M \mapsto \widehat{M}$ is a contractive algebra homomorphism into $C(\Phi_A)$, whose kernel is the Jacobson radical, the intersection of all maximal ideals. Thus the map is injective if and only if the algebra is semisimple. We have $\widehat{M}(\Phi_A) \subset \sigma(M)$, with equality in the commutative case.
\end{lemma}
\begin{proof}
    By direct calculation,
    %
    \[ \widehat{MN}(\phi) = \phi(MN) = \phi(M) \phi(N) = (\widehat{M} \widehat{N})(\phi) \]
    %
    If $\widehat{M} = 0$, then $\phi(M) = 0$ for all $\phi$. Thus $M$ is contained in every maximal ideal. The converse is equally trivial. That $\widehat{M}(\Phi_A) \subset \sigma(M)$ follows from Corollary \ref{spectralhomomorphism}. In the commutative case, if $\lambda \in \sigma(M)$, then $\lambda - M$ is contained in a unique maximal ideal in $A$, which corresponds to a unique homomorphism $\phi \in \Phi_A$ for which $\phi(\lambda - M) = 0$.
\end{proof}

It was Gelfand's idea to surplant the character spaces with a topological structure. This is the Gelfand topology of the character space. We shall now deduce its structure.

\begin{theorem}
    $\Phi_A$ into a compact, Hausdorff space.
\end{theorem}
\begin{proof}
    Let us first verify the Hausdorff condition. Given $\phi, \psi \in \Phi_A$, there is $M \in A$ such that $\phi(M) \neq \psi(M)$ (for otherwise they are equal as functions). If we pick a neighbourhood $U$ of $\phi(x)$ and a neighbourhood $V$ of $\psi(x)$ disjoint from each other, then $\widehat{M}^{-1}(U)$ and $\widehat{M}^{-1}(V)$ are disjoint neighbourhoods in $\Phi_A$ which separate $\phi$ and $\psi$.

    Compactness is a little trickier. We may view $\Phi_A$ as a subset of $A^*$, and $\Phi_A$ has a weaker topology than induced under the relative topology of $A^*$. For each $\phi \in \Phi_A$, $\| \phi \| \leq 1$, so the Gelfand space is contained in the unit ball of $A^*$, which is compact, by the Banach-Alaoglu theorem. Thus we need only verify that $\Phi_A$ is weak $*$ closed in $A^*$. Let $\Lambda$ be in the weak $*$ closure of $\Phi_A$. Fix $M,N \in A$, $\varepsilon > 0$. Consider
    %
    \[ W = \{ \psi \in A^* : | \psi - \Lambda | (z) < \varepsilon\ \text{for}\ z \in \{ 1, M, N, MN \} \} \]
    %
    Then $W$ is a weak $*$ neighbourhood of $\Lambda$, and contains some $\phi \in \Phi_A$. Thus
    %
    \[ |\phi(1) - \Lambda(1)| = | 1 - \Lambda(1) | < \varepsilon \]
    %
    so that $\Lambda(1) = 1$, since $\varepsilon$ was arbitrary. Furthermore,
    %
    \begin{align*}
        \Lambda(MN) -   \Lambda(M) \Lambda(N) &= [\Lambda(MN) - \phi(MN)] + [\phi(M)\phi(N) - \Lambda(M)\Lambda(N)]\\
        &= [\Lambda(MN) - \phi(MN)] + [\phi(N) - \Lambda(N)] \phi(M)\\
        &\ \ \ + [\phi(M) - \Lambda(M)] \Lambda(N)
    \end{align*}
    %
    Hence, since $\phi$ is a complex homomorphism,
    %
    \[ | \Lambda(MN) -   \Lambda(M) \Lambda(N) | < (1 + |\phi(M)| + |\Lambda(N)|) \varepsilon \leq (1 + \| M \| + \| \Lambda \| \| N \|) \varepsilon \]
    %
    Letting $\varepsilon \to 0$, we find $\Lambda(MN) = \Lambda(M) \Lambda(N)$, so $\Lambda \in \Phi_A$, and thus $\Phi_A$ is compact.
\end{proof}

Before we elaborate more on the Gelfand space, let us discuss an important property of semisimple algebras.

\begin{prop}
    Every homomorphism $T: A \to B$ into a semisimple commutative Banach algebra $B$ is continuous.
\end{prop}
\begin{proof}
    Suppose $M_i \to M$, $TM_i \to N$. By the closed graph theorem, we now need only verify that $TM = N$. If $\phi: B \to \mathbf{C}$ is a character of $B$, then $\phi \circ T$ is continuous, and so $\phi(TM) = \phi(N)$. But this implies that $TM - N$ is in the kernel of every character, implying (by the semisimplicity) that $TM = N$. This proves continuity.
\end{proof}

Thus every isomorphism between semisimple Banach algebras is a homeomorphism. The topology of a semisimple algebra is therefore completely determined by its algebraic structure.

\begin{theorem}
    If $A$ is a unital commutative Banach algebra generated by the identity and a single element $M$, then $\Phi_A$ is homeomorphic to $\sigma(M)$.
\end{theorem}
\begin{proof}
    If $\phi: A \to \mathbf{C}$ is a character, then $\phi(M) \in \sigma(M)$, which gives us a map from $\Phi_A$ to $\sigma(M)$. If $\phi(M) = \psi(M)$, then $\phi = \psi$, since thus must agree on all polynomials in $M$, and hence on $A$ by continuity. Thus the map from $\Phi_A$ to $\sigma(x)$ is bijective. The map is continuous, for it is just $\Gamma(M)$. It is then a homeomorphism since $\Phi_A$ is compact.
\end{proof}

Since $\Gamma(M)$ induces the `identity' map on the Gelfand space representation $\sigma(M)$, $\Gamma(\sum a_i M^i) = \sum a_i z^i$. This implies that $\Phi_A$ contains all complex polynomials, and thus $\Gamma(N)$ is the uniform limit of polynomials in $M$, and therefore holomorphic (if $\Phi_A$ is open). Similarily results can be obtained if $A$ is generated by $n$ elements $M_1, \dots, M_n$, in which case the Gelfand space can be represented as a subset of $\mathbf{C}^n$, and $\Gamma(N)$ is the uniform limit of the images of polynomials in the $M_i$.

\begin{example}
    Consider $\delta^1 \in L^1(\mathbf{Z})$. Then $\sigma(\delta^1) = \mathbf{T}$. We have $\sigma(\delta^1) \subset \mathbf{D}$ because $\| \delta^1 \| = 1$. $\delta^1$ has an inverse $\delta^{-1}$, and since $\| \delta^{-1} \| = 1$, $\sigma(\delta^1)^{-1} = \sigma(\delta^{-1}) \subset \mathbf{D}$ (That the inverse of the spectrum is the spectrum of the inverse can be proved easily as an algebraic exercise, but we will soon see it is a special case of a much more general method of mapping spectra), so we conclude $\sigma(\delta^1) \subset \mathbf{T}$. In fact, we have equality. To see this, let us try and invert $\lambda \delta - \delta^1$, for some $|\lambda| = 1$. If $c * (\lambda \delta - \delta^1) = \delta$, then
    %
    \[ \lambda c_n - c_{n-1} = \begin{cases} 1 & n = 0 \\ 0 & n \neq 0 \end{cases} \]
    %
    Solving these equations recursively, we find $c_n = \lambda^{-n} c_0$ for $n \geq 0$, and $c_{-n} = \lambda^{n-1} c_{-1}$. We must have $c \in L^1(\mathbf{Z})$, so that
    %
    \[ |c_0| \sum_{k = 0}^\infty 1 + |c_{-1}| \sum_{k = 1}^\infty 1 < \infty \]
    %
    Hence $c_0 = c_{-1} = 0$, which is impossible if we desire invertibility to hold. Since $\delta^1$ generates $L^1(\mathbf{Z})$, we find that the Gelfand space of $L^1(\mathbf{Z})$ is homeomorphic to $\mathbf{T}$.

    It is interesting to note that if $c \in L^1(\mathbf{Z})$, then $\Gamma(c) = \sum c_n z^n$, so the Gelfand transform is essentially a representation of the Fourier series transform between $\mathbf{Z}$ and $\mathbf{T}$. This follows because $\Gamma(\delta^1)$ induces the identity on $\mathbf{T}$, and the rest follows from linearity.
\end{example}

The key to Gelfand theory is noticing that characters of Banach algebras naturally reflect the structure of the Banach algebra in question. Understanding the character space leads to a natural understanding of the invertibility of the elements of the algebra.

\begin{example}
    Let
    %
    \[ f(z) = \sum a_n z^n \]
    %
    where $\sum |a_n| < \infty$. Then $f$ is the Gelfand transform of the sequence $a = (a_k)$ in $L^1(\mathbf{Z})$. If $f$ never vanishes, then $\sigma(a) = f(\mathbf{T})$ does not contain zero, so $a$ is invertible, and therefore there is a sequence $b \in L^1(\mathbf{Z})$ such that $a * b = \delta$. Then $\Gamma(b) = 1/f$, and so we have shown that
    %
    \[ (1/f)(z) \sim \sum b_n z^n \]
    %
    where $\sum |b_n| < \infty$. Before Gelfand theory, this was an incredibly difficult theorem of Fourier analysis. Similar results can be applied to non-continous functions by approximation, and to higher dimensional trigonometric series, in which case the Gelfand space are the higher dimensional torii $\mathbf{T}^n$.
\end{example}

Wiener had a long, technical and messy proof of the lemma long before Banach algebras were invented. The Gelfand theory reduces the theorem to a few statements. Here's another example from complex analysis.

\begin{example}
    Consider the commutative algebra $A(\mathbf{D})$. We also contend that $\Phi_{A(\mathbf{D})}$ are just the evaluations of functions at a point in the unit disk. The identity function $z$ has norm 1 in this space, implying that if $\phi$ is any homomorphism, then $\phi(z) = w \in \mathbf{D}$. Then
    %
    \[ \phi(\sum_{k = 0}^N a_k z^k) = \sum_{k = 0}^N a_k w^k \]
    %
    and these polynomials are dense in $A(\mathbf{D})$, so $\phi = \phi_w$.

    Now let $f_1, \dots, f_n \in A(\mathbf{D})$ be such that for each $z \in \mathbf{D}$, some $f_i(z)$ is non-zero. Then we claim there are $g_i \in A(\mathbf{D})$ for which $\sum f_i g_i = 1$. It is clear that this is equivalent to the ideal relation
    %
    \[ (f_1, \dots, f_n) = A(\mathbf{D}) \]
    %
    If this were not true, then $(f_1, \dots, f_n)$ would be contained in a closed, maximal ideal, and would therefore be annihilated by some $\phi \in \Phi_{A(\mathbf{D})}$. But $\phi$ corresponds to evaluation at some $w \in \mathbf{D}$, so $f_i(w) = 0$ for all $i$, a contradiction which shows $(f_1, \dots, f_n) = A(\mathbf{D})$. One can apply Runge's theorem to classify the character spaces of arbitrary $A(C)$, but we leave this to the reader.
\end{example}

\subsection{The Non-Unital Gelfand Space}

Let us also address the non-unital case of Gelfand theory. In the Gelfand theory, we have a little trouble defining ideals. We call an ideal $\mathfrak{a}$ (an additive subgroup closed under multiplication) for a non-unital algebra $A$ {\bf modular} if there is $N \in A$ such that $MN - M, NM - M \in \mathfrak{a}$ for all $M \in A$. If $\mathfrak{a}$ is modular, it follows that $A/\mathfrak{a}$ contains an identity. Maximal modular ideals exist with Zorn's help. We need to edit the proof which shows a maximal ideal is closed, which relies on an algebraic trick.

\begin{lemma}
    A maximal modular ideal in an algebra $A$ is closed.
\end{lemma}
\begin{proof}
    If a maximal ideal $\mathfrak{a}$ was not closed, then we would have $\overline{\mathfrak{a}} = A$. Let $N$ be a right modular identity for $\mathfrak{a}$. Then there is $M \in \mathfrak{a}$ with $\| M - N \| < 1$, so
    %
    \begin{align*}
        N &= (N - M) + M = \sum_{k = 1}^\infty (N - M)^k - \sum_{k = 2}^\infty (N - M)^k + M\\
        &= \sum_{k = 1}^\infty (N - M)^k - \left[ \sum_{k = 1}^\infty (N - M)^k \right] (N - M) + M\\
        &= \underbrace{\left[ \left( \sum_{k = 1}^\infty (N - M)^k \right) M + M \right]}_{\in \mathfrak{a}} - \underbrace{\left[ \left( \sum_{k = 1}^\infty (N - M)^k \right) N - \left( \sum_{k = 1}^\infty (N - M)^k \right) \right]}_{\in \mathfrak{a}}
    \end{align*}
    %
    Given any $L \in A$, $LM - M \in \mathfrak{a}$, implying $M \in \mathfrak{a}$. This implies $\mathfrak{a} = A$, an impossibility.
\end{proof}

\begin{lemma}
    The kernel of any {\it nonzero} algebra homomorphism from $A$ to $\mathbf{C}$ is a maximal modular ideal, and in a commutative non-unital algebra, any maximal modular ideal is the kernel of some non-zero algebra homomorphism.
\end{lemma}
\begin{proof}
    If $\phi: A \to \mathbf{C}$ is a non-zero algebra homomorphism, then $\phi$ is surjective, for if $\phi(M) \neq 0$, then $\phi(\mathbf{C} \cdot x) = \mathbf{C}$. We claim that if $\mathfrak{a} = \ker(\phi)$, and if $\phi(M) = 1$, then $M$ is a modular identify for $\mathfrak{a}$, because
    %
    \[ \phi(MN - N) = \phi(M)\phi(N) - \phi(N) = \phi(N) - \phi(N) = 0 \]
    %
    Implying $MN - N \in \mathfrak{a}$ for each $N$, so $\mathfrak{a}$ is a modular ideal. The first isomorphism theorem gives us an isomorphism from $A/\mathfrak{a}$ to $\mathbf{C}$, so $\mathfrak{a}$ is maximal, since $A/\mathfrak{a}$ is a field.

    If $A$ is commutative, and $\mathfrak{a}$ is a maximal modular ideal, then $A/\mathfrak{a}$ is an algebra with identity, whose only ideals consists of $(0)$ and $A/\mathfrak{a}$, since $\mathfrak{a}$ is maximal, so $A/\mathfrak{a}$ is a field, and is therefore isometric to $\mathbf{C}$, giving us an algebra homomorphism.
\end{proof}

We may therefore consider the Gelfand transform of a non-unital algebra, into the character space of all nonzero non-unital algebra homomorphisms.

\begin{lemma}
    If $A$ is an algebra without identity, and $\phi$ is a character, then there is a unique non-zero algebra functional $\tilde{\phi}$ defined on $A \ltimes \mathbf{C}$ which extends $\phi$.
\end{lemma}
\begin{proof}
    Define $\tilde{\phi}(M + \lambda) = \phi(M) + \lambda$. Then $\tilde{\phi}$ is linear, $\tilde{\phi}(1) = 1$, and
    %
    \begin{align*}
        \tilde{\phi}(M + \lambda) \tilde{\phi}(N + \gamma) &= [\phi(M) + \lambda][\phi(N) + \gamma]\\
        &= \phi(MN) + \phi(\lambda N) + \phi(\gamma M) + \lambda \gamma\\
        &= \tilde{\phi}((M + \lambda)(N + \gamma))
    \end{align*}
    %
    The uniqueness of the extension follows from the fact any maximal modular ideal of $A$ can be uniquely extended to a maximal ideal on $A \ltimes \mathbf{C}$, for the projection $M + \lambda \mapsto M$ maps ideals to ideals, and the only ideals of $\mathbf{C}$ are $(0)$ and $(1)$.
\end{proof}

\begin{corollary}
    The Gelfand space extension $\Phi_{A \ltimes \mathbf{C}} = \Phi_A \cup \{ \phi_\infty \}$, where
    %
    \[ \phi_\infty(M + \lambda) = \lambda \]
    %
    and $\sigma(M) \subset \widehat{M}(\Phi_A) \cup \{ 0 \}$, with equality in the commutative case.
\end{corollary}

\begin{example}
    The unitization of $C_0(X)$, where $X$ is locally compact, is equivalent to an embedding in $C(X^*)$. The maximal ideals of $C_0(X)$ correspond to the points in $X$, as well as the `infinity point', which is the one-point added in the compactification.
\end{example}

We may still apply a topology to $\Phi_A$, but it is no longer compact.

\begin{theorem}
    The Gelfand topology on $\Phi_A$ is locally compact when $A$ does not possess an identity, and the Gelfand transform maps $A$ into $C_0(\Phi_A)$.
\end{theorem}
\begin{proof}is the map
    Let $\psi \in \Phi_A$ be arbitrary, and fix $M \in A$ such that $\psi(M) = 1$. We shall verify that
    %
    \[ K = \{ \psi \in \Phi_A : |\psi(M)| \geq 1/2 \} \]
    %
    is compact. We need only verify if it is closed, for if $\psi \in \Phi_A$, and if $\tilde{\psi} \in \Phi_{A \ltimes \mathbf{C}}$ is the unique extension, then $\| \tilde{\psi} \| \leq 1$, so $\| \psi \| \leq 1$. Thus $K$ is contained in the closed unit ball of $A^*$. Since $K = \Lambda^{-1}([1/2, \infty))$, where $\Lambda: A^* \to \mathbf{C}$ is the map $\phi \mapsto |\phi(M)|$, then $\Lambda$ is weak $*$ continuous, so $K$ is closed, hence compact. $K$ contains an open neighbourhood
    %
    \[ U = \{ \psi \in \Phi_A : |\psi(M)| > 1/2 \} \]
    %
    so $\Phi_A$ is locally compact. If $M \in A$, $\widehat{M} \in C_0(\Phi_A)$, since
    %
    \[ \{ \phi \in \Phi_A : |\phi(M)| \geq \varepsilon \} \]
    %
    is compact, and outside of this set $\widehat{M} \leq \varepsilon$.
\end{proof}



\section{The Riesz Calculus}

\subsection{Vector-valued integration}

We would like to define the integral of arbitrary measurable functions $f: \Omega \to X$, where $(\Omega, \mu)$ is a measure space, and $X$ is some Banach space, to be some element $\int f d\mu \in X$. This is not so easy as in the real case, as we have no canonical order through which we can carry summations.

We would like our definition to have properties that ordinary integrals have, namely
%
\[ \Lambda \left( \int f d\mu \right) = \int (\Lambda f) d\mu \]
%
for every $\Lambda \in X^*$. This is certainly true of other measures, since it is true of finite sums, and the integral can be fort of as some limit of a net of sums.

We shall define a function $f: \Omega \to X$ to be {\bf measurable} if $\Lambda f$ is measurable for each $\Lambda \in X^*$, where $\Lambda f = \Lambda \circ f$. This is equivalent to the fact that $(\Lambda \circ f)^{-1}(U)$ is measurable for each measurable subset $U$ of the complex plane, and therefore that $f$ is Borel measurable when $X$ has the weak topology. We shall say that a measurable $f$ is {\bf Pettis Integrable} if there is $x \in X$ for which
%
\[ \Lambda x = \int (\Lambda f) d\mu \]
%
for all $\Lambda \in X^*$. We say $x$ is the integral of $f$, and denote $x$ by $\int f d\mu$. It is clear that $x$ is unique, for if $y$ also has the property, then $x - y$ is annihilated by all linear maps, hence $x - y = 0$.

\begin{theorem}
    Suppose $(\Omega, \mu)$ is a compact measure space with a positive Borel measure, $f: \Omega \to X$ is a continuous map into a Banach space, and $\overline{co}(f(\Omega))$ is compact in $X$, then $f$ is integrable, and $\int f d\mu \in \overline{co}(f(\Omega))$.
\end{theorem}
\begin{proof}
    Assume, without loss of generality, that $\mu$ is a probability measure. Let $H = \text{co}(f(Q))$. For a finite set $L$ is a finite set of functionals, let
    %
    \[ E_L = \left\{ x \in \overline{H} : \Lambda_i(x) = \int (\Lambda_i f) d\mu \right\} \]
    %
    Then $E_L \cap E_{L'} = E_{L \cup L'}$, and each $E_L$ is compact, and therefore has the finite intersection property. If we can show that $E_L$ is non-empty for each $L$, then $\bigcap E_L$ is non-empty, and $\int f d\mu$ exists. Given $L$, fix an ordering $(\Lambda_1, \dots, \Lambda_n)$, which defines a function from $X$ to $\mathbf{R}^n$, and consider $m = (m_1, \dots, m_n)$, where
    %
    \[ m_i = \int (\Lambda_i f) d\mu \]
    %
    We shall show that $m$ is in the convex hull of $(L \circ f)(\Omega)$. If $m$ were not in the hull, then, since $\{ m \}$ is a convex set, there is a linear functional which separates $m$ and the convex hull of $(L \circ f)(\Omega)$, and since we know the structure of linear functionals on $\mathbf{R}^n$, there would be $c_1, \dots, c_n \in \mathbf{R}$ for which
    %
    \[ \sum c_i v_i < \sum c_i m_i \]
    %
    whenever $v = (v_1, \dots, v_n)$ is in the hull. But then
    %
    \[ \sum c_i [(\Lambda_if) (\omega)] < \sum c_i m_i \]
    %
    and integrating,
    %
    \[ \sum c_i m_i = \sum c_i \int (\Lambda_i f) < \sum c_i m_i \]
    %
    Hence $m$ is in the convex hull. Since $L$ is linear, $m = Ly$ for some $y \in H$. At this $y$, we have
    %
    \[ \Lambda_i y = m_i = \int (\Lambda_i f) d\mu \]
    %
    so $y \in E_L$. Thus $\int f d\mu$ exists, and it lies in $\overline{\text{co}}(f(\Omega))$.
\end{proof}

This theorem may be generalized to arbitrary complex valued measures by Jordan decomposition.

\begin{theorem}
    If $X$ is a Banach space, $K \subset X$ is compact, and $\overline{H} = \overline{\text{co}}(K)$, then for each $x \in \overline{H}$ there is a regular Borel probability measure $\mu$ for which $x = \int \text{id}_K d\mu$.
\end{theorem}
\begin{proof}
    The Riesz representation theorem identifies $C(K)^*$ with the space $R$ of all Borel probability measures on $K$. Define $\phi: C(K)^* \to X$ by $\phi(\mu) = \int \text{id}_K d\mu$. The theorem says that $\phi(R) = \overline{H}$. Certainly $K \subset \phi(R)$, for if $\delta_x$ is the unit measure at $x \in K$, then $\phi(\delta_x) = x$, for $\Lambda x = \int \Lambda d \delta_x$. Since $R$ is convex, $\phi(R)$ is convex, and we know from the last theorem that it is a subset of $\overline{H}$. Thus all we need verify is that $\phi(R)$ is closed. This is a consequence of the fact that $R$ is weak $*$ compact in $C(Q)^*$, and $\phi$ is weak $*$ continuous into the weak topology on $X$, so $\phi(R)$ is weakly closed, and all weakly closed sets are strongly closed.

    To see that $R$ is weak $*$ compact, notice that each probability measure has unit variation, and thus lies within the unit ball of $C(K)^*$, which is weak $*$ compact. For $f \in C(K)^*$, let $E_f = \{ \mu : \int f d\mu \geq 0 \}$. Then each $E_f$ is weak $*$ closed, since the map $f \mapsto f(x)$ is weak $*$ continuous. Hence $\bigcap E_f$ is weak $*$ closed. Similarily, the set $E = \{ \mu : \int 1 d\mu = 1 \}$ is weak $*$ closed, and $E \cap \bigcap E_f = R$ is therefore weak $*$ closed.

    It is sufficient to show $\phi$ is weak $*$ continuous at the origin. $X$ has a weak subbasis of neighbourhoods at the origin of the form
    %
    \[ W = \{ x \in X : |\Lambda(x)| < r \} \]
    %
    for $r > 0$, $\Lambda \in X^*$. Thus we must verify that the set of $\mu$ for which $\left| \int \Lambda d\mu \right| < r$ is weak $*$ open. But $\Lambda|_K \in C(K)$, so this set is open almost by definition. Thus $\phi$ is weak $*$ continuous.
\end{proof}

\begin{theorem}
    If $\Omega$ is a compact Hausdorff space with a positive Borel measure $\mu$, $X$ is a Banach space, and $f: \Omega \to X$ is continuous, then
    %
    \[ \left\| \int f d\mu \right\| \leq \int \| f \| d\mu \]
\end{theorem}
\begin{proof}
    For each $\Lambda \in X^*$,
    %
    \[ \left| \Lambda\left(\int f d\mu\right) \right| = \left| \int \Lambda f d\mu \right| \leq \int | \Lambda f | d\mu \leq \| \Lambda \| \int | f | d\mu \]
    %
    If $\Lambda$ is chosen (by the Hahn Banach theorem) such that $\Lambda(\int f d\mu) = \| \int f d\mu \|$, and such that $\| \Lambda \| = 1$, then we obtain the needed inequality.
\end{proof}

If $A$ is a Banach algebra, then $\int f d\mu$ has an additional property, namely, for each $M \in A$,
%
\[ M \int f d\mu = \int Mf d\mu\ \ \ \ \ \ \ \ \ \ \left(\int f d\mu\right) M = \int fM d\mu \]
%
This follows since multiplication on the left and right by $x$ is a bounded linear function. If we let $L: N \mapsto MN$ be the left multiplication operator, and fix $\Lambda \in X^*$, then
%
\[ \Lambda \left( M \left( \int f d\mu \right) \right) = (\Lambda \circ L) \left( \int f d\mu \right) = \int (\Lambda \circ L) f d\mu = \int \Lambda(Mf) d\mu \]
%
the right case is similar.





\subsection{Holomorphic Functional Calculus}

Given a polynomial $P \in \mathbf{C}[X]$, the definition of $P(M)$, for $M \in A$, is obvious -- we just overload the operator. We don't need anything topological here. In this section, we use the topological structure of Banach algebra to evaluate holomorphic functions. For instance, in matrix theory it is useful to define the exponential of $M \in A$ by
%
\[ e^M = \sum_{k = 0}^\infty \frac{M^k}{k!} \]
%
We may define $\sin$ and $\cos$ similarily. More generally, for any power series $f = \sum_{k = 0}^\infty c_k (X - \alpha)^k$ with a radius of convergence $R$, we may define
%
\[ f(M) = \sum_{k = 0}^\infty c_k (M - \alpha)^k \]
%
And this function is defined for any $M$ satisfying $\|M - \alpha \| < R$. It is non-trivial to `overload' arbitrary analytic functions, but it is very useful.

We shall use the formulation of complex analysis in terms of chains, arbitrary finite `integer combinations' of differentiable curves $\gamma = \sum n_i \gamma_i$, where we let $- \gamma_i = \gamma_i^{-1}$ in the path homotopic sense (what we are really doing is considering the free abelian group $\mathbf{Z}\langle C^\infty(\mathbf{R}, \mathbf{C}) \rangle$, modulo the relation $- \gamma = \gamma^{-1}$). The trace of a curve will be its image,
%
\[ \text{tr}(\sum n_i \gamma_i) = \bigcup_{n_i \neq 0} \text{Im}(\gamma_i) \]
%
If $f$ is a continuous function defined on $\text{tr}(\gamma)$ valued in a Banach space, and $\gamma_i$ is defined on $[a_i, b_i]$ we define the integral
%
\[ \int_\gamma f(z) dz = \sum n_i \int_{\gamma_i} f(z) dz = \sum_i n_i \int_{a_i}^{b_i} (f \circ \gamma_i) \gamma_i' \]
%
where the integrals on the right side are Pettis integrals, if they exist.

If $\gamma$ is a chain in $\mathbf{C}$, and $w \in \mathbf{C}$ is given, then we define the {\bf index} of $\gamma$ with respect to $w$, denoted $\text{Ind}_w \gamma$, to be the value
%
\[ \frac{1}{2 \pi i} \int_\gamma \frac{dz}{z - w} \]
%
A chain $\gamma$ is {\bf positively oriented} if $\text{Ind}_w \gamma \in \{ 0, 1 \}$ for each $w \in \mathbf{C}$. We then let
%
\begin{align*}
    \text{Int}(\gamma) = \{ w \in \mathbf{C} : \text{Ind}_w \gamma = 1 \} \\
    \text{Ext}(\gamma) = \{ w \in \mathbf{C} : \text{Ind}_w \gamma = 0 \}
\end{align*}
%
Each is the union of connected components in $\mathbf{C} - \text{tr}(\gamma)$.

\begin{lemma} \label{banachcauchy}
    If $D$ is open, $f: D \to X$ is a holomorphic function into a Banach space, and if $\gamma$ satisfies $\text{Ind}_z \gamma = 0$ for all $z \not\in D$, then $\int_\gamma f(z) dz = 0$.
\end{lemma}
\begin{proof}
    Take $\Lambda \in X^*$. Then $\Lambda f: D \to \mathbf{C}$ is holomorphic, and so
    %
    \[ \int \Lambda f dz = 0 \]
    %
    But this implies that $\Lambda(\int f dz) = 0$ for all $\Lambda$, hence $\int f dz = 0$.
\end{proof}

\begin{lemma}
    Let $D$ be an open set, and $K$ a compact subset. There is a positively oriented chain whose interior contains $K$, and whose exterior contains $D^c$.
\end{lemma}
\begin{proof}
    Split $D$ into its connected components. It suffices to prove the theorem assuming $D$ is a component. Let $\varepsilon = d(K,D^c)$. Then $\varepsilon > 0$, since $K$ is disjoint from $\varepsilon$. Let $G$ be a set of positively oriented squares (a chain consisting of the sum of four clockwise lines), whose interior lies in $D$, and whose vertices are elements of the product $\mathbf{Z}/n \times \mathbf{Z}/n$, where $n > \varepsilon / 2$. If we let $\gamma = \sum_{g \in G} g$, then by construction, $\text{Ind}_z \gamma = 0$ for all $z \in D^c$, and since the interiors of the squares are disjoint, the winding number around each point in $K$ is 1. By construction, we have found a required chain.
\end{proof}

We shall temporarily say that such a chain has the $(K,D)$ separation property. We can now extend holomorphic functions to arbitrary algebras, by artificially introducing the cauchy integral theorem into the algebra. Let $A$ be a Banach algebra, consider an element $M \in A$, $D$ an open set containing all of $\sigma(M)$, and $f:D \to A$ a holomorphic function. Let $\gamma$ be a positively oriented curve satisfying the $(\sigma(M), D)$ separation property, with $\sigma(M) \subset \text{Int}(\gamma)$. Define
%
\[ f(M) = \int_\gamma f(z) (z - M)^{-1} dz \]
%
Then the extension of $f$ is a well-defined holomorphic function. Of course, we must verify the map is well-defined.

\begin{lemma}
    Let $D$ be an open set. If $f: D \to \mathbf{C}$ is holomorphic, and $\gamma, \lambda$ be two chains in $D$ satisfying the $(\sigma(M), D)$ separation property. When $f: D \to A$ is holomorphic, then
    %
    \[ \int_\gamma f(z) (z - M)^{-1} dz = \int_\lambda f(z) (z - M)^{-1} dz \]
\end{lemma}
\begin{proof}
    Then $\gamma - \lambda$ has winding number zero around any point in $\sigma(M)$ and $D^c$, and we have already verified this case in lemma \ref{banachcauchy}.
\end{proof}

We have laid the foundations for the holomorphic functional calculus.

\begin{theorem}
    The map $f \mapsto f(M)$ is an algebra homomorphism from $\mathcal{O}(D)$ into $A$, which is really just evaluation for polynomials, and is continuous in the locally uniform topology.
\end{theorem}
\begin{proof}
    It can easily be verified (plagiarized from the proof in the real case) that uniform convergence preserves convergence of integration. The evaluation map is obviously linear, so we need only show the polynomial evaluation property for monomials $P = X^n$. Select a $(\sigma(M), D)$ chain $\gamma$ far away enough from the origin that we may express inverses in Neumann series. Then
    %
    \begin{align*}
        P(M) &= \frac{1}{2 \pi i} \int_\gamma z^n (z - M)^{-1} dz = \frac{1}{2 \pi i} \int_\gamma z^{n-1} \sum_{k = 0}^\infty \frac{M^k}{z^k} dz\\
        &= \frac{1}{2 \pi i} \sum_{k = 0}^\infty M^k \int_\gamma z^{n - 1 - k} dz = M^n
    \end{align*}
    %
    since $\int_\gamma z^{n-1-k} dz \neq 0$ only when $n - 1 - k = -1$ ($n = k$).

    Now we need to show the multiplicity of the evaluation map. This is done by brute calculation. Let $f,g \in \mathcal{O}(D)$ be given. Pick $\gamma$ be as required, and consider another $\tilde{\gamma}$ satisfying $\text{Int}\ \tilde{\gamma} \supset \overline{\text{Int}\ \gamma}$, $\mathbf{C} - D \subset \text{ext}\ \tilde{\gamma}$. Then
    %
    \begin{align*}
        f(M) g(M) &= \frac{-1}{4 \pi^2} \left( \int_\gamma f(z) (z - M)^{-1} dz \right) \left( \int_{\tilde{\gamma}} g(w) (w - M)^{-1} \right)\\
        &= \frac{-1}{4 \pi^2} \int_\gamma \int_{\tilde{\gamma}} f(z) g(w) (z - M)^{-1} (w - M)^{-1} dw\ dz\\
        &= \frac{-1}{4 \pi^2} \int_\gamma \int_{\tilde{\gamma}} f(z) g(w) \left( \frac{1}{w - z} \left((z - M)^{-1} - (w - M)^{-1} \right) \right) dw\ dz\\
        &= \frac{-1}{4 \pi^2} \int_\gamma f(z) \left( \int_{\tilde{\gamma}} \frac{g(w)}{w - z} dw \right) (z - M)^{-1} dz\\
        &\ \ \ + \frac{1}{4 \pi^2} \int_{\tilde{\gamma}} g(w) \left( \int_\gamma \frac{f(z)}{w - z} dz \right) (w - M)^{-1} dw\\
        &= \frac{1}{2 \pi i} \int_\gamma f(z) \left( \frac{1}{2 \pi i} \int_{\tilde{\gamma}} \frac{g(w)}{w - z} dw \right) (z - M)^{-1} dz\\
        &= \frac{1}{2 \pi i} \int_\gamma f(z) g(z) (z - M)^{-1} dz\\
        &= (fg)(M)
    \end{align*}
    %
    and thus we have an algebra homomorphism.
\end{proof}

In fact, we will show that this homomorphism is the unique one satisfying the properties. This would be trivial if $D = \mathbf{C}$, since we could expand all functions as limits of polynomials via their power series. To prove the theorem in general, we need an advanced theorem in complex analysis, Runge's theorem, which states that any function holomorphic in an open set $D$ is locally uniformly approximated by rational functions whose poles lay outside of $D$. Now if $r_i$ is rational, and $\Lambda$ is a homomorphism such that $\Lambda P = P(M)$, then
%
\[ \Lambda (P/Q) = \Lambda(P) \Lambda(1/Q) = \Lambda(P) \Lambda(Q)^{-1} = P(M) Q(M)^{-1} = (P/Q)(M) \]
%
The homorphism is then uniquely defined, since the $r_i$ are dense in the set of all holomorphic functions.

\begin{example}
    Let $K$ be compact, and let $f \in C(K)$. Let $D$ be an open neighbourhood of $\sigma(f) = f(K)$. Consider the map from $\mathcal{O}(D) \to C(K)$ mapping $g$ to $g \circ f$. This map satisfies the properties of the holomorphic functional calculus, so it really is the holomorphic calculus in disguise.
\end{example}

\begin{theorem}
    Let $f:A \to B$ be a homomorphism between two Banach algebras, $M \in A$. Then $\sigma(f(M)) \subset \sigma(M)$, and if $g$ is holomorphic in a neighbourhood of $\sigma(M)$, then $(g \circ f)(M) = (f \circ g)(M)$.
\end{theorem}
\begin{proof}
    That $\sigma(f(M)) \subset \sigma(M)$ is trivial, for if $\alpha - M$ is invertible, then $f(\alpha - M) = \alpha - f(M)$ is invertible. If $g$ is a polynomial, then the theorem follows from the multiplicative property. But then the theorem is true for limits of polynomials, and hence for all $g$.
\end{proof}

\begin{corollary}
    If $D$ is a neighbourhood of $\sigma(M)$, and $f \in \mathcal{O}(D)$, then
    %
    \[ \widehat{f(M)} = f \circ \widehat{M} \]
\end{corollary}

The ultimatum of the holomorphic functional calculus is the proof of the spectral mapping theorem, that asserts that the `operation' of the spectrum commutes with holomorphic functions. To extend this theorem to non-commutative algebras, we need an algebraic trick. We introduce the center $Z(A)$ of an algebra, which is defined exactly how it is defined in all other algebraic structures. It is a closed subalgebra of $A$, for if $M_i \to M$, and each $M_i \in A$, then
%
\[ MN = \lim M_iN = \lim NM_i = NM \]
%
If $S$ is a subset of $A$, then we may consider
%
\[ Z(S) = \{ M \in A : (\forall N \in S: MN = NM) \} \]
%
which is a commutative subalgebra of $A$. For any subset $S$, the double centralizer $Z(Z(S))$ also contains $S$. $Z(Z(S))$ is commutative if $S \subset Z(S)$, for then $Z(Z(S)) \subset Z(S)$, and thus commutes with itself by definition.

Now consider the commutative subalgebra $B = Z(Z(\{ M \}))$ of an algbera $A$. If $\lambda - M$ is invertible in $A$, there is $N = (\lambda - M)^{-1}$. But then if $K \in Z(\{ M \})$, then $K$ commutes with $\lambda - M$, and
%
\[ KN = N(\lambda - M) K N = NK (\lambda - M) N = NK \]
%
which implies that $\lambda - M$ is invertible in $Z(Z(\{M\}))$, so $\sigma_A(M) = \sigma_B(M)$.

\begin{theorem}[Spectral Mapping Theorem]
    If $\sigma(M) \subset D$, with $D$ open, and $f \in \mathcal{O}(D)$, then $\sigma(f(M)) = f(\sigma(M))$.
\end{theorem}
\begin{proof}
    First, suppose $A$ is commutative. By Gelfand theory,
    %
    \[ \sigma(f(M)) = \widehat{f(M)}(\Phi_A) = f(\widehat{M}(\Phi_A)) = f(\sigma(M)) \]
    %
    If $A$ is not commutative, consider the commutative subalgebra $B = Z(Z(\{M\}))$. If $f$ is analytic in a neighbourhood of $\sigma_A(M)$, then $f(M) \in B$, for there is surely a homomorphism from analytic functions in a neighbourhood of $\sigma_A(M)$ into $B$, and the embedding of $B$ in $A$ produces the unique homomorphism required. Thus we have justified the computation
    %
    \[ \sigma_A(f(M)) = \sigma_B(f(M)) = f(\sigma_B(M)) = f(\sigma_A(M)) \]
    %
    and the spectral mapping theorem is proved in general.
\end{proof}

The holomorphic functional calculus is incredibly useful for it allows us to apply calculations on the complex plane to arbitary algebras. Here are some immediate uses.

\begin{theorem}
    Suppose $A$ is a Banach algebra, $M \in GL(A)$, and $\sigma(M)$ does not separate the origin from $\infty$, then for each $m$, there is $N$ such that $M = N^m$, and there is $N$ such that $M = e^N$. If $\sigma(M) \subset \mathbf{R}^+$, then we may choose the $N$ such that $\sigma(N) \subset \mathbf{R}^+$.
\end{theorem}
\begin{proof}
    Choose a branch of $\sqrt[m]{z}$ holomorphic on a neighbourhood of $\sigma(M)$. If $\sigma(M) \subset \mathbf{R}^+$, choose $\sqrt[m]{z}$ to be the principal branch. Then $\sqrt[m]{M}$ is well defined, and the function $z^m$ is holomorphic everywhere, so $(\sqrt[m]{M})^m$ is defined, and
    %
    \[ \left( {\sqrt[m]{M}}\ \right)^m = (z^m \circ \sqrt[m]{z})(M) = z(M) = M \]
    %
    The spectral mapping theorem implies $\sigma(\sqrt[m]{M}) = \sqrt[m]{\sigma(M)}$, so by the choice of the branch, if $\sigma(M) \subset \mathbf{R}^+$, $\sigma(\sqrt[m]{M}) \subset \mathbf{R}^+$. We may also define $\log(M)$, and $e^{\log(M)} = M$, by similar calculation.
\end{proof}

Even for finite dimensional matrix algebras, this theorem has nontrivial repercussions.

\begin{corollary}
    If $M \in GL_n(\mathbf{C})$, then for each $m$, there is $N \in GL_n(\mathbf{C})$ such that $M = N^m$, and there is $N$ such that $M = e^N$.
\end{corollary}

\begin{example}
    Now consider the exponential function $\exp: A \to A$. If $M$ and $N$ commute, it is easy to see from the power series representation that $\exp(M + N) = \exp(M) \exp(N)$. In particular, this implies that $\exp(M)$ is invertible, and $\exp(-M) = \exp(M)^{-1}$. Thus for each $M$, the image of the map
    %
    \[ t \mapsto \exp(tM) \]
    %
    is a one-parameter subgroup of $G(A)$, which gives us a path from the identity $\exp(0)$ to $\exp(M)$. The connected component of $G(A)$ containing the identity is the {\bf principal component}, denoted $G_0(A)$. Since $N \mapsto MNM^{-1}$ is a continuous map leaving the identity fixed, it maps $G_0(A)$ to itself, so $G_0(A)$ is a normal subgroup. The quotient group $G(A)/G_0(A)$ is sometimes called the {\bf index group}.
\end{example}

\begin{theorem}
    If $P(X) = (X - \alpha_1)^{m_1} \dots (X - \alpha_s)^{m_s}$, and $P(M) = 0$, then $\sigma(M) \subset \{ \alpha_1, \dots, \alpha_n \}$, and if $\Omega$ is an open subset of the plane containing $\alpha_1, \dots, \alpha_n$, then for any $f \in H(\Omega)$, there is a polynomial $Q$ of degree less than $m_1 + \dots + m_s$, such that $f - Q = gP$, for some $g \in H(\Omega)$, so $f(M) = Q(M)$.
\end{theorem}
\begin{proof}
    Applying the spectral theorem, we find $\{ 0 \} = \sigma(P(M)) = P(\sigma(M))$, from which the spectrum of $M$ is restricted to lie in the roots of $P$. The Laurent series about each $\alpha_i$ gives constants $c_{it}$ such that
    %
    \[ g(z) = f(z)/P(z) - \sum_{i = 1}^s \sum_{k = 1}^{m_i} \frac{c_{it}}{(z - \alpha_i)^k} \]
    %
    is holomorphic in $\Omega$, and
    %
    \[ gP = f - \sum_{i = 1}^s \sum_{k = 1}^{m_i} c_{it} P (z - \alpha_i)^{-k} \]
    %
    constructing $Q$ with degree less than $n$.
\end{proof}

In non-unital algebras, we cannot evaluate any polynomial of the form $\sum a_i z^i$, since $a_0 \in \mathbf{C}$ does not necessarily have an interpretation in the algebra. If $a_0 = 0$, we can evaluate $\sum a_i z^i$, and thus any holomorphic function $f$, provided $f(0) = 0$. To verify this, one need only use the simple trick of switching from a non-unital algebra $A$ to a unital algebra $\mathbf{C} \ltimes A$.






\chapter{Involutive Banach Algebras}

An important property of the complex numbers is the existence of the conjugation operation $z \mapsto \overline{z}$, which allows us to `reverse' the orientation of the space, and can therefore be viewed as a function which enables us to study non-holomorphic functions on the space. We shall find that Banach algebras that have a `conjugation operation' have a much richer theory, analogous to `infinite dimensional' complex theory.

Define an {\bf involution} on a Banach algebra $A$ to be an antilinear map $M \mapsto M^*$, which satisfies
%
\[ M^{**} = M\ \ \ \ \ (MN)^* = N^*M^* \]
%
A Banach algebra with a fixed involution possesses a canonical way to reverse the rotational structure provided by the complex structure on the algebra. A {\bf Banach $*$ Algebra} is a Banach algebra with a fixed involution. Note that an arbitrary involution could be highly discontinuous, so we normally require a least some structure on our $*$ algebras. A {\bf $\mathbf{C}^*$ algebra} is a Banach $*$ algebra $A$ which satisfies
%
\[ \| M^* M \| = \| M \|^2 \]
%
for all $M \in A$.

\begin{example}
    Let $H$ be a Hilbert space. Then the map $T \mapsto T^*$ gives an involution in $B(H)$, where $T^*$ is defined by the equation $\langle T^* x, y \rangle = \langle x, Ty \rangle$. Now if $\| x \|, \| y \| \leq 1$, then
    %
    \[ | \langle T^* T x, y \rangle | = | \langle Tx, Ty \rangle | \leq \| T \|^2 \]
    %
    and this is attained, for if there are $\| x_i \| \leq 1$ such that $\| Tx_i \| \to \| T \|$, then
    %
    \[ \langle T^* T x_i, x_i \rangle = \| Tx_i \|^2 \to \| T \|^2 \]
    %
    so $\| T^* T \| = \| T \|^2$. If we are working over $\mathbf{C}^n$, then each element of $B(\mathbf{C}^n)$ can be expressed as a matrix in $M_n(\mathbf{C})$ If $(a_{ij})$ is such a matrix, then one verifies that $(a_{ij})^* = (\overline{a_{ji}})$, so involution here is analogous to the complex transpose on sets of matrices.
\end{example}

\begin{example}
    The map $f \mapsto \overline{f}$ is an involution in $C_0(X)$, where $X$ is a locally compact Hausdorff space, and
    %
    \[ \| f \overline{f} \|_\infty = \| |f|^2 \|_\infty = \| f \|^2_\infty \]
    %
    The central result of commutative involution theory shows that all commutative $*$ algebras are isometric to a subspace of $C_0(X)$.
\end{example}

\begin{example}
    For $f \in A(\mathbf{D})$, define $f^*(z) = \overline{f(\overline{z})}$, then $*$ is an isometric involution. If $f(z) = e^{iz}$, then
    %
    \[ (ff^*)(z) = e^{iz} \overline{e^{i\overline{z}}} = e^{iz} e^{-iz} = 1 \]
    %
    Thus $\| f f^* \|_\infty = 1$, yet $f(1) = f^*(-1) = e > 1$, so
    %
    \[ \| f \|_\infty \| f^* \|_\infty > \| ff^* \|_\infty \]
    %
    which implies $A(\mathbf{D})$ is not a $C^*$ algebra.
\end{example}

\begin{example}
    $L^1(G)$ has the involution $\overline{f}(g) = \overline{f(g^{-1})}$. It also has the involution $\overline{f}(g) = \overline{f(g)}$, but it turns out the first is much more important to the theory of groups. Neither give $L^1(G)$ the structure of a $C^*$ algebra.
\end{example}

\begin{prop}
    In a $C^*$ algebra, $\| M \| = \| M^* \|$.
\end{prop}
\begin{proof}
    Note that
    %
    \[ \| M \|^2 = \| M M^* \| \leq \| M \| \| M^* \| \]
    %
    so $\| M \| \leq \| M^* \|$. But then, by symmetry,
    %
    \[ \| M^* \| \leq \| M^{**} \| = \| M \| \]
    %
    so we have equality.
\end{proof}

For many Banach algebras, involutions will always be continuous, but $C^*$ algebras are still much nicer objects to study.

\begin{prop}
    If a Banach algebra $A$ is commutative and semisimple, then every involution on $A$ is continuous.
\end{prop}
\begin{proof}
    If $T: A \to A$ is an involution, and $\phi$ is a character on $A$, then $\overline{\phi \circ T}$ is a character, and therefore continuous. This implies that if $a_i \to a$, and $Ta_i \to b$, then $\overline{\phi(Ta)} = \overline{\phi(b)}$, hence $\phi(Ta) = \phi(b)$. Since this is true for every character $\phi$, $Ta = b$.
\end{proof}

\begin{prop}
    For any $M$ in a $C^*$ algebra,
    %
    \[ \| M \| = \sup \{ \| MN \| : \| N \| \leq 1 \} = \sup \{ \| NM \| : \| N \| \leq 1 \} \]
\end{prop}
\begin{proof}
    If $N = M^* \|M\|^{-1}$, $\| N \| = 1$, and
    %
    \[ \| MN \| = \| MM^* \| \| M \|^{-1} = \| M \| \]
    %
    Thus the suprema on the right is greater than or equal to $\| M \|$. But if $\| N \| \leq 1$, then
    %
    \[ \| MN \| \leq \| M \| \| N \| \leq \| M \| \]
    %
    so the suprema is also less than or equal to $\| M \|$.
\end{proof}

It follows that we may isometrically embed any $C^*$-algebra $A$ into $B(A)$, by either considering the left or right multiplications. This is known as the {\bf left regular representation}. Thus if $A$ is a $C^*$ algebra without identity, then we may embed $A$ in $B(A)$, and then consider the smallest closed $C^*$-subalgebra of $B(A)$ which contains $A$ and the identity. The representation theory of $C^*$ algebras is quite rich, and we will soon learn about the Gelfand Naimark Segal theorem, which gives rise to a universal representation of $A$ as a subalgebra of operators on a Hilbert space.

Given a (nonunital) $C^*$ algebra $A$, there is a natural $C^*$ structure on $A^\#$, given by
%
\[ (\lambda + M)^* = \overline{\lambda} + M^* \]
%
given that we define a new norm
%
\[ \| \lambda + M \| = \sup \{ \| MN + \lambda \| : N \in A, \| N \| \leq 1 \} \]
%
The above proposition verifies that the embedding of $A$ in $A^\#$ is an isometry. Thus the unital theory essentially includes the non-unital theory.

The theory of $C^*$ algebras is closely related to operator theory. In fact, one can show that all $C^*$ algebras are isometric to some $C^*$-subalgebra of $B(H)$, where $H$ is a Hilbert space. Thus it makes sense to extend terminology from Hilbert space theory to the $C^*$ domain. We say $M \in A$ is {\bf self-adjoint} or {\bf hermitian} if $M^* = M$. The set of all such elements forms a real subalgebra of $A$. The subalgebra of Hermitian elements in a larger complex $C^*$ algebra are analogous to the inclusion of real numbers in the complex numbers. Here is another result extending this analogy.

\begin{prop}
    In a Banach $*$ algebra, any $M \in A$ can be uniquely written as $M = T + iS$, where $T$ and $S$ are self-adjoint elements.
\end{prop}
\begin{proof}
    Write
    %
    \[ T = \frac{M + M^*}{2}\ \ \ \ \ S = \frac{M - M^*}{2i} \]
    %
    Then $T + iS = M$, and they are trivially verified to be self-adjoint. Now suppose $M = N + iL$, where $N$ and $L$ are self-adjoint. Then
    %
    \[ 0 = (N - T) + i(L - S) \]
    %
    is normal. Thus
    %
    \[ (N - T) + i(L - S) = (N - T) - i(L - S) \]
    %
    And $i(L - S) = -i(L - S)$, which can only occur when $L - S = 0$. But then $N - T = 0$ also.
\end{proof}

An element $M$ is {\bf normal} if
%
\[ M^* M = MM^* \]
%
These elements are important because the smallest $C^*$ subalgebra containing $M$ will be commutative, in which case we can apply Gelfand theory to its full extent. {\bf Unitary} elements satisfy
%
\[ MM^* = M^*M = 1 \]
%
the set of all such elements forming a closed subgroup $U(A)$ of $GL(A)$. A {\bf projection} is a hermitian element $M$ such that $M^2 = M$. Thus the terminology from operator theory carries across to arbitrary $*$ algebras, along with some of the interesting properties.

\begin{prop}
    If $A$ is a $C^*$ algebra, and $M \in A$ is normal, then $\| M \| = r(M)$.
\end{prop}
\begin{proof}
    A calculation reveals
    %
    \[ \| M^2 \|^2 = \| (M^2)^* M^2 \| = \| (M^* M) (M^* M)^* \| = \| M^* M \|^2 = \| M \|^4 \]
    %
    Thus $\| M^2 \|^{1/2} = \| M \|$, and more generally by induction,
    %
    \[ \| M^{2^k} \|^{1/2^k} = \left( \| (M^{2^{k-1}})^2 \|^{1/2} \right)^{1/2^{k-1}} = \| M^{2^{k-1}} \|^{1/2^{k-1}} = \| M \| \]
    %
    which implies
    %
    \[ r(M) = \lim_{k \to \infty} \| M^k \|^{1/k} = \lim_{k \to \infty} \| M^{2^k} \|^{1/2^k} = \| M \| \]
    %
    since $\| M^{2^k} \|^{1/2^k}$ is a subsequence of $\| M^k \|^{1/k}$.
\end{proof}

\begin{corollary}
    Every $C^*$ algebra $A$ has a unique norm making it a $C^*$ algebra.
\end{corollary}
\begin{proof}
    The spectral radius $r(M)$ is invariant of any particular $M \in A$. Thus if $\| \cdot \|$ is any $C^*$ norm, then
    %
    \[ \| M \|^2 = \| M^*M \| = r(M^*M) \]
    %
    and thus the norm is uniquely defined.
\end{proof}

In a commutative $C^*$ algebra, every element is trivially normal, so

\begin{corollary}
    The Gelfand transform from a commutative $C^*$ algebra $A$ to $C(\Phi_A)$ is an isometry.
\end{corollary}
\begin{proof}
    $\widehat{M}(\Phi_A) = \sigma(M)$, so $\|\widehat{M}\|_\infty = r(M) = \| M \|$.
\end{proof}

Gelfand's main result is that the isometry is actually surjective, which means all commutative $C^*$ algebras are just algebras of continuous functions in disguise. The surjectivity is the tricky part, which requires some work.

\begin{prop}
    In a $C^*$ algebra, if $M$ is unitary, then $\sigma(M) \subset \mathbf{T}$.
\end{prop}
\begin{proof}
    Since
    %
    \[ \| M \|^2 = \| M^* M \| = \| 1 \| = 1 \]
    %
    and because $M$ is normal, $r(M) = \| M \| = 1$. But $M^{-1} = M^*$ is also unitary, so $r(M^{-1}) = 1$, and by the spectral mapping theorem,
    %
    \[ \sigma(M^{-1}) = \sigma(M)^{-1} \]
    %
    which implies that both spectra lie on $\mathbf{T}$.
\end{proof}

\begin{prop}
    If $M$ is self-adjoint, then $\sigma(M) \subset \mathbf{R}$.
\end{prop}
\begin{proof}
    The operator $e^{iM}$ is unitary, for
    %
    \[ e^{-iM} = \sum_{k = 0}^\infty \frac{(-i)^k M^k}{k!} = \sum_{k = 0}^\infty \left( \frac{(iM)^k}{k!} \right)^* = (e^{iM})^* \]
    %
    The spectral mapping theorem implies that $\sigma(e^{iM}) = e^{i \sigma(M)} \subset \mathbf{T}$, which implies $\sigma(M) \subset \mathbf{R}$.
\end{proof}

\begin{prop}
    If $A$ is a $C^*$ algebra, and $\phi$ a character on $A$, then
    %
    \begin{enumerate}
        \item[(a)] $\phi(M) \in \mathbf{R}$ if $M$ is self-adjoint.
        \item[(b)] $\phi(M^*) = \phi(M)^*$.
        \item[(c)] $\phi(M^*M) \geq 0$ for all $M \in A$.
        \item[(d)] If $M$ is unitary, then $|\phi(M)| = 1$.
    \end{enumerate}
\end{prop}
\begin{proof}
    (a) and (d) follow from Gelfand theory, since we may unitize our algebra, adding at most one extra chracter which is uniformly zero on $A$. Now for any $M$, write $M = T + iS$, and then
    %
    \[ \phi(M^*) = \phi(T - iS) = \phi(T) - i \phi(S) = \overline{\phi(T) + i \phi(S)} = \overline{\phi(M)} \]
    %
    (c) follows immediately.
\end{proof}

The next result fully characterizes commutative $C^*$ algebras. It is really only a more basic version of the more general Gelfand-Naimark theorem, which shows that an arbitrary $C^*$ algebra is isometrically $*$ isomorphic to a $C^*$ subalgebra of a Hilbert space.

\begin{theorem}[Gelfand-Naimark]
    Let $A$ be a commutative $C^*$ algebra. Then the Gelfand transform is a $*$-isomorphism between $A$ and $C(\Phi_A)$.
\end{theorem}
\begin{proof}
    We shall show $\hat{A}$ is dense in $C(\Phi_A)$. It is surely a subalgebra. Given $M \in A$, write $M = N + iL$, with $N$ and $L$ self adjoint. Now
    %
    \[ \widehat{M^*}(\phi) = \overline{\widehat{M}(\phi)} \]
    %
    Thus $\widehat{A}$ is a subalgebra of $C(A)$ closed under conjugation, and which separates points. The complex Stone Weirstra{\ss} tells us $\widehat{A}$ is dense in $C(\Phi_A)$, and since it is closed, we obtain that $\widehat{A} = C(\Phi_A)$.
\end{proof}

The Gelfand Naimark theorem has two important applications. Recall that a space $X$ is {\bf completely regular} if it is $T1$, and if for each closed $C$ and $x \in X - C$, there is $f \in C(X)$ for which $f|_F = 0$, and $f(x) = 1$. Urysohn's lemma tells us all normal spaces (for instance, those which are compact and Hausdorff, or metrizable) are completely regular.

\begin{theorem}
    Let $X$ be a completely regular space. Then there is a compact Hausdorff space $\beta X$ with an embedding $i: X \to \beta X$ onto a dense subset of $\beta X$ satisfying the following universal property; $j: X \to K$ is a continuous map into a compact Hausdorf space, then there is a unique $f: \beta X \to K$ such that
    %
    \begin{center}
    \begin{tikzcd}
        X \arrow{r}{i} \arrow{d}{j} & \beta X \arrow{ld}{f} \\
        K
    \end{tikzcd}
    \end{center}
    %
    The universal property means $\beta X$ is unique up to homeomorphism.
\end{theorem}
\begin{proof}
    The map $f \mapsto \overline{f}$ is an isometric involution on $C(X)$. Then $C(X)$ is a commutative $C^*$ algebra with identity, so $\Phi_{C(X)}$ is compact and Hausdorff. The map
    %
    \[ i: X \to \Phi_{C(X)}\ \ \ \ \ \ \ \ \ \ x \mapsto \phi_x \]
    %
    is continuous and injective. Fix $\phi \in \beta X - \overline{i(X)}$, if it exists. Pick $f \in C(\Phi_A)$ such that $f|_{\overline{i(X)}} = 0$, and $f(\phi) = 1$. But since $\widehat{\cdot}$ is an isomorphism, $f = \widehat{g}$ for some $g \in C(X)$, so $g(x) = 0$ for all $x \in X$, so $\widehat{g} = f = 0$, a contradiction, since $f \not \in i(X)$. Thus $i(X)$ is dense in $\beta X$

    Now we show $i$ is an embedding. Let $\{ x_\alpha \}$ be a net in $X$, and suppose $i(x_\alpha) \to i(x)$. Assume $x_\alpha$ does not tend to $x$. Then there is a neighbourhood $U$ of $x$ and a subnet $\{ x_\beta \}$ of $\{ x_\alpha \}$ such that $x_\beta \not \in U$ for all $\beta$. Since $X$ is completely regular, choose $f \in C(X)$ such that $f|_{X - U} = 0$, and $f(x) = 1$. Then $\widehat{f}(\phi_{x_\beta})$ does not tend to $\widehat{f}(\phi_x)$ even pointwise, and, consequently by the weak topology on $\Phi_{C(X)}$, $i(x_\beta)$ does not tend to $i(x)$, a contradiction.

    Finally, let $K$ be any compact Hausdorff space, and let $j: X \to K$ be continuous. Define
    %
    \[ j^*: C(K) \to C(X)\ \ \ \ \ f \mapsto f \circ j\ \ \ \ \ \ \ \ \ \ \widehat{j}: \Phi_{C(X)} \to \Phi_{C(K)} \cong K\ \ \ \ \ \phi \mapsto \phi \circ j^* \]
    %
    then $\widehat{j}$ is continuous, and for any $f \in C(K)$,
    %
    \[ [(\widehat{j} \circ i)(x)](f) = [\phi_x \circ j^*](f) = f(j(x)) \]
    %
    Thus if $\Gamma$ is the Gelfand transform from $\Phi_{C(K)}$ to $K$, then
    %
    \[ \Gamma^{-1}((\widehat{j} \circ i)(x)) = j(x) \]
    %
    so $g \circ \widehat{j} \circ i = j$, thus $g \circ \widehat{j}$ is a map making the diagram commute, and it must be unique, since $i(X)$ is dense in $\Phi_{C(X)}$.
\end{proof}

\begin{theorem}
    If  $B$ is a $C^*$-subalgebra of $A$, then $GL(A) = GL(B)$.
\end{theorem}
\begin{proof}
    Any involution satisfies $1^* = 1$, since the involution is linear, and satisfies, for any $w \in \mathbf{C}$,
    %
    \[ w^* = (w1)^* = w^*1^* \]
    %
    If $N \in GL(A)$, then $N^* \in GL(A)$, for
    %
    \[ N^*(N^{-1})^* = (N^{-1}N)^* = 1^* = 1\ \ \ \ \ \ \ \ (N^{-1})^* N^* = (NN^{-1})^* = 1^* = 1 \]
    %
    Thus $NN^*$ is an invertible, self adjoint operator, so $\sigma_A(NN^*) \subset \mathbf{R}$, implying $\sigma_A(NN^*) = \sigma_B(NN^*)$, which implies $NN^* \in GL(B)$. But then $N$ has a right inverse in $B$. Computing with $N^*N$ instead we find $N$ has a left inverse, so $N \in GL(B)$.
\end{proof}

\begin{corollary}
    If $A$ is a $C^*$ algebra, and $B$ is a $C^*$-subalgebra, then
    %
    \[ \sigma_A(M) = \sigma_B(M) \]
    %
    so spectral theory over $C^*$-algebras is invariant of a particular $C^*$ algebra.
\end{corollary}

For normal operators, we can extend the holomorphic functional calculus to all continuous functions. Let $A$ be a $C^*$ algebra, and let $M \in A$. We let $C^*(M)$ be the smallest unital $C^*$ algebra containing $M$. It can be described as the smallest closed subalgebra which contains $1$, $M$, and $M^*$. Certainly, the latter is contained in the former. Consider a monomial, consider of products of $M$ and $M^*$. Then the involution of a monomial is a monomial, and by linearity, the span is closed under involution, hence the closed span is also closed under involution.

\begin{theorem}[Continuous Functional Calculus]
    Let $M$ be a normal element of a $C^*$ algebra $A$. Then there is a unique isometric $C^*$-isomorphism between $C(\sigma(M))$ and $C^*(M)$ such that
    %
    \[ 1 \mapsto 1\ \ \ \ \ \text{id}_{\sigma(M)} \mapsto A\ \ \ \ \ \overline{\text{id}_{\sigma(M)}} \mapsto A^* \]
\end{theorem}
\begin{proof}
    The uniqueness assumption is clear, since $C(\sigma(M))$ is generated by the identity function by the Stone Weirstra{\ss} theorem. Since $M$ is normal, $C^*(M)$ is a commutative $C^*$ algebra, so the Gelfand transform gives us an isometric isomorphism from $C^*(M)$ to $C(\Phi_{C^*(M)})$. But the map
    %
    \[ \widehat{M}: \Phi_{C^*(M)} \to \Sigma(M) \]
    %
    is a continuous bijection (for $C^*(M)$ is commutative), which therefore must be a homeomorphism. Then,
    %
    \[ f \mapsto \widehat{\cdot}\ ^{-1}(f \circ \widehat{M}) \]
    %
    is what we desire, where $\widehat{\cdot}\ ^{-1}$ is the inverse Gelfand transform.  We write the image of $f$ under this map as $f(M)$.
\end{proof}

\begin{corollary}
    If $A$ is a $C^*$ algebra, and $M \in A$ is normal, and let $f \in C(\sigma(M))$. Then
    %
    \[ f(\sigma(M)) = \sigma(f(M)) \]
\end{corollary}
\begin{proof}
    We just apply Gelfand theory.
    %
    \[ \sigma(f(M)) = \sigma(\Gamma^{-1}(f \circ \widehat{M})) = (f \circ \widehat{M})(\Phi_{C^*(M)}) = f(\sigma(M)) \]
    %
    Thus the spectrum of normal operators is nicely defined under spectral mappings.
\end{proof}

For Banach $*$ algebras, Gelfand theory is not nearly as successful. We need not even have $\phi(M^*) = \overline{\phi(M)}$, for instance, in $A(\mathbf{D})$ where $f^*(z) = \overline{f}(\overline{z})$, or in $L^1(\mathbf{Z})$, where $(a^*)_n = \overline{a}_n$. We call an algebra with this property a {\bf Symmetric Algebra}. If we define $(a^*)_n = \overline{a_{-n}}$, then $L^1(\mathbf{Z})$ becomes a symmetric algebra, because every character is of the form $\phi(a) = \sum a_k z^k$, for $z \in \mathbf{T}$, and
%
\[ \overline{\phi(a)} = \sum \overline{a_k} z^{-k} = \sum \overline{a_{-k}} z^k = \phi(a^*) \]
%
which is one of the reaons why this involution is so much more useful than pointwise conjugation.




\section{Positivity}

\subsection{Positive Elements of Algebras}

A self-adjoint element $M$ is {\bf positive} if $\sigma(M) \subset [0, \infty)$. In shorthand, we write $M \geq 0$. The set of all positive elements in an algebra $A$ is denoted $A_+$. We write $M \leq N$ if $N - M \geq 0$. The identification of positive elements gives us a useful ordering on the self adjoint elements of $A$, analogous to the ordering of the real numbers.

\begin{example}
    If $f \in C(X)$, then $f$ is normal if and only if $f$ is real-valued. Similarily, $f$ is positive if and only if $f \geq 0$, in the usual sense, for this occurs if and only if $\overline{f(X)} \subset [0,\infty)$.
\end{example}

\begin{prop}
    If $M \in A_+$, there is a unique $N \in A_+$ for which $M = N^2$.
\end{prop}
\begin{proof}
    Consider the map $f(z) = \sqrt{z}$. This is well defined, since $\sigma(M) \subset [0,\infty)$, and $f \in C(\sigma(M))$, so we may consider $f(M)$. Now
    %
    \[ f(M)^2 = (f^2)(M) = (\text{id}_{\sigma(M)})(M) = M \]
    %
    If $N^2 = M$, then
    %
    \[ NM = N^3 = (N^2)N = MN \]
    %
    similarily, $(N^*)^2 = (N^2)^* = M^*$, and since $N$ is normal,
    %
    \[ NM^* = NN^*N^* = N^*N^*N = M^*N \]
    %
    so, since $f(M)$ is a limit of certain sums of monomials of $M$ and $M^*$, under which $N$ commutes,
    %
    \[ Nf(M) = f(M)N \]
    %
    Consider $B = C^*(N, f(M))$. This is a commutative sub $C^*$-algebra of $A$. Applying the Gelfand transform, for any $\phi \in \Phi_B$, $\widehat{N}(\phi) \geq 0$, since $N$ is a positive operator. Similarily, $\widehat{f(A)}(\phi) \geq 0$. Since
    %
    \[ \widehat{N}(\phi)^2 = \widehat{M}(\phi) = \widehat{f(A)}(\phi)^2 \]
    %
    we must have $\widehat{N}(\phi) = \widehat{f(A)}(\phi)$, hence $\widehat{N} = \widehat{f(A)}$, so $N = f(A)$, since the Gelfand transform is injective for commutative algebras.
\end{proof}

We denote the unique positive square root of $M \in A_+$ by $\sqrt{M}$. If self-adjoint operators behave like real numbers, positive operators behave like positive real numbers.

\begin{prop}
    Let $A$ be a $C^*$ algebra, and let $M \in A$ be self adjoint. Then there are unique $M_+$, $M_- \in A_+$ for which $M = M_+ - M_-$, and $M_+ M_- = 0$.
\end{prop}
\begin{proof}
    Let $f_+(t) = \text{max}(t,0)$, and $f_-(t) = -\text{min}(t,0)$. Then $f_+$ and $f_-$ are continuous maps, and we may consider $f_+(M)$, $f_-(M)$. Since
    %
    \[ f_+ - f_- = \text{id}_{\sigma(M)} \]
    %
    $f_+(M) - f_-(M) = M$. By the spectral mapping theorem,
    %
    \[ \sigma(f_+(M)) = f_+(\sigma(M)) \subset [0,\infty)\ \ \ \ \ \sigma(f_-) = f_-(\sigma(M)) \subset [0,\infty) \]
    %
    so $f_+(M), f_-(M) \geq 0$. Since $f_+ f_- = 0$, $f_+(M) f_-(M) = 0$.

    To prove uniqueness, suppose $L - U = M$, and $LU = 0$. Let $B = C^*(L,U)$. Then $M \in B$, and thus $M_+$, $M_- \in B$. For any $\phi \in \Phi_B$, either $\widehat{L}(\phi) = 0$, or $\widehat{U}(\phi) = 0$. If $\widehat{M}(\phi) \geq 0$, then $\widehat{U}(\phi) = 0$, so $\widehat{L}(\phi) = \widehat{M}(\phi)$, and if $\widehat{M}(\phi) \leq 0$, then $\widehat{L}(\phi) = 0$, so $\widehat{U}(\phi) = -\widehat{M}(\phi)$. But then $\widehat{L} = \widehat{f_+(M)}$ and $\widehat{U} = \widehat{f_-(M)}$. Then equality is obtained, since $B$ is commutative, because $L$ and $U$ commute.
\end{proof}

\begin{example}
    Suppose $M$ is a positive element of $B(H)$. Then, for any $x \in H$,
    %
    \[ \langle Mx,x \rangle = \langle \sqrt{M}\sqrt{M}x,x \rangle = \langle \sqrt{M}x, \sqrt{M}x \rangle \geq 0 \]
    %
    Conversely, let $M \in B(H)$ be an operator such that, for any $x \in H$,
    %
    \[ \langle Mx, x \rangle \geq 0 \]
    %
    Write $M = N + iL$, with $N$ and $L$ self adjoint.Then
    %
    \[ \langle Mx, x \rangle = \langle Nx, x \rangle + i \langle Lx, x \rangle \geq 0 \]
    %
    implying $\langle Lx, x \rangle = 0$ for all $x$, so $L = 0$, and $M = N$ is self-adjoint. The last proposition implies that we may write $M = M_+ - M_-$. Then
    %
    \[ 0 \leq \langle MM_-x, M_-x \rangle = - \langle M_-^2 x, M_- x \rangle \leq 0 \]
    %
    Since $H = \overline{M_-H} \oplus \ker M_-$, we find that
    %
    \[ \langle M_-x, x \rangle = 0 \]
    %
    for all $x$, implying $M_- = 0$, so $M = M_+$ is positive. Thus positive operators have a nice characterization on Hilbert spaces.
\end{example}

\begin{lemma}
    If $M$ is self-adjoint, and $\| M - t \| \leq t$, then $M$ is positive. If $M$ is positive, and $\| M \| \leq t$, then $\| M - t \| \leq t$.
\end{lemma}
\begin{proof}
    Suppose $\| M - t \| \leq t$. Then $\sigma(M - t) \in [-t,t]$, so
    %
    \[ \sigma(M) = \sigma(M - t) + t \in [0, 2t] \]
    %
    Conversely, if $M$ is positive and $\| M \| \leq t$, then $\sigma(M) \subset [0,t]$, and
    %
    \[ \sigma(M - t) = \sigma(M) - t \subset [-t,0] \]
    %
    which means $\| M - t \| = r(M - t) \leq t$.
\end{proof}

\begin{prop}
    If $A$ is a $C^*$ algebra, then
    %
    \begin{enumerate}
        \item[(a)] $A_+$ is closed.
        \item[(b)] If $M,N \in A_+$, then $M + N \in A_+$.
        \item[(c)] If $M \in A_+$, and $t \geq 0$, then $tM \in A_+$.
        \item[(d)] $A_+ \cap -A_+ = (0)$.
    \end{enumerate}
\end{prop}
\begin{proof}
    Suppose that $N \in \overline{A_+}$, choose $K > \| N \|$. Fix $\varepsilon > 0$ and pick $M \in A_+$ with $\| N - M \| < \varepsilon$. Then if $\varepsilon$ is chosen small enough, then $\| M \| < K$, and
    %
    \[ \| N - K \| = \| N - M \| + \| M - K \| \leq \varepsilon + K \]
    %
    Letting $\varepsilon \to 0$, we find $N \in A_+$. To prove (b), we apply the inequality in the lemma to conclude
    %
    \[ \| M + N - (\| M \| + \| N \| ) \| \leq \| M - \| M \| \| + \| N - \| N \| \| \leq \| M \| + \| N \| \]
    %
    So $M + N \in A_+$. Since $\sigma(\lambda M) = \lambda \sigma(M)$, we obtain (c). To prove (d), suppose $M \in A_+ \cap -A_+$. Then $\sigma(M) = \{ 0 \}$, and since $M$ is normal, $\| M \| = r(M) = 0$.
\end{proof}

The next corollary says that `$| x | \leq 0$ if and only if $x = 0$', but in a $C^*$-algebra setting.

\begin{lemma}
    If $-M^*M \in A_+$, then $M = 0$.
\end{lemma}
\begin{proof}
    Since
    %
    \[ \sigma(MM^*) \cup \{ 0 \} = \sigma(M^*M) \cup \{ 0 \} \]
    %
    which can be proved by some algebraic tricks, $-MM^* \in A_+$ as well. If we write $M = T + iS$, where $T$ and $S$ are normal, then
    %
    \[ M^*M + MM^* = (T - iS)(T + iS) + (T + iS)(T - iS) = 2(T^2 + S^2) \in A_+ \]
    %
    Thus
    %
    \[ M^*M = 2(T^2 + S^2) - MM^* \in A_+ \]
    %
    which implies $M^*M = 0$.
\end{proof}

\begin{prop}
    The following are equivalent
    %
    \begin{enumerate}
        \item[(a)] $M \in A_+$.
        \item[(b)] There is $N \in A_+$ such that $M = N^2$.
        \item[(c)] There is $N \in A$ such that $N^*N = M$.
    \end{enumerate}
\end{prop}
\begin{proof}
    We have already show (a) implies (b). The proof of (c) from (b) is trivial. To prove (a) from (c), note that if $M = N^*N$, then $M$ is certainly self-adjoint, so we may write $M = M_+ - M_-$. Let $L = NM_-$. Then
    %
    \[ - L^*L = - M_-N^*NM_- = - M_- M M_- = M_-^3 \in A_+ \]
    %
    Implying $NM_- = 0$, hence
    %
    \[ 0 = N^*L = N^*NM_- = -M_-^2 \in A_+ \]
    %
    This implies $M_- = 0$, so $M = M_+ \in A_+$.
\end{proof}

Thus we are inclined to define $|M| = \sqrt{MM^*}$, for any $M \in A$, which is a positive element.

\begin{prop}
    If $M \leq N$, then
    %
    \begin{enumerate}
        \item[(a)] $M + L \leq N + L$.
        \item[(b)] $L^*ML \leq L^*NL$.
    \end{enumerate}
    %
    If in addition, $0 \leq M \leq N$, then
    %
    \begin{enumerate}
        \item[(c)] $\| M \| \leq \| N \|$.
        \item[(d)] $\sqrt{M} \leq \sqrt{N}$.
    \end{enumerate}
    %
    and if $M, N \in GL(A)$, then
    %
    \begin{enumerate}
        \item[(e)] $0 \leq N^{-1} \leq M^{-1}$
    \end{enumerate}
\end{prop}
\begin{proof}
    (a) is trivial. To prove (b), note that
    %
    \[ L^*(N - M)L = L^*\sqrt{N - M}\sqrt{N - M}L = (\sqrt{N - M} L)^* (\sqrt{N - M} L) \in A_+ \]
    %
    Let us prove (c). If $N$ is positive, then $N \leq \| N \|$, which follows by Gelfand theory. Thus if $M$ and $N$ are positive, then
    %
    \[ 0 \leq M \leq N \leq \| N \| \]
    %
    But then $\phi(M) \leq \| N \|$ for each $\phi$, so $\| M \| \leq \| N \|$.

    Fix $\varepsilon > 0$. Write
    %
    \[ (\varepsilon + \sqrt{N} + \sqrt{M})(\varepsilon + \sqrt{N} - \sqrt{M}) = T \]
    %
    where $T,U \in A$ are self adjoint. By calculation
    %
    \[ T = \varepsilon^2 + 2 \varepsilon \sqrt{N} + N - M \geq \varepsilon^2 \]
    %
    Thus $T$ is positive and invertible, so $\varepsilon + \sqrt{N} - \sqrt{M}$ is left invertible and therefore invertible. Thus $\varepsilon \not \in \sigma(\sqrt{M} - \sqrt{N})$, and thus
    %
    \[ \sigma(\sqrt{M} - \sqrt{N}) \subset (-\infty, 0) \]
    %
    But this implies $\sqrt{N} - \sqrt{M}$ is positive.

    If $M \geq \varepsilon$, then $M^{-1} \leq 1/\varepsilon$, which follows by Gelfand theory, since $\phi(M^{-1}) = \phi(M)^{-1}$. Since
    %
    \[ 1 = \sqrt{M}^{-1}M\sqrt{M}^{-1} \leq \sqrt{M}^{-1} N \sqrt{M}^{-1} \]
    %
    this yields
    %
    \[ \sqrt{M} N^{-1} \sqrt{M} \leq 1 \]
    %
    and by conjugation again,
    %
    \[ N^{-1} = \sqrt{M}^{-1} \sqrt{M} N^{-1} \sqrt{M} \sqrt{M}^{-1} \leq M^{-1} \]
\end{proof}

\subsection{Positive Approximate Units}

we show that all $C^*$ algebras have {\bf approximate units}, which are bounded approximate identities which are also increasing nets, in the sense of the positive ordering on the $C^*$ algebra.

\begin{lemma}
    If $0 \leq M \leq N$, then $M(1 + M)^{-1} \leq N(1 + N)^{-1}$.
\end{lemma}
\begin{proof}
    Note that
    %
    \[ M(1 + M)^{-1} = 1 - (1 + M)^{-1}\ \ \ \ \ N(1 + N)^{-1} = 1 - (1 + N)^{-1} \]
    %
    We know $1 + M \leq 1 + N$, so $(1 + N)^{-1} \leq (1 + M)^{-1}$, so
    %
    \[ N(1 + N)^{-1} = 1-(1 + N)^{-1} \leq 1-(1 + M)^{-1} = M(1 + M)^{-1} \]
    %
    which is exactly the inequality we needed.
\end{proof}

\begin{prop}
    For a $C^*$ algebra $A$, the net
    %
    \[ \{ M \in A_+ : \| M \| < 1 \} \]
    %
    ordered by the positivity of the algebra, is an approximate unit.
\end{prop}
\begin{proof}
    The set is a net, because any $M,N \in A_+$ are less than or equal to $\max(\|M\|,\|N\|)$. Fix a self adjoint $M \in A$, and let $\varepsilon > 0$. Let $B$ by the (non-unital) $C^*$ algebra generated by $M$. Then $B$ is commutative, and $\widetilde{M} \in C_0(\Phi_B)$. Let $g \in C_0(\Phi_B)$ be a function such that $0 \leq g < 1$, such that $\| \widetilde{M} - g \widetilde{M} \| < \varepsilon^2$, and set $N$ to be the unique element of $B$ such that $\widetilde{N} = g$. Then $\| N \| \leq 1$, $N$ is positive, and $\| M - MN \| < \varepsilon^2$. Let $N \leq L < 1$. Then, since $\sigma(\sqrt{1 - L}) \in [0,1]$, $\| \sqrt{1 - L} \| < 1$, and
    %
    \begin{align*}
        \| M - LM \|^2 &= \| \sqrt{1 - L}^2 M \|^2 \leq \| \sqrt{1 - L}\ M \|^2 = \| M(1 - L)M \|\\
        &\leq \| M(1 - N)M \| \leq \| M \| \| (1 - N)M \| < \| M \| \varepsilon^2
    \end{align*}
    %
    Decreasing $\varepsilon$, we find $\lim LM = M$. Similar results show $\lim ML = M$.
\end{proof}

If $A$ is an abelian $C^*$ algebra, then the family of self adjoint elements in $A$ forms a real Riesz lattice, which can be easily seen from an isometric representation $A \cong C(X)$. Thus if $0 \leq M \leq N + L$, where $N,L \in A_+$, then there are $N',L'$ such that $N' \leq N$, $L' \leq L$ such that $M = N' + L'$. This theorem does not hold in general nonabelian $C^*$ algebras, but there is a weaker decomposition theorem which does hold.

\begin{prop}
    If $A$ is a $C^*$ algebra, and
    %
    \[ \sum_{i = 0}^m M_i^* M_i = \sum_{j = 0}^n N_i^* N_i \]
    %
    then there are $L_{i,j}$ such that
    %
    \[ M_i^* M_i = \sum_j L_{i,j}^* L_{i,j}\ \ \ \ \ \ N_j^* N_j = \sum_i L_{i,j}^* L_{i,j} \]
\end{prop}
\begin{proof}
    We may assume $A$ is unital. Let $X = \sum M_i^* M_i = \sum N_i^* N_i$. Write
    %
    \[ L_{i,j,t} = s \]
    %
    and so on and so forth FINISH THIS LATER.
\end{proof}







\section{Ideals and Quotients of C Star Algebras}

\subsection{C Star Operations on Quotients}

If we quotient a $C^*$ algebra by a closed ideal, we certainly get a Banach algebra back. But does this algebra still have an involution, or do we need to apply additional structure to our ideals? Such discussions naturally lead to the Gelfand Naimark construction, establishing that every $C^*$ algebra is isometric to some subalgebra of bounded operators on a Hilbert space. Thus such an endeavor is very fruitful to discuss.

\begin{prop}
    If $\mathfrak{a}$ is a closed left/right ideal in a $C^*$ algebra $A$ containing a self-adjoint element $M$, and if $f \in C(\sigma(M))$ satisfies $f(0) = 0$, then $f(M) \in \mathfrak{a}$.
\end{prop}
\begin{proof}
    Choose a sequence $P_i$ of polynomials which uniformly approximating $f$, with $P_i(0) = 0$ for all $i$. If we write
    %
    \[ P_i = \sum a_{ij} X^i \overline{X}^j \]
    %
    then $P_i(M) = \sum a_{ij} X^{i+j} \in \mathfrak{a}$, and since $\mathfrak{a}$ is closed, we find $f(M) \in \mathfrak{a}$.
\end{proof}

\begin{corollary}
    If $M \in \mathfrak{a}$ is self adjoint, then $M_+$, $M_-$, $|M|$, and $\sqrt{M}$ are all in $\mathfrak{a}$.
\end{corollary}

If an ideal is closed under involution, then it certainly must be a two sided ideal, for involution reverses multiplication. We shall show that all two-sided ideals are closed under involution.

\begin{prop}
    A closed ideal in a $C^*$-algebra is closed under involution.
\end{prop}
\begin{proof}
    Given a closed ideal $\mathfrak{a}$ of a $C^*$ algebra $A$, $\mathfrak{a}$ is a (non-unital) $C^*$ subalgebra of $A$. Let $\{ E_\alpha \}$ be a two-sided approximate identity for $\mathfrak{a}$. Then, if $M \in \mathfrak{a}$ is arbitrary,
    %
    \[ \lim M^*E_\alpha = \lim (E_\alpha M)^* = (\lim E_\alpha M)^* = M^* \]
    %
    so $M^*$ is the limit of elements of $\mathfrak{a}$, and is thus in $\mathfrak{a}$.
\end{proof}

Thus if $\mathfrak{a}$ is a closed ideal in a $C^*$-algebra $A$, then $A/\mathfrak{a}$ is verified to be a Banach $*$ algebra. It is a little bit more technical to verify that $A/\mathfrak{a}$ is a $C^*$ algebra, but we shall now carry out the details.

\begin{lemma}
    If $\mathfrak{a}$ is a closed ideal of $A$, and $\{ E_\alpha \}$ is the positive BAI for $\mathfrak{a}$, then $A/\mathfrak{a}$, then for any $M \in A$,
    %
    \[ \| M + \mathfrak{a} \| = \lim \| M - E_\alpha M \| = \lim \| M - M E_\alpha \| \]
\end{lemma}
\begin{proof}
    Fix $M \in A$, and let $\{ E_\alpha \}$ be the positive BAI for $\mathfrak{a}$. Choose $\varepsilon > 0$, and let $N \in \mathfrak{a}$ satisfy $\| M + N \| \leq \| M + \mathfrak{a} \| + \varepsilon$. Then
    %
    \begin{align*}
        \limsup_\alpha \| M - E_\alpha M \| &\leq \limsup_\alpha \| (1 - E_\alpha)(M + N) \| + \| E_\alpha N - N \|\\
        &\leq \limsup_\alpha \| (1 - E_\alpha)(M + N) \| + \limsup_\alpha \| N - E_\alpha N \|\\
        &= \limsup_\alpha \| (1 - E_\alpha) (M + N) \|\\
        &\leq \| M + N \| \leq \| M + \mathfrak{a} \| + \varepsilon
    \end{align*}
    %
    and $\varepsilon$ was arbitrary, so
    %
    \[ \limsup_\alpha \| M - E_\alpha M \| \leq \| M + \mathfrak{a} \| \]
    %
    But $E_\alpha M \in \mathfrak{a}$, so for each $\alpha$,
    %
    \[ \| M - E_\alpha M \| \geq \| M + \mathfrak{a} \| \]
    %
    and since we have bounded the $\limsup$ above and below by the same value, it is in fact a limit, and has value $\| M + \mathfrak{a} \|$. Similar results hold for $\| M - M E_\alpha \|$.
\end{proof}

The lemma makes sense, for $E_\alpha$ approximates elements of $\mathfrak{a}$ as best as possible, so subtracting $E_\alpha M$ subtracts the best approximation of $M$ in $\mathfrak{a}$, thus giving us the quotient norm.

\begin{theorem}
    If $\mathfrak{a}$ is a closed ideal of a $C^*$ algebra $A$, then $A/\mathfrak{a}$ is a $C^*$ algebra.
\end{theorem}
\begin{proof}
    Letting $\{ E_\alpha \}$ be the positive approximate identity for $\mathfrak{a}$, we find
    %
    \begin{align*}
        \| M + \mathfrak{a} \|^2 &= \lim_\alpha \| M (1 - E_\alpha) \|^2\\
        &= \lim_\alpha \| (1 - E_\alpha) M^*M (1 - E_\alpha) \|\\
        &\leq \lim_\alpha \| (1 - E_\alpha) M^*M \|\\
        &= \| M^*M + \mathfrak{a} \|\\
        &\leq \| M^* + \mathfrak{a} \| \| M + \mathfrak{a} \|
    \end{align*}
    %
    And it is easy to see that $\| M^* + \mathfrak{a} \| = \| M + \mathfrak{a} \|$, for $\mathfrak{a}$ is closed under involution, so we find
    %
    \[ \| M + \mathfrak{a} \|^2 = \| M^*M + \mathfrak{a} \| \]
    %
    and this is exactly the $C^*$ identity for the quotient algebra.
\end{proof}

It is often much easier to calculate the norm on $A/\mathfrak{a}$ by applying the theory we've created. If $\pi: A \to B$ is a $C^*$ homorphism with kernel $\mathfrak{a}$, then we obtain an induced diagram of $C^*$ morphisms
%
\begin{center}
\begin{tikzcd}
    A \arrow{r}{\pi} \arrow{d}{} & B\\
    A/\mathfrak{a} \arrow{ru}[below]{\tilde{\pi}}
\end{tikzcd}
\end{center}
%
and we know that $\tilde{\pi}$ is an isometry, since it is injective. Thus
%
\[ \| M + \mathfrak{a} \| = \| \pi(M) \| \]
%
and this is normally easier to calculate than either of the two equations we can use to calculate the quotient norm.

\begin{example}
    We know the closed ideals of $C(X)$ are of the form
    %
    \[ \mathfrak{a}_K = \{ f \in C(X) : (\forall x \in K: f(x) = 0) \} \]
    %
    for some closed set $K$. Then $C(X)/\mathfrak{a}_K$ is isometric to $C(K)$, because we have the $C^*$ homorphism $f \mapsto f|_K$, and $\mathfrak{a}_K$ is the kernel.
\end{example}

\begin{example}
    Let $H$ be a Hilbert space, and $\mathfrak{a}$ a closed ideal of $B(H)$. If $\mathfrak{a} \neq (0)$, we claim $K(H) \subset \mathfrak{a}$. For any $x,y \in H$, let $x \otimes y \in B(H)$ be the map
    %
    \[ z \mapsto \langle z, y \rangle x \]
    %
    It is clear that $K(H)$ is the closed linear hull of $\{ x \otimes y: x,y \in H \}$. If $T \in \mathfrak{a}$ and $T \neq 0$, then there are $v,w$ such that $\langle Tv, w \rangle = 1$. Then
    %
    \[ (x \otimes y) = (x \otimes w) \circ (Tv \otimes y) = (x \otimes w) \circ T \circ (v \otimes y) \in \mathfrak{a} \]
    %
    from which we obtain, since $x \otimes y$ was arbitrary, that $K(H) \subset \mathfrak{a}$.
\end{example}

\begin{example}
    In an infinite dimensional, separable Hilbert space $H$ the only closed ideals of $B(H)$ are $(0)$, $K(H)$, and $B(H)$. Let $\mathfrak{a}$ be an ideal properly containing $K(H)$. Let $T \in \mathfrak{a} - K(H)$, which is without loss of generality positive. FINISH THIS LATER WHEN BETTER AT SPECTRAL MEASURES.

    Thus in an infinite dimensional, separable Hilbert space $H$, the {\bf Calkin Algebra} $C(H) = B(H)/K(H)$ has no nontrivial closed ideals.
\end{example}

\subsection{* Homorphisms}

The natural morphism between $C^*$ algebras is a $*$ homorphism, an algebra homomorphism $T$ which satisfies $T(M^*) = T(M)^*$.

\begin{prop}
    If $T: A \to B$ is a $*$ morphism, then $\| T(M) \| \leq \| M \|$ for all $M$.
\end{prop}
\begin{proof}
    Assume first that $A$ is unital. For any $M$, $\sigma(T(M)) \subset \sigma(M)$. But then if $M$ is normal, then $T(M)$ is normal, and
    %
    \[ \| T(M) \| = r(T(M)) \leq r(M) = \| M \| \]
    %
    In general, for any $M$,
    %
    \[ \| T(M) \|^2 = \| T(M) T(M)^* \| = \| T(MM^*) \| \leq \| MM^* \| = \| M \|^2 \]
    %
    so $\| T \| \leq 1$. In the non-unital case, $T$ induces a $*$-extension between $A^\#$ and $B^\#$, in which we can apply the previous case.
\end{proof}

Thus every $*$-morphism is continuous. We shall find that injective $*$-morphisms are isometries. To prove this, we must first verify a corresponding theorem about continuous functions on a locally compact space, which says that the uniform norm on $C(X)$ is the sharpest algebra norm we can have.

\begin{lemma}
    If $\pi: A \to B$ is an injective $*$-morphism between abelian $C^*$ algebras, then the dual map $\pi^*$ is surjective.
\end{lemma}
\begin{proof}
    Let
    %
    \[ K = \{ \phi|_A : \phi \in \Phi_B \]
    %
    If $K \neq \Phi_A$, then using the continuous functional calculus, we may find non-zero $f$ and $g$ which vanish at $K$, for which $f = fg$. Then $1 \in \sigma(g)$, so there is $\phi \in \Phi_B$ such that $\phi(g) = 1$. But this is clearly impossible.
\end{proof}

\begin{lemma}
    Let $\vvvert{\cdot}$ be a submultiplicative norm on $C(X)$. Then $\| \cdot \|_\infty \leq \vvvert{\cdot}$.
\end{lemma}
\begin{proof}
    Let $A$ be the completion of $C(X)$ with respect to the $\vvvert{\cdot}$ norm. Then $A$ is a Banach algebra, and we obtain
    %
    \[ \| f \|_\infty = r(f) \leq \vvvert{f} \]
    %
    The fact that algebraic facts can hide within them topological facts is essential to the proof.
\end{proof}

Since arbitrary $C^*$ algebras contain abelian subalgebra on mass, we can almost effortlessly harvest facts about $*$ morphisms on arbitrary algebras from the above lemma.

\begin{prop}
    An injective, continuous morphism $\pi: A \to B$ from a $C^*$ algebra to a Banach algebra satisfies
    %
    \[ \| M \| \leq \| \pi \| \| \pi(M) \| \]
\end{prop}
\begin{proof}
    Given $M$, let $A_0 = C^*(M^*M)$. Define a norm on $A_0$ by letting
    %
    \[ \vvvert{N} = \| \pi(N) \| \]
    %
    Then $\| \cdot \| \leq \vvvert{\cdot}$, and
    %
    \[ \| M \|^2 = \| M^*M \| \leq \vvvert{M^*M} = \| \pi(M^*M) \| \leq \| \pi \| \| M \| \| \pi(M) \| \]
    %
    which yields the claim.
\end{proof}

\begin{corollary}
    Let $\pi: A \to B$ be a $*$ homomorphism between $C^*$ algebras. Then $\pi(A)$ is a closed $C^*$ algebra, and $\pi$ is an isometry if it is injective.
\end{corollary}
\begin{proof}
    If $\pi$ is injective, it certainly has closed range by the last proposition. But in general, letting $\mathfrak{a} = \ker(\pi)$, $\pi$ induces an injective map
    %
    \begin{center}
    \begin{tikzcd}
        A \arrow{r}{\pi} \arrow{d}{} & B \\
        A/\mathfrak{a} \arrow{ru}{\tilde{\pi}}
    \end{tikzcd}
    \end{center}
    %
    and $\pi(A) = \tilde{\pi}(A)$ is closed. Thus if $\pi$ is injective we obtain an inverse $*$ homomorphism $\pi^{-1}: \pi(A) \to A$, and both must be contractible, hence isometries.
\end{proof}

\begin{prop}
    If $B$ is a $C^*$ subalgebra of $A$, and $\mathfrak{a}$ a closed ideal of $A$, then $B + \mathfrak{a}$ is a $C^*$ subalgebra of $A$.
\end{prop}
\begin{proof}
    Completeness is a three space property, and since $\mathfrak{a}$ is complete, we need only show that $(B + \mathfrak{a})/\mathfrak{a}$ is complete, and from this we will obtain completeness. Let $\pi: A \to A/\mathfrak{a}$ be the quotient map. But the composed $*$ morphism $\psi$ defined by
    %
    \[ B \to A \to A/\mathfrak{a} \]
    %
    tells us that the image of $B$ is closed, and
    %
    \[ \psi(B) = (B + \mathfrak{a})/\mathfrak{a} \]
    %
    so $(B + \mathfrak{a})/\mathfrak{a}$ is complete.
\end{proof}




\subsection{Positive Operators}

Every $C^*$ algebra is essentially a ring of bounded operators on a Hilbert space. The challenge is to construct a canonical Hilbert space from a $C^*$ algebra. The study of a certain subclass of operators will be essential. A {\bf positive} operator $\phi: A \to \mathbf{C}$ on a $C^*$ algebra maps $A_+$ into $[0,\infty)$. Certainly the set of positive maps form a real subspace of $A^*$.

\begin{example}
    If $\phi \in \Phi_A$, then $\phi$ is positive, for
    %
    \[ \phi(M^*M) = |\phi(M)|^2 \]
    %
    and every positive element can be written in the form $M^*M$.
\end{example}

\begin{example}
    If $X$ is locally compact and Hausdorff, and $\mu$ is a complex Borel measure on $X$, then the functional
    %
    \[ f \mapsto \int f d\mu \]
    %
    from $C_0(X)$ to $\mathbf{C}$ is positive if and only if $\mu$ is a positive measure.
\end{example}

\begin{example}
    Consider the trace map $\text{tr}: M_n(\mathbf{C}) \to \mathbf{C}$. Any positive matrix $M$ can be, by a change of basis, put into the form
    %
    \[ \begin{pmatrix} \lambda_1 & 0 & \dots & 0 \\ 0 & \lambda_2 & \dots & 0 \\ 0 & 0 & \ddots & 0 \\ 0 & 0 & \dots & \lambda_n \end{pmatrix} \]
    %
    where $\lambda_1, \dots, \lambda_n \geq 0$. The trace is invariant of a change in basis, and thus
    %
    \[ \text{tr}(M) = \sum \lambda_i \geq 0 \]
    %
    Thus the trace is positive.
\end{example}

\begin{example}
    For a fixed $x \in H$, the map $T \mapsto \langle Tx, x \rangle$ is positive.
\end{example}

One need not verify continuity for positive operators, for we obtain this automatically.

\begin{prop}
    A positive operator $\phi:A \to \mathbf{C}$ is continuous.
\end{prop}
\begin{proof}
    We claim the supremum
    %
    \[ K = \sup \{ \phi(M): M \in A_+, \| M \| \leq 1 \} \]
    %
    is finite. Otherwise, we may pick a sequence $M_1, M_2, \dots$ with $\| M_i \| \leq 1$ and $\phi(M_i) \geq 4^n$. Let
    %
    \[ M = \sum \frac{M_i}{2^i} \]
    %
    Then for each $n$, $2^n M \geq M_n$, so
    %
    \[ \phi(M) \geq \frac{\phi(M_n)}{2^n} \geq \frac{4^n}{2^n} = 2^n \]
    %
    which yields an immediate contradiction.

    Now suppose that $M$ is an arbitrary operator, choose self adjoint $N$ and $L$ in $A$ such that
    %
    \[ M = N + iL = N_+ - N_- + iL_+ - iL_- \]
    %
    Then if $\| M \| \leq 1$,
    %
    \[ |\phi(M)| = |\phi(N_+) - \phi(N_-) + i\phi(L_+) - i\phi(L_-)| \leq 4K \]
    %
    which shows that $\phi$ is continuous.
\end{proof}

\begin{theorem}
    The following are equivalent
    %
    \begin{enumerate}
        \item $\phi$ is positive operator.
        \item For every BAI $E_\alpha$ contained within the positive identity on $A$,
        %
        \[ \| \phi \| = \lim_\alpha \phi(E_\alpha) \]
        \item There is a BAI $E_\alpha$ in the positive identity on $A$ for which
        %
        \[ \| \phi \| = \lim_\alpha \phi(E_\alpha) \]
    \end{enumerate}
\end{theorem}
\begin{proof}
    Without loss of generality, assume $\| \phi \| = 1$. Suppose $\phi$ is positive. Fix an approximate identity $E_\alpha$. Then
    %
    \[ \limsup_\alpha \phi(E_\alpha) \leq 1 \]
    %
    On the other hand, when $\| M \| \leq 1$,
    %
    \[ |\phi(M)|^2 = \lim |\phi(E_\alpha M)|^2 \leq \liminf_\alpha \phi(E_\alpha^2) \phi(M^*M) \]
\end{proof}







\chapter{Spectral Theory for Normal Operators}

\section{Resolutions of the Identity}

If $X$ is a measure space and $H$ is a Hilbert space, then a \emph{resolution of the identity} on $X$ is a mapping $E$ from measurable subsets of $X$ to self-adjoint projections $B(H)$, such that $E(\emptyset) = 0$, $E(X) = I$, such that $E(S_1 \cap S_2) = E(S_1) E(S_2)$, if $S_1$ and $S_2$ are disjoint, $E(S_1 \cup S_2) = E(S_1) + E(S_2)$, and for any $x,y \in H$, the map $E_{x,y}(S) = \langle E(S) x, y \rangle$ is a complex measure. If $X$ is a topological space equipped with the measure space structure given by Borel sets, then we will implicitly assume that each $E_{x,y}$ is a regular Borel measure.

A consequence of these assumptions is that all projections in the image of $E$ commute with one another, and that if $S_1$ and $S_2$ are disjoint, $E(S_1)$ and $E(S_2)$ are orthogonal to one another. $E$ is certainly finitely additive, but except in trivial situations, is \emph{not} countably additive in the topology of $B(H)$, since projections have norm one. On the other hand, the measure \emph{is} countably additive in the strong operator topology on $B(H)$, i.e. for each $x \in H$, if $\{ S_n \}$ are disjoint, and $S = \bigcup S_n$, then
%
\[ \sum_{n = 1}^\infty E(S_n) x = E(S) x, \]
%
because the series converges absolutely, i.e. by the orthogonality of the projections $E(S_n)$,
%
\[ \sum_{n = 1}^\infty \| E(S_n) x \|^2 \leq \| x \|^2. \]
%
We will use resolutions of the identity to obtain a very general functional calculus on operators.

If $E$ is a resolution to the identity, we let $f: X \to \CC$ be a measurable function. The \emph{essential range} of $f$ is the complement of the set formed from the union of all open discs $D \subset \CC$ with $E(f^{-1}(D)) = 0$. The function $f$ is essentially bounded if it's range is bounded, and we can then define $\| f \|_{L^\infty(X,E)}$ to be the maximum modulus of an element of the essential range of $f$. The quotient of the family of essentially bounded functions by the closed ideal of functions $f$ with $\| f \|_{L^\infty(X,E)} = 0$ is a Banach algebra denote by $L^\infty(X,E)$. In this Banach algebra, $\sigma(f)$ is precisely the essential range of $f$. We now define a way to integrate elements of $L^\infty(X,E)$ against the resolution $E$. The integral will not only be linear, but even \emph{multiplicative}. Recall that a closed normal subalgebra of $B(H)$ is a commutative subalgebra closed under taking adjoints.

\begin{theorem}
    For any resolution of the identity $E$, there exists an isometric map $L^\infty(X,E) \to B(H)$ onto a closed normal subalgebra $A$ of $B(H)$, such that if we denote the image of $f \in L^\infty(X,E)$ under this map by
    %
    \[ \int_X f dE, \]
    %
    then for any $x,y \in H$,
    %
    \[ \left\langle \left(\int_X f dE \right) x, y \right\rangle = \int f dE_{x,y}, \]
    %
    and
    %
    \[ \left\| \left( \int_X f dE \right) x \right\|^2 = \int_X |f|^2 dE_{x,x}. \]
    %
    An operator $M \in B(H)$ commutes with every $E(S)$ if and only if $M$ commutes with every element of $A$, and
    %
    \[ \| f \|_{L^\infty(X,E)} = \sup_{\| x \| \leq 1} \left( \int_X |f|^2 dE_{x,x} \right)^{1/2}. \]
\end{theorem}
\begin{proof}
    TODO: Not that interesting.
\end{proof}

The spectral theorem for normal operators on a Hilbert space says that each normal $T \in B(H)$ induces a canonical resolution $E$ on Borel subsets of $\sigma(T)$ which allows us to reconstruct $T$. We can do things in a more general light, by simultaneously performing the spectral theorem for any subalgebra of $B(H)$, all of whose elements are normal.

\begin{theorem}
    Suppose that $A$ is a closed normal unital subalgebra of $B(H)$, and let $\Phi_A$ be the maximal ideal space of $A$. Then there exists a unique resolution $E$ on the Borel sets of $\Phi_A$ such that
    %
    \[ T = \int_{\Phi_A} \widehat{T}\; dE, \]
    %
    where $\widehat{T}$ is the Gelfand transform of $T$. The inverse of the Gelfand transform from $C(\Phi_A)$ to $A$ extends to an isometric $*$ isomorphism $\Phi$ of $L^\infty(X,E)$ onto a closed subalgebra $B$ of $B(H)$ containing $A$, where the inverse is given by
    %
    \[ f \mapsto \int_{\Phi_A} f dE. \]
    %
    The subalgebra $B$ is the closure (in the topology of $B(H)$) of all finite linear combinations of the projections in the image of $E$. Moreover, if $U$ is a nonempty open subset of $\Phi_A$, then $E(U) \neq 0$. An operator $S \in B(H)$ commutes with every element of $A$ if and only if $S$ commutes with every projection in the image of $E$.
\end{theorem}
\begin{proof}
    TODO
\end{proof}

In particular, if $T \in B(H)$ is a normal operator, then there exists a unique resolution of the identity on Borel subsets of $\sigma(T)$ such that
%
\[ T = \int_{\sigma(T)} \lambda dE(\lambda) \]
%
and if $S$ commutes with $T$, every projection in the image of $E$ commutes with $S$. We can extend $E$ to a resolution on $\CC$ by setting $E(S) = 0$ if $S \cap \sigma(T) = \emptyset$. We call $E$ in this case the \emph{spectral decomposition} of $T$. To obtain this result from the more general result, we consider the smallest closed subalgebra $A$ of $B(H)$ containing $T$, $T^*$ and the identity. Then $A$ is a normal (in fact, a commutative) algebra, and $\Phi_A$ can be identified with $\sigma(T)$ in such a way that $\widehat{T}(\lambda) = \lambda$ for each $\lambda \in \sigma(T)$. In this case, the inverse map from $L^\infty(\sigma(T),E)$ to $B(H)$ is denoted by $f \mapsto f(T)$. In this special case, we find that $\| f(T) \| \leq \| f \|_{L^\infty(X)}$ (defined by a pointwise supremum rather than an essential supremum), where this is an equality if $f \in C(\sigma(T))$, since then $\| f \|_{L^\infty(X,E)} = \| f \|_{L^\infty(X)}$ by virtue of the fact that if $U \subset \CC$ is open, then $E(U) \neq 0$. Intuitively, we can think of $dE(\lambda)$ as giving an `infinitisimal projection' onto the eigenspace corresponding to the eigenvalue $\lambda$. Let us consider some applications of this symbolic calculus.

\begin{theorem}
    If $T \in B(H)$ is normal, then
    %
    \[ \| T \| = \sup_{\|x\| \leq 1} |\langle Tx, x \rangle|. \]
\end{theorem}
\begin{proof}
    If scalars in $\sigma(T)$ always corresponded to eigenvectors, then this result would follow from the fact that $\| T \| = \sigma(T)$. But elements of $\sigma(T)$ do not always correspond to eigenvectors. On the other hand, we will use the spectral theorem to produce `almost eigenvectors' for $T$, which will suffice for our purposes. Since $T$ is normal, we know that $\| T \|$ is equal to the spectral radius of $T$. Thus for any $\lambda_0 \in \sigma(T)$ with $|\lambda_0| = \| T \|$. let $B_\varepsilon$ be a ball of radius $\varepsilon$ centred at $\lambda_0$. Then if $E$ is the resolution associated with $T$, we know that $E(B_\varepsilon) \neq 0$. Since $E(B_\varepsilon)$ is a projection, we may find $x \in H$ such that $E(B_\varepsilon) x = x$ and such that $\| x \| = 1$. If $f(\lambda) = (\lambda - \lambda_0) \mathbf{I}_{B_\varepsilon}$, then
    %
    \[ f(T) = (T - \lambda_0) E(B_\varepsilon) \]
    %
    and so $f(T) x = Tx - \lambda_0 x$. Now $\| f \|_{L^\infty} \leq \varepsilon$, so it follows that $\| f(T) x \| \leq \varepsilon$, which implies that
    %
    \[ |\langle Tx, x \rangle - \lambda_0| = |\langle f(T)x, x \rangle - \lambda_0| \leq \| f \|_{L^\infty} \leq \varepsilon. \]
    %
    This implies the required result.
\end{proof}

\begin{remark}
    If $T: \CC^2 \to \CC^2$ is given by $T(a,b) = (b,0)$, then $\langle T(a,b), (a,b) \rangle = ab$, and so $|\langle Tx, x \rangle| \leq 1/2$ if $\| x \| \leq 1$. But $\| T \| = 1$.
\end{remark}

Similar arguments, left as an exercise, show that a normal operator $T \in B(H)$ is self-adjoint if and only if $\sigma(T) \subset \RR$, and unitary if and only if $\sigma(T)$ is contained on the unit circle. This need not be true for general operators (e.g. non orthogonal projections are not self-adjoint).

\begin{theorem}
    Let $T \in B(H)$ be normal, and let $E$ be it's spectral decomposition. If $f \in C(\sigma(T))$ and $S = f^{-1}(0)$, then the kernel of $f(T)$ is equal to the range of $E(S)$.
\end{theorem}
\begin{proof}
    If $g = \mathbf{I}_S$, then $f \cdot g = 0$, and so $f(T) g(T) = f(T) E(S) = 0$, which shows the range of $E(S)$ is contained in the kernel of $f(T)$. Conversely, suppose $x \in H$ and $f(T) x = 0$. For each $\varepsilon > 0$, let
    %
    \[ g_\varepsilon(x) = \frac{1}{f(x)} \mathbf{I}_{S_\varepsilon^c}(x). \]
    %
    Since $f$ is continuous, $g_\varepsilon \in L^\infty(\sigma(T))$ for all $\varepsilon > 0$. And $f \cdot g_\varepsilon = \mathbf{I}_{S_\varepsilon^c}$, so that $E(S_\varepsilon^c) x = g_\varepsilon(T) f(T) x = 0$. Countable additivity shows that $E(S^c) x = 0$, which shows by orthogonality of $E(S)$ and $E(S^c)$ that $x$ is in the range of $E(S)$.
\end{proof}

It follows that if $\lambda \in \sigma(T)$, and $E = E(\lambda)$, then the range of $E$ is precisely the kernel of $T - \lambda_0$, i.e. the set of eigenvectors for $T$ with eigenvalue $\lambda$. Since $E(U) \neq 0$ for any open $U$, any isolated point of $\sigma(T)$ is an eigenvalue for some eigenvector. If $\sigma(T)$ is countable, $\sigma(T) = \{ \lambda_1, \lambda_2, \dots \}$ then we have, in the strong operator topology, an orthogonal decomposition $I = \sum_{n = 1}^\infty E_i$, where $E_i = E(\lambda_i)$, and $TE_i = \lambda E_i$.

Previously, we saw that for a compact normal operator $T$, $\sigma(T)$ was countable with no limit point but the origin, and $T - \lambda$ had a finite dimensional kernel for each $\lambda \in \sigma(T)$. Using the spectral theorem, we can now prove that this characterizes the compact normal operators from the family of all normal operators.

\begin{theorem}
    A normal operator $T \in B(H)$ is compact if and only if $\sigma(T)$ has no limit point but the origin (which implies $\sigma(T)$ is countable), and for any $\lambda \neq 0$, then $T - \lambda$ has finite dimensional kernel.
\end{theorem}
\begin{proof}
    To show this is sufficient, if $f_\varepsilon(x) = x \cdot \mathbf{I}(|x| \geq \varepsilon)$, then $f_\varepsilon(T)$ is a finite sum of finite rank operators. Thus $f_\varepsilon(T)$ is compact for each $\varepsilon > 0$. On the other hand,
    %
    \[ \| T - f_\varepsilon(T) \| \leq \| x - f_\varepsilon \|_{L^\infty} \leq \varepsilon, \]
    %
    which shows $T$ is the limit of a family of compact operators, and is therefore compact.
\end{proof}

It follows that if $T$ is normal and compact, then $T$ has an eigenvalue $\lambda$ with $\| T \| = |\lambda|$, and $f(T)$ is compact for any $f \in C(\sigma(T))$ with $f(0) = 0$, but $f(T)$ is not compact if $H$ is infinite dimensional and $f(0) \neq 0$.

\begin{theorem}
    If $T \in B(H)$, then $\langle Tx, x \rangle \geq 0$ for each $x \in H$ if and only if $T$ is self-adjoint, and $\sigma(T) \subset [0,\infty)$. We call such an operator \emph{positive semidefinite} and write $T \geq 0$. If $\sigma(T) \subset (0,\infty)$, we say $T$ is \emph{positive definite}, and then $\langle Tx, x \rangle > 0$ for $x \neq 0$, and if $\lambda_0$ is the smallest value in $\sigma(T)$, then $\langle Tx, x \rangle \geq \lambda_0 \| x \|^2$ (and $\lambda_0$ is the largest constant for which this bound holds), in which case we write $T \geq \lambda_0$.
\end{theorem}
\begin{proof}
    Suppose $\langle Tx,x \rangle \geq 0$ for each $x \in H$. Then because $\langle Tx, x \rangle$ is real-valued, $\langle Tx, x \rangle = \langle x, Tx \rangle = \langle T^* x, x \rangle$, so that $\langle (T - T^*) x, x \rangle = 0$ for all $x \in H$. Since $T - T^*$ is normal, it follows that $T - T^* = 0$, from which it follows that $T$ is self adjoint. For any $\lambda < 0$, for $x \neq 0$ we have
    %
    \[ -\lambda \| x \|^2 = \langle -\lambda x, x \rangle \leq \langle (T-\lambda) x, x \rangle \leq \| (T - \lambda) x \| \| x \|, \]
    %
    from which it follows that $\| (T - \lambda) x \| \geq - \lambda \| x \|$, so $T - \lambda$ is an isomorphism, hence $\lambda \not \in \sigma(T)$.

    Conversely, if $T$ is self adjoint, and $\sigma(T) \subset [0,\infty)$, then
    %
    \[ \langle Tx, x \rangle = \int_0^\infty \lambda dE_{x,x}. \]
    %
    Since $E_{x,x}$ is a positiev measure, and $\lambda \geq 0$, it follows that $\langle Tx,x\rangle \geq 0$. If $\sigma(T) \subset (0,\infty)$, and $\lambda_0$ is the smallest element of $\sigma(T)$, then $\sigma(T - \lambda_0) \subset [0,\infty)$, and is self adjoint, so $\langle (T - \lambda_0) x, x \rangle \geq 0$, which gives $\langle Tx, x \rangle \geq \lambda \| x \|^2$.
\end{proof}

If $T \in B(H)$ is positive-semidefinite, then we can let $f(t) = \sqrt{t}$ be a well defined continuous function on $\sigma(T)$, and so $f(T) = \sqrt{T}$ is well define. Thus every positive semidefinite function has a square root. Now $\sigma(f(T)) = \sqrt{\sigma(T)} \subset [0,\infty)$, and since $f$ is real valued, $\sqrt{T}$ is self adjoint. Thus $\sqrt{T}$ is a positive-semidefinite square root for $T$. This is actually the unique such positive semidefinite square root, because if $S \geq 0$ is a positive semidefinite square root for $T$, then we let $A$ be the smallest closed subalgebra of $B(H)$ containing $S$, $T$, and $\sqrt{T}$. Then $A$ is commutative, and is isomorphic to $C(\Phi_A)$ for some compact $\Phi_A$, in such a way that positive-semidefinite elements of $A$ are mapped onto non-negative functions in $C(\Phi_A)$. But since non-negative functions have unique square roots, it follows that $S = \sqrt{T}$.

For any operator $T$, $T^*T$ is a positive semidefinite self-adjoint operator, so we can consider the operator $|T| = \sqrt{T^*T}$. Then
%
\[ \| |T| x \|^2 = \langle \sqrt{T^* T} x, \sqrt{T^* T} x \rangle = \langle T^*T x, x \rangle = \| Tx \|^2. \]
%
This is the \emph{only} positive semidefinite operator with this property, since if $\| Sx \| = \| Tx \|$ for each $x \in H$, then $\langle (S^2 - T^* T) x, x \rangle = 0$ for all $x$, from which it follows that $S^2 = T^*T$.

Over the complex numbers, any $z \in \CC$ can be written as $r w$ for some $r \geq 0$ and $|w| = 1$. This leads to the problem of trying to factorize an operator $T$ into a product $U P$, where $P \geq 0$ and $U$ is unitary. We call $UP$ a \emph{polar decomposition} of $T$.




















\chapter{Unbounded Operators}

We often wish to describe a spectral theory of operators that are not necessarily bounded. For instance, in the spectral analysis of partial differential equations with non constant coefficients, one would like to consider a Banach space of functions $X$ supporting both derivative operators $D: X \to X$ given formally by $Df = f'$, and multiplier operators $M: X \to X$ given formally by $Mf = xf$. Applying the product rule would show that $[D,M] = DM - MD$ is the identity operator. But the following intriguing result shows that we \emph{cannot} find a Banach space of functions $X$ upon which this theory works.

\begin{lemma}
    There does not exist a unital normed algebra $A$ supporting $x,y \in A$ with $xy - yx = 1$.
\end{lemma}
\begin{proof}
    If $x,y \in A$ were found with $xy - yx = 1$, then it would follow that $x^n y - yx^n = nx^{n-1}$ (easily proved e.g. by induction). But this means that
    %
    \[ n \| x^{n-1} \| = \| x^n y - yx^n \| \leq 2 \| x^n \| \| y \| \leq 2 \| x^{n-1} \| \| x \| \| y \|, \]
    %
    so that $\| x \| \| y \| \geq n/2$ for every integer $n$, which is impossible.
\end{proof}

Thus we cannot find a common function space upon which to analyze differentiation and multiplication as bounded operators. Fortunately, we can obtain a theory of \emph{unbounded operators} which allow us to view $D$ and $M$ as operators with a common domain (for instance, the Hilbert space $L^2(\RR)$). We detail the theory in this chapter.

Let $X$ and $Y$ be Banach spaces. An \emph{unbounded operator} $T: X \to Y$ is a linear map $T: \mathcal{D}(T) \to Y$ defined on a subspace $\mathcal{D}(T) \subset X$. Two unbounded operators are equal if they have the same domain, and are equal as functions. We do not assume this operator has any topological properties in general - if the operator was bounded the Hahn-Banach theorem would imply that it was just the restriction of a bounded operator $T: X \to Y$, so there would be no real importance to consider the operator on a smaller subspace.

For any unbounded operator $T: X \to Y$, we can consider the graph $\Gamma(T) = \{ (x,Tx): x \in \mathcal{D}(T) \}$. If $\Gamma(T)$ is a closed subset of $X \times Y$, we say that $T$ is \emph{closed}. The closed graph theorem implies that if $T$ is closed and $\mathcal{D}(T) = X$, then $T$ is actually a bounded operator. For any closed operator $T$, we can give $\mathcal{D}(T)$ the structure of a Banach space by defining $\| x \|_T = \sqrt{\| x \|^2 + \| Tx \|^2}$, and in this topology $T$ will then be a bounded operator from $\mathcal{D}(T)$ to $Y$.

We wish to extend the natural operations on the class of operators to unbounded operators. Given a scalar $\lambda$, and an unbounded operator $T: X \to Y$, we can define an unbounded operator $\lambda T: X \to Y$ with $\mathcal{D}(\lambda T) = \mathcal{D}(T)$. For two unbounded operators $T: \mathcal{D}(T) \to Y$ and $S: \mathcal{D}(S) \to Y$, we can define $T + S$ as an unbounded operator defined on $\mathcal{D}(T) \cap \mathcal{D}(S)$. Similarily, we can define $T - S$. If $T: \mathcal{D}(T) \to Y$ and $S: \mathcal{D}(S) \to Z$ are unbounded operators, we can define $ST$ as an unbounded operator defined on $\mathcal{D}(T) \cap T^{-1}(\mathcal{D}(S))$. The distributive law $(S + U)T = ST + UT$ continues to hold since
%
\[ \mathcal{D}(T) \cap T^{-1}(\mathcal{D}(S) \cap \mathcal{D}(U)) = \left[ \mathcal{D}(T) \cap T^{-1}(\mathcal{D}(S)) \right] \cap \left[ \mathcal{D}(T) \cap T^{-1}(\mathcal{D}(U)) \right], \]
%
but the distributive law $T(S + U) = TS + TU$ might \emph{not} hold in this scheme. If $T: X \to Y$ is an injective unbounded operator, we can define $T^{-1}: Y \to X$ to be the unbounded operator whose domain is the image of $T$.

An unbounded operator is \emph{densely defined} if $\mathcal{D}(T)$ is a dense subset of $X$. Given an unbounded densely defined operator $T: \mathcal{D}(X) \to Y$, we can define an unbounded operator $T^*$ with domain
%
\[ \mathcal{D}(T^*) = \{ \gamma \in Y^*: \gamma \circ T \in \mathcal{D}(T)^* \}, \]
%
where $\mathcal{D}(T)$ is given the subspace topology inherited from $X$. We then set $T^* \gamma$ to be the unique $\lambda \in X^*$ extending the continuous functional $\gamma \circ T$. In inner product notation, we then have
%
\[ \langle T^* \gamma, x \rangle = \langle \gamma, Tx \rangle \]
%
for any $x \in \mathcal{D}(T)$ and $\gamma \in \mathcal{D}(T^*)$. Note that if $T$ is not densely defined, this extension might not be unique, and so the adjoint is not well defined. If $T: X \to Y$ and $S: Y \to Z$ are densely defined unbounded operators such that $ST$ is also densely defined, then $(ST)^*$ is an extension of $T^* S^*$. If $S$ is bounded, then the domains are actually equal to one another, and so we get the familiar identity $T^* S^* = (ST)^*$. We also note that if $T: X \to Y$ is densely defined and unbounded, and $S: Y^* \to X^*$ is another unbounded operator such that $\gamma(Tx) = (S\gamma)(x)$ for any $x \in \mathcal{D}(T)$ and $\gamma \in \mathcal{D}(S)$, then $\mathcal{D}(S) \subset \mathcal{D}(T^*)$, because $\gamma \circ T$ is a bounded functional on $\mathcal{D}(T)$ for any $\gamma \in \mathcal{D}(S)$.

%If $T^* S^* \eta$ is well defined, then firstly, $|\eta(Sy)| \lesssim \| y \|_Y$ for $y \in \mathcal{D}(S)$. Thus there exists $\gamma \in Y^*$ such that $\gamma y = \eta(Sy)$ for all $y \in \mathcal{D}(S)$. Moreover, we know that $|\gamma(Tx)| \lesssim \| x \|$ for all $x \in \mathcal{D}(T)$. Thus we can find $\lambda \in X^*$ such that $\lambda x = \gamma(Tx)$. But then
%
%\[ |\eta(STx)| = |\gamma(Tx)| \lesssim \| x \| \]
%
% So \eta is in the domain of T^* S^*.

If $T: H_1 \to H_2$ is an unbounded operator between two Hilbert spaces $H_1$ and $H_2$, the unbounded adjoint can be naturally identified with an unbounded adjoint $T^*: H_2 \to H_1$, which we will do whenever we are talking about Hilbert spaces. If $T: H \to H$ is unbounded, and $T^*$ is an extension of $T$, then we say $T$ is \emph{symmetric}. This means that for any $x_1,x_2 \in \mathcal{D}(T)$, $\langle Tx_1, x_2 \rangle = \langle x_1, Tx_2 \rangle$. If we actually have $T^* = T$, which means that in addition to the previous property, $x \mapsto \langle Tx, y \rangle$ is continuous if and only if $y \in \mathcal{D}(X)$, then we say $T$ is \emph{self adjoint}. However, it is possible for $T$ to be a strict extension of $T^*$, as the next example shows, and we do not say this operator is symmetric nor self-adjoint. 

\begin{example}
    Let $T: L^2[0,1] \to L^2[0,1]$ be the unbounded operator whose domain consists of all absolutely continuous functions $f: [0,1] \to \CC$ such that $f' \in L^2[0,1]$, and for such functions, let
    %
    \[ Df = if'. \]
    %
    If $g \in L^2[0,1]$ satisfies
    %
    \[ \left| \int f'(x) g(x)\; dx \right| \lesssim \| f \|_{L^2[0,1]} \]
    %
    for all $f \in \mathcal{D}(T)$, then we claim that $g$ is absolutely continuous, and $g(0) = g(1) = 0$. Conversely, if $g$ is absolutely continuous, then
    %
    \[ \int f'(x) g(x)\; dx = f(1) g(1) - f(0) g(0) - \int f(x) g'(x)\; dx, \]
    %
    and we see that this quantity is $\lesssim \| f \|_{L^2[0,1]}$ precisely when $g(0) = g(1) = 0$. All that remains is to argue that $g$ is absolutely continuous, but this follows because $\mathcal{D}(T)$ contains the restriction of $C_c^\infty(0,1)$, so the boundedness equation above implies that the distributional derivative $g'$ of $g$ in $\mathcal{D}(0,1)^*$ is square integrable, which implies that $g$ lies in $\mathcal{D}(T)$, i.e. $g$ is absolutely continuous and $g' \in L^2[0,1]$. Thus we see that, even though we have $\langle Tx, y \rangle = \langle x, Ty \rangle$ for $x \in \mathcal{D}(T)$ and $y \in \mathcal{D}(T^*)$, $T^*$ is only defined on a proper subset of the domain of $T$. On the other hand, the operator $ S = T^*$, which extends $T$, is \emph{symmetric}, but not self adjoint, since one can verify that $S^* = T^{**} = T$, and that $S^*$ therefore extends $S$. The operator $T'$ defined as the restriction of $T$ to functions $f$ with $f(0) = f(1)$, is \emph{self adjoint}, since once can easily verify from the integration by parts formula that $\mathcal{D}((T')^*) = \mathcal{D}(T')$.
\end{example}

We now come to an interesting result which relates the graph of $T$ and $T^*$.

\begin{theorem}
    Let $T$ be a densely defined unbounded operator on a Hilbert space $H$. Then $\Gamma(T^*) = [V \Gamma(T)]^\perp$, where $V(x,y) = (-y,x)$, and $H \times H$ is viewed as a Hilbert space. In particular, $T^*$ is \emph{always} a closed unbounded operator, all self-adjoint operators are closed.
\end{theorem}
\begin{proof}
    The following four statements are equivalent to one another:
    %
    \begin{itemize}
        \item $(y,z) \in \Gamma(T^*)$
        \item $\langle y, Tx \rangle = \langle z, x \rangle$ for every $x \in \mathcal{D}(T)$.
        \item $\langle (y,z), V(x,Tx) \rangle = 0$ for every $x \in \mathcal{D}(T)$.
        \item $(y,z)$ lies in $V(\Gamma(T))^\perp$.
    \end{itemize}
    %
    Thus the proof is complete.
\end{proof}

Here are some corollaries for a symmetric unbounded operator $T: H \to H$:
%
\begin{itemize}
    \item If $\mathcal{D}(T) = H$, then $T$ is a bounded, self-adjoint operator.
    \item If $T$ is injective and self-adjoint, then it has dense image, and $T^{-1}$ is self adjoint.
    \item If the image of $T$ is dense, then $T$ is injective.
    \item If $T$ is surjective (so that (c) applies and $T$ is actually injective), then $T$ is self-adjoint, and $T^{-1}$ is bounded.
\end{itemize}
\begin{proof}
    To see (a), if $\mathcal{D}(T) = H$, then because $T^*$ extends $T$, $T^* = T$, and since $T^*$ is closed, it follows that $T$ is bounded and self-adjoint. 

    To obtain (b), to show the image is dense let $x$ be perpendicular to the image of $T$. Then $x \in \mathcal{D}(T^*)$ because $\langle x, Tx' \rangle = 0$ for all $x' \in \mathcal{D}(T)$, and is thus trivially continuous. But this implies that $x \in \mathcal{D}(T^*) = \mathcal{D}(T)$, and $Tx = T^*x = 0$. Thus $x = 0$, which shows the image of $T$ is dense. Thus $T^{-1}$ is a densely defined unbounded operator, and so we can consider it's adjoint. We claim that $[V\Gamma(T^{-1})]^\perp = \Gamma(T^{-1})$, which would show that $T^{-1}$ is self adjoint. But $V(\Gamma T^{-1}) = \Gamma(-T)$ and $V \Gamma(-T)$, and since $T$ is self-adjoint, $-T$ is self adjoint, and so $\Gamma(-T)^\perp = V \Gamma(-T) = \Gamma(T^{-1})$, which completes the argument.

    To prove (c), we suppose that $Tx = 0$. Then for any $x' \in \mathcal{D}(T)$, $\langle x, Tx' \rangle = \langle Tx, x' \rangle = 0$. Thus $x$ is orthogonal to the image of $T$. But this implies $x = 0$.

    To prove (d), we note that $T^{-1}$ is symmetric, because if $x_1,x_2 \in H$ then $x_i = Ty_i$ for some $y_i \in \mathcal{D}(T)$, and so applying the fact that $T$ is symmetric, we find that
    %
    \[ \langle T^{-1} x_1, x_2 \rangle = \langle y_1, Ty_2 \rangle = \langle Ty_1, y_2 \rangle = \langle x_1, T^{-1} x_2 \rangle. \]
    %
    Since $T^{-1}$ is self adjoint, injective, and has dense iamge, this implies $T = (T^{-1})^{-1}$ is self adjoint. But (a) also implies that $T^{-1}$ is bounded.
\end{proof}

A powerful consequence of this result is that if $T$ is a closed operator, then we have an orthogonal decomposition
%
\[ H \times H = V \Gamma(T) \oplus \Gamma(T^*) \]
%
Thus for any $a,b \in H$, there exists a unique pair $(x,y) \in \mathcal{D}(T) \times \mathcal{D}(T^*)$ such that $a = y - Tx$ and $b = x + T^* y$. Moreover, for this pair we have
%
\[ \| a \|^2 + \| b \|^2 = \| x \|^2 + \| Tx \|^2 + \| y \|^2 + \| T^* y \|^2. \]
%
In particular, setting $a = 0$, we find that with any closed, unbounded operator $T$, we can associate a pair of bounded operators $A_1: H \to \mathcal{D}(T)$ and $A_2: H \to \mathcal{D}(T^*)$, with $\| A_1 \| \leq 1$ and $\| A_2 \| \leq 1$, such that $TA_1 = A_2$, $I = A_1 + T^* A_2$, and for any $x \in H$,
%
\[ \| x \|^2 = \| A_1 x \|^2 + \| TA_1 x \|^2 + \| A_2 x \|^2 + \| T^* A_2 x \|^2. \]
%
In particular, if $Q = I + T^*T$, then $I = QA_1$, so $Q$ is surjective onto $H$. Moreover, $Q$ is also injective, since if $x \in \mathcal{D}(Q)$, $Tx \in \mathcal{D}(T^*)$, and so
%
\[ \langle x, x \rangle + \langle Tx, Tx \rangle = \langle x, x \rangle + \langle T^*Tx, x \rangle = \langle Qx, x \rangle \]
%
which shows by Cauchy-Schwartz that $\| x \|^2 \leq \| x \| \| Qx \|$, so that $\| Qx \| \geq \| x \|$, and thus $Q$ is injective. Thus $Q$ is a bijection from $\mathcal{D}(Q)$ to $H$, which means that $I$ is an extension of $A_1Q$. The fact that $\langle Qx, x \rangle \geq 0$ for all $x \in \mathcal{D}(Q)$, means $\langle A_1 x, x \rangle \geq 0$ for all $x \in H$, i.e. $A_1$ is positive semidefinite. Since $A_1$ is injective and self-adjoint, this implies $Q = A_1^{-1}$ is self-adjoint. And this means $T^*T = Q - I$ is self adjoint for any closed unbounded operator $T$.

\begin{theorem}
    If $T: H \to H$ is a densely defined closed unbounded operator, then $\mathcal{D}(T^*)$ is dense, and $T^{**} = T$.
\end{theorem}
\begin{proof}
    We use the fact that $H \times H = V \Gamma(T) \oplus \Gamma(T^*)$. Suppose $y \perp \mathcal{D}(T^*)$. Then $(y,0)$ is perpendicular to $\Gamma(T^*)$. This means that $(y,0) \in V \Gamma(T)$, so $(0,-y) \in \Gamma(T)$, which implies $y = 0$. Thus $\mathcal{D}(T^*)$ is dense. To show $T^{**} = T$, we simply note that we can also write $H \times H = V \Gamma(T^{**}) \oplus \Gamma(T^*)$, from which it follows that $V \Gamma(T^{**}) = V \Gamma(T)$, and so $T = T^{**}$.
\end{proof}

A symmetric operator $T$ is \emph{maximally symmetric} if it has no proper symmetric extension. Any self-adjoint operator $T$ is maximally symmetric, for if $S$ is symmetric and extends $T$, then $T = T^*$ extends $S^*$, which means $S$ extends $S^*$, and since $S$ is symmetric, $S^*$ extends $S$, hence we conclude $S = S^*$, and as a result, $S = T$. On the other hand, maximally symmetric operators need not be self-adjoint.

\section{The Cayley Transform}

One trick to understanding symmetric operators $T$ is to consider the operator $T + i$. This is because if $T$ is symmetric, a simple calculation shows that for $x \in \mathcal{D}(T)$,
%
\[ \| Tx + i x \|^2 = \| x \|^2 + \| Tx \|^2. \]
%
Thus $\Gamma(T)$ is isometric to the image of $T + i$. Thus $T + i$ is injective, $T$ is closed if and only if the range of $T + i$ is closed, and if $T + i$ is onto $H$, then $T$ is maximally symmetric, because otherwise $T + i$ would have a symmetric extension which was still injective, which is impossible. Exactly the same results are true if we replace $i$ with any other purely imaginary quantity.

The mapping
%
\[ t \mapsto \frac{t - i}{t + i} \]
%
is a one-to-one-correspondence between $\RR$ and the unit circle in $\CC$ (except for one). It follows from the continuous functional calculus that the map
%
\[ T \mapsto (T - i)(T + i)^{-1} \]
%
is a one-to-one-correspondence between the family of self-adjoint bounded operator $T$ on a Hilbert space $H$, and the family of unitary operators on $H$ which do not contain $1$ in their spectrum. We wish to work with `unbounded' isometries, which will extend the theory of the Cayley transform to unbounded symmetric operators. An unbounded isometry is an unbounded operator $U: H \to H$ such that $\| Ux \| = \| x \|$ for each $x \in \mathcal{D}(U)$. Equivalently, this holds if and only if $\langle Ux, Uy \rangle = \langle x, y \rangle$ for each $x,y \in \mathcal{D}(U)$. If $U$ is an unbounded isometry, and $I - U$ has dense range, then $I - U$ is injective, for if $Ux = x$, then for any $y \in \mathcal{D}(U)$,
%
\[ \langle x, (I - U)y \rangle = \langle x, y \rangle - \langle x, Uy \rangle = \langle x,y \rangle - \langle Ux, Uy \rangle = 0. \]
%
Because the range of $I - U$ is dense, $x = 0$. It is similarily easy to prove that the three sets $\mathcal{D}(U)$, $U(\mathcal{D}(U))$, and $\Gamma(U)$ are all closed, or all not closed.

Let us now show the Caylet transform gives a correspondence between symmetric unbounded operators on $H$, and unbounded isometries on $H$ (unbounded operators $U: H \to H$ such that $\| Ux \| = \| x \|$ for each $x \in \mathcal{D}(U)$), or equivalently, $\langle Ux, Uy \rangle = \langle x, y \rangle$ for $x,y \in \mathcal{D}(U)$). For any symmetric operator $T$, since $\| Tx + ix \|^2 = \| x \|^2 + \| Tx \|^2$, the map $U = (T - i)(T + i)^{-1}$ is an unbounded isometry with domain is the image of the map $T + i$. The following properties then holds

\begin{theorem}
    Suppose $U$ is the Cayley transform of $T$.
    %
    \begin{itemize}
        \item $U$ is closed if and only if $T$ is closed.
        \item The map $I - U$ is injective from $\mathcal{D}(U)$ onto $\mathcal{D}(T)$, and $T = i(I + U)(I - U)^{-1}$.
        \item $U$ is unitary if and only if $T$ is self-adjoint.
    \end{itemize}
    %
    Conversely, any unbounded isometry on $H$ such that $I - U$ is injective is the Cayley transform of a symmetric operator.
\end{theorem}
\begin{proof}
    TODO
\end{proof}

If $U_1$ and $U_2$ are Cayley transforms of symmetric operators $T_1$ and $T_2$, then $T_1$ extends $T_2$ if and only if $U_1$ extends $U_2$. Thus problems about extensions of symmetric operators reduce to problems about extensions of isometries. Given an unbounded symmetric closed operator $T$, the range of $T + i$ and $T - i$ are closed, and $U$ is an isometry from the image of the first map to the image of the second map. The dimensions of $(T + i)^\perp$ and $(T - i)^\perp$ are called the \emph{deficiency indices} of the map $T$. Then it follows from the properties above that $T$ is self-adjoint if and only if both of its deficiency indices are zero, $T$ is maximally symmetric if and only if at least one of its deficiency indices is zero, and $T$ has a self-adjoint extension if and only if its two deficiency indices are equal.

\begin{example}
    Let $V$ be the right shift operator on $l^2$. Then $V$ is an isometry, and $I - V$ is injective. Thus $V$ is the Cayley transform of a symmetric operator $T$ on $l^2$. Since $V$ has domain $l^2$, and its range has codimension one, the deficiency indices of $T$ are zero and one. Thus $T$ is an example of a maximally symmetric operator which is not self adjoint.
\end{example}

\section{The Spectral Theory of Unbounded Symmetric Operators}

s






\section{Operator Semigroups}

TODO






\chapter{Von Neumann Algebras}





\section{Spectral Theorem for Normal Operators}

In finite dimensional theory, a normal linear transformation $T: \mathbf{C}^n \to \mathbf{C}^n$ can be written
%
\[ T = \sum \lambda_i P_i \]
%
where $P_i$ is projection onto the eigenspace corresponding to the eigenvalue $\lambda_i$. In this section we apply the theory of $C^*$ algebras to extend this theorem to arbitrary normal bounded transformations from a Hilbert space to itself. The trick is to `integrate' rather than sum up the subspaces.

A {\bf spectral measure} on a measure space $(\Omega, \mathcal{F})$ is a function $E: \mathcal{F} \to B(H)$, for which
%
\[ E(\Omega) = \text{id}_H \]
%
such that $E(S)$ is a projection for each $S$, and such that for a disjoint family of sets $\{ S_1, S_2, \dots \}$ in $\mathcal{F}$, then
%
\[ E \left( \bigcup S_i \right) = \sum E(S_i) \]
%
with convergence pointwise (the strong operator topology), rather than in the operator norm. It follows that $E(\emptyset) = 0$.

\begin{lemma}
    If $S \subset W$, $E(S) \leq E(W)$.
\end{lemma}
\begin{proof}
    Since $E(W - S)$ is a projection, $E(W - S) \geq 0$, so
    %
    \[ E(W) = E(W - S) + E(S) \geq E(S) \]
    %
    Hence a spectral measure is monotone on projections.
\end{proof}

\begin{lemma}
    If $S$ and $W$ are disjoint sets, then $E(S)E(W) = 0$.
\end{lemma}
\begin{proof}
    $E(S \cup W) = E(S) + E(W)$, so
    %
    \[ E(S \cup W) = E(S \cup W) E(S \cup W) = E(S) + E(W) E(S) + E(S)E(W) + E(W) \]
    %
    which implies
    %
    \[ E(S) + E(W) = E(S) + E(W) E(S) + E(S) E(W) + E(W) \]
    %
    hence $E(W) E(S) + E(S) E(W) = 0$.
\end{proof}

\begin{lemma}
    $E(S \cap W) = E(S) \circ E(W)$.
\end{lemma}
\begin{proof}
    We have
    %
    \[ E(S \cap W) + E(S - W) = E(S)\ \ \ \ \ \ \ \ E(S \cap W) + E(W - S) = E(W) \]
    %
    and
    %
    \[ E(S \cup W) = E(S \cap W) + E(S - W) + E(W - S) \]
    %
    From which it follows that
    %
    \begin{align*}
        E(S) + E(W) &= 2E(S \cap W) + E(S - W) + E(W - S)\\
        &= E(S \cap W) + E(S \cup W)
    \end{align*}
    %
    Now multiply both sides of the equation on the right by $E(W)$ gives
    %
    \[ E(S) E(W) + E(W)^2 = E(S \cap W) E(W) + E(S \cup W) E(W) \]
    %
    Since $S \subset S \cup W$, $E(S) \leq E(S \cup W)$, and
    %
    \[ E(S \cup W) E(S) = E(S) \]

    Now $E(S \cup W) E(S) = E(S)^2 + E(W - S) E(S) = E(S)$
\end{proof}

\begin{example}
    If $H = \mathbf{C}$, then $B(H) \cong \mathbf{C}$, and any $\{ 0, 1 \}$ valued measure $\mu$ is a spectral measure, provided $\mu(\Omega) = 1$. If $\mu(S_1) = \mu(S_2) = 1$, then
    %
    \[ 1 = \mu(S_1) = \mu(S_1 \cap S_2) + \mu(S_1 - S_2) \]
    %
    \[ 1 = \mu(S_2) = \mu(S_1 \cap S_2) + \mu(S_2 - S_1) \]
    %
    If $\mu(S_1 \cap S_2) = 0$, then $\mu(S_1 - S_2) = \mu(S_2 - S_1) = 1$, which implies
    %
    \[ \mu((S_1 - S_2) \cup (S_2 - S_1)) = \mu(S_1 - S_2) + \mu(S_2 - S_1) = 2 \]
    %
    an impossibility. Thus $\mu(S_1 \cap S_2) = 1$. If $\mu(S_1) = 0$, then $S_1 \cap S_2 \subset S_1$, so $\mu(S_1 \cap S_2) \leq \mu(S_1) = 0$. The same holds if $\mu(S_2) = 0$. Thus we have verified that $\mu(S_1 \cap S_2) = \mu(S_1) \circ \mu(S_2)$ on a case by case basis. The countable summation property holds by the property of the measure itself.
\end{example}

\begin{example}
    If $H$ is infinite dimensional, and $N \in K(H)$, then
    %
    \[ \sigma(N) = \{ \lambda_0, \lambda_1, \dots \} \]
    %
    where $\lambda_0 = 0$, and each $\lambda_i$ is an eigenvalue. For each $n > 0$, let $P_n$ be orthogonal projection onto $\text{ker}(\lambda_n - N)$, and let $P_0$ project onto the orthogonal complement of the closed linear space of the images of $P_1, P_2, \dots$. For $S \subset \sigma(N)$, let
    %
    \[ E(S) = \sum_{\lambda_n \in S} P_n \]
    %
    In the sense that the sum on the right is interpreted pointwise. Then $E$ is a spectral measure on $(\sigma(N), \mathcal{P}(\sigma(N)))$. Surely
    %
    \[ E(\sigma(N)) = \sum_{n = 0}^\infty P_n = \text{id}_H \]
    %
    And since $P_i \circ P_j = 0$ if $i \neq j$,
    %
    \begin{align*}
        E(S_1) \circ E(S_2) &= \left( \sum_{\lambda_n \in S_1} P_n \right) \circ  \left( \sum_{\lambda_m \in S_2} P_m \right)\\
        &= \sum_{\lambda_n \in S_1, \lambda_m \in S_2} P_n P_m\\
        &= \sum_{\lambda_n \in S_1 \cap S_2} P_n = E(S_1 \cap S_2)
    \end{align*}
    %
    And the countable additivity follows by the pointwise definition of the sum of operators.
\end{example}

\begin{example}
    If $\Omega$ is a $\sigma$-finite measure space with measure $\mu$, and $H = L^2(\Omega)$, define
    %
    \[ E(S) \xi = \chi_S \xi \]
    %
    Then $E$ is a spectral measure. Surely $E(\Omega) = \text{id}_H$, $E(\emptyset) = 0$, because $\chi_\Omega = 1$, $\chi_\emptyset = 0$. Now $E(S_1 \cap S_2) = E(S_1) \circ E(S_2)$ follows because $\chi_{S_1 \cap S_2} = \chi_{S_1} \chi_{S_2}$. Similarily, $\chi_{\bigcup S_i} = \sum \chi_{S_i}$ pointwise if the $S_i$ are disjoint.
\end{example}

\end{document}
