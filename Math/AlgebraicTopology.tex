\documentclass[12pt, dvipsnames]{report}

\usepackage{amsmath}
\usepackage{algorithm}
%\usepackage{algorithmic}
\usepackage[noend]{algpseudocode}

\usepackage{amsmath}
\usepackage{amssymb}
\usepackage{amsthm}
\usepackage{amsopn}

\usepackage{kpfonts}

\usepackage{graphicx}

% Probably don't need this on notes anymore
%\usepackage{kbordermatrix}

% Standard tool for drawing diagrams.
\usepackage{tikz}
\usepackage{tkz-berge}
\usepackage{tikz-cd}
\usepackage{tkz-graph}

\usepackage{comment}

%
\usepackage{multicol}

%
\usepackage{framed}

%
\usepackage{mathtools}

%
\usepackage{float}

%
\usepackage{subfig}

%
\usepackage{wrapfig}

%
\let\savewideparen\wideparen
\let\wideparen\relax
\usepackage{mathabx}
\let\wideparen\savewideparen

% Used for generating `enlightening quotes'
\usepackage{epigraph}

% Forget what this is used for :P
\usepackage[utf8]{inputenc}

% Used for generating quotes.
\usepackage{csquotes}

% Allows what to generate links inside
% generated pdf files
\usepackage{hyperref}

% Allows one to customize theorem
% environments in mathematical proofs.
\usepackage{thmtools}

% Gives access to a proof
\usepackage{lplfitch}

% I forget what this is for.
\usepackage{accents}

% A package for drawing simple trees,
% as a substitute for unnesacary TIKZ code
\usepackage{qtree}

% Enables sequent calculus proofs
\usepackage{ebproof}

% For braket notation
\usepackage{braket}

% To change line spacing when using mathematical notations which require some height!
\usepackage{setspace}

%\usepackage[dvipsnames]{xcolor}

\usepackage{float}

% For block commenting
\usepackage{comment}




\setlength\epigraphwidth{8cm}

\usetikzlibrary{arrows, petri, topaths, decorations.markings}

% So you can do calculations in coordinate specifications
\usetikzlibrary{calc}
\usetikzlibrary{angles}

\theoremstyle{plain}
\newtheorem{theorem}{Theorem}[chapter]
\newtheorem{axiom}{Axiom}
\newtheorem{lemma}[theorem]{Lemma}
\newtheorem{corollary}[theorem]{Corollary}
\newtheorem{prop}[theorem]{Proposition}
\newtheorem{exercise}{Exercise}[chapter]
\newtheorem{fact}{Fact}[chapter]

\newtheorem*{example}{Example}
\newtheorem*{proof*}{Proof}

\theoremstyle{remark}
\newtheorem*{exposition}{Exposition}
\newtheorem*{remark}{Remark}
\newtheorem*{remarks}{Remarks}

\theoremstyle{definition}
\newtheorem*{defi}{Definition}

\usepackage{hyperref}
\hypersetup{
    colorlinks = true,
    linkcolor = black,
}

\usepackage{textgreek}

\makeatletter
\renewcommand*\env@matrix[1][*\c@MaxMatrixCols c]{%
  \hskip -\arraycolsep
  \let\@ifnextchar\new@ifnextchar
  \array{#1}}
\makeatother

\renewcommand*\contentsname{\hfill Table Of Contents \hfill}

\newcommand{\optionalsection}[1]{\section[* #1]{(Important) #1}}
\newcommand{\deriv}[3]{\left. \frac{\partial #1}{\partial #2} \right|_{#3}} % partial derivative involving numerator and denominator.
\newcommand{\lcm}{\operatorname{lcm}}
\newcommand{\im}{\operatorname{im}}
\newcommand{\bint}{\mathbf{Z}}
\newcommand{\gen}[1]{\langle #1 \rangle}

\newcommand{\End}{\operatorname{End}}
\newcommand{\Mor}{\operatorname{Mor}}
\newcommand{\Id}{\operatorname{id}}
\newcommand{\visspace}{\text{\textvisiblespace}}
\newcommand{\Gal}{\text{Gal}}

\newcommand{\xor}{\oplus}
\newcommand{\ft}{\wedge}
\newcommand{\ift}{\vee}

\newcommand{\prob}{\mathbf{P}}
\newcommand{\expect}{\mathbf{E}}
\DeclareMathOperator{\Var}{\mathbf{V}}
\newcommand{\Ber}{\text{Ber}}
\newcommand{\Bin}{\text{Bin}}

%\newcommand{\widecheck}[1]{{#1}^{\ft}}

\DeclareMathOperator{\diam}{\text{diam}}

\DeclareMathOperator{\QQ}{\mathbf{Q}}
\DeclareMathOperator{\ZZ}{\mathbf{Z}}
\DeclareMathOperator{\RR}{\mathbf{R}}
\DeclareMathOperator{\HH}{\mathbf{H}}
\DeclareMathOperator{\CC}{\mathbf{C}}
\DeclareMathOperator{\AB}{\mathbf{A}}
\DeclareMathOperator{\PP}{\mathbf{P}}
\DeclareMathOperator{\MM}{\mathbf{M}}
\DeclareMathOperator{\VV}{\mathbf{V}}
\DeclareMathOperator{\TT}{\mathbf{T}}
\DeclareMathOperator{\LL}{\mathcal{L}}
\DeclareMathOperator{\EE}{\mathbf{E}}
\DeclareMathOperator{\NN}{\mathbf{N}}
\DeclareMathOperator{\DQ}{\mathcal{Q}}
\DeclareMathOperator{\IA}{\mathfrak{a}}
\DeclareMathOperator{\IB}{\mathfrak{b}}
\DeclareMathOperator{\IC}{\mathfrak{c}}
\DeclareMathOperator{\IP}{\mathfrak{p}}
\DeclareMathOperator{\IQ}{\mathfrak{q}}
\DeclareMathOperator{\IM}{\mathfrak{m}}
\DeclareMathOperator{\IN}{\mathfrak{n}}
\DeclareMathOperator{\IK}{\mathfrak{k}}
\DeclareMathOperator{\ord}{\text{ord}}
\DeclareMathOperator{\Ker}{\textsf{Ker}}
\DeclareMathOperator{\Coker}{\textsf{Coker}}
\DeclareMathOperator{\emphcoker}{\emph{coker}}
\DeclareMathOperator{\pp}{\partial}
\DeclareMathOperator{\tr}{\text{tr}}

\DeclareMathOperator{\supp}{\text{supp}}

\DeclareMathOperator{\codim}{\text{codim}}

\DeclareMathOperator{\minkdim}{\dim_{\mathbf{M}}}
\DeclareMathOperator{\hausdim}{\dim_{\mathbf{H}}}
\DeclareMathOperator{\lowminkdim}{\underline{\dim}_{\mathbf{M}}}
\DeclareMathOperator{\upminkdim}{\overline{\dim}_{\mathbf{M}}}
\DeclareMathOperator{\lhdim}{\underline{\dim}_{\mathbf{M}}}
\DeclareMathOperator{\lmbdim}{\underline{\dim}_{\mathbf{MB}}}
\DeclareMathOperator{\packdim}{\text{dim}_{\mathbf{P}}}
\DeclareMathOperator{\fordim}{\dim_{\mathbf{F}}}

\DeclareMathOperator*{\argmax}{arg\,max}
\DeclareMathOperator*{\argmin}{arg\,min}

\DeclareMathOperator{\ssm}{\smallsetminus}

\DeclareMathOperator{\Dom}{Dom}

\title{Topology}
\author{Jacob Denson}

\begin{document}

\pagenumbering{gobble}

\maketitle

\tableofcontents

\pagenumbering{arabic}

\chapter{Homotopy}

To verify two topological spaces are homeomorphic, we need only find a single homeomorphism. On the contrary, verifying two topological spaces are not homeomorphic is much more tricky; we need to show that every function from one space to the other is not a homeomorphism. One trick is to find fundamental topological properties which distinguish two topological spaces. Connectedness, Compactness, and Hausdorffiness are all preserved by homeomorphism, so if one space is connected, and the other is not, the two cannot be homeomorphic. Algebraic topology consists of deep techniques to distinguish between spaces.

Consider two continuous functions $f$ and $g$ between topological spaces $X$ and $Y$\footnote{From now on, we shall assume all functions continuous.}. Though $f$ might not be equal to $g$, they may be in some sense topologically equal -- we may be able to deform one to the other in a continuous fashion. This is a homotopy.

\begin{definition}
    Two continuous functions $f,g: X \to Y$ are {\bf homotopic}, written $f \simeq g$, if there exists a continuous map $H: [0,1] \times X \to Y$, such that $H(0,x) = f(x)$, and $H(1,x) = g(x)$. We say that $H$ is a homotopy. $H$ is a homotopy {\bf relative to} $A \subset X$ if $H(t,a) = a$ for all $a \in A$. If $H$ exists, $f \simeq_A g$.
\end{definition}

Homotopy forms an equivalence relation on continuous maps. It forms a category $\textbf{Toph}$, which is obtained as a quotient category of $\textbf{Top}$. 

\begin{example}
    Any two $\mathbf{R}^n$-valued functions $f$ and $g$ defined on the same domain are homotopic. The map
    %
    \[ H(t,x) = t f(x) + (1 - t) g(x) \]
    %
    is a homotopy between $f$ and $g$. Thus the spaces $\mathbf{R}^n$ are the terminal objects in the homotopy category.
\end{example}

Two spaces are isomorphic if they are isomorphic in $\textbf{Top}$ -- that is, $X$ is homotopic to $Y$, written $X \simeq_h Y$ if there is a map $f:X \to Y$ and $g: Y \to Z$ such that $g \circ f$ and $f \circ g$ are homotopic to the identities on $X$ and $Y$. $g$ is known as the {\bf homotopy inverse} of $f$.

\begin{example}
    The $n$-sphere $S^n$ is homotopic to $\mathbf{R}^{n+1} - \{ 0 \}$, with the embedding $i: S^n \to \mathbf{R}^{n+1} - \{ 0 \}$, and the map $f: \mathbf{R}^{n+1} - \{ 0 \} \to S^n$, defined by
    %
    \[ f(v) = v/\|v\| \]
    %
    One just expands the norm to obtain the homotopy.
\end{example}

We will not be able to approach algebraic topology on all the topological spaces, so we restrict ourselves to nice spaces. Manifolds are the first approximation to nice spaces, but we can get away with a more general construction.

\begin{definition}
    s
\end{definition}

A retraction is a map $r:X \to X$ for which $r^2 = r$. If $Y = \text{im}(X)$, and $r$ is a homotopy isomorphism, then we say $X$ {\bf deformation retracts} to $Y$. The deformation retraction is {\bf strong} if $X \simeq_Y Y$. We showed that $\mathbf{R}^{n+1} - \{ 0 \}$ strong deformation retracts to $S^n$. A space is {\bf contractible} or {\bf null-homotopic} if it is homotopic to a point.

\begin{example}
    A subset $X$ of a vector space is {\bf star-shaped} if there is $x \in X$ such that if $y \in X$, the line segment between $x$ and $y$ is contained in $X$. Then $X$ is (strongly) contractible to $\{ x \}$, by the map
    %
    \[ H(t,y) = ty + (1 - t)x \]
    %
    which shows every star shaped space is final.
\end{example}

\begin{example}
    Consider a connected graph $\Gamma$. Then $\Gamma$ is a {\bf tree} if there are no cycles: paths $(v_0, \dots, v_n)$ of distinct edges such that $v_n$ connects to $v_0$. In a tree, fix a vertex $v$. For any vertex $w$, there is a unique path $(w, k_1, \dots, k_n, v)$ of distinct vertices connecting $w$ and $v$. Take $m$ to be the longest such length of a path. We may identify the path from $w$ to $v$ with the interval $[0,n+2]$, by taking a trajectory $c_w:[0,n+2] \to \Gamma$ travelling at uniform velocity from $w$ to $v$, passing over each vertex at each integer mark. Extend $c_w$ to $[0,\infty]$ by defining $c_w(n+2+t) = v$ for $t > 0$. Define $H_w: [0,m] \times (w, k_1, \dots k_n, v) \to \Gamma$ by
    %
    \[ H_w(t, c_w(u)) = c_w(t + u) \]
    %
    If $w$ and $u$ are vertices, then $H_w$ agress with $H_u$ on the intersection of their domain, so we may put all the maps together to obtain a homotopy between the identity and a point on $\Gamma$, showing $\Gamma$ is contractible.
\end{example}

One may visualize homotopy for lower dimensional spaces in the following way. Let $f:X \to Y$ be a continuous map. Define the {\bf mapping cylinder}
%
\[ Z = (X \times [0,1]) \coprod Y / {\sim} \]
%
where $(x,1) \sim f(x)$. If $f$ is a homotopy equivalence, then $Z$ retracts to both $\pi(X \times \{0\}) = \tilde{X}$ and $\pi(Y) = \tilde{Y}$. Thus two spaces are homotopic if and only if they are both deformation retracts of a bigger space.

The fact that homotopy is an equivalence relation will allow us to distill functions between spaces to their fundamental properties. We need to specialize our definition for it to be more of more use to us.

\begin{definition}
    Two paths in $X$ are path homotopic if they have the same start and end point, and are homotopic to each other.
\end{definition}

Let $f$ and $g$ be two paths in $X$, where the end point of $f$ is the start point of $g$. Then we may compose the two paths to form a new path $f * g$, defined by
%
\[ (f * g)(x) = f(2x): x \in [0,1/2]
                g(2x - 1): x \in [1/2,1] \]
%
By the pasting lemma, this function is a path which connects the start point of $f$ to the end point of $g$. Unfortunately, concatenation is not associative, we do not have that $f * (g * h) = (f * g) * h$. These paths are homotopic to each other, however, and moving to path homotopy classes makes the definition much simpler.

\begin{theorem}
    Let $f$ be path homotopic to $f'$, and $g$ path homotopic to $g'$. Then $f * g$ is homotopic to $f' * g'$.
\end{theorem}
\begin{proof}
    Let $F$ be the path homotopy from $f$ to $f'$, and $G$ the path homotopy from $g$ to $g'$. Define a homotopy $H$ between $f * g$ and $f' * g'$ by
    %
    \[ H(\cdot,y) = F(\cdot, y) * G(\cdot, y) \]
    %
    More specifically
    %
    \[ H(x,y) = \begin{cases}
        F(2x,y) & \text{if } x \in [0,1/2]\\
        G(2x - 1,y) & \text{if } x \in [1/2,1]\\
\end{cases} \]
    %
    The pasting lemma guarentees this function is a new homotopy.
\end{proof}

We now consider homotopy classes of paths, so when we talk about a path $f$, we are really talking about all paths homotopic to $f$.

\begin{theorem}
    $[f] * ([g] * [h]) = ([f] * [g]) * [h]$.
\end{theorem}

\chapter{Compactification}

In the following, we consider only locally connected, locally compact, connected Hausdorff spaces.

\begin{definition}
    An {\bf end} of a $X$ is a map $\varepsilon$ defined on compact subsets of $X$, such that $\varepsilon(C)$ is a connected component of $X - C$ for each compact $C$, and $C \subset D$ implies $\varepsilon(D) \subset \varepsilon(C)$. Denote the set of all ends on $X$ by $\mathcal{E}(X)$.
\end{definition}

\begin{definition}
    The end compactification of a space $X$ is the space $\mathbf{X} = X \cup \mathcal{E}(X)$, where a set is open if it is open in $X$, or if it is of the form $U_{\varepsilon(C)} := \varepsilon(C) \cup \{ \varepsilon' \in \mathcal{E}(X) : \varepsilon'(C) = \varepsilon(C) \}$, where $C$ is compact.
\end{definition}

\begin{lemma}
    The end compactification is Hausdorff.
\end{lemma}
\begin{proof}
    If $\varepsilon, \varepsilon \in \mathcal{E}(X)$ are two unequal ends, then there is some compact set $C$ for which $\varepsilon(C) \neq \varepsilon'(C)$. But then $U_{\varepsilon(C)}$ and $U_{\varepsilon'(C)}$ are disjoint. If $x, y \in X$, then they can surely be separated in the end compactification because $X$ is Hausdorff. If $x$ is a point in $X$, and $\varepsilon$ is an end, then because $X$ is locally compact, $x$ possesses a precompact neighbourhood $V$ of $x$, and then $U_{\varepsilon(\overline{V})}$ is disjoint from $V$.
\end{proof}

\begin{lemma}
    $\mathbf{R}$ has two ends, the `left' and `right' ends.
\end{lemma}
\begin{proof}
    $\mathbf{R} = \bigcup_{n = 1}^\infty [-n, n]$, and each $[-n, n]$ is compact. We contend every end $\varepsilon$ on $\mathbf{R}$ is defined by its action on $[-n, n]$. If $C$ is any compact set, then $C$ is contained in an interval of the form $[-n, n]$. Clearly, $\varepsilon(C)$ must be the unique connected extension of $\varepsilon([-n, n])$, since $\varepsilon(C) \supset \varepsilon([-n,n])$. In fact, $\varepsilon$ is defined solely by its action on $[-1,1]$, since $[-1,1] \subset [-2,2] \subset \dots$. Since the two choices $\varepsilon([-x,x]) = (-\infty, x)$ and $\varepsilon([-x,x]) = (x,\infty)$ constitute ends, the space has two ends.
\end{proof}

In general, if a space $X$ can be written as $C_1 \subset C_2 \subset \dots \to X$, where each $C_i$ is compact, then all ends are defined by their action on $C_1$. We shall call such a space {\bf hemicompact}. Not all choices of components of $X - C_1$ will work, however.

\begin{lemma}
    The end compactification of a hemicompact space is compact.
\end{lemma}
\begin{proof}
    s
\end{proof}

\end{document}