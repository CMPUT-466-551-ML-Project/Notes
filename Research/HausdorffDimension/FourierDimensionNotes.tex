\documentclass[12pt,reqno]{article}

%\documentclass[dvipsnames,letterpaper,12pt]{article}

%\usepackage[margin = 1.5in]{geometry}
\usepackage{amsmath,amssymb,graphicx,mathabx,accents}
\usepackage{enumerate,mdwlist}

%\setlist[enumerate]{label*={\normalfont(\Alph*)},ref=(\Alph*)}

\numberwithin{equation}{section}

\usepackage{amsthm}
\usepackage{verbatim}

\usepackage{nag}

\DeclareMathOperator{\minkdim}{\dim_{\mathbf{M}}}
\DeclareMathOperator{\hausdim}{\dim_{\mathbf{H}}}
\DeclareMathOperator{\lowminkdim}{\underline{\dim}_{\mathbf{M}}}
\DeclareMathOperator{\upminkdim}{\overline{\dim}_{\mathbf{M}}}

\DeclareMathOperator{\lhdim}{\underline{\dim}_{\mathbf{M}}}
\DeclareMathOperator{\lmbdim}{\underline{\dim}_{\mathbf{MB}}}

\DeclareMathOperator{\RR}{\mathbf{R}}
\DeclareMathOperator{\ZZ}{\mathbf{Z}}
\DeclareMathOperator{\QQ}{\mathbf{Q}}

\DeclareMathOperator{\B}{\mathcal{B}}

\newtheorem{theorem}{Theorem}
\newtheorem{lemma}[theorem]{Lemma}
\newtheorem{corollary}[theorem]{Corollary}
\newtheorem{prop}[theorem]{Proposition}
\newtheorem{exercise}[theorem]{Exercise}
\newtheorem{remark}[theorem]{Remark}

\DeclareMathOperator{\EE}{\mathbf{E}}
\DeclareMathOperator{\PP}{\mathbf{P}}

\DeclareMathOperator{\DQ}{\mathcal{Q}}
\DeclareMathOperator{\DR}{\mathcal{R}}

\newcommand{\psitwo}[1]{\| {#1} \|_{\psi_2(L)}}
\newcommand{\TV}[2]{\| {#1} \|_{\text{TV}({#2})}}








\title{Salem Sets Avoiding Rough Configurations}
\author{Jacob Denson}

\begin{document}

\maketitle

\section{Introduction}

Recall that a set $X \subset \RR^d$ is a \emph{Salem set} of dimension $t$ if it has Hausdorff dimension $t$, and for every $\varepsilon > 0$, there exists a probability measure $\mu_\varepsilon$ supported on $X$ such that
%
\begin{equation} \label{equation1}
    \sup_{\xi \in \RR^d} |\xi|^{t - \varepsilon} |\widehat{\mu_\varepsilon}(\xi)| < \infty.
\end{equation}
%
It is a result of the Poisson summation formula that if $\mu_\varepsilon$ is compactly supported, then \eqref{equation1} is equivalent to the equation
%
\begin{equation}
    \sup_{k \in \ZZ^d} |k|^{t - \varepsilon} |\widehat{\mu_\varepsilon}(k)| < \infty.
\end{equation}
%
Our goal in these notes is to obtain high dimensional Salem sets avoiding rough configurations.

\begin{theorem} \label{maintheorem}
    Let $Z \subset [0,1]^{dn}$ be the countable union of sets, each with lower Minkowski dimension at most $s$. Then there exists a Salem set $X \subset \RR^d$ of dimension
    %
    \[ t = \frac{nd - s}{n}, \]
    %
    such that for any $n$ distinct elements $x_1, \dots, x_n \in X$, $(x_1, \dots, x_n) \not \in Z$.
\end{theorem}

We rely on a random selection approach, like in our paper on rough configurations, to obtain such a result, since such random selections give high probability bounds on the Fourier transform of the measures we study.

\section{Concentration Inequalities}

Define a convex function $\psi_2: [0,\infty) \to [0,\infty)$ by $\psi_2(t) = e^{t^2} - 1$, and a corresponding Orlicz norm on the family of scalar valued random variables $X$ over a probability space by setting
%
\[ \psitwo{X} = \inf \left\{ A \in (0,\infty) : \EE(\psi_2(|X|/A)) \leq 1 \right\}. \]
%
The family of random variables with $\psitwo{X} < \infty$ are known as \emph{subgaussian random variables}. Here are some important properties:
%
\begin{itemize}
	\item If $\psitwo{X} \leq A$, then for each $t \geq 0$,
	%
	\[ \PP \left( |X| \geq t \right) \leq 10 \exp \left( -t^2/10A^2 \right). \]
    %
    Thus Subgaussian random variables have Gaussian tails.

	\item If $|X| \leq A$ almost surely, then $\psitwo{X} \leq 10 A$. Thus bounded random variables are subgaussian.

	%\item (Centering) For any random variable $X$,
	%
	%\[ \psitwo{X - \EE(X)} \lesssim \psitwo{X}. \]
	
	%\item (Union Bound) If $X_1, \dots, X_N$ are random variables, then
	%
	%\[ \psitwo{X_1 + \dots + X_N} \leq \psitwo{X_1} + \dots + \psitwo{X_N}. \]
	
	\item If $X_1, \dots, X_N$ are \emph{independent}, then
	%
	\[ \psitwo{X_1 + \dots + X_N} \leq 10 \left( \psitwo{X_1}^2 + \dots + \psitwo{X_N}^2 \right)^{1/2}. \]
    %
    This is an equivalent way to state \emph{Hoeffding's Inequality}, and we refer to an application of this inequality as an application of Hoeffding's inequality.
\end{itemize}

\begin{remark}
    The constants involved in these statements are suboptimal, but will suffice for our purposes. Proofs can be found in Chapter 2 of \cite{Vershynin}.
\end{remark}

Roughly speaking, we can think of a random variable $X$ with $\psitwo{X} \leq A$ as a variable whose magnitude exceeds $A$ with extremely low probability. The Orlicz norm thus provides a convenient way to quantify concentration phenomena.

\section{A Family of Cubes}

Fix two integer-valued sequences $\{ K_m : m \geq 1 \}$ and $\{ M_m : m \geq 1 \}$. For convenience, we also define $N_m = K_m M_m$ for $m \geq 1$. We then define two sequences of real numbers $\{ l_m : m \geq 0 \}$ and $\{ r_m : m \geq 0 \}$, by
%
\[ l_m = \frac{1}{N_1 \dots N_m} \quad\text{and}\quad r_m = \frac{1}{N_1 \dots N_{m-1} M_m}. \]
%
For each $m, d \geq 0$, we define two collections of strings
%
\[ \Sigma_m^d = \ZZ^d \times [M_1]^d \times [K_1]^d \times \dots \times [M_m]^d \times [K_m]^d \]
%
and
%
\[ \Pi_m^d = \ZZ^d \times [M_1]^d \times [K_1]^d \times \dots \times [K_{m-1}]^d \times [M_m]^d. \]
%
For each string $\sigma = \sigma_0 \sigma_1 \dots \sigma_{2k} \in \Sigma_m^d$, we define a vector $a(\sigma) \in (l_m \ZZ)^d$ by setting
%
\[ a(\sigma) = \sigma_0 + \sum_{k = 1}^m \sigma_{2k-1} \cdot r_k + \sigma_{2k} \cdot l_k \]
%
Then each string $\sigma \in \Sigma_m^d$ can be identified with the sidelength $l_m$ cube $Q(\sigma)$ with left-hand corner lies at $a(\sigma)$, i.e. the cube
%
\[ Q(\sigma) = \prod_{i = 1}^d \left[ a(\sigma)_i, a(\sigma)_i + l_m \right]. \]
%
Similarily, for each string $\sigma = \sigma_0 \dots \sigma_{2m-1} \in \Pi_m^d$, we define a vector $a(\sigma) \in (r_m \ZZ)^d$ by setting, for each $1 \leq j \leq d$,
%
\[ a(\sigma) = \sigma_0 + \left( \sum_{k = 1}^{m-1} \sigma_{2k-1} \cdot r_k + \sigma_{2k} \cdot l_k \right) + \sigma_{2m-1} \cdot r_m, \]
%
and then define a sidelength $r_m$ cube
%
\[ R(\sigma) = \prod_{i = 1}^d \left[ a(\sigma)_i, a(\sigma)_i + r_m \right]. \]
%
We let $\DQ_m^d = \{ Q(\sigma) : \sigma \in \Sigma_m^d \}$, and $\DR_m^d = \{ R(\sigma) : \sigma \in \Pi_m^d \}$. We now list some important properties of this collection of cubes:
%
\begin{itemize}
	\item For each $m$, the two collections $\DQ_m^d$ and $\DR_m^d$ form covers of $\RR^d$.

	\item If $Q_1,Q_2 \in \bigcup_{m = 0}^\infty \DQ_m^d$, then either $Q_1$ and $Q_2$ have disjoint interiors, or one cube is contained in the other. Similarily, if $R_1,R_2 \in \bigcup_{m = 1}^\infty \DR_m^d$, then either $R_1$ and $R_2$ have disjoint interiors, or one cube is contained in the other.

	\item For each cube $Q \in \DQ_m$, there is a unique cube $Q^* \in \DR_m$ with $Q \subset Q^*$. We refer to $Q^*$ as the \emph{parent cube} of $Q$. Similarily, if $R \in \DR_m$, there is a unique cube in $R^* \in \DQ_{m-1}$ with $R \subset R^*$, and we refer to $R^*$ as the \emph{parent cube} of $R$.
\end{itemize}

We say a set $E \subset \RR^d$ is $\DQ_m$ discretized if it is a union of cubes in $\DQ_m^d$, and we then let $\DQ_m(E) = \{ Q \in \DQ_m^d : Q \subset E \}$. Similarily, we say a set $E \subset \RR^d$ is $\DR_m$ discretized if it is a union of cubes in $\DR_m^d$, and we then let $\DR_m(E) = \{ R \in \DR_m^d : R \subset E \}$. We set $\Sigma_m(E) = \{ \sigma \in \Sigma_m^d : Q(\sigma) \in \DQ_m(E) \}$, and $\Pi_m(E) = \{ \sigma \in \Pi_m^d : R(\sigma) \in \DR_m(E) \}$. We say a cube $Q_1 \times \dots \times Q_n \in \DQ_m^{dn}$ is \emph{strongly non diagonal} if there does not exist two distinct indices $i,j$, and a third index $\sigma \in \Pi_m^d$, such that $R_\sigma \cap Q_i, R_\sigma \cap Q_j \neq \emptyset$.

\section{A Family of Mollifiers}

We now consider a family of $C^\infty$ mollifiers, which we will use to ensure the Fourier transform of the measure we study have appropriate decay.

\begin{lemma} \label{lemma19020941290}
    There exists a non-negative, $C^\infty$ function $\psi$ supported on $[-1,1]^d$ such that
    %
    \begin{equation} \label{equation1249015901590190}
        \int_{\RR^d} \psi = 1,
    \end{equation}
    %
    and for each $x \in \RR^d$,
    %
    \begin{equation} \label{equation50914902149012}
        \sum_{n \in \ZZ^d} \psi(x + n) = 1.
    \end{equation}
\end{lemma}
\begin{proof}
    Let $\alpha$ be a non-negative, $C^\infty$ function compactly supported on $[0,1]$, such that $\alpha(1/2 + x) = \alpha(1/2 - x)$ for all $x \in \RR$, $\alpha(x) = 1$ for $x \in [1/3,2/3]$, and $0 \leq \alpha(x) \leq 1$ for all $x \in \RR$. Then define $\beta$ to be the non-negative, $C^\infty$ function supported on $[-1/3,1/3]$ defined for $x \in [-1/3,1/3]$ by
    %
    \[ \beta(x) = 1 - \alpha(|x|) = 1 - \alpha(1 - |x|). \]
    %
    Symmetry considerations imply that $\int_{\RR} \alpha + \beta = 1$, and for each $x \in \RR$,
    %
    \begin{equation} \label{equation129410294910}
        \sum_{m \in \ZZ} \alpha(x + m) + \beta(x + m) = 1.
    \end{equation}
    %
    If we set
    %
    \[ \psi_0(x) = \alpha(x) + \beta(x), \]
    %
    The function $\psi(x_1, \dots, x_d) = \psi_0(x_1) \dots \psi_0(x_d)$ then satisfies the constraints of the lemma.
\end{proof}

Fix some choice of $\psi$ given by Lemma \ref{lemma19020941290}. Since $\psi$ is $C^\infty$ and compactly supported, then for each $t \in [0,\infty)$, we conclude
%
\begin{equation} \label{equation682928418931289}
	\sup_{\xi \in \RR^d} |\xi|^t |\widehat{\psi}(\xi)| < \infty.
\end{equation}
%
Now we rescale the mollifier. For each integer $m \geq 1$, we let
%
\[ \psi_m(x) = l_m^{-d} \cdot \psi(l_m \cdot x). \]
%
Then $\psi_m$ is supported on $[-l_m,l_m]^d$. Equation \eqref{equation1249015901590190} implies that for each $x \in \RR^d$,
%
\begin{equation} \label{equation19204910490190190}
	\int_{\RR^d} \psi_m = 1.
\end{equation}
%
Equation \eqref{equation50914902149012} implies
%
\begin{equation} \label{equation990249012409129041290} \sum_{n \in \ZZ^d} \psi(x + l_m \cdot n) = l_m^{-d}. \end{equation}
%
An important property of the rescaling in the frequency domain is that for each $\xi \in \RR^d$,
%
\begin{equation} \label{equation12901902419209012}
    \widehat{\psi_m}(\xi) = \widehat{\psi}(l_m \cdot \xi),
\end{equation}
%
In particular, \eqref{equation12901902419209012} implies that for each $t \geq 0$,
%
\begin{equation}
    \sup_{|\xi| \in \RR^d} |\widehat{\psi_m}(\xi)| |\xi|^t = l_m^{-t} \sup_{|\xi| \in \RR^d} |\widehat{\psi}(\xi)| |\xi|^t.
\end{equation}
%
Intuitiely, $\{ \psi_m \}$ is a `uniform' family of wave packets, with $\psi_m$ supported in phase space on $[-l_m,l_m]^d$, and in frequency space, essentially supported on $[-l_m^{-1}, l_m^{-1}]^d$.

\section{Comparison to Previous Paper}

As in our previous paper, our proof of Theorem \ref{maintheorem} will involve constructing a configuration avoiding set $X$ by considering a nested decreasing family of sets $\{ X_m : m \geq 0 \}$, where $X_m \subset [0,1]^d$ is a $\DQ_m$ discretized set, and then setting $X = \bigcap_{m \geq 0} X_m$. We find a strong cover of $Z$ by sets $\{ B_m \}$, where $B_m$ is $\DQ_m$ discretized. Provided $X_m^d$ is disjoint from strongly non-diagonal cubes in $B_m$, we conclude that for any $n$ distinct elements $x_1, \dots, x_n \in X$, $(x_1,\dots,x_n) \not \in Z$. We now show that the technique of our last paper as stated fails to produce Salem sets.

Let us recap the approach of our last paper. To form $X_{m+1}$, we chose a cube $Q_R \in \DQ_{m+1}(R)$ uniformly at random, for each $R \in \DR_{m+1}(X_m)$. We then let $Y_{m+1} = \bigcup Q_R$. If $s \geq d$, and
%
\begin{equation} \label{equation12904190249102}
    K_{m+1} \approx M_{m+1}^{\frac{s - d}{dn - s}},
\end{equation}
%
then with non-zero probability, we proved there is $X_{m+1} \subset Y_{m+1}$ such that $X_{m+1}^d$ avoids strongly non-diagonal cubes in $B_{m+1}$, and $X_{m+1}$ contains at least half of the cubes in $\DQ_{m+1}(Y_{m+1})$. Then $X_{m+1}$ will be the union of at least $M_{m+1}^{-d}$ cubes with sidelength $l_{m+1}$. Provided that $K_{m+1}, M_{m+1} \gg K_1, M_1, \dots, K_m, M_m$, we have
%
\[ M_{m+1}^{-d} \approx r_{m+1}^{-d} \approx l_{m+1}^{-\frac{dn - s}{n-1}}. \]
%
Thus $X$ has lower Minkowski dimension at most $(dn - s)/(n-1)$, and a more involved analysis shows the set has Hausdorff dimension exactly equal to $(dn - s)/(n-1)$.

The approach detailed in the last paragraph is \emph{not} guaranteed to produce a set with Fourier dimension $t$. Because $X_{m+1}$ is random, it exhibits psuedorandomness properties with high probability. In particular, it supports probability measures whose Fourier transform has sharp decay. However, since the choice of the set $Y_{m+1}$ is \emph{not} chosen randomly from $X_{m+1}$, depending heavily on the set $Z$ and the discretized set $B_{m+1}$, the set $Y_{m+1}$ will in general not possess psuedorandomness properties. For instance, if $\mu$ is the probability measure induced by normalizing Lebesgue measure restricted to $X_{m+1}$, then with high probability,
%
\[ \| \widehat{\mu} \|_{L^\infty(\RR^d)} \approx l_m^t. \]
%
If $\nu$ is the probability measure induced by normalizing Lebesgue measure restricted to $Y_{m+1}$, then it is still possible for us to have
%
\[ \| \widehat{\nu} \|_{L^\infty(\RR^d)} \gtrsim 1. \]
%
For instance, this will be true if $\DQ_{m+1}(X_{m+1}) - \DQ_{m+1}(Y_{m+1})$ is a thickening of a subset of an arithmetic progression. Thus the method of our previous paper is not able to reliably produce Salem sets without further analysis on the psuedorandom properties of the sets $\{ B_m \}$ we have to avoid.

In this paper, we take a different approach which avoids us having to analyze the psuedorandomness of the sets $B_m$. Instead of \eqref{equation12904190249102}, we choose
%
\[ K_{m+1} \approx M_{m+1}^{\frac{s}{dn - s}}. \]
%
Notice that $M_{m+1}^{\frac{s}{dn - s}} \geq M_{m+1}^{\frac{s-d}{dn - s}}$, so the set $Y_{m+1}$ we will obtain will be a thinner set than $X_m$. In particular, $Y_{m+1}$ will be covered by at most $M_{m+1}^{-d}$ sidelength $l_{m+1}$ cubes, and if $K_{m+1}, M_{m+1} \gg K_1,M_1,\dots,K_m,M_m$< 
%
\[ r_{m+1}^{-d} \approx l_{m+1}^{-t} \]
%
sidelength $l_{m+1}$ cubes, which implies $X$ will have upper Minkowski dimension at most $t$. However, as a result, because the set $Y_{m+1}$ is thinner, we find that $Y_{m+1}^d$ is disjoint from the cubes in $B_{m+1}$ with high probability. In particular, we can set $X_{m+1} = Y_{m+1}$. This means that $X_{m+1}$ will be pseudorandom, and we should therefore expect $X$ to be a Salem set of dimension $t$. The remainder of this paper is devoted to showing that these heuristics are correct.

\section{Discrete Lemma}

We now proceed to solve a discretized version of Theorem \ref{maintheorem}.

\begin{prop} \label{discreteLemma}
    Fix $s \in [1,dn)$ and $\varepsilon \in [0,(dn-s)/2)$. Let $T \subset [0,1]^d$ be a non-empty, $\DQ_m$ discretized set, and let $\mu_T$ be a smooth measure compactly supported on $T$. Let $B \subset \RR^{dn}$ be a non-empty, $\DQ_{m+1}$ discretized set such that
    %
    \begin{equation} \label{equation1290419204912090120912}
       \#(\DQ_{m+1}(B)) \leq (1/l_{m+1})^{s + \varepsilon}.
    \end{equation}
    %
    Then there exists a large constant $C(\mu_T,l_m,n,d,s,\varepsilon)$, such that if
    %
    % Must depend on d and \mu
    \begin{equation} \label{equation1095121284102}
        K_{m+1}, M_{m+1} \geq C(\mu_T,l_m,n,d,s,\varepsilon,l_m),
    \end{equation}
    %
    and
    %
    % Must depend on l_m, d, n, and s
    \begin{equation} \label{equation5890129048128941891}
        M_{m+1}^{\frac{s}{dn-s} + c\varepsilon} \leq K_{m+1} \leq 2 M_{m+1}^{\frac{s}{dn-s} + c \varepsilon},
    \end{equation}
    %
    where
    %
    \[ c = \frac{6dn}{(dn - s)^2}, \]
    %
    then there exists a $\DQ_{m+1}$ discretized set $S \subset T$ together with a smooth probability measure supported on $S$ such that
    %
    \begin{enumerate}
        \item[(A)] For any strongly non-diagonal cube
        %
        \[ Q = Q_1 \times \dots \times Q_n \in \DQ_{m+1}(B), \]
        %
        there exists $i$ such that $Q_i \not \in \DQ_{m+1}(S)$.

        \item[(B)] $\mu_S(\RR^d) \geq \mu_T(\RR^d) - M_{m+1}^{-1/2}$.

        \item[(C)] If $|k| \leq 10 l_{m+1}^{-d}$, $|\widehat{\mu_T}(k) - \widehat{\mu_S}(k)| \leq r_{m+1}^{d/2} \log(M_{m+1})$.

        \item[(D)] If $|k| \geq 10 l_{m+1}^{-d}$, $|\widehat{\mu_S}(k)| \leq |k|^{-d/2}$.
    \end{enumerate}
\end{prop}

\begin{remark}
    To make the statement of Proposition \eqref{discreteLemma} more clean, we have hidden the explicit choice of constant $C(\mu_T,l_m,n,d,s,\varepsilon)$. But this constant can certainly be made explicit; such a choice can be made by ensuring that \eqref{equation1095121284102} implies \eqref{equation10491249012}, \eqref{equation194012904129009}, \eqref{equation1940129041}, \eqref{equation129041902412901290129}, \eqref{equation190512919204912901}, and \eqref{equation68943893493849} all hold.%\eqref{equation1290412904129049102}, and \eqref{equation129301923109}.
\end{remark}

\begin{proof}[Proof of Proposition \ref{discreteLemma}]
    \renewcommand{\qedsymbol}{}
    First, we describe the construction of the set $S$, and the measure $\mu_S$. For each string $\sigma \in \Pi_{m+1}^d$, let $j_\sigma$ be a random integer vector chosen from $\{ 0, \dots, K_{m+1} - 1 \}^d$, such that the family $\{ j_\sigma : \sigma \in \Pi_{m+1}^d \}$ is an independent family of random variables. Then it is certainly true that for any $j \in [K_{m+1}]^d$,
    %
    \begin{equation} \label{equation129412904912090}
        \PP(j_\sigma = j) = K_{m+1}^{-d}.
    \end{equation}
    %
    Then $\sigma j_\sigma \in \Sigma_{m+1}^d$. We can thus define a measure $\mu_S$ such that, for each $x \in \RR^d$,
    %
    \[ d\mu_S(x) = r_{m+1}^d \sum\nolimits_{\sigma \in \Pi_{m+1}^d} \psi_{m+1}(x - a(\sigma j_\sigma)) \cdot d\mu_T(x). \]
    %
    If we set
    %
    \[ S = \bigcup \{ Q \in \DQ_{m+1}^d : \mu_S(Q) > 0 \}, \]
    %
    then $S$ is $\DQ_{m+1}$ discretized, $\mu_S$ is supported on $S$, and $S \subset T$. Our goal is to show that, with non-zero probability, some choice of the family of indices $\{ j_\sigma : \sigma \in \Pi_{m+1}^d \}$ yields a set $S$ and a measure $\mu_S$ satisfying Properties (A) and (B) of Proposition \ref{discreteLemma}. In our calculations, it will help us to decompose the measure $\mu_S$ into components roughly supported on sidelength $r_{m+1}^d$ cubes. For each $\sigma \in \Pi_{m+1}(T)$, define a measure $\mu_\sigma$ such that for each $x \in \RR^d$,
    %
    \[ d\mu_\sigma(x) = r_{m+1}^d \psi_{m+1}(x - a(\sigma j_\sigma)) \cdot d\mu_T(x). \]
    %
    Then $\mu_S = \sum_{\sigma \in \Pi_{m+1}^d(T)} \mu_\sigma$. We shall split the proof of Properties (A), (B), and (C) into several more managable lemmas.
\end{proof}

\begin{lemma} \label{nuNormalizationLemma}
    If
    %
    \begin{equation} \label{equation10491249012}
        M_{m+1} \geq \left( 3^d \sqrt{d} \cdot l_m \| \nabla \mu \|_{L^\infty(\RR^d)} \right)^2,
    \end{equation}
    %
    then $\mu_S(\RR^d) \geq \mu_T(\RR^d) - M_{m+1}^{-1/2}$.
\end{lemma}
\begin{proof}
    Fix $\sigma \in \Pi_{m+1}^d$. If $j_0, j_1 \in \{ 0, \dots, K_{m+1} - 1 \}^d$, then
    %
    \begin{equation} \label{equation99210391920}
        |a(\sigma j_0) - a(\sigma j_1)| = |j_0 - j_1| \cdot l_{m+1} \leq (\sqrt{d} \cdot K_{m+1}) \cdot l_{m+1} = \sqrt{d} \cdot r_{m+1}.
    \end{equation}
    %
    Together with \eqref{equation19204910490190190}, \eqref{equation99210391920} implies
    %
    \begin{equation} \label{equation92941294129412919}
    \begin{split}
        &\left| r_{m+1}^d \int_{\RR^d} \psi_{m+1}(x - a(\sigma j_0)) \mu_T(x) - r_{m+1}^d \int_{\RR^d} \psi_{m+1}(x - a(\sigma j_1)) \mu_T(x) \right|\\
        &\ \ \ \ \ \ \ \ \leq r_{m+1}^d \int_{\RR^d} \psi_{m+1}(x) \left| \mu_T(x + a(\sigma j_0)) - \mu_T(x + a(\sigma j_1)) \right|\\
        &\ \ \ \ \ \ \ \ \leq \sqrt{d} \cdot r_{m+1}^{d+1} \cdot \| \nabla \mu_T \|_{L^\infty(\RR^d)} \int_{\RR^d} \psi_{m+1}\\
        &\ \ \ \ \ \ \ \ = \sqrt{d} \cdot r_{m+1}^{d+1} \cdot \| \nabla \mu_T \|_{L^\infty(\RR^d)}.
    \end{split}
    \end{equation}
    %
    %For each $i \in \Sigma_{m+1}^d$, if we set
    %
    %\[ A_i = \EE(\nu_i(\RR^d)) = \frac{1}{N_{n+1}^d} \sum_{j \in [N_{n+1}]^d} \int_{\RR^d} \psi_{m+1}(x - a_{ij}) \mu_T(x)\; dx, \]
    %
    Thus $\eqref{equation92941294129412919}$ implies that for each $\sigma$,
    %
    \begin{equation} \label{equation491040912491}
        |\mu_\sigma(\RR^d) - \EE(\mu_\sigma(\RR^d))| \leq \sqrt{d} \cdot r_{m+1}^{d+1} \| \nabla \mu \|_{L^\infty(\RR^d)}.
    \end{equation}
    %
    Furthermore, \eqref{equation990249012409129041290} implies
    %
    \begin{equation} \label{9921490124912}
    \begin{split}
        &\sum\nolimits_{\sigma \in \Pi_{m+1}^d} \EE(\mu_\sigma(\RR^d))\\
        &\ \ \ \ \ = r_{m+1}^d \sum\nolimits_{(\sigma,j) \in \Sigma_{m+1}^d} \PP(j_\sigma = j) \int_{\RR^d} \psi_{m+1}(x - a(\sigma j_\sigma)) d\mu_T(x)\\
        &\ \ \ \ \ = \frac{r_{m+1}^d}{K_{m+1}^d} \int_{\RR^d} \left( \sum\nolimits_{(\sigma,j) \in \Sigma_{m+1}^d} \psi_{m+1}(x - a(\sigma j)) \right) d\mu_T(x)\\
        &\ \ \ \ \ = \frac{r_{m+1}^d l_{m+1}^{-d}}{K_{m+1}^d} \mu_T(\RR^d) = \mu_T(\RR^d).
    \end{split}
    \end{equation}
    %
    For all but at most $3^d r_{m+1}^{-d}$ indices $\sigma \in \Pi_{m+1}^d$, $\mu_\sigma = 0$ almost surely. Thus we can apply the triangle inequality together with \eqref{equation491040912491} and \eqref{9921490124912} to conclude that
    %
    \begin{equation} \label{equation42214124124102412}
    \begin{split}
        |\mu_S(\RR^d) - \mu_T(\RR^d)| &= \left| \sum\nolimits_{\sigma \in \Pi_{m+1}^d} \left[ \mu_\sigma(\RR^d) - \EE(\mu_\sigma(\RR^d)) \right] \right| \\
        &\leq \sum\nolimits_{\sigma \in \Pi_{m+1}^d} \left| \mu_\sigma(\RR^d) - \EE(\mu_\sigma(\RR^d)) \right|\\
        &\leq (3^d r_{m+1}^{-d}) \left( \sqrt{d} \cdot r_{m+1}^{d+1} \| \nabla \mu \|_{L^\infty(\RR^d)} \right)\\
        &= \left( 3^d \sqrt{d} \| \nabla \mu \|_{L^\infty(\RR^d)} \right) \cdot r_{m+1}\\
        &= \frac{3^d \sqrt{d} \cdot l_m \| \nabla \mu \|_{L^\infty(\RR^d)}}{M_{m+1}}.
    \end{split}
    \end{equation}  
    %
    Thus \eqref{equation10491249012} and \eqref{equation42214124124102412} imply that, $|\mu_S(\RR^d) - \mu_T(\RR^d)| \leq M_{m+1}^{-1/2}$.
\end{proof}

\begin{lemma} \label{propertyALemma}
    If
    %
    \begin{equation} \label{equation194012904129009}
        M_{m+1} \geq \left( 10 \cdot 3^{dn} \cdot l_m^{-(s + \varepsilon)} \right)^{1/\varepsilon},
    \end{equation}
    %
    then
    %
    \[ \PP \left(S\ \text{\normalfont{does not satisfies Property (A)}} \right) \leq 1/10. \]
\end{lemma}
\begin{proof}
    For any cube $Q \in \Sigma_{m+1}^d$, there are at most $3^d$ indices $\sigma j \in \Sigma_{m+1}^d$ such that $Q_{\sigma j} \cap Q \neq \emptyset$, and so a union bound together with \eqref{equation129412904912090} gives
    %
    \begin{equation} \label{equation5901490129129012409}
        \PP(Q \in \DQ_{m+1}(S)) \leq \sum\nolimits_{Q_{\sigma j} \cap Q \neq \emptyset} \PP(j_\sigma = j) \leq 3^d K_{m+1}^{-d}.
    \end{equation}
    %
    Without loss of generality, removing cubes from $B$ if necessary, we may assume all cubes in $B$ are strongly non-diagonal. Let $Q = Q_1 \times \dots \times Q_n \in \DQ_{m+1}(B)$ be such a cube. Since $Q$ is strongly diagonal, the events $\{ Q_k \in S \}$ are independent from one another for $k \in \{ 1, \dots, n \}$, which together with \eqref{equation5901490129129012409} implies that
    %
    \begin{equation} \label{equation190589012590812892189}
       \PP(Q \in \DQ_{m+1}(S^n)) = \PP(Q_1 \in S) \dots \PP(Q_n \in S) \leq 3^{dn} K_{m+1}^{-dn}.
    \end{equation}
    %
    Taking expectations over all cubes in $B$, and applying \eqref{equation1290419204912090120912} and \eqref{equation190589012590812892189} gives
    %
    \begin{equation} \label{equation129041289589128921891289}
    \begin{split}
        \EE(\#(\DQ_{m+1}(B) \cap \DQ_{m+1}(S^n))) &\leq \#(\DQ_{m+1}(B)) \cdot (3^{dn} K_{m+1}^{-dn})\\
        &\leq l_{m+1}^{-(s + \varepsilon)} (3^{dn} K_{m+1}^{- dn})\\
        &= \frac{3^{dn} l_m^{-(s + \varepsilon)} M_{m+1}^{s + \varepsilon}}{K_{m+1}^{dn - s - \varepsilon}}.
    \end{split}
    \end{equation}
    % M^{(s + e)/(dn - s - e)} <= K
    % (s + e)/(dn - s - e) \leq (s + e[s + 3])/(dn - s)
    %
    Since $\varepsilon \leq (dn - s)/2$, we conclude
    %
    \begin{align*}
        \left( dn - s - \varepsilon \right) \left( \frac{s}{dn - s} + c\varepsilon \right) &= s + \varepsilon \left( c(dn - s - \varepsilon) - \frac{s}{dn - s} \right)\\
        &\geq s + \varepsilon \left( \frac{c(dn - s)}{2} - \frac{s}{dn - s} \right)\\
        &= s + \varepsilon \frac{3dn - s}{dn - s} \geq s + 2\varepsilon.
    \end{align*}
    %
    Applying \eqref{equation5890129048128941891}, we therefore conclude that
    %
    \begin{align*}
        K_{m+1}^{dn - s - \varepsilon} &\geq M_{m+1}^{(dn - s - \varepsilon) \left( \frac{s}{dn - s} + c\varepsilon \right)} \geq M_{m+1}^{s + 2 \varepsilon}.
    \end{align*}
    %
    Combined with \eqref{equation194012904129009}, we conclude that
    %
    \begin{equation} \label{equation1290419024190}
        \frac{3^{dn} l_m^{-(s + \varepsilon)} M_{m+1}^{s + \varepsilon}}{K_{m+1}^{dn - s - \varepsilon}} \leq \frac{3^{dn} l_m^{-(s + \varepsilon)}}{M_{m+1}^\varepsilon} \leq 1/10.
    \end{equation}
    %
    We can then apply Markov's inequality with \eqref{equation129041289589128921891289} and \eqref{equation1290419024190} to conclude
    %
    \begin{equation} \label{fourierdim2}
    \begin{split}
        \PP(\DQ_{k+1}(B) \cap \DQ_{k+1}(S^n) \neq \emptyset) &= \PP(\# (\DQ_{k+1}(B) \cap \DQ_{k+1}(S^n)) \geq 1)\\
        &\leq \EE(\#(\DQ_{m+1}(B) \cap \DQ_{m+1}(S^n)))\\
        &\leq 1/10. \qedhere
    \end{split}
    \end{equation}
    %
%    Thus $\DQ_{k+1}(S^n)$ is disjoint from $\DQ_{k+1}(B)$ with high probability.
\end{proof}

\begin{lemma} \label{deviationLemma}
    Set $D = \{ k \in \ZZ^d: |k| \leq 10l_{m+1}^{-1} \}$. Then if
    %
    \begin{equation} \label{equation109519029012}
        K_{m+1} \leq M_{m+1}^{\frac{2dn}{dn - s}},
    \end{equation}
    %
    and
    %
    \begin{equation} \label{equation1940129041}
        M_{m+1} \geq \exp \left( \frac{10^7 (3dn - s) d^2}{dn - s} \right),
    \end{equation}
    %
    then
    %
    \begin{equation} \label{equation12901904192090129102}
        \PP \left( \| \widehat{\mu_S} - \widehat{\mu_T} \|_{L^\infty(D)} \geq r_{m+1}^{d/2} \log(M_{m+1}) \right) \leq 1/10
    \end{equation}
\end{lemma}
\begin{proof}
    For each $\sigma \in \Pi_{m+1}^d$, and $k \in \ZZ$, define $X_{\sigma k} = \widehat{\mu_\sigma}(k) - \widehat{\EE(\mu_\sigma)}(k)$. Applying \eqref{equation50914902149012} gives 
    %
    \begin{equation} \label{equation891248921894128942189}
    \begin{split}
        \sum_{\sigma \in \Pi_{m+1}^d} \widehat{\EE(\mu_\sigma)}(k) &= \sum_{\sigma \in \Pi_{m+1}^d} l_{m+1}^d \sum_{j \in [K_{m+1}]^d} \int_{\RR^d} e^{- 2 \pi i k \cdot x} \psi_{m+1}(x - a(\sigma j)) d\mu_T(x)\\
        &= \int_{\RR^d} e^{-2 \pi i k \cdot x} d\mu_T(x) = \widehat{\mu_T}(k).
    \end{split}
    \end{equation}
    %
    For each $\sigma$ and $k$, the standard $(L^1,L^\infty)$ bound on the Fourier transform, combined with \eqref{equation491040912491}, shows
    %
    \begin{equation} \label{equation12904912049012}
    \begin{split}
        \psitwo{X_{\sigma k}} &\leq 10 | X_{\sigma k} |\\
        &\leq 10[| \mu_\sigma(\RR^d) | + \EE(\mu_\sigma)(\RR^d)]\\
        &\leq 10^2 \left( \EE(\mu_\sigma)(\RR^d) + \sqrt{d} \cdot r_{m+1}^{d+1} \| \nabla \mu_T \|_{L^\infty(\RR^d)} \right).
    \end{split}
    \end{equation}
    %
    For a fixed $k$, the family of random variables $\{ X_{\sigma k} : \sigma \in \Pi_{m+1}^d \}$ are independent. Furthermore, $\sum X_{\sigma k} = \widehat{\mu_S}(k) - \widehat{\EE(\mu_S)}(k)$. Equations \eqref{equation990249012409129041290} and \eqref{equation129412904912090} imply that
    %
    \begin{equation} \label{equation19241902490129021}
    \begin{split}
        \EE(\widehat{\mu_S}(k)) &= \frac{r_{m+1}^d}{K_{m+1}^d} \int_{\RR^d} e^{-2 \pi i k \cdot x} \left( \sum_{(\sigma ,j) \in \Sigma_{m+1}^d} \psi_{m+1}(x - a(\sigma j)) \right) d\mu_T(x)\\
        &= \frac{r_{m+1}^d l_{m+1}^{-d}}{K_{m+1}^d} \int_{\RR^d} e^{-2 \pi i k \cdot x} d\mu_T(x)\\
        &= \frac{r_{m+1}^d l_{m+1}^{-d}}{K_{m+1}^d} \widehat{\mu_T}(k) = \widehat{\mu_T}(k).
    \end{split}
    \end{equation}
    %
    Hoeffding's inequality, together with \eqref{equation12904912049012} and \eqref{equation19241902490129021}, imply that
    %
    \begin{equation} \label{equation190219024901290129041}
    \begin{split}
        & \psitwo{\widehat{\mu}(k) - \widehat{\mu_T}(k)}\\
        &\ \ \ \ \ \ \ \ \ \ \leq 10^3 \sqrt{d} \left( \left( \sum \EE(\mu_\sigma)(\RR^d)^2 \right)^{1/2} + r_{m+1}^{d/2+1} \| \nabla \mu_T \|_{L^\infty(\RR^d)} \right).
    \end{split}
    \end{equation}
    %
    Equation \eqref{equation19204910490190190} shows
    %
    \begin{equation} \label{equation129401924901290412}
    \begin{split}
        \EE(\mu_\sigma)(\RR^d) &= l_{m+1}^d \sum_{j \in [K_{m+1}]^d} \int \psi_{m+1}(x - a(ij)) d\mu_T(x)\\
        &\leq r_{m+1}^d \| \mu_T \|_{L^\infty(\RR^d)}.
    \end{split}
    \end{equation}
    %
    Combining \eqref{equation190219024901290129041} and \eqref{equation129401924901290412} gives
    %
    \begin{equation} \label{equation10491204901290}
        \psitwo{\widehat{\mu}(k) - \widehat{\mu_T}(k)} \leq 10^3 \sqrt{d} \left[ \| \mu_T \|_{L^\infty(\RR^d)} + \| \nabla \mu_T \|_{L^\infty(\RR^d)} \right] r_{m+1}^{d/2}.
    \end{equation}
    %
    We can then apply a union bound over the set $D$, which has cardinality at most $10^{d+1} l_{m+1}^{-d}$, together with \eqref{equation10491204901290} to conclude that
    %
    \begin{equation} \label{equation1290421941021290}
    \begin{split}
        \PP& \left(\| \widehat{\mu_S} - \widehat{\mu_T} \|_{L^\infty(D)} \geq r_{m+1}^{d/2} \log(M_{m+1}) \right)\\
        &\ \ \ \ \ \leq 10^{d+2} \cdot l_{m+1}^{-d} \exp \left( - \frac{\log(M_{m+1})^2}{10^7 d} \right)\\
        &\ \ \ \ \ = 10^{d+2} l_m^{-d} \exp \left( d \log(M_{m+1} K_{m+1}) - \frac{\log(M_{m+1})^2}{10^7 d} \right).
    \end{split}
    \end{equation}
    %
    Combined with \eqref{equation109519029012} and \eqref{equation1940129041}, \eqref{equation1290421941021290} implies
    %
    \begin{equation} \label{equation0148912489128}
        \PP \left(\| \widehat{\mu_S} - \widehat{\mu_T} \|_{L^\infty(D)} \geq r_{m+1}^{d/2} \log(M_{m+1}) \right) \leq 1/10.
    \end{equation}
    %
    Thus $\widehat{\mu_S}$ and $\widehat{\mu_T}$ are highly likely to differ only by a negligible amount over small frequencies.
\end{proof}

Since $\mu_T$ is compactly supported, we can define, for each $t > 0$,
%
\[ A(t) = \sup_{\xi \in \RR^d} |\widehat{\mu_T}(\xi)| |\xi|^t < \infty. \]
%
In light of \eqref{equation12901902419209012}, if we define, for each $t > 0$,
%
\[ B(t) = \sup_{\xi \in \RR^d} |\widehat{\psi}(\xi)| |\xi|^t, \]
%
then
%
\[ \sup_{\xi \in \RR^d} |\widehat{\psi_{m+1}}(\xi)| |\xi|^t = l_{m+1}^{-t} B(t). \]

\begin{lemma} \label{largeFrequencyLemma}
    Suppose that
    %
    \begin{equation} \label{equation129041902412901290129}
        N_{m+1}^d \geq 10 \cdot 2^{3d/2+1} A(3d/2 + 1),
    \end{equation}
    %
    \begin{equation} \label{equation190512919204912901}
    \begin{split}
        N_{m+1}^d \geq \frac{10 \cdot 2^{3d}}{1 + d/2} A(3d/2 + 1),
    \end{split}
    \end{equation}
    %
    and
    %
    \begin{equation} \label{equation68943893493849}
        N_{m+1}^d \geq 10 \cdot 2^{7d/2 + 1} B(3d/2 + 1).
    \end{equation}
    %
    then if $|\eta| \geq 10l_{m+1}^{-1}$,
    %
    \begin{equation} \label{equation66123142190}
        |\widehat{\mu_S}(\eta)| \leq \frac{1}{|\eta|^{d/2}}.
    \end{equation}
\end{lemma}
\begin{proof}
    Define a random measure
    %
    \[ \alpha = r_{m+1}^d \sum\nolimits_{\substack{\sigma \in \Pi_{m+1}^d\\d(a(\sigma),T) \leq 2 r_{m+1}^{-1}}} \delta_{a(ij_i)}. \]
    %
    Then $\mu_S = (\alpha * \psi_{m+1}) \mu_T$.
    %Since $\mu_T$ is supported on $T$, we can really truncate the sum defining $\eta$ to $O_d(r_{m+1}^{-d})$ terms such that the convolution identity remains valid.
    Thus we have $\widehat{\mu_S} = (\widehat{\alpha} \cdot \widehat{\psi_{m+1}}) * \widehat{\mu_T}$. The measure $\alpha$ is the sum of at most $2^d r_{m+1}^{-d}$ delta functions, scaled by $r_{m+1}^d$, so $\| \widehat{\alpha} \|_{L^\infty(\RR^d)} \leq \alpha(\RR^d) \leq 2^d$. Thus
    %
    \begin{equation} \label{equation9942941924912912}
        |\widehat{\mu_S}(\eta)| \leq 2^d \int |\widehat{\mu_T}(\eta - \xi)| |\widehat{\psi_{m+1}}(\xi)|\; d\xi.
    \end{equation}
    %
    If $|\xi| \leq |\eta|/2$, $|\eta - \xi| \geq |\eta|/2$, and since \eqref{equation19204910490190190} implies $\| \widehat{\psi_{m+1}} \|_{L^\infty(\RR^d)} \leq 1$, we find that for all $t > 0$,
    %
    \begin{equation} \label{equation999941204192049012}
        \int_{0 \leq |\xi| \leq |\eta|/2} |\widehat{\mu_T}(\eta - \xi)| |\widehat{\psi_{m+1}}(\xi)|\; d\xi \leq \frac{A(t) 2^{t-d}}{|\eta|^{t-d}}.
    \end{equation}
    %
    Set $t = 3d/2 + 1$. Equation \eqref{equation999941204192049012}, together with \eqref{equation129041902412901290129}, implies
    %
    \begin{equation} \label{equation1111902491209012}
    \begin{split}
        &\int_{0 \leq |\xi| \leq |\eta|/2} |\widehat{\mu_T}(\eta - \xi)| |\widehat{\psi_{m+1}}(\xi)|\; d\xi\\
        &\ \ \ \ \ \ \ \leq \frac{A(3d/2 + 1) 2^{1 + d/2} |\eta|^{-1}}{|\eta|^{d/2}}\\
        &\ \ \ \ \ \ \ \leq \frac{A(3d/2 + 1) 2^{1 + d/2} l_{m+1}}{|\eta|^{d/2}}\\
        &\ \ \ \ \ \ \ \leq \frac{1}{10 \cdot 2^d \cdot |\eta|^{d/2}}.
    \end{split}
    \end{equation}
    %
    Conversely, if $|\xi| \geq 2|\eta|$, then $|\eta - \xi| \geq |\xi|/2$, so for each $t > d$,
    %
    \begin{equation} \label{equation5551092491022109}
    \begin{split}
        \int_{|\xi| \geq 2|\eta|} |\widehat{\mu_T}(\eta - \xi)| |\widehat{\psi}_{m+1}(\xi)|\; d\xi &\leq \int_{|\xi| \geq 2|\eta|} \frac{A(t) 2^t}{|\xi|^t}\\
        &\leq 2^d \int_{2|\eta|}^\infty r^{d-1 - t} A(t) 2^t\\
        &\leq \frac{4^d A(t)}{t - d} |\eta|^{d-t}.
    \end{split}
    \end{equation}
    %
    Set $t = 3d/2 + 1$. Equation \eqref{equation190512919204912901}, applied to \eqref{equation5551092491022109}, allows us to conclude
    %
    \begin{equation} \label{equation123123123}
        \int_{|\xi| \geq 2|\eta|} |\widehat{\mu_T}(\eta - \xi)| |\widehat{\psi}_{m+1}(\xi)|\; d\xi \leq \frac{1}{10 \cdot 2^d \cdot |\eta|^{s/2}}.
    \end{equation}
    %
    Finally, if $t > 0$, we use the fact that $\| \widehat{\mu_T} \|_{L^\infty(\RR^d)} \leq 1$ to conclude that
    %
    \begin{equation} \label{equation59010491092}
    \begin{split}
        \int_{|\eta|/2 \leq |\xi| \leq 2|\eta|} |\widehat{\mu_T}(\eta - \xi)| |\widehat{\psi_{m+1}}(\xi)|\; d\xi &\leq \frac{2^{d+t} B(t)}{|\eta|^{t-d}}.
    \end{split}
    \end{equation}
    %
    Set $t = 3d/2 + 1$. Then \eqref{equation59010491092} and \eqref{equation68943893493849} imply
    %
    \begin{equation} \label{equation2194129}
        \int_{|\eta|/2 \leq |\xi| \leq 2|\eta} |\widehat{\mu_T}(\eta - \xi)| |\widehat{\psi_{m+1}}(\xi)|\; d\xi \leq \frac{1}{10 \cdot 2^d \cdot |\eta|^{d/2}}.
    \end{equation}
    %
    It then suffices to sum up \eqref{equation1111902491209012}, \eqref{equation123123123}, and \eqref{equation2194129}, and apply \eqref{equation9942941924912912}.
\end{proof}

\begin{proof}[Proof of Proposition \ref{discreteLemma}, Continued]
    Let us now put all our calculations together. In light of Lemma \ref{propertyALemma} and Lemma \ref{deviationLemma}, there exists some choice of $j_\sigma$ for each $\sigma$, and a resultant non-random pair $(\mu_S, S)$ such that $S$ satisfies Property (A) of the Lemma, and $\mu_S$ satisfies \eqref{equation12901904192090129102}, implying that $\mu_S$ satisfies Property (C) of the Lemma. But Lemma \ref{nuNormalizationLemma} shows that $\mu_S$ always satisfies Property (B), and Lemma \eqref{largeFrequencyLemma} shows Property (D) is also always satisfied. This completes the proof.
    \begin{comment}

    %
    Now
    %
    \begin{align*}
        r_{m+1}^{d/2} \log(M_{m+1}) &= \left( l_{m+1}^{a\varepsilon - \frac{dn-s}{2n}} \cdot r_{m+1}^{d/2} \log(M_{m+1}) \right) l_{m+1}^{\frac{dn - s}{2n} - a \varepsilon}.
    \end{align*}
    %
    Equation \eqref{equation5890129048128941891} implies
    %
    \begin{align*}
        &l_{m+1}^{a\varepsilon - \frac{dn-s}{2n}} \cdot r_{m+1}^{d/2} \log(M_{m+1})\\
        &\ \ \ \ \ = \frac{l_m^{a\varepsilon - \frac{dn-s}{2n} + d/2} \log(M_{m+1}) K_{m+1}^{\frac{dn-s}{2n} - a\varepsilon}}{M_{m+1}^{a\varepsilon + \frac{s}{2n}}}\\
        &\ \ \ \ \ \leq \left[ l_m^{a\varepsilon - \frac{dn-s}{2n} + d/2} 2^{\frac{dn-s}{2n} - a \varepsilon} \right] \log(M_{m+1}) M_{m+1}^{\left( \frac{s}{dn-s} + c\varepsilon \right)\left(\frac{dn-s}{2n} - a\varepsilon\right) - a\varepsilon - \frac{s}{2n}}.
    \end{align*}
    %
    Now
    %
    \begin{align*}
        \left( \frac{s}{dn - s} + c\varepsilon \right) \left( \frac{dn - s}{2n} - a\varepsilon \right) - a\varepsilon &\leq \left[ \frac{(dn-s)c}{2n} - \left( \frac{s}{dn-s}+1 \right)a \right] \varepsilon\\
        &= \left[ \frac{d(3 - na)}{(dn - s)} \right] \varepsilon\\
        &\leq -2 \varepsilon.
    \end{align*}
    %
    Since $\log(M_{m+1}) \leq (2/\varepsilon) M_{m+1}^{\varepsilon/2}$, if we assume that
    %
    \begin{equation} \label{equation1290412904129049102}
        M_{m+1} \geq \left( l_m^{a\varepsilon - \frac{dn-s}{2n} + d/2} 2^{\frac{dn-s}{2n} - a\varepsilon} (2/\varepsilon) \right)^{2/\varepsilon},
    \end{equation}
    %
    then we conclude
    %
    \begin{align*}
        r_{m+1}^{d/2} \log(M_{m+1}) &\leq \left[ l_m^{a\varepsilon - \frac{dn-s}{2n} + d/2} 2^{\frac{dn-s}{2n} - a \varepsilon} \log(M_{m+1}) M_{m+1}^{-\varepsilon} \right] M_{m+1}^{-\varepsilon} l_{m+1}^{\frac{dn-s}{2n} - a\varepsilon}\\
        &\leq M_{m+1}^{-\varepsilon} l_{m+1}^{\frac{dn-s}{2n} - a \varepsilon}
    \end{align*}
    %
    Applying Lemma \ref{deviationLemma}, we conclude that if $|k| \leq 10 l_{m+1}^{-1}$,
    %
    \begin{align*}
        |\widehat{\nu_{S}}(k)| &\leq |\widehat{\nu_{S}}(k) - \widehat{\mu_T}(k)| + |\widehat{\mu_T}(k)|\\
        &\leq r_{m+1}^{d/2} \log(M_{m+1}) + L |k|^{c\varepsilon - \frac{dn-s}{2n}}\\
        &\leq l_{m+1}^{-\frac{dn-s}{2n}-c\varepsilon + \varepsilon} + L |k|^{c\varepsilon - \frac{dn-s}{2n}}\\
        &\leq \left[ L + 10^d M_{m+1}^{-\varepsilon} \right] |k|^{c\varepsilon - \frac{dn-s}{2n}}.
    \end{align*}
    %
    Applying Lemma \ref{largeFrequencyLemma} with $r = 2c\varepsilon - (dn - s)/n$ implies that for $|k| \geq 10l_{m+1}^{-1}$,
    %
    \[ |\widehat{\nu_{S}}(k)| \leq L |k|^{c\varepsilon-\frac{dn-s}{2n}} \leq [L + 10^d M_{M+1}^{-\varepsilon}]|k|^{c\varepsilon - \frac{dn-s}{2n}}, \]
    %
    Thus we conclude that for all $k \in \ZZ^d$,
    %
    \[ |\widehat{\nu_S}(k)| \leq (L + 10^d M_{m+1}^{-\varepsilon}) |k|^{c\varepsilon - \frac{dn-s}{2n}}. \]
    %
    Applying Lemma \ref{nuNormalizationLemma}, as well as a valid assumption that
    %
    \begin{equation} \label{equation129301923109}
        M_{m+1} \geq 10^{d/\varepsilon} \cdot 4^{m/\varepsilon},
    \end{equation}
    %
    we conclude that for all $k \in \ZZ^d$,
    %
    \begin{align*}
        |\widehat{\mu_S}(k)| &\leq \frac{L + 10^d M_{m+1}^{-\varepsilon}}{1 - M_{m+1}^{-1/2}} |m|^{c\varepsilon - \frac{dn-s}{2n}}\\
        &\leq (1 + M_{m+1}^{-1/2}) [L + 10^d M_{m+1}^{-\varepsilon}] |k|^{c\varepsilon - \frac{dn-s}{2n}}\\
        &\leq (1 + 1/2^m) [L + 10^d M_{m+1}^{-\varepsilon}] |k|^{c\varepsilon - \frac{dn-s}{2n}}\\
        &\leq (1 + 1/2^m) [L + 1/2^m] |k|^{c\varepsilon - \frac{dn-s}{2n}}. \qedhere
    \end{align*}
    \end{comment}
\end{proof}

\section{Construction of the Salem Set}

Let us now choose the parameters to construct our configuration avoiding set. First, we fix some preliminary parameters. Write $Z \subset \bigcup_{i = 1}^\infty Z_i$, where $Z_i$ has lower Minkowski dimension at most $s$ for each $i$. Then choose an infinite sequence $\{ i_m : m \geq 1 \}$ which repeats every positive integer infinitely many times. Also, choose an arbitrary, decreasing sequence of positive numbers $\{ \varepsilon_m : m \geq 1 \}$, with $\varepsilon_m < (dn - s)/2$ for each $m$. We choose our parameters $\{ M_m \}$ and $\{ K_k \}$ inductively. First, set $X_0 = [0,1]^d$, and $\mu_0$ an arbitrary smooth probability measure supported on $X_0$. At the $m$th step of our construction, we have already found a set $X_{m-1}$ and a measure $\mu_{m-1}$. We then choose $K_m$ and $M_m$ such that
%
\[ K_m, M_m \geq C(\mu_{m-1}, n, d, s, \varepsilon_m), \]
%
such that
%
\[ M_m^{\frac{s}{dn-s} + c\varepsilon_m} \leq K_m \leq 2 M_m^{\frac{s}{dn-s} + c\varepsilon}, \]
%
and such that the set $Z_{i_m}$ is covered by at most $l_m^{-(s + \varepsilon_m)}$ cubes in $\DQ_m$, the union of which, we define to be equal to $B_m$. We can then apply Proposition \ref{discreteLemma} with $\varepsilon = \varepsilon_m$, $T = X_{m-1}$, $\mu_T = \mu_{m-1}$, and $B = B_m$. This produces a $\DQ_{m+1}$ discretized set $S \subset T$, and a measure $\mu_S$ supported on $S$. We define $X_m = S$, and $\mu_m = \mu_S$.

The last paragraph recursively generates an infinite sequence $\{ X_m \}$. We set $X = \bigcap X_m$. Just as in our previous paper, it is easy to see $X$ must be a configuration avoiding set. Given any $(x_1, \dots, x_n) \in Z$, there are infinitely many integers $m_k$ such that $(x_1, \dots, x_n) \in B_{m_k}$. If $|x_i - x_j| \geq \varepsilon$ for each $i \neq j$, and $r_{m_k} \leq \varepsilon/2$, then $(x_1, \dots, x_n)$ is contained in a strongly non-diagonal cube in $\DQ_{m_k}(B_k)$, and as such $X^n \subset X_k^n$ does not contain $(x_1, \dots, x_n)$.

\section{Proof that $X$ is Salem}

We now show $X$ is Salem, completing the proof of Theorem \ref{maintheorem}. Since the masses of the sequence of measures $\{ \mu_m \}$ is uniformly bounded, there is some subsequence $\mu_{m_i}$ which converges weakly to some measure $\mu$. Repeated applications of Property (B) of Proposition \ref{discreteLemma} imply
%
\[ \mu(\RR^d) = \lim_{i \to \infty} \mu_{m_i}(\RR^d) \geq 1 - \sum_{m=1}^\infty M_m^{-1/2}. \]
%
In particular, $\mu$ is a non-zero measure if the sequence $\{ M_m \}$ is rapidly increasing. Moreover, for each $k \in \ZZ^d$,
%
\[ \widehat{\mu}(k) = \lim_{i \to \infty} \widehat{\mu_{m_i}}(k). \]
%
Thus
%
\[ |\widehat{\mu}(k)| \leq |\widehat{\mu_0}(k)| + \sum_{m = 0}^\infty |\widehat{\mu_{m+1}}(k) - \widehat{\mu_m}(k)|. \]
%
Fix $\varepsilon > 0$. Since $l_m \leq 2^{-m}/10$, we find that for $m \geq \log(k)$, $|k| \leq 10 l_{m+1}^{-1}$. Thus we can apply Property (C) and (D) of Proposition \ref{discreteLemma} to conclude
%
\begin{align*}
    &\sum_{m = 0}^\infty |\widehat{\mu_{m+1}}(k) - \widehat{\mu_m}(k)|\\
    &\ \ \ \ \ \leq 2 \log(k) |k|^{-d/2} + \sum_{m = \log(k)}^\infty r_{m+1}^{d/2} \log(M_{m+1})\\
    &\ \ \ \ \ \lesssim_\varepsilon |k|^{\varepsilon- t/2} \left( 1 + \sum_{m = \log(k)}^\infty |k|^{t/2-\varepsilon} r_{m+1}^{d/2} \log(M_{m+1}) \right)\\
    &\ \ \ \ \ \leq |k|^{\varepsilon - t/2} \left( 1 + 10^{t/2 - \varepsilon} \sum_{m = \log(k)}^\infty l_{m+1}^{\varepsilon - t/2} r_{m+1}^{d/2} \log(M_{m+1}) \right)\\
    &\ \ \ \ \ \lesssim_\varepsilon |k|^{\varepsilon - t/2} \left( 1 + \sum_{m = \log(k)}^\infty \frac{1}{K_{m+1}^{\varepsilon}} \frac{K_{m+1}^{t/2}}{M_{m+1}^{d/2 - t/2}} \right)\\
    &\ \ \ \ \ \lesssim |k|^{\varepsilon - t/2} \left( 1 + \sum_{m = \log(k)}^\infty \frac{M_{m+1}^{c\varepsilon_m (t/2)}}{K_{m+1}^\varepsilon} \frac{M_{m+1}^{(t/2) \left( \frac{s}{dn-s} \right)}}{M_{m+1}^{d/2 - t/2}} \right)\\
    &\ \ \ \ \ = |k|^{\varepsilon - t/2} \left( 1 + \sum_{m = \log(k)}^\infty \frac{M_{m+1}^{c\varepsilon_m (t/2)}}{K_{m+1}^{\varepsilon}} \right) \lesssim_\varepsilon |k|^{\varepsilon - t/2}.
\end{align*}
%
The last inequality follows because $\varepsilon_m \to 0$, and so the series is summable if the sequence $\{ K_m \}$ increases rapidly enough. Since $\mu_0$ is smooth and compactly supported, we find
%
\[ \sup_{k \in \ZZ^d} |k|^{t/2 - \varepsilon} |\widehat{\mu}(k)| \lesssim_\varepsilon 1 + \sup_{k \in \ZZ} |k|^{t/2 - \varepsilon} |\widehat{\mu_0}(k)| < \infty. \]
%
Since $\varepsilon > 0$ was arbitrary, this shows that the Fourier dimension of $X$ is at least $t$. Because $X_m$ is the union of $(M_1 \dots M_m)^d$ sidelength $l_m$ cubes, one can easily show using \eqref{equation5890129048128941891} that the lower Minkowski dimension of $X$ is upper bounded by $t$. But these two bounds imply that the Hausdorff dimension, Fourier dimension, and Minkowski dimension are all equal to $t$. Thus $X$ is Salem of dimension $t$.

\begin{comment}
It is easy to see from our proof that for each $m$,
%
\[ \#(\DQ_m(X_m)) = \frac{1}{(M_1 \dots M_m)^d}. \]
%
Thus
%
\begin{align*}
    \log_{1/l_m}(\#(\DQ_m(X_m))) &= \frac{d \log(M_1 \dots M_m)}{\log(N_1 \dots N_m)}\\
    &\sim \frac{d \log(M_m)}{\log(N_m)}\\
    &= \frac{d \log(M_m)}{\log(M_m) + \log(K_m)}\\
    &= \frac{d}{1 + \log_{M_m}(K_m)}\\
    &\sim \frac{d}{1 + \left(\frac{s}{dn - s} + c\varepsilon \right)}\\
    &= \frac{d}{\frac{dn}{dn - s} + c\varepsilon_m}.
\end{align*}
%
Thus
%
\[ \lim_{m \to \infty} \log_{1/l_m}(\#(\DQ_m(X_m))) \leq \frac{dn - s}{n}. \]
%
In particular, this means $X$ has lower Minkowski dimension at most $(dn - s)/n$, which, together with the Fourier dimension bound, implies that $X$ is Salem.
\end{comment}

\begin{comment}
\section{K\"{o}rner's Work}

The last sections gives a first, positive result finding Salem sets avoiding rough configurations. However, as we showed in our last paper, given the same assumptions, one can find a set $X$ with \emph{Hausdorff dimension} $(dn - s)/(n-1)$ avoiding configurations. Improving the dimension to obtain this result requires a deeper knowledge of the stochastic behaviour of the set $\DQ_{m+1}(B) \cap \DQ_{m+1}(S^n)$, when $N_{m+1}$ is chosen significantly smaller relative to $M_{m+1}$. In the next section, we provide a summary of an argument due to K\"{o}rner, which deals with a very similar situation.
\end{comment}

\begin{thebibliography}{9}

\bibitem{Vershynin}
    Roman Vershynin,
    \textit{High Dimensional Probability},
    Cambridge Series in Statistical and Probabilistic Mathematics,
    2018.

\end{thebibliography}

\end{document}