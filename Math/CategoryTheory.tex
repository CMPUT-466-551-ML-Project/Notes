\documentclass[12pt, dvipsnames]{report}

\usepackage{amsmath}
\usepackage{algorithm}
%\usepackage{algorithmic}
\usepackage[noend]{algpseudocode}

\usepackage{amsmath}
\usepackage{amssymb}
\usepackage{amsthm}
\usepackage{amsopn}

\usepackage{kpfonts}

\usepackage{graphicx}

% Probably don't need this on notes anymore
%\usepackage{kbordermatrix}

% Standard tool for drawing diagrams.
\usepackage{tikz}
\usepackage{tkz-berge}
\usepackage{tikz-cd}
\usepackage{tkz-graph}

\usepackage{comment}

%
\usepackage{multicol}

%
\usepackage{framed}

%
\usepackage{mathtools}

%
\usepackage{float}

%
\usepackage{subfig}

%
\usepackage{wrapfig}

%
\let\savewideparen\wideparen
\let\wideparen\relax
\usepackage{mathabx}
\let\wideparen\savewideparen

% Used for generating `enlightening quotes'
\usepackage{epigraph}

% Forget what this is used for :P
\usepackage[utf8]{inputenc}

% Used for generating quotes.
\usepackage{csquotes}

% Allows what to generate links inside
% generated pdf files
\usepackage{hyperref}

% Allows one to customize theorem
% environments in mathematical proofs.
\usepackage{thmtools}

% Gives access to a proof
\usepackage{lplfitch}

% I forget what this is for.
\usepackage{accents}

% A package for drawing simple trees,
% as a substitute for unnesacary TIKZ code
\usepackage{qtree}

% Enables sequent calculus proofs
\usepackage{ebproof}

% For braket notation
\usepackage{braket}

% To change line spacing when using mathematical notations which require some height!
\usepackage{setspace}

%\usepackage[dvipsnames]{xcolor}

\usepackage{float}

% For block commenting
\usepackage{comment}




\setlength\epigraphwidth{8cm}

\usetikzlibrary{arrows, petri, topaths, decorations.markings}

% So you can do calculations in coordinate specifications
\usetikzlibrary{calc}
\usetikzlibrary{angles}

\theoremstyle{plain}
\newtheorem{theorem}{Theorem}[chapter]
\newtheorem{axiom}{Axiom}
\newtheorem{lemma}[theorem]{Lemma}
\newtheorem{corollary}[theorem]{Corollary}
\newtheorem{prop}[theorem]{Proposition}
\newtheorem{exercise}{Exercise}[chapter]
\newtheorem{fact}{Fact}[chapter]

\newtheorem*{example}{Example}
\newtheorem*{proof*}{Proof}

\theoremstyle{remark}
\newtheorem*{exposition}{Exposition}
\newtheorem*{remark}{Remark}
\newtheorem*{remarks}{Remarks}

\theoremstyle{definition}
\newtheorem*{defi}{Definition}

\usepackage{hyperref}
\hypersetup{
    colorlinks = true,
    linkcolor = black,
}

\usepackage{textgreek}

\makeatletter
\renewcommand*\env@matrix[1][*\c@MaxMatrixCols c]{%
  \hskip -\arraycolsep
  \let\@ifnextchar\new@ifnextchar
  \array{#1}}
\makeatother

\renewcommand*\contentsname{\hfill Table Of Contents \hfill}

\newcommand{\optionalsection}[1]{\section[* #1]{(Important) #1}}
\newcommand{\deriv}[3]{\left. \frac{\partial #1}{\partial #2} \right|_{#3}} % partial derivative involving numerator and denominator.
\newcommand{\lcm}{\operatorname{lcm}}
\newcommand{\im}{\operatorname{im}}
\newcommand{\bint}{\mathbf{Z}}
\newcommand{\gen}[1]{\langle #1 \rangle}

\newcommand{\End}{\operatorname{End}}
\newcommand{\Mor}{\operatorname{Mor}}
\newcommand{\Id}{\operatorname{id}}
\newcommand{\visspace}{\text{\textvisiblespace}}
\newcommand{\Gal}{\text{Gal}}

\newcommand{\xor}{\oplus}
\newcommand{\ft}{\wedge}
\newcommand{\ift}{\vee}

\newcommand{\prob}{\mathbf{P}}
\newcommand{\expect}{\mathbf{E}}
\DeclareMathOperator{\Var}{\mathbf{V}}
\newcommand{\Ber}{\text{Ber}}
\newcommand{\Bin}{\text{Bin}}

%\newcommand{\widecheck}[1]{{#1}^{\ft}}

\DeclareMathOperator{\diam}{\text{diam}}

\DeclareMathOperator{\QQ}{\mathbf{Q}}
\DeclareMathOperator{\ZZ}{\mathbf{Z}}
\DeclareMathOperator{\RR}{\mathbf{R}}
\DeclareMathOperator{\HH}{\mathbf{H}}
\DeclareMathOperator{\CC}{\mathbf{C}}
\DeclareMathOperator{\AB}{\mathbf{A}}
\DeclareMathOperator{\PP}{\mathbf{P}}
\DeclareMathOperator{\MM}{\mathbf{M}}
\DeclareMathOperator{\VV}{\mathbf{V}}
\DeclareMathOperator{\TT}{\mathbf{T}}
\DeclareMathOperator{\LL}{\mathcal{L}}
\DeclareMathOperator{\EE}{\mathbf{E}}
\DeclareMathOperator{\NN}{\mathbf{N}}
\DeclareMathOperator{\DQ}{\mathcal{Q}}
\DeclareMathOperator{\IA}{\mathfrak{a}}
\DeclareMathOperator{\IB}{\mathfrak{b}}
\DeclareMathOperator{\IC}{\mathfrak{c}}
\DeclareMathOperator{\IP}{\mathfrak{p}}
\DeclareMathOperator{\IQ}{\mathfrak{q}}
\DeclareMathOperator{\IM}{\mathfrak{m}}
\DeclareMathOperator{\IN}{\mathfrak{n}}
\DeclareMathOperator{\IK}{\mathfrak{k}}
\DeclareMathOperator{\ord}{\text{ord}}
\DeclareMathOperator{\Ker}{\textsf{Ker}}
\DeclareMathOperator{\Coker}{\textsf{Coker}}
\DeclareMathOperator{\emphcoker}{\emph{coker}}
\DeclareMathOperator{\pp}{\partial}
\DeclareMathOperator{\tr}{\text{tr}}

\DeclareMathOperator{\supp}{\text{supp}}

\DeclareMathOperator{\codim}{\text{codim}}

\DeclareMathOperator{\minkdim}{\dim_{\mathbf{M}}}
\DeclareMathOperator{\hausdim}{\dim_{\mathbf{H}}}
\DeclareMathOperator{\lowminkdim}{\underline{\dim}_{\mathbf{M}}}
\DeclareMathOperator{\upminkdim}{\overline{\dim}_{\mathbf{M}}}
\DeclareMathOperator{\lhdim}{\underline{\dim}_{\mathbf{M}}}
\DeclareMathOperator{\lmbdim}{\underline{\dim}_{\mathbf{MB}}}
\DeclareMathOperator{\packdim}{\text{dim}_{\mathbf{P}}}
\DeclareMathOperator{\fordim}{\dim_{\mathbf{F}}}

\DeclareMathOperator*{\argmax}{arg\,max}
\DeclareMathOperator*{\argmin}{arg\,min}

\DeclareMathOperator{\ssm}{\smallsetminus}

\title{Category Theory}
\author{Jacob Denson}

\begin{document}

\pagenumbering{gobble}
\maketitle
\tableofcontents

\chapter{Basic Definitions}

\pagenumbering{arabic}

Category Theory is the language of transformations. A great many objects share some common formal behaviour, which we use terminology to describe. Rather than looking at particular elements of particular groups, rings, and sets, we just look at the algebraic objects themselves, and describe the functions connecting them.

A {\bf category} $\mathcal{C}$ consists of objects $\text{Obj}(\mathcal{C})$, such that for each pair of objects $(A,B) \in \text{Obj}(\mathcal{C})$ we have a collection of morphisms $\text{Mor}(A,B)$, which are pairwise disjoint, together with a composition operation. We write $f \in \text{Mor}(A,B)$ as $f:A \to B$. For each triplet $A,B,C$ of objects, we have an associative composition map
%
\[ \circ: \text{Mor}(B,C) \times \text{Mor}(A,B) \to \text{Mor}(A,C) \]
%
For any object $A$, we have a morphism $\text{id}_A \in \text{Mor}(A,A)$, such that $\text{id} \circ f = f$ for any $f: B \to A$, and $g \circ \text{id}_A = g$ for any $g: A \to B$. Morphisms in $\text{Mor}(A,A)$ will be known as {\bf endomorphisms}, and the set of such objects will be abbreviated to $\text{End}(A)$. With the operation of composition, $\text{End}(A)$ becomes a monoid, and all monoids can be realized as endomorphisms over some object in a particular category. An {\bf operation} of a monoid $M$ on an object $A$ is a monoid homomorphism $f: M \to \text{End}(A)$.

\begin{example}
    Perhaps the most basic category is the category $\mathbf{Set}$ of sets, whose objects are sets, and whose morphisms are set-theoretic functions. Category theory can be seen as a generalization of this category, and most often categories will be seen as a subcategory of this category.
\end{example}

\begin{example}
    Algebra makes extensive use of category theory. The category $\mathbf{Grp}$ of groups has groups as objects, and whose morphisms are group homomorphisms. One similarily defines the categories $\mathbf{Rng}$ and $\mathbf{Vect}$ of rings and vectors, with ring homomorphisms and linear transformations as morphisms.
\end{example}

\begin{example}
    Category theory is also useful in analysis. Let $\mathbf{Top}$ be the category of topological spaces, whose morphisms are continuous maps. One may specialize to the category $\mathbf{Man}$ of manifolds, or even further to $\mathbf{Man}^\infty$, which consists of differentiable manifolds with differentiable maps as morphisms.
\end{example}

An {\bf isomorphism} in a category is a morphism $f:A \to B$ for which there is $g: B \to A$ such that
%
\[ g \circ f = \text{id}_A\ \ \ \ \ \ \ \ f \circ g = \text{id}_B \]
%
It follows trivially that $g$ is unique, for if $h$ is another inverse, then
%
\[ g = g \circ \text{id}_B = g \circ f \circ h = \text{id}_A \circ h = h \]
%
we denote $g$ by $f^{-1}$. Examples of isomorphism are algebraic isomorphisms, bijective maps, homeomorphisms, and diffeomorphisms, all in one concept. The set of isomorphisms from an object $A$ to itself will be denoted $\text{Aut}(A)$. It is a group, and all groups are isomorphic to automorphisms over some object in a category. By an {\bf operation} or {\bf representation} of a group $G$ on an object $A$ in a category we mean a group homomorphism $f: G \to \text{Aut}(A)$.

Isomorphisms really are `the same object' in a categorical sense, because there are natural bijections between the morphisms of the object. Let $f: X \to Y$ be an isomorphism. The map $g \mapsto f \circ g \circ f^{-1}$ is then a bijection between $\text{Aut}(X)$ and $\text{Aut}(Y)$. Similarily, $g \mapsto g \circ f$ is a bijection between $\text{Mor}(Y,A)$ and $\text{Mor}(X,A)$.

\begin{example}
    If $M$ is a differentiable manifold, and $X$ a differentiable vector field, then $X$ induces a representation of the additive group $\mathbf{R}$ on the diffeomorphisms of $M$, obtained by taking $t$ to the map which perturbs a point to where it would be in $t$ seconds after the vector field acts on the space.
\end{example}

Other useful maps to describe are {\bf sections}, maps $f: X \to Y$ which are {\it left invertible}, such that there are maps $g: Y \to X$ such that $g \circ f = \text{id}_A$. The {\it right invertible} elements are called {\bf retractions}.

\section{Universal Objects}

Let $\mathcal{C}$ be a category. An {\bf initial object} (or a universal repeller) $X$ in the category is an object such that for any other object $A$, there is a unique map $f: X \to A$. A {\bf final object} (or universal attractor) has unique maps $f: A \to X$. It is easy to see that any two initial and final objects in the same category are isomorphic. Here are some examples,
%
\begin{example}
    The trivial group $(0)$ is both initial and final in the category of groups. Similarily, the trivial module is initial and final in the category of modules over a fixed ring.
\end{example}

\begin{example}
    The empty set is an initial object in the category of sets, and a singleton is a final object in this category. The same is true in the category of topological spaces and differentiable manifolds.
\end{example}

We shall describe other {\bf universal objects} in this section. It is difficult to describe precisely what these objects are, but they are constructed in a manner such that they are initial or final in some category related to another category.

\begin{example}
    As a pedagogical example, let us show that a universal object is reducible to an initial object. Given a category $\mathcal{C}$, consider the category $\mathcal{C}^{\text{rev}}$, with the same objects, but if $f: A \to B$ is a morphism in $\mathcal{C}$, then $f^{\text{rev}}: B \to A$ is a morphism in $\mathcal{C}^{\text{rev}}$. An initial object in $\mathcal{C}^{\text{rev}}$ is simply a final object in $\mathcal{C}$. Suppose we only know that initial objects are isomorphic. Let $A$ and $B$ be final objects in $\mathcal{C}$. Then $A$ and $B$ are final in $\mathcal{C}^{\text{rev}}$, so there is an isomorphism $f^{\text{rev}}: B \to A$, which converts back to an isomorphism $f: A \to B$.
\end{example}

We shall make common use of a certain type of construction. Given any category $\mathcal{C}$, we may form a new category $\mathcal{C}^\to$, whose objects consist of the morphisms in $\mathcal{C}$, and such that a morphism between two morphisms $f: A \to B$ and $g: A' \to B'$ is a pair of maps $\phi: A \to A'$, $\psi: B \to B'$ such that
%
\begin{center}
\begin{tikzcd}
    A \arrow{r}{f} \arrow{d}{\phi} & B \arrow{d}{\psi}\\
    A' \arrow{r}{g} & B'
\end{tikzcd}
\end{center}
%
commutes. There are many variations to this category. For instance, if we fix an object $X$, and consider the category of morphisms $f:A \to X$, whose morphisms are just as morphisms in $\mathcal{C}^\to$, but forcing $\psi$ to be the identity, so that the diagram
%
\begin{center}
\begin{tikzcd}
    A \arrow{rr}{\phi} \arrow{rd}[below]{f} & & B \arrow{ld}{g}\\
    & X &
\end{tikzcd}
\end{center}
%
commutes.

Given two objects $A$ and $B$, we will construct a {\bf product} object $A \times B$. Consider the category whose objects consist of triplets $(X,f,g)$, where $f: X \to A$, and $g: X \to B$ are morphisms. A morphism between $(X, f,g)$ and $(Y, \rho, \pi)$ is a morphism $h: X \to Y$, such that the diagram
%
\begin{center}
\begin{tikzcd}
    & X \arrow{d}{h} \arrow{ld}[above]{f} \arrow{rd}{g} & \\
    A & Y \arrow{l}{\rho} \arrow{r}{\pi} & B
\end{tikzcd}
\end{center}
%
commutes. A product for $A$ and $B$ is then a final object $(A \times B, \pi_A, \pi_B)$ in this category. It is clear that any products are not only isomorphic in the original category, but also isomorphic in the stronger sense in the category above. What we have argued is that if $f: X \to A$ and $g: X \to B$ are any two morphisms, then there exists a morphism $f \times g: X \to A \times B$, such that the diagram
%
\begin{center}
\begin{tikzcd}
    & X \arrow{d}{f \times g} \arrow{ld}[above]{f} \arrow{rd}{g} & \\
    A & A \times B \arrow{l}{\pi_A} \arrow{r}[below]{\pi_B} & B
\end{tikzcd}
\end{center}
%
commutes. One define products of arbitrary families $\{ A_\alpha \}$ of objects, and denotes the final objects as $(\prod A_\alpha, \{ \pi_\alpha \})$.

\begin{example}
    Given two groups $G$ and $H$, one canonically defines the product $G \times H$ to be the set of all tuples $(g,h)$, with $g \in G$ and $h \in H$, and with multiplication structure $(g,h)(x,y) = (gx,hy)$. The same trick works for products of rings, modules, vector spaces, and sets, where the associated operations are adjusted accordingly.
\end{example}

{\bf Coproducts} are obtained from the above definition by reversing the arrows. We consider the category of objects $(X,f,g)$, where $f: A \to X$ and $g: B \to X$ are morphisms. An initial object in this category is the coproduct of $A$ and $B$, denoted $A \coprod B$. Given $f: A \to X$ and $g: B \to X$, we have a unique map $f \coprod g: A \coprod B \to X$, such that
%
\begin{center}
\begin{tikzcd}
    & X  \arrow{ld}[above]{f} \arrow{rd}{g} & \\
    A \arrow{rd}[below]{i_A} & & B \arrow{ld}{i_B} \\
    & A \coprod B \arrow{uu}{f \coprod g} &&
\end{tikzcd}
\end{center}
%
We may also consider coproducts $\coprod A_\alpha$ of an arbitrary family $\{ A_\alpha \}$ of objects.

\begin{example}
    Given two groups, $G$ and $H$, the coproduct is the free product $G * H$, which is a quotient of the monoid of all finite words with elements in $G$ and $H$ (assumed disjoint) whose operation is concatenation. Consider the equivalence which identifies $(g,g')$ with $g * g'$, and $h * h'$ with $hh'$, and if $e$ is the identity in $G$, and $e'$ the identity in $H'$, then identify $e * h$ and $h * e$ with $h$, and $e' * g$ and $g * e'$ with $g$. Extend this to semigroup congruence. The monoid formed is the free product, and is a group, for $G * H$ is generated by $G \cup H$, and each $g \in G$ and $h \in H$ has an inverse in $G * H$. We have canonical embeddings $i_G: G \to G * H$ mapping $g$ to itself, and $i_H: H \to G * H$ mapping $h$ to itself.
\end{example}

\begin{example}
    Given two modules $M$ and $N$ over an abelian ring $R$, the coproduct is the direct sum $M \oplus N$, which is the set $M \times N$ (where $(m,n)$ is denoted $m \oplus n$) with operations $(m \oplus n) + (x \oplus y) = (m + x) \oplus (n + y)$.
\end{example}

Products and Coproducts are the most basic constructions in category theory, but we have some other occasionally useful. Fix an object $Z$ in a category, and consider the category whose objects are morphisms $f: A \to Z$ and $g: B \to Z$, and a morphism between morphisms is a map $h: A \to B$ such that the standard diagram commutes. A product of morphisms in this category is called the {\bf fibre product}, and satisfies the diagram
%
\begin{center}
\begin{tikzcd}
    & A \times_Z B \arrow{dd}{f \times_Z g} \arrow{ld}{\pi_f} \arrow{rd}{\pi_g} & \\
    A \arrow{rd}{f} & & B \arrow{ld}{g} \\
    & Z &
\end{tikzcd}
\end{center}
%
$\pi_f$ is known as the pullback of $f$ by $g$, and $\pi_g$ the pullback of $g$ by $f$. Similarily, we may consider {\bf fiber coproducts}, or {\bf pushouts}.

\begin{example}
    Fibre products exist in the category of groups. Let $f: G \to K$ and $g: H \to K$ be two maps. Let $G \times_K H = \{ (x,y) \in G \times H : f(x) = g(y) \}$, and let $\pi_f$ and $\pi_g$ be the standard projections, then define
    %
    \[ f \times_K g = f \circ \pi_f = g \circ \pi_g \]
    %
    Let $\pi: L \to G$, $\rho: L \to H$, and $\psi: L \to K$ be three maps such that
    %
    \[ f \circ \pi = g \circ \rho = \psi \]
    %
    Then we may consider $\pi \times \rho: L \to G \times H$, and the image of $\pi \times \rho$ is contained in $G \times_K H$, for $f(\pi(x)) = g(\rho(x))$, hence we may consider $\pi \times_K \rho: L \to G \times_K H$, obtained by restricting the domain. This map is unique, for the product map is unique. The normal product $G \times H$ does not satisfy the property of $G \times_K H$, because there may not be maps $\pi_f$ and $\pi_g$ making the diagram commute.
\end{example}

\section{Functors and Natural Transformations}

The main reason to rigorously define groups is to define what a homomorphism is, so we can consider groups with similar structure. Categories were invented to define Functors and natural transformations.

A {\bf (Covariant) Functor} $F$ between two categories $\mathcal{C}$ and $\mathcal{D}$ is an association of an object $X$ in $\mathcal{C}$ with an object $F(X)$ in $\mathcal{D}$, and associating a morphism $f: X \to Y$ with a morphism $F(f): F(X) \to F(Y)$, such that
%
\[ F(g \circ f) = F(g) \circ F(f) \]
%
whenever $g \circ f$ is defined. A {\bf Contravariant Functor} associative a morphism $f: X \to Y$ with a morphism $F(f): F(Y) \to F(X)$ such that
%
\[ F(g \circ f) = F(f) \circ F(g) \]
%
Often $F(f)$ is denoted $f_*$ in the case of a covariant functor, and $f^*$ for a contravariant functor. Functors are the natural morphisms in forming the category of categories, denoted $\mathbf{Cat}$.

\begin{example}
    The association of a set $S$ with the free abelian group $\mathbf{Z}\langle S \rangle$ is a covariant functor from $\mathbf{Set}$ to $\mathbf{Ab}$, since a map $f: S \to T$ induces a map
    %
    \[ f_*: \mathbf{Z} \langle S \rangle \to \mathbf{Z} \langle T \rangle \]
    %
    defined by $f_*(\sum_{s \in S} n_i s) = \sum_{s \in S} n_i f(s)$.
\end{example}

\begin{example}
    Functors most naturally occur in algebraic topology, which studies associations of topological space in $\mathbf{Top}$ with some algebraic structure, for instance, in the categories $\mathbf{Grp}$, $\mathbf{Ab}$, and $\mathbf{Rng}$. Functors were invented to discuss these associations. For instance, the homology is a covariant functor, associating each topological space $X$ a graded $\mathbf{Z}$-module $H(X)$.
\end{example}

Natural transformations are the natural maps relating functors. Given two functors $F$ and $G$ between two categories $\mathcal{C}$ and $\mathcal{D}$, a natural transformation is an association with each object $X \in \mathcal{C}$ a morphism $H(X): F(X) \to G(X)$, such that for each morphism $f: X \to Y$ in $\mathcal{C}$, the diagram
%
\begin{center}
\begin{tikzcd}
    F(X) \arrow{r}{F(f)} \arrow{d}[left]{H(X)} & F(Y) \arrow{d}{H(Y)}\\
    G(X) \arrow{r}{G(f)} & G(Y)
\end{tikzcd}
\end{center}
%
commutes. We may therefore consider isomorphisms of functors, known as {\bf natural equivalences}. We will often say a functor itself is natural if it is naturally equivalent 

\begin{example}
    The classic example of a natural transformation says that a finite dimensional vector space $V$ it `naturally isomorphic' to its double dual $V^{**}$. What we mean is that the endofunctor which associates $V$ with $V^{**}$, and associates $f: V \to W$ with $f^{**}: V^{**} \to W^{**}$ defined by
    %
    \[ f^{**}(\phi) = \phi \circ f^* \]
    %
    is naturally equivalent to the identity functor. Given a vector space $V$, consider the `double dual' map $(\cdot)^{**}: V \to V^{**}$, which maps $v \in V$ to $v^{**}: V^* \to \mathbf{F}$, defined by $v^{**}(f) = f(v)$. We claim this is a natural transformation. Fix $f: V \to W$. Then, for each $v \in V$, and $\phi \in W^*$,
    %
    \[ f(v)^{**}(\phi) = \phi(f(v)) = (\phi \circ f)(v) = f^*(\phi)(v) = (v^{**} \circ f^*)(\phi) = f^{**}(v^{**})(\phi) \]
    %
    The natural transformation has an inverse if we restrict ourselves to finite dimensional vector spaces, since then the double dual is invertible.

    One can also verify that the association of $V$ with $V^*$ is unnatural, for there is no `natural' morphism between $V$ and $V^*$ which does not depend on a choice of basis. However, if we restrict ourselves to the category $\mathbf{Hil}$ of Hilbert spaces, vector spaces with a fixed inner product, then $V$ is naturally isomorphic to $V^*$, and we do not even need to restrict ourselves to finite dimensional spaces, provided we interpret $V^*$ as the continuous dual space.
\end{example}



\chapter{Abelian Categories}

\end{document}