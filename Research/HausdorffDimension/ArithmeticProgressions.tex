\documentclass{article}

\usepackage{amsthm}
\usepackage{amsmath}
\usepackage{multicol}
\usepackage{MnSymbol}

\theoremstyle{plain}
\newtheorem{lemma}{Lemma}
\newtheorem{prop}{Proposition}
\newtheorem*{example}{Example}
\newtheorem*{fact}{Fact}
\newtheorem*{corollary}{Corollary}

\usepackage{algorithm}
\usepackage[noend]{algpseudocode}

\theoremstyle{plain}
\newtheorem{theorem}{Theorem}
\newtheorem{proposition}[theorem]{Proposition}

\title{Squarefree Sets}
\author{Jacob Denson}

\begin{document}

\maketitle

\tableofcontents

\section{Dimensions of Sets Uniformly Avoiding Arithmetic Progressions}

A $k$ length arithmetic progression with gap length $\Delta$ is a sequence of the form
%
\[ \{ a, a + \Delta, a + 2 \Delta, \dots, a + (k-1) \Delta \} \]
%
It is of interest to analyze the dimension of subsets of $\mathbf{R}$ avoiding $k$ length arithmetic progressions, for some $k \geq 3$ (the case $k = 2$ doesn't really make sense). However, such discrete subsets are very difficult to analyze the dimension of. More importantly, Keleti has constructed subsets of the real line with full Hausdorff dimension not containing any $k$ length arithmetic progressions. In this paper, we upper bound the dimension of subsets containing a `wider' family of sequences than arithmetic progressions.

If we consider the arithmetic progression above, then an {\it almost arithmetic progression} with error $\varepsilon > 0$, length $k$ and gap length $\Delta$ is a sequence of the form $b_0, b_1, \dots, b_{k-1}$, which uniformly approximates a $k$ length arithmetic progression, so that $|a + k \Delta - b_k| < \varepsilon \Delta$. We will often abbreviate this discussion by saying a sequence is a $(k,\varepsilon)$ progression The advantage of an almost arithmetic progression is that it restricts our sets from occuring in intervals, rather than points, which leads to a decrease in the Hausdorff dimension of sets.

\begin{theorem}
    If $X \subset \mathbf{R}$ contains no $(k,\varepsilon)$ almost arithmetic progressions, for $\varepsilon \in (0,1)$, then
    %
    \[ \dim_A(X) \leq 1 - \frac{\log(\frac{k}{k-1})}{\log k \lceil 1/2\varepsilon \rceil} \]
    %
    where $\dim_A(X)$ is the {\it Assoud dimension} of $X$, the smallest number such that if $s > \dim_A(X)$, then for radius $R$ and $x \in \mathbf{R}$, the number of balls of radius $r \leq R$ required to cover $B_x(R) \cap X$ is bounded up to a constant by $(R/r)^s$.
\end{theorem}
\begin{proof}
    Consider some interval $I \subset \mathbf{R}$ of length $R$, and fix $r < R$. If $1/2\varepsilon$ was an integer, we could divide $I$ evenly into $k/2\varepsilon$ intervals of length $(2\varepsilon/k) R$. If we partition these intervals into $k$ classes on which the midpoints of the intervals are separated by a multiple of $1/2\varepsilon$, then a given $X$ cannot intersect all the intervals in a given class, because this would give a $(k,\varepsilon)$ progression. We conclude that $X \cap I$ intersects at most
    %
    \[ \frac{k-1}{2 \varepsilon} \]
    %
    intervals of length $(2 \varepsilon/k) R$. Repeating this argument on each of these intervals recursively, we conclude that $X \cap I$ intersects at most
    %
    \[ \left( \frac{k-1}{2 \varepsilon} \right)^m \]
    %
    intervals of length $(2 \varepsilon/k)^m R$. If we choose $(2 \varepsilon/k)^m R \leq r$, then the number of intervals of radius $r$ to cover $I \cap X$ is bounded by $\left( \frac{k-1}{2 \varepsilon} \right)^m$. This occurs, in particular, if
    %
    \[ m = \left\lceil \log_{2\varepsilon/k} r/R \right\rceil \]
    %
    For any $\delta > 0$, we can write
    %
    \[ m = \left\lceil \frac{\log R/r}{\log k/2\varepsilon} \right\rceil \leq (1 + \delta) \frac{\log R/r}{\log k/2\varepsilon} \]
    %
    Provided $R/r$ is sufficiently large, which we may assume. Thus $I \cap X$ is covered by at most
    %
    \[ \left( \frac{k-1}{2 \varepsilon} \right)^{(1 + \delta) \frac{\log R/r}{\log k/2\varepsilon}} = \left(\frac{R}{r}\right)^{(1 + \delta) \frac{\log k-1/2\varepsilon}{\log k/2\varepsilon}} \]
    %
    Since $I$ was arbitrary, it follows that the Assoud dimension of $X$ is less than or equal to
    %
    \[ (1 + \delta) \frac{\log \frac{k-1}{2\varepsilon}}{\log \frac{k}{2\varepsilon}} = (1 + \delta) - \frac{\log k/k-1}{\log k/2\varepsilon} \]
    %
    We then let $\delta \to 0$ to get the required result. If $1/2\varepsilon$ is not an integer, we may simply replace $\varepsilon$ by $\varepsilon' = 1/2\lceil 1/2\varepsilon \rceil$ to get the required result, since any set containing no $(k,\varepsilon)$ arithmetic progressions also contains no $(k,\varepsilon')$ arithmetic progressions since $\varepsilon' \leq \varepsilon$.
\end{proof}

\subsection{Constructing Sets Avoiding Arithmetic Progressions}

Now we have upper bounded the dimension of sets avoiding uniform almost arithmetic progressions, we construct subsets avoiding uniform almost arithmetic progressions with a high enough dimension. We first provide a technical lemma showing that if we remove a wide enough chunk from the middle of an interval, $(k,\varepsilon)$ arithmetic progressions must occur on one side of the chunk or the other.

\begin{lemma}
    Let $I$ be a closed interval of length $|I|$, and $J \subset I$ an open interval of length $|J|$, which is sufficiently wide, so that
    %
    \[ \frac{1 + 2 \varepsilon}{k - 1 - 2\varepsilon} |I| < |J| \]
    %
    Then a $(k,\varepsilon)$ arithmetic progression in $I - J$ must occur either entirely to the left of $J$ or entirely to the right.
\end{lemma}
\begin{proof}
    If $I$ contains a $(k,\varepsilon)$ arithmetic progression with some gap length $\Delta$, then
    %
    \[ |I| \geq (k - 1 - 2\varepsilon) \Delta \]
    %
    On the other hand, if the progression is in $I - J$, and occurs on both sides of $J$, it must `bridge the gap' between $I$ and $J$, so
    %
    \[ \Delta (1 + 2\varepsilon) \geq |J| \]
    %
    But this implies
    %
    \[ |J| \leq \frac{1 + 2 \varepsilon}{k - 1 - 2 \varepsilon} |I| \]
    %
    So the hypothesis of the theorem is contradicted.
\end{proof}

We use this lemma to construct a set avoiding $(k,\varepsilon)$ arithmetic progressions of dimension
%
\[ \frac{\log 2}{\log \frac{2k - 2 - 4\varepsilon}{k - 2 - 4\varepsilon}} \]
%
Suppose $\varepsilon < \min(1, k-2/4)$, and take an increasing sequence $c_m$ which converges to $(k-2-4\varepsilon)/(2k - 2 - 4\varepsilon)$. Set $X_0 = [0,1]$, and then
%
\[ X_{m+1} = c_m X_m \cup (c_m X_m + 1 - c_m) \]
%
Then $\bigcap X_m$ does not contain any $(k,\varepsilon)$ arithmetic progressions, because for each interval $I$ in $X_{m+1}$ we add a hole of length
%
\[ |I|(1 - 2c_m) > |I| \frac{1 + 2 \varepsilon}{k - 1 - 2 \varepsilon} \]

\begin{thebibliography}{9}

\bibitem{RuzsaSetsWithoutSquares}
I. Z. Ruzsa
\textit{Difference Sets Without Squares}

\bibitem{SetsAvoidingUniformProgressions}
Jonathan M. Fraser, Kota Saito, Han Yu
\textit{Dimensions of Sets Which Uniformly Avoid Arithmetic Progressions}

\end{thebibliography}
