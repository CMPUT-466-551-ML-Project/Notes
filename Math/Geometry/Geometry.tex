\documentclass[12pt, dvipsnames]{report}

\usepackage{amsmath}
\usepackage{algorithm}
%\usepackage{algorithmic}
\usepackage[noend]{algpseudocode}

\usepackage{amsmath}
\usepackage{amssymb}
\usepackage{amsthm}
\usepackage{amsopn}

\usepackage{kpfonts}

\usepackage{graphicx}

% Probably don't need this on notes anymore
%\usepackage{kbordermatrix}

% Standard tool for drawing diagrams.
\usepackage{tikz}
\usepackage{tkz-berge}
\usepackage{tikz-cd}
\usepackage{tkz-graph}

\usepackage{comment}

%
\usepackage{multicol}

%
\usepackage{framed}

%
\usepackage{mathtools}

%
\usepackage{float}

%
\usepackage{subfig}

%
\usepackage{wrapfig}

%
\let\savewideparen\wideparen
\let\wideparen\relax
\usepackage{mathabx}
\let\wideparen\savewideparen

% Used for generating `enlightening quotes'
\usepackage{epigraph}

% Forget what this is used for :P
\usepackage[utf8]{inputenc}

% Used for generating quotes.
\usepackage{csquotes}

% Allows what to generate links inside
% generated pdf files
\usepackage{hyperref}

% Allows one to customize theorem
% environments in mathematical proofs.
\usepackage{thmtools}

% Gives access to a proof
\usepackage{lplfitch}

% I forget what this is for.
\usepackage{accents}

% A package for drawing simple trees,
% as a substitute for unnesacary TIKZ code
\usepackage{qtree}

% Enables sequent calculus proofs
\usepackage{ebproof}

% For braket notation
\usepackage{braket}

% To change line spacing when using mathematical notations which require some height!
\usepackage{setspace}

%\usepackage[dvipsnames]{xcolor}

\usepackage{float}

% For block commenting
\usepackage{comment}




\setlength\epigraphwidth{8cm}

\usetikzlibrary{arrows, petri, topaths, decorations.markings}

% So you can do calculations in coordinate specifications
\usetikzlibrary{calc}
\usetikzlibrary{angles}

\theoremstyle{plain}
\newtheorem{theorem}{Theorem}[chapter]
\newtheorem{axiom}{Axiom}
\newtheorem{lemma}[theorem]{Lemma}
\newtheorem{corollary}[theorem]{Corollary}
\newtheorem{prop}[theorem]{Proposition}
\newtheorem{exercise}{Exercise}[chapter]
\newtheorem{fact}{Fact}[chapter]

\newtheorem*{example}{Example}
\newtheorem*{proof*}{Proof}

\theoremstyle{remark}
\newtheorem*{exposition}{Exposition}
\newtheorem*{remark}{Remark}
\newtheorem*{remarks}{Remarks}

\theoremstyle{definition}
\newtheorem*{defi}{Definition}

\usepackage{hyperref}
\hypersetup{
    colorlinks = true,
    linkcolor = black,
}

\usepackage{textgreek}

\makeatletter
\renewcommand*\env@matrix[1][*\c@MaxMatrixCols c]{%
  \hskip -\arraycolsep
  \let\@ifnextchar\new@ifnextchar
  \array{#1}}
\makeatother

\renewcommand*\contentsname{\hfill Table Of Contents \hfill}

\newcommand{\optionalsection}[1]{\section[* #1]{(Important) #1}}
\newcommand{\deriv}[3]{\left. \frac{\partial #1}{\partial #2} \right|_{#3}} % partial derivative involving numerator and denominator.
\newcommand{\lcm}{\operatorname{lcm}}
\newcommand{\im}{\operatorname{im}}
\newcommand{\bint}{\mathbf{Z}}
\newcommand{\gen}[1]{\langle #1 \rangle}

\newcommand{\End}{\operatorname{End}}
\newcommand{\Mor}{\operatorname{Mor}}
\newcommand{\Id}{\operatorname{id}}
\newcommand{\visspace}{\text{\textvisiblespace}}
\newcommand{\Gal}{\text{Gal}}

\newcommand{\xor}{\oplus}
\newcommand{\ft}{\wedge}
\newcommand{\ift}{\vee}

\newcommand{\prob}{\mathbf{P}}
\newcommand{\expect}{\mathbf{E}}
\DeclareMathOperator{\Var}{\mathbf{V}}
\newcommand{\Ber}{\text{Ber}}
\newcommand{\Bin}{\text{Bin}}

%\newcommand{\widecheck}[1]{{#1}^{\ft}}

\DeclareMathOperator{\diam}{\text{diam}}

\DeclareMathOperator{\QQ}{\mathbf{Q}}
\DeclareMathOperator{\ZZ}{\mathbf{Z}}
\DeclareMathOperator{\RR}{\mathbf{R}}
\DeclareMathOperator{\HH}{\mathbf{H}}
\DeclareMathOperator{\CC}{\mathbf{C}}
\DeclareMathOperator{\AB}{\mathbf{A}}
\DeclareMathOperator{\PP}{\mathbf{P}}
\DeclareMathOperator{\MM}{\mathbf{M}}
\DeclareMathOperator{\VV}{\mathbf{V}}
\DeclareMathOperator{\TT}{\mathbf{T}}
\DeclareMathOperator{\LL}{\mathcal{L}}
\DeclareMathOperator{\EE}{\mathbf{E}}
\DeclareMathOperator{\NN}{\mathbf{N}}
\DeclareMathOperator{\DQ}{\mathcal{Q}}
\DeclareMathOperator{\IA}{\mathfrak{a}}
\DeclareMathOperator{\IB}{\mathfrak{b}}
\DeclareMathOperator{\IC}{\mathfrak{c}}
\DeclareMathOperator{\IP}{\mathfrak{p}}
\DeclareMathOperator{\IQ}{\mathfrak{q}}
\DeclareMathOperator{\IM}{\mathfrak{m}}
\DeclareMathOperator{\IN}{\mathfrak{n}}
\DeclareMathOperator{\IK}{\mathfrak{k}}
\DeclareMathOperator{\ord}{\text{ord}}
\DeclareMathOperator{\Ker}{\textsf{Ker}}
\DeclareMathOperator{\Coker}{\textsf{Coker}}
\DeclareMathOperator{\emphcoker}{\emph{coker}}
\DeclareMathOperator{\pp}{\partial}
\DeclareMathOperator{\tr}{\text{tr}}

\DeclareMathOperator{\supp}{\text{supp}}

\DeclareMathOperator{\codim}{\text{codim}}

\DeclareMathOperator{\minkdim}{\dim_{\mathbf{M}}}
\DeclareMathOperator{\hausdim}{\dim_{\mathbf{H}}}
\DeclareMathOperator{\lowminkdim}{\underline{\dim}_{\mathbf{M}}}
\DeclareMathOperator{\upminkdim}{\overline{\dim}_{\mathbf{M}}}
\DeclareMathOperator{\lhdim}{\underline{\dim}_{\mathbf{M}}}
\DeclareMathOperator{\lmbdim}{\underline{\dim}_{\mathbf{MB}}}
\DeclareMathOperator{\packdim}{\text{dim}_{\mathbf{P}}}
\DeclareMathOperator{\fordim}{\dim_{\mathbf{F}}}

\DeclareMathOperator*{\argmax}{arg\,max}
\DeclareMathOperator*{\argmin}{arg\,min}

\DeclareMathOperator{\ssm}{\smallsetminus}

\title{Geometry}
\author{Jacob Denson}

\begin{document}

\pagenumbering{gobble}
\maketitle
\tableofcontents
\pagenumbering{arabic}

\part{Euclid}

I'm writing these notes so that I can understand Euclidean geometry better. We'll build up the axioms from the ground up, so I can understand Euclid's work from the ground up. Thus these notes probably won't be useful for someone trying to understand Euclid themselves, because its just my ramblings about the subject.

\chapter{Book I}

Basic Euclidean geometry consists of three objects: Points, Lines (both finite lines with endpoints, an infinite lines with no extremities), and Circles (defined by a point and a radius). Classically, these objects were seen as distinct, but with the power of set theory, it is easier to model lines and circles as sets of points. This has the advantage of making things notationally simple. These is no real logical difference between switching to this notation -- any theorem provable in one system is provable in the other. However, we'll avoid from using set theory too much, to avoid making the exposition too austere.

Euclid was the first to pioneer the axiomatic method in mathematics. However, the philosophy behind his proofs was different to ours. At the end of the day, his arguments attack a particular model of the planar geometry find in our world, and he proves things like a physicist, adopted some methods of proof not explicitly stated in his assumptions. This causes problems for us when we try and look at his proofs from a modern day perspective. We will eventually look at other logical systems for geometry, but for now a naive approach will be most useful.

Most of Euclid's proofs concern constructions of certain figures in the planes. Rather than a proof of existence, Euclid literally builds these figures from the ground up. In the early parts of the text these figures will all be defined by a simple curve consisting of straight lines, so that we may describe such a figure by the sequence of points which define the figure. If $X_1, \dots, X_n$ are points, then $X_1 \dots X_n$ will denote the figure obtained by drawing the line $X_1 X_2$, then $X_2 X_3$, and so on, finishing off by drawing $X_n X_1$. Two such figures will be considered equal if we may obtain the points of one from the points of the other by performing a cycle permutation of the points. For instance, a {\bf triangle} is just a sequence of distinct points $ABC$, and $ABC = BCA = CAB$, and we can abuse the notation, denoting a line between two points $A$ and $B$ as $AB = BA$. The question of whether this is a unique description of such a line is settled by the first axiom of geometry.

\begin{axiom}
    There is a unique straight line between any pair of points, having those points as endpoints.
\end{axiom}

Euclid does not assume that the straight line which exists between the points is unique, but later he uses the fact that a finite line is defined by its endpoints, so we can only assume that he really wants this fact to hold. In order to discuss the lengths of lines, we shall be required to discuss circles at points, and so we introduce the second axiom.

\begin{axiom}
    A circle may be described with any centre and radius.
\end{axiom}

A circle is {\it defined} by its centre and radius, so the circle which exists by this axiom is unique. Note that circles with a different radii and the same centre may still be equal. Indeed, this happens exactly when the two radii have the same length, a concept we will very shortly discuss.

Euclid defines an {\bf equilateral triangle} as `a triangle whose three sides are equal', which he really means as saying the {\it magnitude}, or length, of the sides are equal. In Euclid's synthetic geometry, there do not exist real numbers to assign length to, and as is well known most Greek's did not even believe in irrational numbers. But we shall find that we can get away with much of the theory of magnitude without ever mentioning the concept of a number, which gives a certain sense of satisfaction.

Right now, we only need equality in the length of lines, and we shall discuss a very aggreeable manner in checking equality. If we have two lines $AB$ and $AC$ with a common point, we can check if they have equal length by checking if the circles constructed with centre $A$ and radii $AB$ and $AC$ are equal. This gives us an equivalence relation on the set of lines extending out from $A$. We shall require that this equivalence relation describes exactly the set of circles with centre $A$, so that a point $C$ lies on the circle with centre $A$ and radius $AB$ if and only if the length of $AC$ is equal to the length of $AB$.

\begin{axiom}
    If $C$ lies on the circle with radius $A$ and radii $AB$, then $AC$ has the same length as $AB$.
\end{axiom}

In order to generalize equality of length of arbitrary lines, we just make the relation transitive. The relation is already reflexive and symmetric, so this generates an equivalence relation on the set of all lines in the plane. Thus we see that the only basic way to check if two lines $AB$ and $CD$ are equal is to form a sequence of lines beginning at $B$, and ending at $C$, which are all equal to one another as lines extending from the same basepoint.

\begin{theorem}
    Any finite line lies on an equilateral triangle.
\end{theorem}
\begin{proof}
    To prove the existence of an equilateral triangle at a line $AB$, Euclid constructs the circle with radius $AB$ and centre $A$, and the circle with centre $B$ and radius $AB$, and considers their point of intersection $C$. Since $C$ lies on the first circle, $AB$ has the same length as $AC$, and since $C$ lies on the second circle, $CB$ has the same length as $AB$. But then the lines $AB, BC$, and $CA$ describe an equilateral triangle, and so $ABC$ is the triangle required.
\end{proof}

There is only one problem remaining in this proof. There is nothing saying that the two circles given will have a common point of intersection. We could describe an axiom which supplies us with such a point, but this axiom would probably be more general than the theorem itself. Indeed, the existence of a point on the intersection of two circles with the same radius but different centres is equivalent to the theorem we set out to prove. Thus we shall have to settle on the fact that theorem one must be treated as an assumption from our current viewpoint.

\begin{theorem}
    Given a point $A$ and line $BC$, to construct a line extending from $A$ with the same length as $BC$.
\end{theorem}
\begin{proof}
    Construct an equilateral triangle $ABD$ on the line $AB$. Then construct the circle with centre $B$ and radius $BC$. Find an intersection point $E$ on the circle which either lies on the line $BD$, or extends the line, and then construct the circle with centre $B$ and radius $BE$. Extend the line $DA$ from the extremity $A$ to an intersection point $F$ on the circle. We claim $AF$ has the same length as $BC$. Indeed, the length of $DF$ is the sum of the length of $DA$ and $AF$, and the length of $DE$ is the sum of $DB$ and $BE$. Since the length of $DF$ is equal to $DE$, since they both lie on the same circle extending from $D$, and the length of $DA$ is equal to the length of $DB$, we may subtract to conclude that the length of $AF$ is the same as the length of $BE$. But $BE$ has the same length of $BC$, which is all that is required to show $AF$ has the same length as $BC$.
\end{proof}

\chapter{Analytic Geometry}

From the perspective of modern axiomatic systems, the axioms of classical Euclidean geometry are a bit of a mess. They not only encompass the theory of lines in planar geometry, but also circles and the interactions between circles and lines. From the perspective of the ancient greeks, this makes sense, because the axioms are meant to describe all the techniques the greeks could interact with in the physical world. The ability to draw circles was provided by a compass; drawing lines was aided by a ruler, which also allowed us to see whether two points were between one another. There was only a single model of the geometry in Euclid's mind -- the universal Euclidean plane $\mathbf{E}^2$ modelling the flat planes he drew his diagrams on. However, when we focus on a single model of the geometry, it is no longer clear what can be {\it proven} from the axioms (Euclid's axioms are not complete, there are many nonstandard Euclidean planes, including the plane of constructible numbers). In order to apply the completeness theorem to determine the limits of the logical axioms, it is important to be able to classify the various models of Euclidean geometry. In this chapter, we weaken the axioms of geometry to obtain a more general class of models. In doing so, we find a beautiful classification theory assuming only a slightly stronger axiom. In particular, we study geometries satisfying three axioms
%
\begin{itemize}
    \item Any two distinct points $X$ and $Y$ lie together on a unique line $XY$.
    \item We say two lines are {\bf parallel} if they are equal to one another, or if they do not intersect. Given a line $l$ and a point $X$, there is a unique line through $X$ parallel to $l$.
    \item There exist three non colinear points in the geometry, and each line has at least two points on it. This axioms is to prevent triviality, such as all points lying on a single line.
\end{itemize}
%
We call the study of models of this axiom set {\bf affine geometry}.

\begin{example}
    Let $K$ be a division ring, and consider the geometry whose points are elements of $K^2$, and whose lines are the affine span $v + Kw = \{ v + xw : x \in K \}$, for a nonzero $w \in K^2$. It is easy to see that every line through a vector $v$ can be written as $v + Kw$, and for two lines $v + Kw_0$ and $v + Kw_1$, either the two lines intersect only at $v$, or the two lines are equal and $w_0$ is a scalar multiple of $w_1$. This tells us that lines intersecting at two or more common points are equal, and therefore there is a unique line $v + K(w-v)$ between any two points $v$ and $w$. A line $v_0 + Kw_0$ is parallel to a line $v_1 + Kw_1$ if and only if $w_0$ is a scalar multiple of $w_1$, and for any point $v_1$ not on a line $v_0 + Kw$, the line $v_1 + Kw$ is a line containing $v_1$ and parallel to $v_0 + Kw$, and this is the unique such line. Finally, the points $(0,1)$, $(1,0)$, and $(0,0)$ are non colinear, so $K^2$ is a model of affine geometry.
\end{example}

Our main result will be that under an additional axiom, all models of affine geometry are describable as $K^2$, for some division ring $K$. Under a slightly stronger axiom, we will also be able to show $K$ is commutative. An affine isomorphism between two fields induces an isomorphism between the fields themselves, so any affine geometry can be {\it uniquely} coordinatized by some field. This justifies the introduction of Cartesian coordinates in synthetic geometry; if we can prove a theorem in affine geometry using coordinates, the theorem must be true in every model of Euclidean geometry, and therefore there must exist a synthetic proof of the theorem, i.e. a proof proceeding directly from the axioms. To begin with this proof, we consider any affine geometry, and fix a line $OI$. We will give $OI$ a division ring structure, under which $O$ and $I$ are additive and multiplicative identities, and $(OI)^2$ is isomorphic to the affine geometry we started with. We begin by exploring what tools the axioms of affine geometry imply we can use.

\begin{lemma}
    Two intersecting lines cannot both be parallel to the same line.
\end{lemma}
\begin{proof}
    Let $l_0$ and $l_1$ be two lines intersecting at a point $X$. For any line $u$, there is a unique line through $X$ parallel to $u$, so since $l_0 \neq l_1$, they cannot both be parallel to $u$.
\end{proof}

\begin{corollary}
    Parallellism is an equivalence relation.
\end{corollary}
\begin{proof}
    The relation is clearly reflexive and symmetric. If $l_0$ was parallel to $l_1$, and $l_1$ was parallel to $l_2$, then the last lemma implies $l_0$ cannot intersect $l_2$, so they must be parallel.
\end{proof}

An equivalence class of parallel lines is called a {\bf pencil} of parallel lines. They are a useful family because they allow us to `translate' in our geometry. For instance, given two distinct lines $l_0$ and $l_1$, and a pencil of parallel lines not parallel to $l_0$ or $l_1$, then for each point $X \in l_0$, there is a unique line in the pencil through $X$ passing through a unique point $Y$ on $l_1$. This generates a bijective map between the points on $l_0$ and the points on $l_1$. We conclude that any two lines in an affine geometry contain the same number of points.

\begin{example}
    Let $l_0$ and $l_1$ be two intersecting lines. Given any point $Z$ in the plane, it passes through a unique line parallel to $l_1$, and therefore intersects a point $X$ on $l_0$. Similarily, it passes through a unique line parallel to $l_0$, which intersects a point $Y$ on $l_1$. This process is clearly reversible, so we have established a bijection between $l_0 \times l_1$ and the points on the plane. In particular, if a line $l$ in an affine plane contains $n$ points, then the entire plane contains $n^2$ points.
\end{example}

Now we return to our scenario of defining field operations on a line through two points $OI$. Given $O$, fix a line $l$ through $O$. Fix a point $M \not \in OI$, and let $r$ be the line through $M$ parallel to $l$. Given $A,B \in l$, the pencil of lines generated by $OM$ allows us to transport $A$ to a point $N$ on $r$. Similarily, the pencil of lines generated by $BM$ allows us to transport $N$ back to a point on $l$, which we define as $A + B$. Unfortunately, in general affine geometries this operation will depend on the point $M$ we choose, and this means addition does not in general even have to be associative. When we try to construct addition on two lines, we require a very subtle property. To ensure that addition is independent of the line we choose, we must assume a theorem of Desargues, which specializes our study to the theory of Desarguian planes.
%
\begin{itemize}
    \item Suppose we are given three lines $A_0A_1, B_0B_1$, and $C_0C_1$, either all parallel or all meeting in a point. Then if $A_0B_0$ is parallel to $A_1B_1$ and $A_0C_0$ is parallel to $A_1C_1$, then $B_0C_0$ is parallel to $B_1C_1$.
\end{itemize}
%
Later on, using the techniques of projective geometry, this theorem will become simple to prove for the models $K^2$. Consider two choices $M_0,N_0$, and $M_1,N_1$ of the parameters, generating two points $(A+B)_0$ and $(A+B)_1$. Then $M_0N_0$, $OI$, and $M_1N_1$ are parallel, as are $OM_0$ and $AN_0$, and $OM_1$ and $AN_1$, so we conclude that $M_0M_1$ is parallel to $N_0N_1$. Since $M_0B$ is parallel to $N_0(A+B)_0$, we conclude that $M_1B$ is parallel to $N_1(A+B)_0$. But $N_1(A+B)_1$ is also parallel to this line, so $(A+B)_0 = (A+B)_1$. Desargues theorem also enables us to show addition is commutative.

\begin{theorem}
    $A+B = B+A$.
\end{theorem}
\begin{proof}
    Consider a choice of $M_0$ and $N_0$ leading to a definition of $A+B$. We can also construct $N_1$ leading to the definition of $B+A$. Our proof would be complete if we could prove that $N_1(A+B)$ is parallel to $MA$. Construct a point $X$ by taking a line parallel to $OM$, $AN_0$, and $BN_1$ through $A+B$, and intersecting it with the line through $M$, $N_0$, and $N_1$. Then $(A+B)X$ is parallel to $AN_0$, $XN_1$ is parallel to $N_0M$, and so by Desargue's theorem (a degenerate form, where one of the three parallel lines is repeated), we conclue that $(A+B)N_1$ is parallel to $AM$. Thus $A+B = B+A$.
\end{proof}

TODO: ASSOCIATIVITY.

Performing an analogous construction to addition, we can also construct the subtraction $A - B$ of two vectors, which is a point on the line through $OI$ such that $(A - B) + B$ is $A$. Thus the addition operation makes the line into an abelian group.

To obtain a well defined multiplication operation, we take a line through $O$ distinct to $OI$, project $I$ onto this line by an arbitrary point $M \neq O$, use this projection to project points onto the line, and then project them off to multiply arbitrary points $A,B$ on the line. Using Desargue's theorem, one can verify that this calculation does not depend on $M$, and the reversibility of this construction implies every element has an inverse, so that we obtain a division ring. If we assume another axiom, known as Pappus' theorem, we can conclude that the division ring is actually commutative, hence a field.
%
\begin{itemize}
    \item Consider two distinct lines $l_0$ and $l_1$ intersecting at a point $O$, where $l_0$ contains three points $A_0,B_0,C_0$, and $l_1$ contains three points $A_1,B_1$, and $C_1$. If $A_0B_1$ is parallel to $B_0C_1$, and $B_0A_1$ is parallel to $C_0B_1$, then $A_0C_1$ is parallel to $B_0A_1$.
\end{itemize}
%
The main result is that $OI$ has a natural structure which turns it into a field. If $OT$ is a line through $O$ distinct to $OI$, then the pencil generated by $IT$ gives a bijection of $OI$ with $OT$, and since every point in the plane is uniquely designated by its projection onto $OI$ with respect to the pencil generated by $OT$, and the projection onto $OT$ with respect to the pencil generated by $OI$, we conclude that the plane is in one to one correspondence with $OT \times OI$, which we now view as $OI^2$. Finally, we note that all the lines in the plane are given by $x + OIy$, for $x,y \in OI^2$, and this is left as an exercise. 

\section{Conic Sections}

In this section we discuss conic sections, which are planar figures obtained from intersecting a plane with a cone. These may be viewed as two dimensional figures by taking a coordinate system on the plane in question, and we will find that these figures describe all quadratic figures in the plane describable by the equation
%
\[ aX^2 + 2bXY + cY^2 + 2dX + 2eY + f = 0 \]
%
These equations seem complicated, but we will find that we can simplify these equations by applying {\it affine transformations} to these equations. In projective space, we cannot consider zero sets of arbitrary polynomials in homogenous coordinates, but we can consider the zero sets of {\it homogenous} polynomials, so we consider any conic as a three dimensional equation adding in the variable $Z$
%
\[ aX^2 + 2bXY + cY^2 + 2dXZ + 2eYZ + fZ^2 = 0 \]
%
We will projectively classify these conics in $\mathbf{RP}^2$.

We can see such a conic as a zero set of a quadratic form in three dimensions, which has a matrix representation
%
\[ (X,Y,Z)^t \begin{pmatrix} a & b & d \\ b & c & e \\ d & e & f \end{pmatrix} \begin{pmatrix} X \\ Y \\ Z \end{pmatrix} = 0 \]
%
If we apply a projective transformation to $(X,Y,Z)$ by a matrix $M$, then this is the same as replacing the interior matrix with
%
\[ M^t \begin{pmatrix} a & b & d \\ b & c & e \\ d & e & f \end{pmatrix} M \]
%
Over the real numbers, it is Sylvester's law of inertia that we can choose a projective transformation $M$ such that the quadratic form is reduced to
%
\[ aX^2 + bY^2 + cZ^2 = 0 \]
%
for some values $a,b,c \in \mathbf{R}$. Since we may always multiply $X$ with a scalar multiple, these polynomials are only affinely distinguished by the sign of their coefficients. We may also multiply the entire equation by a scalar, so we may assume that there are at least as many positive coefficients than negative coefficients. We may also projectively consider a permutation of the variables, like $(X,Y,Z) \mapsto (Z,X,Y)$, so we may assume that $a \geq b \geq c$. These lead to the following five conics, which can be checked not to be projectively equivalent to one another.
%
\begin{itemize}
    \item $X^2 + Y^2 = Z^2$: The standard, {\it non degenerate conic}.
    \item $X^2 + Y^2 = 0$: The single point at the origin.
    \item $X^2 + Y^2 + Z^2 = 0$: No solutions.
    \item $X^2 = 0$: A projective line 
    \item $X^2 = Z^2$: Two distinct projective lines.
    \item $0$: The entire projective plane.
\end{itemize}
%
All of the well known conics, ellipses, circles, hyperbolas, and parabolas, are projectively equivalent to the conic $X^2 + Y^2 = Z^2$, which interestingly enough can be seen as the equation in $\mathbf{R}^3$ for the standard cone. This makes sense, because every plane  not passing through the origin in $\mathbf{R}^3$ corresponds to an affine plane embedded in $\mathbf{RP}^2$, and the intersections of this plane with the cone corresponds to all possible situations of what the conic looks like on the affine plane. The projective transformations which maps this plane to itself correspond to all affine transformations of the plane, so the conics which are affinely equivalent may be described as those corresponding affine transformations. If the plane is $Z = 1$, we obtain the canonical embedding of $\mathbf{R}^2$ in $\mathbf{RP}^2$, and the conic $X^2 + Y^2 = Z^2$ becomes the circle $X^2 + Y^2 = 1$. The plane at infinity here is $Z = 0$, so for any $r \neq 0$, the coordinates $Z = r$ induce an affine transformation showing that the circles $X^2 + Y^2 = r^2$ are all affinely equivalent. Similarily, the scalings $X = aX$ and $Y = bY$ induce the equations $a^2X^2 + b^2Y^2 = r^2$, which are the equations for the ellipses whose axis lie on the $X$ and $Y$ axis. If we have an ellipse whose axis do not lie on these axis, a simple rotation of the plane will map this ellipse affinely to one of the forms we understand. Thus circles and ellipse are affinely one and the same.



These may be projectively equivalent, but are not {\it affinely equivalent}. To distinguish between the projective and affine equations, we distinguish each of the degenerative bullets.
%
\begin{itemize}
    \item $X^2 + Y^2 + Z^2 = 0$: All figures obtained by projective transformations from this set have no solutions, and therefore all the figures are trivially affinely equivalent.
    \item $X^2 + Y^2 = 0$: The projective solution set is just a single point, and the projective transformations can map this point to any other point in the affine plane, or a single point on the line at infinity. These correspond to two solution sets which are not affinely equivalent.
    \item $X^2 = 0$ has a projective line as the solution set, and every projective transformation can map this line to any other line, so the other affine solutions are all just lines, which are affinely equivalent, or the line at infinity, which will have a trivial solution set in the affine plane. Thust we obtain two solution sets which are not affinely equivalent.
    \item $X^2 = Z^2$: The solution set corresponds to the intersection of two projective lines, and every projective transformation will map this solution set to the intersection of two projective lines. If these lines intersect at infinity, then the affine solution set will correspond to two parallel lines, which are all affinely equivalent, or the affine solution set will correspond to two intersecting liens, which are also all affinely equivalent. Thus we obtain two affine solution sets here.
    \item $0$: Whose solution set is the entire plane $\mathbf{RP}^2$, and whose affine solution set will always be $\mathbf{R}^2$.
    \item $X^2 + Y^2 = Z^2$: Whose solution set is a circle with radius one. The projective transformations take this conic into a conic of the form $aX^2 + 2bXY + cY^2 + 2dX + 2eY + f$, where the matrix
    %
    \[ \begin{pmatrix} a & b & d \\ b & c & e \\ d & e & f \end{pmatrix} \]
    %
    is invertible. We can actually classify these conics affinely by looking at the discriminant $b^2 - ac$. It is clear that the only conics affinely equivalent to $X^2 + Y^2 = Z^2$ are the ellipses,
\end{itemize}
%
It is interesting that, in the projective case, two polynomials are projectively equivalent if and only if their solution sets are projectively equivalent, whereas this does not hold in the affine case -- two solution sets can be affinely equivalent, while their polynomials cannot be affinely transformed onto one another.


\chapter{Projective Geometry}

Projective geometry can be viewed as a slight modification to affine geometry. Rather than distinguishing between parallel and intersecting lines, in projective geometry we deduce theorems based on the assumption that all lines intersect. Originally studied as a `paradoxical' system used to prove the parallel postulate by contradiction, projective geometry is now seen as a perfectly sound geometric system, which is often more elegant than classical Euclidean geometry. Projective geometry proceeds in studying geometries satisfying two geometric axioms:
%
\begin{itemize}
    \item Every distinct pair of lines has a unique intersection.
    \item Every distinct pair of points is connected by a unique line.
\end{itemize}
%
as well as a `regularity' axiom, to protect against degeneracy:
%
\begin{itemize}
    \item Every line has 3 or more points, and there is more than one line.
\end{itemize}
%
The most visual way to see a model of projective geometry is as a model of affine geometry, with added `points at infinity'. In particular, if we take a model of projective geometry, and remove any line from the geometry (as well as points on that line), then the resulting geometry will be a model of affine geometry. Conversely, there is a unique way of extending any affine geometry to obtain a projective geometry by adding a `line at infinity'.

\begin{example}
    We can obtain a projective geometry from any affine geometry by adding a single point for each set of parallel lines, causing these parallel lines to intersect. More specifically, in any affine geometry $G$, for each line $l$ we consider the equivalence class $[l]$ of lines parallel to $l$ (the {\it pencil} of lines generated by $l$). We obtain a projective geometry $G_\infty$ by adding a new point $p_{[l]}$ for each pencil of lines, extending each line $l$ so that $p_{[l]}$ lies on $l$, and adding a new {\it line at infinity} $l_\infty$, which consists of all points $p_{[l]}$ not found in the original geometry.
\end{example}

\begin{example}
    There is a particular way to coordinatize the projective plane $K^2_\infty$, where $K$ is some field. We consider the projective geometry $\mathbf{P}K^3$, also denoted $K\mathbf{P}^2$, whose points are lines through the origin in $K^3$, and whose lines are planes through the origin in $K^3$. This is a projective geometry, because two planes through the origin in three dimensions always intersect in a line, and two lines through the origin generate a unique plane through the origin. This space has a {\bf homogenous coordinate system} obtained by the correspondence $K^3 - \{ 0 \} \to \mathbf{P}K^3$, taking a vector $v$ to its span $Kv$. The fibres of this correspondence are written as $[x:y:z]$, and are essentially the smallest equivalence classes such that $[\lambda x: \lambda y: \lambda z] = [x:y:z]$ for all nonzero $\lambda \in K$. It is often very useful to introduce a homogenous coordinate system to a projective geometry, because then we can understand the space through algebraic operations. For instance, a line in $\mathbf{P}K^3$ can be described as the solution set to the equation $aX + bY + cZ = 0$ (which is well defined because the polynomial is homogenous), for some $a,b,c \in K$, which is obvious if we think of a line in $\mathbf{P}K^3$ as a plane through the origin. Since the $a,b,c$ are unique up to scaling, the homogenous coordinates $[a:b:c]$ can also be used to describe the line. We embed $K^2$ in $K\mathbf{P}^2$ by mapping $(x,y)$ to $[x:y:1]$. In this case, the `line at infinity' in $K\mathbf{P}^2$ can be thought of as the points $[x:y:0]$, which are not in the image of this correspondence. In particular, a line in $K^2$, which can be identified as the solution set of the equation $aX + bY = c$, consists of all points which are in the plane $aX + bY = cZ$, and if $Z = 0$, then the added point at infinity of this line is just the solution the line $aX + bY = 0$ through the origin. We switch between the notations $K\mathbf{P}^2$ and $\mathbf{P}K^3$ depending on whether we want to view the object as `two dimensional' or `one dimensional'. In general, for any vector space $V$, $\mathbf{P}V$ is the space of lines through the origin in $V$. It is a more natural functor from the point of view of the category of vector spaces, but less intuitive geometrically, because in the general case we have no natural embedding of some `affine space' $W$ in $\mathbf{P}V$.
\end{example}

If we remove a line from the projective geometry to obtain an affine geometry, and then add a line at infinity back in, we obtain a geometry isomorphic to the original geometry, so the two axiom systems are essentially in one to one correspondence. In particular, any projective geometry satisfying Desargues' theorem and Pappus' theorem must be isomorphic to $\mathbf{P}K^2$ for some field $K$, because if we remove a line at infinity we obtain a geometry isomorphic to $K^2$. Thus systems of homogenous coordinates over an arbitrary field as the `correct' way to introduce analytic geometry to the study of projective geometry, and we will find ourselves focusing on projective geometries which have homogenous coordinates more and more over the course of our study of projective geometry.

\section{Duality}

Unlike affine geometry, projective geometry has an incredibly rich duality theory, because the axioms defining projective geometry are symmetric with respect to points and lines. If we take any model of projective geometry, and exchange the specification of points and lines, we end up with another projective geometry. Thus if we take any theorem of projective geometry and exchange the definition of points and lines, and swap a statement of the form `a point $x$ lies on a line $l$' with `a line $x$ contains a point $l$', we obtain another theorem of projective geometry. We will see this duality theory appear again and again in our study of projective systems.

Another way we can see the duality theory is through projective geometries with homogenous coordinates. Points and lines are both described by homogenous coordinates. A point $x$ lies on a line $l$ if and only if $x \cdot l = 0$, which is a condition completely symmetric with respect to points and lines. Given two points $x$ and $y$, the cross product $x \times y$ (well defined since the cross product on vectors is bilinear) describes a vector orthogonal to $x$ and $y$. Thus $x \times y$ gives the coordinates of the line passing through $x$ and $y$. Conversely, given two lines $l$ and $u$, the coordinates $l \times u$ describe the homogenous coordinates of the point lying at the intersection of $l$ and $u$. Thus techniques for calculating with points dualize to give techniques to calculate with lines.

\begin{example}
    Given two lines $l$ and $l_\infty$, and an additional point $X$, we often want to calculate the unique line through $X$ passing through the point on the intersection of $l$ and $l_\infty$. If $l_\infty$ is viewed as the line at infinity, this means exactly that we want to find the line through $X$ parallel to $l$. Since $l \times l_\infty$ gives the homogenous coordinates of the point we want $X$ to pass through, we can find this value as $X \times (l \times l_\infty)$.
\end{example}

\section{Projective Transformations}

The interesting transformations on projective space are those preserving colinearity. They are know as {\bf perspectivities}, or {\bf homographies}. Since linear transformations in $T: K^3 \to K^3$ map lines to lines, they induce maps from $\mathbf{P}K^2$ to $\mathbf{P}K^2$, and since $T$ maps planes to planes, it preserves colinearity. In coordinates, we can describe these transformations by 3 by 3 invertible matrices
%
\[ \begin{pmatrix} a & b & c \\ d & e & f \\ g & h & i \end{pmatrix} \in GL_3(K) \]
%
The fundamental theorem of projective geometry says that the only transformations on projective spaces are those induced by {\bf semilinear maps} $T: K^3 \to K^3$, which are maps satisfying
%
\[ T(x + y) = Tx + Ty\ \ \ \ T(cx) = \theta(c) Tx \]
%
for some particular field automorphism $\theta: K \to K$. They form the {\bf projective group} $P_2(K)$. For now, lets explore the subgroup of homographies obtained by linear maps on $K^3$, which is the {\bf projective linear group} $PGL_2(K)$, and can be described as $GL_3(K)$ modulo the center $Z(K)$ (which in most cases is just the set of scalar multiples of the identity). First, we note the following `rigid' properties of these maps.

\begin{theorem}
    Given two collections of four points $A,B,C,D$ and $A',B',C',D'$, neither of which contain three colinear points, there is a unique transformation in $PGL_3(K)$ mapping the first collection to the second collection.
\end{theorem}
\begin{proof}
    First, consider the special case where $A = [1:0:0]$, $B = [0:1:0]$, $C = [0:0:1]$, and $D = [1:1:1]$. If $A'$ has homogenous coordinates $[A_1:A_2:A_3]$, $B'$ has homogenous coordinates $[B_1:B_2:B_3]$, and so on and so forth, we essentially must find an invertible matrix
    %
    \[ M = \begin{pmatrix} a & b & c \\ d & e & f \\ g & h & i \end{pmatrix} \]
    %
    such that
    %
    \[ M[1:0:0] = [a:d:g] = [A_1:A_2:A_3] \]
    \[ M[0:1:0] = [b:e:h] = [B_1:B_2:B_3] \]
    \[ M[0:0:1] = [c:f:i] = [C_1:C_2:C_3] \]
    \[ M[1:1:1] = [a+b+c:d+e+f:g+h+i] = [D_1:D_2:D_3] \]
    %
    Trying to reduce this to a problem of linear algebra, we have to try and find four constants $\alpha, \beta, \lambda, \gamma$ such that
    %
    \[ \alpha M(1,0,0) = (A_1,A_2,A_3)\ \ \ \beta M(0,1,0) = (B_1,B_2,B_3) \]
    \[ \lambda M(0,0,1) = (C_1,C_2,C_3)\ \ \ \gamma M(1,1,1) = (D_1,D_2,D_3) \]
    %
    Solving the first three constraints is equivalent to solving the matrix equation
    %
    \[ \begin{pmatrix} \alpha & 0 & 0 \\ 0 & \beta & 0 \\ 0 & 0 & \lambda \end{pmatrix} \begin{pmatrix} a & b & c \\ d & e & f \\ g & h & i \end{pmatrix} = \begin{pmatrix} A_1 & A_2 & A_3 \\ B_1 & B_2 & B_3 \\ C_1 & C_2 & C_3 \end{pmatrix} \]
    %
    and the matrix on the right is invertible because these vectors are not colinear, hence we can uniquely find the matrix of coefficients on the left for any particular nonzero values of $\alpha, \beta$, and $\lambda$. Since $A,B$ and $C$ are not colinear, we may write $(D_1,D_2,D_3) = a(A_1,A_2,A_3) + b(B_1,B_2,B_3) + c(C_1,C_2,C_3)$ for a unique pair of coefficients $a,b$ and $c$, which all must be nonzero, because $(D_1,D_2,D_3)$ is not colinear with any of the other two points. Now we need only to find values $\alpha, \beta$, and $\lambda$ and $\gamma$ such that
    %
    \[ \gamma M(1,1,1) = \gamma \alpha^{-1}(A_1,A_2,A_3) + \gamma \beta^{-1}(B_1,B_2,B_3) + \gamma \lambda^{-1}(C_1,C_2,C_3) \]
    %
    is equal to $(D_1,D_2,D_3)$, and we set $\gamma \alpha^{-1} = a$, $\gamma \beta^{-1} = b$, and $\gamma \lambda^{-1} = c$. If we fix $\gamma = 1$, the values of $\alpha, \beta$, and $\gamma$ are uniquely determined, so the element of $PGL_2(K)$ is also uniquely determined.
\end{proof}

Projective linear transformations which fix the line at infinity in $K\mathbf{P}^2$ can model all sorts of interesting transformations on Euclidean space. First, note that since the line at infinity may be identified with the set of points with homogenous coordinates $[x:y:0]$, a matrix
%
\[ \begin{pmatrix} a & b & c \\ d & e & f \\ g & h & i \end{pmatrix} \]
%
preserves the line at infinity if and only if $g = h = 0$. It then follows that in order to be invertible, $i$ must be nonzero, and by normalizing, we determine that we can describe such maps as
%
\[ \begin{pmatrix} a & b & c \\ d & e & f \\ 0 & 0 & 1 \end{pmatrix} \]
%
restricting this map to points in $K^2$, we find that it takes the form
%
\[ X \mapsto  \begin{pmatrix} a & b & c \\ d & e & f \\ 0 & 0 & 1 \end{pmatrix} [X:1] = \begin{pmatrix} a & b \\ d & e \end{pmatrix} X + \begin{pmatrix} c \\ f \end{pmatrix} \]
%
so projective linear transformations which fix the line at infinity are exactly the affine linear transformations, consisting of all skews of the plane combined with a translation.

\section{The Projective Line and Cross Ratios}

The space $K\mathbf{P}^1 = \mathbf{P}K^2$ is known as the {\bf projective line}, and can be seen as a one dimensional projective geometry. Indeed, it contains the affine line $K$ where $x \in K$ is viewed as $[x:1]$, as well as a unique point at infinity $[1:0]$, which we also denote as $\infty$. In Euclidean geometry, we understand the plane by fixing an arbitrary line and studying the properties of the line invariant under affine transformations. We shall try to do the same in $K\mathbf{P}^1$. Given a line of the form $\{ [\alpha v + \beta w] : \alpha, \beta \in K \}$ in $\mathbf{P}K^3$, we can coordinatize the line once we fix $v$ and $w$ by the map $[\alpha v + \beta w] \mapsto [\alpha: \beta]$, which are the homogenous coordinates in $K\mathbf{P}^1$. This is equivalent to viewing $w$ as the origin of the projective line, $v$ as the unique point at infinity, and then fixing a scale so that $[v+w]$ is expressed as $1$ in the coordinates. Given any other spanning set $v',w'$, there is a change of basis matrix $T$ mapping $v$ to $v'$ and $w$ to $w'$, which by linearity induces a transformation on $K\mathbf{P}^1$. The family of all such transformations is the one dimensional projective linear group, denoted $PGL_1(K)$, which can be described as those maps induced by
%
\[ \begin{pmatrix} a & b \\ c & d \end{pmatrix} \in GL_2(K) \]
%
And also are of the form
%
\[ x \mapsto \frac{ax + b}{cx + d} = [ax + b: cx + d] \]
%
By applying similar techniques to the theory of projective linear transformations in $K\mathbf{P}^2$, we find that a projective linear transformation is uniquely specified by where it maps 3 distinct points. In particular, a projective transformation preserving $0$, $1$ and $\infty$ must be the identity, and any transformation preserving $0$ and $\infty$ must be of the form
%
\[ \begin{pmatrix} \alpha & 0 \\ 0 & \beta \end{pmatrix} \]
%
for nonzero $\alpha, \beta$, and therefore after normalizing by setting $\beta = 1$, we see that the transformation is just $x \mapsto \alpha x$.

\begin{lemma}
    If $l_0 = v_0 \times w_0$ and $l_1 = v_1 \times w_1$ are two lines, and $O$ is a point not lying on either, consider the projection $p \mapsto (p \times o) \times l_1$. If $v_0 \mapsto v_1$ and $w_0 \mapsto w_1$, then the induced function on homogenous coefficients induced by the specification $(v_0,w_0)$ and $(v_1,w_1)$ is just multiplication by some scalar in $K$.
\end{lemma}
\begin{proof}
    It is clear that the map is linear in $p$, because the cross product is bilinear, hence the induced transformation on homogenous cooeficients must be an element of $PGL_1(K)$. Switching to the coefficients, we see that $v_0 = v_1 = \infty$ and $w_0 = w_1 = 0$, so this transformation must preserve these two points, and is therefore just given by multiplication by a scalar.
\end{proof}

If we consider a projective transformation on $K\mathbf{P}^2$ mapping a line $l_0$ to a line $l_1$, then in arbitrary projective coordinates on $l_0$ and $l_1$, the induced transformation will be a projective map on $K\mathbf{P}^1$, and it is therefore of interest to analyze projective transformations on the projective line, and the invariants of these transformations. Building up this theory will enable us to classify the homographies in the projective plane.

We now introduce one of the most important invariants of projective geometry. For notational convinience, given $v,w \in K^2$, let $[v,w] \in K$ denote the determinant of the matrix obtained by stacking the column vectors $v$ and $w$ into a matrix. That is, $[v,w] = v_1w_2 - v_2w_1$. For four points $A,B,C,D \in K^2$, we define the cross ratio to be
%
\[ (A,B;C,D) = \left( [A,C][B,D] : [A,D][B,C] \right) = \frac{[A,C][B,D]}{[A,D][B,C]} \]
%
One way to remember the formula is as a `ratio of ratios' between $A$ and $B$, i.e.
%
\[ \frac{[A,C][B,D]}{[A,D][B,C]} = \frac{[A,C]}{[A,D]} \bigg/ \frac{[B,C]}{[B,D]} \]
%
If $A = [A:1], \dots, D = [D:1]$ are finite, then the cross ratio of the four points is
%
\[ \frac{(A-C)(B-D)}{(A-D)(B-C)} \]
%
The cross ratio is invariant under any transformation $T \in GL_2(K)$, because
%
\begin{align*}
    (TA,TB;TC,TD) &= \left( [TA,TC][TB,TD] : [TA,TD][TB,TC] \right)\\
    &= (\det(T)^2 [A,C][B,D]: \det(T)^2 [A,D][B,D]) = (A,B;C,D)
\end{align*}
%
This also implies the cross ratio descends to homogenous coordinates. For any nonzero $\lambda \in K$, $(\lambda A, \lambda B; \lambda C, \lambda D) = (A,B;C,D)$, and therefore the cross ratio can be considered for points on the projective line. It is invariant under the action of $PGL_1(K)$. This implies, in particular, that for any four points $A,B,C,D$ on a line $l$ in $K\mathbf{P}^2$, the cross ratio is defined irrespective of the particular homogenous coordinate system, and is also invariant under the action of $PGL_2(K)$. What's more, if $O$ is not on the line $l$, then we may calculate the cross ratio in $K\mathbf{P}^2$ as
%
\[ (A,B;C,D) = \frac{[O,A,C][O,B,D]}{[O,A,D][O,B,C]} \]
%
where $[A,B,C]$ is the determinant of the three by three stacked matrix, which is invariant under $PGL_2(K)$. To see this, we may assume $O = [0:0:1]$, $A = [1:0:0]$, and $B = [0:1:0]$, in which case it follows that $C = [C_1:C_2:0]$ and $D = [D_1:D_2:0]$ for some values $C_1,C_2,D_1,D_2$, and then the theorem is obvious.

The cross ratio is obviously not commutative. Swapping two pairs on either side of the equation corresponds to inverting the cross ratio, i.e.
%
\[ (b,a;c,d) = (a,b;d,c) = \frac{1}{(a,b;c,d)} \]
%
Conversely, swapping two elements on either side corresponds to reflecting the ratio about the line $X = 1/2$.
%
\[ (d,b;c,a) = (a,c;b,d) = 1 - (a,b;c,d) \]
%
If $(a,b;c,d) = \lambda$, then the six possible values of the cross ratio obtained by permuting points in the ratio are obtained by composing these two operations, and therefore consist of
%
\[ \lambda, \frac{1}{\lambda}, 1 - \lambda, \frac{1}{1 - \lambda}, \frac{\lambda}{\lambda - 1}, \frac{\lambda - 1}{\lambda} \]
%
and the transformations formed by reflection at $1/2$ and inversion form the anharmonic group, which is isomorphic to the Klein four group. The values of the cross ratio become degenerate when $\lambda = 1 - \lambda$, or $\lambda = \lambda^{-1}$, that is, at the values $\lambda = 1/2$, $\lambda = -1$, $\lambda = 1$, and $\lambda = \infty$. These break down into the orbit sets $\{ -1, 2, 1/2 \}$ and $\{ 1, 0, \infty \}$. In the latter case, $(A,B;C,D) = 0$ implies that one of the points $A$ and $B$ is equal to one of the other points $C,D$, so assuming our points $A,B,C,D$ are distinct, the only degenerate case up to orbits is $(A,B;C,D) = -1$. In this case, we say $(A,B)$ and $(C,D)$ are in {\bf harmonic position}.

Given three distinct points $A$, $B$, and $C$ on a projective line, there is a unique point $D$ such that $(A,B)$ and $(C,D)$ are in Harmonic position. This is clear because for any fixed $A,B$ and $C$, the map $D \mapsto (A,B;C,D)$ is a projective transformation. given by the linear operator
%
\[ D \mapsto ([A,C](B_1D_2 - D_1B_2), [B,C] A_1D_2 - A_2D_1) = \begin{pmatrix} -[A,C]B_2 & [A,C]B_1 \\ -[B,C]A_2 & [B,C]A_1 \end{pmatrix} D \]
%
and the determinant of the operator is $[A,C][B,C][B,A]$, which is nozero provided $A$, $B$ and $C$ are distinct points. One way to construct $D$ is by fixing an auxillary point $O$ off of the line $l$ containing $A,B$, and $C$, choosing another auxilary point $P$ on $O \times C$, and then considering the point $A' = (O \times A) \times (P \times B)$, $B' = (O \times B) \times (P \times A)$, and $D = (A' \times B') \times l$. If we let $C' = (A' \times B') \times (O \times C')$, then we find $(A,B;C,D) = (A',B';C',D')$, and also $(A,B;C,D) = (B',A':C',D')$, so it follows that since the opposite pairs of points are distinct, that $(A,B;C,D) = -1$. If $A = B$, then $(A,B;C,D) = 1$, so this construction is impossible.

\section{Calculating With Determinants}

An important principle of Pl\"{u}cker's theory of geometry is that using determinants is the easiest way to calculate properties of objects in projective geometry that are invariant under projective transformations. The homogenous coordinates of a point obviously change with respect to a projective transformation, so they are not an elegant choice to calculate invariants; instaed, we must view the determinants as the `first class citizens'. Because we will use determinants so often in this chapter, we introduce a simplifying notation. Given a set of $n$ vectors $p_1, \dots, p_n \in \mathbf{R}^n$, we will let $[p_1, \dots, p_n]$ denote the determinant of the matrix formed by adjoining these vectors as columns. Clearly, this gives a multilinear map on the space.

\begin{example}
    Three points $p,q,r \in \mathbf{P}^2$ being colinear is obviously an invariant property of a projective transformation. And indeed, we find that this property can be expressed via determinants: The points are colinear if and only if $[p,q,r] = 0$.
\end{example}

A very useful technique in these calculations, used extensively by Pl\"{u}cker, and known as the Pl\"{u}cker mu. We suppose that we have a family of geometric objects described as the zero sets of a family of functions, such that these functions form a vector space. Given two functions $f$ and $g$, the family $\alpha f + \beta g$ all describe objects containing the intersection points of $f$ and $g$, since if $f(p) = g(p) = 0$, then $\alpha f(p) + \beta g(p) = 0$. Given a point $q$, the object described by the function $g(q) f - f(q) g$ passes not only through the intersection points of $f$ and $g$, but also through $q$. This often gives the fastest and most elegant way to specify a geometric object.

\begin{example}
    Given two lines $l_0$ and $l_1$ in projective space, we wish to find an easy way to describe the homogenous coordinates of the line passing through the intersection points of $l_0$ and $l_1$, and an additional point $q$. Since the set of lines form a vector space in homogenous coordinates, and the zero set corresponding to a line $l$ results from the function $p \mapsto l \cdot p$, the line we desire is precisely $(l_1 \cdot q) l_0 - (l_0 \cdot q) l_1$.
\end{example}

\begin{example}
    What is the point $r$ lying on the intersection of the lines formed by two point pairs $(p_0,p_1)$ and $(q_0,q_1)$. Of course, this is given by $(p_0 \times p_1) \times (q_0 \times q_1)$, but Pl\"{u}cker's mu gives a simpler formulation. A point lying on the line between $p_0$ and $p_1$ must be of the form $\alpha p_0 + \beta p_1$. It must also cause the function $(q_0 \times q_1) \cdot r = [q_0,q_1,r] = 0$, and so this point is precisely
    %
    \[ [q_0,q_1,p_1] p_0 - [q_0,q_1,p_0] p_1 \]
    %
    Alternatively, swapping the places of the $p_i$ and $q_i$ shows that
    %
    \[ [q_0,q_1,p_1] p_0 - [q_0,q_1,p_0] p_1 = [p_0,p_1,q_1] q_0 - [p_0,p_1,q_0] q_1 \]
    %
    which is essentially a reformulation of Cramer's rule. Since
    %
    \[ (r_0 \times r_1) \cdot ([p_0,p_1,q_1] q_0 - [p_0,p_1,q_0] q_1) = [r_0,r_1,q_0][p_0,p_1,q_1] - [p_0,p_1,q_0][r_0,r_1,q_1] \]
    %
    Using this formula, we can express that three point pairs $(p_0,p_1)$, $(q_0,q_1)$, and $(r_0,r_1)$ meet at a single point if and only if $[r_0,r_1,q_0][p_0,p_1,q_1] = [p_0,p_1,q_0][r_0,r_1,q_1]$.
\end{example}

Using this technique, we can prove Pappos' theorem in a simple way. First we reformulate the theorem by saying that if for six points $a,\dots,f$, the lines $a \times d, c \times b, e \times f$, and the lines $c \times f$, $e \times d$, and $a \times b$ meet, then so too do $e \times b$, $a \times f$, and $c \times d$. The two hypothesis say
%
\[ [b,c,e][a,d,f] = [b,c,f][a,d,e]\ \ \ \ \ [c,f,b][d,e,a] = [c,f,a][d,e,b] \]
%
and we must prove
%
\[ [f,a,c][e,b,d] = [f,a,d][e,b,c] \]
%
But this follows because the second term of the first equation is equal to the second term of the first, and transitivity leads exactly to the equation.

Essentially {\it all} properties of projective geometry invariant under projective transformations are expressable in terms of determinants, a fact we will now discuss. If we consider $n$ points $x_1, \dots, x_n$, a property $P(x_1, \dots, x_n)$ will be {\it projectively invariant} if for any projective transformation $T$, $P(Tx_1, \dots, Tx_n) = P(x_1, \dots, x_n)$.

\begin{example}
    If $f$ is a homogenous polynomial in the brackets of $x_1, \dots, x_n$, then since $[Tx_1, Tx_2, Tx_3] = \det(T) [x_1, x_2, x_3]$, then $f(x_1, \dots, x_n) = 0$ if and only if $f(Tx_1, \dots, Tx_n) = 0$. Thus vanishing on such a homogenous polynomial is projectively invariant.
\end{example}

\section{Harmonic Points and Quadrilateral Sets}





\chapter{Conformal Geometry}

Minkowski geometry is the study of Euclidean space compactified at a point at $\infty$, and the study of angle preserving maps on this space. The most basic occurs in the plane, which we can identify with $\CC$, where the compatification is the Riemann sphere, and the angle preserving maps on the Riemann sphere $\Sigma = \CC \cup \{ \infty \}$ are precisely the \emph{M\"{o}bius transformations}
%
\[ z \mapsto \frac{az + b}{cz + d}, \]
%
where $ad - bc \neq 0$. On the other hand, the only angle preserving maps from $\CC$ to itself are maps of the form $z \mapsto az + b$, i.e. compositions of rotations, dilations, and translations. Thus the conformal structure of $\Sigma$ is much richer than that of the complex plane. Because of the conformal structure, the M\"{o}bius transformations map `generalized circles' to `generalized circles', where by a generalized circle we mean either a circle, or the union of a straight line $L$ and $\infty$, which we think of as a circle passing through infinity.

To do this more rigorously, we must equip $\CC \cup \{ \infty \}$ with the structure which will enable us to consider tangent vectors on the space, and the ability to measure angles between tangent vectors on the space, in a way which agrees with the definition of angles for finite points. The trick here is to consider the stereographic projection map $\sigma: \CC \cup \{ \infty \} \to S^2$, which identifies $\CC \cup \{ \infty \}$ with $S^2$, where for $z \in \CC$, $\sigma(z)$ is the unique point on $S^2 - \{ (-1,0,0) \}$ on the line between $(z,0)$ and $(-1,0,0)$,
%We define the map by setting, for each $x \in \CC$, 
%
%\[ \sigma(z) = \left( \frac{1 - |z|^2}{1 + |z|^2}, \frac{2z}{1 + |z|^2}, \right) \]
%
and defining $\sigma(\infty) = (-1,0,\dots,0)$. Identifying $\Sigma$ with $S^2$ gives an alternate Riemannian metric $g$ to $\CC$. However, this metric structure is \emph{conformally equivalent} to the standard metric on $\CC$, i.e. one can calculate simply that there exists a smooth function $u: \CC \to (0,\infty)$ such that for each $z \in \CC$ and $v,w \in \CC$, $g_z(v,w) = u(z) (v \cdot w)$. A consequence of this is that the angles between tangent vectors with respect to $g$ and the standard metric are equivalent. It is also simple to see that straight lines in $\CC$ correspond to circles passing through $\infty$ in $\Sigma$. The other circles on $\Sigma$ correspond precisely to circles in $\CC$.

Identifying $\Sigma$ with $S^2$ also gives us a way to parameterize the circles of $\Sigma$ For each $y \in \RR^3$ with $y \in \RR^3$ with $|y| > 1$, the hyperplane $H = \{ x : x \cdot y = 1 \}$ intersects $S^2$ in a circle $S_y$, corresponding to a circle on $\Sigma$. However, this parameterization $y \mapsto S_y$ does not include all generalized circles; it missed out on the \emph{great circles}. However, the fact that these great circles are achieved `asymptotically' as $|y| \to \infty$ implies we might be able to come up with a consistent parameterization by projectivizing; for each $y = [y_0: \dots : y_3] \in \mathbf{P}^3$ with $y_1^2 + y_2^2 + y_3^2 > y_0^2$, the projective hyperplane $H = \{ x \in \mathbf{P}^3: x_1y_1 + x_2y_2 + x_3y_3 = x_0y_0 \}$ intersects the sphere $S^2$ in a circle $S_y$. If $y = [1:y_1:y_2:y_3]$ is a finite point, then $S_y$ agrees with the definition given earlier. On the other hand, if $y = [0:y_1:y_2:y_3]$ is an infinite point, then $S_y = \{ x \in S^2 : x_1y_1 + x_2y_2 + x_3y_3 = 0 \}$ gives precisely the family of great circles we missed earlier. Thus we can parameterize the family of all circles on $S^2$ by the open set of points in $\mathbf{P}^3$ lying `outside' of $S^2$.

This can all be done in much greater generality on $\RR^n$, in a very simple manner. We consider the space $\RR^n \cup \{ \infty \}$, which, via stereoscopic projection, posseses a conformal structure by identifying it with $S^n$, which we will denote by $\Sigma$. For each $y \in \mathbf{P}^{n+1}$ such that $y_0^2 < y_1^2 + \dots + y_n^2$, we consider the hypersphere $S_y$ in $\Sigma$ which corresponds to the circle obtained by intersecting $S^n$ with the hyperplane $H = \{ x \in \mathbf{P}^{n+1} : x_0y_0 = x_1y_1 + \dots + x_{n+1}y_{n+1} \}$. If $S_y$ does not contain $\infty$, then $S_y \subset \RR^n$ is precisely a hypersphere. On the other hand, if $S_y$ contains $\infty$, then $S_y$ is the union of a hyerplane and $\infty$. Infinite points $y \in \mathbf{P}^{n+1}$ correspond to great circles on $\Sigma$. We call the space $\Sigma$ in general a \emph{M\"{o}bius space}. Thus the points in $\mathbf{P}^{n+1}$ lying outside $S^n$ parameterize the family of hyperspheres and hyperplanes in $\RR^n$.

It is useful in this situation to introduce the Lorentz form on $\RR^{n+2}$, defined by $x,y \in \RR^{n+2}$ by setting
%
\[ (x,y) = x_1 y_1 + \dots + x_{n+1} y_{n+1} - x_0 y_0. \]
%
The projective points lying outside of $\mathbf{P}^{n+1}$ can then be identified with the set $\{ x \in \PP^{n+1} : (x,x) > 0 \}$, and then for any such point $y$,
%
\[ S_y = \{ x \in S^n : (x,y) = 0 \}. \]
%
It is simple to check that two spheres $S_{y_1}$ and $S_{y_2}$ intersect orthogonally if and only if $(y_1,y_2) = 0$. Thus the Lorentz form, together with the parameterization $y \mapsto S_y$, provides an easy way to determine whether two spheres on $S^n$ intersect orthogonally.

A \emph{M\"{o}bius transformation} is a projective transformation $T$ on $\mathbf{P}^{n+1}$ such that $T(S^n) = S^n$. This is equivalent to the algebraic statement that if $x,y \in \mathbf{P}^{n+1}$ and $(x,y) = 0$, then $(Tx,Ty) = 0$. In particular, any M\"{o}bius transformation maps $S_y$ onto $S_{Ty}$, and preserves the orthogonality of spheres.

\chapter{Lie Geometry}

Lie modified the theory of M\"{o}bius geometry to give a geometry which focuses on spheres which `touch' one another. We consider the set
%
\[ W = \{ x \in  \} \]

\end{document}