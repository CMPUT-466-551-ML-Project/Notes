%% The following is a directive for TeXShop to indicate the main file
%%!TEX root = HarmonicAnalysis.tex

\part{Distributional Methods}

\chapter{The Theory of Distributions}

The theory of distributions is a tool which enables us to justify formal manipulations which occur in harmonic analysis in such a way that we can avoid the technical issues which occur from having to interpret such manipulations analytically. For instance, the Fourier transform is only defined for functions in $L^1(\RR^d)$. On the other hand, the theory of tempered distributions enables us to define the Fourier transform of \emph{any} locally integrable function, and even more general functions. Thus distributions are a cornerstone to the formulation of many problems in modern harmonic analysis.

The path of modern analysis has extended analysis from the study of continuous and differentiable functions to measurable functions. The power of this approach is that we can study a very general class of functions. On the other hand, the more general the class of functions we work with, the more restricted the analytical operations we can perform. Nonetheless, $C^\infty_c(\RR^d)$ is dense in almost all the spaces of measurable functions we consider in basic analysis, and for such functions we can apply all the fundamental analytical operations in this region. One approach to studying the general class of measurable functions is to prove results for elements of $C_c^\infty(\RR^d)$, and then apply an approximation result to obtain the result for a wider class of measurable functions. The theory of distributions provides an alternate approach, using \emph{duality} to formally extend analytical operations on $C^\infty_c(\RR^d)$ to larger sets.

From the perspective of set theory, functions $f: X \to Y$ are a way of assigning values in $Y$ to each point in $X$. However, in analysis this is not often the way we view functions. For instance, in measure theory, we are used to identifying functions which are equal almost everywhere, so that functions in this setting are only defined `almost everywhere'. In distribution theory, we view functions as `integrands', whose properties are understand by integration against a family of `test functions'. For instance, recall that for $1 \leq p < \infty$, the dual space of $L^p(\RR^d)$ is $L^q(\RR^d)$. Thus we can think of elements $f \in L^q(\RR^d)$ as `integrands', whose properties can be understood by integration against elements of $L^p(\RR^d)$, i.e. through the linear functional on $L^p(\RR^d)$ given by
%
\[ \phi \mapsto \int_{\RR^d} f(x) \phi(x)\; dx. \]
%
Similarily, the dual space of $C(K)$, where $K$ is a compact topological space, is the space $M(K)$ of finite Borel measures on $K$. Thus we can think of measures as a family of `generalized functions'. For each measure $\mu \in M(K)$, we consider the linear functional on $C(K)$ through the map
%
\[ \phi \mapsto \int_K \phi(x) d\mu(x). \]
%
Notice that as we shrink the family of test functions, the resultant family of `generalized functions' becomes larger and larger, and so elements can behave more and more erratically. A distribution is a `generalized function' tested against functions in $C_c^\infty(\RR^d)$. Since most operations in analysis can be applied to elements of $C_c^\infty(\RR^d)$, we can then use duality to extend these operations to distributions. Moreover, since $C_c^\infty(\RR^d)$ is a very `tame' space of functions, distributions are a very general family of generalized functions.

\begin{remark}
  From the perspective of experimental physics, viewing functions as integrands is more natural than the pointwise sense. Indeed, points in space are idealizations which do not correspond to real world phenomena. One can never measure the exact value of some quantity of a function at a point, but rather only understand the function by looking at it's averages over a small region around that point. Thus the only physically meaningful properties of a `function' are those obtained by testing that function against test functions.
\end{remark}

\section{The Space of Test Functions}

We fix an open subset $\Omega$ of $\RR^n$, and let $C_c^\infty(\Omega)$ denote the family of all smooth functions on $\Omega$ with compact support. Our goal is to equip $C_c^\infty(\Omega)$ with a complete locally convex topology, so that we can consider the dual space $C_c^\infty(\Omega)^*$ of \emph{distributions} on $\Omega$. We could equip $C_c^\infty(\Omega)$ with a locally convex, metrizable topology with respect to the seminorms
%
\[ \| f \|_{C^n(\Omega)} = \max_{|\alpha| \leq n} \| D^\alpha f \|_{L^\infty(\Omega)} \]
%
However, the resultant topology on $C_c^\infty(\Omega)$ is not complete.

\begin{example}
    Let $\Omega = \RR$, pick a bump function $\phi \in C_c^\infty(\RR)$ supported on $[0,1]$ with $\phi > 0$ on $(0,1)$, and define
    %
    \[ \psi_m(x) = \phi(x-1) + \frac{\phi(x-2)}{2} + \dots + \frac{\phi(x-m)}{m} \]
    %
    Then $\psi_m$ is compactly supported on $[1,m]$, and Cauchy, since for $m_1 \geq m_0$,
    %
    \[ \| \psi_{m_0} - \psi_{m_1} \|_{C^n(\RR)} = \frac{ \max_{r \leq n} \| D^r \phi \|_{L^\infty(\RR^d)}}{m_0+1} \lesssim_n 1/m_0. \]
    %
    However, the sequence $\{ \psi_m \}$ does not converge to any element of $C_c^\infty(\RR)$, since the sequence converges uniformly to the function
    %
    \[ \psi(x) = \sum_{n = 1}^\infty \psi(x-n) \]
    %
    an element of $C^\infty(\RR)$ which is not compactly supported.
\end{example}

We instead assign $C_c^\infty(\Omega)$ a stronger locally convex topology which prevents convergent functions from `escaping a set'; the cost, however, is that the topology is no longer metrizable. The process we perform here is quite general and can be viewed as a way to construct the `categorical limit' of a family of complete, locally convex spaces. For each compact set $K \subset \Omega$, the subspace $C_c^\infty(K) \subset C_c^\infty(\Omega)$ is a complete metric space under the family of seminorms $\| \cdot \|_{C^n(K)}$. We consider a convex topology on $C_c^\infty(\Omega)$ by considering the family of sets $\{ \phi + W \}$ as a basis, where $\phi$ ranges over all elements of $C_c^\infty(\Omega)$, and $W$ ranges over all convex, balanced subsets of $C_c^\infty(\Omega)$ such that $W \cap C_c^\infty(K)$ is open in $C_c^\infty(K)$ for each $K \subset \Omega$.

\begin{theorem}
    This gives a basis of a Hausdorff topology on $C_c^\infty(\Omega)$.
\end{theorem}
\begin{proof}
    If $\phi_1 + W_1$ and $\phi_2 + W_2$ both contain $\phi$, then $\phi - \phi_1 \in W_1$ and $\phi - \phi_2 \in W_2$. The functions $\phi, \phi_1$, and $\phi_2$ are all supported on some compact set $K$. By continuity of multiplication on $C_c^\infty(K)$, and the fact that $W_n \cap C_c^\infty(K)$ is open, there is a small constant $\delta$ such that $\phi - \phi_n \in (1 - \delta) W_n$ for each $n \in \{ 1, 2 \}$. The convexity of the $W_n$ implies that $\phi - \phi_n + \delta W_n \subset W_n$. But then $\phi + \delta W_n \subset \phi_n + W_n$, and so $\phi + \delta (W_1 \cap W_2) \subset (\phi_1 + W_1) \cap (\phi_2 + W_2)$. Thus we have verified the family of sets specified above is a basis. Now we show $C_c^\infty(\Omega)$ is Hausdorff under this topology. Suppose $\phi$ is in every open neighbourhood of the origin, then in particular, for each $\varepsilon > 0$, $\phi$ lies in the set $W_\varepsilon = \{ f \in C_c^\infty(\Omega): \| f \|_{L^\infty(\Omega)} < \varepsilon \}$, and it is easy to see these sets are open. Since $\bigcap_{\varepsilon > 0} W_\varepsilon = \{ 0 \}$, this means $\phi = 0$.
\end{proof}

\begin{remark}
    This technique can be formulated more abstractly to give a locally convex topological structure to the direct limit of locally convex spaces. From this perspective, we also see why our metrization doesn't work; if $X = \lim X_n$, with each $X_n$ a locally convex metrizable space, then we cannot give $X$ a complete metrizable topology such that each $X_n$ is an embedding and has empty interior in $X$, because this would contradict the Baire category theorem. In particular, this means that the topology we have given to $C_c(\Omega)$ cannot be metrizable, and therefore the space cannot be first countable. Later we will see a more explicit proof of this.
\end{remark}

\begin{theorem}
    $C_c^\infty(\Omega)$ is a locally convex space.
\end{theorem}
\begin{proof}
    Fix $\phi$ and $\psi$, and consider any neighbourhood $W$ of the origin. By convexity, we have $(\phi + W/2) + (\psi + W/2) \subset (\phi + \psi) + W$. This shows addition is continuous. To show multiplication is continuous, fix $\lambda$, $\phi$, and a neighbourhood $W$ of the origin. Then $\phi$ is supported on some compact set $K$, and $W \cap C_c^\infty(K)$ is open, in particular absorbing, so there is $\varepsilon > 0$ such that if $|\alpha| < \varepsilon$, $\alpha \phi \in W/2$. Then if $|\gamma - \lambda| < \varepsilon$, then because $W$ is balanced and convex,
    %
    \begin{align*}
        \gamma \left(\phi + \frac{W}{2(|\lambda| + \varepsilon)} \right) &= \lambda \phi + (\gamma - \lambda) \phi + \frac{\gamma}{2(|\lambda| + \varepsilon)} W\\
        &\subset \lambda \phi + W/2 + W/2 \subset \lambda \phi + W
    \end{align*}
    %
    so multiplication is continuous.
\end{proof}

\begin{theorem}
    For each compact set $K \subset \Omega$, the canonical embedding of $C_c^\infty(K)$ in $C_c^\infty(\Omega)$ is continuous.
\end{theorem}
\begin{proof}
    We shall prove a convex, balanced neighbourhood $V$ is open in $C_c^\infty(\Omega)$ if and only if $C_c^\infty(K) \cap V$ is open in $C_c^\infty(K)$ for each $K$. Since $V$ is open, $V$ is the union of convex, balanced sets $W_\alpha$ with $W_\alpha \cap C_c^\infty(K)$ open in $C_c^\infty(K)$ for each $K$. But then $V \cap C_c^\infty(K) = (\bigcup W_\alpha) \cap C_c^\infty(K)$ is open in $C_c^\infty(K)$. The converse is true by definition of the topology. But this statement means exactly that the map $C_c^\infty(K) \to C_c^\infty(\Omega)$ is an embedding, because it is certainly continuous, and if $W$ is a convex neighbourhood of the origin equal to the set of $\phi$ supported on $K$ with $\| \phi \|_{C^n(K)} \leq \varepsilon$ for some $n$, then the image is the intersection of $C_c^\infty(K)$ with the set of all $\phi$ supported on $\Omega$ satisfying the inequality, which is open. This shows that the map is open onto its image, hence an embedding.
\end{proof}

It is difficult to see from the definition above why the topology is much stronger than the previous one given. We can see this more numerically by introducing the topology in terms of seminorms. The topology we have given $C_c^\infty(\Omega)$ is the same as the locally convex topology introduced by all norms $\| \cdot \|$ on the space which are continuous when restricted to each $C_c^\infty(K)$. As an example, if we choose an increasing family $U_1, U_2, \dots$ of precompact open sets whose closure is contained in $\Omega$, then any compact set $K$ is contained in some $U_N$ for large enough $N$, and for any increasing sequence $\alpha_1, \alpha_2, \dots$ of positive constants and increasing sequence $k_1, k_2, \dots$ of positive integers the norm
%
\[ \| f \| = \min_{\text{supp}(f) \subset U_n} \alpha_n \| f \|_{C^{k_n}(U_n)} \]
%
is well defined on $C_c^\infty(\Omega)$ and continuous. But if $\{ f_i \}$ is a sequence such that $\lim_{i \to \infty} f_i = 0$, then $\lim_{i \to \infty} \| f_i \| = 0$ for any choice of constants $\alpha_n$ and $k_n$. This means that, asymptotically, as we approach the boundary of $\Omega$, the sequence $\{ f_i \}$ must converge arbitrarily rapidly to zero. The next theorem shows that this implies that the union of the domains $f_n$ must actually be precompact. It is this `uniform compactness' that gives us completeness.

\begin{theorem}
    Consider any $E \subset C_c^\infty(\Omega)$. Then $E$ is a bounded subset of $C_c^\infty(\Omega)$ if and only if $E$ is contained in $C_c^\infty(K)$ for some compact set $K$, and there is a sequence of constants $\{ M_n \}$ such that $\| \phi \|_{C^n(\Omega)} \leq M_n$ for all $\phi \in E$.
\end{theorem}
\begin{proof}
    We shall now prove that if $E$ is not contained in some $C_c^\infty(K)$ for any compact set $K \subset \Omega$, then $E$ is not bounded. If our assumption is true, we can find functions $\phi_n \in E$ and a set of points $x_n \in X$ with no limit point such that $\phi_n(x_n) \neq 0$. For each $n$, set
    %
    \[ W_n = \left\{ \psi \in C_c^\infty(\RR^d): |\psi(x_n)| < n^{-1} |\phi_n(x_n)| \right\}. \]
    %
    Certainly $W_n$ is convex and balanced, and for each compact set $K$, if $\psi \in W_n \cap C_c^\infty(K)$, then there is $\varepsilon > 0$ such that $|\psi(x_n)| < n^{-1} |\phi_n(x_n)| - \varepsilon$. Thus if $\eta \in C_c^\infty(K)$ satisfies $\| \eta \|_{L^\infty(\RR^d)} < \varepsilon$, then $\psi + \eta \in W_n$. In particular, this means $W_n \cap C_c^\infty(K)$ is open in $C_c^\infty(K)$ for each $K$, so $W_n$ is open.

    Now we claim $W = \bigcap_{n = 1}^\infty W_n$ is open. Certainly this set is convex and balanced. Moreover, each compact set $K$ contains finitely many of the points $\{ x_n \}$, so $W \cap C_c^\infty(K)$ can be replaced by a finite intersection of the $W_n$, and is therefore open. Since $\phi_n \not \in nW$ for all $n$, this implies that $E$ is not bounded. The fact that $\| \cdot \|_{C^n(\Omega)}$ specifies the topological structure of $C_c^\infty(K)$ for each compact $K$ now shows that if $E$ is bounded, there exists constants $\{ M_n \}$ such that $\| \phi \|_{C^n(\Omega)} \leq M_n$ for all $\phi \in E$. The converse property follows because $C_c^\infty(K)$ is embedded in $C_c^\infty(\Omega)$.
\end{proof}

\begin{corollary}
    $C_c^\infty(\Omega)$ has the Heine Borel property.
\end{corollary}
\begin{proof}
    This follows because if $E$ is bounded and closed, it is a closed and bounded subset of some $C_c^\infty(K)$ for some $K$, hence $E$ is compact since $C_c^\infty(K)$ satisfies the Heine-Borel property (this can be proved by a technical application of the Arzela-Ascoli theorem).
\end{proof}

\begin{corollary}
    $C_c^\infty(\Omega)$ is quasicomplete.
\end{corollary}
\begin{proof}
    If $\phi_1, \phi_2, \dots$ is a Cauchy sequence in $C_c^\infty(\Omega)$, then the sequence is bounded, hence contained in some common $C_c^\infty(K)$. Since the sequence is Cauchy, they converge in $C_c^\infty(K)$ to some $\phi$, since $C_c^\infty(K)$ is complete, and thus the $\phi_n$ converge to $\phi$ in $C_c^\infty(\Omega)$.
\end{proof}

It is often useful to use the fact that we can perform a `separation of variables' to a smooth function. This is done formally in the following manner. Say $f \in C_c^\infty(\RR^d)$ is a {\it tensor function} if there are $f_1, \dots, f_n \in C_c^\infty(\RR)$ such that $f(x) = f_1(x_1) \dots f_n(x_n)$. We write $f = f_1 \otimes \dots \otimes f_n$. Since the product of two tensor functions is a tensor function, the family of all finite sums of tensor functions forms an algebra.

\begin{theorem}
    Finite sums of tensor functions are dense in $C_c^\infty(\RR^d)$.
\end{theorem}
\begin{proof}
    Recall from the theory of multiple Fourier series that if $f \in C^\infty(\RR^d)$ is $N$ periodic, in the sense that $f(x + n) = f(x)$ for all $x \in \RR^d$ and $n \in (N \ZZ)^d$, then there are coefficients $a_m$ for each $m \in \ZZ^n$ such that $f = \lim_{M \to \infty} S_M f$, where the convergence is dominated by the sminorms $\| \cdot \|_{C^n(\RR^d)}$, for all $n > 0$, and
    %
    \[ (S_M f)(x) = \sum_{\substack{m \in \ZZ^d\\|m| \leq M}} a_m e^{\frac{2 \pi i m \cdot x}{N}}. \]
    %
    Note that since
    %
    \[ e^{\frac{2 \pi i m \cdot x}{N}} = \prod_{k = 1}^d e^{2 \pi i m_ix_i/N} \]
    %
    is a tensor product, $S_M f$ is a finite sum of tensor functions. If $\phi \in C_c^\infty(\RR^d)$ is compactly supported on $[-N,N]^d$, we let $f$ be a $10N$ periodic function which is equal to $\phi$ on $[-N,N]^d$. We then find coefficients $\{ a_m \}$ such that $S_M f$ converges to $f$. If $\psi: \RR \to \RR$ is a compactly supported bump function equal to one on $[-N,N]^d$, and vanishing outside of $[-2N,2N]^d$, then $\psi^{\otimes d} S_M f$ converges to $\psi$ as $M \to \infty$, and each is a finite sum of tensor functions.
\end{proof}

Because $C_c^\infty(\Omega)$ is the limit of metrizable spaces, it's linear operators still have many of the same properties as metrizable spaces.

\begin{theorem}
    If $T: C_c^\infty(\Omega) \to X$ is a map from $C_c^\infty(\Omega)$ to some locally convex space $X$, then the following are equivalent:
    %
    \begin{itemize}
        \item[(1)] $T$ is continuous.
        \item[(2)] $T$ is bounded.
        \item[(3)] If $\{ \phi_n \}$ converges to zero, then $\{ T\phi_n \}$ converges to zero.
        \item[(4)] For each compact set $K \subset \Omega$, $T$ is continuous restricted to $C_c^\infty(K)$.
    \end{itemize}
\end{theorem}
\begin{proof}
    We already known that (1) implies (2). If $T$ is bounded, and we have a sequence $\{ \phi_n \}$ converging to zero, then the sequence is bounded, hence contained in some $C_c^\infty(K)$. Then $T$ is bounded as a map from $C_c^\infty(K)$ to $X$, hence $\{ T\phi_n \} \to 0$. (3) implies (4) because each $C_c^\infty(K)$ is metrizable, and any convergent sequence is contained in some common $C_c^\infty(K)$. To prove that (4) implies (1), we let $V$ be a convex, balanced, open subset of $X$. Then $T^{-1}(V) \cap C_c^\infty(K)$ is open for each $K$, and $T^{-1}(V)$ is convex and balanced, so $T^{-1}(V)$ is an open set.
\end{proof}

Because convergence is so strict in $C_c^\infty(\Omega)$, almost every operation we want to perform on smooth functions is continuous in this space.
%
\begin{itemize}
    \item Since $f \mapsto D^\alpha f$ is a continuous operator from $C_c^\infty(K)$ to itself, it is therefore continuous on the entire space $C_c^\infty(\Omega)$. More generally, any linear differential operator with coefficients in $C_c^\infty(\Omega)$ is a continuous operator from $C_c^\infty(\Omega)$ to itself.

    \item The inclusion $C_c^\infty(\Omega) \to L^p(\Omega)$ is continuous. To prove this, it suffices to prove for each compact $K$, the inclusion $C_c^\infty(K) \to L^p(\Omega)$ is continuous, and this follows because $\| f \|_{L^p(\Omega)} \leq |K|^{1/p} \| f \|_\infty$.

    \item If $f \in L^1(\RR^d)$ is compactly supported, then for any $g \in C_c^\infty(\RR^d)$, $f * g \in C_c^\infty(\RR^d)$. This is because $f * g$ is continuous since $g \in L^\infty(\RR^n)$, and it's support is contained in the algebraic sums of the support of $f$ and $g$, as well as the identity $D^\alpha(f * g) = f * (D^\alpha g)$. In fact, the map $g \mapsto f * g$ is a continuous operator on $C_c^\infty(\RR^n)$. This is because if we restrict our attention to $C_c^\infty(K)$, and $f$ has supported on $K'$, then our convolution operator maps into the compact set $K+K'$, and since
    %
    \[ \| D^\alpha (g * f) \|_{L^\infty(K + K')} = \| D^\alpha g * f \|_{L^\infty(K + K')} \leq \| D^\alpha g \|_{L^\infty(K)} \| f \|_{L^1(K')}, \]
    %
    we conclude
    %
    \[ \| g * f \|_{C^n(K+K')} \leq \| g \|_{C^n(K)} \| f \|_{L^1(K')}, \]
    %
    which gives continuity of the operator as a map from $C_c^\infty(K)$ to $C_c^\infty(K+K')$. Since the latter space embeds in $C_c^\infty(\RR^n)$, we obtain continuity of the operator on $C_c^\infty(\RR^n)$.
\end{itemize}

\begin{theorem}
    If a map $T: C_c^\infty(K_0) \to C_c^\infty(\RR^n)$ is continuous, then the image of $C_c^\infty(K_0)$ is actually $C_c^\infty(K_1)$ for some compact set $K_1$.
\end{theorem}
\begin{proof}
    Suppose there is a sequence $\{ x_i \}$ in $\RR^d$ with no limit point and smooth functions $\{ \phi_i \}$ compactly supported on $C_c^\infty(K_0)$ such that
    %
    \[ (T\phi_i)(x_i) \neq 0. \]
    %
    Then for any sequence $\{ \alpha_i \}$ of positive scalars, the sequence $\{ \alpha_i T\phi_i \}$ does not converge to zero, since the union of the supports of $\alpha_i T\phi_i$ is unbounded. This means $\alpha_i \phi_i$ does not converge to zero. But this is clearly not true, for if we let
    %
    \[ \alpha_i = \frac{1}{2^i \| \phi_i \|_{C^i(\RR^d)}}, \]
    %
    then for any fixed $n$, $\lim_{i \to \infty} \| \alpha_i \phi_i \|_{C^n(\RR^d)} = 0$, so the sequence $\{ \alpha_i \phi_i \}$ converges to zero. Thus there cannot exist a sequence $\{ x_i \}$, and so the union of the supports of $T(C_c^\infty(K_0))$ is supported on some compact set $K_1$.
\end{proof}

Thus the topology on the space $C_c^\infty(\RR^d)$ is as strict as can be. As a consequence, we shall see that the weak-$*$ topology on $C_c^\infty(\RR^d)^*$ is essentially the weakest topology available in analysis. This is surprising, because we are still able to obtain the continuity of many operators in the dual space to $C_c^\infty(\RR^d)$.

\section{The Space of Distributions}

We now have the tools to explain the idea of a distribution. If $f$ is a locally integrable function defined on $\Omega$, then the linear functional $\Lambda[f]$ on $C_c^\infty(\Omega)$ defined for each $\phi \in C_c^\infty(\Omega)$ by setting
%
\[ \Lambda[f](\phi) = \int f(x) \phi(x)\; dx \]
%
is continuous. Moreover, $\Lambda[f]$ determines $f$ up to a set of measure zero, and so we can safely identify $f$ with $\Lambda[f]$ (this is the `distributional viewpoint' of $f$). The idea of the theory of distributions is to treat any continuous linear functional $\Lambda$ on $C_c^\infty(\Omega)$ as if it were given by integration against a function. Using the properties of integration for these integration, we can usually cheat out a definition of operations for general distributions. Thus the operations of analysis generalize to an incredibly large family of objects. As an example, if $f \in C^1(\RR)$, then for any $\phi \in C_c^\infty(\RR)$, we would find
%
\[ \int_{-\infty}^\infty f'(x) \phi(x)\; dx = - \int_{-\infty}^\infty f(x) \phi'(x)\; dx \]
%
Since the right hand side is defined independantly of how nice the function $f(x)$ is, we could define the {\it derivative} of a continuous linear functional $\Lambda$ as
%
\[ \Lambda'(\phi) = - \Lambda(\phi') \]
%
and more generally, for a linear functional on $n$ dimensional space, we could define $(D^\alpha \Lambda)(\phi) = (-1)^{|\alpha|} \Lambda(D^\alpha \phi)$.

\begin{example}
    Let $H(x) = \mathbf{I}(x > 0)$ denote the {\it Heaviside step function}. Then $H$ is locally integrable, and so for any test function $\phi$, we calculate
    %
    \[ \int_{-\infty}^\infty H'(x) \phi(x)\; dx = - \int_{-\infty}^\infty H(x) \phi'(x) = - \int_0^\infty \phi'(x) = \phi(0) \]
    %
    Thus the \emph{distributional derivative} of the Heaviside step function is the Dirac delta function. It is not a function, but if we were to think of it as a `generalized function', it would be zero everywhere except at the origin, where it is infinitely peaked.
\end{example}

\begin{example}
    Consider the Dirac delta function at the origin, which is the distribution $\delta$ such that for any $\phi \in C_c^\infty(\RR)$,
    %
    \[ \int_{-\infty}^\infty \delta(x) \phi(x)\; dx = \phi(0). \]
    %
    Then
    %
    \[ \int_{-\infty}^\infty \delta'(x) \phi(x)\; dx = - \int_{\RR^d} \delta(x) \phi'(x)\; dx = - \phi'(0). \]
    %
    This is a distribution that does not arise from integration with respect to a locally integrable function nor integration against a measure.
\end{example}

In general, we define a \emph{distribution} to be a continuous linear functional on the space of test functions $C_c^\infty(\Omega)$. In the last section, our exploration of continuous linear transformations on $C_c^\infty(\Omega)$ guarantees that a linear functional $\Lambda$ on $C_c^\infty(\Omega)$ is continuous if and only if for every compact $K \subset X$ there is an integer $n_k$ such that $|\Lambda \phi| \lesssim_K \| \phi \|_{C^{n_k}(K)}$ for $\phi \in C_c^\infty(K)$. If one integer $n$ works for all $K$, and $n$ is the smallest integer with such a property, we say that $\Lambda$ is a distribution of \emph{order $n$}. If such an $n$ doesn't exist, we say the distribution has infinite order. If such an $n$ doesn't exist, we say the distribution has infinite order.

\begin{example}
    If $\mu$ is a locally finite Borel measure, or a finite complex valued measure, then we can define a distribution $\Lambda[\mu]$ such that for each $\phi \in C_c^\infty(\RR^d)$.
    %
    \[ \Lambda[\mu](\phi) = \int_{\RR^d} \phi(x) d\mu(x) \]
    %
    Thus $\Lambda[\mu]$ is a distribution, since if $\phi$ is supported on $K$, then
    %
    \[ |\Lambda[\mu](\phi)| \leq \mu(K) \| \phi \|_{L^\infty(K)}. \]
    %
    Thus $\Lambda[\mu]$ is a distribution of order zero.
\end{example}

\begin{example}
    Not all distributions arise from functions or measures. For instance, consider a functional $\Lambda$ defined such that for any $\phi \in C_c^\infty(\RR)$ vanishing in a neighbourhood of the origin,
    %
    \[ \Lambda(\phi) = \int_{-\infty}^\infty \frac{\phi(x)}{x}\; dx. \]
    %
    Such functions are dense in $C_c^\infty(\RR)$. We claim $\Lambda$ extends to a continuous functional on the entirety of $C_c^\infty(\RR)$. To prove this, fix $\phi \in C_c^\infty[-N,N]$ vanishing on a neighbourhood $(-\varepsilon,\varepsilon)$ of the origin. Then
    %
    \[ |\Lambda \phi| = \left| \int_{-\infty}^\infty \frac{\phi(x)}{x}\; dx \right| = \left| \int_{\varepsilon \leq |x| \leq N} \frac{\phi(x) - \phi(0)}{x}\; dx \right|. \]
    %
    Applying the mean-value theorem, we find
    %
    \[ |\Lambda \phi| \leq N \| \phi \|_{C^1[-N,N]}. \]
    %
    Since $N$ was arbitrary, it follows that $\Lambda$ is continuous in the topology induced by that of $C_c^\infty(\RR)$, and thus extends uniquely to a distribution on the entirety of $C_c^\infty(\RR)$. To be precise, we often denote the application of $\Lambda$ to $C_c^\infty(\RR)$ as
    %
    \[ \text{p.v} \int_{-\infty}^\infty \frac{\phi(x)}{x}\; dx. \]
    %
    A simple approximation argument shows that for any $\phi \in C_c^\infty(\RR)$,
    %
    \[ \Lambda \phi = \lim_{\varepsilon \to 0} \int_{|x| \geq \varepsilon} \frac{\phi(x)}{x}\; dx. \]
    %
    The distribution can also be described as the distribution derivative of the locally integrable function $\log |x|$, since an integration by parts shows that for each $\phi \in C_c^\infty(\RR^d)$,
    %
    \begin{align*}
        \int (\log |x|)'\; \phi(x)\; dx &= - \int \log |x| \phi'(x)\; dx\\
        &= \lim_{\varepsilon \to 0} \int_{|x| \geq \varepsilon} \log |x| \phi'(x)\\
        &= \lim_{\varepsilon \to 0} \left( \log(\varepsilon) \cdot \left( \phi(x) - \phi(-x) \right) + \int_{|x| \geq \varepsilon} \frac{\phi(x)}{x} \right)\\
        &= \text{p.v.} \int \frac{\phi(x)}{x}\; dx.
    \end{align*}
    %
    This distribution arises most prominantly in the theory of the Hilbert transform.
\end{example}

As we stated before, given any distribution $\Lambda$, we can define it's {\it derivative} $D^\alpha \Lambda$ to be the distribution
%
\[ D^\alpha \Lambda (\phi) = (-1)^{|\alpha|} \Lambda(D^\alpha \phi) \]
%
which is continuous since the derivative operation is continuous on $C_c^\infty(\Omega)$. Just as the partial derivatives commutes on $C_c^\infty(\Omega)$, the partial differentiation operation commutes on the the space of distributions, i.e. $D^\alpha D^\beta \Lambda = D^\beta D^\alpha \Lambda$, and we take the common value to be $D^{\alpha + \beta} \Lambda$. If $D^\alpha f$ is continuous, then we already know an integration by parts gives $D^\alpha \Lambda[f] = \Lambda[D^\alpha f]$, so we can think of the distributional derivative as a true generalization of the usual derivative. On the other hand, in general the distribution derivative may disagree with the usual derivative if the function is less well behaved (as might be expected, given that the distributional derivative always commutes). More generally, if $P$ is a polynomial, we have
%
\[ P(D)(\Lambda)(\phi) = \Lambda(P(-D)(\phi)) \]
%
if we understand the polynomial applications of derivatives linearly.

\begin{example}
    Let $f$ be a left continuous function on the real line with bounded variation and with $f(-\infty) = 0$. Then $f'$ exists almost everywhere in the classical sense, and $f' \in L^1(\RR)$. By Fubini's theorem, if we let $\mu$ be the measure defined by $\mu([a,b)) = f(b) - f(a)$, then for any $\phi \in C_c^\infty(\RR)$,
    %
    \begin{align*}
        \int_{-\infty}^\infty \phi(x) d\mu(x) &= - \int_{-\infty}^\infty \int_x^\infty \phi'(y)\; dy\; d\mu(x)\\
        &= - \int_{-\infty}^\infty \phi'(y) \int_{-\infty}^y d\mu(x)\; dy\\
        &= - \int_{-\infty}^\infty \phi'(y) f(y) dy
    \end{align*}
    %
    and we know $f(-\infty) = 0$. Thus we find $\smash{\Lambda[f'] = \Lambda[\mu]}$. In particular, we only have $\smash{\Lambda[f]' = \Lambda[f'}]$ if $\smash{f' dx = \mu}$, which only holds if $f$ is absolutely continuous.
\end{example}

\begin{theorem}
  If $u \in C_c^\infty(\RR^d)'$ and $D^i u = 0$, then there exists $v \in C_c^\infty(\RR^d)'$ such that
  %
  \[ \int_{\RR^d} u(x) \phi(x)\; dx = \int_{\RR^{d-1}} v(x) \left( \int_{-\infty}^\infty \phi(x)\; dx^i \right)\; dx. \]
\end{theorem}
\begin{proof}
  Assume $i = d$ without loss of generality. Fix $\phi \in C_c^\infty(\RR^d)$ with
  %
  \[ \int \phi(x_1,\dots,x_{d-1})\; dx^1 \dots dx^{d-1} = 1. \]
  %
  for each $x_d$. Then for any $\psi \in C_c^\infty(\RR^d)$. Then if
  %
  \[ a(t) = \int \psi(x_1,\dots,x_{d-1})\; dx \],
\end{proof}

If $f \in L^1_{\text{loc}}(\RR^d)$, and $g \in C^\infty(\RR^d)$, then $fg$ is locally integrable. The identity
%
\[ \int (f(x)g(x)) \phi(x)\; dx = \int f(x) (g(x) \phi(x))\; dx \]
%
enables us to define the product of a $C^\infty(\Omega)$ function with a distribution. Given any distribution $\Lambda$ on $\Omega$ and $f \in C^\infty(\Omega)$, we define $(f \Lambda)(\phi) = \Lambda(f \phi)$. To see why $f \Lambda$ is a distribution, fix a compact set $K \subset \Omega$, and pick $A$ and $n$ such that for any $\phi \in C_c^\infty(K)$, $|\Lambda(f)| \leq A \| f \|_{C^n(K)}$. The Leibnitz rule tells us that
%
\[ D^\alpha(f \phi) = \sum_{\lambda + \gamma = \alpha} C_{\lambda \gamma} D^\lambda f D^\gamma \phi \]
%
for some constants $C_{\lambda \gamma} > 0$, and so
%
\begin{align*}
  |\Lambda(f \phi)| &\lesssim \| f \phi \|_{C^n(K)}\\
  &\lesssim_n \max_{|\alpha| \leq n} \max_{\lambda + \gamma = \alpha} \| D^\lambda f \|_{L^\infty(K)} \| D^\gamma \phi \|_{L^\infty(K)}\\
  &\leq \| f \|_{C^n(K)} \| \phi \|_{C^n(K)},
\end{align*}
%
which completes the argument.

Since $C_c^\infty(X)^*$ is the dual space of a topological vector space, we can give it a natural topology, the weak $*$ topology. Thus a net of distributions $\Lambda_\alpha$ converges to $\Lambda$ if and only if $\Lambda_\alpha(\phi) \to \Lambda(\phi)$ for all test functions $\phi$. This gives a further topology on the space of measures and functions, and we often write $f_\alpha \to f$ `in the distribution sense' if we have a convergence $\Lambda[f_\alpha] \to \Lambda[f]$ for the corresponding distributions. Since the convergence in $C_c^\infty(\Omega)$ is incredibly strict, convergence of distributions is incredibly weak. The following is thus quite a surprising result.

\begin{theorem}
  Suppose that $\{ \Lambda_i \}$ are a sequence of distributions converging weakly to a distribution $\Lambda$. Then $D^\alpha \Lambda_i$ converges weakly to $D^\alpha \Lambda$ for any multi-index $\alpha$.
\end{theorem}
\begin{proof}
  For each $\phi \in C_c^\infty(\Omega)$, $D^\alpha \phi \in C_c^\infty(\Omega)$, so
  %
  \begin{align*}
    \lim_{i \to \infty} (D^\alpha \Lambda_i)(\phi) &= \lim_{i \to \infty} (-1)^{|\alpha|} \Lambda_i(D^\alpha \phi)\\
    &= (-1)^{|\alpha|} \Lambda(D^\alpha \phi)\\
    &= (D^\alpha \Lambda)(\phi). \qedhere
  \end{align*}
\end{proof}

Thus differentiation is continuous in the space of distributions. So too is multiplication by elements of $C^\infty(\RR^d)$, which turns the space of distributions into a $C^\infty(\RR^d)$ module.

\begin{theorem}
  Fix a sequence $\{ g_i \}$ in $C^\infty(\RR^d)$ and a sequence of distributions $\{ \Lambda_i \}$ such that $g_i \to g$ in $C^\infty(\RR^d)$ and $\Lambda_i \to \Lambda$ weakly. Then $g_i \Lambda_i$ converges weakly to $g \Lambda$.
\end{theorem}
\begin{proof}
  For each $\phi \in C_c^\infty(\RR^d)$, the map $(\Lambda_i \times g_i) \mapsto \Lambda_i g_i$ is bilinear, and continuous in each variable. The result then follows from a variant of Banach-Steinhaus.
\end{proof}

\section{Localization of Distribuitions}

Just as we can consider the local behaviour of functions around a point, we can consider the local behaviour of a distribution around points, and this local behaviour contains most of the information of the distribution. For instance, given an open subset $U$ of $X$, we say two distributions $\Lambda$ and $\Psi$ are equal on $U$ if $\Lambda \phi = \Psi \phi$ for every test function $\phi$ compactly supported in $U$. We recall the notion of a partition of unity, which, for each open cover $U_\alpha$ of Euclidean space, gives a family of $C^\infty$ functions $\psi_\alpha$ which are positive, {\it locally finite}, in the sense that only finitely many functions are positive on each compact set, and satisfy $\sum \psi_\alpha = 1$ on the union of the $U_\alpha$.

\begin{theorem}
    If $X$ is covered by a family of open sets $U_\alpha$, and $\Lambda$ and $\Psi$ are locally equal on each $U_\alpha$, then $\Lambda = \Psi$. If we have a family of distributions $\Lambda_\alpha$ which agree with one another on $U_\alpha \cap U_\beta$, then there is a unique distribution $\Lambda$ locally equal to each $\Lambda_\alpha$.
\end{theorem}
\begin{proof}
    Since we can find a $C^\infty$ partition of unity $\psi_\alpha$ compactly supported on the $U_\alpha$, upon which we find if $\phi$ is supported on $K$, then finitely many of the $\psi_\alpha$ are non-zero on $K$, and so
    %
    \[ \Lambda(\phi) = \sum \Lambda(\psi_\alpha \phi) = \sum \Psi(\psi_\alpha \phi) = \Psi(\phi) \]
    %
    Thus $\Lambda = \Psi$. Conversely, if we have a family of distributions $\Lambda_\alpha$ like in the hypothesis, then we can find a partition of unity $\psi_{\alpha \beta}$ subordinate to $U_\alpha \cap U_\beta$, and we can define
    %
    \[ \Lambda(\phi) = \sum \Lambda_\alpha(\psi_{\alpha \beta} \phi) = \sum \Lambda_\beta(\psi_{\alpha \beta} \phi) \]
    %
    The continuity is verified by fixing a compact $K$, from which there are only finitely many nonzero $\psi_{\alpha \beta}$ on $K$, and the fact that this definition is independant of the partition of unity follows from the first part of the theorem.
\end{proof}

In the language of modern commutative algebra, the association of $C_c^\infty(U)^*$ to each open subset $U$ of $\Omega$ gives a sheaf structure to $\Omega$. Given a distribution $\Lambda$, we might have $\Lambda(\phi) = 0$ for every $\phi$ supported on some open set $U$. The complement of the largest open set $U$ for which this is true is called the \emph{support} of $\Lambda$.

\begin{theorem}
    If a distribution has compact support, the distribution has finite order, and extends uniquely to a continuous linear functional on $C^\infty(X)$.
\end{theorem}
\begin{proof}
    Let $\Lambda$ be a distribution supported on a compact set. If $\psi$ is a function with compact support with $\psi(x) = 1$ on the support of $\Lambda$, then $\psi \Lambda = \Lambda$, because for any $\phi$, $\phi - \phi \psi$ is supported on a set disjoint from the support of $\Lambda$. But if $\psi$ is supported on $K$, then there is $N$ such that for any $\phi \in C_c^\infty(K)$,
    %
    \[ |\Lambda(\phi)| \lesssim \| \phi \|_{N,K} \]
    %
    and so for any other compact set $K$,
    %
    \[ |\Lambda(\phi)| = |\Lambda(\phi \psi)| \lesssim \| \phi \psi \|_{N,K} \lesssim \| \psi \|_{C^N(K)} \| \phi \|_{C^N(K)} \]
    %
    which shows $\Lambda$ has order $N$. We have shown that $\Lambda$ is continuous with respect to the seminorm $\| \cdot \|_{C^N(K)}$ on $C^\infty(X)$, and so by the Hahn Banach theorem, $\Lambda$ extends uniquely to a continuous functional on $C^\infty(X)$.
\end{proof}

\begin{example}
    If $\Lambda(\phi) = \sum_{|\alpha| \leq N} \lambda_\alpha D^\alpha \phi(x)$, then $\Lambda$ is supported on $x$. Conversely, every distribution $\Lambda$ supported on $x$ is of this form. We know $\Lambda$ must have finite order $N$, and consider $\phi$ with $D^\alpha \phi(x) = 0$ for all $|\alpha| \leq N$. We claim $\Lambda(\phi) = 0$. Fix $\varepsilon > 0$, and choose a compact neighbourhood $K$ of the origin with $|D^\alpha \phi(x)| < \varepsilon$ on $K$ for all $|\alpha| = N$. Then for $|\alpha| < N$, the mean value theorem implies that, by induction,
    %
    \[ |D^\alpha \phi(x)| \leq \varepsilon n^{N - |\alpha|} |x|^{N-|\alpha|} \]
    %
    Find $A$ such that for functions $\phi$ supported on $K$,
    %
    \[ |\Lambda(\phi)| \leq A \| \phi \|_{C^N(K)} \]
    %
    Fix a bump function $\psi$ with support on the ball of radius one and $\psi(x) = 1$ in a neighbourhood of the origin, and define $\psi_\delta(x) = \psi(x/\delta)$. If $\delta$ is small enough, then $\psi$ is supported on $K$, and because $\Lambda$ is supported on $x$,
    %
    \begin{align*}
        |\Lambda(\phi)| &= |\Lambda(\phi \psi_\delta)| \leq A \| \phi \psi_\delta \|_{C^N(K)}\\
        &\leq A \sum_{|\alpha + \beta| = N} |c_{\alpha \beta}| \| D^\alpha \phi \|_\infty \| D^\beta \psi_\delta \|\\
        &\leq A \| \psi \|_{C^N} \sum_{|\alpha + \beta| = N} |c_{\alpha \beta}| \delta^{|\beta| - |\alpha|} \| D^\beta \phi \|_{L^\infty(K)}\\
        &\leq \varepsilon A \left( \sum_{|\alpha + \beta| = N} |c_{\alpha \beta}| n^{N - |\beta|} \right)
    \end{align*}
    %
    We can then let $\varepsilon \to 0$ to conclude $\Lambda(\phi) = 0$. But this means that $\Lambda(\phi)$ is a linear function of the partial derivatives of $\phi$ with order $\leq N$, completing the proof.
\end{example}

\begin{example}
    If $\delta$ is the Dirac delta distribution in $\RR^d$, then $f \delta = f(0) \delta$ for any $f \in C^\infty(\RR^d)$. Thus, in particular, $x \delta = 0$. Conversely, if $\Lambda$ is any distribution with $x \Lambda = 0$, then $\Lambda$ is a multiple of the Dirac delta distribution. To see this, we note that this would imply $\Lambda(f) = 0$ for all functions $f$ such that $f/x$ is also smooth and compactly supported. In particular, this is true if the support of $f$ does not contain the origin. Thus $\Lambda$ is supported on the origin, hence there are constants $a_n$ such that
    %
    \[ \Lambda f = \sum_{n = 0}^N a_n f^{(n)}(0) \]
    %
    But $(xf)^{(n)}(0) = n f^{(n-1)}(0)$ only vanishes for all $f$ when $n = 0$, so $\Lambda$ is a multiple of the Dirac delta distribution. A more simple way to see this is that if $f$ is compactly supported on $[-N,N]$, the function
    %
    \[ g(x) = \frac{f(x) - f(0)}{x} = \int_0^1 f'(tx)\; dt \]
    %
    is smooth, and $f = f(0) + xg$. Since $\Lambda$ and $x \Lambda$ have bounded support, they extend uniquely to $C^\infty(\Omega)$, and so $\Lambda f = f(0) \Lambda 1 + \Lambda(xg) = f(0) \Lambda 1$.
\end{example}

In many other ways, distributions act like functions. For instance, any distribution $\Lambda$ can be uniquely written as $\Lambda_1 + i \Lambda_2$ for two distributions $\Lambda_1, \Lambda_2$ that are real valued for any real-valued smooth continuous function. However, we cannot write a real-valued distribution as the difference of two positive distributions, i.e. those which are non-negative when evaluated at any non-negative functional. Given a non-negative functional $\Lambda$ (which is automatically continuous),  we define $\Lambda f$ for a compactly supported continuous function $f \geq 0$ as
%
\[ \Lambda f = \sup \{ \Lambda g: g \in C_c^\infty(\RR^n), g \leq f \} \]
%
and then in general define $\Lambda (f^+ - f^-) = \Lambda f^+ - \Lambda f^-$. Then $\Lambda$ is obviously a positive extension of $\Lambda$ to all continuous functions, and is linear. But then the Riesz representation theorem implies that there is a Radon measure such that $\Lambda = \Lambda_\mu$, completing the proof.

\section{Derivatives of Continuous Functions}

One of the main reasons to consider the theory of distributions is so that we can take the derivative of any function we want. We now show that, at least locally, every distribution is the derivative of some continuous function, which means the theory of distributions is essentially the minimal such class of objects which enable us to take derivatives of continuous functions.

\begin{theorem}
    If $\Lambda$ is a distribution on $\Omega$, and $K$ is a compact set, then there is a continuous function $f$ and $\alpha$ such that for every $\phi$,
    %
    \[ \Lambda \phi = (-1)^{|\alpha|} \int_\Omega f(x) (D^\alpha \phi)(x)\; dx \]
\end{theorem}
\begin{proof}
    TODO
\end{proof}

\begin{theorem}
    If $K$ is compact, contained in some open subset $V$, which in turn is a subset of $\Omega$, and $\Lambda$ has order $N$, then there exists finitely many continuous functions $f_\beta \in C(\Omega)$ supported on $V$, for each $|\beta| \leq N + 2$, with supports on $V$, and with $\Lambda = \sum D^\beta f_\beta$.
\end{theorem}

\begin{theorem}
    If $\Lambda$ is a distribution on $\Omega$, then there exists continuous functions $g_\alpha$ on $\Omega$ such that each compact set $K$ intersects the supports of finitely many of the $g_\alpha$, and $\Lambda = \sum D^\alpha g_\alpha$. If $\Lambda$ has finite order, then only finitely many of the $g_\alpha$ are nonzero.
\end{theorem}

\section{Convolutions of Distributions}

Using the convolution of two functions as inspiration, we will not define the convolution of a distribution $\Lambda$ with a test function $\phi$, and under certain conditions, the convolution of two distributions. Recall that if $f,g \in L^1(\RR^n)$, then their convolution is the function in $L^1(\RR^n)$ defined by
%
\[ (f * g)(x) = \int f(y) g(x - y)\; dy \]
%
If we define the translation operators $(T_y g)(x) = g(x-y)$, then $(f * g)(x) = \int f(y) (T_x g^*)(y)\; dy$, where $g^*$ is the function defined by $g^*(x) = g(-x)$. Thus, if $\Lambda$ is any distribution on $\RR^n$, and $\phi$ is a test function on $\RR^n$, we can define a function $\Lambda * \phi$ by setting $(\Lambda * \phi)(x) = \Lambda(T_x \phi^*)$. Notice that since
%
\begin{align*}
    \int (T_x f)(y) g(y)\; dy &= \int f(y-x) g(y)\; dy = \int f(y) g(x+y)\; dy\\
    &= \int f(y) (T_{-x}g)(y)\; dy,
\end{align*}
%
so we can also define the translation operators on distributions by setting $(T_x \Lambda)(\phi) = \Lambda (T_{-x} \phi)$. One mechanically verifies that convolution commutes with translations, i.e. $T_x (\Lambda * \phi) = (T_x \Lambda) * \phi = \Lambda * (T_x \phi)$.

\begin{theorem}
    $\Lambda * \phi$ is $C^\infty$, and $D^\alpha(\Lambda * \phi) = (D^\alpha \Lambda) * \phi = \Lambda * (D^\alpha \phi)$.
\end{theorem}
\begin{proof}
    It is easy to calculate that
    %
    \begin{align*}
        (D^\alpha \Lambda * \phi)(x) &= (D^\alpha \Lambda)(\phi^*_x) = (-1)^{|\alpha|} \Lambda(D^\alpha (T_x \phi^*))\\
        &= \Lambda(T_x (D^\alpha \phi)^*) = (\Lambda * D^\alpha \phi)(x)
    \end{align*}
    %
    If $k \in \{ 1, \dots, d \}$ and $h \in \RR$, we set
    %
    \[ (\Delta_h f)(x) = \frac{f(x + he_k) - f(x)}{h} \]
    %
    then $\Delta_h \phi$ converges to $D^k \phi$ in $C_c^\infty(\RR^d)$, and as such
    %
    \begin{align*}
      \Delta_h(\Lambda * \phi)(x) &= \frac{(\Lambda * \phi)(x + he_k) - (\Lambda * \phi)(x)}{ h}\\
      &= \Lambda \left( \frac{T_{-x - he_k} \phi^* - T_{-x} \phi^*}{h} \right)
    \end{align*}
    %
    As $h \to 0$, in $C_c^\infty(\RR^d)$ we have
    %
    \[ \frac{T_{-x - he_k} \phi^* - T_{-x} \phi^*}{h} \to - T_{-x} D_k \phi^* = T_{-x} (D_k \phi)^*. \]
    %
    Thus, by continuity,
    %
    \[ \lim_{h \to 0} \Delta_h(\Lambda * \phi)(x) = \Lambda(T_{-x} (D_k \phi)^*) = (\Lambda * D_k \phi)(x) \]
    %
    Iteration gives the general result that $\Lambda * \phi \in C^\infty(\RR^d)$. An easy calculation then shows that for each $x \in \RR^d$,
    %
    \begin{align*}
      [(D^\alpha \Lambda) * \phi](x) &= (D^\alpha \Lambda)(T_{-x} \phi^*)\\
      &= (-1)^{|\alpha|} \Lambda(T_{-x} D^\alpha \phi^*)\\
      &= \Lambda(T_{-x} (D^\alpha \phi)^*)\\
      &= (\Lambda * D^\alpha \phi)(x). \qedhere
    \end{align*}
\end{proof}

There is a certain duality going on here. Distributions can be viewed as linear functionals on $C_c^\infty(\RR^d)$, but one can also view them as a certain family of linear operators from $C_c^\infty(\RR^d) \to C^\infty(\RR^d)$ , and the convolution operator uniquely represents the distribuition. In fact, any such operator that is translation invariant and continuous can be represented as convolution by a distribution.

\begin{theorem}
  Let $T: C_c^\infty(\RR^d) \to C^\infty(\RR^d)$ be a translation invariant continuous operator. Then there exists a distribution $\Lambda$ such that $T\phi = \Lambda * \phi$ for all $\phi \in C_c^\infty(\RR^d)$.
\end{theorem}
\begin{proof}
  If we knew $T\phi = \Lambda * \phi$ for some $\Lambda$, then we could recover $\Lambda$ since
  %
  \[ \int \Lambda(x) \phi(x)\; dx = T \tilde{\phi}(0). \]
  %
  Since $T$ is a continuous operator, the right hand side defines a distribution $\Lambda$, and translation invariance allows us to conclude that $T\phi = \Lambda * \phi$ for all $\phi \in C_c^\infty(\RR^d)$.
\end{proof}

For more general operators that are translation invariant, we cannot represent all operators via convolution by distributions. A significantly more general family of operators can be found if, instead of considering operators of the form
%
\[ T\phi(y) = \int \Lambda(y - x) \phi(x)\; dx \]
%
we instead study \emph{kernel} operators
%
\[ T\phi(y) = \int K(x,y) \phi(x)\; dx \]
%
where $K$ is a distribution on $\RR^n \times \RR^m$ and $\phi \in C_c^\infty(\RR^n)$. To formally interpret the output of this operator, we need to test it against another bump function, i.e. for $\psi \in C_c^\infty(\RR^m)$ we consider
%
\[ \int T\phi(y) \psi(y)\; dy = \int K(x,y) \phi(x) \psi(y)\; dx\; dy. \]
%
Thus $T\phi$ is naturally a distribution on $\RR^m$, and this definition naturally gives a continuous map from $C_c^\infty(\RR^n)$ to $C_c^\infty(\RR^m)'$. In 1953, Schwartz showed that essentially every linear operator encountered in Euclidean analysis is of this form.

\begin{theorem}
  Let $T: C_c^\infty(\RR^n) \to C_c^\infty(\RR^m)'$ be a continuous linear operator. Then there exists a unique distribution $K \in C_c^\infty(\RR^n \times \RR^m)$ such that for $\phi \in C_c^\infty(\RR^n)$ and $\psi \in C_c^\infty(\RR^m)$,
  %
  \[ \int T\phi(y) \psi(y)\; dy = \int K(x,y) \phi(x) \psi(y)\; dx\; dy. \]
\end{theorem}

Looking at the properties of kernels defining an operator is often a useful technique to gain insight in how an operator behaves. For instance, if $T$ is an operator corresponding to a kernel $K(x,y)$, then $D^\alpha \circ T \circ D^\beta$ has kernel $(-1)^{|\beta|} D^\alpha D^\beta K(x,y)$.

\section{Schwartz Space and Tempered Distributions}

We have already encountered the fact that Fourier transforms are well behaved under differentiation and multiplication by polynomials. If we let $\mathcal{S}(\RR^d)$ denote a class of functions under which to study this phenomenon, it must be contained in $L^1(\RR^d)$ and $C^\infty(\RR^d)$, and closed under multiplication by polynomials, and closed under applications of arbitrary constant-coefficient differential operators. A natural choice is then the family of functions which \emph{decays rapidly}, as well as all of it's derivatives; i.e. we let $\mathcal{S}(\RR^d)$ be the space of all functions $f \in C^\infty(\RR^d)$ such that for any integer $n$ and multi-index $\alpha$, $|x|^n D^\alpha f \in L^\infty(\RR^d)$. The space $\mathcal{S}(\RR^d)$ is then locally convex if we consider the family of seminorms
%
\[ \| f \|_{\mathcal{S}^{n,m}(\RR^d)} = \sup_{|\beta| \leq n} \| |1+x|^m D^\beta f \|_{L^\infty(\RR^d)}. \]
%
Elements of $\mathcal{S}(\RR^d)$ are known as \emph{Schwartz functions}, and $\mathcal{S}(\RR^d)$ is often known as the \emph{Schwartz space}. The seminorms naturally give $\mathcal{S}(\RR^d)$ the structure of a Fr\'{e}chet space. Sometimes, it is more convenient to use the equivalent family of seminorms $\| f \|_{\mathcal{S}^{\alpha, \beta}(\RR^d)} = \| x^\alpha D^\beta f \|_{L^\infty(\RR^d)}$, because $x^\alpha$ often behaves more nicely under various Fourier analytic operations. It is obvious that $\mathcal{S}(\RR^d)$ is separated by the seminorms defined on it, because $\| \cdot \|_{L^\infty(\RR^d)} = \| \cdot \|_{\mathcal{S}^{0,0}(\RR^d)}$ is a norm used to define the space. We now show the choice of seminorms make the space complete.

\begin{theorem}
    $\mathcal{S}(\RR^d)$ is a complete metric space.
\end{theorem}
\begin{proof}
    Let $\{ f_i \}$ be a Cauchy sequence with respect to the seminorms $\| \cdot \|_{\mathcal{S}^{n,\alpha}(\RR^d)}$. This implies that for each integer $m$, and multi-index $\alpha$, the sequence of functions $[1 + |x|^m] D^\alpha f_k$ is Cauchy in $L^\infty(\RR^d)$. Since $L^\infty(\RR^d)$ is complete, there are functions $g_{m,\alpha}$ such that $(1 + |x|^m) D^\alpha f_k$ converges uniformly to $g_{m,\alpha}$. If we set $f = g_{0,0}$, then it is easy to see using the basic real analysis of uniform continuity that $f$ is infinitely differentiable, and $(1 + |x|^m) D^\alpha f = g_{m,\alpha}$. It is then easy to show that $f_i$ converges to $f$ in $\mathcal{S}(\RR^d)$.
\end{proof}

\begin{example}
    The Gaussian function $\phi: \RR^d \to \RR$ defined by $\phi(x) = e^{-|x|^2}$ is Schwartz. For any multi-index $\alpha$, there is a polynomial $P_\alpha$ of degree at most $|\alpha|$ such that $D^\alpha \phi = P_\alpha \phi$; this can be established by a simple induction. But this means that for each fixed $\alpha$, $|P_\alpha(x)| \lesssim_\alpha 1 + |x|^{|\alpha|}$. Since $e^{-|x|^2} \lesssim_{m,\alpha} 1/(1 + |x|)^{m + |\alpha|}$ for any fixed $m$ and $\alpha$, we find that for any $x \in \RR^d$,
    %
    \[ | (1 + |x|^m) D^\alpha \phi| \lesssim_{\alpha,m} 1. \]
    %
    Since $m$ and $\alpha$ were arbitrary, this shows $\phi$ is Schwartz.
\end{example}

\begin{example}
    The space $C^\infty_c(\RR^d)$ consists of all compactly supported $C^\infty$ functions. If $f \in C^\infty_c(\RR^d)$, then $f$ is Schwartz. This is because for each $\alpha$ and $m$, $(1 + |x|)^m f_\alpha$ is a continuous function vanishing outside a compact set, and is therefore bounded.
\end{example} 

Because of the sharp control we have over functions in $\mathcal{S}(\RR^d)$, almost every analytic operation we want to perform on $\mathcal{S}(\RR^d)$ is continuous. To show that an operator $T$ on $\mathcal{S}(\RR^d)$ is bounded, it suffices to show that for each $n_0$ and $m_0$, there is $n_1$, $m_1$ such that
%
\[ \| Tf \|_{\mathcal{S}^{n_0,m_0}(\RR^d)} \lesssim_{n_0,m_0} \| f \|_{\mathcal{S}^{n_1,m_1}(\RR^d)}. \]
%
For a functional $\Lambda: \mathcal{S}(\RR^d) \to \RR$, it suffices to show that there exists $n$ and $m$ such that $|\Lambda f| \lesssim \| f \|_{\mathcal{S}^{n,m}(\RR^d)}$. The minimal such choice of $n$ is known as the \emph{order} of the functional $\Lambda$. We normally do not care about the constant behind the operators for these norms, since the norms are not translation invariant and therefore highly sensitive to the positions of various functions. We really just care about proving the existence of such a constant.

\begin{lemma}
  The map $(f,g) \mapsto fg$ for $f,g \in \mathcal{S}(\RR^d)$ gives a bounded bilinear map from $\mathcal{S}(\RR^d) \times \mathcal{S}(\RR^d) \to \mathcal{S}(\RR^d)$.
\end{lemma}
\begin{proof}
  A simple application of the Leibnitz formula shows that for any multi-index $\alpha$ with $|\alpha| = m$, and two non-negative integers $n_1$ and $n_2$ with $n_1 + n_2 = n$,
  %
  \[ \| fg \|_{\mathcal{S}^{n,\alpha}(\RR^d)} \lesssim_n \| f \|_{\mathcal{S}^{n_1,m}(\RR^d)} \| g \|_{\mathcal{S}^{n_2,m}(\RR^d)}. \]
  %
  More generally, this argument shows that the analogoue bilinear map from $C^\infty(\RR^d) \times \mathcal{S}(\RR^d) \to \mathcal{S}(\RR^d)$ is bounded.
\end{proof}

\begin{theorem}
    The following operators are all bounded on $\mathcal{S}(\RR^n)$.
    %
    \begin{itemize}
        \item For each $h \in \RR^n$, the translation operator $(T_h f)(x) = f(x - h)$.

        \item For each $\xi \in \RR^n$, the modulation operator $(M_\xi f)(x) = e(\xi \cdot x) f(x)$.

        \item The $L^p$ norms $\| f \|_{L^p(\RR^n)}$, for $1 \leq p \leq \infty$.

        \item The Fourier transform from $\mathcal{S}(\RR^d)$ to $\mathcal{S}(\RR^d)$.
    \end{itemize}
    %
    Furthermore, the Fourier transform is an isomorphism of $\mathcal{S}(\RR^d)$.
\end{theorem}
\begin{proof}
%   Let $(T_h f)(x) = f(x - h)$. We calculate that if $|\alpha| \leq n$, then
    %
%   \begin{align*}
%       (1 + |x|^m) (T_h f)_\alpha &= T_h((1 + |x + h|^m) f_\beta)\\
%       &\leq 2^m T_h((1 + |x|^m + |h|^m) f_\alpha)\\
%       &\leq 2^m |h|^m \| f_\alpha \|_{n,0} + 2^m \| f \|_{n,m}.
%   \end{align*}
    %
%   Thus $\| T_h f \|_{n,m} \leq 2^m(1 + |h|^m) \| f \|_{n,m}$, so $T_h$ is continuous.

%   Similarily, we calculate using the Leibnitz formula and the formula for the derivatives of $e(\xi \cdot x)$ that if $|\alpha| \leq n$, then
    %
%   \[ (1 + |x|^m) |(e(\xi \cdot x) f)_\alpha| \leq 4^n (2\pi)^n (1 + |\xi|^n) \| f \|_{n,m} \]
    %
%   Thus $\| M_\xi f \|_{n,m} \leq (8 \pi)^n (1 + |\xi|^n) \| f \|_{n,m}$.

%   For any Schwartz function $f$, and $|\alpha| \leq n$,
    %
%   \[ f(x) \leq \frac{\| f \|_{0,d+1}}{1 + |x|^{d+1}} \]
    %
%   Integrating this equation gives
    %
%   \[ \| f_\alpha \|_{L^1(\RR^d)} \leq 2^d \| f \|_{0,d+1}. \]
    %
%   Thus $\| \cdot \|_1$ is a bounded norm on the space. Interpolation then shows that for any $1 < p < \infty$,
    %
%   \[ \| f \|_{L^p(\RR^d)} \leq \| f \|_{L^1(\RR^d)}^{1 - 1/p} \| f \|_{L^\infty(\RR^d)}^{1/p} \leq \| f \|_{L^1(\RR^d)} + \| f \|_{L^\infty(\RR^d)} \leq 2 \| f \|_{0,d+1}. \]
    %
%   This implies $\| \cdot \|_{L^p(\RR^d)}$ is bounded.

%   A simple calculation using the Leibnitz formula shows that if $|\alpha| \leq n$,
    %
%   \begin{align*}
%       (1 + |x|^m) |\mathcal{F}(f)_\alpha| &\leq |\mathcal{F}(f)_\alpha| + \sum_{k = 1}^d |x_k^m \mathcal{F}(f)_\alpha|\\
%       &\leq (2 \pi)^n \left( \| \mathcal{F} f \|_{L^\infty(\RR^d)} + \sum_{k = 1}^d |\mathcal{F}((x^\alpha f)_{me_k})| \right)\\
%       &\leq n! (2 \pi)^n 2^m (n+1) \max_{0 \leq k \leq d} \max_{1 \leq l \leq m} \left( \| \mathcal{F} f \|_{L^\infty(\RR^d)} + \sum_{k = 1}^n \max_{1 \leq l \leq m} \| \mathcal{F}(f_{le_k}) \|_{L^\infty(\RR^d)} \right)\\
%       &\leq n! (2 \pi)^n 2^m \left( \| f \|_{L^1(\RR^d)} + \sum_{k = 1}^n \max_{1 \leq l \leq m} \| f_{le_k} \|_{L^1(\RR^d)} \right)\\
%       &\leq n! (2 \pi)^n 2^m 2^d (n+1) \| f \|_{n,d+1}.
%   \end{align*}

%   there are constants $c_{\alpha \beta \gamma}$ for each $\gamma \leq \alpha \wedge \beta$ such that
    %
%   \begin{align*}
%       |x^\alpha \mathcal{F}(f)_\beta| &= (2 \pi)^{|\beta|} |x^\alpha \cdot \mathcal{F}(x^\beta f)|\\
%       &= (2\pi)^{|\beta| - |\alpha|} \mathcal{F}((x^\beta f)_\alpha)\\
%       &\leq (2\pi)^{|\beta| - |\alpha|} \sum_{\gamma \leq \alpha \wedge \beta} c_{\alpha \beta \gamma} |\mathcal{F}(x^{\beta - \gamma} f_{\alpha - \gamma})|.
%   \end{align*}
    %
%   This calculation shows
    %
%   \begin{align*}
%       \| \mathcal{F} f \|_{\alpha,\beta} &\lesssim_{\alpha,\beta} \sum \| \mathcal{F}(x^{\beta - \gamma} f_{\alpha - \gamma}) \|_{L^\infty(\RR^n)}\\
%       &\leq \sum \| x^{\beta - \gamma} f_{\alpha - \gamma} \|_{L^1(\RR^n)}.
%   \end{align*}
    %
%   The right hand side is a continuous function of $f$, so the Fourier transform is bounded. The smoothness of the Schwartz space implies that $\mathcal{F}$ is a bijective map. But then the open mapping theorem implies that $\mathcal{F}^{-1}$ is a bounded operation, and therefore $\mathcal{F}$ is a homeomorphism.

    We leave all but the last point as exercises. Here it will be convenient to use the norms $\| \cdot \|_{\mathcal{S}^{\alpha,\beta}(\RR^d)}$ as well as the norms $\| \cdot \|_{\mathcal{S}^{n,m}(\RR^d)}$. If $|\alpha| \leq m$, $|\beta| \leq n$, then we can use the Leibnitz formula to conclude that
    %
    \begin{align*}
        \| \xi^\alpha D^\beta \mathcal{F}(f) \|_{L^\infty(\RR^d)} &\lesssim_{\alpha,\beta} \| \mathcal{F}(D^\alpha(x^\beta f)) \|_{L^\infty(\RR^d)}\\
        &\lesssim_{\alpha,\beta} \max_{\gamma \leq \alpha \wedge \beta} \| \mathcal{F}(x^{\gamma} D^\gamma f) \|_{L^\infty(\RR^d)}\\
        &\leq \max_{\gamma \leq \alpha \wedge \beta} \| x^\gamma D^\gamma f \|_{L^1(\RR^d)}\\
        &\lesssim \| f \|_{\mathcal{S}^{\gamma,|\gamma| + d+1}(\RR^d)}.
    \end{align*}
    %
    Thus $\mathcal{F}$ is a bounded linear operator on $\mathcal{S}(\RR^d)$. Since all Schwartz functions are arbitrarily smooth, the Fourier inversion formula applies to all Schwartz functions, and so $\mathcal{F}$ is a bijective bounded linear operator with inverse $\mathcal{F}^{-1}$. The open mapping theorem then immediately implies that $\mathcal{F}^{-1}$ is bounded.
\end{proof}

\begin{corollary}
    If $f$ and $g$ are Schwartz, then $f * g$ is Schwartz.
\end{corollary}
\begin{proof}
    Since $f * g = \mathcal{F}^{-1}(\mathcal{F}(f) \mathcal{F}(g))$, this fact follows from the previous two lemmas.
\end{proof}

Now we get to the interesting part of the theory. We have defined a homeomorphic linear transform from $\mathcal{S}(\RR^d)$ to itself. The theory of functional analysis then says that we can define a dual map, which is a homeomorphism from the dual space $\mathcal{S}(\RR^d)^*$ to itself. Note the inclusion map $C_c^\infty(\RR^d) \to \mathcal{S}(\RR^d)$ is continuous, and $C_c^\infty(\RR^d)$ is dense in $\mathcal{S}(\RR^d)$. This implies that we have an injective, continuous map from $\mathcal{S}^*(\RR^d)$ to $(C_c^\infty)^*(\RR^d)$, so every functional on the Schwarz space can be identified with a distribution. We call such distributions \emph{tempered}. They are precisely the linear functionals on $C_c^\infty(\RR^d)$ which have a continuous extension to $\mathcal{S}(\RR^d)$. Intuitively, this corresponds to an asymptotic decay condition.

\begin{example}
    Recall that for any $f \in L^1_{\text{loc}}(\RR^d)$, we can consider the distribution $\Lambda[f]$ defined by setting
    %
    \[ \Lambda[f](\phi) = \int f(x) \phi(x)\; dx. \]
    %
    However, this distribution is not always tempered. If $f \in L^p(\RR^d)$ for some $p$, then, applying H\"{o}lder's inequality, we obtain that
    %
    \[ |\Lambda[f](\phi)| \leq \| f \|_{L^p(\RR^d)} \| \phi \|_{L^q(\RR^d)}. \]
    %
    Since $\| \cdot \|_{L^q(\RR^d)}$ is a continuous norm on $\mathcal{S}(\RR^d)$, this shows $\Lambda[f]$ is bounded. More generally, if $f \in L^1_{\text{loc}}(\RR^d)$, and $f(x) (1 + |x|)^{-m}$ is in $L^p(\RR^d)$ for some $m$, then $\Lambda[f]$ is a tempered distribution. If $p = \infty$, such a function is known as \emph{slowly increasing}.
\end{example}

\begin{example}
    For any Radon measure, $\mu$, we can define a distribution
    %
    \[ \Lambda[\mu](\phi) = \int \phi(x) d\mu(x) \]
    %
    But this distribution is not always tempered. If $|\mu|$ is finite, the inequality $\| \Lambda[\mu](\phi) \| \leq \| \mu \| \| \phi \|_{L^\infty(\RR^d)}$ gives boundedness. More generally, if $\mu$ is a measure such that for some $n$,
    %
    \[ \int_{\RR^d} \frac{d|\mu|(x)}{1 + |x|^n}\; dx < \infty \]
    %
    then $\mu$ is known as a \emph{tempered measure}, and acts as a tempered distribution since
    %
    \begin{align*}
      |\Lambda[\mu](\phi)| &\leq \int_{\RR^d} |\phi(x)|\; d|\mu|(x)\\
      &\leq \left( \int_{\RR^d} \frac{d|\mu|(x)}{1 + |x|^n}\; dx \right) \cdot \| \phi \|_{\mathcal{S}^{0,n}(\RR^d)}.
    \end{align*}
\end{example}

\begin{example}
  Any compactly supported distribution is tempered. Indeed, if $\Lambda$ is a distribution supported on a compact set $K$, then it has finite order $n$ for some integer $n$, and extends to an operator on $C^\infty(\RR^d)$. We then find
  %
  \[ |\Lambda(\phi)| \lesssim \| \phi \|_{C^n(\RR^d)} \leq \| \phi \|_{\mathcal{S}^{0,n}(\RR^d)}. \]
\end{example}

\begin{example}
  The distribution $\Lambda$ on $\RR$ given by
  %
  \[ \Lambda(\phi) = \text{p.v.} \int_{-\infty}^\infty \frac{\phi(x)}{x}\; dx \]
  %
  is tempered, since
  %
  \[ \int_{|x| \geq 1} \frac{\phi(x)}{x} \lesssim \| \phi \|_{\mathcal{S}^{1,0}(\RR^d)} \]
  %
  and
  %
  \[ \text{p.v.} \int_{-\infty}^\infty \frac{\phi(x)}{x}\; dx \lesssim \| \phi \|_{C^1(\RR^d)} = \| \phi \|_{\mathcal{S}^{0,1}(\RR^d)} \]
  %
  and so $\Lambda$ is tempered of order 1.
\end{example}

Using the same techniques as for distributions, the derivative $D^\alpha \Lambda$ of a tempered distribution $\Lambda$ is tempered, as is $\phi \Lambda$, whenever $\phi$ is a Schwartz function, or $f \Lambda$, where $f$ is a polynomial. Of course, we can multiply by polynomially increasing smooth functions as well.

Let us now apply the distributional method to define the Fourier transform of a tempered distribution. Recall that we heuristically think of $\Lambda$ as formally corresponding to a regular function $f$ such that
%
\[ \Lambda(\phi) = \int f(x) \phi(x)\; dx \]
%
The multiplication formula
%
\[ \int_{\RR^d} \widehat{f}(\xi) g(\xi)\; d\xi = \int_{\RR^d} f(x) \widehat{g}(x)\; dx \]
%
gives us the perfect opportunity to move the analytical operations on $f$ to analytical operations on $g$. Thus if $\Lambda$ is the distribution corresponding to a Schwartz $f \in \mathcal{S}(\RR^d)$, the distribution $\widehat{\Lambda}$ corresponding to $\widehat{f}$, then for any Schwartz $\phi \in \mathcal{S}(\RR^d)$,
%
\[ \widehat{\Lambda}(\phi) = \Lambda \left( \widehat{g} \right). \]
%
In particular, this motivates us to define the Fourier transform of \emph{any} tempered distribution $\Lambda$ to be the unique tempered distribution $\widehat{\Lambda}$ such that the equation above holds for all Schwartz $\phi$. This distribution exists because the Fourier transform is an isomorphism on the space of Schwartz functions. Clearly, the Fourier transform is a homeomorphism on the space of tempered distributions under the weak topology, and moreover, satisfies all the symmetry properties that the ordinary Fourier transform does, once we interpret scalar, rotation, translation, differentiation, etc, in a natural way on the space of distributions.

\begin{example}
    Consider the constant function $1$, interpreted as a tempered distribution on $\RR^d$. Then for any $\phi \in \mathcal{S}(\RR^d)$,
    %
    \[ 1(\phi) = \int \phi(x)\; dx, \]
    %
    Thus for any $\phi \in \mathcal{S}(\RR^d)$,
    %
    \[ \widehat{1} \left( \widehat{\phi} \right) = 1(\phi) = \int \phi(\xi)\; d\xi = \widehat{\phi}(0). \]
    %
    Thus $\widehat{1}$ is the Dirac delta function at the origin. Similarily, the Fourier inversion formula implies that
    %
    \[ \widehat{\delta} \left( \widehat{\phi} \right) = \phi(0) = \int \widehat{\phi}(\xi)\; d\xi = 1 \left( \widehat{\phi} \right) \]
    %
    so the Fourier transform of the Dirac delta function is the constant 1 function.
\end{example}

\begin{example}
  The theory of tempered distributions enables us to take the Fourier transform of $f \in L^p(\RR^d)$, when $p > 2$ or when $p < 1$. The introduction of distributions is in some sense, essential to this process, because for each $p \not \in [1,2]$, there is $f \in L^p(\RR^d)$ such that $\widehat{f}$ is \emph{not} a locally integrable function. Otherwise, we could define an operator $T: L^p(\RR^d) \to L^1(\RR^d)$ given by
  %
  \[ Tf = \widehat{f} \mathbf{I}_{|\xi| \leq 1}. \]
  %
  If a sequence of functions $\{ f_n \}$ converges to $f$ in $L^p(\RR^d)$, and $Tf_n$ converges to $g$ in $L^1(\RR^d)$, then $Tf_n$ converges distributionally to $g$, which implies $Tf = g$. The closed graph theorem thus implies that $T$ is a continuous operator from $L^p(\RR^d)$ to $L^1(\RR^d)$, so there exists $M > 0$ such that
  %
  \[ \int_{|\xi| \leq 1} |\widehat{f}(\xi)| \leq M \| f \|_{L^p(\RR^d)}. \]
  %
  If $f_\alpha(x) = e^{-\pi \alpha |x|^2}$, then $\widehat{f_\alpha}(\xi) = \alpha^{-d/2} e^{-\pi |x|^2 / \alpha}$. We have
  %
  \begin{align*}
    \| f_\alpha \|_{L^p(\RR^d)} &= \left( \int_{\RR^d} e^{- \pi \alpha p |x|^2}\; dx \right)^{1/p}\\
    &= (\alpha p)^{-d/2p} \left( \int_{\RR^d} e^{- \pi |x|^2}\; dx \right)^{1/p} \lesssim_d (\alpha p)^{-1/2p}.
  \end{align*}
  %
  On the other hand, for $|\xi| \leq 1$, $|\widehat{f_\alpha}(\xi)| \geq \alpha^{-d/2} e^{-\pi/\alpha}$, so
  %
  \[ \int_{|\xi| \leq 1} |\widehat{f_\alpha}(\xi)| \gtrsim_d \alpha^{-d/2} e^{-\pi/\alpha}. \]
  %
  Thus we conclude that $\alpha^{-d/2} e^{-\pi/\alpha} \lesssim_d M (\alpha p)^{-d/2p}$, or equivalently,
  %
  \[ \alpha^{d/2(1/p-1)} e^{-\pi/\alpha} \lesssim_d M p^{-d/2p}. \]
  %
  Taking $\alpha \to \infty$ gives a contradiction if $p < 1$. For $p > 2$, we give the Gaussian an oscillatory factor that does not affect the $L^p$ norm but boosts the $L^1$ norm of the Fourier transform. We set
  %
  \[ g_\delta(x) = \prod_{k = 1}^d \frac{e^{- \pi x_k^2 / (1 + i \delta)}}{(1 + i \delta)^{1/2}}. \]
  %
  The Fourier transform formula of the Gaussian, when applied using the theory of analytic continuation, shows that
  %
  \[ \widehat{g_\delta}(\xi) = \prod_{k = 1}^d e^{- \pi (1 + i \delta) \xi_k^2}. \]
  %
  We have
  %
  \[ \int_{|\xi| \leq 1} |\widehat{g_\delta}(\xi)| = \int_{|\xi| \leq 1} e^{- \pi |\xi|^2} \gtrsim 1. \]
  %
  On the other hand, for $\delta \geq 1$,
  %
  \begin{align*}
    \| g_\delta \|_{L^p(\RR^d)} &= \left( \int |g_\delta(x)|^p\; dx \right)^{1/p}\\
    &= |1 + i \delta|^{-d/2} \left( \int_{-\infty}^\infty e^{- p \pi x^2/(1 + \delta^2)}\; dx \right)^{d/p}\\
    &\lesssim_d \delta^{-d/2} \delta^{d/p} p^{-d/p} = \delta^{d(1/p - 1/2)} p^{-d/p}.
  \end{align*}
  %
  Thus we conclude $1 \lesssim_d M \delta^{d(1/p - 1/2)} p^{d/p}$, which gives a contradiction as $\delta \to \infty$ if $p > 2$.
\end{example}

\begin{example}
  Consider the Riesz Kernel on $\RR^d$, for each $\alpha \in \CC$ with positive real part, as the function
  %
  \[ K_\alpha(x) = \frac{\Gamma(\alpha/2)}{\pi^{\alpha/2}} |x|^{-\alpha}. \]
  %
  Then for $0 < \text{Re}(\alpha) < d$, $\widehat{K_\alpha} = K_{d-\alpha}$. We recall that $\Gamma$ is defined by the integral formula
  %
  \[ \Gamma(s) = \int_0^\infty e^{-t} t^{s-1}\; ds, \]
  %
  where $\text{Re}(s) > 0$. We note that if $p = d/\text{Re}(\alpha)$, $K_\alpha \in L^{p,\infty}(\RR^d)$. The Marcinkiewicz interpolation theorem implies that if $d/2 < \text{Re}(\alpha) < d$, then $K_\alpha$ can be decomposed as the sum of a $L^1(\RR^d)$ function and a $L^2(\RR^d)$ function, and so we can intepret the Fourier transform of $\widehat{K_\alpha}$ using techniques in $L^1(\RR^d)$ and $L^2(\RR^d)$, and moreover, the Marcinkiewicz interpolation theorem implies that
  %
  \[ \| \widehat{K_\alpha} \|_{L^{q,\infty}(\RR^d)} \leq \| K_\alpha \|_{L^{p,\infty}(\RR^d)}. \]
  %
  where $q$ is the dual of $p$. In particualr, the Fourier transform of $K_\alpha$ is a function. We note that $K_\alpha$ obeys multiple symmetries. First of all, $K_\alpha$ is radial, so $\widehat{K_\alpha}$ is also radial. Moreover, $K_\alpha$ is homogenous of degree $-\alpha$, i.e. for each $x \in \RR^d$, $K_\alpha(\varepsilon x) = \varepsilon^{-\alpha} K_\alpha(x)$. This actually uniquely characterizes $K_\alpha$ among all locally integrable functions. Taking the Fourier transform of both sides of the equation for homogeneity, we find
  %
  \[ \varepsilon^{-d} \widehat{K_\alpha}(\xi/\varepsilon) = \varepsilon^{-\alpha} \widehat{K_\alpha}(x). \]
  %
  Thus $\widehat{K_\alpha}$ is homogenous of degree $\alpha - d$. But this uniquely characterizes $\widehat{K_{d-\alpha}}$ out of any distribution, up to multiplicity, so we conclude that for $d/2 < \text{Re}(\alpha) < d$, that $\widehat{K_\alpha}$ is a scalar multiple of $K_{d-\alpha}$. But we know that by a change into polar coordinates, if $A_d$ is the surface area of a unit sphere in $\RR^d$, then
  %
  \begin{align*}
    \int_{\RR^d} K_\alpha(x) e^{- \pi |x|^2}\; dx &= \frac{\Gamma(\alpha/2)}{\pi^{\alpha/2}} \int_{\RR^d} |x|^{-\alpha} e^{-\pi |x|^2}\; dx\\
    &= A_d \frac{\Gamma(\alpha/2)}{\pi^{\alpha/2}} \int_0^\infty r^{d-1-\alpha} e^{- \pi r^2}\; dr\\
    &= A_d \frac{\Gamma(\alpha/2)}{2 \pi^{d/2}} \int_0^\infty s^{(d-\alpha)/2 - 1} e^{-s}\; ds\\
    &= A_d \frac{\Gamma(\alpha/2) \Gamma((d-\alpha)/2)}{\pi^{d/2}}.
  \end{align*}
  %
  But this is also the value of
  %
  \[ \int_{\RR^d} K_{d - \alpha}(x) e^{- \pi |x|^2}, \]
  %
  so we conclude $\widehat{K_\alpha} = K_{d-\alpha}$ if $d/2 < \text{Re}(\alpha) < d$. We could apply Fourier inversion to obtain the result for $0 < \text{Re}(\alpha) < d/2$, but to obtain the case $\text{Re}(\alpha) = d/2$, we must apply something different. For each $s \in \CC$ with $0 < \text{Re}(s) < d$, and for each Schwartz $\phi \in \mathcal{S}(\RR^d)$ we define
  %
  \[ A(s) = \int K_s(\xi) \widehat{\phi}(\xi)\; d\xi = \frac{\Gamma(s/2)}{\pi^{s/2}} \int |\xi|^{-s/2} \widehat{\phi}(\xi)\; d\xi. \]
  %
  and
  %
  \[ B(s) = \int K_{d-s}(\xi) \widehat{\phi}(\xi)\; d\xi = \frac{\Gamma((d-s)/2)}{\pi^{(d-s)/2}} \int |\xi|^{(d-s)/2} \widehat{\phi}(\xi)\; d\xi. \]
  %
  The integrals above converge absolutely for $0 < \text{Re}(s) < d$, and the dominated convergence theorem implies that $A$ and $B$ are both complex differentiable. Since $A(s) = B(s)$ for $d/2 < \text{Re}(s) < d$, analytic continuation implies $A(s) = B(s)$ for all $0 < \text{Re}(s) < d$, completing the proof. For $\text{Re}(\alpha) \geq d$, $K_\alpha$ is no longer locally integrable, and so we must interpret the distribution given by integration by $K_\alpha$ in terms of principal values. The fourier transform of these functions then becomes harder to define.
\end{example}

\begin{example}
  Let us consider the complex Gaussian defined, for a given invertible symmetric matrix $T: \RR^d \to \RR^d$, as $G_T(x) = e^{- i \pi (Tx \cdot x)}$. Then
  %
  \[ \widehat{G_T} = e^{- i \pi \sigma/4} |\det(T)|^{-1/2} G_{-T^{-1}}, \]
  %
  where $\sigma$ is the \emph{signature} of $T$, i.e. the number of positive eigenvalues, minus the number of negative eigenvalues, counted up to multiplicity. Thus we need to show that for any Schwartz $\phi \in \mathcal{S}(\RR^d)$,
  %
  \[ e^{-i \pi \sigma/4} |\det(T)|^{-1/2} \int_{\RR^d} e^{i \pi (T^{-1}\xi \cdot \xi)} \widehat{\phi}(\xi)\; d\xi = \int_{\RR^d} e^{- i \pi (Tx \cdot x)} \phi(x)\; dx. \]
  %
  Let us begin with the case $d = 1$, in which case we also prove the theorem when $T$ is a complex symmetric matrix. If $T$ is given by multiplication by $-iz$, and if $\sqrt{\cdot}$ denotes the branch of the square root defined for all non-negative numbers and positive on the real-axis, then we note that when $z = \lambda i$,
  %
  \[ e^{- i \pi \sigma/4} |\det(T)|^{-1/2} = e^{- i \pi \text{sgn}(\lambda)/4} |\lambda|^{-1/2} = \sqrt{z}. \]
  %
  Thus it suffices to prove the analytic family of identities
  %
  \[ \int_{-\infty}^\infty e^{- (\pi/z) \xi^2} \widehat{\phi}(\xi)\; d\xi = \sqrt{z} \int_{-\infty}^\infty e^{-\pi z x^2} \phi(x)\; dx, \]
  %
  where both sides are well defined and analytic whenever $z$ has positive real part. But we already know from the Fourier transform of the Gaussian that this identity holds whenever $z$ is positive and real, and so the remaining identities follows by analytic continuation. We note that the higher dimensional identity is invariant under changes of coordinates in $SO(n)$. Thus it suffices to prove the remaining theorem when $T$ is diagonal. But then everything tensorizes and reduces to the one dimensional case. More generally, if $T = T_0 - i T_1$ is a complex symmetric matrix, which is well defined if $T_1$ is positive semidefinite, then
  %
  \[ \widehat{G_T} = \frac{1}{\sqrt{i \det(T)}} \cdot G_{-T^{-1}}, \]
  %
  which follows from analytic continuation of the case for real $T$.
\end{example}

\begin{example}
    We know $((-2 \pi i x)^\alpha)^\ft = ((- 2 \pi i x)^\alpha \cdot 1)^\ft = \delta_\alpha$, which essentially provides us a way to compute the Fourier transform of any polynomial, i.e. as a linear combination of dirac deltas and the distribution derivatives of dirac deltas, which are derivatives evaluated at points.
\end{example}

\begin{theorem}
    If $\mu$ is a finite measure, $\widehat{\mu}$ is a uniformly continuous bounded function with $\| \widehat{\mu} \|_{L^\infty(\RR^d)} \leq \| \mu \|$, and
    %
    \[ \widehat{\mu}(\xi) = \int e(- 2 \pi i x \cdot \xi) d\mu(x) \]
    %
    The function $\widehat{\mu}$ is also smooth if $\mu$ has moments of all orders, i.e. $\int |x|^k d\mu(x) < \infty$ for all $k > 0$.
\end{theorem}
\begin{proof}
    Let $\phi \in \mathcal{S}(\RR^d)$. We must understand the integral
    %
    \[ \int_{\RR^d} \widehat{\phi}(x)\; d\mu(x). \]
    %
    Applying Fubini's theorem, which applies since $\mu$ has finite mass, we conclude that
    %
    \[ \int_{\RR^d} \widehat{\phi}(x)\; d\mu(x) = \int_{\RR^d} \int_{\RR^d} \phi(\xi) e^{-2 \pi i \xi \cdot x} d\mu(x)\; d\xi = \int_{\RR^d} \phi(\xi) f(\xi)\; d\xi, \]
    %
    where
    %
    \[ f(\xi) = \int_{\RR^d} e^{-2 \pi i \xi x} d\mu(x). \]
    %
    Thus $\widehat{\mu}$ is precisely $f$, and it suffices to show that $\| f \|_{L^\infty(\RR^d)} \leq \| \mu \|$, and that $f$ is uniformly continuous. The inequality follows from a simple calculation of the triangle inequality, and the second inequality follows because for some $y$,
    %
    \begin{align*}
      |f(\xi + \eta) - f(\xi)| &= \left| \int_{\RR^d} e^{-2 \pi i \xi \cdot x} (e^{-2 \pi i \eta \cdot x} - 1)\; d\mu(x) \right|\\
      &\leq \int_{\RR^d} |e^{-2 \pi i \eta \cdot x} - 1|\; d|\mu|(x).
    \end{align*}
    %
    As $\eta \to 0$, the dominated convergence theorem implies that this quantity tends to zero, which proves uniform continuity. On the other hand, if $x_i \mu$ is finite for some $i$, then
    %
    \begin{align*}
      \frac{f(\xi + \varepsilon e_i) - f(\xi)}{\varepsilon} &= \int_{\RR^d} e^{-2 \pi i \xi \cdot x} \frac{(e^{- 2 \pi \varepsilon i x_i} - 1)}{\varepsilon} d\mu(x).
    \end{align*}
    %
    We can apply the dominated convergence theorem to show that as $\varepsilon \to 0$, this quantity converges to the classical partial derivative $f_i$, which has the integral formula
    %
    \[ f_i(\xi) = (-2 \pi i) \int_{\RR^d} e^{-2 \pi i \xi \cdot x} x_i d\mu(x), \]
    %
    which is the Fourier transform of $x_i \mu$. Higher derivatives are similar.
\end{proof}

Not being compactly supported, we cannot compute the convolution of tempered distributions with all $C^\infty$ functions. Nonetheless, if $\phi$ is Schwartz, and $\Lambda$ is tempered, then the definition $(\Lambda * \phi)(x) = \Lambda(T_{-x} \phi^*)$ certainly makes sense, and gives a $C^\infty$ function satisfying $D^\alpha(\Lambda * \phi) = (D^\alpha \Lambda) * \phi = \Lambda * (D^\alpha \phi)$ just as for $\phi \in C_c^\infty(\RR^d)$. Moreover, $\Lambda * \phi$ is a slowly increasing function; to see this, we know there is $n$ such that
%
\[ |\Lambda \phi| \lesssim \| \phi \|_{\mathcal{S}^{n,m}(\RR^d)}. \]
%
Now for $|y| \geq 1$,
%
\[ \| T_y \phi \|_{\mathcal{S}^{n,m}(\RR^d)} \leq |x-y|^n \leq 2^n (1 + |y|^n) \| \phi \|_{\mathcal{S}^{n,m}(\RR^d)}, \]
%
and so
%
\[ (\Lambda * \phi)(x) = \Lambda(T_{-x} \phi^*) \lesssim_n (1 + |x|^n) \| \phi \|_{\mathcal{S}^{n,m}(\RR^d)}, \]
%
which gives that $\Lambda * \phi$ is slowly increasing. In particular, we can take the Fourier transform of $\Lambda * \phi$. Now for any $\psi \in \mathcal{S}(\RR^d)$ with $\widehat{\psi} \in C_c^\infty(\RR^d)$,
%
\begin{align*}
  \int_{\RR^d} \widehat{\Lambda * \phi}(\xi) \psi(\xi)\; d\xi &= \int_{\RR^d} (\Lambda * \phi)(x) \widehat{\psi}(x)\; dx\\
  &= \int_{\RR^d} \Lambda( \widehat{\psi}(x) \cdot T_{-x} \phi^*)\; dx\\
  &= \Lambda \left( \int_{\RR^d} \widehat{\psi}(x) T_{-x} \phi^*\; dx \right)\\
  &= \Lambda \left( \widehat{\psi} * \phi^* \right) = \Lambda \left( \widehat{\psi} * \widehat{\widehat{\phi}} \right)\\
  &= \Lambda \left( \widehat{\psi \widehat{\phi}} \right) = \widehat{\Lambda} \left( \psi \widehat{\phi} \right) = \widehat{\phi} \widehat{\Lambda}(\psi).
\end{align*}
%
We therefore conclude that $\widehat{\Lambda * \phi} = \widehat{\phi} \widehat{\Lambda}$.

\section{Paley-Wiener Theorems}

TODO: See Rudin, Functional Analysis.








\chapter{Psuedodifferential Operators}

Our goal is to consider more general families of operators that are amenable to analysis, but enable us to simultaneously control spatial and frequency properties of functions. The theory of Fourier multipliers can be used to understand constant coefficient differential operators. The most basic spatial multiplier in Fourier analysis are the \emph{position operators} $X^\alpha: \mathcal{S}(\RR^d) \to \mathcal{S}(\RR^d)$ given by
%
\[ X^\alpha f(x) = x^\alpha f(x) \]
%
and the most basic Fourier multipliers are the \emph{momentum operators}
%
\[ D^\alpha f(x) = \frac{1}{(2\pi i)^{|\alpha|}} \partial^\alpha f(x), \]
%
which have the property that $\widehat{D^\alpha f}(\xi) = \xi^\alpha \widehat{f}(\xi)$. If $m \in C^\infty(\RR^d)$ is given, such that $m$ and all it's derivatives are slowly increasing, then we can define a continuous operator $m(X): \mathcal{S}(\RR^d) \to \mathcal{S}(\RR^d)$ by setting
%
\[ m(X) f(x) = m(x) f(x). \]
%
We refer to $m$ as the \emph{symbol} of the operator. Similarily, we can define an operator $m(D): \mathcal{S}(\RR^d) \to \mathcal{S}(\RR^d)$ such that
%
\[ \widehat{m(D) f}(\xi) = m(\xi) \widehat{f}(\xi). \]
%
These give two homomorphisms from the ring of functions $m$ to the ring of continuous operators on $\mathcal{S}(\RR)$.

Our goal here is to associate with a suitably smooth family of functions $a(x,\xi)$, an operator $a(X,D)$ which extends the theory of spatial and Fourier multipliers. This is useful in a variety of contexts, especially in the theory of variable-coefficient linear operators. To begin with, if for each multi-index $|\alpha| \leq n$, we consider a function $c_\alpha \in C^\infty(\RR^d)$ such that itself and all of it's derivatives are slowly increasing, then we can consider a continuous operator $L: \mathcal{S}(\RR^d) \to \mathcal{S}(\RR^d)$ by setting
%
\[ Lf(x) = \sum_{|\alpha| \leq n} c_\alpha(x) D^\alpha f(x). \]
%
We associate with the operator $L$ the function
%
\[ a(x,\xi) = \sum_{|\alpha| \leq n} c_\alpha(x) \xi^k. \]
%
Applying the Fourier inversion formula, we conclude that for $f \in \mathcal{S}(\RR^d)$,
%
\begin{align*}
  Lf(x) &= \sum_{|\alpha| \leq n} c_\alpha(x) D^\alpha f(x)\\
  &= \sum_{|\alpha| \leq n} c_\alpha(x) \int_{\RR^d} \xi^\alpha \widehat{f}(\xi) e^{2 \pi i \xi \cdot x}\; d\xi\\
  &= \int_{\RR^d} a(x,\xi) \widehat{f}(\xi) e^{2 \pi i \xi \cdot x}\; d\xi.
\end{align*}
%
Now consider any smooth function $a \in C^\infty(\RR^n \times \RR^n)$ such that for any $n,m \geq 0$, there exists $k_n$ and $l_m$ such that
%
\[ |\nabla_x^n \nabla_\xi^m a(x,\xi)| \lesssim_{n,m} \langle x \rangle^{k_n} \langle \xi \rangle^{l_m}. \]
%
We call such a function a \emph{symbol}. From this function, we can define a continuous operator $a(X,D): \mathcal{S}(\RR^d) \to \mathcal{S}(\RR^d)$ such that
%
\[ [a(X,D) f](y) = \int_{\RR^d} a(y,\xi) \widehat{f}(\xi) e^{2 \pi i \xi \cdot y}\; d\xi. \]
%
This is known \emph{Kohn-Nirenberg quantization} of the function $a(x,\xi)$. Any such operator is known as a \emph{Psuedo-differential operator}, or $\Psi$DO. Applying the Schwartz kernel theorem, we can also view such an operator as a kernel operator, and it is a simple verification to show that $a(X,D)$ corresponds to the kernel
%
\[ K(x,y) = \int_{\RR^d} a(y,\xi) e^{2 \pi i \xi \cdot (y - x)}\; d\xi, \]
%
where this integral must be interpreted distributionally.

The choice of symbol associated to a psuedodifferential operator is not standard in the literature. For instance, for any symbol $a(x,\xi)$ we can also define the \emph{adjoint Kohn-Nirenberg quantization}
%
\[ a^*(X,D) f(y) = \int_{\RR^d} \int_{\RR^d} a(x,\xi) f(x) e^{2 \pi i (y - x) \cdot \xi}\; d\xi\; dx. \]
%
which is related to the Kohn-Nirenberg quantization by the relation $a^*(X,D) = \overline{a}(X,D)^*$, with $a^*(X,D)$ corresponding to the kernel
%
\[ K(x,y) = \int_{\RR^d} a(x,\xi) e^{2 \pi i (y - x) \cdot \xi}\; d\xi. \]
%
Since the conjugate of a symbol is a symbol, and the adjoint of a psuedodifferential operator is a psuedodifferential operator, we still get the same family of operators using the adjoint Kohn-Nirenberg quantization. For instance, the symbol corresponding to the differential operator
%
\[ Lf(x) = \sum_{|\alpha| \leq n} c_\alpha(x) D^\alpha f(x) \]
%
under the adjoint Kohn-Nirenberg scheme corresponds to the operator $a^*(X,D)$, where $a$ is the symbol
%
\[ a(x,\xi) = \sum_{|\alpha| \leq n} c_\alpha'(x) \xi^\alpha \]
%
and
%
\[ \sum_{|\alpha| \leq n} D^\alpha (c_\alpha' f) = \sum_{|\alpha| \leq n} c_\alpha D^\alpha f. \]
%
This is similar to the choice of notation for an elliptic equation, either given in divergence form, or nondivergence form. Note that, $c_\alpha' = c_\alpha$ if $|\alpha| = n$. In other words $a(X,D) - a^*(X,D)$ is a differential operator with order lower than that of $L$.

Note that, in the case when $a$ is a polynomial, the Kohn-Nirenberg scheme applies the operator $D$ first, then the operator $X$. On the other hand, the adjoint Kohn-Nirenberg scheme applies the operator $X$ first, then applies the operator $D$. It is often useful to treat the application in a symmetric manner. To do this, we introduce the \emph{Weyl quantization}, which for each symbol $a$ gives the operator $a_W(X,D)$ with
%
\[ a_W(X,D) f(y) = \int_{\RR^d} \int_{\RR^d} a \left( \frac{x + y}{2}, \xi \right) e^{2 \pi i (y - x) \cdot \xi} f(x) \; d\xi\; dx \]
%
(an equation that can only be interpreted distributionally if $a$ is not compactly supported), which has kernel
%
\[ K(x,y) = \int_{\RR^d} a \left( \frac{x + y}{2}, \xi \right) e^{2 \pi i (y - x) \cdot \xi}\; d\xi. \]
%
TODO: PROVE WEYL AND KOHN-NIRENBERG SCHEMES ARE EQUIVALENT.

It is often useful in harmonic analysis to think of a function $f \in \mathcal{S}(\RR^d)$ in terms of a `phase portrait' $\tilde{f}(x,\xi)$ which combines the physical space function $f(x)$ with the frequency space function $\widehat{f}(\xi)$. Thinking in this scheme, the phase portrait of $a(X,D) f$ should be approximately $a(x,\xi) f(x,\xi)$, though the uncertainty principle prevents this from being much more than an intuitive, though useful, picture.

The fact that $X$ and $D$ do not commute prevents the map $a(x,\xi) \mapsto a(X,D)$ is from being an algebra homomorphism (though it is linear). However, we hope that this intuition holds at least approximately. In particular, this should imply that $(a_1a_2)(X,D) \approx a_1(X,D) a_2(X,D)$. We shall develop this fact later on.

\section{Symbol Classes}

To build a more elegant theory, it is useful to restrict our class of symbols. Classically, it is natural to focus on the following family: a symbol $a \in C^\infty(\RR^d \times \RR^d)$ has \emph{order} $k$ and \emph{type} $\rho \in [0,1]$ if it obeys the bounds
%
\[ |\nabla_x^n \nabla_\xi^m a(x,\xi)| \lesssim_{n,m} \langle \xi \rangle^{k - \rho m}. \]
%
We let this class of symbols be denoted $S_\rho^k(\RR^d)$. We mostly deal with the case $\rho = 1$, in which case the class will just be denoted $S^k(\RR^d)$. Together with the class of norms
%
\[ \| a \| = \sup_{x,\xi} |\nabla_x^n \nabla_\xi^m a(x,\xi)| \cdot \langle \xi \rangle^{\rho m - k}, \]
%
for all $n,m \geq 0$, the set $S^k_\rho(\RR^d)$ becomes a Fr\'{e}chet space. Any linear differentiable operator of order $k$ corresponds to a symbol in $S^k(\RR^d)$, provided that the coefficient functions of the operator are all bounded, and all of the derivatives of the coefficients. If $a_1 \in S^{k_1}_\rho(\RR^d)$ and $a_2 \in S^{k_2}_\rho(\RR^d)$, then $a_1 \cdot a_2 \in S^{k_1 + k_2}_\rho(\RR^d)$. If $a \in S^k(\RR^d)$, we say $a(X,D)$ is a $\Psi DO$ of \emph{order $k$}.

\begin{theorem}[Calder\'{o}n-Vallaincourt]
  Let $1 < p < \infty$, and $a \in S^0(\RR^d)$, then
  %
  \[ \| a(X,D) f \|_{L^p(\RR^d)} \lesssim_{a,p} \| f \|_{L^p(\RR^d)}. \]
\end{theorem}
\begin{proof}
  TODO
\end{proof}

\section{Order Theory}

Recall the Sobolev spaces $H^s(\RR^d)$ consisting of functions $f \in L^2(\RR^d)$ such that the quantity
%
\[ \| f \|_{H^s(\RR^d)} = \left( \int_{\RR^d} (1 + |\xi|^2)^s |\widehat{f}(\xi)|^2\; d\xi \right)^{1/2}. \]
%
is finite. Such spaces can be defined for all $s \in \RR$. We say an operator $T: \mathcal{S}(\RR^d) \to \mathcal{S}(\RR^d)$ has \emph{order} $t$ if for each $s \in \RR$,
%
\[ \| Tf \|_{H^s(\RR^d)} \lesssim_s \| f \|_{H^{s+t}(\RR^d)}. \]
%
The \emph{true order} of $T$ is the infinum of the orders for $T$. Recall that if $|m(\xi)| \lesssim (1 + |\xi|^2)^\sigma$, then the Fourier multiplier $m(D)$ has order $2\sigma$. In particular, if $m$ is compactly supported, $m(D)$ has true order equal to $-\infty$.

To begin studying the orders of psuedodifferential operators, let us begin by assuming strong conditions on the function $a$. In particular, we assume $a(x,\xi)$ is homogenous of degree zero in $\xi$, and that moreover, there exists a smooth, homogenous function $b(\xi)$ such that for any integers $n_1,n_2$, and $n_3$,
%
\[ \nabla^{n_1}_x \nabla^{n_2}_\xi[a(x,\xi) - b(\xi)] \lesssim_{n_1,n_2,n_3} \frac{1}{1 + |x|^{n_3}} \]
%
We define $a_0(x,\xi) = a(x,\xi) - b(\xi)$.

\begin{theorem}
  The function $a(X,D)$ has order zero.
\end{theorem}
\begin{proof}
  Since $b \in L^\infty(\RR^d)$ due to it's homogeneity, we conclude that $b(D)$ has order zero, i.e.
  %
  \[ \| b(D) f \|_{H^s(\RR^d)} = \| b \widehat{f} (1 + |\xi|^2)^{s/2} \|_{L^2(\RR^d)} \lesssim \| \widehat{f} (1 + |\xi|^2)^{s/2} \|_{L^2(\RR^d)}. \]
  %
  It suffices to show $a_0^*(X,D)$ has order zero since $a_0^*$ satisfies the same hypothesis as $a_0$. We calculate that the adjoint of $a_0^*(X,D)$ is the operator $a_0^*(X,D)^*$ such that for $g \in \mathcal{S}(\RR^d)$,
  %
  \begin{align*}
    \widehat{a_0^*(X,D)^* g}(\xi) &= \int_{\RR^d} a(x,\xi) g(x) e^{-2 \pi i \xi \cdot x}\; dx.
  \end{align*}
  %
  By duality (noting $H^s(\RR^d)^* = H^{-s}(\RR^d)$), it therefore suffices to show $a_0^*(X,D)^*$ has order zero, which is easier because the manipulations here are in the Fourier domain. For each $\xi$, let $a_{0,\xi}(x) = a_0(x,\xi)$. Then
  %
  \[ \widehat{a_0^*(X,D)^* g}(\xi) = \widehat{a_{0,\xi} g}(\xi) = \int_{\RR^d} \widehat{a_{0,\xi}}(\xi - \eta) \widehat{g}(\eta)\; d\eta. \]
  %
  Thus
  %
  \begin{align*}
    \| a_0^*(X,D)^* g \|_{H^s(\RR^d)} &= \left\| \int_{\RR^d} \left[ \frac{(1 + |\xi|)^{s/2}}{(1 + |\eta|^2)^{s/2}} \widehat{a_{0,\xi}}(\xi - \eta) \right] \left[(1 + |\eta|^2)^{s/2}) \widehat{g}(\eta)\right]\; d\eta \right\|_{L^2_\xi(\RR^d)}.
  \end{align*}
  %
  By Schur's test, it suffices to show that
  %
  \[ \left\| \int_{\RR^d} \left[ \frac{(1 + |\xi|)^{s/2}}{(1 + |\eta|^2)^{s/2}} \widehat{a_{0,\xi}}(\xi - \eta) \right] \right\|_{L^1_\xi L^\infty_\eta}, \left\| \int_{\RR^d} \left[ \frac{(1 + |\xi|)^{s/2}}{(1 + |\eta|^2)^{s/2}} \widehat{a_{0,\xi}}(\xi - \eta) \right] \right\|_{L^1_\eta L^\infty_\xi} < \infty, \]
  %
  for then we find the upper quantity is bounded up to a constant by $\| g \|_{H^s(\RR^d)}$. TODO LATER.
\end{proof}

\begin{remark}
  Instead of the Kohn-Niremberg quantization $a(X,D) = a_{KN}(X,D)$, one can associate the \emph{adjoint Kohn-Niremberg quantization} $a(X,D) = a_{KN^*}(X,D)$ such that for $g \in \mathcal{S}(\RR^d)$,
  %
  \[ \widehat{a_{KN^*}(X,D) g}(\xi) = \int_{\RR^d} a(x,\xi) g(x) e^{-2 \pi i \xi \cdot x}\; dx. \]
  %
  Thus $a_{KN^*}(X,D) = a^*_{KN}(X,D)^*$. For the purposes of order theory, these operators are essentially equivalent. Indeed, in our situation the operator $a_{KN}(X,D) - a_{KN^*}(X,D)$ is an operator of order $-1$. We calculate that
  %
  \[ \widehat{(a_{KN}(X,D) - a_{KN^*}(X,D))(f)}(\xi) = \int_{\RR^d} [\widehat{a_\xi}(\xi - \eta) - \widehat{a_\eta}(\xi - \eta)] \widehat{f}(\eta)\; d\eta. \]
  %
  TODO (we do some singular integral type manipulations).
\end{remark}

\section{An Algebra of Operators}

s
















\chapter{Spectral Analysis of Singularities}

Suppose $u$ is a compactly supported distribution on $\RR^d$. The \emph{singular support} of a distribution $u$ are the set of points $x_0 \in \RR^d$ which \emph{do not} have an open neighbourhood upon which $u$ acts as integration against a $C^\infty$ function. Understanding the singular support of a distribution, and how to control it, is often a useful perspective in harmonic analysis. For instance, to reduce the study of $u$ to the study of a $C^\infty$ function one need only smoothen around the singular support of $u$.

The smoothness of a distribution is linked to the decay of it's Fourier transform. In particular, suppose there is a compactly supported bump function $\phi \in C^\infty(\RR^d)$ with $\phi(x) = 1$ in a neighbourhood of some point $x_0 \in \RR^d$. Since $\phi u$ is compactly supported, $\widehat{\phi u}$ is an analytic function. If for all $N \geq 0$, we find
%
\begin{equation}
  |\widehat{\phi u}(\xi)| \lesssim_N \frac{1}{1 + |\xi|^N}
\end{equation}
%
then we conclude $\phi u \in C^\infty(\RR^d)$. Thus we can infer the singular support of $u$ via purely spectral means, provided we are first able to localize about a point.

We can also gain more detailed information about the singularities of a distribution $u$ through the Fourier transform. If $x_0$ is a singularity of $u$, then for any bump function $\phi \in C^\infty(\RR^d)$ with $\phi(x) = 1$ in a neighbourhood of $x_0$, there must exist a value $\xi_0 \neq 0$ such that there exists no conical neighbourhood $U$ from the origin containing $\xi_0$ such that for all $\xi \in U$ and all $N > 0$,
%
\begin{equation} \label{nonsingularfourierdecay}
  |\widehat{u \phi}(\xi)| \lesssim_N \frac{1}{1 + |\xi|^N}.
\end{equation}
%
Since the set of such values $\xi_0$ itself forms a closed conical set about the origin, a compactness argument shows that the set of values $\xi_0$ which does not satisfy \eqref{nonsingularfourierdecay} for any choice of bump function $\phi$ around $x_0$ is nonempty. This is called the \emph{wavefront} of $u$ about the singularity $x_0$. The set
%
\[ \text{WF}(u) = \{ (x_0,\xi_0) : \xi_0\ \text{is in the wavefront of $u$ at $x_0$} \} \]
%
is the \emph{wavefront} of the distribution, and provides a deeper characterization of the singularities of $u$. For instance, in order to smoothen out a distribution $u$ one need only average along the directions in the wave-front set.

\begin{remark}
  Why does this not defy the uncertainty principle heuristically?
\end{remark}

Let us now argue a little more precisely. If $u$ is a compactly supported distribution on $\RR^d$, we define $\Gamma(u)$ to be the set of $\xi_0 \in \RR^d$ which have no conical neighbourhood $U$ such that for each $N > 0$ and $\xi \in U$,
%
\begin{equation} \label{fastDecayEquation}
  |\widehat{u}(\xi)| \lesssim_N \frac{1}{1 + |\xi|^N}.
\end{equation}
%
It is simple to verify that if $\Gamma(u) = \emptyset$, then $u \in C^\infty(\RR^d)$.

\begin{lemma} \label{wavefrontlocalizationlemma}
  If $u$ is a compactly supported distribution and $\phi \in C_c^\infty(\RR^d)$, then
  %
  \[ \Gamma(\phi u) \subset \Gamma(u). \]
\end{lemma}
\begin{proof}
  Suppose $\xi_0 \not \in \Gamma(u)$, so $\xi_0$ has a conical neighbourhood $U$ such that \eqref{fastDecayEquation} holds. Then there exists $\varepsilon > 0$ such that $U$ contains
  %
  \[ \left\{ \eta \in \RR^d : \frac{\xi_0 \cdot \eta}{|\xi_0| |\eta|} \geq 1 - 2\varepsilon \right\} \]
  %
  Let $V$ be the conical neighbourhood of $\xi_0$ defined by setting
  %
  \[ V = \left\{ \eta \in \RR^d : \frac{\xi_0 \cdot \eta}{|\xi_0| |\eta|} \geq 1 - \varepsilon \right\}. \]
  %
  We claim $V$ satisfies \eqref{fastDecayEquation}. Fix $\xi \in V$. Then
  %
  \[ |\widehat{\phi u}(\xi)| = (\widehat{\phi} * \widehat{u})(\xi) = \int_{\RR^d} \widehat{\phi}(\eta) \widehat{u}(\xi - \eta)\; d\xi. \]
  %
  If $|\xi - \eta| \leq 0.25 \varepsilon |\xi|$, then it is simple to verify that
  %
  \[ (\xi_0 \cdot \eta) \geq (1 - 2\varepsilon) |\xi_0| |\eta| \]
  %
  so $\eta \in U$. Thus for any $N > 0$, $\widehat{u}(\eta) \lesssim_N 1/(1 + |\eta|)^N$. Since $\phi \in L^\infty(\RR^d$, we conclude
  %
  \begin{align*}
    \int_{|\eta| \leq 0.25 \varepsilon |\xi|} \widehat{\phi}(\eta) \widehat{u}(\xi - \eta)\; d\xi &\lesssim_{\phi} \int_{|\eta| \leq 0.25 \varepsilon |\xi|} \frac{1}{1 + |\xi - \eta|^N}\\
    &\lesssim_{\varepsilon,d} \frac{|\xi|^d}{(1 + 2 |\xi|^{N})} \lesssim \frac{1}{1 + |\xi|^{N-d}}.
  \end{align*}
  %
  On the other hand, since $u$ is compactly supported, $\widehat{u}$ is slowly increasing, i.e. there exists $m > 0$ such that
  %
  \[ |\widehat{u}(\xi)| \leq 1 + |\xi|^m. \]
  %
  Since $\phi \in C_c^\infty(\RR^d)$, we have $|\widehat{\phi}(\eta)| \lesssim_M 1/(1 + |\eta|^M)$ for all $M > 0$ and thus we conclude that if $M > m + d$
  %
  \begin{align*}
    \int_{|\eta| \geq 0.25 \varepsilon |\xi|} \widehat{\phi}(\eta) \widehat{u}(\xi - \eta) &\lesssim_M \int_{|\eta| \geq 0.25 \varepsilon |\xi|} \frac{1 + |\xi - \eta|^m}{1 + |\eta|^M}\\
    &\lesssim_{\varepsilon,m} \int_{|\eta| \geq 0.25 \varepsilon |\xi|} \frac{1 + |\eta|^m}{1 + |\eta|^M}\\
    &\lesssim_{\varepsilon,d} \frac{1}{1 + |\xi|^{M-m-d}}.
  \end{align*}
  %
  Choosing the parameter $M$ and $N$ appropriately, we obtain the required bound which shows that $\xi_0 \not \in \Gamma(\phi u)$.
\end{proof}

This fact means we can obtain a consistant localization about a point. If $u$ is a distribution, and $\phi_1,\phi_2 \in C_c^\infty(\RR^d)$ are given and the support of $\phi_2$ is compactly supported on the support of $\phi_1$, then $\phi_2/\phi_1 \in C_c^\infty(\RR^d)$, and so we conclude that
%
\[ \Gamma(\phi_2 u) = \Gamma((\phi_2/\phi_1) \phi_1 u) \subset \Gamma(\phi_1 u). \]
%
Thus if $u$ is a distribution, and $x \in \RR^d$, then we define $\Gamma_x(U)$ to be equal to
%
\[ \bigcap \left\{ \Gamma(\phi u) : \phi \in C_c^\infty(\RR^d), x \in \text{supp}(\phi) \right\}. \]
%
It is simple to see that if $\{ \phi_n \}$ is a sequence in $C_c^\infty(\RR^d)$ such that $\text{supp}(\phi_{n+1})$ is compactly supported in $\text{supp}(\phi_n)$ for each $n$, and if $\bigcap \text{supp}(\phi_n) = \{ x \}$, then $\Gamma_x(u) = \lim_{n \to \infty} \Gamma(\phi_n u)$. Finally, we define
%
\[ \text{WF}(u) = \{ (x,\xi): \xi \in \Gamma_x(u) \}. \]
%
This is the \emph{wavefront set} of $u$.

\begin{lemma}
  If $u$ is a compactly supported distribution, then the projection $\pi_x(\text{WF}(u))$ is the singular support of $u$, and the projection $\pi_\xi(\text{WF}(u))$ is the set $\Gamma(u)$ of singular frequencies.
\end{lemma}
\begin{proof}
  Fix $x \in \RR^d$ and suppose $x \not \in \pi_x(\text{WF}(u))$. Then by a compactness argument, there exists $\phi \in C_c^\infty(\RR^d)$ with $\phi(x) \neq 0$ and with $\Gamma(\phi u) = \emptyset$, which implies $\phi u \in C^\infty(\RR^d)$. But this means that $u$ is $C^\infty$ in a neighbourhood of $x$, so $x$ is not a singular point.

  Now suppose $\xi \in \RR^d$ and $\xi \not \in \pi_\xi(\text{WF}(u))$.
\end{proof}

\begin{example}
  Suppose $u$ is a homogenous distribution which is $C^\infty$ away from the origin. Then $\widehat{u}$ is homogenous and $C^\infty$ away from the origin, and we claim that
  %
  \[ \text{WF}(u) = \{ (0,\xi): \xi \in \text{supp}(\widehat{u}) \}. \]
  %
  Since the singular support of $u$ is $\{ 0 \}$, it suffices to calculate $\Gamma_0(u)$. Fix a radial bump function $\phi \in C_c^\infty(\RR^d)$ with $\phi(0) = 1$. If $\xi_0$ is \emph{not} in the support of $\widehat{u}$, then $\widehat{u}$ vanishes on a conical neighbourhood of $\xi_0$, and so it follows from similar arguments to Lemma \ref{wavefrontlocalizationlemma} that $\xi_0 \not \in \Gamma(u \phi)$. Conversely, suppose $\widehat{u}(\xi_0) \neq 0$. Let $\beta > 0$ be the degree of $\widehat{u}$. For each $\varepsilon > 0$, let
  %
  \[ u_\varepsilon = \text{Dil}_{1/\varepsilon} \phi \cdot u. \]
  %
  To show $\xi_0 \in \Gamma_0(u)$ it suffices to show that $\xi_0 \in \Gamma(u_\varepsilon)$ for all $\varepsilon > 0$. Without loss of generality, we may assume $|\xi_0| = 1$ and $u(\xi_0) = 1$. We calculate using homogeneity that
  %
  \begin{align*}
    \widehat{u_\varepsilon} &= \varepsilon^d \cdot (\text{Dil}_\varepsilon \widehat{\phi}) * \widehat{u}\\
    &= \text{Dil}_\varepsilon (\widehat{\phi} * \text{Dil}_{1/\varepsilon} \widehat{u})\\
    &= \varepsilon^{-\beta} \cdot \text{Dil}_\varepsilon (\widehat{\phi} * \widehat{u}).
  \end{align*}
  %
  Thus it suffices to show that $\xi_0 \in \Gamma(\phi u)$ to conclude that $\xi_0 \in \Gamma(u_\varepsilon)$ for each $\varepsilon > 0$. For each $R > 0$, we have
  %
  \[ \widehat{u_\varepsilon}(R \xi_0) = \int_{\RR^d} \widehat{\phi}(R \xi_0 - \xi) \widehat{u}(\xi)\; d\xi. \]
  %
  Now for any $K,N > 0$, for $|R \xi_0 - \xi| \geq cR$ we have
  %
  \[ |(\nabla^K \phi)(R \xi_0 - \xi)| \lesssim_{K,N} \frac{1}{1 + R^N}, \]
  %
  which implies, interpreting the integral as a principal value if $\beta < 0$, that for any $N > 0$,
  %
  \[ \left| \int_{|R \xi_0 - \xi| \geq cR} \widehat{\phi}(R \xi_0 - \xi) \widehat{u}(\xi)\; d\xi \right| \lesssim_N \frac{1}{1 + R^N}. \]
  %
  Since $\widehat{u}$ is continuous and homogenous away from the origin, there exists $c \in (0,1)$ such that for any $R > 0$ and any $\xi \in \RR^d$ with $|\xi - R \xi_0| \leq cR$,
  %
  \[ \left| \widehat{u}(\xi) - |\xi|^\beta \right| \leq |\xi|^\beta / 2. \]
  %
  Thus
  %
  \[ \int_{|R \xi_0 - \xi| \leq cR} \widehat{\phi}(R \xi_0 - \xi) \widehat{u}(\xi)\; d\xi = R^\beta \int_{|R \xi_0 - \xi| \leq cR} \widehat{\phi}(R \xi_0 - \xi)\; d\xi + O \left( R^{d+\beta} \right). \]
\end{example}


One important relation between $u$ and $\text{WF}(u)$ is the \emph{propogation of singularities theorem}. If $u$ is a solution to a linear partial differential equation
%
\[ \sum_{|\alpha| \leq K} a_\alpha(x) (\partial_\alpha u)(x) = v \]
%
where $v$ is a distribution, then for any $(x,\xi) \in \text{WF}(u) - \text{WF}(v)$,
%
\[ q(x,\xi) = \sum_{|\alpha| \leq K} a_\alpha(x) \xi^\alpha = 0, \]
%
and $\text{WF}(u) - \text{WF}(v)$ is invariant under the flow generated by the Hamiltonian vector field
%
\[ H_{x,\xi} = \sum_{i = 1}^d \frac{\partial q}{\partial x^j} \frac{\partial}{\partial \xi^j} - \frac{\partial q}{\partial \xi_j} \frac{\partial}{\partial x^j}. \]
%
As a particular example, if $u(t,x,y)$ is a distributional solution to the wave equation $u_{tt} = \Delta u$ and we let $v_t(x,y) = u(t,x,y)$, then $\Delta v_t = u_{tt}$, and so by the propogation of singularities theorem $\text{WF}(v_t) \subset \text{WF}(u_{tt})$.

Then the Paley-Wiener theorem implies that $\widehat{u}$ is an analytic function on $\RR^d$. If $\widehat{u}$ decays rapidly, then $u$ is also a smooth function. However, even if $u$ is not smooth, $\widehat{u}$ may still decrease rapidly in certain directions, which implies that the singularities of $u$ `propogate' in certain directions and understanding these directions is often useful to understanding the distribution $u$. We can also get even more information about the distribution $u$ by looking at the singular frequencies.

To begin with, let 

To begin with, a distribution $u$ is \emph{nonsingular} at a point $x \in \RR^d$ if $u$ is locally a $C^\infty$ function in a neighbourhood of $x$, i.e. there exists a bump function $\phi \in C^\infty(\RR^d)$ with $\phi(x) \neq 0$ such that $\phi u \in C^\infty(\RR^d)$. The  \emph{singular support} of a compactly supported distribution $u$ to be the set of all points $x \in \RR^d$ upon which $u$ is not nonsingular.