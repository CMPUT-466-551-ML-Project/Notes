\documentclass{article}

\usepackage{amsthm}
\usepackage{amsmath}
\usepackage{multicol}
\usepackage{amssymb}
\usepackage{mathabx}
\usepackage{accents}
\usepackage[margin=0.7in]{geometry}
\usepackage[english]{babel}
\usepackage{blindtext}

\theoremstyle{plain}
\newtheorem{lemma}{Lemma}
\newtheorem{prop}{Proposition}
\newtheorem*{example}{Example}
\newtheorem*{fact}{Fact}
\newtheorem*{corollary}{Corollary}

\usepackage{algorithm}
\usepackage[noend]{algpseudocode}

\theoremstyle{plain}
\newtheorem{theorem}{Theorem}
\newtheorem{proposition}[theorem]{Proposition}
\newtheorem*{remark}{Remark}

\def\changemargin#1#2{\list{}{\rightmargin#2\leftmargin#1}\item[]}
\let\endchangemargin=\endlist 

\DeclareMathOperator{\codim}{codim}

\title{Fractals Avoiding Fractal Sets}

\author{Jacob Denson\\ \and Malabika Pramanik\\ \and Josh Zahl}

\begin{document}

\maketitle

\begin{multicols}{2}

\begin{abstract}
	\blindtext[1]
\end{abstract}

Suppose you have to determine the stability of an operator on function spaces. Thinking geometrically, one can study how the operator acts on indicator functions. The existence of interesting characteristics in sets often gives insight into how the operator behaves on the corresponding indicator function. Proving the operator's boundedness then reduces to determining how large sets can be possessing certain characteristics.

% TODO: Put particular example in introduction.

In this paper, we describe methods to find large sets avoiding fine-scale patterns. Important examples include affine configurations, such as sets not containing the vertices of equilateral triangles, sets not containing three term arithmetic progressions, sets not generating particular families of angles, and sets not containing points in a common hyperplane. For these examples, the Lebesgue density theorem implies any set of positive measure contains these patterns. Thus we quantify the size of sets by their Hausdorff dimension.

There are two approaches to the pattern avoidance problem. We get upper bounds by proving sets with large Hausdorff dimension contains patterns. Constructing sets avoiding patterns with large Hausdorff dimension give lower bounds. In this paper, we focus on the {\it construction problem} for lower bounding pattern avoidance problems.

There are already generic pattern avoidance methods in the literature. We compare our method to them in detail in section 6. But these rely on the non-singularity of the patterns, lying on a smooth function. The novel feature of our method is we can avoid points with patterns lying on an {\it arbitrary} fractal set, and the Hausdorff dimension of our constructions is still comparable to previous methods.

The {\it key idea} of our method is the introduction of a new geometric framework for pattern avoidance problems, described in section 1. A simple combinatorial argument, described in section 2, exploited repeatedly in section 3 via a queueing process analogous to the method in \cite{MalabikaRob} is all that is needed to construct the pattern avoiding set. We believe this new geometric framework should help find further methods in the field. We show this by proving another pattern avoidance result in section 5, assuming extra geometric information on the patterns.

\section{A Fractal Avoidance Framework}

The common framework used to think about pattern avoidance problems is to think of the pattern as being specified by the zero set of some function.
%
\begin{itemize}
	\item A set $X \subset \mathbf{R}^d$ contains the vertices of no equilateral triangles if and only if for any three distinct $x,y,z \in X$,
	%
	\[ f(x,y,z) = d(x,y) + d(y,z) - 2d(x,z) \neq 0 \]

	\item A set $X \subset \mathbf{R}$ contains no three term arithmetic progressions if and only if for any distinct $x,y,z \in X$,
	%
	\[ f(z,y,z) = (x - y) - (x - z) - 2(x - z) \neq 0 \]

	\item A set $X \subset \mathbf{R}^d$ does not contain a family of angles $\{ \alpha_i \}$ if and only if for any distinct $x,y,z \in X$, and $i$,
	%
	\[ f(x,y,z) = \frac{(x - z) \cdot (y - z)}{|x - z||y - z|} \neq \cos(\alpha_i) \]

	\item A set $X \subset \mathbf{R}^d$ does not contain $d$ points in a hyperplane if and only if for any distinct $x_1, \dots, x_d \in X$,
	%
	\[ f(x_1, \dots, x_d) = \det(x_1, \dots, x_d) \neq 0 \]
\end{itemize}
%
Thus the common way to phrase generic solutions to the configuration avoidance problem is in terms of a function:

\begin{changemargin}{0.5em}{0em}
{\bf The Configuration Avoidance Problem:} Given a function $f: (\mathbf{R}^d)^n \to \mathbf{R}$ as input, find $X \subset \mathbf{R}^d$ with high Hausdorff dimension such that for any {\it distinct} $x_1, \dots, x_n \in X$, $f(x_1, \dots, x_n) \neq 0$.
\end{changemargin}

This is the viewpoint of \cite{MalabikaRob} and \cite{Mathe}, who give results assuming various regularity conditions on the function $f$. It is the viewpoint of this paper that the function $f$ contains extraneous information which is not really useful to the problem. The only important information we need to extract from the function $f$ is the geometric structure of it's zero set. If we denote the zero set of $f$ by $Z$, the problem becomes equivalent to a more flexible framework:

\begin{changemargin}{0.5em}{0em}
	{\bf The Fractal Avoidance Problem:} Given $Z \subset (\mathbf{R}^d)^n$, find a set $X \subset \mathbf{R}^d$ such that $X^d \cap Z \subset \Delta$, where $\Delta = \{ x \in (\mathbf{R}^d)^n : x_i = x_j\; \text{for some $i$ and $j$} \}$, with as high a Hausdorff dimension as possible.
\end{changemargin}

A natural goal is to solve the generic fractal avoidance problem with minimal assumptions. Thus $Z$ can take the form of an arbitrary fractal, and the only assumptions we place on $Z$ are it's fractal dimension.

\begin{theorem}
	If $Z \subset (\mathbf{R}^d)^n$ be a set with Minkowski dimension $\alpha$, then there exists a set $X$ solving the fractal avoidance problem for $Z$ with
	%
	\[ \dim_{\mathbf{H}}(X) = \frac{dn - \alpha}{n - 1} = \frac{\codim_{\mathbf{H}}(Z)}{n - 1} \]
\end{theorem}

% TODO: Describe why Delta is important in the problem description.

A second goal is to show an example where assuming additional geometric conditions on $Z$ can be used to derive a better result, showing this framework extends to further methods in pattern avoidance. We consider the condition where $Z$ has {\it low rank}, in a certain sense.

\begin{theorem}
	Let $Z \subset (\mathbf{R}^d)^n$ be a set together with a projection $\pi$ such that $\pi(Z)$ is $\alpha$ dimensional. Then there exists a set $X$ solving the fractal avoidance problem for $Z$ with dimension
\end{theorem}

Because of the lack of any {\it rigid} geometric information about the set $Z$, such as smoothness, really the only way we can avoid $Z$ is by covering arguments. This can be neatly summarized as a combinatorial argument on graphs, which we detail in the next section.

\section{Avoidance at a Single Scale}

We now develop a discrete technique used to construct solutions to the fractal avoidance problem. It depends very little on the Euclidean structure of the plane. As such, we rephrase the construction as a combinatorial problem on graphs.

Recalling definitions, we say an {\it $n$ uniform hypergraph} is a collection of {\it vertices} and {\it hyperedges}, where a hyperedge is a set of $n$ distinct vertices. An {\it independent set} is a subset of vertices containing no complete set of vertices in any hyperedge of the graph. A {\it coloring} is a partition of the vertex set into finitely many independent sets, each of which we call a {\it color}. Such a coloring is {\it $K$ uniform} if each color class has $K$ elements.

The next lemma is a variant of Tur\'{a}n's theorem on independent sets. For technical reasons, we need an extra restriction on the independent set so it is `uniformly' chosen over the graph. This is why colorings are introduced.

\begin{lemma}
	Let $G$ be an $n$ uniform hypergraph with a $K$ uniform coloring. Then there is an independent set $W$ containing elements from all but $|E|/K^n$ colors.
\end{lemma}
\begin{proof}
	Let $U$ be a random vertex set chosen by selecting a vertex of each color uniformly randomly. Every vertex occurs in $U$ with probability $1/K$. For any edge $e = (v_1, \dots, v_n)$, the vertices $v_i$ all have different colors. Thus they have an independent chance of being added to $U$, and we calculate
	%
	\begin{align*}
		\mathbf{P}(v_1 \in U, \dots, v_n \in U) = \mathbf{P}(v_1 \in U) \dots \mathbf{P}(v_n \in U) = 1/K^n
	\end{align*}
	%
	If we let $E'$ denote the edges $e = (u_1, \dots, u_n)$ with $u_1, \dots, u_n \in U$, then
	%
	\[ \mathbf{E}|E'| = \sum_{e \in E} \mathbf{P}(e \in E') = \sum_{e \in E} 1/K^n = \frac{|E|}{K^n} \]
	%
	This means we may choose a {\it particular}, nonrandom $U$ for which $|E'| \leq |E|/K^n$. If we form a vertex set $W \subset U$ by removing, for each $e \in E'$, a vertex in $U$ adjacent to the edge, then $W$ is an independent set containing all but $|E'| \leq |E|/K^n$ colors.
\end{proof}

\begin{corollary}
	If $|V| \gtrsim N^a$, $|E| \lesssim N^b$, and $K \gtrsim N^c$, where $b < a + c(n-1)$, then as $N \to \infty$ we can find an independent set containing all but a fraction $o(1)$ of the colors.
\end{corollary}
\begin{proof}
	A simple calculation on the quantities of the previous lemma yields
	%
	\begin{align*}
		\frac{\# ( \text{colors removed} )}{\# ( \text{all colors} )} = \frac{|E|/K^n}{|V|/K} = \frac{|E|}{|V|K^{n-1}} \lesssim \frac{N^b}{N^{a + c(n-1)}}
	\end{align*}
	%
	This is $o(1)$ if $b < a + c(n-1)$.
\end{proof}

We now apply these constructions to a problem clearly related to the fractal avoidance problem. It will form our key method to construct fractal avoidance solutions. Given an integer $N$, we subdivide $\mathbf{R}^d$ into a lattice of side length $1/N$ cubes with corners on $\mathbf{Z}^d/N$, the collection of such cubes we will denote by $\mathcal{B}(1/N)$. This grid is used to granularize configuration avoidance.

\begin{theorem}
	Suppose $\mathcal{I}_1, \dots, \mathcal{I}_n$ are disjoint collections of cubes in $\mathcal{B}(1/N)$, with $|\mathcal{I}_i| \gtrsim N^d$. We assume the lower Minkowski dimension of $Y$ is bounded above by $\alpha$, and we have a rational parameter $\beta > d(n-1)/(n-\alpha)$. Then there exists arbitrarily large $N$ such that $N^\beta$ is an integer, and there exists collections of cubes $\mathcal{J}_1, \dots, \mathcal{J}_n \in \mathcal{B}(1/N^\beta)$ with each cube in $\mathcal{J}_1 \times \dots \times \mathcal{J}_n$ disjoint from $Y$, and as $N \to \infty$, each $\mathcal{J}_i$ contains cubes in all but a fraction $o(1)$ of cubes in $\mathcal{I}_i$.
\end{theorem}
\begin{proof}
	If $\mathcal{K} \subset \mathcal{B}(1/N^\beta)^n$ is the collection of all cubes in a side length $1/N^\beta$ lattice intersecting $Y$, then $|\mathcal{K}| \lesssim N^{\alpha \beta}$. We then let $\mathcal{I}'_i$ be all cubes in $\mathcal{B}(1/N^\beta)$ contained in $\mathcal{I}_i$. Considering these cubes as vertices gives us an $n$ uniform hypergraph $G$ with a hyperedge between $I_1 \in \mathcal{I}'_1, \dots, I_n \in \mathcal{I}'_n$ if $I_1 \times \dots \times I_n \in \mathcal{K}$. We say two cubes in $G$ are the same color if they are contained in a common cube in $\mathcal{I}_i$.

	Using the fact that a side length $1/N$ cube contains $N^{d(\beta - 1)}$ side length $1/N^\beta$ cubes, we conclude that $G$ has $\sum |\mathcal{I}_i| = N^{d(\beta - 1)} \sum |\mathcal{I}_i| \gtrsim N^{d \beta}$ vertices. The number of edges in $G$ is bounded by $|\mathcal{K}| \lesssim N^{\alpha \beta}$. Finally, the coloring is $N^{d(\beta - 1)}$ uniform. Thus in the terminology of the previous corollary, $a = d \beta$, $b = \alpha \beta$, and $c = d(\beta - 1)$, and the inequality in the hypothesis of this theorem is then equivalent to the inequality in the hypothesis of the corollary. Applying the corollary gives the required result.
\end{proof}

The value $d(n-1)/(n-\alpha)$ in the theorem is directly related to the dimension $(n-\alpha)/(n-1)$ we obtain in our main result. Any improvement on this bound for special cases of the fractal avoidance problem immediately will lead to improvements on the Hausdorff dimension of the set constructed. The fact that our hypergraph result is tight indicates that for the general fractal avoidance problem, our construction gives tight bounds.

% TODO: Include Tightness Calculation?

\section{A Fractal Avoiding Set}

Our solutions $X$ to fractal avoidance problems will be obtained by breaking the problem down into a sequence of discrete configuration problems. The central idea was first used in \cite{MalabikaRob}. We construct $X$ as a limit $\lim X_N$, where $X_N$ is a disjoint union of side length $L_N$ cubes, and $X_{N+1}$ is obtained from $X_N$ by subdividing the cubes into length $R_N$ cubes, then further subdividing these cubes into cubes of smaller side length $L_{N+1}$, and removing a portion of them.

At each step $N$, we consider a disjoint collection of side length $R_N$ cubes $\mathcal{I}_1(N), \dots, \mathcal{I}_n(N) \subset \mathcal{B}(R_N)$, each cube contained in $X_N$. The main result of the previous section allows us to find a collection of side length $L_{N+1} = R_N^{\beta_N}$ cubes $\mathcal{J}_i(N) \subset \mathcal{I}_i(N)$ with all cubes in $\mathcal{J}_1(N) \times \dots \times \mathcal{J}_n(N)$ disjoint from $Y$, and where $\beta_N$ converges to $\beta = d(n-1)/(n-\alpha)$ from above. We then form $X_{N+1}$ from $X_N$ by removing the parts of cubes in $\mathcal{I}_i(N)$ which are not contained in the cubes in $\mathcal{J}_i(N)$. Once parameters are fixed, and an initial set $X_0$ is chosen, which we might as well assume to be $[0,1]^d$, we obtain a sequence $X_0, X_1, \dots$ converging to a set $X$. A simple constraint detailed below is all that is required to ensure that $X$ is a solution to the fractal avoidance problem.

\begin{lemma}
	Suppose that for any choice of distinct $x_1, \dots, x_n \in X$, there exists $N$ such that each $x_i$ is contained in a cube in $\mathcal{I}_i(N)$. Then $X^d \cap Y \subset \Delta$.
\end{lemma}
\begin{proof}
	For then $x_1, \dots, x_n \in X_{N+1}$, so $x_1 \in \mathcal{J}_1(N), \dots, x_n \in \mathcal{J}_n(N)$, and so the tuple $(x_1, \dots, x_n)$ is contained in a cube in $\mathcal{J}_1(N) \times \dots \times \mathcal{J}_n(N)$, which is disjoint from $Y$. Taking contrapositives of this argument shows that if $y \in X^d \cap Y$, then there must be some $i$ and $j$ for which $y_i = y_j$, so $y \in \Delta$.
\end{proof}

% TODO: Include diagram of construction of queueing.

We achieve the constraint in the lemma by dynamically choosing parameters subject to a queueing process. The queue will consist of an ordered sequence of tuples $(I_1, \dots, I_n)$, where $I_1 ,\dots, I_n$ are disjoint cubes. At stage $N$ of the construction, we take off the front tuple $(I_1, \dots, I_n)$ from the queue, and set $\mathcal{I}_i(N)$ to be the set of all cubes in $\mathcal{B}(R_N)$ which are a subset of both $I_i$ and $X_N$. We then subdivide $X_N$ using these parameters to form the set $X_{N+1}$ as a union of length $L_{N+1} = R_N^\beta$ intervals. After this, for {\it any} ordered choice of distinct intervals $I_1, \dots, I_n \in \mathcal{B}(L_{N+1})$, with each interval $I_i$ a subset of $X_{N+1}$, we add the tuple $(I_1, \dots, I_n)$ to the end of the queue.

Provided that $L_N \to 0$, which will of course be the case, then for any distinct choice of $x_1, \dots, x_n \in X$, there exists $N$ and $L_N$ such that $|x_i - x_j| \geq 2 L_N$ for all $i \neq j$. Thus at stage $N$ of the construction, a tuple $(I_1, \dots, I_n)$ is added to the end of the queue with $x_i \in I_i$, and at a {\it much} {\it much} later stage $M$ of the construction, this tuple is popped off the front of the queue, and so each $x_i$ is contained in a cube in $\mathcal{I}_i(M)$. Thus we conclude that $X$ is a solution to the fractal avoidance problem.

\section{Dimension Bounds}

To complete the proof, it suffices to choose the parameters $R_N$ and $\beta_N$ which lead to the correct Hausdorff dimension bound on $X$. The actual choice of $\beta_N$ doesn't matter, only that it is an increasing sequence converging to $\beta$ in the limit. We also fix a decreasing sequence $\lambda_N$ such that $\lambda_N \beta_N > d$, to be used later on in our argument. Since $\beta_N$ converges to $\beta$ from above, we can let $\lambda_N$ tend to $\lambda = (dn - \alpha)/(n - 1)$ from below. The fact that the dissection of $X_{N+1}$ for $X_N$ occurs uniformly over the will aid us in annihilating the super-exponentially increasing constants which inherently occur from the exponentially decreasing values of $L_N$ we are forced to choose.

We rely on the mass distribution principle to construct a probability measure $\mu$ supported on $X$. This enables us to calculate the Hausdorff dimension of $X$ using Frostman's lemma. We begin by putting the uniform probability measure $\mu_0$ on $X_0 = [0,1]^d$. Then, at each stage of the construction, we construct $\mu_{N+1}$ from $\mu_N$ by taking the mass on a certain side length $L_N$ cube in $X_N$, and uniformly distributing it's mass over the side length $L_{N+1}$ cubes in $I \cap X_{N+1}$. Using the weak compactness of the unit ball in $L^1(\mathbf{R}^d)^*$, we obtain a weak limit $\mu = \lim \mu_n$. The fact that $\mu_n$ is supported on $X_n$ for each $n$ implies $\mu$ is supported on $X$.

It is intuitive that the mass on $\mu$ will be distributed more thinly at each stage the fatter the cubes that are kept. Quantifying this precisely allows us to apply Frostman's lemma. More precisely, we will prove that for each length $L$ interval $I$, $\mu(I) \lesssim_N L^{\lambda_N}$. Thus Frostman's lemma guarantees that $\dim_{\mathbf{H}}(X) \geq \lambda_N$, and taking $\lambda_N \to \lambda$ will complete the proof.

\begin{lemma}
	For $R_N \gg 0$, if $I \in \mathcal{B}(L_{N+1})$ and $J \in \mathcal{B}(L_N)$,
	%
	\[ \mu(I) \leq 2 (R_N/L_N)^d \mu(J)\ \ \ \mu(I) \leq 2^N R_0^{d - \beta_0} \dots R_N^{d - \beta_N} R_N^d \]
\end{lemma}
\begin{proof}
	If $I$ is not a cube in $X_{N+1}$, then $\mu(I) = \mu_{N+1}(I) = 0$, so the inequality is obviously true. Otherwise, we can find a cube $J \in \mathcal{B}(L_N)$ in $I \cap X_N$. $J$ contains $(L_N/R_N)^d$ sidelength $R_N$ cubes. Our main discrete result implies that $X_{N+1}$ contains a sidelength $L_{N+1}$ cube in all but a fraction $o(1)$ of these cubes,. In particular, if we choose $R_N$ sufficiently large, then we know that we keep a sidelength $L_N$ portion of at least half of these cubes. Thus
	%
	\begin{align*}
		\mu(I) &= \mu_{N+1}(I) \leq \frac{\mu_N(J)}{(L_N/R_N)^d/2}\\
		&= 2 \mu_N(J) (R_N/L_N)^d = 2 \mu(J) (R_N/L_N)^d
	\end{align*}
	%
	completing the calculation. Applying this calculation iteratively, we conclude
	%
	\begin{align*}
		\mu(I) &\leq 2^N (R_0/L_0)^d (R_1/L_1)^d \dots (R_N/L_N)^d\\
		&= 2^N R_0^{d - \beta_0} \dots R_{N-1}^{d - \beta_{N-1}} R_N^d
	\end{align*}
	%
	completing the calculation.
\end{proof}

\begin{corollary}
	If $R_N \gg 0$, $\mu(I) \leq L_N^{\lambda_N}$ for $I \in \mathcal{B}(L_N)$.
\end{corollary}
\begin{proof}
	We write the inequality in the last problem as
	%
	\[ \mu(I) \leq [2^N R_0^{d - \beta_0} \dots R_{N-1}^{d - \beta_{N-1}} R_N^{d - \lambda_N \beta_N}] L_{N+1}^{\lambda_N} \]
	%
	Since $\lambda_N \beta_N > d$, the quantity in the square brackets is $o(1)$ as $R_N \to \infty$. Thus for sufficiently large $R_N$, we conclude that $\mu(I) \leq L_N^{\lambda_N}$.
\end{proof}

This is almost the required inequality, except we have only proven it for intervals at particular scales. To obtain a general inequality, we use the fact that our construction is obtained uniformly across all intervals.

\begin{theorem}
	If $R_N$ is chosen large enough that the previous inequalities hold, then we have $\mu(I) \leq 2^{d+1} L^{\lambda_N}$ for all intervals $I$ with side length $L \leq L_N$.
\end{theorem}
\begin{proof}
	We break our analysis into three cases, depending on the size of $L$:
	%
	\begin{itemize}
		\item If $R_N \leq L \leq L_N$, we can cover $I$ by at most $2^d(L/R_N)^d$ cubes in $\mathcal{B}(R_N)$. For each such cube, we know that the mass on each side length $R_N$ cube is at most $2(R_N/L_N)^d$ times the mass on an element of $\mathcal{B}(L_N)$. Thus
		%
		\begin{align*}
			\mu(I) &\leq [2^d(L/R_N)^d] [2(R_N/L_N)^d] [L_N^{\lambda_N}]\\
			&\leq \frac{2^{d+1} L^d}{L_N^{d - \lambda_N}} \leq 2^{d+1} L^{\lambda_N}
		\end{align*}
		%
		which gives the required result.

		\item If $L_{N+1} \leq L \leq R_N$, we can cover $L$ by at most $2^d$ cubes in $\mathcal{B}(R_N)$. Each cube in $\mathcal{B}(R_N)$ contains at most one cube in $\mathcal{B}(L_{N+1})$ which is also contained in $X_{N+1}$, so the bound in the last corollary gives that $\mu(I) \leq 2^d L_{N+1}^{\lambda_N} \leq 2^d L^{\lambda_N}$.

		\item If $L \leq L_{N+1}$, there certainly exists $M$ such that $L_{M+1} \leq L \leq L_M$, and one of the previous cases yields that $\mu(I) \leq 2^{d+1} L^{\lambda_M} \leq 2^{d+1} L^{\lambda_N}$.
	\end{itemize}
	%
	This covers all possible situations, completing the proof.
\end{proof}

To employ Frostman's lemma, we need the result $\mu(I) \lesssim L^{\lambda_N}$ for an {\it arbitrary} interval, not just one with $L \leq L_N$. But this is no trouble; it is only the behavior of the measure on arbitrarily small scales that matters. This is because if $L \geq L_N$, then $\mu(I)/L^{\lambda_N} \leq 1/L_N^{\lambda_N} \lesssim_N 1$, so $\mu(I) \lesssim_N L^{\lambda_N}$ holds automatically for all sufficiently large intervals. Thus all problems with the Hausdorff dimension argument are complete, and we have proven that there is a choice of parameters which constructs a set $X$ with Hausdorff dimension no less than $(nd - \alpha)/(n-1)$ (and by looking at the way we dissect our intervals at each scale, it is easy to see that $X$ has {\it precisely} this dimension).

\section{Low Rank Projection Method}

\section{Comparison with Other Generic Avoidance Schemes}

\section{Concluding Remarks}

\begin{thebibliography}{9}

\bibitem{RuzsaSetsWithoutSquares}
I. Z. Ruzsa
\textit{Difference Sets Without Squares}

\bibitem{KeletiDimOneSet}
Tam\'{a}s Keleti
\textit{A 1-Dimensional Subset of the Reals that Intersects Each of its Translates in at Most a Single Point}

\bibitem{MalabikaRob}
Robert Fraser, Malabika Pramanik
\textit{Large Sets Avoiding Patterns}

\bibitem{Sudakov}
B. Sudakov, E. Szemer\'{e}di, V.H. Vu
\textit{On a Question of Erd\"{o}s and Moser}

\bibitem{Mathe}
A. Ma\'{t}h\'{e}
\textit{Sets of Large Dimension Not Containing Polynomial Configurations}

\end{thebibliography}

\end{multicols}

%- Berend's counterexample is a discrete version of Kaleti's continuous counterexample for 3APs
%- Look up Wisewell functions
%- (x_2 - x_1)^2 = x_3 - x_1: Comes up in Bourgain & Chang

%- Principal character gives main term, rest should be thrown into the error term.   

\end{document}