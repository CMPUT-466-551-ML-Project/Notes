\documentclass[dvipsnames,letterpaper,12pt]{article}

\usepackage[margin = 1.0in]{geometry}
\usepackage{amsmath,amssymb,graphicx,mathabx,accents}
\usepackage{enumerate,mdwlist}

\usepackage{tikz}

%\setlist[enumerate]{label*={\normalfont(\Alph*)},ref=(\Alph*)}

\numberwithin{equation}{section}

\usepackage{amsthm}

\usepackage{hyperref}

\usepackage{verbatim}

\usepackage{nag}

\DeclareMathOperator{\minkdim}{\dim_{\mathbb{M}}}
\DeclareMathOperator{\hausdim}{\dim_{\mathbb{H}}}
\DeclareMathOperator{\lowminkdim}{\underline{\dim}_{\mathbb{M}}}
\DeclareMathOperator{\upminkdim}{\overline{\dim}_{\mathbb{M}}}
\DeclareMathOperator{\fordim}{\dim_{\mathbb{F}}}

\DeclareMathOperator{\lhdim}{\underline{\dim}_{\mathbb{M}}}
\DeclareMathOperator{\lmbdim}{\underline{\dim}_{\mathbb{MB}}}

\DeclareMathOperator{\RR}{\mathbb{R}}
\DeclareMathOperator{\ZZ}{\mathbb{Z}}
\DeclareMathOperator{\QQ}{\mathbb{Q}}
\DeclareMathOperator{\TT}{\mathbb{T}}
\DeclareMathOperator{\CC}{\mathbb{C}}

\DeclareMathOperator{\B}{\mathcal{B}}

\newtheorem{theorem}{Theorem}
%\newtheorem{lemma}{Lemma}
%\newtheorem{corollary}{Corollary}
\newtheorem{lemma}[theorem]{Lemma}
\newtheorem{corollary}[theorem]{Corollary}
%\newtheorem{prop}[theorem]{Proposition}
\newtheorem{remark}[theorem]{Remark}
\newtheorem{remarks}[theorem]{Remarks}
%\newtheorem*{concludingremarks}{Concluding Remarks}
\numberwithin{theorem}{section}

\DeclareMathOperator{\EE}{\mathbb{E}}
\DeclareMathOperator{\PP}{\mathbb{P}}

\DeclareMathOperator{\DQ}{\mathcal{Q}}
\DeclareMathOperator{\DR}{\mathcal{R}}

\newcommand{\psitwo}[1]{\| {#1} \|_{\psi_2(L)}}
\newcommand{\TV}[2]{\| {#1} \|_{\text{TV}({#2})}}








\title{Large Salem Sets Avoiding Nonlinear Configurations}
\author{Jacob Denson\footnote{University of Madison Wisconsin, Madison, WI, jcdenson@wisc.edu}}

\begin{document}

\maketitle

\begin{abstract}
    We construct large Salem sets avoiding patterns, complementing previous constructions of pattern avoiding sets with large Hausdorff dimension. For a potentially uncountable family of uniformly Lipschitz functions $\{ f_i : (\TT^d)^{n-2} \to \TT^d \}$, we obtain a Salem subset of $\TT^d$ with dimension $d/(n-1)$ avoiding nontrivial solutions to the equation $x_n - x_{n-1} = f_i(x_1,\dots,x_{n-2})$ for each $i$. For a countable family of smooth functions $\{ f_i : (\TT^d)^{n-1} \to \TT^d \}$ satisfying a modest geometric condition, we obtain a Salem subset of $\TT^d$ with dimension $d/(n-3/4)$ avoiding nontrivial solutions to the equation $x_n = f(x_1,\dots,x_{n-1})$. For a set $Z \subset \TT^{dn}$ formed from the countable union of a family of sets, each with lower Minkowski dimension $s$, we obtain a Salem subset of $\TT^d$ of dimension $(dn - s)/(n - 1/2)$ whose Cartesian product does not intersect $Z$ except at points with non-distinct coordinates.
\end{abstract}

In these notes, we try to use decoupling to construct large sets avoiding solutions to certain equations.

\end{document}
