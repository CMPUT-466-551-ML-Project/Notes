\documentclass{article}

\usepackage{amsthm}
\usepackage{amsmath}
\usepackage{multicol}

\theoremstyle{plain}
\newtheorem{lemma}{Lemma}
\newtheorem{prop}{Proposition}
\newtheorem*{example}{Example}
\newtheorem*{fact}{Fact}
\newtheorem*{corollary}{Corollary}

\usepackage{algorithm}
\usepackage[noend]{algpseudocode}

\theoremstyle{plain}
\newtheorem{theorem}{Theorem}
\newtheorem{proposition}[theorem]{Proposition}

\title{Vehicle Routing}
\author{Jacob Denson}

% TODO: Zac mentioned Swalmie / Post as inventing the assymetric metric tree finding algorithm,
% but his paper mentions a technique developed in one of his other papers

\begin{document}

\maketitle

The travelling salesman problem asks to find the shortest cycle in a connected graph passing through each vertex. A practical formulation of this problem is that a door to door deliveryman is attempting to deliver packages to each house in a neighbourhood, and wants to travel between houses as efficiently as possible. The {\it vehicle routing problem} adds an additional challenge to the scenario: the deliveryman can only carry a limited supply of goods at any once time, and must periodically return to a central depot to resupply. The deliveryman still wants to deliver goods as efficiently as possible, but must now simultaneously cluster potential customers together while optimizing routing within these clusters to ensure that time spent refueling is not wasted. As one of the most famous NP complete problems, most work on the travelling salesman problem is spent deriving heuristics which give approximation guarantees on solutions. In this work, we collect research on heuristics obtained for the vehicle routing problem.

In the vehicle routing problem, we take a complete graph $G$, with vertices $V \cup \{ B \}$, where $B$ is a distinguished {\it depot node}. We are also given a metric $d$ between the vertices in the graph, as well as a capacity $C$. Each node on the graph, except for the depot node, has a fixed demand $D_v$. We shall let $r_v$ denote $d(B,v)$, the {\it radius} from $B$ to $v$. To begin with, we consider two variants of the vehicle routing problem. In the {\it split demand} problem, a feasible solution to this problem consists of a collection of cycles $T_1, \dots, T_n$, each cycle containing $B$, together with quantities $D^i_v$ for each vertex $v$ on the tour $T_i$, such that
%
\begin{itemize}
    \item Each tour remains within the capacity of the delivery vehicle, so that $\sum_v D^i_v \leq C$ holds for each tour $i$.

    \item Each vertex is completely supplied with the capacity which it demands, so that $\sum_i D^i_v = D_v$.
\end{itemize}
%
In the {\it unsplit demand} version of the problem, we are unable to split demand satisfaction of a single node between different tours; each node must be completely supplied by a single vertex. In this case, we can assume a feasible solution is a collection of cycles $T_1, \dots, T_n$ forming a disjoint cover of the vertices $V$, intersecting only at $B$, such that
%
\begin{itemize}
    \item Each tour remains within capacity, so that $\sum_{v \in T_i} D_v \leq C$ holds for each tour $T_i$. Of course, in this version of the problem we must therefore assume $D_v \leq C$ for each vertex $v$.
\end{itemize}
%
The goal of both the split and unsplit versions of vehicle routing is to find a feasible solution minimizing the combined total length of all cycles.

\part{Basic Results}

\section{Lower Bounds to Vehicle Routing}

In our introduction to vehicle routing, we emphasized how vehicle routing is an extension of the travelling salesman problem. The additional problem is to partition the graph into efficient, bounded demand clusters of nodes to travel through at once. Indeed, if the capacity of the salesman is greater than the sum of the demands of all vertices, than an optimal solution to the travelling salesman problem will give an optimal solution to the vehicle routing problem. Here we introduce strategies showing that feasible solutions to the travelling salesman problem can be adapted using certain refueling strategies to yield efficient approximations to vehicle routing. Given a vehicle routing instance $G$, define $T^*(G)$ to be the optimal length of a travelling salesman solution on the graph $G$, and let $V^*(G)$ denote the optimal length of a vehicle routing solution. The bounds below are established for the {\it split} version of the vehicle routing problem; since any solution to the unsplit problem is also a solution to the split version, the split optimal value is at least as optimal as the unsplit optimal value.

\begin{proposition} $T^*(G) \leq V^*(G)$
\end{proposition}
\begin{proof}
    The optimal solution of the vehicle routing is essentially just a feasible solution to the travelling salesman problem, because we can concatenate cycles to obtain a tour.
\end{proof}

\begin{proposition}$\frac{2}{C} \sum_v D_v r_v \leq V^*$
\end{proposition}
\begin{proof}
    We first note that for any vertex $v$ on the tour $T_i$, the tour $T_i$ consists of two paths from the depot to $v$, and hence the tour is longer than twice the distance from the depot to $v$. Thus $T^*(T_i) \geq 2 \max_{v \in T_i} r_v$. What's more, the values $D^i_v$, once normalized by $\sum D^i_v$, constitute a probability distribution over the vertices in tour $i$, and employing the general fact that $\max_{v \in T_i} r_v \geq \mathbf{E}[r_X]$ for any random variable $X$ distributed over the vertices in tour $i$, we conclude that
    %
    \[ T^*(T_i) \geq 2 \max_{v \in V_i} r_v \geq 2 \frac{\sum_v D^i_v r_v}{\sum_v D^i_v} \geq 2 \frac{\sum_v D^i_v r_v}{C} \]
    %
    where we have used the fact that $\sum_v D^i_v \leq C$. But now $V^*(G) = \sum T^*(T_i)$, and so we therefore conclude that
    %
    \[ V^*(G) = \sum_i T^*(T_i) \geq \sum_i 2 \frac{\sum_v D^i_v r_v}{C} = \frac{2}{C} \sum_v \left( \sum_i D^i_v \right) r_v = \frac{2}{C} \sum_v D_v r_v \]
    %
    This essentially says that the average radius of demands, weighted by the demands at each node with respect to the capacity, lower bounds the distance a routing problem must travel.
\end{proof}

This bound shows that, roughly, the expected radii of a random vertex in the graph, weighted by demand, lower bounds an optimum vehicle routing solution. The result implies that if we take a particular travelling salesman solution, and we are able to randomly refuel at a rate weighted by demand, then the expected resulting distance added by refueling along the cycle is bounded by the optimal vehicle routing solution. For brevity, we let $\text{LB} = \frac{2}{C} \sum_v D_v r_v$ stand for the left hand side of the inequality above. It is the canonical bound used to obtain approximations to the vehicle routing problem. However, one disadvantage with how this bound is often employed is that it isolates the {\it refill cost} of a particular solution to the vehicle routing problem, while discarding the {\it routing cost} of travelling between different nodes. This is why the bound works particularly well with vehicle routing approximations augmenting travelling salesman solutions, but is not sufficient to obtain better approximations in other scenarios.

\section{Random Refill Approximations}

As we mentioned, the bound above shows that an effective random refill policy results in a bounded approximation to the vehicle routing problem. We detail one such algorithm for the split vehicle routing problem below, assuming the existence of a polynomial time $1 + \alpha$ approximation to the travelling salesman problem.

\begin{algorithm}
\begin{algorithmic}[1]
\State Compute an $1 + \alpha$ optimal travelling salesman tour of $V$. Write this tour as a sequence $Q = \{ v_1, \dots, v_n \}$.
\State Uniformly randomly pick the starting supply $C' = C_0 \in [0,C]$.
\State Start the vehicle route at $B$.
\While {$Q$ is non-empty}
\State Pick a new vertex $v$ from the beginning of $Q$.
\If{$D_v \leq C'$} supply $v$ completely.
\ElsIf{$D_v > C'$}
    \State Supply $v$ with $C'$ units.
    \State Return to the depot and fill the vehicle up completely.
    \State Return to $v$ and completely supply $v$.
\EndIf
\EndWhile
\State Return to $B$.
\end{algorithmic}
\caption{Augmenting a TSP to Obtain a Routing Approximation.}
\label{alg:TSPAugmentationSplit}
\end{algorithm}

In the above algorithm, we say a vertex is a {\it breakpoint} if $D_v > C'$ when we visit the vertex, so we must return to the depot to refuel. The expected distance travelled by the algorithm above is then bounded by
%
\[ (1 + \alpha) T^*(G) + 2 \sum_v \mathbf{P}(v\ \text{is a breakpoint}) r_v \]
%
The event that the node $v_i$ is a breakpoint is the same as the event that there is some integer $n$ with $\sum_{j < i} D_{v_j} \leq C_0 + nC$, and $D_{v_i} + \sum_{j < i} d_{v_j} > C_0 + nC$, and we see that this occurs when $0 \leq C_0 < D_{v_i}$, so the probability that $v_i$ is a breakpoint is $D_{v_i}/C$. This implies the expected distance of the route the algorithm returns is
%
\[ (1 + \alpha) T^*(G) + \frac{2}{C} \sum_v D_v r_v = (1 + \alpha) T^*(G) + \text{LB} \leq (2 + \alpha) V^*(G) \]
%
The current standard approximation algorithm for the travelling salesman problem is a $3/2$ approximation algorithm. Thus the random refill policy gives a $7/2$ approximation algorithm for the vehicle routing problem. If we work over graphs in Euclidean space, or more generally, we work over graphs with a bounded genus, then we have a polynomial time approximation scheme giving $1 + \varepsilon$ approximation algorithms for each value of $\varepsilon$. This implies that there are $2 + \varepsilon$ approximation algorithms for vehicle routing on graphs of bounded genus.

The random refill technique is certainly unnatural, especially when compared to just refilling to the maximum value each time. However, the random refill technique is useful for avoiding being `stuck' in certain inefficient scenarios, and the technique is provably better: Refilling to maximum does result in an approximation algorithm with a constant approximation guarantee, but there are examples of graphs in which the full refuel strategy is unable to reach the $2 + \alpha$ constant guaranteed by the random refill value (The constant value approximation bound to non stochastic refill policies is left as an exercise in  Gupta's {\it Approximation Algorithms for VRP with Stochastic Demands}, which I have yet to do).

\section{Simulating Split Routing in Unsplit Scenarios}

We can obtain an approximation algorithm for the unsplit version of vehicle routing by simulating the TSP approximation for the split version of the problem. We proceed as in the split case, until we end up with a vertex $v$ with $D_v > C'$. In this case, we return to the depot, fill up our supply to $D_v$, return to $v$, supply $v$ completely, then return back to the depot, fill up to $C - (D_v - C')$ capacity, and continue with the TSP cycle. The reason for this is that the algorithm will then follow the route provided by the split version of the algorithm, except that whenever the original algorithm goes back and forth from the depot, the split version of the algorithm goes back and forth twice. The expected length of the route will be
%
\[ (1 + \alpha) T^*(G) + (4/C) \sum_v d_v r_v = (1 + \alpha) T^*(G) + 2 \text{LB} \]
%
which is bounded by $3 + \alpha$ times the length of the optimal solution to the {\it split} solution to the algorithm, and the split solution will always be shorter than the optimal solution to the unsplit algorithm\footnote{It might be a profitable question to ask when there is a measurable difference between the performance of split and unsplit optimal solutions on graphs, in which case we can bound the unsplit optimal algorithm performance even better, as well as understand what features of a graph limit the capabilities of the unsplit graph algorithm to perform well. I have yet to see an analysis of unsplit vehicle routing which does not just bound the output of an algorithm by the split demand cost.}.

\section{NEW: Shortcutting in Unsplit Routing}

The refill approximations to unsplit vehicle routing break down their approximations into two factors, the {\it refill cost}, which is obtained by returning to the vehicle depot, and the {\it TSP cost}. One idea to improving these approximation algorithms is to optimize the TSP route by skipping certain parts of the TSP journey when we travel to the depot to refuel. In the split case, we are unlikely to decrease the TSP cost without coming up with a completely new algorithm for VRP, because with random refills there is a zero percent chance of having to refuel from a vertex which is completely satisfied, so we must travel between the TSP tour completely to refuel all vertices. However, in the unsplit algorithm we are able to {\it shortcut} the parts of the TSP, because, when we travel back to the depot to refuel a second time, we have completely satisfied the current vertex, and so we have no need to return to it, so we may proceed directly to the next vertex, shortcutting the TSP path.

\begin{example}
    Consider a routing problem with uniform demands over vertices, equal to $C/k$ for some integer $k$. If we follow a tour labelled with elements of $\mathbf{Z}_n$, then the probability that we skip an edge $(i\ i+1)$ is the probability that we begin a refill action at $i$ or $i+1$. The event that we refill at $i$ is disjoint from the event that we refill at $i+1$, because once we refill at $i$ we will start with at least $1-1/k$ capacity at the next vertex. Thus we skip an edge $(i\ i+1)$ with probability $2/k$. The event that we refuel at $i$ is $1/k$, and then we travel twice along the radius $r_i$, and once each along each radius $r_{i-1}$ and $r_{i+1}$. Thus the expected distance travelled is
    %
    \[ (1 - 2/k) T^*(G) + \sum 1/k (r_{i-1} + 2r_i + r_{i+1}) = (1 - 2/k) T^*(G) + 2 \text{LB} \]
    %
    Thus we obtain a $(1 - 2/k)(1 + \alpha) + 2 = (3 - 2/k) + (1-2/k) \alpha$ approximation algorithm. For $k = 2$, we obtain a $2$ approximation algorithm, and for $k = 3$, we obtain a $2 + 1/3 + \alpha/3$ approximation. However, these cases are solvable exactly in polynomial time (think of matching demand vertices), and therefore the approximations are just toy values. For $k = 4$, we obtain a $2.5 + \alpha/2$ approximation.
\end{example}

When the demands of vertices are high, the refuel cost of the algorithm should vastly outweigh the TSP cost of the approximation, and it is probable that shortcutting will yield similar approximation results. That is, we hope to obtain similar results assuming only that the demands of vertices are lower bounded by $C/k$. Then the event that we refill at two adjacent vertices $i$ and $i+1$ no longer occurs with a non zero probability. If $D_i + D_{i+1} > C$, then the probability that we refill at both vertices is $(D_i + D_{i+1})/C - 1$, so the probability that we skip the edge $i(i+1)$ is equal to
%
\[ \frac{D_i + D_{i+1}}{C} - \max \left(0, \frac{D_i + D_{i+1}}{C} - 1 \right) = \min \left( 1, \frac{D_i + D_{i+1}}{C} \right) \]
%
This means that if $D_i + D_{i+1} > C$, then we skip an edge with probability one. Overall, we find the length of the route is
%
\begin{align*}
    &\left( 1 - \sum \min \left( 1, \frac{D_i + D_{i+1}}{C} \right) \right) T^*(G) + \frac{1}{C} \sum \left( D_{i-1} + 2D_i + D_{i+1} \right) r_i\\
    &\leq \left( 1 - \sum \min \left( 1, \frac{D_i + D_{i+1}}{C} \right) \right) T^*(G) + \text{LB} + \frac{1}{C} \sum D_{i-1} r_i + \frac{1}{C} \sum D_{i+1} r_i
\end{align*}
%
Unfortunately, we have yet to find a feasible way to bound the sums $\sum D_{i-1} r_i$ and $\sum D_{i+1} r_i$ in terms of the cost of the optimal VRP solution, because it is difficult to compare distances and weights when weighed against different vertices.

TODO: Describe method of directed arrows to strategize shortcutting the path, and how we haven't been able to succeed with this approach.

% I Just realized that this section is essentially the same as the large demand analysis, except where we normalize the demands to 1 instead of required the demands to each be an equal portion of the capacity, so I feel this section doesn't really add anything





%\section{NEW: Random Start Positions in Unit Demand Routing}

%We can obtain a slightly better approximation to vehicle routing using the same techniques as the last chapter if we assume that all nodes on the graph have the same unit demand, in which case the bounds which required the introduction of a random starting supply $C_0$ vanish. Instead, the main problem we try to avoid is the difference between the radii $r_v$ between a vertex and the depot at any one point. Because $D_v = 1$ for all vertices $v$, the lower bound we derived for the general problem takes the form
%
%\[ V^*(X) \geq (2/C) \sum r_v = (2n/C) \frac{\sum r_v}{n} = \frac{2n \overline{r}}{C} \]
%
%where $n$ is the number of nodes in $V$, and $\overline{r} = (1/n) \sum r_v$ is the average radius from the depot.

%\newpage

%\begin{algorithm}
%\begin{algorithmic}[1]
%\State Compute an $1 + \alpha$ optimal travelling salesman tour of $G - \{ D \}$.
%\State Uniformly randomly pick a starting vertex $v_1$ for the vehicle routing tour, writing the tour as a sequence $Q = \{ v_1, \dots, v_n \}$.
%\State Start the route at $D$, with the supply $C'$ set to the maximal capacity $C$.
%\While {$Q$ is non-empty}
%\State Pick a new vertex $v$ from the beginning of $Q$.
%\State Supply $v$.
%\If{$C' = 0$}
%    \State Return to $D$ and fill the vehicle up completely.
%\EndIf
%\EndWhile
%\State Return to $D$.
%\end{algorithmic}
%\caption{Unit Cost TSP Augmentation Algorithm.}
%\label{alg:UnitTSPAugmentation}
%\end{algorithm}

%Due to the random choice of a starting vertex, each vertex on the tour has an equal chance of being chosen as a breakpoint of the random route, and the total number of breakpoints is $\lceil n/C \rceil$, so the expected number of times we take the routes from a vertex back to the depot is $2 \lceil n/C \rceil / n$. This implies that the expected distance of the solution is
%
%\[ (1 + \alpha) T^*(G) + \sum (2 \lceil n/C \rceil / n) r_v = (1 + \alpha) T^*(G) + 2 \lceil n/C \rceil \overline{r} \approx (1 + \alpha) T^*(G) + \text{LB} \]
%
%yielding essentially the same approximation ratio as for the split case analysis.

%However, in the unit demand case, when we return to a depot we can be guaranteed that the vertex we just travelled from is completely satisfied, so we can immediately travel to the next vertex rather than the vertex we just travelled to. We skip a particular edge of the tour when we refill at the vertex beginning the edge, so we ignore each edge with equal probability $\lceil n/C \rceil / n$. Thus, with skipping, we find that the expected length of the routing solution is bounded by
%
%\begin{align*}
%    (1 + \alpha) &\left( 1 - \lceil n/C \rceil / n \right) T^*(G) + 2 \sum (\lceil n/C \rceil / n) r_v \approx \left[(1 + \alpha)(1 - 1/C) + 1\right] V^*(G)
%\end{align*}
%
%so we obtain slightly better results, depending on how large the capacity $C$ is. This analysis does not work if we randomize starting demands instead, because then the edges of the graph have non uniform probabilities of being skipped, depending on the demand of their starting vertex.

\section{Reserving Refills in Small Demand Unsplit Vehicle Routing}

TODO: Talk about reserve tank to reduce unsplit refill distances to two laps rather than 4.

\part{Prize Collecting Vehicle Routing}

In most combinatorial optimization problems, a feasible solution must accomplish the entire task thoroughly. In a feasible solution to the travelling salesman problem, we must visit {\it all} the vertices of the graph. But in a real life application of the travelling salesman problem, we might reject possible sales opportunities that are outlandishly far from our start point, unless these opportunities are significantly valuable. {\it Prize collection} versions of combinatorial optimization problems place values on subtasks of a particular problem, and ask to find an optimal solution, given that we need only solve a certain number of the subtasks of a problem. In a similar vein, {\it $K$-subtask} problems ask us to find the optimal solution to a task, given that we need only complete $K$ subtasks. Certain techniques have been developed to reduce the $K$-subtask version of the travelling salesman problem to the prize collecting version. Good approximations for the prize collecting version of the vehicle routing problem have been developed, and one project we plan to work on this summer is to generalize the $K$-subtask prize collecting reduction used in travelling salesman problem to yield a good approximation to the $K$-subtask vehicle routing problem. We begin by detailing the prize collecting travelling salesman problem reduction which is likely to yield good results in the vehicle routing arena.

\section{$K$ Minimal Spanning Tree}

In the $K$ minimal spanning tree problem, we are given a root node $r$, where we have some cost function $d$ on the edges. The goal is to find a minimal edge set in the graph which connects the root to at least $K$ other edges. We denote the optimum value of this problem by $\text{OPT}_K$. We begin by describing a basic linear program reduction. Let $X_e \in \{ 0, 1 \}$ be a variable denoting whether we choose an edge $e$, and $Z_v \in \{ 0, 1 \}$ describe whether $v$ is connected to the root note. We can obtain an easy linear program relaxation by letting $X_e$ denote choosing the edge $e$, and $Z_v$ denote not connecting the vertex $v$ to the root. The linear program is then to minimize $\sum d_eX_e$, such that for each vertex $v$, and for each subset $S$ of edges containing $v$ but not containing $r$, $\sum_{e \in \delta(S)} X_e + Z_v \geq 1$, and also $\sum Z_v \leq n - k$. Unfortunately, this approximation algorithm has a poor integrality gap.

\begin{example}
    Consider a graph $G$ consisting of vertices $\{ r, v_1, \dots, v_n \}$, with an edge $(r,v_1)$ that costs $n$, and $n-1$ edges $(v_1,v_i)$ each with cost $1$. Then for $K = 2$, $\text{OPT}_K = n+1$, yet for fractional solutions we can set $Z = 1 - 2/n$ and $X = 2/n$, which is feasible, and has value $n(2/n) + (n-1)(2/n) \leq 4$.
\end{example}

Nonetheless, we can still apply some tricks on the LP to obtain a constant factor approximation, by `Lagrangifying' the cardinality constraint. Rather than forcing $K$ vertices to be connected to the root, we will assign a penalty for each vertex not connected to the root. As we increase the penalty, the optimum will connect more and more vertices. If we can find the `sweet spot' where the optimum to this problem has exactly $K$ vertices, then we can actually substitute this answer in the original problem and obtain a good approximation.

For each $\lambda$, we consider the linear program $K\text{-MST}(\lambda)$ which attempts to minimize $\sum d_eX_e + \lambda (\sum Z_v - (n-k))$, subject only to the subset constraint that $Z_v + \sum_{e \in \delta(S)} X_e \geq 1$ for any $S$ with $r \not \in S$, $v \in S$. We let the optimal value to this problem be $\text{OPT}_K(\lambda)$. It shall also be helpful to consider the problem $\text{PCST}(\lambda)$, which has the same constraints but attempts to optimize $\sum d_eX_e + \lambda \sum Z_v$, with optimal value $\text{OPT}_{ST}$. The optimal solution for $K\text{-MST}(\lambda)$ and $\text{PCST}(\lambda)$ will always be the same, but the optimal solution values will be different, and as such they may have different integrality gaps. The reason for this name is that this problem is a linear relaxation of the {\it prize collecting Steiner tree problem}, which asks to find the minimal Steiner tree, subject to a uniform penalty $\lambda$ for each terminal vertex we don't include in a solution.

\begin{lemma}
    For any $\lambda \geq 0$, $\text{OPT}_K(\lambda) \leq \text{OPT}_K$.
\end{lemma}
\begin{proof}
    The optimum solution to the $K$ MST problem is certainly feasible to the linear program, and the cost of this solution is equal to
    %
    \[ \sum d_e X_e + \lambda(\sum Z_v - (n-k)) = \text{OPT} + \lambda(\sum Z_v - (n-k)) \leq \text{OPT}_K \]
    %
    because we know $\sum Z_v \leq n-k$ in the optimal solution. In particular, this implies that the value of the optimal solution to $K\text{-MST}(\lambda)$ is upper bounded by $\text{OPT}_K$.
\end{proof}

\begin{lemma}
    For any $\lambda \geq 0$, $\text{OPT}_K(\lambda) + \lambda(n-K) = \text{OPT}_{ST}(\lambda)$.
\end{lemma}
\begin{proof}
    Obvious.
\end{proof}

To analyze the $K$-MST problem, we will assume a result we will later prove, which is slightly stronger than saying that $\text{PCST}(\lambda)$ linear programs have an integrality gap of 2.

\begin{fact}
    For any $\lambda \geq 0$, there is a polynomial time algorithm to find a set of edges $E$ such that if $V$ is the set of all the vertices not connected to the root by edges in $E$, then $d(E) + 2 \lambda |V| \leq 2 \text{OPT}_{ST}(\lambda)$.
\end{fact}

Such an algorithm is known as a {\it Lagrangian multiplier preserving approximation}. It is a stronger result that just the existence of a 2-approximation for the prize collecting Steiner tree, because we double the cost of missed vertices while still lower bounding the optimal result. This will be important to our approximation of $K$-MST. 

Using the integrality gap, we present a $5 + \varepsilon$ algorithm with running time polynomial in the number of vertices in $G$ and $\log(1/\varepsilon)$. As a preprocessing step, we guess the furthest node from the root spanned by the optimum solution, and discard all nodes further than this. Thus we may assume we know the radius of the optimum solution. In particular, we may assume that $d(r,v) \leq \text{OPT}$ for all vertices $v$. It is easy to check if there is a 0 cost solution to the problem, so we also assume $\text{OPT} > 0$. Let $\delta \leq \text{OPT}$ be the minimum non-zero distance $d(r,v)$ over all vertices $v \in V$.

If we invoke the algorithmic result with $\lambda = 0$, then we will obtain $E = \emptyset$. If $2 \lambda  > 2 |\text{MST}(G)| + 2 \lambda(n-k)$, i.e. if $\lambda \geq \text{OPT}_{MST}$, then the optimal solution to $\text{OPT}_{ST}(\lambda)$ is just a minimal spanning tree, because the penalty of missing a vertex is too much, and the algorithm will just return a minimal spanning tree. As we vary $\lambda$ over $[0,\text{OPT}_{MST}]$, the number of vertices will vary between $0$ and $n$. We would hope this varies continuously, so we can find a $\lambda$ which misses exactly $K$ vertices, but this is not always possible.

\begin{example}
    Consider the graph used to disprove the integrality gap for the basic linear relaxation of the $K$ MST problem, now viewed as an instance of the prize collecting steiner tree problem. With some basic computation, we find that if if $\lambda \leq n$, an optimal solution will cover no vertices apart from the root, and for $\lambda \geq n$, the optimal solutions will be the minimal spanning tree covering all nodes. Thus we see a rapid transition in the optimal solutions to the graph, so we cannot expect the algorithm approximating this problem to have a smooth transition between vertices.
\end{example}

However, we can perform a binary search on the value of $\lambda \in [0,\text{OPT}_{MST}]$ to find two values $\lambda_1$ and $\lambda_2$ which are incredibly close together, such that the algorithm for $\lambda_1$ returns a solution with slightly less than $K$ vertices covered, and on the input $\lambda_2$ the algorithm returns a solution with slightly more than $K$ vertices covered. More precisely, we binary search using $\lambda_1$ and $\lambda_2$ as lower and upper bounds until $\lambda_1 \leq \lambda_2 \leq \lambda_1 + \varepsilon \delta / 4n$, such that if the solution on input $\lambda_1$ covers vertices $V_1$, and the solution for $\lambda_2$ is $V_2$, then $|V_1| \geq n - k \geq |V_2|$.

If $V_1$ contains exactly $n-K$ vertices, then if $E_1$ are the respective edges returned by the algorithm, we find that
%
\[ d(E_1) + 2 \lambda_1 |V_1| = d(E_1) + 2 \lambda_1(n-K) \leq 2\text{OPT}_{ST}(\lambda_1) = 2\text{OPT}_K(\lambda_1) + 2\lambda_1(n-K) \]
%
It follows that $d(E_1) \leq 2 \text{OPT}_K(\lambda_1) \leq 2 \text{OPT}$, so $E_1$ gives a 2-approximation solution. The same result holds for $E_2$ if $V_2$ contains exactly $n_K$ vertices.

Given our construction of an approximation algorithm, we can now assume the strict inequality $|V_1| > n - k > |V_2|$. This means that the values
%
\[ \alpha_1 = \frac{(n-K) - |V_2|}{|V_1| - |V_2|}\ \ \ \ \ \alpha_2 = \frac{|V_1| - (n-K)}{|V_1| - |V_2|} \]
%
which measure the relative closeness to an exact $K$ MST, are well defined. Note that these values are both positive, and $\alpha_1 + \alpha_2 = 1$. Thus we may interpret these values as a probability distribution over the two solutions $E_1$ and $E_2$. Since $\alpha_1 |V_1| + \alpha_2 |V_2| = n-K$, `on average' these solutions are exactly a $K$ MST.

\begin{lemma}
    $\alpha_1 d(E_1) + \alpha_2 d(E_2) \leq (2 + \varepsilon/2) \text{OPT}$.
\end{lemma}
\begin{proof}
    Note that $\text{OPT}_{ST}(\lambda_1) \leq \text{OPT}_{ST}(\lambda_2)$, because the two problems have the exact same feasible solutions, except that the cost function of the second problem is always at least as large as the first. Next, note that
    %
    \[ (\lambda_2 - \lambda_1) \alpha_2 |V_2| \leq \frac{\varepsilon \delta}{4 n} \alpha_2 |V_2| \leq \frac{\varepsilon \text{OPT}}{4n} \alpha_2 |V_2| \leq (\varepsilon/4) \text{OPT} \]
    %
    It follows that
    %
    \begin{align*}
        \alpha_1 d(E_1) + \alpha_2 d(E_2) &\leq 2[\alpha_1 \text{OPT}_{ST}(\lambda_1) - \lambda_1 \alpha_1 |V_1| + \alpha_2 \text{OPT}_{ST}(\lambda_2) - \lambda_2 \alpha_2 |V_2|]\\
        &\leq 2[(\alpha_1 + \alpha_2) \text{OPT}_{ST}(\lambda_2) - \lambda_1 \alpha_1 |V_1| - \lambda_2 \alpha_2 |V_2|]\\
        &\leq 2[\text{OPT}_{ST}(\lambda_2) - \lambda_2(\alpha_1 |V_1| + \alpha_2 |V_2|) + (\lambda_2 - \lambda_1) \alpha_1 |V_1|]\\
        &\leq 2[ \text{OPT}_{ST}(\lambda_2) - \lambda_2(n-K) + (\lambda_2 - \lambda_1) \alpha_1 |V_1|]\\
        &\leq 2 \text{OPT} + 2 (\lambda_2 - \lambda_1) \alpha_1 |V_1|\\
        &\leq (2 + \varepsilon/2) \text{OPT}
    \end{align*}
    %
    and this completes the proof.
\end{proof}

$E_2$ will always be a feasible solution to the $K$-MST problem. If $\alpha_2 \geq 1/2$, it is actually a good approximate solution, because the last lemma gives
%
\[ d(E_2) \leq 2 \alpha_2 d(E_2) \leq 2[(2 + \varepsilon/2) \text{OPT} - \alpha_1 d(E_1)] \leq (4 + \varepsilon) \text{OPT} \]
%
The only tricky case is where $\alpha_2 < 1/2$, in which case $E_1$ has a low cost, but is not a feasible solution to the problem. We shall find that we can {\it graft} part of the solution of $E_2$ onto $E_1$ to obtain a feasible solution, while only increasing the approximation ratio by $\text{OPT}$. This yields a $5 + \varepsilon$ approximation for any input to the problem. First, notice that $|V_1 - V_2|$ contains at least $|V_1| - |V_2|$ nodes, and also $|V_2| \leq n-K$, so it makes sense to double the edges $E_1$, and to find the cheapest cycle $C$ in $E_1$ which covers $|V_1| - (n-k)$ vertices in $V_1 - V_2$. We can view $C$ as a cycle of these vertices in $V_1 - V_2$, by passing over vertices in $V_2$. By taking the cheapest subpath of length $|V_1| - (n-K)$ in this graph, we obtain a set of edges we can graft onto $E_1$, and after adding the edges to $E_1 \cup P$ forming the shortest path from $r$ to $P$, so the graph is connected, we obtain a feasible solution. We know that the cost added by connecting $r$ to $P$ is bounded by $\text{OPT}$, because we have discarded all vertices of radius greater than $\text{OPT}$ in a preprocessing step. Since $C$ is a subset of the doubling of $E_2$, we obtain that $d(C) \leq d(E_2)$. What's more, since $P$ is the cheapest subpath of the collapsing of $C$ containing $|V_1| - (n-k)$ edges, and $C$ contains $|V_1| - |V_2|$ edges, we find that
%
\[ d(P) \leq \frac{|V_1| - (n-K)}{|V_1| - |V_2|} d(C) \leq 2\alpha_2 d(E_2) \]
%
and therefore
%
\[ d(E_1) + 2 \alpha_2 d(E_2) + \text{OPT} \leq 2\alpha_1 d(E_1) + 2\alpha_2 d(E_2) + \text{OPT} \leq (4 + \varepsilon) \text{OPT} + \text{OPT} \]
%
this completes the analysis of the algorithm.

\section{Prize Collecting Steiner Tree}

In order to obtain the integrality bound algorithm above, we have to look into the prize collecting Steiner tree problem in more detail. In general, the prize collecting Steiner tree problem assigns penalties $\pi$ to each vertex not covered by the Steiner tree, in addition to the cost function $c$ on edges. The problem then tries to minimize $c(E) + \pi(V)$, where $E$ is a tree from a fixed root $r$, and $V$ are the vertices not covered by $E$.

We will consider a nontrivial linear relaxation to obtain the required construction for the $K$-MST problem. We add the variables $X_e$ for each edge, but also variables $Z_S$ for each nonempty set $S$ of vertices not containing $r$, and try to minimize $\sum c(e) X_e + \sum \pi(S) Z_S$, such that for each subset $S$ of vertices, $\sum_{e \in \delta(S)} X_e + \sum_{S \subset R} Z_R \geq 1$, with $X,Z \geq 0$. This program has exponentially many constraints and variables, but we can still form a primal dual algorithm which can be solved in polynomial time; the dual asks to maximize $\sum y_S$, where $y$ is a vector indexed by nonempty sets of vertices, such that $\sum_{e \in \delta(S)} y_S \leq c_e$ for each edge $e$, and $\sum_{R \subset S} y_R \leq \pi(S)$ for each nonempty set $S$. We denote this linear program by $\text{PCST}$, and the optimal value of this linear program by $\text{OPT}_{PCST}$.

\begin{lemma}
    If $\pi_v = \lambda \geq 0$ for each vertex $v$, then $\text{OPT}_{ST}(\lambda) = \text{OPT}_{PCST}$.
\end{lemma}
\begin{proof}
    Given an optimal solution $(X^*,Z^*)$ to the $\text{PCST}$ linear program we just defined, we construct an optimal solution to  $\text{PCST}(\lambda)$. We set $X_e = X^*_e$ and $Z_v = \sum_{v \in S} Z^*_S$. This gives a $\text{PCST}(\lambda)$ solution. On the other hand, given a $\text{PCST}(\lambda)$ solution $(X^*,Z^*)$, we can order the vertices $v_1, \dots, v_n$ such that $Z^*_{v_1} \leq Z^*_{v_2} \leq \dots \leq Z^*_{v_n}$, and we can then define $X_e = X^*_e$ and for $S_i = \{ i, \dots, n \}$, we can set $Z_{S_i} = Z^*_{v_i} - Z^*_{v_{i-1}}$, with $Z_{S_1} = Z^*_{v_1}$. It is clear that $(X,Z)$ gives the same cost as $(X^*,Z^*)$, and is a feasible solution.
\end{proof}

We now work on a polynomial time algorithm to find a feasible prize collecting Steiner tree instance $(E,V)$ with $c(E) + 2 \pi(V) \leq 2 \text{OPT}_{PCST}$. The idea is to grow `moats' around the vertices in the graph until they connect to $r$. However, each moat has a certain growth limit, which if reached, allows us to discard the entire set of vertices in the moat from the tree. Once we have done this, we prune the solution, discarding vertices not connected to $r$ and edges whose deletion produces a component which exceeded its growth limit in the course of the algorithm.

The general idea is to grow a forest while also constructing a dual solution $y_S$, modifying the terms $y_T$ corresponding to the components of the forest. An {\it active component} will be a component whose dual constraint is slack. We begin by constructing a vector $\gamma$ with $\gamma(v) = \pi_v$ for each vertex, and initializing the forest $F = \emptyset$, with components $\{ v \}$. We set $y = 0$. While there are still active components in the graph, we take the active components $C_1, \dots, C_m$, and increase $y_{C_i}$ and decrease $\gamma(C_i)$ at the same rate simultaneously, until either
%
\begin{itemize}
    \item $\gamma(C_i) = 0$ for some $C_i$, in which case $C_i$ becomes {\it inactive}.
    \item The constraint corresponding to some edge $e$ becomes tight. We then add $e$ to the forest $F$, which identifies two components $A$ and $B$. We set $\gamma(A \cup B) = \gamma(A) + \gamma(B)$, and then continue the algorithm.
\end{itemize}
%
Once there are no more active components, we begin the pruning process. We let $C_r$ denote the component of the forest containing the root node, and let $F(C_r)$ denote its edges in the forest. For each edge $e \in F(C_r)$ such that the nonroot component $C$ of $F - \{ e \}$ was considered in the algorithm, and $\gamma(C) = 0$, we remove the edge $e$ from the forest. Once this is done, we remove all the edges not connected to the root, and the remaining tree is our output.

\begin{lemma}
    For any two components $C$ and $C'$ considered in the algorithm, either $C$ and $C'$ are disjoint, $C$ is a subset of $C'$, or $C'$ is a subset of $C$.
\end{lemma}
\begin{proof}
    From the construction of the components in the algorithm, we see that we successively form decompositions of the components, so the components considered are always decompositions of prior components.
\end{proof}

Now let $F$ be the returned solution, where $V$ is the set of vertices not connected to the root node. We set $X$ to be the family of components $C$ considered in the algorithm which aren't a subset of $V$, and we set $Y$ to be the family of components $C$ which are a subset of $V$.

\begin{lemma}
    $\pi(V) = \sum_{C \in Y} y_C$.
\end{lemma}
\begin{proof}
    We prove this as a loop algorithm of the construction, that for each component $C$ of $F$ (active or not), $\sum_{S \subset C} y_S + \gamma(C) = \pi(C)$. Initially this is obvious. Similarily, it holds when increasing and decreasing the $y_{C_i}$ and $\gamma(C_i)$. Now we note that $y_S$ is only nonzero if $S$ was a component considered in the algorithm. In the last lemma, we showed this was a laminar family, which will become important when bridging components. If an edge $e$ bridges two components $A$ and $B$, then any $S \subset A \cup B$ with $y_S \neq 0$ must either be a subset of $A$ or a subset of $B$. Thus
    %
    \[ \sum_{S \subset A \cup B} y_S + \gamma(A \cup B) = \sum_{S \subset A} y_S + \sum_{S \subset B} y_S + \gamma(A) + \gamma(B) = \pi(A) + \pi(B) = \pi(A \cup B) \]
    %
    Thus the loop invariant holds throughout the algorithm. When our algorithm finishes, we have a sequence of inactive components $C_1, \dots, C_m$, and we therefore know that $\sum_{S \subset C_i} y_S = \pi(C)$. But then the lemma is proved by summing over the components whose union is $V$.
\end{proof}

\begin{lemma}
    $c(F) \leq 2 \sum_{S \in X} y_S$
\end{lemma}

Combining these two points, we find that
%
\[ c(F) + 2 \pi(V) \leq 2 \sum_{S \in X} y_S + 2 \sum_{C \in Y} y_S \leq 2 \sum y_S \leq 2 \text{OPT}_{PCST} \]
%
and this gives the inequality required in our analysis of $K$ minimum spanning trees, while also giving us a 2 approximation for the prize collecting Steiner tree problem.

\section{Prize Collecting Travelling Salesman}

The analysis of the $K$ travelling salesman and its relation to prize collecting travelling salesman is essentially the same as with the minimum spanning tree problem. The linear program for $K$ travelling salesman is to minimize $\sum c_eX_e$ subject to the constraints that $\sum_{e \in \delta(S)} X_e + 2 Z_v \geq 2$ for all sets $S$ with $v \in S$, and $\sum_v Z_v \leq n - K$. The corresponding Lagragian problem is to minimize $\sum c_eX_e + \lambda (\sum Z_v - (n-K))$. We utilize the same notation as for MST, using $\text{OPT}_K$, $\text{OPT}_K(\lambda)$.

TODO: Write more about prize collecting travelling salesman.

\section{Prize Collecting Vehicle Routing Problem}

Consider a $1 + \alpha$ approximation algorithm for $K$-MST. For instance, Arora's {\it A $2 + \varepsilon$ approximation algorithm for the $K$-MST problem} introduces a polynomial time computable heuristic with $\alpha = 1 + \varepsilon$. Given a particular instance of vehicle routing, define $d_{vw} = d(v,w) + \frac{d_vr_v + d_wr_w}{C} + d(v,w)$. This new metric now takes into account {\it both} the travelling salesman cost of an algorithm, and the refill cost of the algorithm. Indeed, a given travelling salesman tour using the metric gives rise to a vehicle routing solution whose expected cost using the random refill strategy. In fact, this metric should guarantee at least as good an approximation as any other vehicle routing strategy augmenting the travelling salesman solution, because all the other solutions are feasible in this solution! Thus we just need to explore which travelling salesman solutions are optimal with respect to this metric.

\part{Metric Augmentation to Improve Vehicle Routing Approximations}

%\begin{multicols}{2}

One obtains the most basic vehicle routing approximation by augmenting near-optimal TSP tours with a {\it refill strategy}, which adds to the length of the tour but ensures the capacity constraint introduced in the VRP problem is satisfied so the solution is feasible. An obvious refill strategy is to follow the path taken by a given tour, satisfying each node sequentially with it's full demand, but periodically returning to the root node to refuel when the capacity is drained. We call this the {\it random refill} strategy if, less obviously, one begins the tour with a uniformly random starting capacity between $0$ and $C$. The expected total distance travelled during refueling using the random refill strategy is given by
%
\[ \text{LB} := \frac{2}{C} \sum_{v \in V} D_v r_v \]
%
where $D_v$ is the demand of the vertex $v$, and $r_v = c(r,v)$ the radius from the root node. It is well established in the literature that $\text{LB}$ lower bounds the optimal cost of the vehicle routing tour on the vertices $V$ \cite{RandomRefillPaper}. Because of this, if we have a $1 + \alpha$ approximation to TSP, we immediately obtain a $2 + \alpha$ approximation to vehicle routing. A slight modification to the random refueling policy in unsplit vehicle routing leads to a refueling strategy with expected cost $2 \text{LB}$, hence the technique yields $3 + \alpha$ approximations to unsplit vehicle routing as well.

A problem with existing refuelling strategy techniques is that they separate the {\it refill cost} of a solution from the {\it routing cost} of the tour. Standard approximation algorithms for the travelling salesman problem do not take the refill cost of the graph into account, since they just need to optimize total length of the tour. This makes it difficult to obtain efficient bounds on the refilling cost when using a near optimal TSP tour. We can see this in the $\text{LB}$ bound we introduced above, which provides a tour independent bound on the cost of refueling, but also adds $+1$ to the approximation ratio of the algorithm. One feature of our approach to vehicle routing is that we attempt to take the refill cost into account when constructing our TSP tour. Though we do not find a new way to bound the refuelling cost, we use the $\text{LB}$ bound on variants of the vehicle routing problem to obtain novel approximation algorithms which do form TSP tours taking the refill bound into account. With further work, we believe more sophisticated refueling bounds can be applied to the main vehicle routing problem, in which case our ideas may yield stronger approximations to vehicle routing.

One disadvantage of the $\text{LB}$ bound is that it doesn't depend on the travelling salesman tour chosen, so it cannot be used to choose a particular travelling salesman tour over the set of vertices. However, if we consider the lower bound as a function of the vertex set $V$ chosen to perform a VRP tour over, then the function is linear, which implies that the lower bound may be a useful feature in a linear program used to find approximations to variants of the vehicle routing problem which only find tours over subsets of vertices. In this case, the tours we find will not only take the routing cost of the subset into consideration, but also the refill cost. We apply this idea first to the prize collecting vehicle routing problem, and then to other variants.

Currently, our knowledge of approximations for variants of the minimum spanning tree problem is much better than our knowledge for approximations for variants of the travelling salesman problem. For this reason, to construct our travelling salesman tours for use with a random refill strategy, we construct a tree, using the double and shortcutting edges technique to construct a Hamiltonian tour of the vertices using each edge on the tree at most twice. What's more, we will rely on a result of Bang-Jensen, Frank, and Jackson which reduces the construction of a tree to the construction of an appropriate preflow on the routing graph.

\begin{fact}[\cite{FlowToTreeResult}, \cite{FlowToTreePolyTimeResult}]
    Let $x$ be a preflow on a directed graph starting at $r$. Then there exists a polynomial time computable random $r$-rooted arboresence $A$ such that $\mathbf{P}(e \in A) \leq x(e)$, and $\mathbf{P}(v \in A)$ is equal to the max $r-v$ flow with respect to $x$.
\end{fact}

What remains is to construct a linear program which can find the appropriate preflows, which is fairly simple, given that flow problems have proven particularly amenable to LP solutions.

\section{A Prize-Collecting Vehicle Routing LP}

We begin with an analysis of the most basic `subset' variant of vehicle routing, the prize collecting vehicle routing problem. In this problem, we only need to find an optimum vehicle routing tour over a subset of vertices in the graph. However, for each vertex $v$ our optimum vehicle routing tour does not cover, we are given a penalty $\pi_v \geq 0$, which encourages us to at least cover efficiently covered vertices. If a given vehicle routing tour $T$ satisfies the vertices $W \subset V$, we assign it a value $c(T) + \pi(V - W)$, and we attempt to find a vehicle routing tour minimizing this value. Since we intend to augment a travelling salesman tour with the random refill strategy, we design an LP which discourages choosing vertex subsets to cover with a high refill cost.

Standard LP prize collection variants of graph problems introduce an extra set of variables $z \in [0,1]^V$ to designate whether a given vertex is included in a solution of not, which can then be used directly as input into the linear penalty function \cite{MSTLPApprox}. We also use this technique to construct LP using this technique for constructing a `prize collecting preflow' $x \in [0,1]^E$ for obtaining a prize collecting VRP solution. To ensure the vertex subset considered in the prefill takes the refill cost into account, we explicitly include it as a factor in the LP objective function. More specifically, we consider an LP over the variables $(x,z)$ given below:
%
\begin{equation*}
\begin{aligned}
& \text{minimize}
& & c(x) + (1/2) \text{LB}(z) + (3/2) \pi(z)\\
& & & = \sum c_ex_e + (1/C) \sum (1 - z_v) D_vr_v + 2\sum \pi_v z_v \\
& \text{subject to} & & x(\delta^{\text{out}}(v)) \leq x(\delta^{\text{in}}(v)) \leq 1 - z_v, \; v \in V\\
& & & x(\delta^{\text{in}}(S)) \geq 1 - z_v, \; v \in S \subset V\\
& & & x(\delta^{\text{out}}(r)) = 1\\
& & & x,z \geq 0
\end{aligned}
\end{equation*}
%
We let $(x^*,z^*)$ denote the optimal solution to the given LP.

\begin{lemma}\label{thm:VRPTourBound}
    For any subset of vertices $W \subset V$, let $T^*(W)$ denote the optimal VRP tour covering the nodes in $W$. Then
    %
    \[ c(x^*) + (1/2)\text{LB}(z^*) + (3/2) \pi(z^*) \leq (3/2)[c(T^*(W)) + \pi(V-W)] \]
\end{lemma}
\begin{proof}
    Assume first that $T^*(W)$ does not visit a node more than twice. Let $T^*(W)$ consist of the tour of vertices $v_1, \dots, v_m$. Consider the solution $(x,z)$ to the LP above with $x_{v_i v_{i+1}} = 1$, and $z_v = 1$ if $v \not \in W$, with all other values zero. Since the solution is a cycle, the $x$ values certainly give a preflow, and we may assume without loss of generality that $T^*(W)$ only passes through the vertices in $W$, which gives the bounds $x(\delta^{\text{in}}(v)) \leq 1 - z_v$ for each vertex $v$ (this is where we use the fact that $T^*(W)$ only visits a node once). Since the solution is feasible, we conclude that
    %
    \begin{align*}
        c(T^*(W)) &+ (1/2) \text{LB}(W) + (3/2) \pi(V-W)\\
        &= c(x) + (1/2) \text{LB}(z) + (3/2) \pi(z)\\
        &\geq c(x^*) + (1/2) \text{LB}(z^*) + (3/2) \pi(z^*)
    \end{align*}
    %
    The only problem if $T^*(W)$ visits a node more tha once is that the constraint $x(\delta^{\text{in}}(v)) \leq 1 - z_v$ need not hold in the construction above. However, we can still obtain the inequality if we {\it shortcut} the tour $T^*(W)$ so each vertex is visited a single time. This only decreases the length of the tour, so the lower bound property still holds. Since $T^*(W)$ is the optimal VRP solution over the vertices $W$, we conclude from the standard refill cost lower bound that $\text{LB}(W) \leq c(T^*(W))$.
\end{proof}

\begin{corollary}
    In the last lemma, if we take $W$ to be the set of vertices covered by an optimum solution to the prize collecting VRP problem, we conclude that
    %
    \[ c(x^*) + (1/2) \text{LB}(z^*) + (3/2) \pi(z^*) \leq (3/2) \text{OPT}_{\text{PCVRP}} \]
    %
    a `Lagrangian Preserving Approximation' type bound for the LP.
\end{corollary}

Now employing the result of Bang-Jensen, we now use the preflow $x^*$ to find a random arboresence $A$. This arboresence has the following useful properties:
%
\begin{itemize}
    \item Directly taking the promises of Bang-Jensen gives
    %
    \[ \mathbf{E}(c(A)) = \sum c_e \mathbf{P}(e \in A) \leq \sum c_e x_e = c(x^*) \]

    \item We can interpret the subcollection of constraints $x(\delta^{\text{in}}(S)) + z_v \geq 1$, for a fixed $v$, as saying that the max $r-v$ flow with respect to a feasible preflow solution is lower bounded by $1 - z_v$. The constraint $x(\delta^{\text{in}}(v)) \leq 1 - z_v$ provides an instance of a min cut of value less than $1 - z_v$, so we conclude the min $r-v$ cut is upper bounded by $1 - z_v$, but since the max $r-v$ flow value is equal to the min $r-v$ cut value, we conclude the max $r-v$ flow has value exactly $1 - z_v$, and therefore the result of Bang-Jensen, Frank, and Jackson gives $\mathbf{P}(v \in A) = 1 - z_v$.

%    Noting that the arboresence includes a particular vertex $v$ if and only if it includes an edge attached to the vertex, we apply a union bound to conclude that $\mathbf{P}(v \in A) \leq \sum_{v \in e} \mathbf{P}(e \in A) \leq \sum_{v \in e} x^*_e$. (We needed this union bound when using the bidirected metric, but now it is unneeded because we can make the r-v cut have value precisely 1 - z_v).
\end{itemize}

Using the doubling and shortcutting technique, we can turn the random arboresence into a random tour covering the same vertices, using each edge at most twice. Furthermore, we can perform the random refill strategy on the tour to obtain a random vehicle routing solution $T$. We split the analysis of the expected routing cost, refill cost, and penalty cost of the solution.
%
\begin{itemize}
    \item The expected routing cost is easy to bound, because it is twice the expected routing cost of the random arboresence $A$, and thus bounded by $2c(x^*)$.

    \item The expected refill cost is equal to
    %
    \[ \frac{2}{C} \sum_v \mathbf{P}(v \in T) D_v r_v = \frac{2}{C} \sum_v (1 - z^*_v) D_vr_v = \text{LB}(z^*) \]

    \item The expected penalty cost is equal to
    %
    \[ \sum \pi_v \mathbf{P}(v \not \in T) = \sum \pi_vz^*_v = \pi(z^*) \]
\end{itemize}
%
We have thus obtained a random tour $T$ over a random collection of vertices $W$ such that
%
\[ \mathbf{E}[c(T) + 3\pi(V - W)] \leq 2c(x^*) + \text{LB}(z^*) + 3\pi(z^*) \]
%
Applying the corollary to Lemma \ref{thm:VRPTourBound}, and choosing the minimum value tour over all vertices, we obtain our final approximation to prize collecting vehicle routing.

\begin{theorem}
    In polynomial time, we can produce a vehicle routing tour $T$ over a collection of vertices $W$ such that
    %
    \[ c(T) + 3\pi(V - W^*) \leq 3 \text{OPT}_{\text{PCVRP}} \]
    %
    a Lagrangian preserving 3 approximation for prize collecting vehicle routing.
\end{theorem}

Before we finish our discussion of prize collection, we note that the unsplit demand prize collecting vehicle routing problem offers a similar approximation. One proceeds through the calculations, but using the double random refill strategy to simulate split demand refills through unsplit demand refills, and using the cost function $c(x) + \text{LB}(z) + 2 \pi(z)$ instead of $c(x) + (1/2) \text{LB}(z) + (3/2) \pi(z)$ in the LP, we can obtain a Lagrangian preserving 4 approximation to unsplit demand prize collecting vehicle routing.

%NOTE: WE DIDN'T USE THE BIDIRECTED METRIC STRATEGY, BUT IT MIGHT DISCOURAGE UNSPLIT TOURS FROM TAKING THE SAME NODE TWICE. I DON'T KNOW WHETHER IT HAS A GOOD INTEGRALITY GAP FOR SPLIT TOURS.

\section{A $K$-VRP Approximation}

To summarize, in terms of uniform Lagrangian penalties $\lambda$ in place of the variable $\pi$, we can find a VRP solution $T$ covering the vertices $W$ such that $c(T) + 3 \pi(W^c) \lambda \leq 3 \text{OPT}_{\text{VRP}}(\lambda)$, where $\text{OPT}_{\text{VRP}}(\lambda)$ denotes the value of the optimal prize collecting steiner tree with uniform penalties $\lambda$. Here we employ the `standard' technique of Lagrangifying a `$K$-subset problem' which asks to find the cheapest VRP solution satisfying exactly $K$ nodes. This has been employed most notably in the $K$-MST problem, where it yields a $5$ approximation \cite{MSTLPApprox}. Our ideas are essentially the same, but included in full for completeness. The only new idea is the way we graft two VRP solutions together to obtain an exact $K$ VRP solution, which we can do more efficiently than in the $K$ MST problem, leading to a slightly improving approximation factor increase.

The main idea is to try and solve prize collecting vehicle routing problems with a uniform penalty $\pi_v = \lambda$ for a fixed $\lambda > 0$, while varying the $\lambda$ penalty so we obtain a solution with exactly $K$ nodes. If $\lambda$ is small, then the penalty for ignoring nodes is negligible, and we will obtain solutions with fewer than $K$ nodes. If $\lambda$ is large, then the penalty for ignoring vertices will be too strong, and we will obtain solutions with more than $K$ nodes. If we find $\lambda$ with exactly $K$ vertices, then we obtain a $K$ VRP solution. This will not always be possible, in which case we find ways of modifying solutions with `essentially' $K$ vertices to obtain better approximations. Define $\text{OPT}_K(\lambda) = \text{OPT}_{\text{VRP}}(\lambda) - \lambda (n - K)$, and $\text{OPT}_K$ to be the optimal cost of a $K$ VRP instance.

\begin{lemma}
    For any $\lambda > 0$, $\text{OPT}_K(\lambda) \leq \text{OPT}_K$.
\end{lemma}
\begin{proof}
    Any $K$ VRP solution $T$ covering the $K$ vertices $W$ is feasible in $\text{OPT}_K(\lambda)$, and with respect to the $\text{PCVRP}$ problem with penalties $\lambda$, has value
    %
    \[ c(T) + \lambda |W^c| = c(T) + (n - K) \geq \text{OPT}_{\text{VRP}}(\lambda) \]
    %
    It follows that $c(T) \geq \text{OPT}_K(\lambda)$, and therefore that $\text{OPT}_K \geq \text{OPT}_K(\lambda)$.
\end{proof}

Using a binary search, we find $\lambda_1 < \lambda_2$ infinitisimally close with corresponding tours $T_1,T_2$ (covering the corresponding vertices $W_1,W_2$) obtained from the Lagrangian preserving approximation algorithm presented in the last section, with $|W_1| \leq K \leq |W_2|$. If $|W_1| = K$, we find that
%
\[ c(T_1) + 3 \lambda |W_1^c| = c(T_1) + 3 \lambda (n - K) \leq 3 \text{OPT}_{\text{VRP}}(\lambda) = 3 \text{OPT}_K(\lambda) + 3 \lambda(n - K) \]
%
Hence $c(T_1) \leq 3 \text{OPT}_K(\lambda) \leq 3 \text{OPT}_K$, and we obtain a $3$-approximation. The same holds true if $|W_2| = K$, so we may assume that we have the {\it strict} inequality $|W_1| < K < |W_2|$. Otherwise, we form the constants
%
\[ \alpha_1 = \frac{|W_2| - K}{|W_2| - |W_1|}\ \ \ \ \ \alpha_2 = \frac{K - |W_1|}{|W_2| - |W_1|} \]
%
These values work just as in the lagrangian $K$ MST approximation, and as in this case we immediately verify that $\alpha_1, \alpha_2 \geq 0$, and $\alpha_1 + \alpha_2 = 1$, so we can view the $\alpha$ values as defining a probability distribution over the two tours. It is also easy to see that $\alpha_1 |W_1| + \alpha_2 |W_2| = K$, so `on average', the random tour is a $K$ tour. The final property we need is that `on average', the random tour is also a good approximation.

\begin{lemma}
    $\alpha_1 c(T_1) + \alpha_2 c(T_2) \leq (3 + \varepsilon) \text{OPT}_K$, where $\varepsilon$ depends in some way on how close we choose the values $\lambda_1$ and $\lambda_2$.
\end{lemma}
\begin{proof}
    Note that $\text{OPT}(\lambda_1) \leq \text{OPT}(\lambda_2)$, because the two problems have the exact same feasible solutions, except that the cost function of the second problem is always at least as large as the first. It follows that
    %
    \begin{align*}
        \alpha_1 d(T_1) + \alpha_2 d(T_2) &\leq 3[\alpha_1 \text{OPT}(\lambda_1) - \lambda_1 \alpha_1 |W_1^c| + \alpha_2 \text{OPT}(\lambda_2) - \lambda_2 \alpha_2 |W_2^c|]\\
        &\leq 3[(\alpha_1 + \alpha_2) \text{OPT}(\lambda_2) - \lambda_1 \alpha_1 |W_1^c| - \lambda_2 \alpha_2 |W_2^c|]\\
        &\leq 3[\text{OPT}(\lambda_2) - \lambda_2(\alpha_1 |W_1^c| + \alpha_2 |W_2^c|) + (\lambda_2 - \lambda_1) \alpha_1 |W_1^c|]\\
        &\leq 3[ \text{OPT}(\lambda_2) - \lambda_2(n-K) + (\lambda_2 - \lambda_1) \alpha_1 |W_1^c|]\\
        &\leq 3 \text{OPT}_K(\lambda_2) + (\lambda_2 - \lambda_1) \alpha_1 |W_1^c|\\
        &\leq (3 + \varepsilon) \text{OPT}_K
    \end{align*}
    %
    We can choose $\lambda_2 - \lambda_1 \leq (2 \min_v r_v) (\varepsilon/3n)$, because $2 \min_v r_v \leq \text{OPT}_K$ (we may assume $\min_v r_v > 0$, for otherwise the problem is solvable exactly in polynomial time), and since we have the inequalities $\alpha_1 \leq 1$, $|W_1^c| \leq n$, we find
    %
    \[ (\lambda_2 - \lambda_1) \alpha_1 |W_1^c| \leq \frac{|W_1^c|}{n} (2 \min_v r_v) (\varepsilon/3) \leq (\varepsilon/3) \text{OPT}_K \]
    %
    It is also important to note that the binary search to find $\lambda_2$ and $\lambda_1$ therefore only takes $\approx -\lg[(2 \min_v r_v) (\varepsilon/3n)]$ queries, which is polynomial in the bit complexity of the radii $r_v$ and $n$. This completes the proof of the inequality.
\end{proof}

Now since $|W_2| \geq K$, we could simply return $T_2$ as our solution to the $K$ MST problem. The next lemma shows how good an approximation ratio this solution has.

\begin{lemma}
    $c(T_2) \leq \alpha_2^{-1}(3 + \varepsilon) \text{OPT}_K$.
\end{lemma}
\begin{proof}
    $c(T_2) \leq \alpha_2^{-1}[\alpha_1 c(T_1) + \alpha_2 c(T_2)] \leq \alpha_2^{-1} (3 + \varepsilon) \text{OPT}_K$.
\end{proof}

If $\alpha_2$ is close to one, we obtain a good approximation to $K$ MST, since $T_2$ is an actual approximation. On the other hand, if $\alpha_2$ is small, then $T_1$ is very close to a $K$ MST tour, and we will use a grafting strategy to place part of the $T_2$ tour onto $T_1$. We consider the shortest subpath $C$ of the TSP component of $T_2$ containing $K - |T_1|$ vertices that are not on $T_1$ (counting the refill parts included on the TSP component as part of the cost of the subpath tour), and then adding two paths onto the endpoints of the subpath to connect them to $r$.

\begin{lemma}
    If we assume that the further vertex from $r$ in the graph is included in the $K$ VRP tour (which can be obtained by minimizing the algorithm over the furthest vertex), then
    %
    \[ d(T_1) + d(C) \leq [\alpha_1^{-1} (3 + \varepsilon) + 1] \text{OPT}_K \]
\end{lemma}
\begin{proof}
    The cost of the optimal $K$ VRP solution must include the furthest node $v$ from $r$, and therefore $\text{OPT}_K \geq 2r_v \geq r_w + r_u$ for any two vertices $w$ and $u$ in the graph. This implies that the two edges used in the construction of $C$ to make the subpath into a cycle about $r$ add at most $\text{OPT}_K$ to the cost of the algorithm. Next, note that we are taking $K - |W_1|$ vertices in the tour $T_2$ out of the total $|W_2| - |W_1|$ new vertices in $T_2$, and therefore the shortest subpath has cost less than the average cost of a uniformly chosen subpath of length $K - |W_1|$, which is equal to
    %
    \[ \frac{K - |W_1|}{|W_2| - |W_1|} c(T_2) = \alpha_2 c(T_2) \]
    %
    In conclusion, this means that $c(C) \leq \text{OPT}_K + \alpha_2 c(T_2)$, and so
    %
    \begin{align*}
        d(T_1) + d(C) &\leq d(T_1) + \alpha_2 d(T_2) + \text{OPT}_K\\
        &\leq \alpha_1^{-1}[\alpha_1 d(T_1) + \alpha_2 d(T_2)] + \text{OPT}_K\\
        &\leq [\alpha_1^{-1} (3 + \varepsilon) + 1] \text{OPT}_K
    \end{align*}
    %
    This gives the required bound.
\end{proof}

For any two values of $\alpha_1$ and $\alpha_2$, we have constructed two approximations to the $K$ VRP problem, and therefore choosing the minimum of these two solution gives a
%
\[ \max_{0 \leq x \leq 1} \min \left( \frac{3 + \varepsilon}{x} + 1, \frac{3 + \varepsilon}{1 - x} \right) = \frac{7 + \sqrt{37}}{2} + O(\varepsilon) \approx 6.541 + O(\varepsilon) \]
%
approximation to $K$ VRP. Proceeding along the same as with the split case, we can also find an approximation to the $K$ unsplit vehicle routing problem, using the Lagrangian preserving $4$ approximation hinted at in the last section. The minimization here is a
%
\[ \max_{0 \leq x \leq 1} \min \left( \frac{4 + \varepsilon}{x} + 1, \frac{4 + \varepsilon}{1 - x} \right) = \frac{9 + \sqrt{65}}{2} + O(\varepsilon) \approx 8.531 + O(\varepsilon) \]
%
approximation to unsplit $K$ VRP.

\section{Orienteering}

s

\begin{thebibliography}{9}

\bibitem{RandomRefillPaper}
Anupam Gupta, Viswanath Nagarajan, R. Ravi
\textit{Approximation Algorithms for VRP with Stochastic Demands.}

\bibitem{FlowToTreeResult}
Jorgen Bang-Jensen, Andr\'{a}s Frank, Bill Jackson
\textit{Preserving and Increasing Local Edge Connectivity in Mixed Graphs.}

\bibitem{FlowToTreePolyTimeResult}
Ian Post, Chaitanya Swamy
\textit{Linear-Programming Based Approximation Algorithms For Multi-Vehicle Minimum Latency Problems.}

\bibitem{MSTLPApprox}
Avrim Blum, R. Ravi, Santosh Vempala
\textit{A Constant-Factor Approximation Algorithm for the k-MST Problem.}

\end{thebibliography}

\end{document}