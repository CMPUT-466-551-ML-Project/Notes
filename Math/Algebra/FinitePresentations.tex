\documentclass{article}

\usepackage{amsmath}
\usepackage{algorithm}
%\usepackage{algorithmic}
\usepackage[noend]{algpseudocode}

\usepackage{amsmath}
\usepackage{amssymb}
\usepackage{amsthm}
\usepackage{amsopn}

\usepackage{kpfonts}

\usepackage{graphicx}

% Probably don't need this on notes anymore
%\usepackage{kbordermatrix}

% Standard tool for drawing diagrams.
\usepackage{tikz}
\usepackage{tkz-berge}
\usepackage{tikz-cd}
\usepackage{tkz-graph}

\usepackage{comment}

%
\usepackage{multicol}

%
\usepackage{framed}

%
\usepackage{mathtools}

%
\usepackage{float}

%
\usepackage{subfig}

%
\usepackage{wrapfig}

%
\let\savewideparen\wideparen
\let\wideparen\relax
\usepackage{mathabx}
\let\wideparen\savewideparen

% Used for generating `enlightening quotes'
\usepackage{epigraph}

% Forget what this is used for :P
\usepackage[utf8]{inputenc}

% Used for generating quotes.
\usepackage{csquotes}

% Allows what to generate links inside
% generated pdf files
\usepackage{hyperref}

% Allows one to customize theorem
% environments in mathematical proofs.
\usepackage{thmtools}

% Gives access to a proof
\usepackage{lplfitch}

% I forget what this is for.
\usepackage{accents}

% A package for drawing simple trees,
% as a substitute for unnesacary TIKZ code
\usepackage{qtree}

% Enables sequent calculus proofs
\usepackage{ebproof}

% For braket notation
\usepackage{braket}

% To change line spacing when using mathematical notations which require some height!
\usepackage{setspace}

%\usepackage[dvipsnames]{xcolor}

\usepackage{float}

% For block commenting
\usepackage{comment}




\setlength\epigraphwidth{8cm}

\usetikzlibrary{arrows, petri, topaths, decorations.markings}

% So you can do calculations in coordinate specifications
\usetikzlibrary{calc}
\usetikzlibrary{angles}

\theoremstyle{plain}
\newtheorem{theorem}{Theorem}
\newtheorem{axiom}{Axiom}
\newtheorem{lemma}[theorem]{Lemma}
\newtheorem{corollary}[theorem]{Corollary}
\newtheorem{prop}[theorem]{Proposition}
\newtheorem{exercise}{Exercise}
\newtheorem{fact}{Fact}

\newtheorem*{example}{Example}
\newtheorem*{proof*}{Proof}

\theoremstyle{remark}
\newtheorem*{exposition}{Exposition}
\newtheorem*{remark}{Remark}
\newtheorem*{remarks}{Remarks}

\theoremstyle{definition}
\newtheorem*{defi}{Definition}

\usepackage{hyperref}
\hypersetup{
    colorlinks = true,
    linkcolor = black,
}

\usepackage{textgreek}

\makeatletter
\renewcommand*\env@matrix[1][*\c@MaxMatrixCols c]{%
  \hskip -\arraycolsep
  \let\@ifnextchar\new@ifnextchar
  \array{#1}}
\makeatother

\renewcommand*\contentsname{\hfill Table Of Contents \hfill}

\newcommand{\optionalsection}[1]{\section[* #1]{(Important) #1}}
\newcommand{\deriv}[3]{\left. \frac{\partial #1}{\partial #2} \right|_{#3}} % partial derivative involving numerator and denominator.
\newcommand{\lcm}{\operatorname{lcm}}
\newcommand{\im}{\operatorname{im}}
\newcommand{\bint}{\mathbf{Z}}
\newcommand{\gen}[1]{\langle #1 \rangle}

\newcommand{\End}{\operatorname{End}}
\newcommand{\Mor}{\operatorname{Mor}}
\newcommand{\Id}{\operatorname{id}}
\newcommand{\visspace}{\text{\textvisiblespace}}
\newcommand{\Gal}{\text{Gal}}

\newcommand{\xor}{\oplus}
\newcommand{\ft}{\wedge}
\newcommand{\ift}{\vee}

\newcommand{\prob}{\mathbf{P}}
\newcommand{\expect}{\mathbf{E}}
\DeclareMathOperator{\Var}{\mathbf{V}}
\newcommand{\Ber}{\text{Ber}}
\newcommand{\Bin}{\text{Bin}}

%\newcommand{\widecheck}[1]{{#1}^{\ft}}

\DeclareMathOperator{\diam}{\text{diam}}

\DeclareMathOperator{\QQ}{\mathbf{Q}}
\DeclareMathOperator{\ZZ}{\mathbf{Z}}
\DeclareMathOperator{\RR}{\mathbf{R}}
\DeclareMathOperator{\HH}{\mathbf{H}}
\DeclareMathOperator{\CC}{\mathbf{C}}
\DeclareMathOperator{\PP}{\mathbf{P}}
\DeclareMathOperator{\MM}{\mathbf{M}}
\DeclareMathOperator{\VV}{\mathbf{V}}
\DeclareMathOperator{\TT}{\mathbf{T}}
\DeclareMathOperator{\LL}{\mathcal{L}}
\DeclareMathOperator{\EE}{\mathbf{E}}
\DeclareMathOperator{\NN}{\mathbf{N}}
\DeclareMathOperator{\DQ}{\mathcal{Q}}
\DeclareMathOperator{\IA}{\mathfrak{a}}
\DeclareMathOperator{\IB}{\mathfrak{b}}
\DeclareMathOperator{\IC}{\mathfrak{c}}
\DeclareMathOperator{\IP}{\mathfrak{p}}
\DeclareMathOperator{\IM}{\mathfrak{m}}
\DeclareMathOperator{\IN}{\mathfrak{n}}
\DeclareMathOperator{\ord}{\text{ord}}
\DeclareMathOperator{\Ker}{\textsf{Ker}}
\DeclareMathOperator{\Coker}{\textsf{Coker}}
\DeclareMathOperator{\emphcoker}{\emph{coker}}
\DeclareMathOperator{\pp}{\partial}
\DeclareMathOperator{\tr}{\text{tr}}

\DeclareMathOperator{\supp}{\text{supp}}

\DeclareMathOperator{\codim}{\text{codim}}

\DeclareMathOperator{\minkdim}{\dim_{\mathbf{M}}}
\DeclareMathOperator{\hausdim}{\dim_{\mathbf{H}}}
\DeclareMathOperator{\lowminkdim}{\underline{\dim}_{\mathbf{M}}}
\DeclareMathOperator{\upminkdim}{\overline{\dim}_{\mathbf{M}}}
\DeclareMathOperator{\lhdim}{\underline{\dim}_{\mathbf{M}}}
\DeclareMathOperator{\lmbdim}{\underline{\dim}_{\mathbf{MB}}}
\DeclareMathOperator{\packdim}{\text{dim}_{\mathbf{P}}}
\DeclareMathOperator{\fordim}{\dim_{\mathbf{F}}}

\DeclareMathOperator*{\argmax}{arg\,max}
\DeclareMathOperator*{\argmin}{arg\,min}

\title{Finite Presentations and Tensor Products}
\author{Jacob Denson}

\begin{document}

\maketitle

A \emph{finite free presentation} of an $R$-module $N$ is an exact sequence
%
\[ R^n \to R^m \to N \to 0. \]
%
More explicitly, a finite free presentation is a pair of maps $i: R^n \to R^m$ and $f: R^m \to N$ such that $f$ is surjective, and $f(x) = 0$ if and only if $x$ is in the image of $i$. Why are they useful for the qual?
%
\begin{itemize}
	\item They're super duper useful to understand the structure of $M \otimes_R N$ without having to think too hard about what's going on in the tensor of the two modules.
	\item They always exist for a finitely generated module over a Noetherian ring, and they're often really easy to compute.
\end{itemize}
%
In these notes I'll explain how to construct a finite presentation easily and show how they can be used to understand the tensor product $M \otimes_R N$. To begin with, let's show how to construct a finite free presentation for any finitely generated module $N$ in three easy steps:
%
\begin{itemize}
	\item[(Step 1):] If $N$ is finitely generated, we can find a basis $v_1,\dots,v_n$ for $N$, and so there exists a surjective $R$-morphism $f: R^m \to N$ such that for $x \in R^m$,
	%
	\[ f(x) = x_1v_1 + \dots + x_mv_m. \]
	%
	This is already the first map in your finite presentation of $N$.

	\item[(Step 2):] Compute the kernel $K$ of the map $f$ and consider the inclusion map $i': K \to R^m$. The sequence
	%
	\[ 0 \to K \to R^m \to N \to 0 \]
	%
	is exact, but $K$ is not necessarily free so this isn't a finite presentation. The last step fixes this.

	\item[(Step 3):] Since $R$ is Noetherian, all finitely generated modules over $R$ are Noetherian. In particular, this implies $K$ \emph{has to be} finitely generated. Find a generating set $k_1, \dots, k_n$ for $K$, which induces a surjective map $g: R^n \to K$ defined by setting
	%
	\[ g(x) = x_1k_1 + \dots + x_nk_n. \]
	%
	Setting $i = i' \circ g$ gives the second map in the finite presentation.
\end{itemize}
%
Not too bad, right? Let's consider two examples.

\begin{example}
	Let's consider the $\ZZ$ module $\ZZ/n \ZZ$. It is generated by a single element, $1 + n \ZZ$, so we let $f: \ZZ \to \ZZ / n \ZZ$ be the map
	%
	\[ f(k) = k + n \ZZ. \]
	%
	The kernel of this map is precisely $K = (n)$. Now as a $\ZZ$ module, $K$ is generated by a single element, namely $n$. Thus we let $i: \ZZ \to \ZZ$ be the map given by setting $i(k) = nk$. The pair of maps $i$ and $f$ gives a finite presentation of $\ZZ / n \ZZ$.
\end{example}

\begin{example}
	Let $I = (x,y)$ be an ideal in $R = \CC[x,y]$. Then $I$ is a finitely generated module over $R$, and so we can compute a finite presentation of $I$. The elements $\{ x,y \}$ give a generating set for $I$, so we consider the map $f: R^2 \to I$ such that for each $a \in R^2$,
	%
	\[ f(a) = a_1x + a_2y. \]
	%
	Next, we compute the kernel of $f$; this is equal to
	%
	\[ K = \{ a \in R^2 : a_1x + a_2y = 0 \}. \]
	%
	Now we have to compute a generating set for $K$. Note that if $a_1x + a_2y = 0$, then $a_1$ is divisible by $y$, and $a_2$ is divisible by $x$. Thus we can write $a_1 = y b_1$, and $a_2 = x b_1$, and we find
	%
	\[ (b_1 + b_2) xy = 0 \]
	%
	which implies $b_1 + b_2 = 0$. Thus we can write
	%
	\[ K = \{ (yb, -xb): b \in R \} = R (y,-x). \]
	%
	Thus the element $(y,-x)$ generates $K$. In particular, we can consider the map $i: R \to R^2$ given by $i(a) = a(y,-x) = (ay,-ax)$, which completes the construction of the finite presentation of $I$.
\end{example}

Now let's show how we can use finite presentations to make the study of tensor products easy. There are two principles at work here:
%
\begin{itemize}
	\item Tensoring is a \emph{right exact} operation.
	\item Tensoring with a free module is a fairly benign operation.
\end{itemize}
%
Firstly, tensoring is a \emph{right exact} operation. That means that if $M$, $N$, $L$, and $K$ are modules, and
%
\[ N \to L \to K \to 0 \]
%
is an exact sequence, then
%
\[ M \otimes N \to M \otimes L \to M \otimes K \to 0 \]
%
is an exact sequence. In other words, suppose $f: N \to L$ and $g: L \to K$ are a pair of $R$-linear homomorphisms such that $g$ is surjective, and $g(x) = 0$ if and only if $x$ is in the image of $f$. Consider the two maps $f': M \otimes N \to M \otimes L$ and $g': M \otimes L \to M \otimes K$ given by
%
\[ f'(m \otimes x) = m \otimes f(x) \quad\text{and}\quad g'(m \otimes x) = m \otimes g(x), \]
%
Then $g'$ is a surjective map, and for any $a \in M \otimes L$, $g'(a) = 0$ if and only if $a$ is in the image of the map $f'$.

Why does this relate to finite presentations? Well, if we have an exact sequence
%
\[ R^n \to R^m \to N \to 0 \]
%
then the above technique gives us an exact sequence
%
\[ M \otimes R^n \to M \otimes R^m \to M \otimes N \to 0 \]
%
The modules $M \otimes R^n$ and $M \otimes R^m$ are easily understood. In particular, for any $k$, $M \otimes R^k$ is isomorphic to $M^k$ under the isomorphism $t_k: M^k \to M \otimes R^k$ given by
%
\[ t_k(m_1,\dots,m_k) = m_1 \otimes (1,0,\dots,0) + m_2 \otimes (0,1,0,\dots,0) + \dots + m_k \otimes (0,\dots,0,1). \]
%
with inverse map
%
\[ t_k^{-1}(m \otimes (x_1,\dots,x_k)) = (x_1m, \dots, mx_k). \]
%
Using these isomorphisms, we can easily compute an exact sequence
%
\[ M^n \to M^m \to M \otimes N \to 0 \]
%
That is, we can find two maps $i_1: M^n \to M^m$ and $f_1: M^m \to M \otimes N$ such that $f_1$ is surjective, and $f_1(x) = 0$ if and only if $x$ is in the image of $i_1$. These maps are explicitly computable using the equations
%
\[ i_1 = t_m^{-1} \circ i' \circ t_n \quad\text{and}\quad f_1 = f \circ t_m. \]
%
Provided that the modules $M^n$ and $M^m$ are understood, these structures essentially give a complete description of the algebraic structure of $M \otimes N$. In particular, the first isomorphism theorem implies that $M \otimes N$ is isomorphic to the quotient of $M^m$ by $i_1(M^n)$.

\begin{example}
	Previously, we gave a finite presentation of the module $\ZZ / n \ZZ$ by the maps $i: \ZZ \to \ZZ$ and $f: \ZZ \to \ZZ / n \ZZ$ given by setting $i(k) = nk$ and $f(k) = k + n\ZZ$. Let us use this presentation to compute the structure of $(\ZZ / m \ZZ) \otimes (\ZZ / n \ZZ)$. The maps $i$ and $f$ induce maps
	%
	\[ i': (\ZZ / m \ZZ) \otimes \ZZ \to (\ZZ / m \ZZ) \otimes \ZZ \quad\text{and}\quad f': (\ZZ / m \ZZ) \otimes \ZZ \to (\ZZ / m \ZZ) \otimes (\ZZ / n \ZZ), \]
	%
	where
	%
	\[ i'((k_1 + m \ZZ) \otimes k_2) = (k_1 + m \ZZ) \otimes nk_2 \]
	%
	and
	%
	\[ f'((k_1 + m \ZZ) \otimes k_2) = (k_1 + m \ZZ) \otimes (k_2 + n \ZZ). \]
	%
	Using the isomorphism $t_1: (\ZZ / m \ZZ) \to (\ZZ / m \ZZ) \otimes \ZZ$ we obtain an exact sequence
	%
	\[ \ZZ / m \ZZ \to \ZZ / m \ZZ \to (\ZZ / m \ZZ) \otimes (\ZZ / n \ZZ) \to 0 \]
	%
	Given by maps $i_0: \ZZ / m \ZZ \to \ZZ / m \ZZ$ and $f_0: \ZZ / m \ZZ \to (\ZZ / m \ZZ) \otimes (\ZZ / n \ZZ)$ where
	%
	\[ i_0(k + m \ZZ) = nk + m \ZZ \quad\text{and}\quad f_0(k + m \ZZ) = (k + m \ZZ) \otimes (1 + n \ZZ). \]
	%
	Before we compute these maps, let us show how they give the structure of $(\ZZ / m \ZZ) \otimes (\ZZ / n \ZZ)$. Firstly, the fact that $f_0$ is surjective implies that any element of $(\ZZ / m \ZZ) \otimes (\ZZ / n \ZZ)$ can be written as
	%
	\[ (k + m \ZZ) \otimes (1 + n \ZZ) \]
	%
	for some integer $k$. Secondly, the exactness implies that
	%
	\[ (k + m \ZZ) \otimes (1 + n \ZZ) = 0 \]
	%
	if and only if there exists an integer $k_0$ such that $k = n k_0$ modulo $m$, which means that there is an integer $k_1$ such that $k = nk_0 + mk_1$. This completely describes the structure of the tensor product. In particular, it shows that the map $k \mapsto (k + m \ZZ) \otimes (1 + n \ZZ)$ is a surjective homomorphism from $\ZZ$ to $(\ZZ / m \ZZ) \otimes (\ZZ / n \ZZ)$ with kernel $n \ZZ + m \ZZ = \text{gcd}(n,m) \ZZ$, so we conclude
	%
	\[ (\ZZ / m \ZZ) \otimes (\ZZ / n \ZZ) \cong \ZZ / \text{gcd}(n,m) \ZZ. \]
	%
	In particular, if $n$ and $m$ are relatively prime, then
	%
	\[ (\ZZ / m \ZZ) \otimes (\ZZ / n \ZZ) = 0. \]
	%
	To obtain these explicit formulas for $i_0$ and $f_0$, we just remember that $i_0 = t_1^{-1} \circ i' \circ t_1$ and $f_0 = f' \circ t_1$, and compute that
	%
	\begin{align*}
		i_0(k + m \ZZ) &= (t_1^{-1} \circ i' \circ t_1)(k + m \ZZ)\\
		&= (t_1^{-1} \circ i')((k + m \ZZ) \otimes 1)\\
		&= t_1^{-1}((k + m \ZZ) \otimes n)\\
		&= n(k + m \ZZ) = nk + m \ZZ,
	\end{align*}
	%
	and
	%
	\begin{align*}
		f_0(k + m \ZZ) &= (f' \circ t_1)(k + m \ZZ)\\
		&= f'((k + m \ZZ) \otimes 1)\\
		&= (k + m \ZZ) \otimes (1 + n \ZZ).
	\end{align*}
\end{example}

The calculation of $(\ZZ/m \ZZ) \otimes (\ZZ / n \ZZ)$ is quite easy to calculate without using finite presentations, so let us move on to a more nontrivial tensor product.

\begin{example}
	Let us use the finite presentation of the ideal $I = (x,y)$ of $R = \CC[x,y]$ to show that the element
	%
	\[ x \otimes y - y \otimes x \in I \otimes I \]
	%
	is non-zero. In this situation the finite presentation of $I$ was given by the pair of maps $i: R \to R^2$ and $f: R^2 \to I$ given by the formulae
	%
	\[ i(a) = (ay,-ax) \quad\text{and}\quad f(a_1,a_2) = xa_1 + ya_2. \]
	%
	These maps induce maps $i_0: I \otimes R \to I \otimes R^2$ and $f_0: I \otimes R^2 \to I \otimes I$ given by
	%
	\[ i_0(a \otimes b) = a \otimes (by, -bx) \quad\text{and}\quad f_0(a \otimes (b_1,b_2)) = a \otimes (xb_1 + yb_2). \]
	%
	Whatmore, they give an exact sequence
	%
	\[ I \otimes R \to I \otimes R^2 \to I \otimes I \to 0. \]
	%
	We have isomorphisms $t_1: I \to I \otimes R$ and $t_2: I^2 \to I \otimes R^2$ given by
	%
	\[ t_1(a) = a \otimes 1 \quad\text{and}\quad t_2(a_1,a_2) = a_1 \otimes (1,0) + a_2 \otimes (0,1). \]
	%
	and we therefore obtain an exact sequence
	%
	\[ I \to I^2 \to I \otimes I \to 0 \]
	%
	given by maps $i_1: I \to I^2$ and $f_1: I^2 \to I \otimes I$ which we will explicitly compute as
	%
	\[ i_1(a) = (ay,-ax) \quad\text{and}\quad f_1(a_1,a_2) = a_1 \otimes x + a_2 \otimes y. \]
	%
	Now we can easily show that $x \otimes y - y \otimes x$ is nonzero. Note that
	%
	\[ f_1(-y,x) = -y \otimes x + x \otimes y = x \otimes y - y \otimes x. \]
	%
	If $x \otimes y - y \otimes x = 0$, then $f_1(-y,x) = 0$. Exactness implies that there must exist some $a \in I$ such that $i_1(a) = (-y,x)$. But $i_1(a) = (ay,-ax)$, so we find that there is $a \in I$ such that $ay = -y$, and $-ax = x$. This can only be true if $a = -1$, which is impossible since $-1 \not \in I$. Thus we have shown $x \otimes y - y \otimes x \neq 0$! The computations here might not be as simple as using the bilinear map given by partial derivatives discussed in the class today, but we didn't really have to think at all during this process; all the computations were fairly automatic. Moreover, this calculation gives much greater knowledge of $I \otimes I$. Namely, the surjectivity of $f_1$ implies that any element of $I \otimes I$ can be written in the form
	%
	\[ a_1 \otimes x + a_2 \otimes y \]
	%
	where $a_1, a_2 \in I$, and moreover, this element is equal to zero in $I \otimes I$ if and only there are polynomials $t_1,t_2 \in \CC[x,y]$ such that $a_1 = t_1xy + t_2y^2$, and $a_2 = t_1y^2 - t_2 xy$. This might be useful on the qual if we were asked other questions about $I \otimes I$. To finish this calculation, we explicitly calculate $f_1$ and $i_1$. But we simply calculate that
	%
	\begin{align*}
		i_1(a) &= (t_2^{-1} \circ i' \circ t_1)(a)\\
		&= (t_2^{-1} \circ i')(a \otimes 1)\\
		&= t_2^{-1}(a \otimes (y,-x))\\
		&= (ay,-ax)
	\end{align*}
	%
	and
	%
	\begin{align*}
		f_1(a_1,a_2) &= (f' \circ t_2)(a_1,a_2)\\
		&= f'(a_1 \otimes (1,0) + a_2 \otimes (0,1))\\
		&= a_1 \otimes x + a_2 \otimes y.
	\end{align*}
	%
	As long as you're careful, these calculations are trivial.
\end{example}

This technique completely fails if both modules in a tensor product are not finitely generated. However, they still work if one of your modules are finitely generated. Let us consider an example. Try and use the techniques we have discussed to compute the tensor product before looking at the solution.

\begin{example}
	Let us compute the tensor product $\RR \otimes_{\ZZ} (\ZZ^2 \times \ZZ_3)$. We begin by noting that $\ZZ^2 \times \ZZ_3$ has a finite presentation
	%
	\[ \ZZ \to \ZZ^3 \to \ZZ^2 \times \ZZ_3 \to 0 \]
	%
	given by maps $i: \ZZ \to \ZZ^3$ and $f: \ZZ^3 \to \ZZ^2 \times \ZZ_3$ where
	%
	\[ f(n_1,n_2,n_3) = (n_1,n_2,n_3 + 3 \ZZ) \quad\text{and}\quad i(n) = (0,0,3n). \]
	%
	Tensoring gives an exact sequence
	%
	\[ \RR \otimes \ZZ \to \RR \otimes \ZZ^3 \to \RR \otimes (\ZZ^2 \times \ZZ_3) \to 0 \]
	%
	and thus an exact sequence
	%
	\[ \RR \to \RR^3 \to \RR \otimes (\ZZ^2 \times \ZZ^3) \to 0. \]
	%
	I leave it as an exercise to show that the maps
	%
	\[ i_1: \RR \to \RR^3 \quad\text{and}\quad f_1: \RR^3 \to \RR \otimes (\ZZ^2 \times \ZZ^3) \]
	%
	are given by
	%
	\[ i_1(t) = (0,0,3t) \]
	%
	and
	%
	\[ f_1(t_1,t_2,t_3) = t_1 \otimes (1,0,0 + 3\ZZ) + t_2 \otimes (0,1,0 + 3\ZZ) + t_3 \otimes (0,0,1 + 3 \ZZ). \]
	%
	Thus we conclude from the exactness of the sequence $\RR \to \RR^3 \to \RR \otimes (\ZZ^2 \times \ZZ_3)$ that every element of $\RR \otimes (\ZZ^2 \times \ZZ_3)$ can be written as
	%
	\[ t_1 \otimes (1,0,0 + 3 \ZZ) + t_2 \otimes (0,1,0 + 3 \ZZ) + t_3 \otimes (0,0,1 + 3 \ZZ). \]
	%
	Moreover, this element is equal to zero in the tensor product if and only if $t_1 = 0$ and $t_2 = 0$. In particular, it is easy to see from these properties that
	%
	\[ \RR \otimes_{\ZZ} (\ZZ^2 \times \ZZ_3) \cong \RR \otimes_{\ZZ} \ZZ^2 \cong \RR^2. \]
\end{example}

\end{document}