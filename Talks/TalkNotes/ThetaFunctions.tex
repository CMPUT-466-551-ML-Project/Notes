\documentclass{article}

\usepackage{amsmath}
\usepackage{amssymb}
\usepackage{amsthm}
\usepackage{esint}
\usepackage{MnSymbol}

\theoremstyle{plain}
\newtheorem{theorem}{Theorem}
\newtheorem{lemma}[theorem]{Lemma}
\newtheorem{corollary}[theorem]{Corollary}
\newtheorem{prop}[theorem]{Proposition}

\theoremstyle{remark}
\newtheorem*{example}{Example}
\newtheorem*{remark}{Remark}

\theoremstyle{definition}
\newtheorem*{defi}{Definition}
\newenvironment{definition}
    {\begin{samepage}\begin{framed}\begin{defi}}
    {\end{defi}\end{framed}\end{samepage}}

\title{Theta Functions}
\author{Jacob Denson}
\date{\today}

\begin{document}

\maketitle

In this talk, we discuss a family of modular forms with some important applications to number theory, known as Theta functions. The most basic is the {\bf Jacobi Theta function}, a holomorphic function defined by the $q$ series
%
\[ \theta(\tau) = \sum_{n = -\infty}^\infty e^{\pi i n^2 \tau} \]
%
It arises both as a fundamental solution to the heat equation on the real line, and as a generating function representing certain number theory problems involving squares of integers. It is a special case of the more general {\bf Theta function}
%
\[ \Theta(z,\tau) = \sum_{n = -\infty}^\infty e^{\pi i n^2 \tau} e^{2 \pi i n z} \]
%
For each disk in $z$ and proper upper half plane, the series converges uniformly, and so we get a holomorphic function entire in $z$, and holomorphic for $\tau \in \mathbf{H}$. For convention, even though we have used $q$ to denote $e^{2 \pi i \tau}$, we in this talk denote $q$ by $e^{\pi i \tau}$. Thus $\smash{\theta(\tau) = \sum q^{n^2}}$ and $\smash{\Theta(z,\tau) = \sum e(n z) q^{n^2}}$.

The great thing about $\Theta$ is it's elliptic properties in $z$, and modular properties in $\tau$. It is easy to see that $\Theta(z + 1|\tau) = \Theta(z|\tau)$, so the function is periodic. A manipulation gives
%
\begin{align*}
    \Theta(z+ \tau|\tau) &= \sum_{n = -\infty}^\infty e^{\pi i (n^2 + 2n) \tau} e^{2 \pi i n z}\\
    &= e^{-\pi i \tau} \sum_{n = -\infty}^\infty e^{\pi i (n+1)^2 \tau} e^{2 \pi i n z} = q^{-1} e^{-2 \pi i z} \Theta(z|\tau)
\end{align*}
%
so we have an `almost $\tau$ periodicity' for $\Theta$. When it comes to modular transformations, we have $\Theta(z|\tau + 2) = \Theta(z|\tau)$. To obtain an interesting modular transformation rule, we apply Poisson summation. Consider the function
%
\[ f(y) = e^{- \pi t(y + x)^2} \]
%
Then $(\delta_{-x} \circ M_{t^{1/2}})(e^{- \pi y^2}) = f$, and so since $e^{- \pi y^2}$ is it's own Fourier transform,
%
\[ \widehat{f}(\xi) = \frac{e^{- 2 \pi i \xi x}}{t^{1/2}} e^{- \pi \xi^2/t} \]
%
Applying Poisson's summation formula, we conclude that
%
\[ e^{- \pi t x^2} \sum_{n = -\infty}^\infty e^{- \pi t n^2} e^{- 2 \pi n t x} = \sum_{n = -\infty}^\infty e^{- \pi t(n + x)^2} = \frac{1}{t^{1/2}} \sum_{n = -\infty}^\infty e^{2 \pi i n x} e^{-\pi n^2/t} \]
%
We can rearrange this formula to give something interesting about the Theta function. The left hand side is $e^{- \pi t x^2} \Theta(xit|it)$ and the right hand side is $\Theta(x|i/t)/t^{1/2}$. Writing $z = x$ and $\tau = it$, then the equation can be rearranged to read
%
\[ \Theta(z|-1/\tau) = \sqrt{-i\tau} \cdot e^{\pi i \tau z^2} \cdot \Theta(z \tau | \tau) \]
%
Both sides of this equation are holomorphic, and we have shown the equation holds for a non isolated set of values $z$ and $\tau$. Because of this, the equation actually holds {\it everywhere}. An important case of this is that
%
\[ \theta(-1/\tau) = \Theta(0|-1/\tau) = \sqrt{-i\tau} \cdot \theta(\tau) \]
%
Thus $\theta$ has a modular character with respect to the subgroup of $\Gamma$ generated by the transformations $T: \tau \mapsto \tau + 2$ and $S: z \mapsto -1/\tau$. These transformations give two, nonequivalent cusp orbits. We have given the $q$ expansion about the cusp orbit corresponding to $\infty$. Using $\Theta$, we can cheat out an expansion about the other set of cusps containing 1. We calculate that
%
\[ \theta(1 + 2 \tau) = \sum_{n = -\infty}^\infty (-1)^{n^2} e^{i \pi n^2 \tau} = \Theta(1/2|\tau) \]
%
So
%
\begin{align*}
    \theta(1 - 1/\tau) &= \Theta(1/2|-1/\tau) = \sqrt{-i\tau} \cdot e^{\pi i \tau/4} \cdot \Theta(\tau/2 | \tau)\\
    &= \sqrt{-i\tau} \cdot \sum_{n = -\infty}^\infty e^{\pi i \tau(n^2 + n + 1/4)} = \sqrt{-i\tau} \cdot \sum_{n = -\infty}^\infty e^{\pi i \tau(n + 1/2)^2}
\end{align*}
%
In particular, we find $\theta(1 - 1/\tau) \sim 2 \sqrt{-i\tau} \cdot e^{i \pi \tau/4}$ as $\text{Im}(\tau) \to \infty$.

\section{The Modular Character}

The transformations $T$ and $S$ generate a subgroup of $\Gamma$, which $\Theta$ has modular symmetry to. Following essentially the same proof as for the generators of $\Gamma$, we can prove that these two transformations generate
%
\[ \Gamma_2(2) = \left\{ z \mapsto \frac{az + b}{cz + d}: a \equiv d\ (\text{mod}\ 2), b \equiv c\ (\text{mod}\ 2), a \not \equiv b\ (\text{mod}\ 2) \right\} \]
%
If we consider the reduction map $R: SL_2(\mathbf{Z}) \to SL_2(\mathbf{Z}_2)$, then
%
\[ \Gamma_2(2) = R^{-1} \left( \begin{pmatrix} 1 & 0 \\ 0 & 1 \end{pmatrix}, \begin{pmatrix} 0 & 1 \\ 1 & 0 \end{pmatrix} \right) \]
%
Since $GL_2(\mathbf{Z}_2) = SL_2(\mathbf{Z}_2)$ has $6$ elements, $\Gamma_2(2)$ is the inverse of an index 3 subgroup of $SL_2(\mathbf{Z}_2)$. In particular, this means $\Gamma_2(2)$ is also an index 3 subgroup of $\Gamma$. And we notice that $D = \{ \tau \in \mathbf{H}: |\text{Re}(\tau)| \leq 1\ |\tau| \geq 1 \}$ is a fundamental domain for $\Gamma_2(2)$, which is essentially three copies of the fundamental domain of $\Gamma$. We have two orbits of cusp points, and three orbits of elliptic points. In particular, the dimension of $M_k(\Gamma_2(2))$ is bounded above by $k/4 + 1$. Thus $M_1(\Gamma_2(2)), M_2(\Gamma_2(2))$, $M_3(\Gamma_2(2))$ are all at most $1$ dimensional, whereas the space $M_4(\Gamma_2(2))$ is at most $2$ dimensional.

Unfortunately, $\theta$ is not really a modular form over $\Gamma_2(2)$, even of half weight, because of the presence of the factor $\sqrt{i}$ in it's definition. One can define modular forms with a `nebentypus coefficient', and prove facts about the dimension of such spaces, which would suffice to classify the behaviour of $\theta^2$, but we prefer to do things in a much more elementary way. Given $f$ and $g$ of the same weight, with
%
\[ f(-1/\tau) = (-i\tau)^{k/2} f(\tau)\ \ \ \ g(-1/\tau) = (-i\tau)^{k/2} g(\tau) \]
%
We find
%
\[ \frac{f(-1/\tau)}{g(-1/\tau)} = (-i\tau)^{k/2 - k/2} \frac{f(\tau)}{g(\tau)} = \frac{f(\tau)}{g(\tau)} \]
%
Thus, if we can prove that $f/g$ is holomorphic at cusps, then $f/g$ is actually a universal constant, so there is $A$ such that $f(\tau) = Ag(\tau)$.

\section{Sums of Squares}

We let $r_2(N)$ denote the number of integers $n,m \in \mathbf{Z}^2$ such that $n^2 + m^2 = N$. Since $\theta(z) = \sum q^{n^2}$, $\theta^2(z) = \sum q^{n^2 + m^2}$, so the $N$'th coefficient in the $q$ series expansion of $\theta^2$ is precisely $r_2(N)$. Our proof that integers can be expanded by sums of perfect squares follows from a simple identity between modular forms, that
%
\[ \theta(\tau)^2 = 2 \sum_{n = -\infty}^\infty \frac{1}{q^n + q^{-n}} = 1 + 4 \sum_{n = 1}^\infty \frac{q^{n}}{1 + q^{2n}} = 1 + 4 \sum_{n = 1}^\infty \frac{q^n}{1 - q^{4n}} - \frac{q^{3n}}{1 - q^{4n}} \]
%
Now plugging in the identity
%
\[ \frac{1}{1 - q^{4n}} = \sum_{m = 0}^\infty q^{4nm} \]
%
Thus expanding out the right hand side gives
%
\[ \sum \frac{q^{2n}}{1 - q^{4n}} = \sum q^{n(4m + 1)} = \sum d_1(k) q^k\ \ \ \ \sum \frac{q^{3n}}{1 - q^{4n}} = \sum q^{n(4m + 3)} = d_3(N) q^N \]
%
where $d_1(N)$ and $d_3(N)$ denote the number of divisors of an integer of the form $4k + 1$ and $4k + 3$ respectively. Thus provided we are able to obtain the first identity, we conclude that $r_2(N) = 4(d_1(N) - d_3(N))$. In particular, we conclude that an integer $N$ is the sum of two perfect squares if and only if every prime $p$ congruent to three modulo four divides into $N$ an even number of times. TODO: Prove this.

Thus it remains to prove the initial identity. We start by studying the right hand side, which can be written temporarily as
%
\[ f(\tau) = 2 \sum \frac{1}{q^n + q^{-n}} = \sum_{n = -\infty}^\infty \frac{1}{\cos(n \pi \tau)} \]
%
Note $f(\tau + 2) = f(\tau)$. We consider the trigonometric identity
%
\[ \sum_{n = -\infty}^\infty \frac{1}{\cosh(\pi n t)} = \frac{1}{t} \sum_{n = -\infty}^\infty \frac{1}{\cosh(\pi n/t)} \]
%
which gives $f(-1/\tau) = -i \tau f(\tau)$, the same transformation law as for $\theta^2$. They also satisfy $f(i \infty) = 1 = \theta^2(i \infty)$. Since the dimension of meromorphic functions satisfying the modular transformation law is one, this gives the required equality. However, we don't really know this, because we don't know about nebentypus coefficient modular forms, so we have to do a bit of extra work. There is an extra trigonometric identity
%
\[ \sum_{n = -\infty}^\infty \frac{(-1)^n}{\cosh(\pi n / t)} = t \sum_{n = -\infty}^\infty \frac{1}{\cosh(\pi(n + 1/2)t)} \]
%
Analytically continuing this identity, we conclude
%
\[ f(1 - 1/\tau) = -i \tau \sum_{n = -\infty}^\infty \frac{1}{\cos(\pi(n + 1/2)\tau)} \]
%
Which gives $f(1 - 1/\tau) \sim -4 i \tau \cdot e^{\pi i \tau/2}$ as $\text{Im}(\tau) \to \infty$. But this is also true of the square of the theta function $\theta^2(1 - 1/\tau) \sim -4i \tau \cdot e^{\pi i \tau/2}$. But now we have proved that $\theta(\tau) / f(\tau)$ is a modular form on $\Gamma_2(2)$ which is bounded at the cusps, and thus constant. Since $\theta^2(i\infty)/f(i\infty) = 1$, we conclude $\theta^2(i\infty) = f(i \infty)$, completing our discussion of squares.

\section{Four Squares}

We now prove Lagrange's theorem that every positive integer is the sum of four squares. Moreover, to do this we will define an exact formula for the number of ways $r_4(N)$ that we can do this for each $N$. If we let $\sigma_1^*(N)$ denote the number of divisors of $N$ which are {\it not} divisible by 4, then we shall prove $r_4(N) = 8 \sigma_1^*(N)$. Since $\sigma_1^*(N) \geq 0$ for all $N \geq 1$, this will complete our proof. Like with two squares, it shall suffice to prove an identity for $\theta^4(\tau)$. To do this, we must dive into the `forbidden' Eisenstein series of weight two. We consider
%
\[ E_2^*(\tau) = \sum_m \sum_n \frac{1}{(m\tau/2 + n)^2} - \sum_m \sum_n \frac{1}{(m \tau + n/2)^2} \]
%
where the order of $m$ and $n$ certainly matters, carefully chosen so that the calculation for the higher weight Eisenstein series still applies, and we find that
%
\[ E_2^*(\tau) = \left( \frac{\pi^2}{3} - 8 \pi^2 \sum_{k = 1}^\infty \sigma_1(k) q^k \right) - 4 \left( \frac{\pi^2}{3} - 8 \pi^2 \sum_{k = 1}^\infty \sigma_1(k) q^{4k} \right) \]
%
The fact that $\sigma_1^*(N) = \sigma_1(N)$ if $N$ is not divisible by four, and $\sigma_1^*(N) = \sigma_1(N) - 4 \sigma_1(N/4)$ if $N$ is divisible by four, gives that
%
\[ E_2^*(\tau) = - \pi^2 - 8 \pi^2 \sum_{k = 1}^\infty \sigma_1^*(k) q^k \]
%
It therefore suffices to prove that $\theta^4(\tau) = -E_2^*(\tau)/\pi^2$. To do this, we need only verify that $E_2^*$ satisfies the same transformational properties as $\theta^4$, and has the same asymptotic properties near the cusps. TODO: PROVE THIS.

\section{Theta Series in Multiple Variables}

We now consider a multivariate generalization of theta series. Given any positive definite quadratic form $Q: \mathbf{Z}^m \to \mathbf{Z}$, we associate a series $\Theta_Q(\tau) = \sum q^{Q(x_1, \dots, x_m)}$, whose $q^N$ coefficient measures the number of representations of an integer as the number of vectors $x$ with $Q(x) = N$. We write $Q(x) = \sum A_{ij} x_ix_j/2$ where $A$ is a positive definite, integral symmetric $m \times m$ matrix, whose diagonal elements are even. Then $A^{-1}$ is a rational matrix, and we define the {\bf level} of $Q$ to be the smallest integer $N$ such that $NA^{-1}$ is an even integral matrix. The {\bf discriminant} is $\Delta_Q = (-1)^m \cdot \det A$. If $m$ is an even integer, we get a precise description of the modular behaviour of the theta series.

\begin{example}
    If $Q(x) = x^2$, then $\Theta_Q = \theta$. It is level one, with discriminant $-1$.
\end{example}

\begin{theorem}[Hecke, Schoenberg]
    Let $Q: \mathbf{Z}^{2k} \to \mathbf{Z}$ be a quadratic form, then $\Theta_Q$ satisfies
\end{theorem}

%\section{Euler's Pentagonal Theorem}

%We will be discussing Theta functions, and their connections with various estimation problems in analytic number theory. Given an integer $N$, we define the {\bf partition function} $p(N)$ to be the number of unordered representations of $N$ as a sum of positive integers. For instance $p(1) = 1$, and $p(4) = 5$, since
%
%\[ 4 = 3 + 1 = 2 + 2 = 2 + 1 + 1 = 1 + 1 + 1 + 1 \]%
%
%We set $p(0) = 0$. The classical way to study sequences of numbers representing additive properties is to form a power series, so in this case we consider the generating function
%
%\[ \sum_{n = 0}^\infty p(n) q^n \]
%
%Just as the product formula for the zeta functions gives analytical information about the function, a product formula for this expansion will also give analytical information.
%\begin{theorem}
%    If $|q| < 1$, then
    %
%    \[ \sum_{n = 0}^\infty p(n) q^n = \prod_{k = 1}^\infty \frac{1}{1 - q^k} \]
%\end{theorem}
%\begin{proof}
%    We provide the gist of the proof. We can write
    %
%    \[ \frac{1}{1 - q^k} = \sum_{n = 1}^\infty q^{nk} \]
    %
%    If we take the infinite product of all these infinite sums, then the coefficient corresponding to each term in the power series expansion corresponding to $q^N$ to the number of ways to represent $N$ as $\sum_{n = 1}^\infty k_n \cdot n$ for some positive integers $k_n$, and this is easily seen to be equal to $p(N)$.
%\end{proof}

%By manipulating the product formula, we get certain variants to the partition function. The infinite product
%
%\[ \prod_{k = 1}^\infty \frac{1}{1 - q^{2k - 1}} \]
%
%gives a function $p_o(N)$, which counts the number of ways we can write $N$ as an unordered sum of {\it odd} integers. If we consider the infinite product
%
%\[ \prod_{k = 1}^\infty (1 + q^k) \]
%
%we get the function $p_u(N)$ which counts the number of ways to write $N$ as a sum of {\it unequal } integers.

%\begin{theorem} $p_u(N) = p_o(N)$ \end{theorem}
%\begin{proof}
%    We calculate
%
%\[ \prod_{k = 1}^\infty (1 - q^k) = \prod_{k = 1}^\infty (1 - q^{2n})(1 - q^{2n - 1}) = \prod (1 + q^k)(1 - q^k)(1 - q^{2k-1}) \]
%
%and dividing both sides by $(1 - q^k)(1 - q^{2k - 1})$ gives
%
%\[ \prod_{k = 1}^\infty (1 + q^k) = \prod_{k = 1}^\infty \frac{1}{1 - q^{2k-1}} \]
%
%from which we can take power series to get the result.
%\end{proof}



\end{document}