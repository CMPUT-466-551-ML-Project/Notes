\documentclass[handout,usenames,dvipsnames]{beamer}

\usepackage{tikz}
\usepackage{tkz-berge}
\usepackage{tkz-graph}
\usepackage{subcaption}

\usetikzlibrary{patterns,arrows,decorations.pathreplacing}

\usepackage{xcolor}
\definecolor{dblue}{RGB}{20,66,129}
\definecolor{rose}{RGB}{255,101,122}
\definecolor{crimsonred}{RGB}{132,22,23}
\definecolor{darkblue}{RGB}{72,61,139}

\definecolor{deepblue}{RGB}{36,123,160}
\definecolor{deepred}{RGB}{255,22,84}
\definecolor{deeporange}{RGB}{240,111,62}

\definecolor{olive}{rgb}{0.3, 0.4, .1}
\definecolor{fore}{RGB}{249,242,215}
\definecolor{back}{RGB}{51,51,51}
\definecolor{title}{RGB}{255,0,90}
\definecolor{dgreen}{rgb}{0.,0.6,0.}
\definecolor{gold}{rgb}{1.,0.84,0.}
\definecolor{JungleGreen}{cmyk}{0.99,0,0.52,0}
\definecolor{BlueGreen}{cmyk}{0.85,0,0.33,0}
\definecolor{RawSienna}{cmyk}{0,0.72,1,0.45}
\definecolor{Magenta}{cmyk}{0,1,0,0}


\DeclareMathOperator{\RR}{\mathbb{R}}
\DeclareMathOperator{\QQ}{\mathbb{Q}}
\DeclareMathOperator{\ZZ}{\mathbb{Z}}
\DeclareMathOperator{\TT}{\mathbb{T}}
\DeclareMathOperator{\PP}{\mathbb{P}}
\DeclareMathOperator{\EE}{\mathbb{E}}
\DeclareMathOperator{\fordim}{\text{dim}_{\mathbb{F}}}
\DeclareMathOperator{\hausdim}{\text{dim}_{\mathbb{H}}}
\DeclareMathOperator{\minkdim}{\text{dim}_{\mathbb{M}}}
\DeclareMathOperator{\lowminkdim}{\text{\underline{dim}}_{\mathbb{M}}}

\title{Sets, Patterns, and Fourier Decay}
\author{Jacob Denson}
\institute{}

\begin{document}

\maketitle

\begin{frame}
    \frametitle{Fourier Analysis and Patterns in Sets}

    \begin{itemize}
        \item What can one learn about the geometry of a compact set $E \subset \TT^d$ via analytical properties of probability measures $\mu$ supported on $E$?
        \pause

        \item A set $E$ has \emph{Minkowski dimension $s$} if $|N_\delta(E)| \lesssim \delta^{d-s}$.
        \pause

        \item A set $E$ has \emph{Hausdorff dimension $s$} if for any $t < s$, $E$ supports a probability measure $\mu_t$ with
        %
        \[ \sum_{k \neq 0} |\widehat{\mu_t}(k)|^2 |k|^{t-d} < \infty. \]
        %
        Very similar to Minkowski dimension, but `multiscale'.
        \pause

        \item A set has \emph{Fourier Dimension} $s$ if it supports $\mu_t$ with $|\widehat{\mu_t}(k)| \lesssim |k|^{-t/2}$ for \emph{all} $n$.

        \item $\fordim(E) \leq \hausdim(E) \leq \minkdim(E)$.
    \end{itemize}
\end{frame}

\begin{frame}
    \frametitle{Pattern Avoidance}

    \begin{itemize}
        \item If $\dim(E)$ is large, does $E$ `contain patterns'.

        \pause
        \item Basic Example: If $\dim(E)$ is large, are there $m_1,\dots,m_n \in \ZZ$ and distinct $x_1,\dots,x_n \in E$ such that $m_1x_1 + \dots + m_nx_n = 0$? (Can large sets be linearly independent over $\QQ$)

        \pause
        \item (Keleti, 1999) There is $E \subset \TT$ with $\hausdim(E) = 1$ such that for any $m_1,\dots,m_n$ and distinct $x_1,\dots,x_n \in E$, $m_1x_1 + \dots + m_nx_n \neq 0$.

        \pause
        \item If $\fordim(E) > 0$, there is $n$, $m_1,\dots,m_n \in \ZZ$ and distinct $x_1,\dots,x_n \in E$ such that $m_1x_1 + \dots + m_nx_n = 0$.

        \begin{itemize}
            \pause
            \item ($E + \dots + E$ actually contains an interval for some large sum)
            
            \pause
            \item Consider $\mu$ with $\text{supp}(\mu) \subset E$ and $|\widehat{\mu}(k)| \lesssim |k|^{-\varepsilon}$.
            
%            \pause
%            \item Pick $n > 1/\varepsilon$. If $\nu = \mu * \dots * \mu$, $|\widehat{\nu}(k)| = |\widehat{\mu}(k)|^n \lesssim |k|^{-n\varepsilon}$.
            
%            \pause
%            \item Thus $\widehat{\nu} \in L^1(\ZZ)$, so $\nu \in C(\TT)$.
            
%            \pause
%            \item Thus $\text{supp}(\nu) \subset E + \dots + E$ contains an interval.
        \end{itemize}

        \pause
        \item If $\fordim(E) > 2/n$, then there are $m_1,\dots,m_n$ and distinct $x_1,\dots,x_n \in E$ such that $m_1x_1 + \dots + m_nx_n = 0$.
    \end{itemize}
\end{frame}

\begin{frame}
    \frametitle{Independent Sets}

    \begin{itemize}
        \item (Rudin, 1960): There exists $E \subset \TT$ and a finite Borel measure $\mu$ with $\text{supp}(\mu) \subset E$ such that $E$ is independent but $|\widehat{\mu}(k)| \to 0$ as $|k| \to \infty$.

        \pause
        \item (K\"{o}rner, 2007): There exists independent $E$ supporting measures converging to zero as `fast as possible'.

        \pause
        \item (K\"{o}rner, 2009): There exists $E \subset \TT$ with $\fordim(E) = 1/(n-1)$ such that $E$ avoids solutions to all $n$-term linear equations.
    \end{itemize}
\end{frame}


\begin{frame}
    \frametitle{Arithmetic Progressions ($x_1 - 2x_2 + x_3 = 0$)}

    \begin{itemize}
        \pause
        \item (Łaba and Pramanik, 2007): For some small $\varepsilon > 0$, if $|\widehat{\mu}(k)| \leq C_1 |k|^{-(1-\varepsilon)/2}$ and $\mu((x,x+r)) \leq C_2 r^\alpha$ for appropriate $C_1,C_2$, and $\alpha$, $\text{supp}(\mu)$ contains arithmetic progressions.

        \pause
        \item (Schmerkin, 2015): There is $E \subset \TT$ avoiding arithmetic progressions with $\fordim(E) = 1$.

        \pause
        \item (Liang and Pramanik, 2020): Generalized Schmerkin's construction to all translation-invariant patterns.
    \end{itemize}
\end{frame}





\begin{frame}
    \frametitle{Fourier Dimension and Nonlinear Patterns}

    \begin{itemize}
        \pause
        \item (Henriot and Łaba and Pramanik, 2015): For certain linear maps $A_1,\dots,A_n$ and polynomials $Q$, there is $\varepsilon > 0$ such that if $E \subset \TT$ and $\fordim(E) \geq 1 - \varepsilon$, $E$ contains a family of points of the form
        %
        \[ \{ x, x + A_1y, \dots, x + A_{n-1}y, x + A_ny + Q(y) \}. \]
        %
        The pattern $\{ x, x + t, x + t^2 \}$ is \emph{not} covered.

        \pause
        \item (Fraser and Guo and Pramanik, 2019): If $\deg(f) > 1$ and $f(0) = 0$, then patterns of the form $\{ x, x + t, x + f(t) \}$ exist in $\text{supp}(\mu)$ if $\mu$ satisfies explicit estimates ala Łaba and Pramanik.

        \pause
        \item (Kuca, Orponen, Sahlsten, Preprint 2021): If $E \subset \TT^2$ and $\hausdim(E) \geq 2 - \varepsilon$, then $E$ contains solutions to $y_2 - x_2 = (y_1 - x_1)^2$ for distinct $x,y \in E$.
    \end{itemize}
\end{frame}




\begin{frame}
    \frametitle{Sets Avoiding Nonlinear Patterns for Hausdorff Dimension}

    \begin{itemize}
        \pause
        \item Find large $E \subset \TT^d$ such that for distinct $x_1,\dots,x_n \in E$,
        %
        \[ x_n \neq f(x_1,\dots,x_{n-1}). \]
    \end{itemize}

    \pause
    \begin{center}
    \begin{tabular}{| p{4cm} | p{4cm} | p{1.6cm} |}
        \hline
        \textbf{Author} & \textbf{Property of $f$} & $\hausdim(X)$\\
        \hline
        Math\'{e} (2017) & A degree $r$ polynomial & $d/r$\\
        \hline
        Fraser Pramanik (2018) & $f$ is $C^1$ & $m/(n-1)$\\
        \hline
        D. Pramanik Zahl (2020) & $f$ Lipschitz & $m/(n-1)$\\
        \hline
        D. (2020) & $f = g \circ \pi$ where the linear map $\pi: \RR^{n-1} \to \RR^{m-1}$ is is surjective & $1/(m-1)$\\
        \hline
    \end{tabular}
    \end{center}

    \begin{itemize}
        \item Can we modify these constructions to obtain Salem sets?
    \end{itemize}
\end{frame}







\begin{frame}
    \frametitle{Main Result}

    \begin{theorem}
        Suppose $f(x_1,\dots,x_{n-1})$ is $C^{d+1}$, and for each $1 \leq i \leq n-1$,
        %
        \[ D_{x_k} f = \left( \frac{\partial f_i}{\partial x_{kj}} \right) \]
        %
        is invertible. Then there exists $E \subset \TT^d$ with
        %
        \[ \fordim(E) = \frac{d}{n - 3/4} \]
        %
        avoiding solutions to the equation $x_n = f(x_1,\dots,x_{n-1})$.
    \end{theorem}

    \begin{itemize}
        \item (Fraser and Pramanik, 2016) obtains a set $E \subset \RR$ with
        %
        \[ \hausdim(E) = \frac{d}{n - 1}. \]
    \end{itemize}
\end{frame}

\begin{frame}
    \frametitle{Linear Result}

    \begin{theorem}
        Suppose $f$ is Lipschitz. Then there exists $E \subset \TT^d$ with
        %
        \[ \fordim(E) = \frac{d}{n-1} \]
        %
        avoiding solutions to the equation
        %
        \[ x_n - x_{n-1} = f(x_1,\dots,x_{n-2}). \]
    \end{theorem}
\end{frame}

\end{document}