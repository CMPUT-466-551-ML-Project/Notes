% LaTeX template for Summer Schools
%
% !!!!!!!!!!!!   INSTRUCTIONS  !!!!!!!!!!!!!!

% 0) Submit your work in tex format (not pdf). It will be pasted into the proceedings file.
%
% 1) please name your file yourlastname.tex, e.g. thiele.tex.
%
% As several files will be concatenated, please 
% use some discipline as to the following:
%
% 2) Whenever you use \label{} to get automatic cross references
% through \ref{} (this is the preferred option for cross references)
% please add your initials to the label such as 
% \label{SLEct} for the Stochastic Loewner equation
% with initials of author Christoph Thiele
% Same with other citations such as in bibitem.
%
% 3) Please STRONGLY avoid using \def or \newcommand unless really necessary.
% We do have macros for black board bold. If you use your favorite
% macros while preparing your summary, please expand them
% (replace by the original definition) everywhere in your file.
% This will save me the work of doing the very same thing.
% Thank you! If you have to use \def, please also add your
% initials to the definition.
%
% 4) if you want to compile the header of the document, 
% uncomment remove the corresponding paragraph signs below
%
% 5) There is a sample lecture below. For the header 
% it is best to keep most of the
% commands and just change the name, title, text. etc
%
% 6) Please follow the conventions below in terms of capitalization of 
% headings etc:
% Only the beginning of a sentence and names are capitalized.


\documentclass[12pt]{article}
\usepackage{amssymb,amsmath,amsthm}
\usepackage{esint}
\usepackage{mathrsfs}
\usepackage{mathtools}
\usepackage{comment}
\usepackage{bbm,dsfont}

\newtheorem{theorem}{Theorem}
\newtheorem{lemma}[theorem]{Lemma}
\newtheorem{corollary}[theorem]{Corollary}
\newtheorem{conjecture}[theorem]{Conjecture}
\newtheorem{definition}[theorem]{Definition}
\newtheorem{example}[theorem]{Example}
\newtheorem{remark}[theorem]{Remark}
\newtheorem{proposition}[theorem]{Proposition}

\newcommand{\talktitle}[1]{\section{#1}}
\newcommand{\talkafter}[1]{\textbf{After #1} \addcontentsline{toc}{subsection}{after #1}}
\newcommand{\talkspeaker}[2]{\begin{center}
\textit{A summary written by #1}
\end{center}
\addcontentsline{toc}{subsection}{#1, #2}
}

\usepackage{graphicx}
\usepackage[utf8]{inputenc}
\usepackage[T1]{fontenc}
\usepackage[hidelinks]{hyperref}

\newcommand*{\Z}{\mathbb{Z}}
\newcommand*{\Q}{\mathbb{Q}}
\def\C{\mathbb{C}}
\newcommand*{\N}{\mathbb{N}}
\newcommand*{\R}{\mathbb{R}}

% Absolute values and norms using mathtools.
% \\[lr][vV]ert produces correct spacing, as opposed to | and \|.
\DeclarePairedDelimiter\abs{\lvert}{\rvert}
\DeclarePairedDelimiter\norm{\lVert}{\rVert}

\title{Nodal Domains and Landscape Functionals}

% \author{Summer School, Kopp}
% \thanks{supported by Hausdorff Center for Mathematics, Bonn}

% \date{September 2021}



\begin{document}

% \maketitle

% {
% \center{Organizers:}
% \center{
% Christoph Thiele, Universit\"at Bonn}
% \center{
% Pavel Zorin-Kranich, Universit\"at Bonn}
% \center{$\ $}
% }
% \newpage
% \tableofcontents

% \newpage


\talktitle{Nodal Domains via Diffusion Processes}
\talkafter{Georgiev and Muckherjee \cite{georgiev}}
\talkspeaker{Jacob Denson}{University of Madison, Wisconsin}

\setcounter{equation}{0}
\setcounter{theorem}{0}

\begin{abstract}
 {We describe the theory of It\^{o} diffusions on compact manifolds, and it's application by Georgiev and Muckherjee to the study of the geometry of nodal sets to the Laplace-Beltrami operator on manifolds.}
\end{abstract}

Let $M^d$ be a compact Riemannian manifold, and let $e_\lambda \in C^\infty(M)$ be an eigenfunction for the Laplacian, such that $\Delta e_\lambda = -\lambda^2 e_\lambda$. Our goal is to use the theory of stochastic diffusions to study asymptotic properties of \emph{nodal domains} $D_\lambda$, open connected components of $\{ x \in M : e_\lambda(x) \neq 0 \}$, as $\lambda \to \infty$. In particular, we show it is impossible to fit $D_\lambda$ in a $O(\lambda^{-1})$ tubular neighborhood of a `flat' embedded surface $\Sigma_\lambda \subset M$. Steinerberger \cite{steinerberger} gave the first version of this result using diffusion. Here we discuss a more robust result, using similar techniques, due to Georgiev and Muckherjee \cite{georgiev}.

\begin{theorem}
    Let $\Sigma_\lambda$ be a smooth surface in $M$ of dimension $k$ such that for any $x \in M$ with $d(x,\Sigma_\lambda) \leq \lambda^{-1}$, there exists a unique point on $\Sigma_\lambda$ closest to $x$. Then there is $c > 0$ such that no nodal domain $D_\lambda$ is contained in a $c \cdot \lambda^{-1}$ neighborhood of $\Sigma_\lambda$ for all $\lambda > 0$.
\end{theorem}

Why should we expect the theory of diffusions to give us information about nodal sets? A major reason is that eigenfunctions to the Laplace-Beltrami operator behave well under the heat equation, i.e. if $e^{t \Delta}$ are the propogators for the heat equation $\partial_t = \Delta$ on $M$, then $(e^{t \Delta} e_\lambda)(x) = e^{-\lambda^2 t} e_\lambda(x)$. The heat equation mathematically describes the distribution of particles diffusing through a medium in which they are subject to random molecular bombardments. The theory of \emph{diffusions} in probability gives an alternate mathematical model of this situation, so it is reasonable that applying the theory will bring light upon the theory of eigenfunctions. In particular, we will see that it gives us a theory of \emph{exit times}, that give us a way to study the rate of propogation of a diffusion process.
%In particular, to study $N_\delta$, we will take a \emph{nodal domain} $D$ of $N_\lambda$ (a connected component of $M - N_\lambda$), and study the diffusion

\section{Probabilistic Tools}

Let us begin by introducing the probabilistic machinery required to describe It\^{o} diffusions. We work over a fixed probability space $S$, and study a \emph{continuous stochastic process} on $S$, valued in a space $M$. There are three useful ways to think of such a process. The first is as a Borel-measurable function $X$ from $S$ to $C([0,\infty), M)$, intuitively, a random continuous path on $M$. The second is as a family of variables $\{ X_t : t \in [0,\infty) \}$. The third is as a \emph{law} which describes how random variables in the `future' depend on random variables in the past, which leads to the study of the conditional operators $\mathbb{E}^x[f(X)]$, defined for $x \in M$ and a random statistic $f(X)$ associated with the process $X$, which give the average value of the statistic given that we start the process at the state $x$, i.e. we let $X_0 = x$, and then let the process evolve according to the law defining the process.

The most basic It\^{o} diffusion is \emph{Brownian motion}. A one-dimensional Brownian
%which (via central limit theorem type arguments) is essentially the unique continuous stochastic process $B$, whose increments $B_t - B_s$ are independent and time homogeneous. More precisely, a one-dimensional 
motion $B$ is a continuous stochastic process on $\R$ such that for any interval $I = [t,s]$, the increments $\Delta_I B = B_s - B_t$ are mean zero, variance $s - t$ Gaussian random variables, and for any almost disjoint family of intervals $\{ I_1,\dots, I_n \}$ in $[0,\infty)$, the random variables $\{ \Delta_{I_1} B, \dots, \Delta_{I_n} B \}$ are independent. A Brownian motion on $\R^d$ is precisely a continuous process whose coordinates are independent one-dimensional Brownian motions.

By tweaking Brownian motion locally, we end up with a more general \emph{It\^{o} diffusion}. Suppose that for each $x \in \R^d$, we are given a $d \times d$ positive semidefinite symmetric matrix $A(x)$. Then we obtain a continuous process $X$ defined by the law given by the \emph{stochastic differential equation}
%
\[ dX = A(X) dB. \]
%
The formal definition of this differential equation is quite technical, but for our purposes, the equation means that there a Brownian motion $B$ such that
%
\[ X_{t + \delta} = X_t + A(X_t) \cdot [B_{t + \delta} - B_t] + o(\delta), \]
%
where the $o(\delta)$ term is a random variable with mean $o(\delta)$, and with $L^3$ norm $O(\delta)$.
%
% Find f such that Df = (A o f)^T.
% then X_t = f(B_t)
%\[ dX^i = \sum A(X)_{ij} dB^j \]
%where the remainder term $R_{t,\delta}$ becomes more and more neglibile as $\delta \to 0$, namely
%
%\[ \EE[(X_{t + \delta} - X_t - A(X_t) (B_{t + \delta} - B_t))] = \int_t^s A(X_t)|A(X_t)|^2 |B_{t + \delta} - B_t|^2 \EE(X_{t + \delta} - X_t)^2 \]
%\[ \nabla_x f(t,X_t) G_t = A(X_t) \]
As one might expect, one can analogously define It\^{o} diffusions on a compact manifold $M$ given a section $A$ of $\text{Hom}(TM)$, which will satisfy analogous formulae. Thus the diffusion acts like Brownian motion, except that instead of acting radially, it spreads out unevenly from a point $x$ in the directions dictated by the extent of the matrix $A(x)$.

%An important fact about the diffusions above is that they are \emph{Markov}, i.e. if one observes the values of the Brownian motion $\{ X_t \}$, up to a certain time $T$, the only information one observes which tells us about the evolution of the process past the time $T$ is the position the diffusion arrives to at time $T$, i.e. the random variable $X_T$. More formally, a probabilist would write this by stating that for $S > T$, and for a function $f$ on $M$,
%
%\[ \mathbb{E}[ f(X_S) | \Sigma_T ] = \mathbb{E}[ f(X_S) | X_T ], \]
%
%where the conditional expectation is taken with respect to a $\sigma$ algebra $\Sigma_T$, which models the information one sees when observing the process up to time $T$. One can also let the times $T$ and $S$ be random \emph{stopping times}, so a diffusion satisfies the \emph{strong Markov property}.

Now we connect diffusions to semielliptic differential operators. For any diffusion $X$, we can associate such an operator $L$, known as the \emph{generator} of the diffusion, and this is a one to one correspondence. As an example, Brownian motion on $\R^d$ has $\Delta / 2$ as it's generator. This motivates us to \emph{define} Brownian motion on a manifold $M$ as a process generated by $\Delta/2$, i.e. half the Laplace-Beltrami operator. This correspondence becomes useful in several scenarios. First, for any $f \in C^2(M)$, we have
%
\[ \lim_{t \to 0} \frac{\mathbb{E}^x[f(X_t)] - f(x)}{t} = (Lf)(x). \]
%
To see how $L$ emerges, use the approximation $X_t = x + A(x) B_t + o(t)$ and expand $f(X_t)$ in a Taylor series about $x$. On a sample by sample basis, the first order terms in the expansion of $f(X_t)$ will behave badly, on the order of $O(t^{1/2})$, since Brownian motion is non-differentiable. But thankfully these terms vanish in the expectation since $B$ is highly oscillatory. The second order terms will involve squares of Brownian motion, and the fact that $\mathbb{E}^0[B_t^2] = t$ implies that these terms will have expected value $O(t)$. Higher order terms will be $O(t^{3/2})$, and thus not affect the value of the limit as $t \to 0$.
%The connection between diffusions and partial differential equations comes from the following formulas. There is a one to one correspondence between diffusion processes $dX = A dB$ and homogeneous semielliptic partial differential operators $L$, and we call the operator the \emph{generator} of the diffusion. For instance, Brownian motion on $\R^d$ has $\Delta / 2$ as it's generator, which motivates us to \emph{define} Brownian motion on a compact Riemannian manifold to be the diffusion process with generator $\Delta / 2$. The connection between diffusions and generators is realized by the fact that for any $f \in C^2(M)$,
%
%First, for any $f \in C^2(M)$, we define
%
%\[ Lf(x) = \frac{A(x)^2 \cdot Hf(x)}{2} = \frac{\text{Tr}(A(x) \cdot G(x) \cdot Hf(x) \cdot G(x)^{-1} \cdot A(x))}{2}, \]
%
%to be the \emph{generator} of the diffusion, where $G = \{ g_{ij} \}$.
%Brownian motion on $\R^d$ has $\Delta / 2$ as it's generator, so we \emph{define} Brownian motion on a Riemannian manifold $M$ as the diffusion process with generator $\Delta / 2$.
\emph{Dynkin's formula} follows formally from this calculation by time-homogeneity and the fundamental theorem of calculus, implying that for any time $T$ (possibly a \emph{random time}, provided it is an `integral stopping time'),
%
\[ \mathbb{E}^x[f(X_T)] = f(x) + \mathbb{E}^x \left[ \int_0^T (Lf)(X_s)\; ds \right]. \]
%
To test Dynkin's formula, if $B$ is a Brownian motion on $\R^n$, $T$ is the first time that $B$ exits an open ball of radius $R$ about the origin (an \emph{exit time}, which will always be an integrable stopping time for any bounded open set), and if $f(y) = |y|^2$, then $f(X_T) = R^2$ and $(\Delta/2) f = n$, so
%
\[ R^2 = \mathbb{E}^x[f(X_T)] = f(x) + \mathbb{E}^x \left[ \int_0^T n\; ds \right] = n \cdot \mathbb{E}^x[T]. \]
%
Thus we see that Brownian motion diffuses a distance $O(R)$ in $R^2$ units of time. It is interesting to note that if $R$ is sufficiently small, one can perform a similar calculation for Brownian motion on a Riemannian manifold with the exit time of a geodesic ball. If $x$ is fixed, and $f(y) = d(x,y)^2$ is the geodesic distance to $x$, then one can show (see Section 2.4 of \cite{rosenberg}) that $(\Delta/2) f(x) - n$ is proportional to the \emph{rate of volume expansion} in the manifold at the point $x$.
%\[ \Delta f(x) = n + \frac{\partial_r \det(D \exp_{x_0} )(x)}{\det(D \exp_{x_0}(x)} r. \]
%Plugging this calculation into Dynkin's formula shows that the time to leave a ball of radius $R$ depends on the \emph{rate of volume expansion} in the manifold $M$.
Thus we expect Brownian motion to diffuse more slowly in positively curved regions (with a negative rate of expansion), and faster in regions of negative curvature (with positive rate of expansion).

The correspondence also applies in the reverse manner, giving the \emph{Feynman-Kac formula} and it's variants; if $f$ is a function on $M$, and we define $u(x,t) = \mathbb{E}^x[f(X_t)]$, then $u$ is the solution to the partial differential equation $\partial_t u = Lu$ on $M$, with initial conditions $f$.
%Locally, Brownian motion diffuses in a manner resembling Euclidean Brownian motion, diffusing slower radially near regions of positive curvature, and faster radially in regions of negative curvature (volume expansion formulas tell us that for negatively curved spaces there is exponentially more room for Brownian motion to travel away from it's starting position, and vice versa for positively curved spaces).
We can also consider boundary value problems. If $D$ is a bounded region of $\R^d$, and $\tau_D = \inf \{ t > 0 : X_t \not \in D \}$ is the \emph{exit time} for $D$, then the Dynkin formula tells us that the unique solution to the Dirichlet problem $Lv = -h$ on $D$ with boundary conditions $\phi$ on $\partial D$ is given by
%
\[ v(x) = \mathbb{E}^x[\phi(X_{\tau_D})] + \mathbb{E}^x \left[ \int_0^{\tau_D} h(X_t)\; dt \right]. \]
%
One can also solve the heat equation $\partial_t u = Lu$ with absorbing boundary conditions $u(x,t) = 0$ for $x \in \partial D$, and initial condition $u(x,0) = f(x)$ by setting $u(x,t) = \mathbb{E}^x[f(X_t) \chi_t ]$, where $\chi_t = 1$ if $t < \tau_D$, and $\chi_t = 0$ otherwise (we kill paths that reach the boundary and are `absorbed'). %Using linearity, gives a solution to  we can now solve the problem $\partial_t u - Lu = h$ with a general boundary condition $u(x,t) = \phi(x)$ on $\partial D$ and with initial conditions $u(x,0) = f(x)$, namely, we have
%
%\[ u(x,t) = \mathbb{E}^x[\phi(X_{\tau_D})] + \mathbb{E}^x \left[ \int_0^{\tau_D} h(X_t)\; dt \right] + \mathbb{E}^x[f(X_t) \chi_t]. \]
%
%Note that if $D$ is bounded, and $f \in L^\infty(D)$, then dominated convergence implies that for each $x \in D$, $u(x,t) \to v(x)$ as $t \to \infty$, i.e. so the solution to Dirichlet's problem is really the long term distribution of the process.
There is also a way to consider solutions to the heat equation with an insulating boundary, i.e. finding a solution to $\partial_t u = \Delta u$ such that $\partial u / \partial \eta = 0$ on $\partial D$ by considering a \emph{reflecting Brownian motion} which `bounces off the boundary' instead of being killed.

\begin{comment}
As an example, consider the equation $\Delta u = \delta_x$ defined on a bounded open subset $\mathcal{S}$ of $\R^d$, with $x \in \Omega$, and subject to the condition that $u$ vanishes on the boundary of $\Omega$. One can view this as the steady state distribution of heat obtained by placing a candle underneath the surface at the point $x$, while fixing the temperature of the boundary. In this situation heat is constantly emitted from the point $x$ at all times, allowed to travel around $\Omega$ at random, but is eradicated if it ever touches the boundary $\partial \Omega$ by virtue of the boundary condition. The probabilistic model of this equation is to consider a Brownian motion $\{ B^{x,t_0} \}$ starting at a point $x \in \Omega$ and released at time $t_0$. The density function of these particles at time $T$, viewed as a random measure $\mu^T$ on $\Omega$, is given by
%
\[ \int \phi(x) d\mu^T(x) = \int_0^T \phi(B^{x,t}_T) \chi^{x,t}_T\; dt, \]
%
where $\chi^{x,t_0}_t$ is the indicator function, equal to one if $B^{x,t_0}$ has remained in $\Omega$ up to time $T$, and equal to zero if $B^{x,t_0}$ has ever exited $\Omega$. We then expect that the solution $u$ above is given by
%
\[ \int \phi(x) u(x)\; dx = \lim_{T \to \infty} \EE \left[ \int_0^T \phi(B^{x,t}_T) \chi^{x,t}_T\; dt \right] = \lim_{T \to \infty} \int_0^T \EE[ \phi(B^x_t) \chi^x_t ]\; dt, \]
%
and indeed, this gives a solution to the problem.

We have therefore found a Green's function for the Dirichlet problem. Namely, if we define a distribution $g$ on $\Omega \times \Omega$ such that
%
\[ \int \phi(y) g(x,y)\; dy = \lim_{T \to \infty} \int_0^T \EE[ \phi(B^x_t) \chi^x_t ]\; dt, \]
%
then for a harmonic function $u$ on $\Omega$, we should have
%
\[ u(x) = \int_{\partial \Omega} \frac{\partial g}{\partial \eta} u(y)\; dy. \]
%
This leads to the \emph{harmonic measure}, i.e. we have
%
\[ \frac{\partial g}{\partial \eta} = d\mu^x, \]
%
where
%
\[ \int_{\partial \Omega} \phi(y) d\mu^x(y) = \EE \left[ \phi( V^x ) \right], \]
%
where $V^x$ is a random variable valued on the boundary, equal to $B^x_T$, where $T$ is the \emph{hitting time} of $\partial \Omega$, i.e. the first time $t$ that $B^x_t$ touches $\partial \Omega$. And it makes sense that the function $u(x) = \EE[\phi(V^x)]$ solves the Dirichlet problem on $\Omega$; if we consider a small ball $B \subset \Omega$ of radius $r$ around a point $x$, then Brownian motion starting at $x$ is equally likely to exit $B$ at each point uniformly on the boundary, and so by the Markov property, we find that
%
\[ \EE[\phi(V^x)] = \fint_{|y - x| = r} \EE[\phi(V^y)]\; dy, \]
%
which is equivalent to being harmonic if $u$ is sufficiently regular.

The \emph{Feynman-Kac formula} extends this kind of analysis from the Dirichlet problem to the study of more general solutions to the heat equation $\partial_t u = \Delta u$.

 is a version of this result

For our purposes, we are interested in the \emph{Feynman-Kac} formula. For an open set $\Omega \subset \RR^d$, $f \in L^2(\Omega)$, $x \in \Omega$, and $t > 0$, we find that
%
\[ e^{t \Delta} f(x) = \frac{\EE[ f(B^x_t) | H_\Omega > t ]}{\PP(H^x_\Omega > t)}, \]
%
where
%
\begin{itemize}
    \item $B^x$ is a standard Brownian motion beginning at $x$.
    \item $H^x_\Omega$ is the time that $B^x$ hits $\partial \Omega$.
\end{itemize}
%
Thus, in particular,
%
\[ e^{-\lambda^2 t} e_\lambda(x) = \frac{\EE[ e_\lambda(B^x_t) | H^x_\Omega ]}{\PP(H^\Omega_x > t)} \]
%
One can again understand $H_{\Omega,x}$ by `reversing time' in a certain sense. The probability that Brownian motion will have left $\Omega$ at a certain time is equal to the probability that a Brownian motion starting uniformly on the boundary will reach $x$. Thus
%
\[ u(x,t) = \PP(H_{\Omega,x} > t) \]
%
is a solution to the heat equation, such that $u(x,t) = 1$ on $\partial \Omega$, and starts equal to zero on $\Omega$.
\end{comment}

\section{Nodal Sets Via Brownian Motion}

Let us use the theory we have introduced to prove Theorem 1. Let $e_\lambda$ be an eigenfunction, and $D_\lambda$ a nodal domain of $e_\lambda$. We may assume without loss of generality that $e_\lambda$ is positive on $D_\lambda$. Consider two solutions $p(x,t)$ and $u(x,t)$ to the heat equation on $D_\lambda$, with initial conditions $p(x,0) = 0$, $u(x,0) = e_\lambda(x)$, and with boundary conditions $p(x,t) = 1$ and $u(x,t) = 0$ for $x \in \partial D_\lambda$. The Feynman-Kac formula tells us that
%
\[ p(x,t) = 1 - \mathbb{E}^x[\chi_t] = \mathbb{P}^x[t > \tau_\lambda] \quad\text{and}\quad u(x,t) = \mathbb{E}^x[ e_\lambda(B_t) \chi_t ]. \]
%
where $\chi_t = \mathbb{I}(t \leq \tau_\lambda)$, and $\tau_\lambda = \inf \{ t > 0 : B_t \in D_\lambda^c \}$ is the exit time of $D_\lambda$. If $x_0 \in D_\lambda$ maximizes $e_\lambda$ on $D_\lambda$, then
%
\[ e^{- \lambda^2 t} e_\lambda(x) = u(x,t) = \mathbb{E}^x[ e_\lambda(B_t) \chi_t ] \leq e_\lambda(x_0) \mathbb{E}^x[\chi_t] = e_\lambda(x_0) (1 - p(x,t)). \]
%
In particular, $p(x_0,t) \leq 1 - e^{-\lambda^2 t}$, so $\mathbb{P}^{x_0}[\tau_\lambda \geq t \lambda^{-2}] \geq e^{-t}$. Thus $\mathbb{E}[\tau_\lambda] \gtrsim \lambda^{-2}$, and the heuristic diffusion rate of Brownian motion therefore leads us to believe that $x_0$ lies roughly $\gtrsim \lambda^{-1}$ from $\partial D_\lambda$. If we had $M = \R^d$, this would immediately yield a contradiction if $D_\lambda$ was contained in a $c \cdot \lambda^{-1}$ neighborhood of a $k$ dimensional plane $\Sigma$, because we have a strong quantification of this heuristic; if $B'$ is the projection of Brownian motion onto the $d-k$ dimensional plane normal to $\Sigma$, then a result due to Kent \cite{kent} implies that for any $\varepsilon > 0$, there is $c > 0$ such that
%
\[ \mathbb{P}^0 \left( \sup_{0 \leq t \leq \lambda^{-2}} |B_t'| \leq c \cdot \lambda^{-1} \right) \leq \varepsilon. \]
%
Letting $c$ corresponding to $\varepsilon = (1/2) e^{-1}$ gives a contradiction, since then
%
\[ e^{-1} \leq \mathbb{P}^{x_0}[\tau_\lambda \geq \lambda^{-2}] \leq \mathbb{P}^0 \left( \sup_{0 \leq t \leq \lambda^{-2}} |B_t'| \leq c \cdot \lambda^{-1} \right) \leq (1/2) e^{-1}. \]
%
Extending this to the non-Euclidean setting is not too difficult. Given a general $k$ dimensional hypersurface $\Sigma$ on a manifold $M$ satisfying the assumptions of the result we are trying to prove, we suppose that $D_\lambda$ is contained in a $c \cdot \lambda^{-1}$ neighborhood $U_\lambda$ of $\Sigma_\lambda$. The flatness assumptions imply that we can find a normal coordinate system $\phi$ on a $2c \cdot \lambda^{-1}$ neighborhood of $\Sigma_\lambda$ using geodesics. Then $\phi(\Sigma_\lambda)$ is a $k$ dimensional plane, and $\phi(U_\lambda)$ is a $c \cdot \lambda^{-1}$ neighborhood of this plane. Because the Euclidean and Riemannian metrics are comparable, Brownian motion on $M$ should behave in coordinates analogously to Brownian motion on $\R^d$. The rate of diffusion of both processes leads us to believe the behaviour should be similar up to times $c \cdot \lambda^{-2}$, before which both Brownian motions are highly unlikely to leave the neighborhoods $U_\lambda$ and $\phi(U_\lambda)$. Thus our proof can be completed by an application of the following `comparison result' for hitting times, which is Theorem 2.2 of \cite{georgiev}.

%Now suppose the result we are proving is false. Then $D_\lambda$ is contained in a $c \lambda^{-1}$ neighborhood $U_\lambda$ of $\Sigma_\lambda$. Let $2 U_\lambda$ denote a $2c \lambda^{-1}$ neighborhood of $\Sigma_\lambda$. By the admissability assumption, we may switch to a geodesic coordinate system $\phi: 2U_\lambda \to \R^d$, normal to $\Sigma_\lambda$, i.e. such that
%
%\[ \phi(\Sigma_\lambda) = \phi(2U_\lambda) \cap \left\{ \R^k \times \{ 0 \}^{d-k} \right\} \quad\text{and}\quad \phi(U_\lambda) = \phi(2U_\lambda) \cap \left\{ \R^k \times \mathbb{B}_\lambda \right\} \]
%
%where $\mathbb{B}_\lambda$ is the closed ball of radius $c \lambda^{-1}$ in $\R^{d-k}$, and such that the metric in the coordinates $\phi$ is comparable to the Euclidean metric. If we let $\tau'_\lambda$ be the exit time of $U_\lambda$, then $\tau'_\lambda \geq \tau_\lambda$, and so we have
%
%\[ \mathbb{P}( \tau_\lambda \leq t ) \geq \mathbb{P}( \tau'_\lambda \leq t ). \]
%
%For small times, Brownian motion on a manifold should behave like Euclidean Brownian motion in geodesic coordinates. We can calculate Euclidean stopping times much more explicitly which we can exploit to give us more information of the set we're considering. Since the metric is comparable to the Euclidean metric on this $\lambda^{-1}$ ball, we should expect Brownian motion in the Euclidean and Riemannian setting to be roughly similar up to a time $\lambda^{-2}$, before which both Brownian motions are unlikely to exit this ball.

\begin{theorem}
    Let $M^d$ be a compact Riemannian manifold, and consider an open geodesic ball $B \subset M$ around a point $x_0$ with radius $r$ smaller than the injectivity radius of $M$. Let $(U,\phi)$ be a chart on $M$ with $B \subset U$, and suppose that the metric of $M$ is comparable to the Euclidean metric in the coordinates $\phi$. Fix a compact set $K \subset B$. Let $B^1$ be a Brownian motion on $M$, and $B^2$ be a Brownian motion on $\R^d$. If $\tau_1$ denotes the time that $B^1$ exits $K$, and $\tau_2$ denotes the time that $B^2$ exits $\phi(K)$, then for any $c > 0$, there exists $C > 0$ such that
    %
    \[ (1/C) \cdot \mathbb{P}( \tau_2 \leq c r^2 ) \leq \mathbb{P}( \tau_1 \leq c r^2 ) \leq C \cdot \mathbb{P}( \tau_2 \leq c r^2 ). \]
\end{theorem}


\begin{comment}

Given that $e^{t \Delta} e_\lambda = e^{-\lambda^2 t} e_\lambda$, i.e. we see that the mass of $e_\lambda$ decays exponentially. To gain information on $e_\lambda$, it therefore makes sense to consider the heat equation $\partial_t u = \Delta u$ on $D$ with initial conditions $e_\lambda$, and with Neumann boundary conditions $\partial u / \partial \eta = 0$, which will conserve the mass of $e_\lambda$.

 this relation to control nodal sets, we let $\Omega$ be a nodal domain of $e_\lambda$, i.e. a connected component of $M$. The Feynman-Kac formula tells us that solutions to the equation $\partial_t = \Delta$ on an open set $\Omega \subset M$, which are held equal to one on $\partial \Omega$ and vanish initially on the interior can be understood in terms of the probability that Brownian motion will hit $\partial \Omega$ in a given number of units of time.



We will get information about the nodal set by constructing diffusion processes which behave different near the boundary via the Feynman-Kac formula,

TODO: WHY IS NEUMANN VS DIRICHLET CONNECTED TO THE NODAL SET OF AN EIGENFUNCTION? NEUMANN CONDITIONS DON'T WORK, BECAUSE THE ASSOCIATED BROWNIAN MOTION MUST BE REFLECTED ABOUT THE SINGULAR SET OF $e_\lambda$, WHICH CAUSES PROBLEMS.

Instead of imposing boundary conditions, we instead construct a stochastic process which looks like a diffusion for small times, but converges back to initial data at large times. This leads to a bound of the form
%
\[ H^{d-1}(N_\lambda) \gtrsim_M \lambda^{- \frac{n-3}{4}} \]
%
A related problem is to show that nodal domains can be `concentrated' near flat surfaces. Call a smooth, embedded $d-1$ dimensional hypersurface $\Sigma$ in $M$ \emph{admissable up to distance $r$} if the geodesic flow outward from $\Sigma$ is injective up to a distance $r$, i.e. any point in the $r$-thickening of $\Sigma$ has a unique closest point on $\Sigma$. Roughly speaking, this implies that $\Sigma$ is `locally flat at a scale $r$'.

\begin{theorem}
    For any manifold $M$, there exists a constant $c > 0$ such that if $\Sigma$ is admissable up to a distance $\lambda^{-1}$, then no nodal domain $N_\lambda$ can be a contained in a $c \lambda^{-1}$ neighborhood of $\Sigma$.
\end{theorem}

The proof will show one can generalize this result to $\Sigma$ formed from a finite union of embedded manifolds admissable up to a distance $\lambda^{-1}$, provided these manifolds are sufficiently transversal to one another. This leads to a new proof of a class result due to Hayman.

\begin{corollary}
    There exists $c \geq 1/900$ such that if $\Omega \subset \R^2$ is simply connected with \emph{inradius} $\rho$ ($\rho$ is the largest radius of a ball contained in $\Omega$), $\lambda_1(\Omega) \geq c/\rho^2$.
\end{corollary}

TODO: DO WE EXPECT DENSITY OF THE NODAL SET FOR LARGE $\lambda$, i.e. $\lambda^{-1}$ density.

\end{comment}

\begin{thebibliography}{3}

\bibitem{chung}
    Kai Lai Chung,
    \emph{Green, Brown, \& Probability and Brownian Motion on the Line},
    World Scientific Publishing Company. 2002.

\bibitem{georgiev}
    Bogdan Georgiev,
    \emph{Nodal Geometry, Heat Diffusion and Brownian Motion},
    Anal. PDE. {\bf 12} (2017), 133-148.

\bibitem{kent}
    John Kent,
    \emph{Eigenfunction Expansion for Diffusion Hitting Times},
    Probab. Theory Related Fields. {\bf 52} (1980), 309-319.

\bibitem{oksendal}
    Bernt {\O}ksendal,
    \emph{Stochastic Differential Equations},
    Springer, 2003.

\bibitem{rosenberg}
    Steven Rosenberg,
    \emph{The Laplacian on a Riemannian Manifold},
    Cambridge Univesity Press, 1997.

\bibitem{steinerberger}
    Stefan Steinerberger,
    \emph{Lower Bounds on Nodal Sets of Eigenfunctions via the Heat Flow},
    Comm. Partial Differential Equations. {\bf 39} (2014), 2240-2261.

\end{thebibliography}

\noindent \textsc{Jacob Denson, UW Madison}\\
\textit{email:} \texttt{jcdenson@wisc.edu}

\end{document}