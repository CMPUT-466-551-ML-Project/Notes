\documentclass[12pt, dvipsnames]{report}

\usepackage{amsmath}
\usepackage{algorithm}
%\usepackage{algorithmic}
\usepackage[noend]{algpseudocode}

\usepackage{amsmath}
\usepackage{amssymb}
\usepackage{amsthm}
\usepackage{amsopn}

\usepackage{kpfonts}

\usepackage{graphicx}

% Probably don't need this on notes anymore
\usepackage{kbordermatrix}

% Standard tool for drawing diagrams.
\usepackage{tikz}
\usepackage{tkz-berge}
\usepackage{tikz-cd}
\usepackage{tkz-graph}

%
\usepackage{multicol}

%
\usepackage{framed}

%
\usepackage{mathtools}

%
\usepackage{float}

%
\usepackage{subfig}

%
\usepackage{wrapfig}

%
\usepackage{mathabx}

% Used for generating `enlightening quotes'
\usepackage{epigraph}

% Forget what this is used for :P
\usepackage[utf8]{inputenc}

% Used for generating quotes.
\usepackage{csquotes}

% Allows what to generate links inside
% generated pdf files
\usepackage{hyperref}

% Allows one to customize theorem
% environments in mathematical proofs.
\usepackage{thmtools}

% Gives access to a proof
\usepackage{lplfitch}

% I forget what this is for.
\usepackage{accents}

% A package for drawing simple trees,
% as a substitute for unnesacary TIKZ code
\usepackage{qtree}

% Enables sequent calculus proofs
\usepackage{ebproof}

% For braket notation
\usepackage{braket}

% To change line spacing when using mathematical notations which require some height!
\usepackage{setspace}

%\usepackage[dvipsnames]{xcolor}

\usepackage{float}



\setlength\epigraphwidth{8cm}

\usetikzlibrary{arrows, petri, topaths, decorations.markings}

% So you can do calculations in coordinate specifications
\usetikzlibrary{calc}
\usetikzlibrary{angles}

\theoremstyle{plain}
\newtheorem{theorem}{Theorem}[chapter]
\newtheorem{axiom}{Axiom}
\newtheorem{lemma}[theorem]{Lemma}
\newtheorem{corollary}[theorem]{Corollary}
\newtheorem{prop}[theorem]{Proposition}
\newtheorem{exercise}{Exercise}[chapter]
\newtheorem{fact}{Fact}[chapter]

\newtheorem*{example}{Example}
\newtheorem*{proof*}{Proof}

\theoremstyle{remark}
\newtheorem*{exposition}{Exposition}

\theoremstyle{definition}
\newtheorem*{defi}{Definition}
\newenvironment{definition}
    {\begin{samepage}\begin{framed}\begin{defi}}
    {\end{defi}\end{framed}\end{samepage}}





\usepackage{hyperref} 
\hypersetup{
    colorlinks = true,
    linkcolor = black,
}

\makeatletter
\renewcommand*\env@matrix[1][*\c@MaxMatrixCols c]{%
  \hskip -\arraycolsep
  \let\@ifnextchar\new@ifnextchar
  \array{#1}}
\makeatother

\renewcommand*\contentsname{\hfill Table Of Contents \hfill}

\newcommand{\optionalsection}[1]{\section[* #1]{(Important) #1}}
\newcommand{\deriv}[3]{\left. \frac{\partial #1}{\partial #2} \right|_{#3}} % partial derivative involving numerator and denominator.
\newcommand{\lcm}{\operatorname{lcm}} % Lowest Common Multiple.
\newcommand{\im}{\operatorname{im}} % Image of a function.
\newcommand{\bint}{\mathbf{Z}} % Bold integer Symbol.
\newcommand{\gen}[1]{\langle #1 \rangle} % Generator of a group.
\newcommand{\End}{\operatorname{End}}
\newcommand{\Mor}{\operatorname{Mor}}
\newcommand{\Id}{\operatorname{id}}
\newcommand{\visspace}{\text{\textvisiblespace}}
\newcommand{\Gal}{\text{Gal}}
\newcommand{\coker}{\text{coker}}
\newcommand{\xor}{\oplus}