\documentclass[12pt, dvipsnames]{report}

\usepackage{amsmath}
\usepackage{algorithm}
%\usepackage{algorithmic}
\usepackage[noend]{algpseudocode}

\usepackage{amsmath}
\usepackage{amssymb}
\usepackage{amsthm}
\usepackage{amsopn}

\usepackage{kpfonts}

\usepackage{graphicx}

% Probably don't need this on notes anymore
%\usepackage{kbordermatrix}

% Standard tool for drawing diagrams.
\usepackage{tikz}
\usepackage{tkz-berge}
\usepackage{tikz-cd}
\usepackage{tkz-graph}

\usepackage{comment}

%
\usepackage{multicol}

%
\usepackage{framed}

%
\usepackage{mathtools}

%
\usepackage{float}

%
\usepackage{subfig}

%
\usepackage{wrapfig}

%
\let\savewideparen\wideparen
\let\wideparen\relax
\usepackage{mathabx}
\let\wideparen\savewideparen

% Used for generating `enlightening quotes'
\usepackage{epigraph}

% Forget what this is used for :P
\usepackage[utf8]{inputenc}

% Used for generating quotes.
\usepackage{csquotes}

% Allows what to generate links inside
% generated pdf files
\usepackage{hyperref}

% Allows one to customize theorem
% environments in mathematical proofs.
\usepackage{thmtools}

% Gives access to a proof
\usepackage{lplfitch}

% I forget what this is for.
\usepackage{accents}

% A package for drawing simple trees,
% as a substitute for unnesacary TIKZ code
\usepackage{qtree}

% Enables sequent calculus proofs
\usepackage{ebproof}

% For braket notation
\usepackage{braket}

% To change line spacing when using mathematical notations which require some height!
\usepackage{setspace}

%\usepackage[dvipsnames]{xcolor}

\usepackage{float}

% For block commenting
\usepackage{comment}




\setlength\epigraphwidth{8cm}

\usetikzlibrary{arrows, petri, topaths, decorations.markings}

% So you can do calculations in coordinate specifications
\usetikzlibrary{calc}
\usetikzlibrary{angles}

\theoremstyle{plain}
\newtheorem{theorem}{Theorem}[chapter]
\newtheorem{axiom}{Axiom}
\newtheorem{lemma}[theorem]{Lemma}
\newtheorem{corollary}[theorem]{Corollary}
\newtheorem{prop}[theorem]{Proposition}
\newtheorem{exercise}{Exercise}[chapter]
\newtheorem{fact}{Fact}[chapter]

\newtheorem*{example}{Example}
\newtheorem*{proof*}{Proof}

\theoremstyle{remark}
\newtheorem*{exposition}{Exposition}
\newtheorem*{remark}{Remark}
\newtheorem*{remarks}{Remarks}

\theoremstyle{definition}
\newtheorem*{defi}{Definition}

\usepackage{hyperref}
\hypersetup{
    colorlinks = true,
    linkcolor = black,
}

\usepackage{textgreek}

\makeatletter
\renewcommand*\env@matrix[1][*\c@MaxMatrixCols c]{%
  \hskip -\arraycolsep
  \let\@ifnextchar\new@ifnextchar
  \array{#1}}
\makeatother

\renewcommand*\contentsname{\hfill Table Of Contents \hfill}

\newcommand{\optionalsection}[1]{\section[* #1]{(Important) #1}}
\newcommand{\deriv}[3]{\left. \frac{\partial #1}{\partial #2} \right|_{#3}} % partial derivative involving numerator and denominator.
\newcommand{\lcm}{\operatorname{lcm}}
\newcommand{\im}{\operatorname{im}}
\newcommand{\bint}{\mathbf{Z}}
\newcommand{\gen}[1]{\langle #1 \rangle}

\newcommand{\End}{\operatorname{End}}
\newcommand{\Mor}{\operatorname{Mor}}
\newcommand{\Id}{\operatorname{id}}
\newcommand{\visspace}{\text{\textvisiblespace}}
\newcommand{\Gal}{\text{Gal}}

\newcommand{\xor}{\oplus}
\newcommand{\ft}{\wedge}
\newcommand{\ift}{\vee}

\newcommand{\prob}{\mathbf{P}}
\newcommand{\expect}{\mathbf{E}}
\DeclareMathOperator{\Var}{\mathbf{V}}
\newcommand{\Ber}{\text{Ber}}
\newcommand{\Bin}{\text{Bin}}

%\newcommand{\widecheck}[1]{{#1}^{\ft}}

\DeclareMathOperator{\diam}{\text{diam}}

\DeclareMathOperator{\QQ}{\mathbf{Q}}
\DeclareMathOperator{\ZZ}{\mathbf{Z}}
\DeclareMathOperator{\RR}{\mathbf{R}}
\DeclareMathOperator{\HH}{\mathbf{H}}
\DeclareMathOperator{\CC}{\mathbf{C}}
\DeclareMathOperator{\AB}{\mathbf{A}}
\DeclareMathOperator{\PP}{\mathbf{P}}
\DeclareMathOperator{\MM}{\mathbf{M}}
\DeclareMathOperator{\VV}{\mathbf{V}}
\DeclareMathOperator{\TT}{\mathbf{T}}
\DeclareMathOperator{\LL}{\mathcal{L}}
\DeclareMathOperator{\EE}{\mathbf{E}}
\DeclareMathOperator{\NN}{\mathbf{N}}
\DeclareMathOperator{\DQ}{\mathcal{Q}}
\DeclareMathOperator{\IA}{\mathfrak{a}}
\DeclareMathOperator{\IB}{\mathfrak{b}}
\DeclareMathOperator{\IC}{\mathfrak{c}}
\DeclareMathOperator{\IP}{\mathfrak{p}}
\DeclareMathOperator{\IQ}{\mathfrak{q}}
\DeclareMathOperator{\IM}{\mathfrak{m}}
\DeclareMathOperator{\IN}{\mathfrak{n}}
\DeclareMathOperator{\IK}{\mathfrak{k}}
\DeclareMathOperator{\ord}{\text{ord}}
\DeclareMathOperator{\Ker}{\textsf{Ker}}
\DeclareMathOperator{\Coker}{\textsf{Coker}}
\DeclareMathOperator{\emphcoker}{\emph{coker}}
\DeclareMathOperator{\pp}{\partial}
\DeclareMathOperator{\tr}{\text{tr}}

\DeclareMathOperator{\supp}{\text{supp}}

\DeclareMathOperator{\codim}{\text{codim}}

\DeclareMathOperator{\minkdim}{\dim_{\mathbf{M}}}
\DeclareMathOperator{\hausdim}{\dim_{\mathbf{H}}}
\DeclareMathOperator{\lowminkdim}{\underline{\dim}_{\mathbf{M}}}
\DeclareMathOperator{\upminkdim}{\overline{\dim}_{\mathbf{M}}}
\DeclareMathOperator{\lhdim}{\underline{\dim}_{\mathbf{M}}}
\DeclareMathOperator{\lmbdim}{\underline{\dim}_{\mathbf{MB}}}
\DeclareMathOperator{\packdim}{\text{dim}_{\mathbf{P}}}
\DeclareMathOperator{\fordim}{\dim_{\mathbf{F}}}

\DeclareMathOperator*{\argmax}{arg\,max}
\DeclareMathOperator*{\argmin}{arg\,min}

\DeclareMathOperator{\ssm}{\smallsetminus}

\title{Radial Multipliers}
\author{Jacob Denson}

\begin{document}

\maketitle

\tableofcontents

\newpage

\chapter{General Introduction}

The question of the regularity of translation-invariant operators on $\RR^n$ has proved central to the development of modern harmonic analysis. Indeed, answers to these questions underpin any subtle understanding of the Fourier transform, since with essentially any such operator $T$, we can associate a function $m: \RR^n \to \CC$ (technically a tempered distribution), known as the \emph{symbol} of $T$, such that for any function $f$,
%
\[ Tf(x) = \int m(\xi) \widehat{f}(\xi) e^{2 \pi i \xi \cdot x}, \]
%
i.e. $\widehat{Tf} = m \widehat{f}$. This is why these operators are also called \emph{Fourier multipliers}. Using the spectral calculus of unbounded operators, one can also write this operator as $m(D)$, where $D = (2 \pi i)^{-1} \nabla$ is an appropriate scalar multiple of the gradient. Thus the study of the boundedness of translation invariant operators is closely connected to the study of the interactions between the characters $e_\xi(x) = e^{2 \pi i \xi \cdot x}$, which are eigenfunctions for the components of $D$.

The study of translation invariant operators emerges from many classical questions in analysis, like that of the convergence properties of Fourier series, or in mathematical physics, through the study of the heat, wave, and Schr\"{o}dinger equation. These operators also have rotational symmetry, so it is natural to restrict our attention to translation-invariant operators which are also rotation-invariant. These operators are precisely those represented by symbols $m: \RR^n \to \CC$ which are \emph{radial}, i.e. such that $m(\xi) = h(|\xi|)$ for some function $h: [0,\infty) \to \CC$, and so this class of operators are also called \emph{radial Fourier multipliers}. Using the spectral calculus of unbounded operators, one can write such operators as $h(\sqrt{-\Delta})$, where $\Delta = \sum \partial^2 / \partial x_i$ is the usual Laplacian. This hints at the fact that the study of radial multipliers is closely connected to interactions between eigenfunctions for the Laplacian. This research project studies necessary and sufficient conditions to guarantee the $L^p$ boundedness of radial multiplier operators in both the Euclidean setting, and generalizations of such operators on compact Riemannian manifolds, stimulated by recent developments which indicate lines of attack for three related problems in the field.

\chapter{Background}

The general study of the boundedness of Fourier multipliers was intiated in the 1960s. It was quickly realized that the most fundamental estimates were those of the form
%
\[ \| Tf \|_{L^q(\RR^d)} \lesssim \| f \|_{L^p(\RR^d)}, \]
%
for $1 \leq p \leq 2$, and $q \geq p$, and that one should determine these properties by studying the regularity of the associated symbol $m$. This began the study of the characterization of the function spaces $M^{p,q}(\RR^d)$, which consist of the family of symbols $m$ whose associated Fourier multiplier operator $T$ is bounded from the $L^p$ norm to the $L^q$ norm. The space $M^{p,q}(\RR^d)$ is then naturally a Banach space by taking the operator norm
%
\[ \| m \|_{M^{p,q}(\RR^d)} = \sup \left\{ \frac{\| T f \|_{L^q(\RR^d)}}{\| f \|_{L^p(\RR^d)}} : f \in \mathcal{S}(\RR^d) \right\}. \]
%
The spaces $M^p(\RR^d) = M^{p,p}(\RR^d)$ are often of the most interest.

It is simple to characterize the spaces $M^{1,q}(\RR^d)$, since for any symbol $m$, if $k = \widehat{m}$, then
%
\[ \| m \|_{M^{1,q}(\RR^d)} = \begin{cases} \| k \|_{L^q(\RR^d)} &: q > 1 \\ \| k \|_{M(\RR^d)} &: q = 1, \end{cases}. \]
%
where $M(\RR^d)$ is the space of finite signed Borel measures equipped with the total variation norm. By virtue of the fact that the Fourier transform is an isometry, the space $M^{2,2}(\RR^d)$ is characterized by the fact that
%
\[ \| m \|_{M^{2,2}(\RR^d)} = \| m \|_{L^\infty(\RR^d)}. \]
%
But for any other pair of exponents $p$ and $q$, finding a simple characterization of the space $M^{p,q}(\RR^d)$ proved to be an impenetrable problem.

It is therefore suprising that recent results have indicated a conjectured characterization of the subspace of $M^{p,q}(\RR^d)$ for \emph{radial symbols} in this class, for an appropriate range of exponents. This conjecture is best phrased in terms of the \emph{homogeneous Besov spaces} $\dot{B}^{p,q}_s(\RR^d)$, the space consisting of all distributions $f$ such that the norm
%
\[ \| f \|_{\dot{B}^{p,q}_s(\RR^d)} = \left( \sum_{j = -\infty}^\infty \left( 2^{js} \| P_j f \|_{L^p(\RR^d)} \right)^q \right)^{1/q} = \| 2^{js} P_j f \|_{l^q(\ZZ) L^p(\RR^d)}, \]
%
is finite, where $P_j$ is the Littlewood-Paley projection operator onto a dyadic frequency band of radius $2^j$. This conjecture was most clearly indicated in the result of \cite{GarrigosandSeeger}, which concerned radial multipliers $m$ whose associated operator $T$ is bounded from the $L^p$ norm to the $L^q$ norm \emph{restricted to radial functions}, i.e. such that the norm
%
\[ \| m \|_{M^{p,q}_{\text{rad}}(\RR^d)} = \sup \left\{ \frac{\| Tf \|_{L^q(\RR^d)}}{\| f \|_{L^p(\RR^d)}} : f \in \mathcal{S}(\RR^d)\ \text{and $f$ is radial} \right\} \]
%
is finite. The main result of \cite{GarrigosandSeeger} was that if $d > 1$, if $1 < p < 2d/(d+1)$, and if $p \leq q < 2$, then $M^{p,q}_{\text{rad}}(\RR^d)$ is a subset of $L^1_{\text{loc}}(\RR^d)$, and for any radial symbol $m$,
%
\[ \| m \|_{M^{p,q}_{\text{rad}}(\RR^d)} \sim_{p,q,d} \| \widehat{m} \|_{\dot{B}^{q,\infty}_{-d/p^*}} = \sup\nolimits_j \| 2^{-(d/p^*) j} P_j \widehat{m} \|_{L^q(\RR^d)}. \]
%
It is natural to conjecture that the same constraint continues to hold when we remove the constraint that our inputs $f$ are radial, i.e. that for radial symbols $m$, for $d > 1$, $1 < p < 2d/(d+1)$, and for $p \leq q < 2$,
%
\[ \| m \|_{M^{p,q}} \sim_{p,q,d} \| \widehat{m} \|_{\dot{B}_{-d/p^*}^{q,\infty}} \]
%
In the sequel, we shall call this the \emph{radial multiplier conjecture}.

% x has units M
% xi has units 1/M
% m is unitless
% Then k = m^ has units M^{-d}
% t also has units 1/M
% The projections P_t m^ have units M^{-d} as well.
% The L^p norm of P_t m^ has units M^{-d} * M^{d/p} = M^{-d/p^*}
% t^s | P_t m^ |_{L^p} has units A M^{-s-d/p^*}
% The L^q_t norm of t^s | P_t m^ |_{L^p} has units M^{-s-d/p^*-d/q}
% The Besov norm of m^ has units M^{-s-d/p^*-d/q}

% k has units M^{-d}
% For a dimensionless function f, k * f is dimensionless
% So | k * f |_{L^q} has units M^{d/q}
% And |f|_p has units M^{d/p}
% so the M^{p,q} norm of m, which is the supremum of quantities | k * f |_q / |f|_p, which have units M^{d/q-d/p}, also has units M^{d/q - d/p}.

% r has units M
% The projections P_r m are unitsless
% The L^p norm of P_r m has units M^{-d/p}
% r^s | P_r m^ |_{L^p} has units M^{s - d/p}
% The L^q_r norm of r^s |P_r m^ |_{L^p} has units M^{s - d/p + d/q}

% M^{1,q} norm has units M^{-d/q^*}
% The B^{p,2}_0 norm of m^ has units M^{-d/p^*-d/2}
% Suppose -d/q^* = -d/p^* - d/2
% 1/p^* - 1/q^* + 1/2 = 0
% Not quite matching up.

\begin{remark}
    We can choose our Littlewood-Paley projections such that
    %
    \[ \widehat{P_j f}(\xi) = \phi(\xi / 2^j) \widehat{f}(\xi) = (\text{Dil}_{2^j} \phi)(\xi) \widehat{f}(\xi). \]
    %
    for a suitable radial function $\phi \in C_c^\infty(\RR^d)$ with $1 = \sum \text{Dil}_{2^j} \phi$. Then
    %
    \[ P_j \widehat{m}(\xi) = 2^{jd} T(\text{Dil}_{2^{-j}} \widehat{\phi}) \]
    %
    and so
    %
    \[ \| P_j \widehat{m} \|_{L^q(\RR^d)} \leq 2^{jd} \| m \|_{M^{p,q}(\RR^d)} \| \text{Dil}_{2^{-j}} \phi \|_{L^p(\RR^d)} \lesssim_{d,p} 2^{jd/p^*} \| m \|_{M^{p,q}(\RR^d)}. \]
    %
    Thus we conclude quite simply that for any values of $p$ and $q$,
    %
    \[ \| m \|_{M^{p,q}(\RR^d)} \gtrsim_{d,p} \sup\nolimits_j 2^{-jd/p^*} \| P_j \widehat{m} \|_{L^q(\RR^d)} = \| \widehat{m} \|_{\dot{B}_{-d/p^*}^{q,\infty}}. \]
    %
    The difficult part of the conjecture is thus justifying the upper bound on the $M^{p,q}(\RR^d)$ norm in terms of the Besov norm for the appropriate range of exponents.
\end{remark}

\begin{remark}
    Let $m(\xi) = \mathbf{I}(|\xi| \leq 1)$ be the ball multiplier. Then we can write
    %
    \[ m(\xi) = m_0(\xi) + \sum_{j = 1}^\infty m_j(\xi) = m_0(\xi) + m_{\geq 1}(\xi). \]
    %
    where $m_0(\xi)$ is smooth and compactly supported on $|\xi| \leq 1/2$, and where $m_j(\xi) = \phi(2^j (1 - x))$ lives at the frequency scale $\approx 2^j$ and is supported on an annulus of width $1/2^j$ and radius $\approx 1$.    Now
    %
    \[ \| m_0 \|_{\cdot{B}^{p,\infty}_s(\RR^d)} \sim \sup_{j \leq 0} 2^{js} \| P_j m_0 \|_{L^p(\RR^d)} \sim \sup_{j \leq 0} 2^{j(s + d/p^*)} \]
    %
    which is $O(1)$ for $s + d/p^* \geq 0$. On the other hand,
    %
    \[ \| m_{\geq 1} \|_{\cdot{B}^{p,\infty}_s(\RR^d)} \sim \sup_{j > 0} 2^{js} \| m_j \|_{L^p(\RR^d)} \sim \sup_{j > 0} 2^{j(s - 1/p)}. \]
    %
    This is $O(1)$ provided that $s \leq 1/p$. Since
    %
    \[ \| m \|_{\cdot{B}^{p,\infty}_s(\RR^d)} \sim_{p,q,s} \| m_0 \|_{\cdot{B}^{p,\infty}_s(\RR^d)} + \| m_{\geq 1} \|_{\cdot{B}^{p,\infty}_s(\RR^d)}, \]
    %
    we find $m \in \cdot{B}^{p,\infty}_s(\RR^d)$ for $-d/p^* \leq s \leq 1/p$. Now it is a result of Fefferman (TODO: CITE?) that $m$ does not lie in any of the spaces $M^{p,q}(\RR^d)$ except for when $p = q = 2$. Looking at the results of the radial multiplier conjecture, this example shows that the generalization of the radial multiplier conjecture to all exponents cannot possibly be true for $-d/p^* \leq 1/p$,
\end{remark}

We now know, by the results of \cite{HeoandNazarovandSeeger} and \cite{Cladek}, that the radial multiplier conjecture is true when $n > 4$ and $1 < p < 2 - 4/(n+1)$, and when $n = 4$ and $1 < p < 2 - 3.79/(n+1)$. We also know \cite{Cladek} the criterion in the conjecture is sufficient to obtain a \emph{restricted weak type} bound
%
\[ \| Tf \|_{L^p(\RR^n)} \lesssim \| f \|_{L^{p,1}(\RR^n)} \]
%
when $n = 3$ and $1 < p < 2 - 3.66/(n+1)$. But the radial multiplier conjecture has not yet been completely resolved in any dimension $n$, we do not have any strong type $L^p$ bounds when $n = 3$, and we don't have any bounds whatsoever when $n = 2$. One goal of this research project is to investigate whether one can use modern research techniques to improve upon these bounds.

The full proof of the radial multiplier is likely far beyond current research techniques. Indeed, it remains a major open problem in harmonic analysis to determine the range of exponents for which \emph{specific} radial Fourier multipliers are bounded in the range where the conjecture would apply, such as the Fourier multiplier on $\RR^d$ with symbol $m_\lambda(\xi) = \left( 1 - |\xi| \right)^\lambda_+$, the family of \emph{Bochner-Riesz multipliers}. The radial multiplier conjecture characterizes the range of the Bochner-Riesz multipliers, and thus the conjecture would also imply the Kakeya and restriction conjectures. All three of these results are major unsolved problems in harmonic analysis. On the other hand, the Bochner Riesz conjecture is completely resolved when $n = 2$, while in contrast, no results related to the radial multiplier conjecture are known in this dimension at all. And in any dimension $n > 2$, the range under which the Bochner-Riesz multiplier is known to hold \cite{GuoandOhandWangandWuandZhang} is strictly larger than the range under which the radial multiplier conjecture is known to hold, even for the restricted weak-type bounds obtained in \cite{Cladek}. Thus it still seems within hope that the techniques recently applied to improve results for Bochner-Riesz problem, such as broad-narrow analysis \cite{BourgainandGuth}, the polynomial Wolff axioms \cite{KatzandRogers}, and methods of incidence geometry and polynomial partitioning \cite{Zahl2} can be applied to give improvements to current results characterizing the boundedness of general radial Fourier multipliers.

Our hopes are further emboldened when we consult the proofs in \cite{HeoandNazarovandSeeger} and \cite{Cladek}, which reduce the radial multiplier conjecture to the study of upper bounds of quantities of the form
%
\[ \| \sum_{(y,r) \in \mathcal{E}} F_{y,r} \|_{L^p(\RR^n)}, \]
%
where $\mathcal{E} \subset \RR^n \times (0,\infty)$ is a finite collection of pairs, and $F_{y,r}$ is an oscillating function supported on a $O(1)$ neighborhood of a sphere of radius $r$ centered at a point $y$. The $L^p$ norm of this sum is closely related to the study of the tangential intersections of these spheres, a problem successfully studied in more combinatorial settings using incidence geometry and polynomial partitioning methods \cite{Zahl}, which provides further estimates that these methods might yield further estimates on the radial multiplier conjecture.

When $n = 3$, the results of \cite{Cladek} are only able to obtain bounds on the $L^p$ sums in the last paragraph when $\mathcal{E}$ is a Cartesian product of two subsets of $(0,\infty)$ and $\RR^n$. This is why only restricted weak-type bounds have been obtained in this dimension. It is therefore an interesting question whether different techniques enable one to extend the $L^p$ bounds of these sums when the set $\mathcal{E}$ is \emph{not} a Cartesian product, which would allow us to upgrade the result of \cite{Cladek} in $n = 3$ to give strong $L^p$ bounds. This question also has independent interest, because it would imply new results for the `endpoint' local smoothing conjecture, which concerns the regularity of solutions to the wave equation in $\RR^n$. Incidence geometry has been recently applied to yield results on the `non-endpoint' local smoothing conjecture \cite{GuthandWangandZhang}, which again suggests these techniques might be applied to yield the estimates needed to upgrade the result of \cite{Cladek} to give strong $L^p$-type bounds.

A third line of questioning about the radial multiplier conjecture is obtained by studying natural analogues of Fourier multiplier operators on Riemannian manifolds. On any geodesically complete Riemannian manifold $X$, we can define a Laplace-Beltrami operator $\Delta$ on functions on $X$, this operator will be essentially self adjoint, and so we have a spectral theory associated with this operator. In particular, we can consider operators of the form $h(\sqrt{-\Delta})$ for functions $h: \RR \to \CC$, which are analogous to the radial multipliers studied above in the Euclidean setting. Just like multiplier operators on $\RR^n$ are crucial to an understanding of the interactions between the functions $e_\xi(x) = e^{2 \pi i \xi \cdot x}$ on $\RR^n$, understanding the operators $h(\sqrt{-\Delta})$ is crucial to understanding the interactions of eigenfunctions of the Laplace-Beltrami operator on $X$. We let $M^{p,q}(X)$ denote the family of all symbols $h: \RR \to \CC$ such that the operator $T_h = h(\sqrt{-\Delta})$ is bounded from $L^p(X)$ to $L^q(X)$, with the analogoue norm.

One problem that prevents a direct generalization of the radial multiplier conjecture in this setting comes from studying multipliers on compact Riemannian manifolds. This is because on a compact Riemannian manifold, $1 / \sqrt{-\Delta}$ is a compact, bounded operator from $L^2(M)$ to itself, and so there exists $0 \leq \lambda_1 \leq \lambda_2 \leq \dots$ with $\lambda_i \to \infty$, and eigenfunctions $e_n \in C^\infty(M)$ such that
%
\[ \sqrt{-\Delta} = \sum \langle f, e_n \rangle e_n = \sum E_{\lambda_n} f. \]
%
Thus
%
\[ h \left( \sqrt{-\Delta} \right) f = \sum h(\lambda_n) \langle f, e_n \rangle e_n = \sum h(\lambda_n) E_{\lambda_n} f. \]
%
If $h$ has compact support, this sum will be finite, and thus trivially bounded from $L^p(M)$ to $L^q(M)$ for any choice of exponents $p$ and $q$, regardless of the function $h$. A more accurate formulation of this problem, at least to prevent trivialities, it to determine when one can get a bound of the form
%
\[ \sup_{t > 0} t^{d(1/q - 1/p)} \| \text{Dil}_t h \|_{M^{p,q}(X)} < \infty. \]
%
In fact, a transference principle of Mitjagin \cite{Mitjagin} shows that if $X$ is a compact Riemannian, and $m: \RR^d \to \CC$ is radial, with $m(\xi) = h(|\xi|)$, then
%
\[ \| m \|_{M^{p,q}(\RR^d)} \lesssim_{X,p,q} \sup_{t > 0} t^{d(1/q - 1/p)} \| \text{Dil}_t h \|_{M^{p,q}(X)}. \]
%
Thus the Fourier multiplier problem on a compact manifold $X$ is at least as hard as it is on $\RR^n$. One direction of this research project here is to try and extend the radial multiplier conjecture to this setting. Of course, since we are no longer dealing with functions on $m$, the precise regularity conditions are naturally rephrased in a slightly different manner. In fact, if $m(\xi) = h(|\xi|)$, then for the range of exponents to which the radial multiplier conjecture applies,
%
\begin{align*}
    \| \widehat{m} \|_{B^{p,\infty}_{-d/q^*}(\RR^d)} &\sim \sup_{j > 0} 2^{jd(1/p - 1/q)} \left( \int_{-\infty}^\infty |P_j \widehat{h}(t)|^q (1 + |x|)^{(d-1)(1 - q/2)}\; dx \right)^{1/q}\\
    &= \sup_{j > 0} 2^{jd(1/p-1/q)} \| P_j \widehat{h} \|_{L^q(\RR, (1 + |x|)^{(d-1)(1 - q/2)})},
\end{align*}
%
which, we can write as a weighted non-homogeneous Besov space
%
\[ B^{q,\infty}_{d(1/p - 1/q)}(\RR, (1 + |x|)^{(d-1)(1-q/2)}). \]

On general compact manifolds, there are difficulties arising from a generalization of the radial multiplier conjecture, connected to the fact that analogues of the Kakeya / Nikodym conjecture are false in this general setting \cite{Minicozzi}. But these problems do not arise for constant curvature manifolds, like the sphere. The sphere also has over special properties which make it especially amenable to analysis, such as the fact that solutions to the wave equation on spheres are periodic. Best of all, there are already results which achieve the analogue of \cite{GarrigosandSeeger} on the sphere. Thus it seems reasonable that current research techniques can obtain interesting results for radial multipliers on the sphere, at least in the ranges established in \cite{HeoandNazarovandSeeger} or even \cite{Cladek}.

In conclusion, the results of \cite{HeoandNazarovandSeeger} and \cite{Cladek} indicate three lines of questioning about radial Fourier multiplier operators, which current research techniques place us in reach of resolving. The first question is whether we can extend the range of exponents upon which the conjecture of \cite{GarrigosandSeeger} is true, at least in the case $n = 2$ where Bochner-Riesz has been solved. The second is whether we can use more sophisticated arguments to prove the $L^p$ sum bounds obtained in \cite{Cladek} when $n = 3$ when the sums are no longer Cartesian products, thus obtaining strong $L^p$ characterizations in this settiong, as well as new results about the endpoint local smoothing conjecture. The third question is whether we can generalize these bounds obtained in these two papers to study radial Fourier multipliers on the sphere. 

\chapter{Papers / Books To Read In More Detail}

\begin{itemize}
    \item Sogge, $L^p$ Estimates For the Wave Equation and Applications (1993).

    A survey of results on regularity results for the wave equation. In particular, reviews (without proof) the ideas of Mockenhaipt, Seeger, and Sogge which give local smoothing for Fourier integral operators satisfying the cone condition, as well as mixed norm estimates for non-homogeneous results on wave equations.

    \item In Sogge's Book, he mentions the main developments in harmonic / microlocal analysis he couldn't discuss in the book were the following:
    \begin{itemize}
    \item Bennett, Carbery, Tao, On the Multilinear Restriction and Kakeya Conjecture (2006).

    Introduction to multilinear methods in harmonic analysis.

    \item Bourgain, Guth, Bounds on Oscillatory Integral Operators Based on Multilinear Estimates (2010).

    Application of multilinear methods to bounding oscillatory integrals.

    \item Bourgain, Demeter, The Proof of the l2 Decoupling Conjecture (2014).

    Introduction to Decoupling.

    \item Peetre, New Thoughts on Besov-Spaces.

    Characterizes boundedness of Fourier multipliers on homogeneous Besov spaces.

    \item Johnson, Maximal Subspaces of Besov-Spaces Invariant Under Multiplication By Characters.

    Shows a Fourier multiplier operator is bounded in the $L^p$ norm if and only if it's translates are all localizably bounded as in Seeger.
    \end{itemize}

    \item For more background reading in microlocal analysis:
    \begin{itemize}
        \item H\"{o}rmander, The Analysis of Linear Partial Differential Operators, Volumes I-IV.
        \item Treves, Introduction to Pseudodifferential and Fourier Integral Operators, Volumes I-II.
        \item Taylor.
    \end{itemize}

%Chapter 4 describes the work of
    - Hormander, The Spectral Function of an Elliptic Operator
    - Avakumovic, Uber die Eigenfunktionen auf Geschlossenen Riemannschen Mannigfaltigkeiten
    - Levitan, On the Asymptotic Behaviour of the Spectral Function of a Self-Adjoint Differential Equation of Second Order.

%- Read Hormander, Estimates for Translation Invariant Operators on Lp Spaces For More In Depth Foundations of Lp Boundedness of Multiplier Operators
%- See Strichartz [1] and Keel Tao [1], Ginibre Velo [1], Lindblad Sogge [1] for sharp embeding of wave operator using orthogonality argument introduced in introduction.
%- Seeger, Roos, Po Lam Yung. Maximal Functions for Families of Hilbert Transforms.
%- Guo, Oh, Wang. The Bochner-Riesz Problem: An Old Approach Revisited.
%- Find Stuff about the Transference Principle
%- Hickman, Guth, Illiopoulos. Sharp Estimates for Oscillatory Integral Operators via Polynomial Partitioning.

%- Fourier Restriction for Hypersurfaces in Three Dimensions and Newton Polyhedra

\end{itemize}


\chapter{Seeger: Singular Convolution Operators in $L^p$ Spaces}

Let $m: \RR^d \to \CC$ be the symbol for a Fourier multiplier operator $m(D)$. If the resulting operator $m(D)$ was bounded from $L^p(\RR^d)$ to $L^p(\RR^d)$ with operator norm $A$, then the operator would also be bounded `at all scales'. That is, if we consider a littlewood Paley decomposition, i.e. taking
%
\[ f = \sum_{i = 0}^\infty f_i \]
%
where $\widehat{f_i} = \eta_i \widehat{f}$ is supported on $2^i \leq |\xi| \leq 2^{i+1}$ for $i \geq 1$, and $|\xi| \leq 2$ for $i = 0$, then we would have estimates of the form
%
\begin{equation} \label{piecewiseBound}
    \| m(D) f_i \|_{L^p(\RR^d)} \lesssim \| f_i \|_{L^p(\RR^d)} \lesssim \| f \|_{L^p(\RR^d)},
\end{equation}
%
where the implicit constant is uniform in $i$. The main focus of the paper in question is to determine whether a uniform bound of the form \eqref{piecewiseBound} implies $m(D)$ is bounded. More precisely, is it true that
%
\begin{equation} \label{operatorbound}
    \| m \|_{M^p(\RR^d)} \lesssim_p \sup\nolimits_{i \geq 0} \| m_i \|_{M^p(\RR^d)},
\end{equation}
%
where $m_i = \eta_i m$.

The Hilbert transform $H$ is a Fourier multiplier with symbol $m(\xi) = \text{sgn}(\xi)$. For each $i > 0$, $m_i(\xi) = \eta_i \text{sgn}(\xi)$, so that
%
\[ K_i(x) = \widehat{\eta_i \text{sgn}(\xi)} = 2^i H \eta(2^i x). \]
%
Thus
%
\[ \| K_i \|_{L^1(\RR)} = \| H \eta \|_{L^1(\RR)}. \]
%
TODO

It is clear that \eqref{operatorbound} is true for $p = 2$, since in this case the bound is equivalent to an inequality of the form
%
\[ \| m \|_{L^\infty(\RR^d)} \lesssim \sup\nolimits_{i \geq 0} \| m_i \|_{L^\infty(\RR^d)}, \]
%
which is true because the supports of the symbols $\{ m_i \}$ are almost all pairwise disjoint. On the other hand, \eqref{operatorbound} does not hold when $p = 1$ or $p = \infty$, which makes sense, since Littlewood-Paley runs into all kinds of problems for these values of $p$. Arguing more precisely, the condition would be equivalent to showing that for any $K: \RR^d \to \CC$,
%
\[ \| K \|_{L^1(\RR^d)} \lesssim \sup\nolimits_{i \geq 0} \| K * \widehat{\eta_i} \|_{L^1(\RR^d)}. \]
%
If
%
\[ K_N(x) = \int_{|\xi| \leq 2^N} e^{2 \pi i \xi \cdot x}\; d\xi \]
%
is the Dirichlet kernel, then $\| K_N \|_{L^1(\RR)} \sim N$. On the other hand, for $i \leq N-1$, we have $K_N * \widehat{\eta_i} = \widehat{\eta_i}$, so that
%
\[ \| K_N * \widehat{\eta_i} \|_{L^1(\RR)} = \| \widehat{\eta_i} \|_{L^1(\RR)} \lesssim 1. \]
%
For $i \geq N+1$, we have $K_N * \widehat{\eta_i} = 0$, so that
%
\[ \| K_N * \widehat{\eta_i} \|_{L^1(\RR)} = 0 \lesssim 1. \]
%
For $i = N$, we have
%
\[ (K_N * \widehat{\eta_N})(x) = 2^N \int_0^1 \eta(\xi) e^{2 \pi i 2^N (\xi \cdot x)} + \int_1^2 \eta(-\xi) e^{-2 \pi i 2^N (\xi \cdot x)}\; d\xi  \]
%
\[ \int |K_N * \widehat{\eta_i}| \]

whereas one
% 
%
\[ K_N * \widehat{\eta_i} = \begin{cases} \widehat{\eta_i} &: i \lesssim N \\ 0 &: i \gtrsim N \end{cases}, \]
%
and so $\| K_N * \widehat{\eta_i} \|_{L^1(\RR)} \lesssim 1$ uniformly in $N$ and $i$. We can then use Baire category techniques to find a kernel $K$ not in $L^1(\RR)$, but such that $\| K * \eta_i \|_{L^1(\RR)} \lesssim 1$, uniformly in $i$.

The result actually fails for $2 < p < \infty$, due to an examples of Triebel. For simplicity, let's work in $\RR$. If we fix a bump function $\phi \in C_c^\infty(\RR)$ supported in $[-1,1]$, and set
%
\[ m_N(\xi) = \sum_{k = N}^{2N} e^{2 \pi i (2^k \xi)} \phi(\xi - 2^k), \]
%
then $m_N(\xi) \eta_i(\xi) = m_{N,i}(\xi)$, where $m_{N,i}(\xi) = e^{2 \pi i (2^k \xi)} \phi(\xi - 2^k)$, and so $K_{N,i}(x) = \widehat{m_{N,i}}(x) = e^{2 \pi i 2^k(x - 2^k)} \widehat{\phi}(x - 2^k)$, hence
%
\[ \| m_{N,i}(D) f \|_{L^p(\RR^d)} = \| K_{N,i} * f \|_{L^p(\RR^d)} \leq \| \widehat{\phi} \|_{L^1(\RR)} \| f \|_{L^p(\RR)} \lesssim \| f \|_{L^p(\RR)}. \]
%
On the other hand, the operator norm of $m_N(D)$ from $L^p(\RR)$ to $L^p(\RR)$ is actually $\gtrsim_p N^{|1/p - 1/2|}$, and thus not bounded uniformly in $N$, so Baire category shows things don't work so well here.

This paper shows that one \emph{can} get uniform bounds assuming an additional, very weak smoothness condition, which rules out the example $m_N$ above. Under the most simple assumptions, if \eqref{piecewiseBound} holds, and $\| m_i \|_{\Lambda^\varepsilon} \lesssim 2^{-ik}$, where $\Lambda^\varepsilon$ is the $\varepsilon$-Lipschitz norm, then $\| m(D) f \|_{L^r(\RR^d)} \lesssim \| f \|_{L^r(\RR^d)}$ whenever $|1/r - 1/2| < |1/p - 1/2|$. Under slightly stronger smoothness assumptions, we can actually conclude $\| m(D) f \|_{L^p(\RR^d)} \lesssim \| f \|_{L^p(\RR^d)}$.

To prove the result, we rely on Littlewood-Paley theory and the Fefferman-Stein sharp maximal function. Without loss of generality we may assume that $2 < p < \infty$. We will actually show that if for all $i$ and $\omega \geq 0$,
%
\[ \int_{|x| \geq \omega} |K_i(x)|\; dx \leq B (1 + 2^i \omega)^{-\varepsilon}, \]
%
consistent with the fact that, if $m_i$ was smooth, the uncertainty principle would say that $K_i$ would live on a ball of radius $1/2^i$. We will then prove that $\| m(D) f \|_{L^p(\RR^d)} \leq A \widetilde{\log}(B/A)^{|1/2 - 1/p|}$. Our goal is to show that if
%
\[ S^\# f(x) = \sup_{x \in Q} \fint_Q \left( \sum_{i = 0}^\infty \left| m_i(D) f(y) - \fint_Q m_i(D) f(z)\; dz \right|^2 \right)^{1/2}\; dy, \]
%
then $\| S^\# f \|_{L^p(\RR^d)} \lesssim A \widetilde{\log}(B/A)^{1/2 - 1/p} \| f \|_{L^p(\RR^d)}$. It then follows by Littlewood-Paley theory implies
%
\begin{align*}
    \| m(D) f \|_{L^p(\RR^d)} &\lesssim_p \left\| \left( \sum_{k = 0}^\infty |m_i(D) f|^2 \right)^{1/2} \right\|_{L^p(\RR^d)}\\
    &\leq \left\| M \left[ \left( \sum_{k = 0}^\infty |m_i(D) f|^2 \right)^{1/2} \right] \right\|_{L^p(\RR^d)}\\
    &\lesssim \left\| S^\# \left( \sum_{k = 0}^\infty |m_i(D) f|^2 \right)^{1/2} \right\|_{L^p(\RR^d)}\\
    &\lesssim A \widetilde{\log}(B/A)^{1/2 - 1/p}.
\end{align*}
%
To bound $S^\#$, we linearize using duality, picking $Q_x$ for each $x$, and a family of functions $\chi_i(x,y)$ such that $\left( \sum |\chi_i(x,y)|^2 \right)^{1/2} \leq 1$, such that
%
\[ S^\# f(x) \approx \fint_{Q_x} \sum_{i = 0}^\infty \left( m_i(D) f(y) - \fint_{Q_x} m_i(D) f(z)\; dz \right) \chi_i(x,y)\; dy. \]
%
Thus $S^\# f = S_1 f + S_2 f$, where if $Q_x$ has sidelength $2^{l(x)}$,
%
\[ S_1 f(x) = \fint_{Q_x} \sum_{|i + l(x)| \leq \tilde\log(B/A)} \left( m_i(D) f(y) - \fint_{Q_x} m_i(D) f(z)\; dz \right) \chi_i(x,y)\; dy \]
%
and
%
\[ S_2 f(x) = \fint_{Q_x} \sum_{|i + l(x)| \geq \tilde\log(B/A)} \left( m_i(D) f(y) - \fint_{Q_x} m_i(D) f(z)\; dz \right) \chi_i(x,y)\; dy. \]
%
If $|i + l(x)| \lesssim 1$, then the uncertainty principle tells us that $m_i(D) f$ is roughly constant on squares on radius $Q_x$, up to some small error, so that we should expect
%
\[ \left| m_i(D) f(y) - \fint_{Q_x} m_i(D) f(z)\; dz \right| \lesssim \left| \fint_{Q_x} m_i(D) f(z)\; dz \right|. \]
%
Thus it is natural to use the bound, $|S_1 f(x)| \lesssim M(\sum_{i = 0}^\infty |m_i(D) f|^2)^{1/2}$, which implies
%
\begin{align*}
    \| S_1 f \|_{L^2(\RR^d)} &\lesssim \| M(\sum_{i = 0}^\infty |m_i(D) f|^2)^{1/2} \|_{L^2(\RR^d)}\\
    &\lesssim \left\| (\sum_{i = 0}^\infty |m_i(D) f|^2 )^{1/2} \right\|_{L^2(\RR^d)}\\
    &= \left( \sum_{i = 0}^\infty \| m_i(D) f \|_{L^2(\RR^d)}^2 \right)^{1/2}
\end{align*}
%
and
%
\begin{align*}
    \| S_1 f \|_{L^\infty(\RR^d)} &\leq \| M(\sum_{|i + l(x)| \leq \tilde{\log}(B/A)}^\infty |m_i(D) f|^2)^{1/2} \|_{L^\infty(\RR^d)}\\
    &\leq \left\| \left( \sum_{|i + l(x)| \leq \tilde{\log}(B/A)} |m_i(D) f|^2 \right)^{1/2} \right\|_{L^\infty(\RR^d)}\\
    &\lesssim \tilde{\log}(B/A)^{1/2} \sup_i  \| m_i(D) f \|_{L^\infty(\RR^d)}
\end{align*}
%
Interpolation gives $\| S_1 f \|_{L^p(\RR^d)} \lesssim \tilde{\log}(B/A)^{1/2 - 1/p} \| m_i(D) f \|_{L^p_x(l^p_i)}$. But now Littlewood-Paley theory shows that
%
\[ \| m_i(D) f \|_{L^p_x(l^p_i)} \leq A \left( \sum_{i = 0}^\infty \| P_i f \|_{L^p(\RR^d)} \right)^{1/p} \leq A \left( \sum_{i = 0}^\infty \| P_i f \|_{L^p(\RR^d)}^2 \right)^{1/2} \lesssim A \| f \|_{L^p}. \]
%
Thus $\| S_1 f \|_{L^p(\RR^d)} \lesssim A \tilde{\log}(B/A)^{1/2 - 1/p} \| f \|_{L^p(\RR^d)}$.

On the other hand, if $i$ is much smaller than $l(x)$, we should expect the error between $m_i(D) f(y)$ and $\fint_{Q_x} m_i(D) f(z)\; dz$ to be even smaller, and if $i$ is much bigger, then $m_i(D) f$ is no longer constant at this scale, and so the averages should be small, so $m_i(D) f(x)$ should dominate $\fint_{Q_x} m_i(D) f(z)$. Now since our assumption implues that $\| m(D) f \|_{L^2(\RR^d)} \lesssim \| f \|_{L^2(\RR^d)}$, it is not so difficult to prove that
%
\[ \| S_2 f \|_{L^2(\RR^d)} \lesssim A \| f \|_{L^2(\RR^d)} \sim A \left\| \left( \sum |P_i f|^2 \right)^{1/2} \right\|_{L^2(\RR^d)}. \]
%
The difficulty is proving $\| S_2 f \|_{L^\infty(\RR^d)} \lesssim A \| \left( \sum |P_i f|^{1/2} \right) \|_{L^\infty(\RR^d)}$, which we can interpolate into an inequality like above where we can apply Littlewood-Paley theory. To do this we perform another decomposition, writing
%
\[ S_2 f = I f + {II}f \]
%
where
%
\[ If(x) = \fint_{Q_x} \sum_{|i + l(x)| \geq \tilde\log(B/A)} \left( m_i(D) (\mathbf{I}_{2Q_x}f)(y) - \fint_{Q_x} m_i(D)(\mathbf{I}_{2Q_x} f)(z)\; dz \right) \chi_i(x,y)\; dy. \]
%
and
%
\[ If(x) = \fint_{Q_x} \sum_{|i + l(x)| \geq \tilde\log(B/A)} \left( m_i(D) (\mathbf{I}_{(2Q_x)^c}f)(y) - \fint_{Q_x} m_i(D)(\mathbf{I}_{(2Q_x)^c} f)(z)\; dz \right) \chi_i(x,y)\; dy. \]
%
Now
%
\[ \|If \|_{L^\infty} \leq \sup_x \fint_{Q_x} \left( \sum | m_i(D) (\mathbf{I}_{2Q_x} f) |^2 \right)^{1/2}\; dy \leq \sup_x |Q_x|^{-1/2} \left( \sum \| m_i(D) (\mathbf{I}_{2Q_x} f) \|_{L^2(\RR^d)}^2 \right)^{1/2} \lesssim A |Q_x|^{-1/2} \left( \| \mathbf{I}_{2Q_x} f \|_{L^2(\RR^d)}^2 \right)^{1/2} \lesssim A \left( \sum_{i = 0}^\infty \fint_{R_x} |f|^2 \right)^{1/2} \]





\chapter{Seeger}





\chapter{Heo, Nazarov, and Seeger}

In this chapter we give a description of the techniques of Heo, Nazarov, and Seeger's paper 2011 \emph{Radial Fourier Multipliers in High Dimensions} \cite{HeoNazrovSeeger2011}. Recall that if $m \in L^\infty(\RR^d)$ is the symbol of a Fourier multiplier operator $T_m: L^2(\RR^d) \to L^2(\RR^d)$. We let $\| m \|_{M^p(\RR^d)}$ denote the operator norm of $T_m$ from $L^p(\RR^d)$ to $L^p(\RR^d)$. The goal of this paper is to show that if $m \in L^\infty(\ZZ)$ is a radial function, $d \geq 4$, and $\eta \in \mathcal{S}(\RR^d)$ is nonzero, then
%
\[ \| m \|_{M^p(\RR^d)} \sim \sup_{t > 0} t^{d/p} \| T_m(\text{Dil}_t \eta) \|_{L^p(\RR^d)} \quad \text{for}\ p \in \left(1, \frac{2(d-1)}{d+1} \right), \]
%
where the implicit constant depends on $p$ and $\eta$. Since
%
\[ \sup_{t > 0} t^{d/p} \| T_m(\text{Dil}_t \eta) \|_{L^p(\RR^d)} \sim \sup_{t > 0} \frac{\| T_m(\text{Dil}_t \eta) \|_{L^p(\RR^d)}}{\| \text{Dil}_t \eta \|_{L^p(\RR^d)}} \]
%
we find that the boundedness of $T_m$ on $\mathcal{S}(\RR^d)$ is equivalent to it's boundedness on the family $\{ \text{Dil}_t \eta \}$.

Note that the assumption of this result, if true, for $m$ is compactly supported is equivalent to the assumption that $\widehat{m}$ is in $L^p(\RR^n)$ (See Theorem 9.3 of this paper).

Note that, applying Littlewood-Paley theory, the assumption of this theorem is equivalent to the fact that
%
\[ F_{p,2}^0 = t^d \| \left( \sum_{n = 0}^\infty |\widehat{m \text{Dil}_{1/t} \widehat{\eta} \psi_n}|^2 \right)^{1/2} \|_{L^p(\RR^n)} \]
% int m(x/xi)
% int eta(x/t) e^{2 pi i xi \cdot x} = t^n widehat{eta}(t xi)

In Garrig\'{o}s and Seeger's 2007 paper \emph{Characterizations of Hankel Multipliers}, it was proved that if $\eta \in \mathcal{S}(\RR^d)$ is a nonzero, radial Schwartz function, then
%
\[ \| m \|_{M^p_{\text{rad}}(\RR^d)} \sim \sup_{t > 0} t^{d/p}  \| T_m(\text{Dil}_t \eta) \|_{L^p(\RR^d)}\quad\text{for}\ p \in \left( 1, \frac{2d}{d+1} \right), \]
%
where $M^p_{\text{rad}}(\RR^d)$ is the operator norm of $T_m$ from $L^p_{\text{rad}}(\RR^d)$ to $L^p(\RR^d)$. Thus, at least in the range $p \in \left( 1, 2\frac{d-1}{d+1} \right)$, boundedness of $T_m$ on radial functions is equivalent to boundedness on all functions.

Another consequence of the techniques of this paper is that an `endpoint' result for local smoothing is proved for the wave equation.

\section{Discretized Reduction}

It is obvious that
%
\[ \| m \|_{M^p(\RR^d)} \gtrsim_\eta \sup_{t > 0} t^{d/p} \| T_m(\text{Dil}_t \eta) \|_{L^p(\RR^d)}, \]
%
so it suffices to show that
%
\[ \| m \|_{M^p(\RR^d)} \lesssim_\eta \sup_{t > 0} t^{d/p} \| T_m(\text{Dil}_t \eta) \|_{L^p(\RR^d)}, \]
%
We will show this via a convolution inequality, which can also be used to prove local smoothing results for the wave equation.

Let $\sigma_r$ be the surface measure for the sphere of radius $r$ centered at the origin in $\RR^d$. Also fix a nonzero, radial Schwartz function $\psi \in \mathcal{S}(\RR^d)$. Given $x \in \RR^d$ and $r \geq 1$, define $f_{x r} = \text{Trans}_x (\sigma_r * \psi)$, which we view as a smoothened indicator function on a thickness $\approx 1$ annulus of radius $r$ centered at $x$. Our goal is to prove the following inequality.

\begin{lemma} \label{lemma1}
    For any $a : \RR^d \times [1,\infty) \to \CC$, and $1 \leq p < 2(d-1)/(d+1)$,
    %
    \[ \left\| \int_{\RR^d} \int_1^\infty a_r(x) f_{x r}\; dx\; dr \right\|_{L^p(\RR^d)} \lesssim \left( \int_{\RR^d} \int_1^\infty |a_r(x)|^p r^{d-1} dr dx \right)^{1/p}. \]
    %
    The implicit constant here depends on $p$, $d$, and $\psi$.
\end{lemma}

Why is Lemma \ref{lemma1} useful? Suppose $m: \RR^d \to \CC$ is a radial multiplier given by some function $\tilde{m}: [1,\infty) \to \CC$, and we set $a_r(x) = \tilde{m}(r) f(x)$ for some $f: \RR^d \to \CC$. Then it is simple to check that
%
\[ \int_{\RR^d} \int_1^\infty a_r(x) f_{x r}\; dx\; dr = K * \psi * f \]
%
where $K(x) = |x|^{1-d} m(x)$. In this setting, Lemma \ref{lemma1} says that
%
\[ \| K * \psi * f \|_{L^p(\RR^d)} \lesssim \| m \|_{L^p(\RR^d)} \| f \|_{L^p(\RR^d)}, \]
%
which is clearly related to the convolution bound we want to show if $\psi = \widehat{\eta}$, provided that we are dealing with a multiplier supported away from the origin. To understand Lemma \ref{lemma1} it suffices to prove the following discretized estimate.

\begin{theorem} \label{lemma2}
    Fix a finite family of pairs $\mathcal{E} \subset \RR^d \times [1,\infty)$, which is \emph{discretized} in the sense that $|(x_1,r_1) - (x_2,r_2)| \geq 1$ for each distinct pair $(x_1,r_1)$ and $(x_2,r_2)$ in $\mathcal{E}$. Then for any $a: \mathcal{E} \to \CC$ and $1 \leq p < 2(d - 1)/(d+1)$, 
    %
    \[ \left\| \sum_{(x,r) \in \mathcal{E}} a_r(x) f_{x r} \right\|_{L^p(\RR^d)} \lesssim \left( \sum_{(x,r) \in \mathcal{E}} |a_r(x)|^p r^{p-1} \right)^{1/p}, \]
    %
    where the implicit constant depends on $p$, $d$, and $\psi$, but most importantly, is independant of $\mathcal{E}$.
\end{theorem}

\begin{proof}[Proof of Lemma \ref{lemma1} from Lemma \ref{lemma2}]
    For any $a: \RR^d \times [1,\infty) \to \CC$,
    %
    \[ \int_{\RR^d} \int_1^\infty a_r(x) f_{x r} = \int_{[0,1)^d} \int_0^1 \sum_{n \in \ZZ^d} \sum_{m \in \ZZ} \text{Trans}_{n,m} (a f_{rx})\; dr\; dx \]
    %
    Minkowski's inequality thus implies that
    %
    \begin{align*}
    \left\| \int_{\RR^d} \int_1^\infty a_r(x) f_{x r} \right\|_{L^p(\RR^d)} &\leq \int_{[0,1)^d} \int_0^1 \| \sum_{n \in \ZZ^d} \sum_{m \in \ZZ} \text{Trans}_{n,m} (a f_{rx}) \|_{L^p(\RR^d)}\; dr\; dx\\
    &\lesssim \int_{[0,1)^d} \int_0^1 \left( \sum_{n \in \ZZ^d} \sum_{m \in \ZZ} |a_r(x)|^p r^{p-1} \right)^{1/p}\; dr\; dx\\
    &\leq \left( \int_{[0,1)^d} \int_0^1 \sum_{n \in \ZZ^d} \sum_{m \in \ZZ} |a_r(x)|^p r^{p-1}\; dr\; dx \right)^{1/p}\\
    &= \left( \int_{\RR^d} \int_1^\infty |a_r(x)|^p r^{d-1} dr dx \right)^{1/p}. \qedhere
    \end{align*}
\end{proof}

Lemma \ref{lemma2} is further reduced by considering it as a bound on the operator $a \mapsto \sum_{(x,r) \in \mathcal{E}} a_r(x) f_{xr}$. In particular, applying real interpolation, it suffices for us to prove a restricted strong type bound. Given any discretized set $\mathcal{E}$, let $\mathcal{E}_k$ be the set of $(x,r) \in \mathcal{E}$ with $2^k \leq r < 2^{k+1}$. Then Lemma \ref{lemma2} is implied by the following Lemma.

\begin{lemma} \label{lemma3}
    For any $1 \leq p < 2(d - 1)/(d+1)$ and $k \geq 1$,
    %
    \[ \left\| \sum_{(x,r) \in \mathcal{E}_k} f_{xr} \right\|_{L^p(\RR^d)} \lesssim 2^{k(d - 1)} \#(\mathcal{E}_k)^{1/p} = 2^k \cdot (2^{k(d-p-1)} \#(\mathcal{E}_k))^{1/p}. \]
\end{lemma}

\begin{proof}[Proof of Lemma \ref{lemma2} from Lemma \ref{lemma3}]
    Applying a dyadic interpolation result (Lemma 2.2 of the paper), Lemma \ref{lemma3} implies that
    %
    \[ \| \sum_{(x,r) \in \mathcal{E}} f_{xr} \|_{L^p(\RR^d)} \lesssim \left( \sum 2^{kp} 2^{k(d-p-1)} \#(\mathcal{E}_k) \right)^{1/p} = \left( \sum 2^{k(d-1)} \#(\mathcal{E}_k) \right)^{1/p} \]
    %
    This is a restricted strong type bound for Lemma \ref{lemma2}, which we can then interpolate.
\end{proof}

If $\psi$ is compactly supported, and $r$ is sufficiently large depending on the size of this support, then $f_{xr}$ is supported on an annulus with centre $x$, radius $r$, and thickness $O(1)$. Thus $\| f_{xr} \|_{L^p(\RR^d)} \sim r^{(d-1)/p}$, which implies that
%
\[ \| \sum_{(x,r) \in \mathcal{E}_k} f_{xr} \|_{L^p(\RR^d)} \gtrsim 2^{k(d-1)/p} \#(\mathcal{E}_k)^{1/p}. \]
%
Thus this bound can only be true if $p \geq 1$, and becomes tight when $p = 1$, where we actually have
%
\[ \| \sum_{(x,r) \in \mathcal{E}_k} f_{xr} \|_{L^1(\RR^d)} \sim 2^{k(d-1)} \#(\mathcal{E}_k) \]
%
because there can be no constructive interference in the $L^1$ norm. Understanding the sum in Lemma \ref{lemma3} for $1 < p < 2(d-1)/(d+1)$ will require an understanding of the interference patterns of annuli with comparable radius. We will use almost orthogonality principles to understand these interference patterns.

\begin{lemma} \label{lemma4}
    For any $N > 0$, $x_1,x_2 \in \RR^d$ and $r_1,r_2 \geq 1$,
    %with $|x_1 - x_2| \geq 1$ or $x_1 = x_2$, and $r_1,r_2 > 1$,
    %
    \begin{align*}
        |\langle f_{x_1r_1}, f_{x_2r_2} \rangle| &\lesssim_N (r_1r_2)^{(d-1)/2} (1 + |r_1 - r_2| + |x_1 - x_2|)^{-(d-1)/2}\\
        &\quad\quad\quad\sum_{\pm,\pm} (1 + ||x_1 - x_2| \pm r_1 \pm r_2|)^{-N}.
    \end{align*}
    %
    In particular,
    %
    \[ |\langle f_{x_1r_1}, f_{x_2r_2} \rangle| \lesssim \left( \frac{r_1r_2}{|(x_1,r_1) - (x_2,r_2)|} \right)^{(d-1)/2} \]
\end{lemma}

\begin{remark}
    Suppose $r_1 \leq r_2$. Then Lemma \ref{lemma4} implies that $f_{x_1r_1}$ and $f_{x_2r_2}$ are roughly uncorrelated, except when $|x_1 - x_2|$ and $|r_1 - r_2|$ is small, and in addition, one of the following two properties hold:
    %
    \begin{itemize}
        \item $r_1 + r_2 \approx |x_1 - x_2|$, which holds when the two annuli are `approximately' externally tangent to one another.

        \item $r_2 - r_1 \approx |x_1 - x_2|$, which holds when the two annuli are `approximately' internally tangent to one another.
    \end{itemize}
    %
    Heo, Nazarov, and Seeger do not exploit the tangency information, though utilizing the tangencies seems important to improve the results they obtain. In particular, Laura Cladek's paper exploits this tangency information.
\end{remark}

\begin{proof}
%    We may assume $|x_1 - x_2| \geq 1$, for otherwise the inequality holds trivially since unless $|r_1 - r_2| \lesssim 1$, $f_{x_1r_1}$ and $f_{x_2r_2}$ have disjoint support, and if $|r_1 - r_2| \lesssim 1$ then Cauchy Schwartz implies that
    %
%    \begin{align*}
%        |\langle f_{x_1r_1}, f_{x_2r_2} \rangle| &\lesssim (r_1 r_2)^{(d-1)/2}\\
%        &\lesssim_{N,d} (r_1r_2)^{(d-1)/2} (1 + |r_1 - r_2| + |x_1 - x_2|)^{-(d-1)/2} \sum_{\pm,\pm} (1 + ||x_1 - x_2| \pm r_1 \pm r_2|)^{-N}
%    \end{align*}
%
    We write
    %
    \begin{align*}
        \langle f_{x_1 r_1}, f_{x_2 r_2} \rangle &= \left\langle \widehat{f}_{x_1 r_1}, \widehat{f}_{x_2 r_2} \right\rangle\\
        &= \int_{\RR^d} \widehat{\sigma_{r_1} * \psi}(\xi) \cdot \overline{\widehat{\sigma_{r_2} * \psi}(\xi)} e^{2 \pi i (x_2 - x_1) \cdot \xi}\; d\xi\\
        &= (r_1 r_2)^{d-1} \int_{\RR^d} \widehat{\sigma}(r_1 \xi) \overline{\widehat{\sigma}(r_2 \xi)} |\widehat{\psi}(\xi)|^2 e^{2 \pi i (x_2 - x_1) \cdot \xi}\; d\xi.
    \end{align*}
    %
    Define functions $A$ and $B$ such that $B(|\xi|) = \widehat{\sigma}(\xi)$, and $A(|\xi|) = |\widehat{\psi}(\xi)|^2$. Then
    %
    \[ \langle f_{x_1 r_1}, f_{x_2 r_2} \rangle = C_d (r_1r_2)^{d-1} \int_0^\infty s^{d-1} A(s) B(r_1 s) B(r_2 s) B(|x_2 - x_1| s)\; ds. \]
    %
    Using well known asymptotics for the Fourier transform for the spherical measure, we have
    %
    \[ B(s) = s^{-(d-1)/2} \sum_{n = 0}^{N-1} (c_{n,+} e^{2 \pi i s} + c_{n,-} e^{-2 \pi i s}) s^{-n} + O_N(s^{-N}). \]
    %
    But now substituting in, assuming $A(s)$ vanishes to order $100N$ at the origin, we conclude that
    %
    \begin{align*}
        \langle f_{x_1 r_1}, f_{x_2 r_2} \rangle &= C_d \left( \frac{r_1r_2}{|x_1 - x_2|} \right)^{(d-1)/2} \sum_{n,\tau} c_{n,\tau}  r_1^{-n_1} r_2^{-n_2} |x_2 - x_1|^{-n_3}\\
        &\quad\quad\quad \Bigg\{ \int_0^\infty A(s) s^{-(d-1)/2}  s^{-n_1-n_2-n_3} e^{2 \pi i (\tau_1 r_1 + \tau_2 r_2 + \tau_3 |x_2 - x_1|) s}\; ds \Bigg\}\\
        &\lesssim_N \left( \frac{r_1r_2}{|x_1 - x_2|} \right)^{\frac{d-1}{2}} \left(1 + \frac{1}{|x_1 - x_2|^N} \right) \sum_{\tau} \left( 1 + |\tau_1 r_1 + \tau_2 r_2 + \tau_3 |x_2 - x_1|| \right)^{-5N}\\
        &\lesssim_N \left( \frac{r_1r_2}{|x_1 - x_2|} \right)^{\frac{d-1}{2}} \left(1 + \frac{1}{|x_1 - x_2|^N} \right) \sum_\tau \left( 1 + |\tau_1 \tau_3 r_1 + \tau_2 \tau_3 r_2 + |x_2 - x_1|| \right)^{-5N}.
    \end{align*}
    %
    This gives the result provided that $1 + |x_1 - x_2| \geq |r_1 - r_2| / 10$ and $|x_1 - x_2| \geq 1$. If $1 + |x_1 - x_2| \leq |r_1 - r_2| / 10$, then the supports of $f_{x_1r_1}$ and $f_{x_2r_2}$ are disjoint, so the inequality is trivial. On the other hand, if $|x_1 - x_2| \leq 1$, then the bound is trivial by the last sentence unless $|r_1 - r_2| \leq 10$, and in this case the inequality reduces to the simple inequality
    %
    \[ \langle f_{x_1r_1}, f_{x_2r_2} \rangle \lesssim_N (r_1 r_2)^{(d-1)/2}. \] 
    %
    But this follows immediately from the Cauchy-Schwartz inequality.
\end{proof}

The exponent $(d-1)/2$ in Lemma \ref{lemma4} is too weak to apply almost orthogonality directly to obtain $L^2$ bounds on $\sum_{(x,r) \in \mathcal{E}_k} f_{xr}$. To fix this, we apply a `density decomposition', somewhat analogous to a Calderon Zygmund decomposition, which will enable us to obtain $L^2$ bounds. We say a 1-separated set $\mathcal{E}$ in $\RR^d \times [R,2R)$ is of \emph{density type} $(u,R)$ if
%
\[ \#(B \cap \mathcal{E}) \leq u \cdot \diam(B) \]
%
for each ball $B$ in $\RR^{d+1}$ with diameter $\leq R$. A covering argument then shows that for any ball $B$,
%
\[ \#(B \cap \mathcal{E}) \lesssim_d u \cdot \left( 1 + \frac{\diam(B)}{R} \right)^d \cdot \diam(B). \]
%
(NOTE: WE MIGHT BE ABLE TO DO BETTER USING THE FACT THAT $\mathcal{E} \subset \RR^d \times [R,2R)$, USING THE VALUE $R$).

\begin{theorem} \label{DecompositionTheorem}
    For any 1-separated set $\mathcal{E} \subset \RR^d \times [R,2R)$, we can consider a disjoint union $\mathcal{E} = \bigcup_{k = 1}^\infty \bigcup_{m = 1}^\infty \mathcal{E}_k(2^m)$ with the following properties:
    %
    \begin{itemize}
        \item For each $k$ and $m$, $\mathcal{E}_k(2^m)$ has density type $(2^m,2^k)$.
        \item If $B$ is a ball of radius $\leq 2^m$ containing at least $2^m \text{rad}(B)$ points of $\mathcal{E}_k$, then
        %
        \[ B \cap \mathcal{E}_k \subset \bigcup_{m' \geq m} \mathcal{E}_k(2^{m'}). \]
        \item For each $k$ and $m$, there are disjoint balls $\{ B_i \}$ of radius at most $2^k$, such that
        %
        \[ \sum_i \text{rad}(B) \leq \frac{\#(\mathcal{E}_k)}{u} \]
        %
        such that $\bigcup B_i^*$ covers $\bigcup_{m' \geq m} \mathcal{E}_k(2^{m'})$, where $B_i^*$ denotes the ball with the same center as $B_i$ but 5 times the radius.
    \end{itemize}
\end{theorem}
\begin{proof}
    Vitali Covering.
\end{proof}

Given a sum $F = \sum_{(x,r) \in \mathcal{E}} f_{xr}$, decompose $\mathcal{E}$ as $\mathcal{E}_k(2^m)$, and define $F_{km}$ to be the sum over $\mathcal{E}_k(2^m)$. It follows from the convering argument above that measure of the support of $F_{km}$ is $O(2^{k(d-1)-m} \#(\mathcal{E}_k))$. We define $F_m = \sum_k F_{km}$. To Prove Lemma \ref{lemma3}, it will suffice to prove the following $L^2$ estimate on $F_m$.

\begin{lemma} \label{lemma6}
    Suppose $\mathcal{E}$ is a set with density type $(2^m,2^k)$. Then
    %
    \[ \left\| \sum_{(x,r) \in \mathcal{E}} f_{x,r} \right\|_{L^2(\RR^d)} \lesssim 2^{m/(d-1)} \sqrt{\log(2 + 2^m)} 2^{k(d-1)/2} \cdot \#(\mathcal{E}_k)^{1/2}. \]
\end{lemma}

\begin{proof}[Proof of Lemma \ref{lemma3} from Lemma \ref{lemma6}]
    We have
    %
    \[ \| F_{km} \|_{L^2(\RR^d)} \lesssim 2^{m/(d-1)} \sqrt{\log(2 + 2^m)} 2^{k(d-1)/2} \#(\mathcal{E}_k)^{1/2}. \]
    %
    If we interpolate this bound with the support bound for $F_{km}$, we conclude that for $0 < p \leq 2$,
    %
    \begin{align*}
        \| F_{km} \|_{L^p(\RR^d)} &\leq |\text{Supp}(F_{km})|^{1/p - 1/2} \| F_{km} \|_{L^2(\RR^d)}\\
        &\lesssim ( 2^{(k(d-1)-m)} )^{1/p - 1/2} 2^{m/(d-1)} \sqrt{\log(2 + 2^m)} 2^{k(d-1)/2} \#(\mathcal{E}_k)^{1/2}\\
        &\lesssim 2^{m(1/p_d - 1/p)} \sqrt{\log(2 + 2^m)} \cdot 2^{k(d-1)/p} \#(\mathcal{E}_k)^{1/2}.
    \end{align*}
    %
    where $p_d = 2(d-1)/(d+1)$. This bound is summable in $m$ for $p < p_d$, which enables us to conclude that
    %
    \[ \| F_k \|_{L^p(\RR^d)} \lesssim_p 2^{k(d-1)/p} \#(\mathcal{E}_k)^{1/2}. \]
    %
    NOTE: THIS SEEMS LIKE A TYPO. Thus for $1 \leq p < p_d$, we obtain the bound stated in Lemma \ref{lemma3}.
\end{proof}

Proving \ref{lemma6} is where the weak-orthogonality bounds from Lemma \ref{lemma4} come into play.

\begin{proof}[Proof of Lemma \ref{lemma6}]
    Split the interval $[2^k,2^{k+1}]$ into $\lesssim 2^{(1 - \alpha)k}$ intervals of length $2^{\alpha k}$, for some $\alpha$ to be optimized later. For appropriate integers $a$, let $I_a = [2^k + (a-1) 2^{\alpha k}, 2^k + a 2^{\alpha k}]$. Let $\mathcal{E}_a = \{ (x,r) \in \mathcal{E}: r \in I_a \}$, and write $F = \sum f_{xr}$, and $F_a = \sum_{(x,r) \in \mathcal{E}_a} f_{xr}$. Without loss of generality, splitting up the sum appropriately, we may assume that the set of $a$ such that $\mathcal{E}_a$ is nonempty is 10-separated. We calculate that
    %
    \[ \| F \|_{L^2(\RR^d)}^2 = \sum_a \| F_a \|_{L^2(\RR^d)}^2 + 2 \sum_{a_1 < a_2} |\langle F_{a_1}, F_{a_2} \rangle| \]
    %
    Given $a_1 < a_2$, $(x_1,r_1) \in \mathcal{E}_{a_1}$, and $(x_2,r_2)$ such that $\langle f_{x_1r_1}, f_{x_2r_2} \rangle \neq 0$, then $|x_1 - x_2| \leq 2^{k+2}$. Since $|r_1 - r_2| \leq 2^{k+1}$ follows because $r_1,r_2 \in [2^k,2^{k+1}]$, it follows that $|(x_1,r_1) - (x_2,r_2)| \leq 3 \cdot 2^{k+1}$. For each such pair, since we may assume that $a_2 - a_1 \geq 10$ without loss of generality, it follows that $|r_1 - r_2| \geq 2^{\alpha k}$, and so applying Lemma \ref{lemma4} together with the density property, we conclude that for $d \geq 4$,
    %
    \begin{align*}
        |\langle f_{x_1r_1}, F_{a_2} \rangle| &\leq \sum_{l = 1}^{(1 - \alpha)k + 1} \sum_{2^l 2^{\alpha k} \leq |(x_1,r_1) - (x_2,r_2)| \leq 2^{l+1} 2^{\alpha k}} \langle f_{x_1r_1}, f_{x_2r_2}|\\
        &\lesssim \sum_{l = 1}^{(1 - \alpha)k + 1} (2^m 2^l 2^{\alpha k}) \left( \frac{2^{2k}}{2^l 2^{\alpha k}} \right)^{(d-1)/2}\\
        &\lesssim \sum_{l = 1}^{(1 - \alpha)k + 1} 2^m (2^k)^{ (d-1) - (d-3)/2 \alpha } 2^{-(d-3)/2 \cdot l}\\
        &\lesssim 2^m (2^k)^{(d-1) - (d-3)/2 \alpha}.
    \end{align*}
    %
    Summing over all choices of $x_1$ and $r_1$, we conclude that
    %
    \[ 2 \sum_{a_1 < a_2} |\langle F_{a_1}, F_{a_2} \rangle| \lesssim 2^m (2^k)^{(d-1) - (d-3)/2 \alpha} \#(\mathcal{E}). \]
    %
    On the other hand, TODO
\end{proof}

\chapter{Cladek: Improvements to Radial Multiplier Problem Using Incidence Geometry}

\chapter{Mockenhaupt, Seeger, and Sogge: Exploiting Wave-Equation Periodicity}

The main goal of the paper \emph{Local Smoothing of Fourier Integral Operators and {C}arleson-{S}j\"{o}lin Estimates} is to prove local regularity theorems for a class of Fourier integral operators in $I^\mu(Z,Y;\mathcal{C})$, where $Y$ is a manifold of dimension $n \geq 2$, and $Z$ is a manifold of dimension $n+1$, which naturally arise from the study of wave equations. A consequence of this result will be a local smoothing result for solutions to the wave equation, i.e. that if $2 < p < \infty$, then there is $\delta$ depending on $p$ and $n$, such that if $T: Y \to Y \times \RR$ is the solution operator to the wave equation, and $Y$ is a compact manifold whose geodesics are periodic, then $T$ is continuous from from $L^p_c(Y)$ to $L^p_{\alpha,\text{loc}}(Y \times \RR)$ for $\alpha \leq -(n-1)|1/2 - 1/p| + \delta$. Such a result is called local smoothing, since if we define $Tf(t,x) = T_tf(x)$, then the operator $T_t$ is, for each $t$, a Fourier integral operator of order zero, with canonical relation
%
\[ \mathcal{C}_t = \{ (x,y;\xi,\xi) : x = y + t \widehat{\xi} \}, \]
%
where $\widehat{\xi} = \xi / |\xi|$ is the normalization of $\xi$. Standard results about the regularity of hyperbolic partial differential equations show that each of the operators $T_t$ is continuous from $L^p_c(Y)$ To $L^p_{\alpha,\text{loc}}(Y \times \RR)$ for $\alpha \leq -(n-1)|1/2 - 1/p|$, and that this bound is sharp. Thus $T$ is \emph{smoothing} in the $t$ variable, so that for any $f \in L^p$, the functions $T_t f$ `on average' gain a regularity of $\delta$ over the worst case regularity at each time. The local smoothing conjecture states that this result is true for any $\delta < 1/p$.

The class of Fourier integral operators studied are those satisfying the following condition: as is standard, the canonical relation $\mathcal{C}$ is a conic Lagrangian manifold of dimension $2n + 1$. The fact that $\mathcal{C}$ is Lagrangian implies $\mathcal{C}$ is locally parameterized by $(\nabla_\zeta H(\zeta, \eta), \nabla_\eta H(\zeta, \eta),\zeta,\eta)$, where $H$ is a smooth, real homogeneous function of order one. If we assume $\mathcal{C} \to T^* Y$ is a submersion, then $D_\xi [\nabla_\eta H(\zeta,\eta)]$ has full rank, which implies $D_\eta [\nabla \xi H(\zeta, \eta)] = (D_\xi [\nabla_\eta H(\zeta, \eta)])^T$ has full rank, and thus the projection $\mathcal{C} \to T^* Z$ is an immersion. We make the further assumption that the projection $\mathcal{C} \to Z$ is a submersion, from which it follows that for each $z$ in the image of this projection, the projection of points in $\mathcal{C}$ onto $T^*_z Z$ is a conic hypersurface $\Gamma_z$ of dimension $n$. The final assumption we make is that all principal curvatures of $\Gamma_z$ are non-vanishing.

\begin{remark}
    The projection properties of $\mathcal{C}$ imply that, in $T^* (Z \times Y)$, there exists a smooth phase $\phi$ defined on an open subset of $Z \times T^* Y$, homogeneous in $T^* Y$, such that locally we can write $\mathcal{C}$ as $(z, \nabla_z \phi(z,\eta), \nabla_\eta \phi(z,\eta), \eta)$ for $\eta \neq 0$. Then, working locally on conic sets,
    %
    \[ \Gamma_z = \{ (\nabla_z \phi(z,\eta)) \}, \]
    %
    and the curvature condition becomes that the Hessian $H_{\eta \eta} \langle \nabla_z \phi, \nu \rangle$ has constant rank $n-1$, where $\nu$ is the normal vector to $\Gamma_z$. This is a natural homogeneous analogue of the Carleson-Sj\"{o}lin condition for non-homogeneous oscillatory integral operators, i.e. the Carleson-Sj\"{o}lin condition is allowed to assume $H_{\eta \eta} \phi$ has rank $n$, which cannot be possible in our case, since $\phi$ is homogeneous here. An approach using the analytic interpolation method of Stein or the Strichartz / Fractional Integral approach generalizes the Carleson-Sj\"{o}lin theorem to show that for any smooth, non-homogeneous phase function $\Phi: \RR^{n+1} \times \RR^n \to \RR$, and any compactly supported smooth amplitude $a$ on $\RR^{n+1} \times \RR^n$. Consider the operators
    %
    \[ T_\lambda f(z) = \int a(z,y) e^{2 \pi i \lambda \Phi(z,y)} f(y)\; dy. \]
    %
    If the associated canonical relation $\mathcal{C}$, if $\mathcal{C}$ projects submersively onto $T^* \RR^n$, so that for each $z \in \RR^{n+1}$ in the image of the projection map $\mathcal{C}$, the set $S_z \subset \RR^{n+1}$ obtained from the inverse image of the projection of $\mathcal{C} \to Z$ at $z$ is a $n$ dimensional hypersurface with $k$ non-vanishing curvatures. Then for $1 \leq p \leq 2$,
    %
    \[ \| T_\lambda f \|_{L^q(\RR^{n+1})} \lesssim \lambda^{-(n+1)/q} \| f \|_{L^p(\RR^n)}. \]
    %
    where $q = p^*(1 + 2/k)$.
\end{remark}

\begin{remark}
    We can also see these assumptions as analogues in the framework of cinematic curvature, splitting the $z$ coordinates into `time-like' and 'space-like' parts. Working locally, because $\mathcal{C} \to T^* Y$ is a submersion, we can consider coordinates $z = (x,t)$ so that, with the phase $\phi$ introduced above, $D_x (\nabla_\eta \phi)$ has full rank $n$, and that $\partial_t \phi(x,t,\eta) \neq 0$. Then for each $z = (x,t)$, we can locally write $\partial_t \phi(x,t,\eta) = q(x,t,\nabla_x \phi(x,t,\eta))$, homogeneous in $\eta$, and then
    %
    \[ \mathcal{C} = \{ (x,t,y;\xi,\tau,\eta) : (x,\xi) = \chi_t(y,\eta), \tau = q(x,t,\xi) \}, \]
    %
    where $\chi_t$ is a canonical transformation. Our curvature conditions becomes that $H_{\xi \xi} q$ has full rank $n-1$. This is the cinematic curvature condition introduced by Sogge. %TODO: READ SOGGE, PROPOGATION OF SINGULARITIES AND MAXIMAL FUNCTIONS IN THE PLANE, WHICH INTRODUCES CINEMATIC CURVATURE?
\end{remark}

Under these assumptions, the paper proves that any Fourier integral operator $T$ in $I^{\mu - 1/4}(Z,Y;\mathcal{C})$ maps $L^2_c(Y)$ to $L^q_{\text{loc}}(Z)$ if
%
\[ 2 \left( \frac{n+1}{n-1} \right) \leq q < \infty \quad\text{and}\quad \mu \leq - n (1/2 - 1/q) + 1/q. \]
%
and maps $L^p_c(Y)$ to $L^p_{\text{loc}}(Z)$ if
%
\[ p > 2 \quad\text{and}\quad \mu \leq -(n-1)(1/2 - 1/p) + \delta(p,n). \]
%
If we introduce time and space variables locally as in the remark above, any operator in $I^{\mu - 1/4}(Z,Y;\mathcal{C})$ can be written locally as a finite sum of operators of the form
%
\[ Tf(x) = \int_{-\infty}^\infty T_t f(x), \]
%
where
%
\[ T_t f(x) = \int a(t,x,\eta) e^{2 \pi i \phi(x,t,y,\eta)} f(y)\; dy\; d\eta. \]
%
is a Fourier integral operator whose canonical relation is a locally a canonical graph, then the general theory implies that each of the maps $T_t$ maps $L^2_c(Y)$ to $L^q_{\text{loc}}(X)$ if
%
\[ 2 \leq q \leq \infty \quad\text{and}\quad \mu \leq -n(1/2 - 1/q) \]
%
so that here we get local smoothing of order $1/q$, and also maps $L^p_c(Y)$ to $L^p_{\text{loc}}(X)$ if
%
\[ 1 < p < \infty \quad\text{and}\quad \mu \leq -(n-1)|1/p - 1/2| \]
%
so we get $\delta(p,n)$ smoothing. A consequence of the smoothing, via Sobolev embedding, is a maximal theorem result for the operator $T_t$, i.e. that for any finite interval $I$, the operator
%
\[ Mf = \sup_{t \in I} |T_t f| \]
%
maps $L^p_c(Y)$ to $L^p_{\text{loc}}(X)$ if $\mu < -(n-1)(1/2 - 1/p) - (1/p - \delta(p,n))$. If the local smoothing conjecture held, we would conclude that, except at the endpoint $T^*$ has the same $L^p_c(Y)$ to $L^p_{\text{loc}}(X)$ mapping properties as each of the operators $T_t$. We also get square function estimates, such that for any finite interval $I$, if we consider
%
\[ Sf(x) = \left( \int_I |T_t f(x)|^2\; dt \right)^{1/2}, \]
%
then for
%
\[ 2 \frac{n+1}{n-1} \leq q < \infty \quad\text{and}\quad \mu \leq -n(1/2 - 1/q) + 1/2, \]
%
the operator $S$ is bounded from $L^2_c(Y)$ to $L^q_{\text{loc}}(X)$.

Our main reason to focus on this paper is the results of the latter half of the paper applying these techniques to radial multipliers on compact manifolds with periodic geodesics. Thus we consider a compact Riemannian manifold $M$, such that the geodesic flow is periodic with minimal period $2 \pi \cdot \Pi$. We consider $m \in L^\infty(\RR)$, such that $\sup_{s > 0} \| \beta \cdot \text{Dil}_s m \|_{L^2_\alpha(\RR)} = A_\alpha$ is finite for some $\alpha > 1/2$ and some $\beta \in C_c^\infty(\RR)$. We define a `radial multiplier' operator
%
\[ Tf = \sum_\lambda m(\lambda) E_\lambda f \]
%
where $E_\lambda$ is the projection of $f$ onto the space of eigenfunctions for the operator $\sqrt{-\Delta}$ on $M$ with eigenvalue $\lambda$. We can also write this operator as $m(\sqrt{-\Delta})$. Then the wave propogation operator $e^{2 \pi i t \sqrt{-\Delta}}$ is periodic of period $\Pi$. The Weyl formula tells us that the number of eigenvalues of $\sqrt{-\Delta}$ which are smaller than $\lambda$ is equal to $V(M) \cdot \lambda^n + O(\lambda^{n-1})$.

\begin{theorem}
    Let $m \in L^2_\alpha(\RR)$ be supported on $(1,2)$, and assume $\alpha > 1/2$, then for $2 \leq p \leq 4$, $f \in L^p(M)$, and for any integer $k$,
    %
    \[ \left\| \sup_{2^k \leq \tau \leq 2^{k+1}} |\text{Dil}_\tau m(\sqrt{-\Delta}) f| \right\|_{L^p(M)} \lesssim_\alpha \| m \|_{L^2_\alpha(M)} \| f \|_{L^p(M)}. \]
\end{theorem}
\begin{proof}
    To understand the radial multipliers we apply the Fourier transform, writing
    %
    \[ T_\tau f = (\text{Dil}_\tau m)(\sqrt{-\Delta}) f = m(\sqrt{-\Delta} / \tau) f = \int_{-\infty}^\infty \tau \widehat{m}(t \tau) e^{2 \pi i t \sqrt{-\Delta}} f\; dt. \]
    %
    If we define $\beta \in C_c^\infty((1/2,8))$ such that $\beta(s) = 1$ for $1 \leq s \leq 4$, and set $L_k f = \text{Dil}_{2^k} \beta(\sqrt{-\Delta}) f$, then for $2^k \leq \tau \leq 2^{k+1}$
    %
    \[ T_\tau f = (\text{Dil}_\tau m)(\sqrt{-\Delta}) f = (\text{Dil}_\tau m \cdot \text{Dil}_{2^k} \beta)(\sqrt{-\Delta}) = T_\tau L_k f. \]
    %
    so Cauchy-Schwartz implies that
    %
    \begin{align*}
        |T_\tau f(x)| &= \left| \int_{-\infty}^\infty \tau \widehat{m}(\tau) e^{2 \pi i t \sqrt{-\Delta}} L_k f(x)\; dt \right|\\
        &\leq \| m \|_{L^2_\alpha(M)} \left( \int_{-\infty}^\infty \frac{\tau}{(1 + |t \tau|^2)^\alpha} |e^{2 \pi i t \sqrt{-\Delta}} L_k f(x)|^2 \right)^{1/2}\\
        &\leq \| m \|_{L^2_\alpha(M)} \left( \int_{-\infty}^\infty \frac{2^k}{(1 + |2^k t|^2)^\alpha} |e^{2 \pi i t \sqrt{-\Delta}} L_k f(x)|^2 \right)^{1/2}
    \end{align*}
    %
    Because of periodicity, if we set $w_k(t) = 2^k / (1 + |2^k t|^2)^\alpha$, it suffices to prove that for $\alpha > 1/2$,
    %
    \[ \left\| \left( \int_0^\Pi w_k(t) |e^{2 \pi i t \sqrt{-\Delta}} L_k f(x)|^2\; dt \right)^{1/2} \right\|_{L^p(M)} \lesssim_{\alpha,p} \| f \|_{L^p(M)}. \]
    %
    This is a weighted combination of the wave propogators, roughly speaking, assigning weight $2^k$ for $t \lesssim 1/2^k$, and assigning weight $1/t$ to values $t \gtrsim 1/2^k$.

    For a fixed $0 < \delta$, we can split this using a partition of unity into a region where $t \gtrsim \delta$ and a region where $t \lesssim \delta$, where $\delta$ is independent of $k$. For each $t$, the wave propogation $e^{2 \pi i t \sqrt{-\Delta}}$ is a Fourier integral operator of order zero (we have an explicit formula for small $t$, and the composition calculus for Fourier integral operators can then be used to give a representation of the propogation operators for all times $t$, such that the symbols of these operators are locally uniformly bounded in $S^0$). Thus the square function estimate above can be applied in the region where $t \gtrsim \delta$, because the weighted square integral above has weight $O_\delta(1)$ uniformly in $k$.

    Next, we move onto the region $t \lesssim 1/2^k$. The symbol of the operator $e^{2 \pi i t \sqrt{-\Delta}}$

    Finally we move onto the region $1/2^k \lesssim t \lesssim \delta$. On this region we have $w_k(t) \sim 1/t$, which hints we should try using dyadic estimates. In particular, suppose that for $\gamma \leq \delta$, we have a family of dyadic estimates of the form
    %
    \[ \left\| \left( \int_\gamma^{2\gamma} |e^{2 \pi i t \sqrt{-\Delta}} L_k f|^2\; dt \right)^{1/2} \right\|_{L^p(M)} \lesssim \gamma^{1/2} (1 + \gamma 2^k)^\varepsilon \cdot \| f \|_{L^p(M)}. \]
    %
    Summing over the $O(k)$ dyadic numbers between $1/2^k$ and $\delta$ gives
    %
    \[ \left\| \left( \int_{1/2^k \lesssim t \lesssim \delta} |e^{2 \pi i t \sqrt{-\Delta}} L_k f|^2\; \frac{dt}{t} \right)^{1/2} \right\|_{L^p(M)} \lesssim 2^{\varepsilon k} \| f \|_{L^p(M)} \]






    If we were able to obtain this inequality for some $\varepsilon > 0$, then we could bound


     that for all $0 < \gamma < \Pi/2$


    If we localize near $t \lesssim 1/2^k$ by multiplying by $\phi(2^k t)$ for some compactly supported smooth $\phi$ supported on $|t| \lesssim 1$, then for $t$ on the support of $\phi(2^k t)$ we have a weight proportional to $2^k$, and rescaling shows that it suffices to bound the quantities
    %
    \[ \left\| \left( \int \phi(t) |e^{2 \pi i (t/2^k) \sqrt{-\Delta}} L_k f(x)|^2\; dt \right)^{1/2} \right\| \]

     the family of functions
    %
    \[ \left\| \left( \int |\phi(t) e^{2 \pi i (t / 2^k) \sqrt{-\Delta}} L_k f(x)|^2\; Dt \right)^{1/2} \right\|_{L^p_x} \lesssim \sup \| e^{2 \pi i (t / 2^k) \sqrt{-\Delta}} L_k f \|_{L^p_x} \]


    $a_k(t) = 2^{-k/2} \widehat{\phi}(t/2^k) \beta(\tau/2^k)$

    it suffices to uniformly bound quantities of the form
    %
    \[ \left\| \left( \int 2^k \phi(2^k t) |e^{2 \pi i \sqrt{-\Delta}} L_k f(x)|^2\; dt \right)^{1/2} \right\|_{L^p(M)} \lesssim_{\alpha,p} \| f \|_{L^p(M)} \]
    %
    We now apply a dyadic decomposition to deal with the smaller values of $t$. Let us assume for simplicity of notation that $\delta < 1$, and then consider a partition of unity $1 = \sum_{j = 1}^\infty \phi(2^j t)$ for $0 \leq t \leq 1$, and such that $\phi$ is localized near $1/4 \leq t \leq 2$, then our goal is to bound the quantities
    %
    \[ \left\| \left( \int_{-\infty}^\infty \phi(2^j t) \frac{2^k}{(1 + |2^k t|^2)^\alpha} |A_t L_k f(x)|^2\; dt \right)^{1/2} \right\|_{L^p(M)}, \]
    %
    which are each proportional to
    \[ s \]
\end{proof}







\chapter{Lee and Seeger: Decomposition Arguments For Estimating Fourier Integral Operators}

In the paper \emph{Lebesgue Space Estimates For a Class of Fourier Integral Operators Associated With Wave Propogation}, Lee and Seeger prove a variable coefficient version of the result of Heo, Nazarov, and Seeger, i.e. generalizing their result from proving results about the boundedness of radial Fourier multipliers on $\RR^n$ to certain Fourier integral operators satisfying the cinematic curvature condition. Let $T = m(-\sqrt{\Delta})$ be a radial multiplier on $\RR^n$, i.e. such that
%
\[ T f(x) = \int m(|\xi|) e^{2 \pi i \xi \cdot (x - y)} f(y)\; d\xi\; dy. \]
%
If $m$ is a symbol, then we can interpret $T$ directly as a Pseudodifferential Operator. But Heo, Nazarov, and Seeger's result discuss families of multipliers $m$ that are not even necessarily smooth, but do satisfy certain integrability conditions. To fix this, we assume a priori that we have applyied a decomposition argument, so we may assume $m$ is compactly supported away from the origin. Then (by Paley-Wiener) $\widehat{m}$ is a smooth symbol of some finite order satisfying some integrability properties, which indicates how we might apply the theory of Fourier integral operators, i.e. by taking the Fourier transform of $m$, we get that
%
\[ Tf(x) = \int \widehat{m}(\rho) e^{2 \pi i [\rho |\xi| + \xi \cdot (x-y)]} f(y)\; d\rho\; d\xi\; dy. \]
%
This is `almost' a Fourier integral operator, except the phase is not smooth unless $\widehat{m}$ is supported away from the origin (fixed by a decomposition argument), and the phase is non-homogeneous. To fix the non-homogeneity, we just isolate the operator in $\rho$, writing
%
\[ Tf(x) = \int_{-\infty}^\infty \widehat{m}(\rho) T_\rho f(x)\; d\rho, \]
%
where
%
\[ T_\rho f(x) = e^{2 \pi i \rho \sqrt{-\Delta}} f(x) = \int e^{2 \pi i [\rho |\xi| + \xi \cdot (x - y)]} f(y)\; d\xi\; dy \]
%
is the propogation operator for the half-wave equation $\partial_t u = \sqrt{-\Delta} \cdot u$. It has phase $\phi(x,y,\xi) = \rho |\xi| + \xi \cdot (x - y)$, and thus we have a stationary frequency value when $x = y - \rho \widehat{\xi}$, where $\widehat{\xi} = \xi / |\xi|$ is the normalization of $\xi$. This has canonical relation



\chapter{Our Approach}

Let us now try and prove certain special cases of the radial multiplier conjecture on the sphere $S^n$. Thus we fix a symbol $h$, and study operators of the form
%
\[ T_R = h \left( \sqrt{-\Delta} / R \right) = \sum h(\lambda / R) E_\lambda, \]
%
where $E_\lambda$ is the projection operator onto the eigenspace corresponding to the eigenvalue $\lambda$. In particular, we wish to characterize the boundedness properties of the operators $T_{h,R}$, in terms of appropriate control of the Fourier transform of the function $h$. More precisely, we fix an exponent $p$, and assume that the quantity
%
\[ A_p(h) = \left( \int_{-\infty}^\infty |\widehat{h}(t)|^p (1 + |t|)^{(d-1)(1 - p/2)}\; dt \right)^{1/q} \]
%
is finite, which is a necessary condition for the multiplier $h(\sqrt{-\Delta})$ to be bounded on $\RR^d$, and thus by the result of Mityagin, necessary for the family of operators $\{ T_R : R > 0 \}$ to be uniformly bounded in $R$.

To exploit the fact that $A_p(h)$ is finite, we apply the Fourier transform to the sum defining $T_R$, writing $T_R f$ as the vector valued integral
%
\[ T_R f = \int_{-\infty}^\infty R \widehat{h}(R t) (e^{2  \pi i t \sqrt{-\Delta}} f). \]
%
Now H\"{o}lder's inequality, together with the fact that the wave propogators $\{ e^{2 \pi i t \sqrt{-\Delta}} \}$ is periodic on $S^n$ with periodicity $1$, and the fact that $1 < p < 2d/(d+1)$, we write
%
\begin{align*}
    & \left| \int_{-\infty}^\infty R \widehat{h}(Rt) e^{2 \pi i t \sqrt{-\Delta}} f\; dt \right|\\
    &\quad\quad\leq R \left( \int_{-\infty}^\infty |\widehat{h}(Rt)|^p (1 + |Rt|)^{(d-1)(1 - p/2)}\; dt \right)^{1/p}\\
    &\quad\quad\quad\quad\quad\quad\quad\left( \int_{-\infty}^\infty |e^{2 \pi i t \sqrt{-\Delta}} f|^{p^*} (1 + |Rt|)^{- (d-1)\frac{(2 - p)}{2(p-1)}} \right)^{1/p^*}\\
    &\quad\quad\leq R^{1 - 1/p} A_p(h) \left( \sum_{k = -\infty}^\infty \int_0^1 |e^{2 \pi i t \sqrt{-\Delta}} f|^{p^*} (1 + |R(t + k)|)^{- (d-1)\frac{(2 - p)}{2(p-1)}}\; dt \right)^{1/p^*}\\
    &\quad\quad\lesssim_p A_p(h) \left( \int_0^1 |e^{2 \pi i t \sqrt{-\Delta}} f|^{p^*}\; dt \right)^{1/p^*}.
\end{align*}
%
The small time parameterix for the half-wave operator, combined with the composition calculus of Fourier integral operators, allows us to write
%
\[ e^{2 \pi i t \sqrt{-\Delta}} f = T_t f + S^\infty_t f, \]
%
where $S^\infty_t$ is a \emph{smoothing operator}, i.e. an integral operator with
%
\[ S^\infty_t f(x) = \int K(t,x,y) f(y)\; dy, \]
%
where $K \in C^\infty([0,1] \times S^n \times S^n)$, and where
%
\[ T_t f(x) = \int_{\RR^n} a(t,x,y,\xi) e^{2 \pi i \Phi(x,y,\xi)} f(y)\; d\xi\; dy \]
%
for some symbol $a \in S^0$, and some symbol $\Phi \in S^1$ satisfying
%
\[ \Phi(x,y,\xi) = (x - y) \cdot \xi + t g_y(\xi,\xi) + O(|x - y|^2 |\xi|), \]
%
where $g$ is the Riemmanian metric of $S^n$. In particular, if we work with the coordinates $z_{\pm}$ in the strict upper and lower hemispheres given by
%
\[ z_{\pm}^{-1}(t_1,\dots,t_n) = (t_1,\dots,t_n, \pm (1-t_1^2 - \dots - t_n^2)^{1/2}), \]
%
then in coordinates we have
%
\[ g_y(\xi,\xi) = |\xi|^2 - (1 - |y|^2)^{-1/2} |y \cdot \xi|^2, \]
%
and so in these coordinates we have the explicit form
%
\[ \Phi(x,y,\xi) = (x - y) \cdot \xi + t |\xi|^2 - (1 - |y|^2)^{-1/2} |y \cdot \xi|^2 + O(|x - y|^2 |\xi|). \]

Let us begin with the qualitative assumption that $h$ is compactly supported. Then, by breaking things up into a finite sum, we may assume that $h$ is supported on $[1/2,2]$. Fix a function $\chi \in C_c^\infty(\RR)$ equal to one on $[1/2,2]$, and vanishing outside of $[1/4,4]$. Write
%
\[ P_R = \chi \left( \sqrt{-\Delta} / R \right) = \sum \chi(\lambda / R) E_\lambda. \]
%
Then for any function $f \in C^\infty(S^n)$, $T_R f = T_R \{ P_R f \}$. Thus when bounding the behaviour of the operator $T_R$, we may assume inputs are linear combinations of eigenfunctions to $\sqrt{-\Delta}$ with eigenvalues $\lambda \sim R$.

%- Nonlinearity of Fourier integral operator, does it cause problems?
%- Why can we get half-wave equation with symbol only depending on x.
%- Is Mockenhaupt Seeger Sogge definition of order agree with Sogge's definition of order?
%- Doesn't 6.2 only apply in two dimensions?
%- Does the formula I mentioned matter?
%- Is non-smooth phase a problem in the theory of Fourier integral operators?
%- Is normalization in Sogge the same as Hormander's normalization?
%- To investigate: Normal Coordinates and relation to canonical relation for the wave equation
%- Riemann curvature tensor: Seems very related to wave equation.
%- Review Fredholm Alternative

\bibliographystyle{plain}
\bibliography{RadialMultipliers}

\end{document}