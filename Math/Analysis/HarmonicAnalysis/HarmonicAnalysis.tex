\documentclass[12pt, dvipsnames]{report}

\usepackage{amsmath}
\usepackage{algorithm}
%\usepackage{algorithmic}
\usepackage[noend]{algpseudocode}

\usepackage{amsmath}
\usepackage{amssymb}
\usepackage{amsthm}
\usepackage{amsopn}

\usepackage{kpfonts}

\usepackage{graphicx}

% Probably don't need this on notes anymore
%\usepackage{kbordermatrix}

% Standard tool for drawing diagrams.
\usepackage{tikz}
\usepackage{tkz-berge}
\usepackage{tikz-cd}
\usepackage{tkz-graph}

\usepackage{comment}

%
\usepackage{multicol}

%
\usepackage{framed}

%
\usepackage{mathtools}

%
\usepackage{float}

%
\usepackage{subfig}

%
\usepackage{wrapfig}

%
\let\savewideparen\wideparen
\let\wideparen\relax
\usepackage{mathabx}
\let\wideparen\savewideparen

% Used for generating `enlightening quotes'
\usepackage{epigraph}

% Forget what this is used for :P
\usepackage[utf8]{inputenc}

% Used for generating quotes.
\usepackage{csquotes}

% Allows what to generate links inside
% generated pdf files
\usepackage{hyperref}

% Allows one to customize theorem
% environments in mathematical proofs.
\usepackage{thmtools}

% Gives access to a proof
\usepackage{lplfitch}

% I forget what this is for.
\usepackage{accents}

% A package for drawing simple trees,
% as a substitute for unnesacary TIKZ code
\usepackage{qtree}

% Enables sequent calculus proofs
\usepackage{ebproof}

% For braket notation
\usepackage{braket}

% To change line spacing when using mathematical notations which require some height!
\usepackage{setspace}

%\usepackage[dvipsnames]{xcolor}

\usepackage{float}

% For block commenting
\usepackage{comment}




\setlength\epigraphwidth{8cm}

\usetikzlibrary{arrows, petri, topaths, decorations.markings}

% So you can do calculations in coordinate specifications
\usetikzlibrary{calc}
\usetikzlibrary{angles}

\theoremstyle{plain}
\newtheorem{theorem}{Theorem}[chapter]
\newtheorem{axiom}{Axiom}
\newtheorem{lemma}[theorem]{Lemma}
\newtheorem{corollary}[theorem]{Corollary}
\newtheorem{prop}[theorem]{Proposition}
\newtheorem{exercise}{Exercise}[chapter]
\newtheorem{fact}{Fact}[chapter]

\newtheorem*{example}{Example}
\newtheorem*{proof*}{Proof}

\theoremstyle{remark}
\newtheorem*{exposition}{Exposition}
\newtheorem*{remark}{Remark}
\newtheorem*{remarks}{Remarks}

\theoremstyle{definition}
\newtheorem*{defi}{Definition}

\usepackage{hyperref}
\hypersetup{
    colorlinks = true,
    linkcolor = black,
}

\usepackage{textgreek}

\makeatletter
\renewcommand*\env@matrix[1][*\c@MaxMatrixCols c]{%
  \hskip -\arraycolsep
  \let\@ifnextchar\new@ifnextchar
  \array{#1}}
\makeatother

\renewcommand*\contentsname{\hfill Table Of Contents \hfill}

\newcommand{\optionalsection}[1]{\section[* #1]{(Important) #1}}
\newcommand{\deriv}[3]{\left. \frac{\partial #1}{\partial #2} \right|_{#3}} % partial derivative involving numerator and denominator.
\newcommand{\lcm}{\operatorname{lcm}}
\newcommand{\im}{\operatorname{im}}
\newcommand{\bint}{\mathbf{Z}}
\newcommand{\gen}[1]{\langle #1 \rangle}

\newcommand{\End}{\operatorname{End}}
\newcommand{\Mor}{\operatorname{Mor}}
\newcommand{\Id}{\operatorname{id}}
\newcommand{\visspace}{\text{\textvisiblespace}}
\newcommand{\Gal}{\text{Gal}}

\newcommand{\xor}{\oplus}
\newcommand{\ft}{\wedge}
\newcommand{\ift}{\vee}

\newcommand{\prob}{\mathbf{P}}
\newcommand{\expect}{\mathbf{E}}
\DeclareMathOperator{\Var}{\mathbf{V}}
\newcommand{\Ber}{\text{Ber}}
\newcommand{\Bin}{\text{Bin}}

%\newcommand{\widecheck}[1]{{#1}^{\ft}}

\DeclareMathOperator{\diam}{\text{diam}}

\DeclareMathOperator{\QQ}{\mathbf{Q}}
\DeclareMathOperator{\ZZ}{\mathbf{Z}}
\DeclareMathOperator{\RR}{\mathbf{R}}
\DeclareMathOperator{\HH}{\mathbf{H}}
\DeclareMathOperator{\CC}{\mathbf{C}}
\DeclareMathOperator{\AB}{\mathbf{A}}
\DeclareMathOperator{\PP}{\mathbf{P}}
\DeclareMathOperator{\MM}{\mathbf{M}}
\DeclareMathOperator{\VV}{\mathbf{V}}
\DeclareMathOperator{\TT}{\mathbf{T}}
\DeclareMathOperator{\LL}{\mathcal{L}}
\DeclareMathOperator{\EE}{\mathbf{E}}
\DeclareMathOperator{\NN}{\mathbf{N}}
\DeclareMathOperator{\DQ}{\mathcal{Q}}
\DeclareMathOperator{\IA}{\mathfrak{a}}
\DeclareMathOperator{\IB}{\mathfrak{b}}
\DeclareMathOperator{\IC}{\mathfrak{c}}
\DeclareMathOperator{\IP}{\mathfrak{p}}
\DeclareMathOperator{\IQ}{\mathfrak{q}}
\DeclareMathOperator{\IM}{\mathfrak{m}}
\DeclareMathOperator{\IN}{\mathfrak{n}}
\DeclareMathOperator{\IK}{\mathfrak{k}}
\DeclareMathOperator{\ord}{\text{ord}}
\DeclareMathOperator{\Ker}{\textsf{Ker}}
\DeclareMathOperator{\Coker}{\textsf{Coker}}
\DeclareMathOperator{\emphcoker}{\emph{coker}}
\DeclareMathOperator{\pp}{\partial}
\DeclareMathOperator{\tr}{\text{tr}}

\DeclareMathOperator{\supp}{\text{supp}}

\DeclareMathOperator{\codim}{\text{codim}}

\DeclareMathOperator{\minkdim}{\dim_{\mathbf{M}}}
\DeclareMathOperator{\hausdim}{\dim_{\mathbf{H}}}
\DeclareMathOperator{\lowminkdim}{\underline{\dim}_{\mathbf{M}}}
\DeclareMathOperator{\upminkdim}{\overline{\dim}_{\mathbf{M}}}
\DeclareMathOperator{\lhdim}{\underline{\dim}_{\mathbf{M}}}
\DeclareMathOperator{\lmbdim}{\underline{\dim}_{\mathbf{MB}}}
\DeclareMathOperator{\packdim}{\text{dim}_{\mathbf{P}}}
\DeclareMathOperator{\fordim}{\dim_{\mathbf{F}}}

\DeclareMathOperator*{\argmax}{arg\,max}
\DeclareMathOperator*{\argmin}{arg\,min}

\DeclareMathOperator{\ssm}{\smallsetminus}

\title{Harmonic Analysis}
\author{Jacob Denson}


\makeatletter
\renewcommand*\l@section{\@dottedtocline{1}{1.5em}{3em}}
\makeatother

\begin{document}

\pagenumbering{gobble}
\maketitle
\tableofcontents
\pagenumbering{arabic}

%% The following is a directive for TeXShop to indicate the main file
%%!TEX root = HarmonicAnalysis.tex

\part{Classical Fourier Analysis}

Deep mathematical knowledge often arises hand in hand with the characterization of symmetry. Nowhere is this more clear than in the foundations of harmonic analysis, where we attempt to understand mathematical `signals' by the `frequencies' from which they are composed. In the mid 18th century, problems in mathematical physics led D. Bernoulli, D'Alembert, Lagrange, and Euler to consider periodic functions representable as a trigonometric series
%
\[ f(t) = A + \sum_{m = 1}^\infty B_n \cos(2 \pi mt) + C_n \sin(2 \pi mt). \]
%
In his book, Th\'{e}orie Analytique de la Chaleur, published in 1811, Joseph Fourier had the audacity to announce that {\it all} functions were representable in this form, and used it to solve linear partial differential equations in physics. His conviction is the reason the classical theory of harmonic analysis is often named Fourier analysis, where we analyze the degree to which Fourier's proclamation holds, as well as it's paired statement on the real line, that a function $f$ on the real line can be written as
%
\[ f(t) = \int_{-\infty}^\infty A(\xi) \cos(2 \pi \xi t) + B(\xi) \sin(2 \pi\xi t)\; d\xi. \]
%
for some functions $A$ and $B$.

In the 1820s, Poisson, Cauchy, and Dirichlet all attempted to form rigorous proofs that `Fourier summation' holds for all functions. Their work is responsible for most of the modern subject of analysis we know today. And in order to interpret the validity of Fourier summation, we will need to utilize all the convergence techniques developed during this time. Under pointwise convergence, the representation of a function by Fourier series need not be unique. Uniform convergence is useful, but a function is not uniformly summable for all functions, even if they are continuous! This means we must introduce more subtle methods to measure convergence.

\chapter{Introduction}

The classic oscillatory functions are the trigonometric ones, given by
%
\[ f(t) = A \cos(st) + B \sin(st) = C \cos(st + \phi). \]
%
The value $\phi$ is the \emph{phase} of the oscillation, $C$ is the \emph{amplitude}, and $s/2\pi$ is the \emph{frequency} of the oscillation. Fourier analysis is the topic devoted to studying the representation of a function as an analytical combination of these functions. In the discrete, periodic setting, we fix a periodic function $f: \RR \to \CC$ (unless otherwise noted, by periodic we always mean $1$ periodic, i.e. a function such that $f(x + 1) = f(x)$ for all $x \in \RR$), and try and find coefficients $\{ A_m \}$, $\{ B_m \}$, and $C$ such that
%
\[ f(t) \sim C + \sum_{m = 1}^\infty A_m \cos(2 \pi mt) + B_m \sin(2 \pi mt). \]
%
In the continuous setting, we fix a function $f: \mathbf{R} \to \mathbf{C}$, trying to find values $A(s)$, $B(s)$, and $C$ such that
%
\[ f(t) \sim C + \int_0^\infty A(s) \cos(2 \pi st) + B(s) \sin(2 \pi st)\; ds. \]
%
The main contribution of Fourier was a method to formally find a reliable choice of $A(s)$ and $B(s)$, $A_m$ and $B_m$, which represents $f$. This choice is given by the \emph{Fourier transform} of $f$ in the continuous case, and the \emph{Fourier series} in the discrete case.

\section{Obtaining the Fourier Coefficients}

A \emph{formal trigonometric series} is a formal sum of the form
%
\[ C + \sum_{m = 1}^\infty A_m \cos(2\pi mt) + B_m \sin(2\pi mt). \]
%
Our goal is to find $\{ A_m \}$, $\{ B_m \}$, and $C$ which `represents' a given function $f$. In particular, we say a periodic function $f$ \emph{admits a trigonometric expansion} if there is a series such that for each $t \in \mathbf{R}$,
%
\[ f(t) = C + \sum_{m = 1}^\infty A_m \cos(2 \pi mt) + B_m \sin(2 \pi mt). \]
%
It is a \emph{very difficult question} to characterize which functions $f$ admit a trigonometric expansion. Nonetheless, using the method of Fourier, we can associate a formal trigonometric series with every function, known as the \emph{Fourier series}. Later, we will show all differentiable periodic functions exhibit a trigonometric expansion in this Fourier series.

\section{Orthogonality}

The key technique Fourier realized could be used to come up with a canonical trigonometric series for a function is \emph{orthogonality}. Note that the various frequencies of sine functions are orthogonal to one another, in the sense that
%
\[ \int_0^1 \sin(2 \pi mt) \sin(2\pi nt) = \int_0^1 \cos(2 \pi mt) \cos(2 \pi nt) = \begin{cases} 0 & : m \neq n, \\ 1/2 & : m = n, \end{cases} \]
%
and for any $m,n \in \ZZ$,
%
\[ \int_0^1 \sin(2 \pi mt) \cos(2 \pi nt) = 0. \]
%
This means that for a finite trigonometric sum
%
\[ f(t) = C + \sum_{m = 1}^N A_m \cos(2 \pi mt) + B_m \sin(2 \pi mt), \]
%
we have
%
\[ C = \int_0^1 f(t)\; dt, \]
\[ A_m = 2 \int_0^1 f(t) \cos(2 \pi mt)\; dt, \quad\text{and}\quad B_m = 2 \int_{-\pi}^\pi f(t) \sin(2 \pi mt)\; dt. \]
%
We note that these values may still be defined even if $f$ is not a trigonometric polynomial. Thus given \emph{any} periodic function $f$, a reasonable candidate for the coefficients is given by the values $A_m$, $B_m$, and $C$ above.

There is an additional choice of oscillatory functions, which replaces the sine and cosine with a single family of trigonometric functions. For $\xi \in \RR$, we let $e_\xi(t) = e^{2 \pi \xi i t}$. For each integer $m$, $e_m$ is periodic with period 1. Applying orthogonality again, we find
%
\[ \int_0^1 e_n(t) \overline{e_m(t)}\; dt = \int_0^1 e_{n-m}(t) = \begin{cases} 0 & : m \neq n, \\ 1 & : m = n. \end{cases}  \]
%
Thus a natural choice of an expansion
%
\[ f(t) \sim \sum_{m \in \mathbf{Z}} C_m e^{2 \pi mit}, \]
%
is given by
%
\[ C_m = \int_0^1 f(t) \overline{e_m(t)}\; dt. \]
%
Euler's formula $e^{mit} = \cos(mt) + i \sin(mt)$ shows this is the same as the Fourier expansion in sines and cosines. Thus the values $\{ A_m, B_m, C : m \geq 0 \}$ can be recovered from the values of $\{ C_m : m \in \mathbf{Z} \}$. Because of it's elegance, unifying the three families of coefficients, the expansion by complex exponentials is the most standard in modern Fourier analysis.

To summarize, we have shown a periodic function $f: \mathbf{R} \to \mathbf{C}$ gives rise to a formal trigonometric series
%
\[ \sum_{m \in \mathbf{Z}} C_m e_m(t). \]
%
This is the \emph{Fourier series} of $f$. Because we will be concentrating on the Fourier series of a function, it is worth reserving a particular notation for them. Given a periodic function $f$, and an integer $m \in \ZZ$, we set
%
\[ \widehat{f}(m) = \int_0^1 f(t) \overline{e_m(t)}\; dt. \]
%
The Fourier series representation in terms of complex exponentials will be our choice throughout the rest of these notes. No deep knowledge of the complex numbers is used here. For most purposes, the exponential notation is just a simple way to represent sums of sines and cosines.

\section{The Fourier Transform}

For a general function $f: \mathbf{R} \to \mathbf{C}$, we cannot rely \emph{just} on orthogonality, because the functions $\sin(2 \pi mx)$ are not integrable on the entirety of $\mathbf{R}$, and therefore cannot be integrated against one another. Nonetheless, we can consider the functions $g_N: [0,1] \to \mathbf{C}$ by setting $g_N(s) = f(N(s-1/2))$. Then for $|t| \leq N/2$, we can apply the usual Fourier series to conclude
%
\begin{align*}
    f(t) &= g_N(t/N+1/2)\\
    &\sim \sum_{m \in \mathbf{Z}} \widehat{g_N}(m) e^{2 \pi m i (t/N + 1/2)}\\
    &=  \sum_{m \in \mathbf{Z}} (-1)^m \left( \int_0^1 f(N(s-1/2)) e^{-2 \pi mis}\; ds \right) e^{2 \pi (m/N) it}\\
    &= \sum_{m \in \mathbf{Z}} \frac{1}{N} \left( \int_{-N/2}^{N/2} f(s) e^{-2\pi (m/N) i s}\; ds \right) e^{(m/N)it}.
\end{align*}
% u = N(s - 1/2)
% du = Nds
%
If we take $N \to \infty$, the exterior sum operates like a Riemann sum, so we might expect
%
\[ f(t) \sim \int_{-\infty}^\infty \left( \int_{-\infty}^\infty f(s) e^{-2 \pi \xi is}\; ds \right) e^{2 \pi \xi i t}\; d\xi. \]
%
The interior integral defines the \emph{Fourier transform} of the function $f$, given for each $\xi \in \RR$ as
%
\[ \widehat{f}(\xi) = \int_{-\infty}^\infty f(s) e^{- 2 \pi \xi is}\; ds. \]
%
Thus the resultant \emph{Fourier inversion formula} takes the form
%
\[ f(t) \sim \int_{-\infty}^\infty \widehat{f}(\xi) e^{2 \pi \xi i s}\; ds. \]
%
As the \emph{limit} of a discrete series defined in terms of orthogonality, the Fourier transform possesses many of the same properties at the Fourier series. But the limit does cause technical issues which are not present in the case of Fourier series, and so we begin by concentrating on the Fourier series.

%
%\begin{example}
%    This method can be used to find all harmonic functions $f$ on a rectangle $[0,\pi] \times [0,1]$, such that $f(0,y) = f(\pi,y) = 0$. Let us first attempt to find all separable solutions $f(x,y) = u(x) v(y)$. Then the equations defining harmonic functions tell us that
%    %
%    \[ u''v + v''u = 0 \]
%    %
%    or
%    %
%    \[ \frac{u''}{u} = - \frac{v''}{v} = - \lambda^2 \]
%    %
%    (we assume the constant factor is negative, since the constraints on $u$ would force $f$ to be trivial otherwise). Then we have
%    %
%    \[ u'' = - \lambda^2 u \]
%    %
%    so $u(x) = A \cos(\lambda x) + B \sin(\lambda x)$. The constraints that $u(0) = u(\pi) = 0$ force $A = 0$, and $\lambda \in \mathbf{Z}$. We may similarily solve the equation
%    %
%    \[ v'' = \lambda^2 v \]
%    %
%    to conclude $v(y) = M e^{\lambda y} + N e^{- \lambda y}$, so we obtain the solution set
%    %
%    \[ f(x,y) = \sin(n x) (Ae^{n y} + Be^{-ny}) \]
%    %
%    where $n \in \mathbf{Z}$, $A,B \in \RR$.

%    Now suppose we can write
%    %
%    \[ f(x,y) = \sum_{n = -\infty}^\infty \sin(nx) (A_n e^{ny} + B_n e^{-ny}) \]
%    %
%    Then
%    %
%    \[ f_0(x) = \sum_{n = -\infty}^\infty (A_n + B_n) \sin(nx) \]
%    \[ f_1(x) = \sum_{n = -\infty}^\infty (A_n e^n + B_n e^{-n}) \sin(nx) \]
%    %
%    So if $\widehat{f_0}$ and $\widehat{f_1}$ denote the sine coefficients of $f_0$ and $f_1$, then
%    %
%    \[ A_n + B_n = \widehat{f_0}(n)\ \ \ \ \ A_n e^n + B_n e^{-n} = \widehat{f_1}(n) \]
%    %
%    \[ A_n = \frac{\widehat{f_1}(n) - \widehat{f_0}(n) e^{-n}}{e^{n} - e^{-n}} \]
%    %
%    \[ B_n = \widehat{f_0}(n) - \frac{\widehat{f_1}(n) - \widehat{f_0}(n) e^{-n}}{e^{n} - e^{-n}} = \frac{e^n \widehat{f_0}(n) - \widehat{f_1}(n)}{e^n - e^{-n}} \]
%    %
%    Thus
%    %
%    \begin{align*}
%        f(x,y) &= \sum_{n = -\infty}^\infty \sin(nx) \left( \frac{(\widehat{f_1}(n) - \widehat{f_0}(n) e^{-n}) e^{ny} + (e^n \widehat{f_0}(n) - \widehat{f_1}(n)) e^{-ny}}{e^n - e^{-n}} \right)\\
%        &= \sum_{n = -\infty}^\infty \frac{\sin(nx)}{e^n - e^{-n}} [(e^{n(1-y)} - e^{n(y-1)}) \widehat{f_0}(n) + (e^{ny} - e^{-ny}) \widehat{f_1}(n)]\\
%        &= \sum_{n = -\infty}^\infty \left( \frac{\sinh n(1-y)}{\sinh n} \widehat{f_0}(n) + \frac{\sinh ny}{\sinh n} \widehat{f_1}(n) \right) \sin(nx)
%    \end{align*}
%\end{example}

%
%First, define the circle group $\mathbf{T}$ to be the set of complex numbers $z$ with $|z| = 1$. Functions from $\mathbf{T}$ to $\RR$ naturally correspond to $2 \pi$-periodic functions; given $g: \mathbf{T} \to \RR$, the correspondence is given by the equation $f(t) = g(e^{it})$. Thus, when defining $2\pi$ periodic functions, we shall make no distinction between a function `defined in terms of $t$' and a function `defined in terms of $z$', after making the explicit identification $z = e^{it}$. Then an expansion of the form
%
%\[ f(t) = \sum_{k = 0}^\infty A_k \cos(kt) + \sum B_k \sin(kt) \]
%
%leads to an expansion
%
%\begin{align*}
%    f(z) &= \sum_{k = 0}^\infty A_k \Re[z^k] + B_k \Im[z^k]\\
%    &= \sum_{k = 0}^\infty A_k \left( \frac{z^k + z^{-k}}{2} \right) - i B_k \left( \frac{z^k - z^{-k}}{2} \right) = \sum_{k = -\infty}^\infty C_k z^k
%\end{align*}
%
%so a Fourier expansion on $[0,2\pi]$ is really just a power series expansion on the circle in disguise.
%
%Thus expanding a real-valued function in the exponentials $e_n(t) = e^{nit}$ is the same as expanding the function in terms of sines and cosines. The complex exponentials $e_n$ have the same orthogonality properties as $\sin$ and $\cos$, so given a function $f$, the coefficients $C_n$ can be found by the expansion
%
%\[ C_n = \frac{1}{2\pi} \int_{-\pi}^\pi f(t) e_n(-t) dt \]

\section{Basic Properties of Fourier Series}

One of the most important properties of the Fourier series is that the coefficients are controlled by reasonable transformations. A basic, but unappreciated property of the Fourier transform is \emph{linearity}: For any two functions $f$ and $g$,
%
\[ \widehat{f+g} = \widehat{f} + \widehat{g}. \]
%
Linearity is \emph{essential} to most methods in this book, and much remains unknown about nonlinear transforms. Fourier series are also stable under various other transformations which occur in analysis, which makes the transform useful. We summarize these properties here:
%
\begin{itemize}
    \item For each periodic function $f$, define $f^*(x) = \overline{f(x)}$. Then for each $n \in \ZZ$
    %
    \[ \widehat{f^*}(n) = \overline{\widehat{f}(-n)}. \]
    %
    As a corollary, we find that if $f$ is real-valued, then $\widehat{f}(-n) = \overline{\widehat{f}(n)}$.

    \item Define the translation and frequency modulation operators, for each $s \in \RR$, $m \in \ZZ$, and $2\pi$ periodic $f$, by setting
    %
    \[ (T_s f)(t) = f(t-s) \quad \text{and} \quad (M_m f)(s) = e_m(s) f(s). \]
    %
    Similarily, for a function $C: \ZZ \to \CC$, and for each $m \in \ZZ$ and $\xi \in \RR$, define
    %
    \[ (T_m C)(n) = C(n-m) \quad \text{and} \quad (M_\xi C)(n) = e_\xi(n) C(n) \]
    %
    Then $\widehat{T_s f} = M_{-s} \widehat{f}$ and $\widehat{M_m f} = T_m \widehat{f}$, for each $s \in \RR$ and $m \in \ZZ$.

    \item If $f$ is odd, then $\widehat{f}$ is odd, and if $f$ is even, $\widehat{f}$ is even.

    \item If $f$ has a derivative $f'$, then $\widehat{f'}(n) = 2 \pi in \widehat{f}(n)$.
\end{itemize}

All but the last relation can be proved by easy exercises in manipulating periodic integrals, when $f$ is a periodic measurable function with
%
\[ \int_0^1 |f(x)|\; dx < \infty. \]
%
The space of all such functions will be denoted by $L^1(\mathbf{T})$, which is a Banach space under the norm
%
\[ \| f \|_{L^1(\mathbf{T})} = \int_0^1 |f(x)|\; dx. \]
%
The latter property involving the derivative of $f$ holds by an easy integration by parts. This proof therefore holds whenever $f$ has a weak derivative in $L^1(\mathbf{T})$, i.e. for $f \in W^{1,1}(\mathbf{T})$.

\begin{remark}
    We note that if $f$ is even, then $\widehat{f}$ is even, so formally
    %
    \[ f(t) \sim \widehat{f}(0) + \sum_{m = 1}^\infty \widehat{f}(m) [e_m(t) + e_{-m}(t)] = \widehat{f}(0) + 2 \sum_{m = 1}^\infty \widehat{f}(m) \cos(mt). \]
    %
    Moreover,
    %
    \[ \widehat{f}(m) = \frac{1}{2\pi} \int_{-\pi}^\pi f(t) \cos(mt)\; dt \]
    %
    If $f$ is an odd function, then the fact that $\widehat{f}$ is odd implies formally that
    %
    \[ f(t) \sim \sum_{m = 1}^\infty \widehat{f}(m) [e_m(t) - e_{-m}(t)] = 2i \sum_{m = 1}^\infty \widehat{f}(m) \sin(mt). \]
    %
    Thus we get a sine expansion, and moreover,
    %
    \[ \widehat{f}(m) = \frac{1}{2\pi i} \int_{-\pi}^\pi f(t) \sin(mt)\; dt. \]
    %
    This is one way to reduce the study of complex exponentials back to the study of sines and cosines, since every function can be written as a sum of an even and an odd function.
\end{remark}

\begin{comment}

\section{Examples of Expansions}

Before we get to the real work, let's start by computing some Fourier series, to use as examples. We also illustrate the convergence properties of these series, which we shall look at in more detail later.

\begin{example}
    Consider the function $f: [0,\pi] \to \RR$ defined by $f(x) = x(\pi - x)$. Then a series of integration by parts gives that
    %
    \[ \int x(\pi - x) \sin(nx) = \frac{x(\pi - x) \cos(nx)}{n} + \frac{(\pi - 2x) \sin(nx)}{n^2} - \frac{2\cos(nx)}{n^3}. \]
    %
    Thus
    %
    \[ \frac{2}{\pi} \int_0^\pi x(\pi - x) \sin(nx) = \frac{4(1 - \cos(n\pi))}{n^3} = \begin{cases} \frac{8}{\pi n^3} & n\ \text{odd}, \\ 0 & n\ \text{even}. \end{cases}  \]
    %
    Thus we have a formal representation
    %
    \[ f(x) = \sum_{n\ \text{odd}} \frac{8}{\pi n^3} \sin(nx). \]
    %
    This sum converges absolutely and uniformly for $x \in [0,\pi]$. If we extend the domain of $f$ to $[-\pi,\pi]$ by making $f$ odd, then
    %
    \[ \widehat{f}(n) = \begin{cases} \frac{4}{\pi i n^3} & : n\ \text{odd}, \\ 0 & : n\ \text{even}. \end{cases} \]
    %
    In this case, we still have
    %
    \[ f(x) = \sum_{\substack{n\ \text{odd}\\ n > 0}} \frac{4}{\pi i n^3} [e_n(x) - e_n(-x)] = \sum_{n\ \text{odd}} \frac{8}{\pi n^3} \sin(nx). \]
    %
    This sum converges absolutely and uniformly on the entire real line.
\end{example}

\begin{example}
    The tent function
    %
    \[ f(x) = \begin{cases} 1 - \frac{|x|}{\delta} & : |x| < \delta, \\ 0 & : |x| \geq \delta. \end{cases} \]
    %
    is even, and therefore has a purely real Fourier expansion
    %
    \[ \widehat{f}(0) = \frac{\delta}{2\pi},\quad\widehat{f}(n) = \frac{1 - \cos(n\delta)}{\delta \pi n^2}. \]
    %
    Thus we obtain an expansion
    %
    \[ f(x) = \frac{\delta}{2\pi} + \sum_{n \neq 0} \frac{1 - \cos(n\delta)}{\delta \pi n^2} e_n(x) = \frac{\delta}{2 \pi} + 2 \sum_{n = 1}^\infty \frac{1 - \cos(n\delta)}{\delta \pi n^2} \cos(nx). \]
    %
    This sum also converges absolutely and uniformly.
\end{example}

\begin{example}
    Consider the characteristic function
    %
    \[ \chi_{(a,b)}(x) = \begin{cases} 1 & : x \in (a,b), \\ 0 & : x \not \in (a,b). \end{cases} \]
    %
    Then
    %
    \[ \widehat{\chi}_{(a,b)}(n) = \frac{1}{2\pi} \int_a^b e_n(-x) = \frac{e_n(-a) - e_n(-b)}{2\pi i n}. \]
    %
    Hence we may write
    %
    \begin{align*}
        \chi_{(a,b)}(x) &= \frac{b-a}{2\pi} + \sum_{n \neq 0} \frac{e_n(-a) - e_n(-b)}{2 \pi i n} e_n(x)\\
        &= \frac{b-a}{2\pi} + \sum_{n = 1}^\infty \frac{\sin(nb) - \sin(na)}{\pi n} \cos(nx) + \frac{\cos(na) - \cos(nb)}{\pi n} \sin(nx).
    \end{align*}
    %
    This sum does not converge absolutely for any value of $x$ (except when $a$ and $b$ are chosen trivially). To see this, note that
    %
    \[ \left|\frac{e_n(-b) - e_n(-a)}{2 \pi n}\right| = \left| \frac{1 - e_n(b-a)}{2 \pi n} \right| \geq \left| \frac{\sin(n(b-a))}{2 \pi n} \right|, \]
    %
    so that it suffices to show $\sum |\sin(nx)| n^{-1} = \infty$ for every $x \not \in \pi \mathbf{Z}$. This follows because the values of $|\sin(nx)|$ are often large, so that we may apply the divergence of $\sum n^{-1}$. First, assume $x \in (0,\pi/2)$. If
    %
    \[ m \pi - x/2 < nx < m \pi + x/2 \]
    %
    for some $m \in \mathbf{Z}$, then
    %
    \[ m \pi + x/2 < (n+1)x < m \pi + 3x/2 < (m+1) \pi - x/2. \]
    %
    Thus if $nx \in (-x/2,x/2) + \pi \mathbf{Z}$, $(n+1)x \not \in (-x/2,x/2) + \pi \mathbf{Z}$. For $y$ outside of $(-x/2,x/2) + \pi \mathbf{Z}$, we have $|\sin(y)| > |\sin(x/2)|$, and therefore for any $n$,
    %
    \[ \frac{|sin(nx)|}{n} + \frac{|\sin((n+1)x)|}{n+1} > \frac{|\sin(x/2)|}{n+1}. \]
    %
    This means
    %
    \begin{align*}
        \sum_{n = 1}^\infty \frac{|\sin(nx)|}{n} &= \sum_{n = 1}^\infty \frac{|\sin(2nx)|}{2n} + \frac{|\sin((2n+1)x)|}{2n+1}\\
        &> |\sin(x/2)| \sum_{n = 1}^\infty \frac{1}{2n+1} = \infty
    \end{align*}
    %
    In general, we may replace $x$ with $x - k \pi$, with no effect to the values of the sum, so we may assume $0 < x < \pi$. If $\pi/2 < x < \pi$, then
    %
    \[ \sin(nx) = \sin(n(\pi - x)), \]
    %
    and $0 < \pi - x < \pi/2$, completing the proof, except when $x = \pi$, in which case
    %
    \[ \sum_{n = 1}^\infty \left| \frac{1 - e_n(\pi)}{2 \pi n} \right| = \sum_{n\ \text{even}} \left| \frac{1}{\pi n} \right| = \infty. \]
    %
    Thus the convergence of a Fourier series need not be absolute.
\end{example}

\begin{example}
    We can often find formulas for certain Fourier summations from taking the corresponding power series. This is because if we set $z = e^{it}$, then
    %
    \[ \sum_{n = -\infty}^\infty a_n e^{nit} = \sum_{n = -\infty}^\infty a_n z^n \]
    %
    becomes a Laurent series in $z$. For instance, we have a power series expansion
    %
    \[ \log \left( \frac{1}{1-x} \right) = \sum_{k = 1}^\infty \frac{z^k}{k}. \]
    %
    This converges pointwise for every $z \in \mathbf{D}$ but $z = 1$. Thus for $x \not \in 2 \pi \mathbf{Z}$,
    %
    \begin{align*}
        \sum_{k = 1}^\infty \frac{\cos(kx)}{k} &= \Re \left( \log \left( \frac{1}{1 - e(x)} \right) \right) = -\frac{1}{2} \log(2 - 2\cos(x)),\\
        \sum_{k = 1}^\infty \frac{\sin(kx)}{k} &= \Im \left( \log \left( \frac{1}{1 - e(x)} \right) \right) = \arctan \left( \frac{\sin(x)}{1 - \cos(x)} \right).
    \end{align*}
    %
    Here we agree that $\arctan(\pm \infty) = \pm \pi/2$. One can check that, indeed, the Fourier series of these two functions corresponds precisely to these summations. If a power series' radius of convergence exceeds $1$, then it is likely that the corresponding Fourier series taken on the circle will be pleasant, whereas if the power series' radius is equal to $1$, we can expect nasty behaviour on the boundary. In Complex analysis, one avoids talking about the boundary of the holomorphic function's definition, whereas in Fourier analysis we have to embrace the boundary points because this is the domain we are interested in. This makes the theory a little more pathological.
\end{example}

\end{comment}

\chapter{Fourier Series}

Let's focus in on the problem we introduced in the last chapter. We write $\TT = \RR / \ZZ$, so that a function $f: \TT \to \CC$ is a complex-valued periodic function on the real line. We then have a metric on $\TT$ given by setting $d(t,s) = |t - s|$, where $|t| = \min_{n \in \mathbf{Z}} |t + 2\pi n|$. The Lebesgue measure on $\RR$ induces a natural Borel measure on $\TT$, such that for any periodic function $f: \mathbf{T} \to \mathbf{C}$,
%
\[ \int_{\TT} f(t)\; dt = \int_0^1 f(t)\; dt. \]
%
For each integrable $f: \mathbf{T} \to \mathbf{C}$, we then have a \emph{formal trigonometric series}
%
\[ \sum_{n = -\infty}^\infty \widehat{f}(n) e_n(t). \]
%
In some sense, $f$ should be able to be approximated by the trigonometric polynomials obtained by truncating this series. However, at this point, we haven't deduced any reason for these sums to converge to $f$ analytically.

To understand the convergence of Fourier series to the function from which they are defined, we consider the partial sums
%
\[ S_Nf = \sum_{|n| \leq N} \widehat{f}(n) e_n. \]
%
If $f$ is real-valued, then the complex parts of $\widehat{f}(n)$ and $\widehat{f}(-n)$ cancel out, so $S_N f$ is a real-valued sum of cosines and sines.

An initial hope is that the Fourier series of a function converges pointwise, i.e. that for every $t$, $\lim_{N \to \infty} S_N(f)(t) = f(t)$. This is true if $f$ is periodic and differentiable everywhere. But even if we try and look at continuous periodic functions, then $S_N f$ can diverge. Thus we look for more exotic forms of convergence, and different quantitative descriptions which determine how the Fourier series represents the function $f$.

\section{Unique Representation of a Function?}

If the Fourier series of every function converged pointwise, we could conclude that if $f$ and $g$ have the same Fourier coefficients, they must necessarily be equal. This is clearly not true, for if we alter a function at a point, the Fourier series, defined by averaging over the entire region, remains the same. Nonetheless, if a function is continuous, editing the function at a point will break continuity, so we may have some hope of uniqueness of the expansion.

\begin{theorem}
    If $\widehat{f}(n) = 0$ for all $n \in \ZZ$, then $f$ vanishes at all it's continuity points.
\end{theorem}
\begin{proof}
    We shall prove, without loss of generality, that if $f$ is continuous at the origin, then $f(0) = 0$. We treat the real-valued case first. For every trigonometric polynomial $g(x) = \sum a_n e_n(-x)$, we have
    %
    \[ \int_{\mathbf{T}} f(x) g(x) dx = \sum a_n \widehat{f}(n) = 0. \]
    %
    Suppose that $f$ is continuous at zero, and assume without loss of generality that $f(0) > 0$. Pick $\delta$ such that if $|x| < \delta$, $|f(x)| > f(0)/2$. Consider the trigonometric polynomial
    %
    \[ g(x) = \varepsilon + \cos(2 \pi x) = \varepsilon + \frac{e_1(x) + e_1(-x)}{2}. \]
    %
    where $\varepsilon$ is small enough that $g(x) > A > 1$ for $|x| < \delta/2$, $P(x) > 0$ for $\delta/2 \leq |x| < \delta$, and $g(x) < B < 1$ for $|x| \geq \delta$. The series of trigonometric polynomials $g_n(x) = (\varepsilon + \cos (2 \pi x))^n$ satisfy
    %
    \begin{align*}
        \left| \int_{\mathbf{T}} g_n(x) f(x) dx \right| &\geq \int_{|x| < \delta} g_n(x) f(x) dx - \left| \int_{\delta \leq |x|} g_n(x) f(x) dx \right|.
    \end{align*}
    %
    H\"{o}lder's inequality then guarantees that as $n \to \infty$,
    %
    \[ \left| \int_{\delta \leq |x|} g_n(x) f(x) dx \right| \lesssim B^n = o(1). \]
    %
    On the other hand,
    %
    \[ \left| \int_{|x| < \delta} g_n(x) f(x) dx \right| \geq \int_{|x| < \delta/2} g_n(x) f(x) \gtrsim A^n. \]
    %
    which increases exponentially fast as $n \to \infty$. Thus we conclude
    %
    \[ 0 = \left| \int_0^1 g_n(x) f(x) dx \right| \gtrsim A^n - o(1). \]
    %
    For suitably large values of $n$, the right hand side is positive, whereas the left hand side is zero, which is impossible. By contradiction, we conclude $f(0) = 0$. In general, if $f$ is complex valued, then we may write $f = u + iv$, where
    %
    \[ u(x) = \frac{f(x) + \overline{f(x)}}{2}\ \ \ \ v(x) = \frac{f(x) - \overline{f(x)}}{2i}. \]
    %
    The Fourier coefficients of $\overline{f}$ all vanish, because the coefficients of $f$ vanish, and so we conclude the coefficients of $u$ and $v$ vanish. $f$ is continuous at $x$ if and only if it is continuous at $u$ and $v$, and we know from the previous case this means that both $u$ and $v$ vanish at that point.
\end{proof}

\begin{corollary}
    If $f,g \in C(\mathbf{T})$ and $\widehat{f} = \widehat{g}$, then $f = g$.
\end{corollary}
\begin{proof}
    Then $f - g$ is continuous with vanishing Fourier coefficients.
\end{proof}

\begin{corollary}
    If $f \in C(\mathbf{T})$ and $\widehat{f} \in L^1(\mathbf{Z})$, $S_N f \to f$ uniformly as $N \to \infty$.
\end{corollary}
\begin{proof}
    If $\sum |\widehat{f}(n)| < \infty$, then the triangle inequality implies
    %
    \[ |(S_{N + M} f)(x) - (S_N f)(x)| \leq \sum_{N < |k| \leq M} |\widehat{f}(k)|. \]
    %
    Since the series $\widehat{f}$ is absolutely summable, for any $\varepsilon$, there is sufficiently large $N$, such that the quantity above is bounded uniformly by $\varepsilon$. Thus the functions $\{ S_N f \}$ are a Cauchy sequence in $L^\infty(\mathbf{T})$, and therefore converge uniformly to some function $Sf \in C(\mathbf{T})$. Uniform convergence also implies we can interchange integrals to conclude
    %
    \[ \widehat{Sf}(n) = \lim_{N \to \infty} \int_{\TT} S_N(f)(t) e_n(-t) = \widehat{f}(n). \]
    %
    Thus $\widehat{Sf} = \widehat{f}$, so $Sf = f$ since both functions are continuous.
\end{proof}

Later we show that if $f \in C^m(\mathbf{T})$, then $\widehat{f}(n) = O(1/|n|^m)$. In particular, if $m \geq 2$, then $S_N f \to f$ uniformly. Moreover, if $f \in C^\infty(\mathbf{T})$, then this shows the $k$'th derivatives $(S_N f)^{(k)}$ converge uniformly to $f^{(k)}$ for each $k$, and $\widehat{f}(m) = O_m(|n|^{-m})$ for each $m \geq 1$. Conversely, if $\{ a_m \}$ is a sequence with $|a_m| = O_m(|n|^{-m})$ for each $m$, then the infinite sum $\sum a_m e^{mix}$ and all it's derivatives converge uniformly to an infinitely differentiable function $f \in C^\infty(\mathbf{T})$, and $\widehat{f}(n) = a_n$. Thus there is a perfect duality between infinitely differentiable functions and arbitrarily fast decaying sequences of integers. In more advanced contexts, like the theory of distributions, this duality is very useful for studying the Fourier transform.

\section{Quantitative Bounds on Fourier Coefficients}

In practical contexts, most functions we deal with are arbitrarily smooth, so the picture established in the last section seems rather complete. However, a deeper understanding of the Fourier series involves studying more quantitative questions. For instance, does the Fourier series of a function which is uniformly small converge faster than a function which is only small on average. In modern terms, do we get faster convergence rates if $\| f \|_{L^\infty(\mathbf{T})}$ is small rather than just if $\| f \|_{L^1(\mathbf{T})}$ is small. Does the convergence get faster if we consider the convergence with respect to an $L^p$ norm rather than an $L^\infty$ norm? Thus we want to understand the behaviour of the Fourier series with respect to a family of \emph{norm spaces}. Similarily, how does the Fourier series of a function $f$ change under a small pertubation in a particular norm space. Of course, these norms are defined in a more general space of measurable functions, and to apply functional analysis arguments it is essential to `complete' the picture of these norms, so we will find that many of our arguments, initially invented to study smooth functions, also work naturally with arbitrarily integrable functions. On the other hand, the density of regular functions in these norm spaces indicates that the behaviour of Fourier summation on an infinitely differentiable space with respect to a norm is at least as bad as the behaviour of Fourier summation on an integrable function.

We note these quantitative problems are still interesting even if we knew everything there was to know about the pointwise convergence of Fourier series, because a series of functions may converge pointwise, whereas none of the individual functions may `look' like the function they converge to. So we may want to look at quantitative measures of how globally similar two functions are, and this leads to norm space estimates.

\begin{example}
    If we consider a square wave $\chi_I$ for some interval $I$, then the techniques of the following section allow us to prove that
    %
    \[ \| \chi_I - S_N \chi_I \|_{L^2(\mathbf{T})} = O(1/\sqrt{N}), \]
    %
    independently of $I$. This means that if we want to simulate square waves with a musical instrument up to some square mean error $\varepsilon$, then we will need $\Omega(\varepsilon^{-2})$ different notes to represent the sound accurately. Thus a piano with 88 keys can only approximate square waves slightly better than a keyboard with 20 keys. If $f$ is $C^{m+1}(\mathbf{T})$, then
    %
    \[ \| f - S_N f \|_{L^2(\mathbf{T})} = O(1/N^{m/2}), \]
    %
    so we require significantly less notes to simulate this sound, i.e. $\Omega(\varepsilon^{-2/m})$. In this case a piano can simulate these sounds much more accurately.
\end{example}

One initial equation which might summarize how well behaved the Fourier series is with respect to suitable norms would be to obtain an estimate of the form $\smash{\| \widehat{f} \|_{L^q(\mathbf{Z})} \lesssim \| f \|_{L^p(\mathbf{T})}}$ for particular values of $p$ and $q$. If this was established, we could conclude that the Fourier series is stable in the $L^q$ norm under small pertubations in the $L^p$ norm. The first inequality we give is trivial, but is certainly tight, e.g. for $f(t) = e_n(t)$.

\begin{theorem}
    For any $f$, $\| \widehat{f} \|_{L^\infty(\mathbf{T})} \leq \| f \|_{L^1(\mathbf{T})}$.
\end{theorem}
\begin{proof}
    We just take absolute values into the oscillatory integral defining the Fourier coefficients, calculating that for any $n$,
    %
    \[ |\widehat{f}(n)| = \left| \int_{\mathbf{T}} f(t) e_n(-t) \right| \leq \int_{\mathbf{T}} |f(t)| = \| f \|_{L^1(\mathbf{T})}, \]
    %
    which was the required bound.
\end{proof}

This proof doesn't really take any deep features of the Fourier coefficients. The same bound holds for any integral
%
\[ \int_{\mathbf{T}} f(t) K(t)\; dt, \]
%
where $|K(t)| \leq 1$ for all $t$. But the bound is still tight, which might be explained by the fact that the Fourier series gives oscillatory information which is not immediately present in the $L^p$ norms of the phase spaces, other than by taking a naive absolute bound into the $L^p$ norm. The only $L^p$ space where we can get a completely satisfactory bound is for $p = 2$, where we can use Hilbert space technique; this should be expected since orthogonality was used to motivate the definition of the Fourier series.

\begin{theorem}
    For any function $f$, $\| \widehat{f} \|_{L^2(\mathbf{Z})} = \| f \|_{L^2(\mathbf{T})}$.
\end{theorem}
\begin{proof}
    With respect to the normalized inner product on the space $L^2(\mathbf{T})$,the calculations of the last chapter tell us that the exponentials are an orthonormal family of functions, in the sense that for distinct $n$ and $m$, $(e_n,e_m) = 0$, and $(e_n,e_n) = 1$. Since $\smash{\widehat{f}(n) = (f,e_n)}$, we apply Bessel's inequality to conclude
    %
    \[ \| \widehat{f} \|_{L^2(\mathbf{Z})} \leq \| f \|_{L^2(\mathbf{T})}. \]
    %
    The exponentials $\{ e_n \}$ are actually an orthonormal basis for $L^2(\mathbf{T})$; This can be seen from the Stone Weirstrass theorem, since trigonometric polynomials separate points, or by results we prove independently, later on in these notes. Thus Parsevel's equality tells us $\| \widehat{f} \|_{L^2(\mathbf{Z})} = \| f \|_{L^2(\mathbf{T})}$.
\end{proof}

This equality makes the Hilbert space $L^2(\mathbf{T})$ often the best place to understand Fourier expansion techniques, and general results are often achieved by reduction to this well understood case. For instance, the inequality above, combined with the trivial inequality, is easily interpolated using the Riesz-Thorin technique to give the Hausdorff Young inequality.

\begin{theorem}
    If $1 \leq p \leq 2$, $\| \widehat{f} \|_{L^q(\mathbf{Z})} \leq \| f \|_{L^p(\mathbf{T})}$.
\end{theorem}

It might be surprising to note that the Hausdorff Young inequality essentially completes the bounds on the Fourier series with respect to the $L^p$ norms. There is no interesting result one can obtain for $p > 2$ other than the obvious inequality
%
\[ \| \widehat{f} \|_{L^2(\mathbf{Z})} \leq \| f \|_{L^2(\mathbf{T})} \leq \| f \|_{L^p(\mathbf{T})}. \]
%
Thus we can control the magnitude of the Fourier coefficients in terms of the width of the original function, but we are limited in our ability to control the width of the Fourier coefficients in terms of the magnitudes of the original function. This makes sense, because the $L^p$ norm of $f$ measures fairly different aspects of the function than the $L^q$ norm of the Fourier transform of $f$. It is only in the case of the $L^2$ norm where results are precise, and where $p$ is small that we can take a trivial bound, that we get an inequality like the Hausdorff Young result.

\section{Asymptotic Decay of Fourier Series}

The next result, known as Riemann-Lebesgue lemma, shows that the Fourier series of any integrable function decays, albeit arbitrarily slowly. The proof we give is an instance of an important principle in Functional analysis that we will use over and over again. Suppose for each $n$, we have a bounded operator $T_n: X \to Y$ between Banach spaces, and we want to show that for each $x \in X$, $\lim T_n(x) = T(x)$, where $T$ is another bounded operator. Suppose that it is obvious that $\lim T_n(x') = T(x')$ for a dense family of points $X' \subset X$, and the operators $T_n$ are {\it uniformly} bounded. Then for any $x \in X$,
%
\begin{align*}
    \| T_n(x) - T(x) \| &\leq \| T_n(x) - T_n(x') \| + \| T_n(x') - T(x') \| + \| T(x') - T(x) \|.
\end{align*}
%
If we choose $x'$ such that $\| x - x' \| \leq \varepsilon$, then for $n$ large enough we find that $\| T_n(x) - T(x) \| \lesssim \varepsilon$. Since $\varepsilon$ was arbitrary, this means that $T_n(x) \to T(x)$ as $n \to \infty$. The advantage of the principle is that it is suitably abstract, and can thus be used very flexibly. But the disadvantage is that it is a very soft analytical argument, and cannot be used to obtain results on the rate of convergence of $T_n(x)$ to $T(x)$. Here is a simple application.

\begin{lemma}[Riemann-Lebesgue]
    If $f \in L^1(\mathbf{T})$, then $\widehat{f}(n) \to 0$ as $|n| \to \infty$.
\end{lemma}
\begin{proof}
    We claim the lemma is true for the characteristic function of an interval $\chi_I$. If $I = [a,b]$, then
    %
    \[ \widehat{\chi_I}(n) = \int_a^b e_n(-t) = \frac{e_n(-b) - e_n(-a)}{-in} = O(1/n). \]
    %
    By linearity of the integral, the Fourier transform of any step function vanishes at $\infty$. But if $\smash{\Lambda_n(f) = \widehat{f}(n)}$, then $|\Lambda_n f| \leq \| \widehat{f} \|_{L^\infty(\mathbf{T})} \leq \| f \|_{L^1(\mathbf{T})}$, which shows that the sequence of functionals $\{ \Lambda_n \}$ are uniformly bounded as maps from $L^1(\mathbf{T})$ to $L^\infty(\mathbf{Z})$. Since $\lim_{|n| \to \infty} \Lambda_n(f) = 0$ for any step function $f$, and the step functions are dense in $L^1(\mathbf{T})$, we conclude that $\lim_{|n| \to \infty} \Lambda_n(f) = 0$ for all $f \in L^1(\mathbf{T})$.
\end{proof}

Even though the Fourier series of any step function decays at a rate $O(1/n)$, it is {\it not} true that a general Fourier series decays at a rate of $O(1/n)$. And in fact, for any sequence of non-negative numbers $\{ \varepsilon_m \}$ with $\varepsilon_m \to 0$ as $|m| \to \infty$, there exists a continuous function $f$ such that $|\widehat{f}(n)| \geq \varepsilon_n$ for infinitely many values $n$. One simply chooses a subsequence $\varepsilon_{m_k}$ with $\sum \varepsilon_{m_k} < \infty$, and consider the trigonometric series attached to this subsequence. This is precisely the penalty for using a soft type analytical argument. Nonetheless, for smooth functions, we can obtain a uniform decay rate. This is an instance of a general result related to the duality between decay and smoothness in phase and frequency space.

\begin{theorem}
    If $f \in C^m(\mathbf{T})$, then $|\widehat{f}(n)| \leq |n|^{-m} \| f^{(m)} \|_{L^1(\mathbf{T})}$.
\end{theorem}
\begin{proof}
    Applying the derivative law for Fourier series $m$ times, we find that
    %
    \[ |\widehat{f}(n)| = |n|^{-m} |\widehat{f^{m}}(n)| \leq |n|^{-m} \| f^{(m)} \|_{L^1(\mathbf{T})}. \qedhere \]
\end{proof}

If $0 < \alpha < 1$, we say a function $f$ is \emph{H\"{o}lder continuous} of order $\alpha$ if there exists a constant $A$ such that $|f(x + h) - f(x)| \leq A |h|^\alpha$ for all $x, h \in \mathbf{T}$. We define
%
\[ \| f \|_{C^{0,\alpha}(\mathbf{T})} = \sup_{x,h \in \mathbf{T}} \frac{|f(x + h) - f(x)}{|h|^\alpha} \]
%
Then the space $C^{0,\alpha}(\mathbf{T})$ of all functions satisfying a H\"{o}lder condition of order $\alpha$ forms a Banach space.

\begin{theorem}
    If $f \in C^{0,\alpha}(\mathbf{T})$, then $|\widehat{f}(n)| \lesssim \| f \|_{C^{0,\alpha}(\mathbf{T})} |n|^{-\alpha}$.
\end{theorem}
\begin{proof}
    We calculate that by periodicity,
    %
    \[ \widehat{f}(n) = - \int_{\mathbf{T}} f(x + 1/n) e_n(-x)\; dx, \]
    %
    so
    %
    \[ \widehat{f}(n) = \frac{1}{2} \int_{\mathbf{T}} [f(x) - f(x + 1/n)] e_n(-x)\; dx. \]
    %
    Thus taking in absolute values and applying H\"{o}lder continuity gives
    %
    \[ |\widehat{f}(n)| \leq \frac{\| f \|_{C^{0,\alpha}(\mathbf{T})}}{2 |n|^\alpha}. \qedhere \]
\end{proof}

\begin{remark}
    Suppose that $\mu$ is a measure on $\mathbf{T}$ with finite variation, for which we write $\mu \in M(\mathbf{T})$. Then one can define the Fourier series of $\mu$ by
    %
    \[ \widehat{\mu}(n) = \int_0^1 e_n(-x) d\mu(x). \]
    %
    If $\mu$ is absolutely continuous with respect to the normalized Lebesgue measure on $\mathbf{T}$, and $d\mu = f dx$, then $\widehat{\mu} = \widehat{f}$, so this is an extension of the Fourier series from integrable functions to measures with finite variation. One can verify that
    %
    \[ \| \widehat{\mu} \|_{L^\infty(\mathbf{Z})} \leq \| \mu \|_{M(\mathbf{T})}. \]
    %
    If $\delta$ is the Dirac delta measure at the origin, i.e. $\mu(E) = 1$ if $0 \in E$, and $\mu(E) = 0$ otherwise, then for all $n$,
    %
    \[ \widehat{\delta}(n) = 1. \]
    %
    Thus the Fourier series of $\delta$ has no decay at all. Once can view this as saying functions are `smoother' than measures, and therefore have a Fourier decay, albeit one that is qualitative rather than quantitative.
\end{remark}

\section{Convolution and Kernel Methods}

The notion of the convolution of two functions $f$ and $g$ is a key tool in Fourier analysis, both as a way to regularize functions, and as an operator that transforms nicely when we take Fourier series. Given two integrable functions $f$ and $g$, we define
%
\[ (f * g)(t) = \int_{\mathbf{T}} f(s) g(t-s)\; ds. \]
%
Thus we smear the values of $g$ with respect to a density function $f$.

\begin{lemma}
    For any $1 \leq p < \infty$, and $f \in L^p(\mathbf{T})$, $\lim_{h \to 0} T_h f = f$ in $L^p(\mathbf{T})$.
\end{lemma}
\begin{proof}
    If $f$ is $C^1(\mathbf{T})$, then $|f(x + h) - f(x)| \lesssim h$ uniformly in $x$, implying that $\| T_h f - f \|_{L^p(\mathbf{T})} \leq \| T_hf - f \|_{L^\infty(\mathbf{T})} \lesssim h$, and so $T_h f \to f$ in all the spaces $L^p(\mathbf{T})$. We have $\| T_h f \|_{L^p(\mathbf{T})} = \| f \|_{L^p(\mathbf{T})}$, so the $T_h$ are a bounded family of operators, and since $C^1(\mathbf{T})$ is dense in $L^p(\mathbf{T})$ for $1 \leq p < \infty$, we conclude that $\lim_{h \to 0} T_h f = f$ for all $f \in L^p(\mathbf{T})$.
\end{proof}

\begin{theorem}
    Convolution has the following properties:
    %
    \begin{itemize}
        \item If $f \in L^p(\mathbf{T})$ and $g \in L^q(\mathbf{T})$, for $1/p + 1/q = 1$, then $f * g$ is uniformly continuous.

        \item If $f \in L^p(\mathbf{T})$ and $g \in L^q(\mathbf{T})$, and if we define $r$ so that $1/r = 1/p + 1/q - 1$, with $1 \leq r \leq \infty$, then $f * g$ is well-defined by the integral formula almost everywhere, and
        %
        \[ \| f * g \|_{L^r(\mathbf{T})} \leq \| f \|_{L^p(\mathbf{T})} \| g \|_{L^q(\mathbf{T})}. \]
        %
        This is known as {\it Young's inequality} for convolutions.

        \item Convolution is a commutative, associative, bilinear operation.

        \item If $f,g \in L^1(\mathbf{T})$, then $\widehat{f * g} = \widehat{f} \widehat{g}$.

        \item If $f$ has a weak derivative $f'$ in $L^1(\mathbf{T})$, then $f * g$ has a weak derivative in $L^1(\mathbf{T})$, and $(f * g)' = f' * g$. Thus convolution is `additively smoothing'. In particular, if $f \in C^k(\mathbf{T})$ and $g \in C^l(\mathbf{T})$, then $f * g \in C^{k+l}(\mathbf{T})$.

        \item If $f$ is supported on $E$, and $g$ on $F$, then $f * g$ is supported on $E + F$.
    \end{itemize}
\end{theorem}
\begin{proof}
    Suppose $f \in L^p(\mathbf{T})$, and $g \in L^q(\mathbf{T})$, then
    %
    \begin{align*}
        |(f * g)(t - h) - (f * g)(t)| &\leq \int_{\mathbf{T}} |f(t-h-s) - f(t-s)| |g(s)|\; ds\\
        &\leq \| f_h - f \|_{L^p(\mathbf{T})} \| g \|_{L^q(\mathbf{T})}.
    \end{align*}
    %
    The right hand side is a bound independant of $t$ and converges to zero as $h \to 0$, so $f * g$ is uniformly continuous. Applying H\"{o}lder's inequality again gives that $\| f * g \|_{L^\infty(\mathbf{T})} \leq \| f \|_{L^p(\mathbf{T})} \| g \|_{L^q(\mathbf{T})}$. If $f \in L^p(\mathbf{T})$, and $g \in L^1(\mathbf{T})$, we use Minkowski's inequality to conclude that
    %
    \begin{align*}
        \| f * g \|_{L^p(\mathbf{T})} &= \left( \int_{\mathbf{T}} \left| \int_{\mathbf{T}} f(t-s)g(s)\; ds \right|^p\; dt \right)^{1/p}\\
        &\leq \int_{\mathbf{T}} \left( \int_{\mathbf{T}} |f(t-s)g(s)|^p\; dt \right)^{1/p}\; ds\\
        &= \int_{\mathbf{T}} g(s) \| f \|_{L^p(\mathbf{T})}\; ds = \| f \|_{L^p(\mathbf{T})} \| g \|_{L^1(\mathbf{T})}.
    \end{align*}
    %
    Thus $f * g$ is finite almost everywhere. The inequality also implies that
    %
    \[ \| f * g \|_{L^p(\mathbf{T})} \leq \| f \|_{L^1(\mathbf{T})} \| g \|_{L^p(\mathbf{T})} \]
    %
    if $f \in L^1(\mathbf{T})$, and $g \in L^p(\mathbf{T})$. But now implying Riesz-Thorin interpolation gives the general Young's inequality. Elementary applications of change of coordinates and Fubini's theorem establish the commutativity and associativity of convolution for functions $f, g \in L^1(\mathbf{T})$.
    %
    %But $L^1(\mathbf{T}) \cap L^p(\mathbf{T})$ is dense in $L^p(\mathbf{T})$. Since $f * g = g * f$ for a dense family of functions, and convolution is continuous from $L^p(\mathbf{T}) \times L^q(\mathbf{T}) \to L^r(\mathbf{T})$, we obtain the identity for the more general families of functions.
    Similarily, one can apply Fubini's theorem to obtain associativity for $f,g,h \in L^1(\mathbf{T})$. To obtain the product identity for the Fourier series, we can apply Fubini's theorem to write
    %
    \begin{align*}
        \widehat{f * g}(n) &= \int_{\mathbf{T}} (f * g)(t) e_n(-t)\ dt\\
        &= \int_{\mathbf{T}} \int_{\mathbf{T}} f(s)g(t-s) e_n(-t)\ ds\ dt\\
        &= \int_{\mathbf{T}} f(s) \int_{\mathbf{T}} (L_{-s}g)(t) e_n(-t)\ dt\ ds\\
        &= \int_{\mathbf{T}} f(s) e_n(-s) \widehat{g}(n)\ ds\\
        &= \widehat{f}(n) \widehat{g}(n),
    \end{align*}
    %
    and this is exactly the identity required. To calculate the weak derivative of $f * g$, we fix $\phi \in C^\infty(\mathbf{T})$, and calculate using two applications of Fubini's theorem that
    %
    \begin{align*}
        \int_{\mathbf{T}} (f' * g)(t) \phi(t)\; dt &= \int_\mathbf{T} \int_{\mathbf{T}} f'(t-s) g(s) \phi(t)\; ds\; dt\\
        &= \int_{\mathbf{T}} g(s) \int_{\mathbf{T}} f'(t-s) \phi(t)\; dt\; ds\\
        &= - \int_{\mathbf{T}} g(s) \int_{\mathbf{T}} f(t-s) \phi'(t)\; dt\; ds\\
        &= - \int_{\mathbf{T}} \left( \int_{\mathbf{T}} g(s) f(t-s)\; ds \right) \phi'(t)\; dt\\
        &= - \int_{\mathbf{T}} (f * g)(t) \phi'(t)\; dt.
    \end{align*}
    %
    If $f = 0$ a.e outside $E$, and $g = 0$ a.e. outside $F$, then $(f * g)(t)$ can be nonzero only when there is a set $G$ of positive measure such that for any $s \in G$, $f(s) \neq 0$ and $g(t-s) \neq 0$. But this means that $E \cap G \cap (t-F)$ has positive measure, so that there is $s \in E$ such that $t-s \in F$, meaning that $t \in E + F$.
\end{proof}

We know that suitably smooth functions have convergent Fourier series. The advantage of convolution is if we want to study the properties of a function $f$, convolution with a smooth function $g$ gives a smooth function, and provided $\smash{\widehat{g}}$ is close to 1, $\smash{\widehat{f*g}}$ will be close to $\widehat{f}$. If we can establish the convergence properties on the convolution $f * g$, then we can probably obtain results about $f$. From the frequency side, $\sum \widehat{f}(n) e_n$ might not converge, but $\sum a_n \widehat{f}(n) e_n$ might converge for a suitably fast decaying sequence $a_n$. But if $a_n$ is close to one, this sequence might still reflect properties of the original sequence.

To make rigorous the idea of approximating the Fourier series of a function, we introduce families of \emph{good kernels}. A good kernel is a sequence of integrable functions $\{ K_n \}$ on $\mathbf{T}$ bounded in $L^1$ norm, for which
%
\[ \int_{\mathbf{T}} K_n(t) = 1. \]
%
so that integration against $K_n$ operates essentially like an average, and for any $\delta > 0$,
%
\[ \lim_{n \to \infty} \int_{|t| > \delta} |K_n(t)| \to 0. \]
%
Thus the functions $\{ K_n \}$ become concentrated at the origin as $n \to \infty$. If in addition, we have an estimate $\| K_n \|_{L^\infty(\mathbf{T})} \lesssim n$, we say it is an {\bf approximation to the identity}.

\begin{theorem}
    Let $\{ K_n \}$ be a good kernel. Then
    %
    \begin{itemize}
        \item $(K_n * f)(t) \to f(t)$ for any continuity point $t$ of $f$.
        \item $(K_n * f) \to f$ uniformly if $f \in C(\mathbf{T})$, and $K_n * f$ converges to $f$ in $L^p(\mathbf{T})$ if $f \in L^p(\mathbf{T})$, for $1 \leq p < \infty$.
        \item If $K_n$ is an approximation to the identity, $(K_n * f)(t) \to f(t)$ for all $t$ in the Lebesgue set of $f$.
    \end{itemize}
\end{theorem}
\begin{proof}
    The operators $T_nf = K_n * f$ are uniformly bounded as operators on $L^p(\mathbf{T})$. Basic analysis shows that $(K_n * f)(t) \to f(t)$ at each point $t$ where $f$ is continuous, and converges uniformly to $f$ if $f$ is in $C(\mathbf{T})$. But a density argument allows us to conclude that $K_n * f \to f$ in $L^p(\mathbf{T})$ for each $f \in L^p(\mathbf{T})$, for $1 \leq p < \infty$. To obtain pointwise convergence, we calculate
    %
    \[ |(K_n * f)(t) - f(t)| \leq \int_{\mathbf{T}} |f(t - s) - f(t)| |K_n(s)|\; ds. \]
    %
    Let $A(\delta) = \delta^{-1} \int_{|s| < \delta} |f(t-s) - f(t)|$. Then as $\delta \to 0$, $A(\delta) \to 0$ because $t$ is in the Lebesgue set of $f$. And we find that for each $k$, since $|K_n(s)| \lesssim n$,
    %
    \[ \int_{2^k/n < |t| < 2^{k+1}/n} |f(t-s) - f(t)| |K_n(s)| \lesssim \frac{A(2^{k+1}/n)}{2^{k+1}}. \]
    %
    Thus we have a bound
    %
    \[ |(K_n * f)(t) - f(t)| \lesssim \sum_{k = 0}^\infty \frac{A(2^k/n)}{2^k}. \]
    %
    Because $f$ is integrable, $A$ is continuous, and hence bounded. This means that for each $m$,
    %
    \[ |(K_n * f)(t) - f(t)| \lesssim \sum_{k = 0}^m \frac{A(2^k/n)}{2^k} + \| A \|_\infty \sum_{k = m}^\infty \frac{1}{2^k} = \sum_{k = 0}^m \frac{A(2^k/n)}{2^k} + O\left( 1/2^m \right). \]
    %
    For any fixed $m$, the finite sum tends to zero as $n \to \infty$, so we obtain that $|(K_n * f)(t) - f(t)| = o(1) + O(1/2^m)$. Taking $m \to \infty$ proves the result.
\end{proof}

\section{The Dirichlet Kernel}

We calculate that
%
\[ (S_Nf)(t) = \sum_{|n| \leq N} \widehat{f}(n) e_n(t) = \frac{1}{2\pi} \int f(x) \left( \sum_{|n| \leq N} e_n(t - x) \right)\; dx.  \]
%
The bracketed part of the final term in the equation is independant of the function $f$, and is therefore key to understanding the behaviour of the sums $S_N$. We call it the Dirichlet kernel $D_N$, defined as
%
\[ D_N(t) = \sum_{n = -N}^N e_n(t). \]
%
Thus $S_N f = f * D_N$, so analyzing convolution with this kernel gives results about the sums of Fourier series.

\begin{theorem}
    For any integer $N$ and $t \in \RR$,
    %
    \[ D_N(t) = \frac{\sin(2\pi(N+1/2)t)}{\sin(\pi t)}. \]
\end{theorem}
\begin{proof}
    By the geometric series summation formula, we may write
    %
    \begin{align*}
        D_N(t) &= 1 + \sum_{n = 1}^N e_n(t) + e_n(-t) = 1 + e(t) \frac{e_N(t) - 1}{e(t) - 1} + e(-t) \frac{e_N(-t) - 1}{e(-t) - 1}\\
        &= 1 + e(t) \frac{e_N(t) - 1}{e(t) - 1} + \frac{e_N(-t) - 1}{1 - e(t)} = \frac{e_{N+1}(t) - e_N(-t)}{e(t) - 1}\\
        &= \frac{e_{N+1/2}(t) - e_{N+1/2}(-t)}{e_{1/2}(t) - e_{1/2}(-t)} = \frac{\sin(2 \pi (N + 1/2)t)}{\sin(\pi t)}.
    \end{align*}
    %
    Thus as $N \to \infty$, $D_N$ oscillates highly rapidly.
\end{proof}


If $D_N$ was a good kernel, then we would obtain that the partial sums of $S_N$ converge uniformly. This initially seems a good strategy, because $\int D_N(t) = 1$. However, we find
%
\begin{align*}
    \int_0^1 |D_N(t)| &= \int_0^1 \left| \frac{\sin(2 \pi (N + 1/2)t)}{\sin(\pi t)} \right|\\
    &\gtrsim \int_0^1 \frac{|\sin(2 \pi (N+1/2) t)|}{\sin(\pi t)}\\
    &\gtrsim \int_0^1 \frac{|\sin(2 \pi (N+1/2) t)|}{t}\; dt\\
    &= \int_0^{2 \pi N + \pi} \frac{|\sin(t)|}{t}\\
    &\gtrsim \sum_{n = 0}^N \frac{1}{t} \gtrsim \log(N).
\end{align*}
%
Thus the $L^1$ norm of $D_N$ grows, albeit slowly, to $\infty$. This reflects the fact that $D_N$ oscillates very frequently, and also that the pointwise convergence of the Fourier series is much more subtle than that provided by good kernels. In fact, a simple functional analysis argument shows that pointwise convergence of Fourier series fails for continuous functions.

\begin{theorem}
    There exists $f \in C(\mathbf{T})$ such that $(S_N f)(0)$ diverges as $N \to \infty$.
\end{theorem}
\begin{proof}
    If we consider the linear operators $\Lambda_N f = (S_N f)(0) = (f * D_N)(0)$ as maps from $C(\mathbf{T})$ to $L^1(\mathbf{T})$, and if we let $f$ be a continuous function approximating $\text{sgn}(D_N)$, then we can obtain a sequence $f_N$ such that $|\Lambda_N f_N| = \Omega(\log N) \| f_N \|_\infty$. This implies that $\| \Lambda_N \| \to \infty$ as $N \to \infty$. The uniform boundedness principle thus implies that there exists a {\it single} function $f \in C(\mathbf{T})$ such that $\sup |\Lambda_N f| = \infty$, so $(S_N f)(0)$ diverges as $N \to \infty$.
\end{proof}

The situation is even worse than this for general integrable functions. In 1927, Andrey Kolmogorov constructed an integrable function whose Fourier series diverges everywhere. But there is some hope. In 1928, Marcel Riesz showed, using methods we will develop in these notes, that if $1 < p < \infty$, and $f \in L^p(\RR)$, that $S_N f$ converges in the $L^p$ norm to $f$, by showing the Hilbert transform was bounded from $L^p(\mathbf{T})$ to $L^p(\mathbf{T})$. And after a half century of the development of techniques in harmonic analysis, in 1966, Carleson proved that for each $f \in L^p(\mathbf{T})$, for $p > 1$, the Fourier series of $f$ converges almost everywhere to $f$.

\section{Countercultural Methods of Summation}

We now interpret our convergence of series according to a different kernel, so we do get a family of good kernels, and therefore we obtain pointwise convergence for suitable reinterpretations of partial sums. One reason why the Dirichlet kernel fails to be a good kernel is that the Fourier coefficients of the kernel have a sharp drop -- the coefficients are either equal to one or to zero. If we mollify, then we will obtain a family of good kernels. And the best way to do this is to alter our summation methods slightly.

The standard method of summation suffices for much of analysis. Given a sequence $a_0, a_1, \dots$, we define the infinite sum as the limit of partial sums. Some sums, like $\sum_{k = 1}^\infty k$, obviously diverge, whereas other sums, like $\sum 1/n$, `just' fail to converge because they grow suitably slowly towards infinity over time. Since the time of Euler, a new method of summation developed by Cesaro was introduced which `regularized' certain terms by considering averaging the sums over time. Rather than considering limits of partial sums, we consider limits of averages of sums, known as Cesaro means. Letting $s_n = \sum_{k = 0}^n a_k$, we define the Cesaro means
%
\[ \frac{s_0 + \dots + s_n}{n+1}, \]
%
A sequence is Cesaro summable to some value if these averages converge. If the normal summation exists, then the Cesaro limit exists, and is equal to the original sum. However, the Cesaro summation is stronger than normal convergence.

\begin{example}
In the sense of Cesaro, we have $1 - 1 + 1 - 1 + \dots = 1/2$, which reflects the fact that the partials sums do `converge', but to two different numbers $0$ and $1$, which the series oscillates between, and the Cesaro means average these two points of convergence out to give a single method of convergence.
\end{example}

Another notion of regularization sums emerged from Complex analysis, called Abel summation. Given a sequence $\{ a_i \}$, we can consider the power series $\sum a_k r^k$. If this is well defined for $|r| < 1$, we can consider the Abel means $A_r = \sum a_k r^k$, and ask if $\lim_{r \to 1} A_r$ exists, which should be `almost like' $\sum a_k$. If this limit exists, we call it the Abel sum of the sequence.

\begin{example}
    In the Abel sense, we have $1 - 2 + 3 - 4 + 5 - \dots = 1/4$, because
    %
    \[ \sum_{k = 0}^\infty (-1)^k (k + 1) z^k = \frac{1}{(1 + z)^2}. \]
    %
    The coefficients here are $\Omega(N)$, so they can't be Cesaro summable.
\end{example}

%Abel summation is even more general than Cesaro summation.

%\begin{theorem}
%    A Cesaro summable sequence is Abel summable.
%\end{theorem}
%\begin{proof}
%    Let $\{ a_i \}$ be a Cesaro summable sequence, which we may without loss of generality assume converges to $0$. Now $(n + 1)\sigma_n - n \sigma_{n-1} = s_n$, so
    %
%    \[ (1 - r)^2 \sum_{k = 0}^n (k + 1) \sigma_k r^k = (1 - r) \sum_{k = 0}^n s_k r^k = \sum_{k = 0}^n a_k r^k \]
    %
%    As $n \to \infty$, the left side tends to a well defined value for $r < 1$, hence the same is true for $\sum_{k = 0}^n a_k r^k$. Given $\varepsilon > 0$, let $N$ be large enough that $|\sigma_n| < \varepsilon$ for $n > N$, and let $M$ be a bound for all $|\sigma_n|$. Then
    %
%    \begin{align*}
%        \left| (1 - r)^2 \sum_{k = 0}^\infty (k + 1) \sigma_k r^k \right| &\leq (1 - r)^2 \left( \sum_{k = 0}^N (k + 1) |\sigma_k| r^k + \varepsilon \sum_{k = N+1}^\infty (k + 1) r^k \right)\\
%        &= (1 - r)^2 \left( \sum_{k = 0}^N (k + 1) (|\sigma_k| - \varepsilon) r^k + \varepsilon \left[ \frac{r^{n+1}}{1-r} + \frac{1}{(1 - r)^2} \right] \right)\\
%        &\leq (1 - r)^2 M \sum_{k = 0}^N (k + 1) r^k + \varepsilon r^{n+1} (1 - r) + \varepsilon\\
%        &\leq (1 - r)^2 M \frac{(N+1)(N+2)}{2} + \varepsilon r^{n+1} (1 - r) + \varepsilon
%    \end{align*}
    %
%    Fixing $N$, and letting $r \to 1$, we may make the complicated sum on the end as small as possible, so the absolute value of the infinite sum is less than $\varepsilon$. Thus the Abel limit converges to zero.
%\end{proof}

\section{Fejer's Theorem}

Note that the Cesaro means of the Fourier series of $f$ are given by
%
\[ \sigma_N(f) = \frac{S_0(f) + \dots + S_{N-1}(f)}{N} = f * \left( \frac{D_0 + \dots + D_{N-1}}{N} \right) = f * F_N. \]
%
The convergence properties of the Cesaro means therefore relate to the properties of the {\bf Fej\'{e}r kernel} $F_N$. We find that
%
\[ F_N(x) = \sum_{n = -N}^N \left( 1 - \frac{|n|}{N} \right) e_n(t) = \frac{1}{N} \frac{\sin^2(Nx/2)}{\sin^2(x/2)}. \]
%
so the oscillations of the Dirichlet kernel are slightly dampened, and as a result, $F_N$ is an approximation to the identity.

\begin{theorem}[Fej\'{e}r's Theorem] For any $f \in L^1(\mathbf{T})$,
    \begin{itemize}
        \item $(\sigma_N f)(x) \to f(x)$ for all $x$ in the Lebesgue set of $f$.
        \item $\sigma_N f \to f$ uniformly if $f \in C(\mathbf{T})$.
        \item $\sigma_N f \to f$ in the $L^p$ norm for $1 \leq p < \infty$, if $f \in L^p(\mathbf{T})$.
    \end{itemize}

\end{theorem}

\begin{corollary}
    If $\widehat{f} = 0$, then $f = 0$ almost everywhere.
\end{corollary}
\begin{proof}
    If $\widehat{f} = 0$, then $\sigma_N f = 0$ for all $N$. But $\sigma_N f \to f$ in $L^1$, which means that $f = 0$ in $L^1(\mathbf{T})$, so $f = 0$ almost everywhere.
\end{proof}

If we look at the Fourier expansion of the trigonometric polynomial $\sigma_N(f)$, we see that
%
\[ \sigma_N f = \sum_{n = -N}^N \frac{N-|n|}{N} \widehat{f}(n) e_n. \]
%
Thus the Fourier coefficients are slowly added to the expansion, rather than a sharp cutoff as with ordinary Dirichlet summation. This is one reason for the nice convergence properties the kernel has as compared to the Dirichlet kernel.

\section{Abel Summation and Harmonics on the Disk}

Relating Abel summations to Fourier series requires a little bit more careful work, since we do not consider limits of finite sums. Note that the Abel sum is
%
\[ A_r(f) = \sum_{n = -\infty}^\infty \widehat{f}(n) r^n e_n(t). \]
%
Thus, if we define the {\it Poisson kernel}
%
\[ P_r(t) = \sum_{n = -\infty}^\infty r^{|n|} e_n(t) \]
%
which is defined by a uniformly convergent series over $\mathbf{T}$, we calculate that $A_r(f) = P_r * f$. Thankfully, we find $P_r$ is a good kernel. To see this, we can apply an infinite geometric series summation to obtain that
%
\begin{align*}
    \sum r^{|n|} e_n(t) &= 1 + \frac{re(t)}{1 - re(t)} + \frac{re(-t)}{1 - re(-t)} = 1 + \frac{2r \cos 2 \pi t - 2r^2}{(1 - re(t))(1 - re(-t))}\\
    &= 1 + \frac{2r \cos 2\pi t - 2r^2}{1 - 2r \cos 2\pi t + r^2} = \frac{1 - r^2}{1 - 2r \cos 2 \pi t + r^2}.
\end{align*}
%
As $r \to 1$, the function concentrates at the origin, because as $r \to 1$, if $\delta \leq |t| \leq \pi$, then $1 - \cos 2\pi t$ is bounded away from the origin, so
%
\begin{align*}
    \left| \frac{1 - r^2}{1 - 2r \cos 2\pi t + r^2} \right| &= \left| \frac{1 + r}{(1+(1-2\cos 2\pi t)r) + 2(1 - \cos 2\pi t) r^2/(1-r)} \right|\\
    &= O \left( \frac{1 - r}{1 - \cos 2\pi t} \right) = O_\delta(1 - r).
\end{align*}
%
Thus the oscillation in the Poisson kernel cancels out as $r \to 1$, and so the Poisson kernel is a good kernel.

\begin{theorem}
    For any $f \in L^1(\mathbf{T})$,
    %
    \begin{itemize}
        \item $(A_r f)(t) \to f(t)$ for all $x$ in the Lebesgue set of $f$.
        \item $A_r f \to f$ uniformly if $f \in C(\mathbf{T})$.
        \item $A_r f \to f$ in the $L^p$ norm for $1 \leq p < \infty$, if $f \in L^p(\mathbf{T})$.
    \end{itemize}
\end{theorem}

The Poisson kernel is not a trigonometric polynomial, and therefore not quite as easy to work with as the F\'{e}jer kernel. However, it is the real part of the Cauchy kernel
%
\[ \frac{1 + re^{2 \pi it}}{1 - re^{2 \pi it}}, \]
%
and therefore links the study of trigonometric series and the theory of analytic functions. 

\begin{theorem}
    $u(r e^{2 \pi it}) = (A_r f)(t)$ is $C^\infty(\mathbf{T})$, and harmonic for $r < 1$. Moreover, the function $u$ is the \emph{unique} $C^2(\mathbf{T})$ harmonic function such that as $r \to 1$, $u(re^{2 \pi it}) \to f$ in the $L^1$ norm.
\end{theorem}
\begin{proof}
    The function $u$ is infinitely differentiable, because of the rapid convergence of the series defining the Poisson kernel. In particular, we note that
    %
    \[ \frac{\partial^2 u}{\partial \theta^2} = - \sum_{n = -\infty}^\infty \widehat{f}(n) |n|^2 r^{|n|} e_n(t), \]
    %
    \[ \frac{\partial u}{\partial r} = \sum_{n = -\infty}^\infty \widehat{f}(n) |n| r^{|n| - 1} e_n(t), \]
    %
    and
    %
    \[ \frac{\partial^2 u}{\partial r^2} = \sum_{n = -\infty}^\infty \widehat{f}(n) |n|(|n| - 1) r^{|n| - 2} e_n(t). \]
    %
    But in polar coordinates, we have
    %
    \begin{align*}
        \Delta u &= \frac{\partial^2 u}{\partial r^2} + \frac{1}{r} \frac{\partial u}{\partial r} + \frac{1}{r^2} \frac{u}{\theta^2}\\
        &= \sum_{|n| \geq 2} \widehat{f}(n) |n|(|n|-1) r^{|n|-2} e_n(t)\\
        &\ \ \ + \sum_{n \neq 0} \widehat{f}(n) |n| r^{|n|-2} e_n(t) - \sum_{n = -\infty}^\infty \widehat{f}(n) |n|^2 r^{|n|-2} e_n(t) = 0.
    \end{align*}
    %
    Thus $u$ is harmonic.

    Conversely, suppose $u \in C^2(\mathbf{T})$ is harmonic. Then we can find $a_n(t)$ such that
    %
    \[ u(re^{it}) = \sum_{n = -\infty}^\infty a_n(r) e_n(t), \]
    %
    where
    %
    \[ a_n(r) = \int_{\mathbf{T}} u(re^{it}) e_n(-t)\; dt. \]
    %
    Then
    %
    \[ \int_{\mathbf{T}} \frac{\partial^2 u}{\partial \theta^2}(re^{it}) e_n(-t)\; dt = -n^2 a_n(r), \]
    %
    and so $a_n''(r) + (1/r) a_n'(r) - (n^2/r^2) a_n(r) = 0$. This is an ordinary differential equation, whose only bounded solutions are given by $a_n(r) = A_n r^{|n|}$. If $u(re^{it}) \to f$ in the $L^1$ norm as $r \to 1$, then we conclude
    %
    \[ A_n = \lim_{r \to 1} \int_{\mathbf{T}} u(re^{it}) e_n(-t)\; dt = \int_{\mathbf{T}} f(t) e_n(-t) = \widehat{f}(n), \]
    %
    so
    %
    \[ u(re^{it}) = \sum \widehat{f}(n) r^{|n|} e_n(t) = g(re^{it}). \qedhere \]
\end{proof}

Thus we can represent any function on $\mathbf{T}$ as a harmonic function on the interior of the unit disk.

%If $u$ is only required to converge to $f$ {\it pointwise} on the boundary, then the function we found is no longer required to be unique. Below is an example of a function $u$ which tends to zero pointwise on the boundary, yet does not vanish on the interior of the unit disk.

%\begin{example}
%    If $P_r$ is the Poisson kernel, define $u(r,\theta) = P_r(\theta)$. Then $u$ is harmonic in the unit disk, because $\Delta u = (\Delta P_r)'' = 0$. We calculate
    %
%    \begin{align*}
%        u(r,t) &= \sum_{n = 1}^\infty in r^n [e_n(t) - e_n(-t)]\\
%        &= i \left[ \frac{r e(t)}{(re(t) - 1)^2} - \frac{r e(-t)}{(re(-t) - 1)^2} \right]\\
%        &= i \left[ \frac{re(-t) + r^{-1}e(t) - re(t) - r^{-1}e(-t)}{(re(t) - 1)^2(re(-t) - 1)^2} \right]\\
%        &= \frac{(r - r^{-1}) \sin(t)}{(re(t) - 1)^2(re(-t) - 1)^2}\\
%        &= \frac{(r^2 - 1) \sin(t)}{r (re(t) - 1)^2(re(-t) - 1)^2}
%    \end{align*}
    %
%    In this form, it is easy to see that for a fixed $t$, as $r \to 1$, $u(r,t) \to 0$. However, the denominator tells us this convergence isn't uniform.
%\end{example}

\section{The De la Valle\'{e} Poisson Kernel}

By taking a kernel halfway between the Dirichlet kernel and the Fejer kernel, we can actually obtain important results about ordinary summation. For two integers $M > N$, we define
%
\[ \sigma_{N,M}(f) = \frac{M\sigma_M(f) - N\sigma_N(f)}{M-N}. \]
%
If we take a look at the Fourier expansion of $\sigma_{n,m} f$, we find
%
\[ \sigma_{N,M} f = \sum_{n = -M}^M \frac{M - |n|}{M-N} e_n - \sum_{n = -N}^N \frac{N - |n|}{M-N} e_n = S_N f + \sum_{|n| = N+1}^M \frac{M - |n|}{M - N} e_n. \]
%
So we still have a slow decay in the Fourier coefficients. And as a result, if we look at the associated De la Velle\'{e} Poisson kernel, we find that a suitable subsequence is an approximation to the identity. In particular, for any fixed integer $k$, the sequence $\sigma_{kN,(k+1)N}$ leads to a good kernel. More interestingly, if the Fourier coefficients of $f$ have some decay, then the De la Vall\'{e}e does not differ that much from the ordinary sum, which gives useful results.

\begin{theorem}
    If $\widehat{f}(n) = O(|n|^{-1})$, then for any integers $N$ and $k$, if
    %
    \[ kN \leq M < (k+1)N, \]
    %
    then
    %
    \[ \| \sigma_{kN,(k+1)N} f - S_M f \|_{L^\infty(\mathbf{T})} \lesssim 1/k. \]
    %
    Where the implicit constant is independant of $N$ and $k$.
\end{theorem}
\begin{proof}
    We just calculate that, since the Poisson sum has essentially the same weight for low term coefficients as the sum $S_M f$,
    %
    \[ \| \sigma_{kN,(k+1)N} f - S_M f \|_{L^\infty(\mathbf{T})} \lesssim \sum_{kN \leq |n| < (k+1)N} |\widehat{f}(n)| \lesssim \sum_{n = kN}^{(k+1)N} \frac{1}{n} \leq \frac{N}{kN} = \frac{1}{k}. \qedhere \]
\end{proof}

\begin{corollary}
    If $f$ is a function with $\widehat{f}(n) = O(|n|^{-1})$,
    %
    \begin{itemize}
        \item $S_Nf$ converges to $f$ in the $L^p$ norm for $1 \leq p < \infty$.
        \item $S_Nf$ converges uniformly to $f$ if $f \in C(\mathbf{T})$.
        \item $(S_N f)(x) \to f(x)$ for each Lebesgue point $x$ of $f$.
    \end{itemize}
\end{corollary}
\begin{proof}
    The idea is quite simple. Fix $N$. Given any $\varepsilon$, we can use the last theorem to find $k$ large enough such that if $kN \leq M < k(N+1)$,
    %
    \[ \| \sigma_{kN,(k+1)N} f - S_M f \|_{L^\infty(\mathbf{T})} \leq \varepsilon. \]
    %
    But this gives the first and second result, up to perhaps a $\varepsilon$ of error. The latter result is given by similar techniques.
\end{proof}

\section{Pointwise Convergence}

One way around around the blowup in the $L^1$ norm of $D_N$ is to consider only functions $f$ which provide a suitable dampening condition on the oscillation of $D_N$ near the origin. This is provided by smoothness of $f$, manifested in various ways. The first thing we note is that the convergence of $(S_N f)(t)$ for a \emph{fixed} $x_0$ depends only \emph{locally} on the function $f$.

\begin{lemma}[Riemann Localization Principle]
    If $f_0$ and $f_1$ agree in an interval around $t_0$, then
    %
    \[ (S_N f_0)(t_0) = (S_N f_1)(t_0) + o(1). \]
\end{lemma}
\begin{proof}
    Let
    %
    \[ X = \{ f \in L^1(\mathbf{T}) : f(x) = 0\ \text{for almost every $x \in (t_0 - \varepsilon, t_0 + \varepsilon$)} \}. \]
    %
    Then $X$ is a closed subset of $L^1(\mathbf{T})$. Note that for all $x \in [-\pi,\pi]$,
    %
    \[ \sin(t/2) \gtrsim t \quad\text{and}\quad \sin((N+1/2)t) \leq 1. \]
    %
    Thus if $|t| \geq \varepsilon$,
    %
    \[ |D_N(t)| = \frac{|\sin(2 \pi (N+1/2)t)|}{|\sin(\pi t)|} \lesssim 1/\varepsilon. \]
    %
    In particular, by H\"{o}lder's inequality, the functionals $T_Nf = (S_N f)(t_0)$ are uniformly bounded on $X$, i.e. $\| T_N \| \lesssim 1/\varepsilon$. If $f$ is smooth, and vanishes on $(t_0 - \varepsilon, t_0 + \varepsilon)$, then $T_N f \to 0$ as $N \to \infty$. But the space of such functions is dense in $X$, which implies that $T_N f \to 0$ for \emph{any} $f \in X$. Thus if $f_0, f_1$ are two functions that agree in $(t_0 - \varepsilon, t_0 + \varepsilon)$, then $f_0 - f_1 \in X$, so $(S_N f_0)(t_0) = (S_N f_1)(t_0) + o(1)$. In particular, the pointwise convergence properties of $f_0$ and $f_1$ are equivalent at $t_0$.
\end{proof}

Thus any result about the pointwise convergence of Fourier series must depend on the local properties of a function $f$. Here, we give two of the main criteria, which corresponds to the smoothness of a function about a point $x$: either $f$ is in a sense, `locally Lipschitz', or `locally of bounded variation'.

\begin{theorem}[Dini's Criterion]
    If there exists $\delta$ such that
    %
    \[ \int_{|t| < \delta} \left| \frac{f(x+t) - f(x)}{t} \right|\; dt < \infty, \]
    %
    then $(S_N f)(x) \to f(x)$.
\end{theorem}
\begin{proof}
    Assume without loss of generality that $x = 0$ and $f(x) = 0$. Fix $\varepsilon > 0$, and pick $\delta_0$ such that
    %
    \[ \int_{|t| < \delta_0} \left| \frac{f(t)}{t} \right|\; dt < \varepsilon. \]
    %
    We have
    %
    \begin{align*}
        |(S_N f)(0)| &= \left| \left( \int_{|t| < \delta_0} + \int_{|t| \geq \delta_0} \right) f(t) D_N(t)\; dt \right|.
    \end{align*}
    %
    Now
    %
    \[ \int_{|t| \geq \delta_0} f(t) D_N(t)\; dt = (D_N * \left( \mathbf{I}_{|t| \geq \delta_0} f \right))(0) = S_N( \mathbf{I}_{|t| \geq \delta_0} f )(0) = o(1) \]
    %
    since $f \mathbf{I}_{|t| \geq \delta_0}$ vanishes in a neighbourhood of the origin. On the other hand, we note that $t/\sin(\pi t)$ is a bounded function on $\mathbf{T}$, so
    %
    \begin{align*}
        \int_{|t| < \delta_0} f(t) D_N(t)\; dt &= \int_{|t| < \delta_0} \left( \sin(2 \pi (N + 1/2)t) \frac{f(t)}{t} \right) \left( \frac{t}{\sin(\pi t)} \right)\; dt\\
        &\lesssim \| f(t)/t \|_{L^1[-\delta_0,\delta_0]} \leq \varepsilon.
    \end{align*}
    %
    Thus, for suitably large $N$, $|(S_N f)(0)| \lesssim \varepsilon$. Since $\varepsilon$ was arbitrary, the proof is complete.
\end{proof}

This proof applies, in particular, if $f$ is locally Lipschitz at $x$. Note the application of the Riemann Lebesgue lemma to show that to analyze the pointwise convergence of $(S_N f)(x)$, it suffices to analyze
%
\[ \lim_{N \to \infty} \int_{|t| < \delta} f(x+t) D_N(t)\; dt \]
%
for any fixed $\delta > 0$.

\begin{lemma}[Jordan's Criterion]
    If $f \in L^1(\mathbf{T})$ locally has bounded variation about $x$, then
    %
    \[ (S_N f)(x) \to \frac{f(x^+) + f(x^-)}{2}. \]
\end{lemma}
\begin{proof}
    By Riemann's localization principle, we may assume $f$ has bounded variation everywhere. Then without loss of generality, we may assume $f$ is an increasing function, since a bounded variation function is the difference of two monotonic functions. Since
    %
    \[ \int_{-1/2}^{1/2} D_N(t)\; dt = \int_0^{1/2} [f(x + t) + f(x - t)] D_N(t), \]
    %
    it suffices without loss of generality to show that
    %
    \[ \lim_{N \to \infty} \int_0^{1/2} f(x+t) D_N(t)\; dt = \frac{f(x+)}{2}. \]
    %
    Since $\int_0^{1/2} D_N(t) = 1/2$, this is equivalent to showing
    %
    \[ \lim_{N \to \infty} \frac{1}{2\pi} \int_0^\pi [f(x + t) - f(x+)] D_N(t)\; dt = 0. \]
    %
    Because of this, we may assume without loss of generality that $x = 0$ and $f(x+) = 0$. Then by the mean value theorem for integrals (which only applies for monotonic functions), for each $N$, there exists $0 \leq \nu_N \leq 1/2$ such that
    %
    \begin{align*}
        \int_0^{1/2} f(t) D_N(t)\; dt &= \| f \|_\infty \int_{\nu_N}^{1/2} D_N(t)\; dt.
    \end{align*}
    %
    Now an integration by parts gives
    %
    \begin{align*}
        \int_{\nu_N}^{1/2} D_N(t) &\lesssim \int_{\nu_N}^{1/2} \frac{\sin((N + 1/2) t)}{t}\; dt = \int_{\nu_N/(N + 1/2)}^{1/2(N + 1/2)} \frac{\sin(t)}{t}\; dt \lesssim \frac{1}{N+1/2}.
    \end{align*}
    %
    Thus
    %
    \[ \int_0^{1/2} f(t) D_N(t) \lesssim \frac{1}{N + 1/2} \to 0. \qedhere \]
\end{proof}

\begin{remark}
    The calculations in this proof also show that if $f \in L^1(\mathbf{T})$ has bounded variation, then
    %
    \[ \widehat{f}(n) = O(1/|n|). \]
    %
    We have seen that this implies $S_N f$ converges to $f$ at every point on the Lebesgue set of $f$, $S_N f$ converges uniformly to $f$ if $f \in C(\mathbf{T})$, and for any $1 \leq p < \infty$, if $f \in L^p(\mathbf{T})$, $S_N f$ converges to $f$ in $L^p(\mathbf{T})$. Dirichlet's theorem says that the Fourier series of a continuous function $f$ with only finitely many maxima and minima converges uniformly to $f$ everywhere. Such a function has bounded variation, and so Dirichlet's theorem is an easy consequence of our discussion.
\end{remark}

Of course, applying various better decay rates leads to a more uniform version of this theorem. The decay of the Fourier series depends on the decay of the Fourier coefficients of $yg(y)$ and $g(y) \cos(y/2)(y/\sin(y/2))$. In particular, if these coefficients is $O(|n|^{-m})$, then the convergence rate is also $O(|n|^{-m})$. If this decay rate is independent of $x$ for suitable values of $x$, the convergence will be uniform over these values of $x$.

\begin{example}
    Consider the sawtooth function defined on $[-1/2,1/2)$ by $s(t) = t$, and then made periodic on the entire real line. We can easily calculate the Fourier series here, obtaining that
    %
    \[ s(t) = i \sum_{n \neq 0} \frac{(-1)^n e_n(t)}{2 \pi n} = -2 \sum_{n = 1}^\infty \frac{(-1)^n \sin(2 \pi nt)}{n}. \]
    %
    Thus for any $t \in (-1/2,1/2)$,
    %
    \[ \sum_{n = 1}^\infty \frac{(-1)^n \sin(2 \pi nt)}{n} = -t/2. \]
\end{example}

\begin{theorem}
    If $\widehat{f}(n) = O(|n|^{-1})$, and $f(t_0-)$ and $f(t_0+)$ exist, then
    %
    \[ (S_N f)(t_0) \to \frac{f(t_0-) + f(t_0+)}{2}. \]
\end{theorem}
\begin{proof}
    The idea of our proof is to break $f$ into a nice continuous function, and the sawtooth function, where we already understand the convergence of Fourier series. Without loss of generality, let $t_0 = 1/2$. Define $g(t) = f(t) + (f(1+) - f(1-)) s(t) / 2$ on $(-1/2,1/2)$, where $s$ is the sawtooth function. Then
    %
    \[ \lim_{t \uparrow 1/2} g(t) = \lim_{t \downarrow -1/2} g(t) = \frac{f(1/2+) + f(1/2-)}{2}. \]
    %
    Thus $g$ can be defined on $\mathbf{T}$ so it is continuous at $t_0$. Now we find $|\widehat{g}| \lesssim |\widehat{f}| + |\widehat{s}| = O(|n|^{-1})$, and so
    %
    \[ (S_N g)(1/2) \to \frac{f(1/2+) + f(1/2-)}{2}. \]
    %
    We also have $(S_N s)(1/2) \to 0$. Thus
    %
    \[ (S_N f)(1/2) = (S_N g)(1/2) - (S_N s)(1/2) \to \frac{f(1/2+) + f(1/2-)}{2}. \qedhere \]
\end{proof}

% Another way to fix the convergence is to use a more quantitative argument in terms of $L^p$ spaces. It is obvious that $S_N f \to f$ in any feasible norm if $f$ is a trigonometric polynomial, because if $f$ has degree $M$, then $S_N f = f$ for $N \geq M$. The Stone-Weirstrass theorem says that we can uniformly approximate any continuous function on $\mathbf{T}$ by a trigonometric polynomial, so provided we can show that the operators $S_N$ are uniformly bounded in the $L^p$ norm for $1 \leq p < \infty$, we obtain convergence for all $f \in L^p(\mathbf{T})$. The fact that the $S_N$ are not bounded in the $L^\infty$ norm is why the Fourier series can diverge pointwise for continuous functions. In fact, the $S_N$ are not bounded as operators on $L^1(\mathbf{T})$, and as such, Fourier series do not converge in the $L^1$ norm. The reason for this is that if $\{ K_M \}$ is a good kernel, then $S_N(K_M) = D_N * K_M \to D_N$ as $M \to \infty$, and so as $M \to \infty$, we find $\| S_N(K_M) \|_{L^1(\mathbf{T})} = \Omega(\log N)$, hence $\| S_N \|_{L^1(\mathbf{T})}$ is unbounded. Later on, using the theory of conjugate functions, we will show that the operators $S_N$ are uniformly bounded in all $L^p(\mathbf{T})$ for $1 < p < \infty$, and so the Fourier series of any function $f \in L^p(\mathbf{T})$ converges to $f$ in the $L^p$ norm.

\section{Pointwise Behaviour at Discontinuity Points}

This isn't the end of our discussion about points of discontinuity. There is an interesting phenomenon which occurs locally around the point of discontinuity. If $f$ is continuous locally around a discontinuity point $t_0$, $S_N f \to f$ pointwise locally around $t_0$. Thus, being continuous, $S_N f$ must `jump' from $(S_N f)(t_0-)$ to $(S_N f)(t_0+)$ locally around $t_0$. Interestingly enough, we find that the jump is not precise, the jump is overshot and then must be corrected to the left and right of $t_0$. This is known as the {\it Gibb's phenomenon}, after the man who clarified the reason for why this phenomenon occured in physical measurements where first thought to be a defect in the equipment used to take the measurements. Gibb's phenomenon is one instance where a series of functions $\{ f_k \}$ converges pointwise to some function $f$, whereas qualitatively with respect to the $L^\infty$ norm, $\{ f_k \}$ does not converge to $f$.

\begin{theorem}
    Given $f$ with finitely many discontinuity points and with $\widehat{f}(n) = O(|n|^{-1})$, in particular one at $t_0$, we find
    %
    \[ \lim_{N \to \infty} (S_N f)(t_0 \pm 1/N) = f(t_0 \pm ) \pm C \cdot \frac{f(t_0+) - f(t_0-)}{2}, \]
    %
    where
    %
    \[ C = 2 \pi \int_0^\pi \frac{\sin x}{x} \approx 16.610. \]
\end{theorem}
\begin{proof}
    First consider the jump function $s$, with $t_0 = 1/2$. Then
    %
    \begin{align*}
        (S_N s)(1/2 + 1/N) &= -2 \sum_{n = 1}^N \frac{\sin(2 \pi n/N)}{n} = -2  \sum_{n = 1}^N \frac{2 \pi }{N} \left( \frac{\sin(2 \pi n/N)}{2 \pi n / N} \right).
    \end{align*}
    %
    Here we're just taking averages of values of $\sin(x)/x$ at $x = 2\pi/N$, $x = 4\pi/N$, and so on and so forth up to $x = 2 \pi$. Thus is a Riemann sum, so as $N \to \infty$, we get that
    %
    \[ (S_N s)(\pi + 1/N) \to - 2 \int_0^{2\pi} \frac{\sin x}{x}. \]
    %
    The same calculations give
    %
    \[ (S_N s)(\pi - \pi/N) \to 2 \pi \int_0^\pi \frac{\sin x}{x}. \]
    %
    In general, given $f$, we can write $f = g + \sum \lambda_j h_j$, where $g$ is continuous, and $h_j$ is a translate of the sawtooth function. Then $S_N g$ converges to $g$ uniformly, and $S_N h_j \to 0$ for all $h_j$ uniformly in an interval outside of their discontinuity point. To see this, we note that an integration by parts gives
    %
    \[ \left| \int_{-\pi}^\pi D_N(y)[s(x-y) - s(x)]\; dy \right| \leq |G_N(x - \pi)|, \]
    %
    where $G_N(y) = -i \sum_{|n| \leq N} e_n(t)/n$, so $G_N' = D_N$. It now suffices to show $G_N(x - \pi) \to 0$ outside a neighbourhood of $\pi$. But if $A(u,t) = \sum_{|n| \leq u} e_n(t)$, summation by parts gives
    %
    \[ \sum_{|n| \leq N} \frac{e_n(t)}{n} = \frac{A(N,t)}{N} + \int_1^N \frac{A(u,t)}{u^2}. \]
    %
    Now a simple geometric sum shows $A(u,t) \lesssim 1/|e(t) - 1|$, so provided $d(t, 2 \pi \mathbf{Z})$ is bounded below, the quantity above tends to zero uniformly. This gives the required result.
\end{proof}

\chapter{Applications of Fourier Series}

\section{Tchebychev Polynomials}

If $f$ is everywhere continuous, then for every $\varepsilon$, Fej\'{e}r's theorem says that we can find $N$ such that $\| \sigma_N(f) - f \| \leq \varepsilon$. But $\sigma_N f$ is just a trigonometric polynomial, and so we have shown that with respect to the $L^\infty$ norm, the space of trigonometric polynomials is dense in the space of all continuous functions.  Now if $f$ is a continuous function on $[0,\pi]$, then we can extend it to be even and $2\pi$ periodic, and then the trigonometric series $S_N(f)$ of $f$ will be a cosine series, hence $\sigma_N(f)$ will also be a cosine series, and so for each $\varepsilon$, we can find $N$ and coefficients $a_1, \dots, a_N$ such that
%
\[ \left| f(x) - \sum_{n = 1}^N a_n \cos(nx) \right| < \varepsilon. \]
%
Now we use a surprising fact. For each $n$, there exists a degree $n$ polynomial $T_n$ such that $\cos(nx) = T_n(\cos x)$. This is clear for $n = 0$ and $n = 1$. More generally, we can write
%
\begin{align*}
    \cos((m+1)x) &= \cos((m+1)x) + \cos((m-1)x) - \cos((m-1)x)\\
    &= \cos(mx + x) + \cos(mx - x) - \cos((m-1)x)\\
    &= 2 \cos x \cos(mx) - \cos((m-1)x).
\end{align*}
%
Thus we have the relation  $T_{m+1}(x) = 2xT_m(x) - T_{m-1}(x)$. These polynomials are known as {\bf Tchebyshev polynomials}, enabling us to move between `periodic coordinates' and standard Euclidean coordinates.

\begin{corollary}[Weirstrass]
    The polynomials are uniformly dense in $C[0,1]$.
\end{corollary}
\begin{proof}
    If $f$ is a continuous function on $[0,1]$, we can define $g(t) = f(|\cos(t)|)$. Then $g$ is even, and so for every $\varepsilon > 0$, we can find $a_1, \dots, a_N$ such that
    %
    \[ \left|g(t) - \sum_{n = 1}^N a_n \cos(nt) \right| = \left| g(t) - \sum_{n = 1}^N a_n T_n(\cos t) \right| < \varepsilon. \]
    %
    But if $x = \cos t$, for $\cos t \geq 0$, this equation says
    %
    \[ \left| f(x) - \sum_{n = 1}^N a_n T_n(x) \right| < \varepsilon, \]
    %
    and so we have uniformly approximated $f$ by a polynomial.
\end{proof}

\section{Exponential Sums and Equidistribution}

The next result uses Fourier analysis to characterize the asymptotic distribution of a certain sequence $a_1, a_2, \dots$. In particular, it is most useful in determining when this distribution is distributed when we consider $2 \pi a_1, 2 \pi a_2, \dots$ as elements of $\mathbf{T}$, i.e. so we only care about the fractional part of the numbers, or in other terms their behaviour modulo one. We say the sequence is {\it uniformly distributed} if for any interval $I \subset \mathbf{T}$, $\# \{ 2 \pi a_n \in I : n \leq N \} \sim N |I|$ as $N \to \infty$. By approximating continuous functions by step functions, this implies that if $f: \mathbf{T} \to \mathbf{C}$ is continuous, then
%
\[ \frac{f(2 \pi a_1) + \dots + f(2 \pi a_N)}{N} \to \int_{\mathbf{T}} f(t)\; dt. \]
%
It is the right hand side to which we can apply Fourier summation to obtain a very useful condition. We let $S_Nf$ denote the left hand side of the equation, and $Tf$ the right hand side.

\begin{theorem}[Weyl Condition]
    A sequence $a_1, a_2, \dots \in \mathbf{T}$ is uniformly distributed if and only if for every $n$, as $N \to \infty$, $e_n(2 \pi a_1) + \dots + e_n(2 \pi a_N) = o(N)$.
\end{theorem}
\begin{proof}
    The condition in the theorem implies that for any trigonometric polynomial $f$, $S_Nf \to Tf$. The $S_N$ are uniformly bounded as functions on $L^\infty(\mathbf{T})$, and $T$ is a bounded functional on this space as well. But this means that $\lim S_N f = T f$ for all $f$ in $C(\mathbf{T})$, since this equation holds on the dense subset of trigonometric polynomials.
\end{proof}

This technique enables us to completely characterize the equidistribution behaviour of arithmetic sequences. Given a particular $\gamma$, we consider the equidistribution of the sequence $\gamma, 2 \gamma, \dots$, which depends on the irrationality of $\gamma$.

\begin{example}
    Let $\gamma$ be an arbitrary real number. Then for any $n$, if $e_n(2 \pi \gamma) \neq 1$,
    %
    \[ \sum_{m = 1}^N e_n(2 \pi m \gamma) = \frac{e_n(2 \pi (N + 1) \gamma) - 1}{e_n(2 \pi \gamma) - 1} \lesssim 1 = o(N). \]
    %
    If $\gamma$ is an irrational number, then $e_n(2 \pi \gamma) \neq 1$ for all $n$, which implies that $\gamma, 2\gamma, \dots$ is equidistributed. Conversely, if $e_n(2 \pi \gamma) = 1$ for some $n$, we have
    %
    \[ \sum_{m = 1}^N e_n(a_m) = N. \]
    %
    which is not $o(N)$, so the sequence $\gamma, 2\gamma, \dots$ is {\it not} equidistributed. If $\gamma$ is rational, there certainly is $n$ such that $n \gamma \in \mathbf{Z}$, and so $e_n(2 \pi \gamma) = 1$.
\end{example}

On the other hand, it is still an open research to characterize, for which $\gamma$ the sequence $\gamma, \gamma^2, \gamma^3, \dots$ is equidistributed. Here is an example showing that there are $\gamma$ for which the sequence is not equidistributed.

\begin{example}
    Let $\gamma$ be the golden ratio $(1 + \sqrt{5})/2$. Consider the sequence
    %
    \[ a_n = \left( \frac{1 + \sqrt{5}}{2} \right)^n + \left( \frac{1 - \sqrt{5}}{2} \right)^n = b_n + c_n. \]
    %
    Then one checks that $a_n$ is a kind of Fibonacci sequence, with $a_{n+1} = a_n + a_{n-1}$, and initial conditions $a_0 = 2$, $a_1 = 1$. One checks that $c_n$ is always negative for odd $n$, and positive for even $n$, and tends to zero as $n \to \infty$. Since $a_n$ is an integer, this means that $d(b_n, \mathbf{Z}) = d(\gamma^n, \mathbf{Z}) \to 0$. But this means that the average distribution of the $\gamma^n$ modulo one is concentrated at the origin.
\end{example}

\section{The Isoperimetric Inequality}

TODO

\section{Heat Propagation Into the Ground}

Let us consider an application of the Fourier series taken from Fourier's original work. Consider heat moving from above ground to below ground, and vice versa. If we let $H(t,y)$ denote the temperature at a depth $y$ into the ground at time $t$, for $y > 0$. Assuming that the material of the ground is homogenous, by choosing appropriate units, the differential equation becomes $H_t = H_{yy}$, a variant of the heat equation. We assume that the heat at the surface changes periodically over the days and seasons, so
%
\[ H(t,0) = A \cos(2\pi t / D) + B \cos(2 \pi t/Y) + C, \]
%
where $A,B,C$ are arbitrary constants, $D$ is the length of a day, and $Y$ is the length of a year, so $Y = 365 D$. In our calculation, we assume the regularity condition that $H \in L^\infty [0,\infty)^2$, so the temperature does not magnify infinitely at large depths or large times.

To solve this equation, we use two tricks: linearity, and Fourier series. We can solve the heat equation by solving the three heat equations with initial conditions $H_D(t,0) = \cos(2\pi t/D)$, $H_Y(t,0) = \cos(2 \pi t/Y)$, and $H_C(t,0) = 1$, and then obtain a general solution by letting $H = A H_D + B H_Y + C H_C$. The third equation is easiest: we let $H_C(t,y) = 1$ for all $t$ and $y$. To solve the other equations, we can use variable separation. Assuming $H_D$ and $H_Y$ are bounded, this means we have
%
\[ H_D(t,y) = \cos((2 \pi /D) t - (\pi / D)^{1/2} y) e^{- (\pi / D)^{1/2} y}, \]
\[ H_Y(t,y) = \cos((2\pi/Y)t - (\pi/Y)^{1/2} y) e^{- (\pi/Y)^{1/2} y}. \]
%
Thus the temperature in the ground splits into a daily heating effect $H_D$, a seasonal heating effect $H_Y$, and a constant temperature $H_C$. From these equations we get several interesting qualitative properties. As we go deeper into the ground, the temperature decays at a rate inversely dependant on the length of time, so even at small depths, the daily temperature becomes neglible, and only the seasonal temperature is important. Experimently, determining the constants in our equation, we determine this happens about half a foot into the ground. Next, the deeper we go in the ground, the more a `time lag' exists, where the seasonal temperature back in time has now travelled to the temperature at the current point in the ground. Experimentally, we determine that about 2-3 metres below ground, the temperature lags by six months. Fourier mentions this is a good depth to build a wine cellar which is cool during the summer months.

\section{Seafaring with Fourier}

Here we discuss two problems in seafaring that can be solved quite accurately with Fourier analysis, first done by Kelvin in the late 1800s. Consider first the problem in determining the error of compass measurement on a ship when taking an initial bearing at harbor travelling. Thus for each angle $\theta$, we consider an error $g(\theta)$ such that if, at an angle $\theta$, we take a measurement $f(\theta)$, then $f(\theta) = \theta + g(\theta)$. Often $g$ is up to 20 degrees, but it will suffice to know $g$ up to an angle of two or three degrees, since other systematic errors in travel disturb the angle the ship actually travels by this amount anyway. And thus experimentally we find it suffices to approximate $g$ by a degree four trigonometric polynomial, i.e. we subtitute $g$ for an approximate value
%
\[ g_1(\theta) = A_0 + A_1 \cos \theta + B_1 \sin \theta + A_2 \cos(2\theta) + B_2 \sin(2\theta). \]
%
We can obtain measurements $g(\theta)$ for certain values of $\theta$ by locating landmarks, and 6 measurements suffice to uniquely identify $g_1$ from all other degree five trigonometric polynomials.

Another seafaring problem is to determine the future height of the tide. We expect the height of the tides to be due to periodic forces in nature. If $h(t)$ is the height of the tide, we might expect by linearity of the wave equation that $h(t) = h_1(t) + h_2(t) + \dots$, where $h_1(t)$ is the height with relation to the rotation of the earth and the moon, $h_2(t)$ the height with respect to the rotation of the earth and the sun, and so on and so forth to more neglible values. Each $h_k$ is periodic with some period $\omega_k$. If we assume that each $h_k$ is a trigonometric polynomial, then there is a way to reduce the calculation of the coefficients to a certain integral formula which one can approximate by taking samples of the height of the tides over time. Unfortunately, one must take a large number of samples to obtain this integral formula, but Kelvin designed one of the first automated calculators to approximate this without hard work on the part of the navigator.

\begin{theorem}
    If $h(t) = \sum_{n = 1}^N A_n \cos(\omega_n t)$, where $\omega_1, \dots, \omega_n$ are distinct, then for any $S$,
    %
    \[ A_n = \lim_{T \to \infty} \frac{2}{T} \int_S^{S + T} h(t) \cos(\omega_n t)\; dt. \]
\end{theorem}
\begin{proof}
    We just change variables. If $2 \pi N / \omega_n < T \leq 2 \pi (N + 1)/\omega_n$,
    %
    \begin{align*}
        \int_S^{S+T} h(t) \cos(\omega_n t)\; dt &= N \int_0^{2 \pi/ \omega_n} h(t) \cos(\omega_n t)\; dt + O(1)\\
        &= \frac{N}{\omega_n} \int_0^{2 \pi} \left( \frac{1}{N} \sum_{n = 1}^N h(S + t/\omega_n + 2 \pi k / \omega_n) \right) \cos(t)\; dt + O(1).
    \end{align*}
    %
    We calculate that
    %
    \[ \frac{1}{N} \sum_{n = 1}^N h(S + t/\omega_n + 2 \pi k / \omega_n) = A_n \cos(t) + O(1), \]
    %
    and so
    %
    \[ \frac{2}{T} \int_S^{S+T} h(t) \cos(\omega_n t)\; dt = A_n + O(1/T). \]
    %
    and we then take $T \to \infty$.
\end{proof}








\chapter{The Fourier Transform}

In the last few chapters, we discussed the role of analyzing the frequency decomposition of a periodic function on the real line. In this chapter, we explore the ways in which we may extend this construction to perform frequency analysis for not necessarily periodic functions on the real line, and more generally, in higher dimensional Euclidean space. The only periodic trigonometric functions on $[0,1]$ on the real line had integer frequencies of the form $2\pi n$, whereas on the real line periodic functions can have frequencies corresponding to any real number. The analogue of the discrete Fourier series formula
%
\[ f(x) = \sum_{k = -\infty}^\infty \widehat{f}(k) e(kx) \]
%
is the Fourier inversion formula
%
\[ f(x) = \int_{-\infty}^\infty \widehat{f}(\xi) e(2 \pi \xi x)\; d\xi, \]
%
where for each real number $\xi$, we define
%
\[ \widehat{f}(\xi) = \int_{-\infty}^\infty f(x) e(- 2 \pi \xi x)\; dx. \]
%
The function $\widehat{f}$ is known as the {\bf Fourier transform} of the function $f$. It is also denoted by $\mathcal{F}(f)$. The role to which we can justify this formula is the main focus of this chapter. Without too much more work, we will also analyze the Fourier transform on $\RR^n$, which, given $f: \RR^n \to \RR^n$, considers the quantities
%
\[ f(x) \sim \int_{\RR^n} \widehat{f}(\xi) e(\xi \cdot x)\ d\xi,\quad\text{where}\quad \widehat{f}(\xi) = \int_{\RR^n} f(x) e(- \xi \cdot x)\ dx \]
%
for $\xi \in \RR^n$, where $e(t) = \exp(2 \pi i t)$ for any $t \in \RR$. The basic theory is the same, though as $n$ increases the transform certainly becomes more complicated. In particular, the issue of pointwise convergence becomes more difficult to understand.

Later, we will interpret the Fourier transform in a very general manner for a very arbitrary class of functions. But first we must interpret the Fourier transform as a Lebesgue integral, and the weakest assumptions we can make in order to do this are that $f$ is an integrable function, i.e. that $f \in L^1(\RR^d)$. During arguments, we can often assume additional regularity properties of $f$, and then apply density arguments to get the result in general. Most of the properties of the Fourier transform are exactly the same as for Fourier series. The only new phenomenon in the basic theory is that the Fourier transform of an integrable function is continuous.

\begin{theorem}
    For any $f \in L^1(\RR^d)$, $\smash{\| \widehat{f} \|_{L^\infty(\RR^d)} \leq \| f \|_{L^1(\RR^d)}}$, and $\widehat{f} \in C_0(\RR^d)$.
\end{theorem}
\begin{proof}
    For any $\xi \in \RR^d$,
    %
    \[ |\widehat{f}(\xi)| = \left| \int f(x) e(- \xi \cdot x)\; dx \right| \leq \int |f(x)| |e(- \xi \cdot x)|\; dx = \| f \|_{L^1(\RR^d)}. \]
    %
    If $\chi_I$ is the characteristic function of an $n$ dimensional box, i.e.
    %
    \[ I = [a_1,b_1] \times \dots \times [a_n,b_n] = I_1 \times \dots \times I_n, \]
    %
    then
    %
    \[ \widehat{\chi_I}(\xi) = \int_I e(- \xi \cdot x) = \prod_{k = 1}^n \int_{a_k}^{b_k} e(- \xi_k x_k) = \prod_{k = 1}^n \widehat{\chi_{I_k}}(\xi_k). \]
    %
    where
    %
    \[ \widehat{\chi_{I_k}}(\xi_k) = \begin{cases} \frac{e(- \xi_k a_k) - e(- \xi_k b_k)}{2 \pi i \xi_k} & \xi_k \neq 0, \\ b_k - a_k & \xi_k = 0. \end{cases} \]
    %
    L'Hopital's rule shows $\widehat{\chi_{I_k}}$ is a continuous function. We also have the upper bound
    %
    \[ \widehat{\chi_{I_k}}(\xi_k) \lesssim_{I_k} (1 + |\xi_k|)^{-1} \]
    %
    for all $\xi_k \in \RR$, which implies that
    %
    \[ \widehat{\chi_I}(\xi) = \prod \widehat{\chi_{I_k}}(\xi_k) \lesssim_I \prod \frac{1}{1 + |\xi_k|} \lesssim_n \frac{1}{1 + |\xi|}. \]
    %
    Thus $\widehat{\chi_I}(\xi) \to 0$ as $|\xi| \to \infty$. But this implies the Fourier transform of any step function is continuous and vanishes at $\infty$. Since step functions are dense in $L^1(\RR^d)$, a density argument then gives the result for all integrable functions.
\end{proof}

\begin{remark}
    The space
    %
    \[ \mathbf{A}(\RR^d) = \left\{ \widehat{f}: f \in L^1(\RR^d) \right\} \]
    %
    is called the \emph{Fourier algebra}. The last theorem shows $\mathbf{A}(\RR^d) \subset C_0(\RR^d)$, but it is {\it not} the case that $\mathbf{A}(\RR^d) = C_0(\RR^d)$. As of yet, current research cannot give a satisfactory description of the elements of $\mathbf{A}(\RR^d)$.
\end{remark}

The next lemma will be used to show $\mathbf{A}(\RR^d) \neq C_0(\RR^d)$.

\begin{lemma}
    For any $0 \leq a < b < \infty$, independantly of $a$ and $b$,
    %
    \[ \left| \int_a^b \frac{\sin x}{x} \right| = O(1). \]
\end{lemma}
\begin{proof}
    Since $\| \sin(x)/x \|_{L^\infty(\RR)} \leq 1$, we may assume $b > 1$, for otherwise we obtain a trivial bound. This also implies
    %
    \begin{align*}
        \left| \int_a^b \frac{\sin x}{x}\; dx \right| \leq 1 + \left| \int_1^b \frac{\sin x}{x}\; dx \right|.
    \end{align*}
    %
    An integration by parts then shows that
    %
    \[ \left| \int_1^b \frac{\sin x}{x}\; dx \right| \leq \left| \left( \cos 1 - \frac{\cos b}{b} \right) \right| + \left| \int_1^b \frac{\cos x}{x^2}\; dx \right| \lesssim 1. \qedhere \]
\end{proof}

\begin{theorem}
    $\mathbf{A}(\RR) \neq C_0(\RR)$. In particular, $\mathbf{A}(\RR)$ does not contain any odd functions $g$ in $C_0(\RR)$ such that
    %
    \[ \limsup_{b \to \infty} \left| \int_1^b \frac{g(\xi)}{\xi}\; d\xi \right| = \infty. \]
\end{theorem}
\begin{proof}
    Suppose $f \in L^1(\RR)$, and $\widehat{f} \in C_0(\RR)$ is an odd function. Then we know
    %
    \[ \widehat{f}(\xi) = -i \int_{-\infty}^\infty f(x) \sin(2 \pi \xi x)\; dx. \]
    %
    If $b \geq 1$, an application of Fubini's theorem shows that
    %
    \[ \left| \int_1^b \frac{\widehat{f}(\xi)}{\xi}\; d\xi \right| = \left| \int_{-\infty}^\infty f(x) \left( \int_1^b \frac{\sin(2 \pi \xi x)}{\xi}\; d\xi \right)\; dx \right|. \]
    %
    But
    %
    \[ \left| \int_1^b \frac{\sin(2 \pi \xi x)}{\xi}\; d\xi \right| = \left| \int_{2 \pi x}^{2 \pi b x} \frac{\sin \xi}{\xi}\; d\xi \right| \lesssim 1. \]
    %
    Thus we obtain that
    %
    \[ \left| \int_1^b \frac{\widehat{f}(\xi)}{\xi}\; d\xi \right| \lesssim \| f \|_{L^1(\RR)}. \]
    %
    For instance, this implies that there is no $f \in L^1(\RR)$ such that
    %
    \[ \widehat{f}(\xi) = \left| \frac{\sin(2 \pi \xi)}{\log | \xi |} \right| \]
    %
    for all $\xi \in \RR$.
\end{proof}

Elementary properties of integration give the following relations among the Fourier transforms of functions on $\RR^d$. They are strongly related to the translation invariance of the Lebesgue integral on $\RR^d$:
%
\begin{itemize}
    \item If $\overline{f}(x) = \overline{f(x)}$ is the conjugate of a function $f$, then
    %
    \[ (\overline{f})^\ft(\xi) = \int \overline{f(x)} e(- x \cdot \xi)\; dx = \overline{\int f(x) e(x \cdot \xi)} = \overline{\widehat{f}(-\xi)}. \]
    %
    If $f$ is real, the formula above says $\widehat{f}(\xi) = \overline{\widehat{f}(-\xi)}$, and so if we define $a(\xi) = \text{Re}(\widehat{f}(\xi))$, $b(\xi) = \text{Im}(\widehat{f}(\xi))$, then formally we have
    %
    \[ \int_{-\infty}^\infty \widehat{f}(\xi) e(\xi x)\; d\xi = 2 \int_0^\infty a(\xi) \cos(2 \pi \xi x) - b(\xi) \sin(2 \pi \xi x)\; d\xi. \]
    %
    Thus the Fourier representation formula expresses the function $f$ as an integral in sines and cosines.
    
    \item There is a duality between translation and frequency modulation. For $y \in \RR^d$, we define $(T_y f)(x) = f(x - y)$. If $\xi \in \RR^d$, then we define $(M_\xi f)(x) = e(\xi \cdot x) f(x)$. We then find that
    %
    \begin{align*}
        \widehat{T_y f}(\xi) &= \int f(x - y) e(- \xi \cdot x)\; dx\\
        &= e(- \xi \cdot y) \int f(x) e(- \xi \cdot x)\; dx = (M_{-y} \widehat{f})(\xi).
    \end{align*}
    %
    and
    %
    \begin{align*}
        \widehat{M_\xi f}(\eta) = \int e(\xi \cdot x) f(x) e(- \eta \cdot x)\; dx = \widehat{f}(\eta - \xi) = (T_\xi \widehat{f})(\eta).
    \end{align*}
    %
    Thus we conclude $\mathcal{F} \circ T_y = M_{-y} \circ \mathcal{F}$, and $\mathcal{F} \circ M_\xi = T_\xi \circ \mathcal{F}$.

    \item Let $T: \RR^d \to \RR^d$ be an invertible linear transformation. Then a change of variables $y = Tx$ gives
    %
    \begin{align*}
        \widehat{f \circ T}(\xi) &= \int f(Tx) e(-\xi \cdot x)\; dx\\
        &= \frac{1}{|\det(T)|} \int f(y) e(- \xi \cdot T^{-1}y)\; dy\\
        &= \frac{1}{|\det(T)|} \int f(y) e(- T^{-t} \xi \cdot y)\; dy\\
        &= \frac{1}{|\det(T)|} (\widehat{f} \circ T^{-t})(\xi).
    \end{align*}
    %
    Thus we conclude that if $T^*(f) = f \circ T$, then $\mathcal{F} \circ T^* = |\det(T)|^{-1} (T^{-t})^* \circ \mathcal{F}$.

    \item As a special case of the theorem above, if $a \in \RR$ and $(D_a f)(x) = f(ax)$, then
    %
    \[ \widehat{D_a f}(\xi) = a^{-d} \widehat{f}(\xi/a) \]
    %
    If we dilate by a small value of $a$, then the values of $f$ are traced over more slowly, so $D_a f$ has smaller frequencies. But the magnitude of these frequencies is increase to compensate.

    \item If $R \in O_n(\RR)$, then $\widehat{f \circ R}(\xi) = \widehat{f}(R \xi)$, i.e. $\mathcal{F} \circ R^* = R^* \circ \mathcal{F}$. In particular, if $f$ is a radial function, so $f \circ R = f$ for any $R$, then $\widehat{f}(R \xi) = \widehat{f}(\xi)$ for any $R \in O_n(\RR)$, so $\widehat{f}$ is also a radial function. If $f$ is even, so $f(x) = f(-x)$ for all $x$, then $\widehat{f}(\xi) = \widehat{f}(-\xi)$ for all $\xi$, so $\widehat{f}$ is even. Similarily, if $f$ is odd, then $\widehat{f}$ is odd.

    \item Given $f,g \in L^1(\RR^d)$, we define the convolution
    %
    \[ (f * g)(x) = \int f(y) g(x-y)\; dy. \]
    %
    This convolution possesses precisely the same properties as convolution on $\mathbf{T}$. Most importantly for us,
    %
    \[ \mathcal{F}(f * g) = \mathcal{F}(f) \cdot \mathcal{F}(g), \]
    %
    so convolution in phase space is just a product in frequency space.
\end{itemize}

Just as with Fourier series, we have a duality between decay of a function and smoothness of it's transform. We say $f$ has a {\bf strong derivative} $f_k$ in $L^p(\RR^d)$ if the family of functions
%
\[ (\Delta_h f)(x) = \frac{f(x + h e_k) - f(x)}{h} \]
%
converge in $L^p(\RR^d)$ to $f_k$. Essentially, this means that the approximations of $f$ to it's derivative quantitatively converge in the mean. If $f$ has a strong derivative in $L^p(\RR^d)$, then $f$ is actually differentiable almost everywhere with derivative $f_k$. However, even if $f$ has a pointwise partial derivative $f_k$, the differences $\Delta_h f$ may not converge to $f_k$ fast enough to conclude that $f$ has a strong derivative. It is fairly easy to prove using the mean value theorem that if $\delta_h f$ converges to $f_k$ in the $L^\infty$ norm, and $f$ has compact support, then $f$ has a strong derivative in all other $L^p$ spaces. If $f$ is not compactly supported, but decays rapidly at $\infty$, then the classical derivative may be a strong derivative. In particular, this is true of a Schwartz function, i.e. a function lying in the space
%
\[ \mathcal{S}(\RR^d) = \{ f \in C^\infty(\RR^d): |(D_\alpha f)(x)| \lesssim_{\alpha,N} |x|^{-N}\ \text{for all $N, \alpha, x$} \} \]
%
which is often a natural space to consider the relation of the Fourier transform to various analytical operations.

\begin{theorem}
    If $f \in L^1(\RR^d)$, and $x_k f \in L^1(\RR^d)$, then $\widehat{f}$ has a strong derivative in the $L^\infty$ norm, and $\widehat{f}_k(\xi) = - 2 \pi i (x_k f)^\ft(\xi)$.
\end{theorem}
\begin{proof}
    Note that a change of variables implies
    %
    \[ (\Delta_h \widehat{f})(\xi) = \int f(x) \frac{e(-h x_k) - 1}{h} e(- \xi \cdot x)\; dx = \widehat{g_h}(\xi), \]
    %
    where
    %
    \[ g_h(x) = f(x) \frac{e(h x_k) - 1}{h}. \]
    %
    Note that
    %
    \[ \left| \frac{e(h x_k) - 1}{h} \right| = O(1 + |x_k|). \]
    %
    Since $x_k f$ is integrable, we can apply the dominated convergence theorem. Because $(e(h x_k) - 1)/h$ tends to $-2 \pi i x_k f(x)$ as $h \downarrow 0$, the function $g_h$ tends to $-2\pi i x_k f$ in $L^1(\RR^d)$. Taking Fourier transforms, we conclude that $\Delta_h \widehat{f} = \widehat{g_h}$ converges uniformly to $(-2 \pi i x_k f)^\ft(\xi)$.
\end{proof}

\begin{remark}
    In particular, the Fourier transform of a compactly supported function lies in $C^\infty(\RR^d)$, and has strong derivatives of all orders, in all the $L^p$ spaces.
\end{remark}

%\begin{remark}
%   If $f$ no longer has compact support, but $D_k f$ vanishes rapidly at infinity, then we can normally still establish that $D_k f$ is the derivative of $f$ in $L^1(\RR^n)$. Indeed, suppose $|(D_k f)(x)| \leq g(|x|)$, where $g$ is an increasing function with $\int_0^\infty t^{n-1} g(t) < \infty$, then surely $\Delta_h f$ converges to $D_k f$ in $L^1$ on any compact set, which implies that for any $M$, using the mean value theorem again,
    %
%   \begin{align*}
%       \int_{\RR^n} &|(\Delta_h f)(x) - D_k f(x)|\; dx \leq o_M(1) + \int_{|x| > M} |(\Delta_h f)(x)| + |D_k f(x)| \\
%       &\leq o_M(1) + O \left( \int_{|x| > M} g(|x| + |h|)\; dx \right) = o_M(1) + O \left( \int_M^\infty t^{n-1}g(t)\; dt \right)\\
%   \end{align*}
    %
%   If we choose $M$ large enough that the big $O$ term is $\leq \varepsilon$, then we find $\| \Delta_h f - D_k f \|_1 \leq \varepsilon + o_M(1)$, and taking $\varepsilon \to 0$ shows the convergence. This shows the derivatives exist if, for instance, $f$ is a Schwarz function, since then $|D_k f(x)| \lesssim 1/(1 + |x|^{n+1})$.
%\end{remark}

\begin{theorem}
    If $f$ has a strong derivative $f_k$ in the $L^1$ norm, $\widehat{f_k}(\xi) = 2 \pi i \xi_k \widehat{f}(\xi)$.
\end{theorem}
\begin{proof}
    It suffices to note that
    %
    \[ \widehat{\Delta_h f}(\xi) = \frac{e(h \xi_k) - 1}{h} \widehat{f}(\xi). \]
    %
    Since $\Delta_h f \to f_k$ in $L^1$, $\widehat{\Delta_h f} \to \widehat{f_k}$ uniformly, and in particular, converges to $\widehat{f_k}$ pointwise. But we know $\widehat{\Delta_h f}$ converges pointwise to $2 \pi i \xi_k \widehat{f}(\xi)$.
\end{proof}

%\begin{theorem}
%   If $X$ is a homogenous space of functions, the $f * K_\delta$ converges to $f$ in the norm associated with $X$.
%\end{theorem}
%\begin{proof}
%   Given a continuous function function $F: \mathbf{T} \to X$, we define the formal Riemann integral of functions as
    %
%   \[ \int_{\mathbf{T}} F(x)\; dx = \lim_{N \to \infty} \frac{1}{N} \sum_{n = 1}^N F(2 \pi /N) \]
    %
%   which exists for the same reason the Riemann integral of a continuous real valued function exists. Now we can consider the formal function theoretic convolution
    %
%   \[ \int_{\mathbf{T}} K_\delta(x) f_x\; dx \]
    %
%   This is equal to $K_\delta * f$, because the $L^1$ norm lower bounds the norm on $X$, so that the limit with respect to the $L^1$ norm is the same as with respect to the norm on $X$, and
    %
%   \[ s \]
    %
%   \[ \int_{\mathbf{T}} K_\delta(x) f_x\; dx - f = \int_0^{2\pi} K_\delta(x)[f_x - f]\; dx \]
    %
%   If we choose 
%\end{proof}
%
%More generally, if we equip a translation invariant subspace of $L^1(\RR^n)$ with a norm lower bounded up to a constant by the $L^1$ norm which turns the space into a Banach space, then $f * K_\delta$ converges to $f$ in that norm. If in addition, the $K_\delta$ satisfy $|K_\delta(x)| \lesssim \delta^{-n}$, and $|K_\delta(x)| \lesssim \delta/|x|^{n+1}$, then $f * K_\delta$ converges to $f$ almost everywhere. 

\section{Convergence Using Alternative Summation}

As we might expect from the Fourier series theory, the formula
%
\[ f(x) = \int \widehat{f}(\xi) e(\xi \cdot x)\; dx \]
%
does not hold for every integrable $f$, nor even for all continuous $f$. In particular, the Fourier transform of $f$ need not even lie in $L^1(\RR^d)$, so the integral formula may not even make sense. Nonetheless, just as with Fourier series, one can obtain general results by `dampening' the integration.

\begin{example}
    Even if $f$ is a non integrable function, the functions $f(x) e^{-\delta |x|}$ may be integrable for $\delta > 0$. We say $f$ is \emph{Abel summable} to a value $A$ if
    %
    \[ \lim_{\delta \to 0} \int f(x) e^{-\delta |x|}\; dx = A \]
    %
    For each $\delta > 0$ and $f \in L^1(\RR^d)$, we let
    %
    \[ (A_\delta f)(x) = \int \widehat{f}(\xi) e(\xi \cdot x) e^{-\delta |\xi|}\; d\xi. \]
\end{example}

If $f \in L^1(\RR^d)$, then the dominated convergence theorem implies that
%
\[ \int f(x) e^{-\delta |x|}\; dx \to \int f(x)\; dx. \]
%
so $f$ is Abel summable. However, $f$ may be Abel summable even if $f$ is not integrable. For instance, if $f(x) = \sin(x)/x$, then $f$ is not integrable, yet $f$ is Abel summable to $\pi$ over the real line.

\begin{example}
    Similarily, we can consider the Gauss sums
    %
    \[ \int f(x) e^{-\delta |x|^2}\; dx \]
    %
    We say $f$ is Gauss summable to if these values converge as $\delta \to 0$. For $f \in L^1(\RR^d)$, we let
    %
    \[ (G_\delta f)(x) = \int \widehat{f}(\xi) e(\xi \cdot x) e^{-\delta |\xi|^2}\; d\xi. \]
\end{example}

\begin{example}
    For $d = 1$, we can also consider the Fej\'{e}r sums
    %
    \[ (\sigma_\delta f)(x) = \int_{-\infty}^\infty \widehat{f}(\xi) e(\xi \cdot x) \left( \frac{\sin(\delta \pi \xi)}{\delta \pi \xi} \right)^2\; d\xi. \]
\end{example}

\begin{example}
    In basic calculus, the integral of a function $f$ over the entire real line is defined as
    %
    \[ \int_{-\infty}^\infty f(x)\; dx = \lim_{t \to \infty} \int_{-t}^t f(x)\; dx. \]
    %
    These integrals can be written as the integral of $f \chi_{[-t,t]}$, and so in a generalized sense, we can integrate a function $f$ if $f \chi_{[-t,t]}$ is integrable for each $N$, and the integrals of these functions converge as $t \to \infty$. Thus we study
    %
    \[ (S_R f)(x) = \int_{-R}^R \widehat{f}(\xi) e(\xi \cdot x)\; d\xi. \]
\end{example}

Abel summability is more general than the piecewise limit integral considered in the last example, as the next lemma proves.

\begin{lemma}
    Suppose $f \in L^1_{\text{loc}}(\RR^d)$, that
    %
    \[ \lim_{t \to \infty} \int_{-t}^t f(x)\; dx \]
    %
    exists, and that $f(x) e^{-\delta x^2}$ is absolutely integrable for each $\delta > 0$. Then $f$ is Abel summable, and
    %
    \[ \lim_{\delta \to 0} \int_{-\infty}^\infty f(x) e^{-\delta |x|^2} = \lim_{t \to \infty} \int_{-t}^t f(x)\; dx. \]
\end{lemma}
\begin{proof}
    Let
    %
    \[ \lim_{t \to \infty} \int_{-t}^t f(x)\; dx = A. \]
    %
    For each $x \geq 0$, write
    %
    \[ F(x) = \int_{-x}^x f(x)\; dx. \]
    %
    Then $F$ is continuous, and $F(x) \to A$ as $x \to \infty$. We know that $F'(x) = f(x) + f(-x)$, and an integration by parts gives for each $s > 0$,
    %
    \begin{align*}
        \int_{-s}^s f(x) e^{-\delta x^2}\; dx &= \int_0^s [f(x) + f(-x)] e^{-\delta x^2}\; dx = F(s) e^{-\delta s^2} + 2 \delta \int_0^s x F(x) e^{-\delta x^2}\; dx.
    \end{align*}
    %
    Taking $s \to \infty$, using the fact that $F$ is bounded so that $F(s) e^{-\delta s^2} \to 0$, we conclude
    %
    \[ \int f(x) e^{-\delta x^2}\; dx = 2 \delta \int_0^\infty x F(x) e^{-\delta x^2}\; dx. \]
    Given $\varepsilon > 0$, fix $t$ such that $|F(s) - A| \leq \varepsilon$ for $s \geq t$. Then
    %
    \begin{align*}
        \left| \int f(x) e^{-\delta x^2}\; dx - A \right| &\leq 2 \delta \left| \int_0^t x F(x) e^{-\delta x^2}\; dx \right|\\
        &\quad + 2 \delta \varepsilon \left| \int_t^\infty x e^{-\delta x^2} \right|\\
        &\quad + \left| 2 \delta A \int_t^\infty x e^{-\delta x^2}\; dx - A \right|.
    \end{align*}
    %
    The first and second components of this upper bound can each be made smaller than $\varepsilon$ for small enough $\delta$. And
    %
    \[ 2 \delta \int_t^\infty x e^{-\delta x^2}\; dx = e^{-\delta t^2} \]
    %
    So the 3rd term is equal to
    %
    \[ |A| |1 - e^{-\delta t^2}| \]
    %
    and for small enough $\delta$, we can also bound this by $\varepsilon$. Thus we have shown for small enough $\delta$ that
    %
    \[ \left| \int f(x) e^{-\delta x^2}\; dx - A \right| \leq 3 \varepsilon. \]
    %
    It now suffices to take $\varepsilon \to 0$.
\end{proof}

Abel summation is even more general than Gauss summation.

\begin{lemma}
    If $f$ is Gauss summable, and $f(x) e^{-\delta |x|}$ is absolutely integrable for each $\delta > 0$, then $f$ is Abel summable, and
    %
    \[ \lim_{\delta \to 0} \int f(x) e^{-\delta |x|^2}\; dx = \lim_{\delta \to 0} \int f(x) e^{-\delta |x|}\; dx. \]
\end{lemma}
\begin{proof}
    Let
    %
    \[ \lim_{\delta \to 0} \int f(x) e^{-\delta |x|^2}\; dx = A. \]
    %
    If there existed constants $c_n$ and $\lambda_n$ such that $e^{-\delta |x|} = \sum c_n e^{-(\lambda_n \delta |x|)^2}$, this theorem would be easy. This is not exactly true, but we do have the {\it subordination principle}, which says
    %
    \[ e^{-\delta |x|} = \int_0^\infty \frac{e^{-u}}{\sqrt{\pi u}} e^{-\delta^2 |x|^2/4u}\; du. \]
    %
    This formula, which is proved using basic complex analysis, is shown later on in this chapter. Applying Fubini's theorem, this means that
    %
    \[ \int f(x) e^{-\delta |x|} = \int_0^\infty \frac{e^{-u}}{\sqrt{\pi u}} \int f(x) e^{-\delta^2 |x|^2/4u}\; dx\; du. \]
    %
    For any fixed $t > 0$, we certainly have
    %
    \[ \lim_{\delta \to 0} \int_t^\infty \frac{e^{-u}}{\sqrt{\pi u}} \int f(x) e^{-\delta^2 |x|^2/4u}\; dx\; du = A \int_t^\infty \frac{e^{-u}}{\sqrt{\pi u}} \]
    %
    And this is equal to $A(1 + o(1))$ as $t \to 0$. And now we calculate
    %
    \[ \int_0^t \frac{e^{-u}}{\sqrt{\pi u}} \int f(x) e^{-\delta^2 |x|^2/4u}\; du \leq \left\| \frac{e^{-u}}{\sqrt{\pi u}} \right\|_{L^1[0,t]} \left\| \int f(x) e^{-\delta^2 |x|^2/4u} \right\|_{L^\infty[0,t]} \]
    %
    The left norm tends to zero as $t \to 0$. And as $u \downarrow 0$, the dominated convergence theorem implies that
    %
    \[ \int f(x) e^{-\delta |x|^2/4u} \to 0. \]
    %
    This completes the proof.
\end{proof}

For any family of functions $\Phi_\delta$, we can consider the `$\Phi$ sums'
%
\[ \int f(x) \Phi_\delta(x)\; d\xi \]
%
and the corresponding Fourier transform operators
%
\[ S_\delta(f,\Phi)(x) = \int \widehat{f}(x) e(\xi \cdot x) \Phi_\delta(\xi)\; d\xi. \]
%
We say $f$ is $\Phi$ summable to a value if
%
\[ \int f(x) \Phi_\delta(x)\; d\xi \]
%
converges. In all the examples we will consider, we construct $\Phi$ sums by fixing a function $\Phi \in C_0(\RR^n)$ with $\Phi(0) = 1$, and defining $\Phi_\delta(x) = \Phi(\delta x)$. When this is the case $f(x) \Phi_\delta(x)$ converges to $f(x)$ pointwise for each $x$ as $\delta \to 0$. Thus if $f \in L^1(\RR^d)$, the dominated convergence theorem implies that $f$ is $\Phi$ summable to it's usual integral. We now use these summability kernels to understand the Fourier summation formula.

\begin{theorem}[The Multiplication Formula]
    If $f,g \in L^1(\RR^n)$,
    %
    \[ \int f(x) \widehat{g}(x)\; dx = \int \widehat{f}(\xi) g(\xi)\; dx. \]
\end{theorem}
\begin{proof}
    If $f, g \in L^1(\RR^n)$, then $\widehat{f}$ and $\widehat{g}$ are bounded, continuous functions on $\RR^n$. In particular, $\widehat{f} g$ and $f \widehat{g}$ are integrable. A simple use of Fubini's theorem gives
    %
    \[ \int f(x) \widehat{g}(x)\; dx = \int \int f(x) g(\xi) e(- \xi \cdot x)\; dx\; d\xi = \int g(\xi) \widehat{f}(\xi)\; d\xi. \qedhere \]
\end{proof}

If $\Phi$ is integrable, then the multiplication formula shows
%
\begin{align*}
    S_\delta(f,\Phi) &= \int \widehat{f}(\xi) e(\xi \cdot x) \Phi(\delta \xi) d\xi\\
    &= \int f(x) (M_x (\delta_\delta \Phi))^\ft(x)\; dx = \delta^{-n} \int f(x) \cdot \widehat{\Phi} \left( \frac{x - y}{\delta} \right)\; dx.
\end{align*}
%
Thus if we define $K^\Phi_\delta(x) = \delta^{-n} \widehat{\Phi}(-x/\delta)$, then $S_\delta(f,\Phi) = K^\Phi_\delta * f$. Thus we have expressed the summation operators as convolution operations.

We now recall some notions of convolution kernels that help us approximate functions. Recall that if a family of kernels $\{ K_\delta \}$ satisfies
%
\begin{itemize}
    \item For any $\delta > 0$,
    %
    \[ \int K_\delta(\xi)\; d\xi = 1. \]

    \item The values $\{ \| K_\delta \|_{L^1(\RR^n)} \}$ are uniformly bounded in $\delta$.

    \item For any $\varepsilon > 0$,
    %
    \[ \lim_{\delta \to 0} \int_{|\xi| \geq \varepsilon} |K_\delta(\xi)|\; d\xi \to 0. \]
\end{itemize}
%
then the family forms a \emph{good kernel}. If this is the case, then $f * K_\delta$ converges to $f$ in the $L^p$ norms if $f \in L^p(\RR^n)$, and converges to $f$ uniformly if $f$ is continuous and bounded. If we have the stronger conditions that
%
\begin{itemize}
    \item For any $\delta > 0$,
    %
    \[ \int K_\delta(\xi)\; d\xi = 1. \]

    \item $\| K_\delta \|_{L^\infty(\RR^d)} \lesssim 1/\delta^d$.
    \item For any $\delta > 0$ and $\xi \in \RR^d$,
    %
    \[ |K_\delta(\xi)| \lesssim \frac{\delta}{|x|^{d+1}}. \]
\end{itemize}
%
then the family $\{ K_\delta \}$ is an approximation to the identity, and so $(K_\delta * f)(x)$ converges to $f(x)$ for any $x$ in the Lebesgue set of $f$.

\begin{example}
    We obtain the {\it Fej\'{e}r kernel} $F_\delta$ from the initial function
    %
    \[ F(x) = \left( \frac{\sin \pi x}{\pi x} \right)^2 \]
    %
    Using contour integration, we now show
    %
    \[ \widehat{F}(\xi) = \begin{cases} 1 - |\xi| & : |\xi| \leq 1\\ 0 &: |\xi| > 1 \end{cases} \]
    %
    Since this functions is compactly supported, with total mass one, it is easy to see the corresponding Kernel $K^F_\delta$ are an approximation to the identity. Thus $\sigma_\delta f$ converges to $f$ in all the manners described above.

    Since $F$ is an even function, $\widehat{F}$ is even, and so we may assume $\xi \geq 0$. We initially calculate
    %
    \[ \widehat{F}(\xi) = \int_{-\infty}^\infty \left( \frac{\sin(\pi x)}{\pi x} \right)^2 e(- \xi x)\; dx = \frac{1}{\pi} \int_{-\infty}^\infty \left( \frac{\sin x}{x} \right)^2 e(- 2 \xi x) \; dx. \]
    %
    Now we have
    %
    \[ (\sin z)^2 = \left( \frac{e(z) - e(-z)}{2i} \right)^2 = \frac{(2 - e^{2iz}) - e^{-2iz}}{4}. \]
    %
    This means
    %
    \begin{align*}
        \frac{(\sin z)^2}{z^2} e^{- 2 i \xi z} &= \frac{2e^{-2 i \xi z} - e^{-2(\xi + 1) i z}) - e^{-2(\xi - 1)iz}}{4z^2 } = \frac{f_\xi(z) + g_\xi(z)}{4}.
    \end{align*}
    %
    For $\xi \geq 0$, $f_\xi(z)$ is $O_\xi(1/|z|^2)$ in the lower half plane, because if $\text{Im}(z) \leq 0$,
    %
    \[ |2e^{-2 i \xi z} - e^{-2(\xi + 1) z}| \leq 2e^{2\xi} + e^{2(\xi + 1)} = O_\xi(1). \]
    %
    For $\xi \geq 1$, $g_\xi(z)$ is also $O_\xi(1/|z|^2)$ in the lower half plane, because
    %
    \[ |e^{-2(\xi - 1)iz}| \leq e^{2(\xi - 1)}.  \]
    %
    Now since $(\sin x/x)^2 e^{-2 i \xi x}$ can be extended to an entire function on the entire complex plane, which is bounded on any horizontal strip, we can apply Cauchy's theorem and take limits to conclude that
    %
    \begin{align*}
        \widehat{F}(\xi) = \frac{1}{\pi} \int_{-\infty}^\infty \frac{(\sin x)^2}{x^2} e^{-2 i \xi x}\; dx &= \frac{1}{\pi} \int_{-\infty}^{\infty} \frac{(\sin (x - iy)^2}{(x - iy)^2} e^{-2 i \xi x  -2 \xi y}\; dx\\
        &= \frac{1}{4 \pi} \int_{-\infty}^\infty f_\xi(x - iy) + g_\xi(x - iy)\; dx.
    \end{align*}
    %
    If $\xi \geq 1$, the functions $f_\xi$ and $g_\xi$ are both negligible in the lower half plane, and have no poles in the lower half plane, so if we let $\gamma$ denote the curve of length $2 \pi n$ travelling anticlockwise along the lower semicircle with vertices $-n - iy$ and $n - iy$, then because $|z| \geq n$ on $\gamma$,
    %
    \begin{align*}
        \int_{-n}^n f_\xi(x - iy) + g_\xi(x - iy)\; dx &= \int_\gamma f_\xi(z) + g_\xi(z)\; dz\\
        &= \text{length}(\gamma) \| f_\xi + g_\xi \|_{L^\infty(\gamma)}\\
        &= (2 \pi n) O_\xi(1/n^2) = O_\xi(1/n),
    \end{align*}
    %
    and so we conclude that
    %
    \[ \int_{-\infty}^\infty f_\xi(x - iy) + g_\xi(x - iy)\; dx = 0. \]
    %
    This means $\widehat{F}(\xi) = 0$. If $0 \leq \xi \leq 1$, then $f_\xi$ is still small in the lower half plane, so we can conclude that
    %
    \[ \int_{-\infty}^\infty f_\xi(x - iy)\; dx = 0. \]
    %
    But $g_\xi$ is now small in the upper half plane. For $\text{Im}(z) \geq -y$,
    %
    \[ |e^{-2(\xi - 1)iz}| = |e^{2(1 - \xi)iz}| \leq e^{2(1 - \xi)y}, \] 
    %
    so $g_\xi(z) = O_\xi(1/|z|^2)$ in the half plane above the line $\RR - iy$. The only problem now is that $g_\xi$ has a pole in this upper half plane, at the origin. Taking Laurent series here, we find that the residue at this point is $2i(\xi - 1)$. Thus, if we let $\gamma$ be the curve obtained from travelling anticlockwise about the upper semicircle with vertices $-n - iy$ and $n - iy$, then $|z| \geq n - y$ on this curve, and the residue theorem tells us that
    %
    \[ \int_{-n}^n g_\xi(x - iy)\; dx + \int_\gamma g_\xi(z)\; dz = 2\pi i (2i(\xi - 1)) = 4 \pi (1 - \xi), \]
    %
    and we now find that, as with the evaluation of the previous case,
    %
    \[ \int_\gamma g_\xi(z)\; dz \leq (2 \pi n) O_{\xi,y}(1/n^2) = O_{\xi,y}(1/n). \]
    %
    Taking $n \to \infty$, we conclude
    %
    \[ \int_{-\infty}^\infty g_\xi(x - iy)\; dx = 4 \pi (1 - \xi), \]
    %
    and putting this all together, we conclude that $\widehat{F}(\xi) = 1 - \xi$.
%   It is interesting in this particular case to note that
    %
%   \begin{align*}
%       \int_{-1}^1 (1 - |\xi|) e^{2 \pi i\xi x}\; d\xi &= 2 \int_0^1 (1 - \xi) \cos(2 \pi \xi x)\; d\xi\\
%       &= 2 \left( \left. \frac{(1 - \xi) \sin(2 \pi \xi x)}{2 \pi x} - \frac{\cos(2 \pi \xi x)}{(2 \pi x)^2} \right|_0^1 \right)\\
%       &= 2 \frac{1 - \cos(2 \pi x)}{(2 \pi x)^2} = \frac{\sin^2(\pi x)}{(\pi x)^2} = F(x)
%   \end{align*}
    %
%   which is exactly the inversion formula we want for all $L^1$ functions.
\end{example}

\begin{example}
    In the next paragraph, we calculate that if $\Phi(x) = e^{-\pi |x|^2}$, then $\widehat{\Phi} = \Phi$. Thus if we define the \emph{Weirstrass kernel} by
    %
    \[ W_\delta(\xi) = \delta^{-d} e^{-\pi |x|^2/\delta^2}, \]
    %
    then $G_\delta(f) = W_\delta * f$. Since the family $\{ W_\delta \}$ is an approximation to the identity, this shows $G_\delta(f)$ converges to $f$ in all the appropriate senses.

    Since $\Phi$ breaks onto products of exponentials over each coordinate, it suffices to calculate the Fourier transform in one dimension, from which we can obtain the general transform by taking products. In the one dimensional case, since $\Phi'(x) = -2 \pi x e^{- \pi x^2}$ is integrable, we conclude that $\widehat{\Phi}$ is differentiable, and
    %
    \[ (\widehat{\Phi})'(\xi) = (- 2 \pi i \xi \Phi)^\ft(\xi) = i (\Phi')^\ft(\xi) = i (2 \pi i \xi) \widehat{\Phi}(\xi) = - 2 \pi \xi \widehat{\Phi}(\xi) \]
    %
    The uniqueness theorem for ordinary differential equations says that since
    %
    \[ \widehat{\Phi}(0) = \int_{-\infty}^\infty e^{- \pi x^2} = 1 = \Phi(0) \]
    %
    Thus we must have $\widehat{\Phi} = \Phi$.
\end{example}

\begin{example}
    The Fourier transform of the function $e^{- |x|}$ is the \emph{Poisson kernel}
    %
    \[ P(\xi) = \frac{\Gamma((d+1)/2)}{(\pi(1 + |\xi|^2))^{(d+1)/2}} \]
    %
    Later on we show the corresponding scaled kernel $\{ P_\delta \}$ is an approximation to the identity, and thus $A_\delta f = P_\delta * f$ converges to $f$ in all appropriate senses.

    The Abel kernel $A_\delta$ on $\RR^n$ is obtained from the initial function $A(x) = e^{-2 \pi |x|}$. The calculation of the Fourier transform of this function indicates a useful principle in Fourier analysis: one can reduce expressions involving $e^{-x}$ into expressions involving $\smash{e^{-x^2}}$ using the subordination principle. In particular, for $\beta > 0$ we have the formula
    %
    \[ e^{-\beta} = \int_0^\infty \frac{e^{-u}}{\sqrt{\pi u}} e^{-\beta^2/4u}\; du \]
    %
    We establish this by letting $v = \sqrt{u}$, so
    %
    \[ \int_0^\infty \frac{e^{-u}}{\sqrt{\pi u}} e^{-\beta^2/4u}\; du = \frac{2}{\sqrt{\pi}} \int_0^\infty e^{-v^2 - \beta^2/4v^2}\; dv = \frac{2e^{-\beta}}{\sqrt{\pi}} \int_0^\infty e^{-(v - \beta/2v)^2}\; dv \]
    %
    But the map $v \mapsto v - \beta/2v$ is measure preserving by Glasser's master theorem, so this integral is
    %
    \[ \frac{2e^{-\beta}}{\sqrt{\pi}} \int_0^\infty e^{-v^2}\; dv = e^{-\beta} \]
    %Because using the theory of residues,
    %
    %\begin{align*}
    %   e^{-\beta} &= \frac{2}{\pi} \int_0^\infty \frac{\cos \beta x}{1 + x^2} = \frac{1}{\pi} \int_{-\infty}^\infty \frac{e^{\beta i x}}{1 + x^2}\; dx\; du\\
    %   &= \frac{1}{\pi} \int_{-\infty}^\infty e^{\beta i x} \int_0^\infty e^{-u} e^{-ux^2}\; du\; dx\\
    %   &= \frac{1}{\pi} \int_0^\infty e^{-u} \int_{-\infty}^\infty e^{-ux^2} e^{\beta i x}\; dx\; du\\
    %   &= \frac{1}{\pi} \int_0^\infty \sqrt{\pi/u} e^{-u} e^{-\beta^2/4u}\; du
    %\end{align*}
    %
    In tandem with Fubini's theorem, this formula implies
    %
    \begin{align*}
        \widehat{A}(\xi) &= \int e^{-2 \pi |x|} e^{- 2 \pi i \xi \cdot x}\; dx = \int \int_0^\infty \frac{e^{-u}}{\sqrt{\pi u}} e^{- |\pi x|^2/u} e^{-2 \pi i \xi \cdot x}\; du\; dx\\
        &= \int_0^\infty \frac{e^{-u}}{\sqrt{\pi u}} \int e^{-|\pi x|^2/u} e^{-2 \pi i \xi \cdot x}\; dx\; du = \int_0^\infty \frac{e^{-u}}{\sqrt{\pi u}} (\delta_{\sqrt{\pi/u}} \Phi)^\ft(\xi)\; du\\
        &= \frac{1}{\pi^{(n + 1)/2}} \int_0^\infty e^{-u} u^{(n-1)/2} e^{- u|\xi|^2}\; du
    \end{align*}
    %
    Setting $v = (1 + |\xi|^2) u$, we conclude that since by definition,
    %
    \[ \int_0^\infty e^{-v} v^{(n-1)/2} = \Gamma \left( \frac{n+1}{2} \right) \]
    %
    \[ \widehat{A}(\xi) = \frac{\Gamma((n+1)/2)}{[\pi(1 + |\xi|^2)]^{(n+1)/2}} \]
    %
    Thus the Abel mean is the Fourier inverse of the Poisson kernel on the upper half plane $\mathbf{H}^{n+1}$.

    In order to conclude $\{ P_\delta \}$ is a good kernel, it now suffices to verify that
    %
    \[ \int_{\RR^n} \frac{d\xi}{(1 + |\xi|^2)^{(n+1)/2}} = \frac{\pi^{(n+1)/2}}{\Gamma((n+1)/2)} \]
    %
    The right hand side is half the surface area of the unit sphere in $\RR^{n+1}$. Denoting this quantity by $S_n$, and switching to polar coordinates, we find that
    %
    \[ \int_{\RR^n} \frac{d\xi}{(1 + |\xi|^2)^{(n+1)/2}} = S_{n-1} \int_0^\infty \frac{r^{n-1}}{(1 + r^2)^{(n+1)/2}}\; dr \]
    %
    Setting $r = \tan u$, we find
    %
    \[ \int_0^\infty \frac{r^{n-1}}{(1 + r^2)^{(n+1)/2}}\; dr = \int_0^{\pi/2} (\sin u)^{n-1} du \]
    %
    The theorem now follows from noticing that $S_{n-1} (\sin u)^{n-1}$ is the surface area of the $n-1$ sphere obtained by slicing $S^n$ with the hyperplane $x_n = \cos u$. Fubini's theorem implies that the integral is $S_n/2$, which is what we wanted to verify.
\end{example}

\begin{example}
    We note that
    %
    \[ \int_{-R}^R e(- \xi x)\; dx = \frac{e(- \xi R) - e(\xi R)}{-2 \pi i \xi} = \frac{\sin(2 \pi \xi R)}{\pi \xi}. \]
    %
    so the Fourier transform of $\chi_{[-R,R]}$ is the \emph{Dirichlet kernel}
    %
    \[ D_R(\xi) = \frac{\sin(2 \pi \xi R)}{\pi \xi} \]
    %
    We note that $D_R \not \in L^1(\mathbf{R})$. Thus $D_R$ is {\it not} a good kernel, which makes the convergence rates of $S_R f$ more subtle. Nonetheless, $D_R$ does lie in $L^p(\mathbf{R})$ for all $p \in (1,\infty]$, and is \emph{uniformly bounded} in $L^p(\mathbf{R})$ for all $p \in (1,\infty)$, a fact we will prove later.
    %
%    \[ \int_{|\xi| \geq 1/R} \left| \frac{\sin(2 \pi \xi R)}{\pi \xi} \right|^p \lesssim \int_{1/R}^\infty \frac{1}{|\xi|^p} = R^{p-1} \]
%    \[ \int_{|\xi| \leq 1/R} \left| \frac{\sin(2 \pi \xi R)}{\pi \xi} \right|^p \lesssim_p R^{p-1} \]
    This is enough to conclude that for all $p \in (1,\infty)$, $S_R f \to f$ in $L^p(\mathbf{R})$.
\end{example}

Thus we now know there are a large examples of functions $\Phi \in C_0(\RR^d)$ with $\Phi(0) = 1$, and such that for any $x$ in the Lebesgue set of $f$,
%
\[ f(x) = \lim_{\delta \to 0} \int \widehat{f}(\xi) e(\xi \cdot x) \Phi(\delta x). \]
%
If $\widehat{f}$ is integrable, then the bound $| \widehat{f}(\xi) e(\xi \cdot x) \Phi(\delta \xi) | \leq \| \Phi \|_\infty | \widehat{f}(\xi) |$ implies that we can use the dominated convergence theorem to conclude that for any point $x$ in the Lebesgue set of $f$,
%
\[ f(x) = \lim_{\delta \to 0} \int \widehat{f}(\xi) e(\xi \cdot x) \Phi(\delta x) = \int \widehat{f}(\xi) e(\xi \cdot x) \]
%
Thus the inversion theorem holds pointwise almost everywhere.

\begin{theorem}
    If $f$ and $\widehat{f}$, then for any $x$ in the Lebesgue set of $f$,
    %
    \[ f(x) = \int \widehat{f}(\xi) e(\xi \cdot x)\; d\xi. \]
\end{theorem}

We define, for any integrable $f: \RR^n \to \RR$, the \emph{inverse} Fourier transform
%
\[ \widecheck{f}(x) = \int f(\xi) e(\xi \cdot x)\; d\xi \]
%
The inverse transform is also denoted by $\mathcal{F}^{-1}(f)$. The last theorem says that $\mathcal{F}^{-1}$ really is the inverse operator to the operator $\mathcal{F}$, at least on the set of functions $f$ where $\widehat{f}$ is integrable. In particular, this is true if $f$ has strong derivatives in the $L^1$ norm for any multi-index $|\alpha| \leq n+1$, and so the Fourier inversion formula holds for sufficiently smooth functions.

\begin{corollary}
    If $f \in C(\RR)$ is integrable and $\widehat{f} \in L^1(\RR)$, $S_R f \to f$  uniformly.
\end{corollary}
\begin{proof}
    The dominated convergence theorem implies that for each $x \in\RR$,
    %
    \[ f(x) = \int_{\RR} \widehat{f}(\xi) e(\xi \cdot x) = \lim_{R \to \infty} \int_{-R}^R \widehat{f}(\xi) e(\xi \cdot x) = \lim_{R \to \infty} (S_R f)(x). \]
    %
    And
    %
    \[ \int_{|x| \geq R} \widehat{f}(\xi) e(\xi \cdot x) \leq \| \widehat{f} \|_{L^1(\RR)}. \]
    %
    so the pointwise convergence is uniform.
\end{proof}

\begin{remark}
    This theorem also generalizes to $\RR^n$. Here, the operators $S_R$ are no longer canonically defined, but if we consider any increasing nested family of sets $B_R$ with $\lim B_R = \RR^n$, then the corresponding operators
    %
    \[ S_R f = \int_{B_R} \widehat{f}(\xi) e(\xi \cdot x) \]
    %
    also converge uniformly to $f$.
\end{remark}

\begin{corollary}
    The map $\mathcal{F}: L^1(\RR^d) \to C_0(\RR^d)$ is injective.
\end{corollary}
\begin{proof}
    If $\widehat{f} = 0$, then $\widehat{f}$ is certainly integrable. But this means that the Fourier inversion theorem can apply, giving that for almost every point $x$,
    %
    \[ f(x) = \int_{-\infty}^\infty \widehat{f}(x) e(\xi \cdot x) = 0. \]
    %
    Thus $f = 0$ almost everywhere.
\end{proof}

This corollary is underestimated in utility. Even if the Fourier inversion theorem doesn't hold, we can still view the Fourier transform as another way to represent a function, since the Fourier transform does not lose any information. For instance, it can be used very easily to verify identities involving convolutions.

\begin{corollary}
    For any $\delta_1, \delta_2$,
    %
    \[ W_{\delta_1 + \delta_2} = W_{\delta_1} * W_{\delta_2}\quad\text{and}\quad P_{\delta_1 + \delta_2} = P_{\delta_1} * P_{\delta_2}. \]
\end{corollary}
\begin{proof}
    We recall that
    %
    \[ W_{\delta_1 + \delta_2} = \mathcal{F}(e^{-(\delta_1 + \delta_2) |x|^2}). \]
    %
    But $e^{-(\delta_1 + \delta_2) |x|^2} = e^{-\delta_1 |x|^2} e^{-\delta_2 |x|^2}$ breaks into a product, which allows us to calculate
    %
    \[ \mathcal{F}(e^{-\pi \delta_1 |x|^2} e^{-\pi \delta_2 |x|^2}) = \mathcal{F}(e^{-\pi \delta_1 |x|^2}) * \mathcal{F}(e^{-\pi \delta_2 |x|^2}) = W_{\delta_1} * W_{\delta_2}.  \]
    %
    Thus $W_{\delta_1} * W_{\delta_2} = W_{\delta_1 + \delta_2}$. Similarily, $P_{\delta_1 + \delta_2}$ is the Fourier transform of $e^{-(\delta_1 + \delta_2)|x|}$, which breaks into a product, whose individual Fourier transforms are $P_{\delta_1}$ and $P_{\delta_2}$.
\end{proof}

% TODO: Prove using De la Vallee Poisson that if f^(xi) = O(1/|xi|), then S_R f converges uniformly.

%\begin{theorem}
%   If $f$ is an integrable, function continuous at the origin, and $\widehat{f} \geq 0$, then $\widehat{f}$ is integrable.
%\end{theorem}
%\begin{proof}
%   This follows because
    %
%   \[ f(0) = \lim_{\delta \to 0} \int \widehat{f}(\xi) e^{-\delta |x|} \]
    %
%   By Fatou's lemma,
    %
%   \[ f(0) = \lim_{\delta \to 0} \int \widehat{f}(\xi) e^{-\delta |x|} \geq \int \liminf_{\delta \to 0} \widehat{f}(\xi) e^{-\delta |x|} = \int \widehat{f}(\xi) \]
    %
%   so $\widehat{f}$ is finitely integrable.
%\end{proof}

%Note that this implies that we obtain the general inversion theorem, so in particular, it is only continuous functions, and functions almost everywhere equal to continuous functions, which can have non-negative Fourier transforms.

\section{The $L^2$ Theory}

One integral component of Fourier series on $L^2(\mathbf{T})$ is Plancherel's equality
%
\[ \sum |\widehat{f}(n)|^2 = \frac{1}{2\pi} \int_0^{2\pi} |f(x)|^2 \]
%
On $\RR^n$, we would like to justify that
%
\[ \int |\widehat{f}(\xi)|^2\; d\xi = \int |f(x)|^2\; dx \]
%
However, on the non-compact Euclidean space, a general element of $L^2(\RR^n)$ is not necessarily integrable, so we cannot take it's Fourier transform using the integral formula. Nonetheless, we can take the Fourier transform of an element of $L^1(\RR^n) \cap L^2(\RR^n)$, and we find the equation holds.

\begin{theorem}
    If $f \in L^1(\RR^n) \cap L^2(\RR^n)$, then $\| \widehat{f} \|_{L^2(\RR^d)} = \| f \|_{L^2(\RR^d)}$.
\end{theorem}
\begin{proof}
    The theorem is an easy consequence of the multiplication formula, since
    %
    \[ |\widehat{f}(\xi)| = \widehat{f}(\xi) \overline{\widehat{f}}(\xi), \]
    %
    and
    %
    \[ \left( \overline{\widehat{f}} \right)^\ft(\xi) = \overline{(f^\ft)^\ft(-\xi)} = \overline{f(\xi)}. \]
    %
    This implies
    %
    \[ \int |\widehat{f}(\xi)|^2\; d\xi = \int \widehat{f}(\xi) \overline{\widehat{f}(\xi)}\; d\xi = \int f(x) \overline{f(x)}\; dx = \int |f(x)|^2\; dx. \qedhere \]
\end{proof}

A simple interpolation argument leads to the following corollary, which is a variant of the Hausdorff-Young inequality for functions on $\RR^n$.

\begin{corollary} If $f \in L^1(\RR^n) \cap L^p(\RR^n)$ for $1 \leq p \leq 2$, then
    %
    \[ \| \widehat{f} \|_{L^q(\RR^n)} \leq \| f \|_{L^p(\RR^n)}. \]
    %
    where $2 \leq q \leq \infty$ is the conjugate of $p$.
\end{corollary}

Though the integral formula of an element of $L^2(\RR^n)$ does not make sense, the bounds above provide a canonical way to define the Fourier transform of an element of $L^p(\RR^n)$, for $1 \leq p \leq 2$. The space $L^1(\RR^n) \cap L^p(\RR^n)$ is a dense subset of $L^p(\RR^n)$, so we can use the Hahn-Banach theorem to define the Fourier transform $\mathcal{F}: L^p(\RR^n) \to L^q(\RR^n)$ as the {\it unique} bounded operator agreeing with the integral formula on the common domain. The extended Fourier transform on $L^2(\RR^n)$ is still unitary, because the multiplication formula extends to $L^2(\RR^n)$, so that
%
\[ (\mathcal{F}(f),g) = \int \widehat{f}(\xi) \overline{g(\xi)}\; d\xi = \int f(x) \overline{\widehat{g}(-\xi)}\; dx = (f,\mathcal{F}^{-1}(g)). \]
%
Thus the adjoint of $\mathcal{F}$ is $\mathcal{F}^{-1}$, which means exactly that $\mathcal{F}$ is unitary.


\section{The Hausdorff-Young Inequality}

For functions on $\mathbf{T}$, it is unclear how to provide examples which show why the Hausdorff-Young inequality cannot be extended to give results for $p > 2$. Over $\RR$, we can provide examples which explicitly indicate the tightness of the appropriate constants by applying symmetry arguments.

\begin{example}
    Given an integrable function $f$, let $f_r(x) = f(rx)$. Then we find $\widehat{f_r}(\xi) = r^{-n} \widehat{f}(\xi/r)$, and so
    %
    \[ \| f_r \|_{L^p(\RR^n)} = r^{-n/p} \| f \|_{L^p(\RR^n)} \quad \text{and} \quad \| \widehat{f_r} \|_{L^q(\RR^n)} = r^{n/q-n} \| \widehat{f} \|_{L^q(\RR^n)}. \]
    %
    In order for a bound to hold in terms of $p$ and $q$ uniformly for all values of $r$, we need $r^{-n/p} = r^{n/q-n}$, which means $1/q + 1/p = 1$, so $p$ and $q$ must be conjugates of one another.
\end{example}

\begin{example}
    Consider the family of functions $f_s(x) = s^{-n/2} e^{- \pi |x|^2/s}$, where $s = 1 + it$ for some $t \in \RR$. One can easily calcluate that $\widehat{f_s}(\xi) = e^{- \pi s |\xi|^2}$. We calculate
    %
    \[ \| f_s \|_{L^p(\RR^n)} = |s|^{-n/2} \left( \int e^{- (p/|s|^2) \pi |x|^2}\; dx \right)^{1/p} = |s|^{n/p - n/2} p^{-n/p} \]
    %
    whereas $\| \widehat{f_s} \|_q = q^{-n/2}$. Thus to be able  compare the two quantities as $t \to \infty$, we need $n/p - n/2 \leq 0$, so $p \leq 2$. As $t \to \infty$, $\smash{|f_s(x)| \sim t^{-n/2} e^{-\pi |x/t|^2}}$, so the $t$ gives us a decay in $f_s$. However, when we take the Fourier transform the $t$ only corresponds to oscillatory terms. Thus we need $p \leq 2$ so that the decay in $t$ isn't too important in relation to the overall width of the function.
\end{example}

The Hausdorff-Young inequality shows that the Fourier transforms narrowly supported functions into a function with small magnitude. But the example above shows that the Fourier transform is not so good at transforming functions with small magnitude into functions which are narrowly supported, because the Fourier transform can absorb the small magnitude into an oscillatory property not reflected in the norms. Some kind of way of measuring oscillation needs to be considered to get a tighter control on the function. Of course, in hindsight, we should have never expected too much control of the Fourier transform in terms of the $L^p$ norms, since the Fourier transform measures the oscillatory nature of the input function, and oscillatory properties of a function in phase space are not very well reflected in the $L^p$ norms, except when applying certain orthogonality properties with an $L^2$ norm, or destroying the oscillation with an $L^\infty$ norm.

\section{The Poisson Summation Formula}

We now show a connection between the Fourier transform on $\RR$, and the Fourier transform on $\TT$. If $f \in \mathcal{S}(\RR)$, there are two ways of obtaining a `periodic' version of $f$ on $\TT$. Firstly, we can define, for each $x \in \mathbf{T}$,
%
\[ f_1(x) = \sum_{n = -\infty}^\infty f(x + 2 \pi n), \]
%
which is a well defined element of $C^\infty(\mathbf{T})$. Secondly, we can define
%
\[ f_2(x) = \sum_{n = -\infty}^\infty \widehat{f}(n) e_n(x), \]
%
The Poisson summation formula says that these two functions are the same function.

\begin{theorem}
    If $f \in \mathcal{S}(\RR)$, then
    %
    \[ \sum_{n = -\infty}^\infty f(x + n) = \sum_{n = -\infty}^\infty \widehat{f}(n) e_n(x). \]
\end{theorem}

\section{Sums of Random Variables}

TODO

We now switch to an application of harmonic analysis to studying sums of random variables probability theory. If $X$ is a random vector, it's probabilistic information is given by it's distribution on $\RR^n$, which can be seen as a measure $\mathbf{P}_X$ on $\RR^n$, with $\mathbf{P}_X(E) = \mathbf{P}(X \in E)$. Given two independant random vectors $X$ and $Y$, $\mathbf{P}_{X+Y}$ is the convolution $\mathbf{P}_X * \mathbf{P}_Y$ between the measures $\mathbf{P}_X$ and $\mathbf{P}_Y$, in the sense that
%
\[ \mathbf{P}_{X+Y}(E) = \int \chi_E(x+y)\; d\mathbf{P}_X(x)\; d\mathbf{P}_Y(y) \]
%
If $d\mathbf{P}_X = f_X \cdot dx$ and $d\mathbf{P}_Y = f_Y \cdot dx$, then $d(\mathbf{P}_X * \mathbf{P}_Y) = (f_X * f_Y) \cdot dx$ is just the normal convolution of functions. This is why harmonic analysis becomes so useful when analyzing sums of independant random variables.

It is useful to express the Fourier transform in a probabilistic language. Given a random variable $X$,
%
\[ \widehat{\mathbf{P}_X}(\xi) = \int e^{i \xi \cdot x} d\mathbf{P}_X(x) \]
%
Thus the natural Fourier transform of a random vector $X$ is the {\bf characteristic function} $\varphi_X(\xi) = \mathbf{E}(e^{i \xi \cdot X})$. It is a continuous function for any random variable $X$. We can also express the properties of the Fourier transform in a probabilistic language.

\begin{lemma}
    Let $X$ and $Y$ be independant random variables. Then
    %
    \begin{itemize}
        \item $\varphi_X(0) = 1$, and $|\varphi_X(\xi)| \leq 1$ for all $\xi$.

        \item (Symmetry) $\varphi_X(\xi) = \overline{\varphi_X(-\xi)}$.

        \item (Convolution) $\varphi_{X+Y} = \varphi_X \varphi_Y$.

        \item (Translation and Dilation) $\varphi_{X+a}(\xi) = e^{i a \cdot \xi} \varphi_X(\xi)$, and $\varphi_{\lambda X}(\xi) = \varphi_X(\lambda \xi)$.

        \item (Rotations) If $R \in O(n)$ is a rotation, then $\varphi_{R(X)}(\xi) = \varphi_X(R(X))$.
    \end{itemize}
\end{lemma}

Using the Fourier inversion formula, if $\varphi_X$ is integrable, then $X$ is a continuous random variable, with density
%
\[ f(x) = \int e^{- i \xi x} \varphi_X(\xi)\; d\xi \]
%
In particular, if $\varphi_X = \varphi_Y$, then $X$ and $Y$ are identically distributed. This already gives interesting results.

\begin{theorem}
    If $X$ and $Y$ are independant normal distributions, then $aX + bY$ is normally distributed.
\end{theorem}
\begin{proof}
    Since $\varphi_{aX+bY}(\xi) = \varphi_X(a \xi) \varphi_Y(b \xi)$, it suffices to show that the product of two such characteristic functions is the characteristic function of a normal distribution. If $X$ has mean $\mu$ and covariance matrix $\Sigma$, then $X \cdot \xi$ has mean $\mu \cdot \xi$ and variance $\xi^T \Sigma \xi$, and one calculates that $\mathbf{E}[e^{i \xi \cdot X}] = e^{- i \mu \cdot \xi - \xi^T \Sigma \xi / 2}$ using similar techniques to the Fourier transform of a Gaussian. One verifies that the class of functions of the form $e^{-i \mu \cdot \xi - \xi^T \Sigma \xi / 2}$ is certainly closed under multiplication and scaling, which completes the proof. 
\end{proof}

Now we can prove the celebrated central limit theorem. Note that if

\begin{theorem}
    Let $X_1, \dots, X_N$ be independant and identically distributed with mean zero and variance $\sigma^2$. If $S_N = X_1 + \dots + X_N$, then
    %
    \[ \mathbf{P}(S_N \leq \sigma \sqrt{N} t) \to \Phi(t) = \frac{1}{\sqrt{2x}} \int_{-\infty}^t e^{-y^2/2}\; dy \]
\end{theorem}
\begin{proof}
    We calculate that
    %
    \[ \varphi_{S_N/\sigma \sqrt{N}}(\xi) = \varphi_X(\xi/\sigma \sqrt{N})^N \]
    %
    Define $R_n(x) = e^{ix} - 1 - (ix) - (ix)^2/2 - \dots - (ix)^n/n!$. Then because of oscillation and the fundamental theorem of calculus,
    %
    \[ |R_0(x)| = \left| i \int_0^x e^{iy}\; dy \right| \leq \min(2,|x|) \]
    %
    Next, since $R_{n+1}'(x) = i R_n$,
    %
    \[ R_{n+1}(x) = i  \int_0^x R_n(y)\; dy \]
    %
    This gives that $|R_n(x)| \leq \min(2|x|^n/n!,|x|^{n+1}/(n+1)!)$. In particular, we conclude
    %
    \[ |\varphi_X(\xi) - 1 - \sigma^2 \xi^2/2| = |\mathbf{E}(R_2(\xi X))| \leq \mathbf{E}|R_2(\xi X)| \leq |\xi|^2 \mathbf{E} \left( \min \left( |X|^2, |\xi X|^3/6 \right) \right) \]
    %
    By the dominated convergence theorem, as $\xi \to 0$, $\varphi_X(\xi) = 1 - \xi^2 \sigma^2/2 + o(\xi^2)$. But this means that
    %
    \[ \varphi_{S_N/\sigma \sqrt{N}}(\xi) = (1 - \xi^2 / 2 N + o(\xi^2/\sigma^2 N))^N = \exp(-\xi^2/2) \]
    %
    This implies the random variables converge weakly to a normal distribution.
\end{proof}

\section{Characteristic Functions}




\chapter{Finite Character Theory}

Let us review our achievements so far. We have found several important families of functions on the spaces we have studied, and shown they can be used to approximate arbitrary functions. On the circle group $\mathbf{T}$, the functions take the form of the power maps $\phi_n: z \mapsto z^n$, for $n \in \mathbf{Z}$. The important properties of these functions is that
%
\begin{itemize}
    \item The functions are orthogonal to one another.
    \item A large family of functions can be approximated by linear combinations of the power maps.
    \item The power maps are multiplicative: $\phi_n(zw) = \phi_n(z) \phi_n(w)$.
\end{itemize}
%
The existence of a family with these properties is not dependant on much more than the symmetry properties of $\mathbf{T}$, and we can therefore generalize the properties of the fourier series to a large number of groups. In this chapter, we consider a generalization to any finite abelian group.

The last property of the power maps should be immediately recognizable to any student of group theory. It implies the exponentials are homomorphisms from the circle group to itself. This is the easiest of the three properties to generalize to arbitrary groups; we shall call a homomorphism from a finite abelian group to $\mathbf{T}$ a {\bf character}. For any abelian group $G$, we can put all characters together to form the character group $\Gamma(G)$, which forms an abelian group under pointwise multiplication $(fg)(z) = f(z)g(z)$. It is these functions which are `primitive' in synthesizing functions defined on the group.

\begin{example}
    If $\mu_N$ is the set of $N$th roots of unity, then $\Gamma(\mu_N)$ consists of the power maps $\phi_n: z \mapsto z^n$, for $n \in \mathbf{Z}$. Because
    %
    \[ \phi(\omega)^N = \phi(\omega^N) = \phi(1) = 1 \]
    %
    we see that any character on $\mu_N$ is really a homomorphism from $\mu_N$ to $\mu_N$. Since the homomorphisms on $\mu_N$ are determined by their action on this primitive root, there can only be at most $N$ characters on $\mu_N$, since there are only $N$ elements in $\mu_N$. Our derivation then shows us that the $\phi_N$ enumerate all such characters, which completes our proof. Note that since $\phi_n \phi_m = \phi_{n+m}$, and $\phi_n = \phi_m$ if and only if $n - m$ is divisible by $N$, this also shows that $\Gamma(\mu_N) \cong \mu_N$.
\end{example}

\begin{example}
    The group $\mathbf{Z}_N$ is isomorphic to $\mu_N$ under the identification $n \mapsto \omega^n$, where $\omega$ is a primitive root of unity. This means that we do not need to distinguish functions `defined in terms of $n$' and `defined in terms of $\omega$', assuming the correspondance $n = \omega^n$. This is exactly the same as the correspondence between functions on $\mathbf{T}$ and periodic functions on $\RR$. The characters of $\mathbf{Z}_n$ are then exactly the maps $n \mapsto \omega^{kn}$. This follows from the general fact that if $f: G \to H$ is an isomorphism of abelian groups, the map $f^*: \phi \mapsto \phi \circ f$ is an isomorphism from $\Gamma(H)$ to $\Gamma(G)$.
\end{example}

\begin{example}
    If $K$ is a finite field, then the set $K^*$ of non-zero elements is a group under multiplication. A rather sneaky algebraic proof shows the existence of elements of $K$, known as primitive elements, which generate the multiplicative group of all numbers. Thus $K$ is cyclic, and therefore isomorphic to $\mu_N$, where $N = |K| - 1$. The characters of $K$ are then easily found under the correspondence.
\end{example}

\begin{example}
    For a fixed $N$, the set of invertible elements of $\mathbf{Z}_N$ form a group under multiplication, denoted $\mathbf{Z}_N^*$. Any character from $\mathbf{Z}_N^*$ is valued on the $\varphi(N)$'th roots of unity, because the order of each element in $\mathbf{Z}_N^*$ divides $\varphi(N)$. The groups are in general non-cyclic. For instance, $\mathbf{Z}_8^* \cong \mathbf{Z}_2^3$. However, we can always break down a finite abelian group into cyclic subgroups to calculate the character group; a simple argument shows that $\Gamma(G \times H) \cong \Gamma(G) \times \Gamma(H)$, where we identify $(f,g)$ with the map $(x,y) \mapsto f(x)g(y)$.
\end{example}

\section{Fourier Analysis on Cyclic Groups}

We shall start our study of abstract Fourier analysis by looking at Fourier analysis on $\mu_N$. Geometrically, these points uniformly distribute themselves over $\mathbf{T}$, and therefore $\mu_N$ provides a good finite approximation to $\mathbf{T}$. Functions from $\mu_N$ to $\mathbf{C}$ are really just functions from $[n] = \{ 1, \dots, n \}$ to $\mathbf{C}$, and since $\mu_N$ is isomorphic to $\mathbf{Z}_N$, we're really computing the Fourier analysis of finite domain functions, in a way which encodes the translational symmetry of the function relative to translational shifts on $\mathbf{Z}_N$.

There is a trick which we can use to obtain quick results about Fourier analysis on $\mu_N$. Given a function $f: [N] \to \mathbf{C}$, consider the $N$-periodic function on the real line defined by
%
\[ g(t) = \sum_{n = 1}^N f(n) \chi_{(n-1/2,n+1/2)}(t) \]
%
Classical Fourier analysis of $g$ tells us that we can expand $g$ as an infinite series in the functions $e(n/N)$, which may be summed up over equivalence classes modulo $N$ to give a finite expansion of the function $f$. Thus we conclude that every function $f: [N] \to \mathbf{C}$ has an expansion
%
\[ f(n) = \sum_{m = 1}^N \widehat{f}(m) e(nm) \]
%
where $\widehat{f}(m)$ are the coefficients of the {\bf finite Fourier transform} of $f$. This method certainly works in this case, but does not generalize to understand the expansion of general finite abelian groups.

The correct generalization of Fourier analysis is to analyze the set of complex valued `square integrable functions' on the domain $[N]$. We consider the space $V$ of all maps $f: [N] \to \mathbf{C}$, which can be made into an inner product space by defining
%
\[ \langle f, g \rangle = \frac{1}{N} \sum_{n = 1}^N f(n) \overline{g(n)} \]
%
We claim that the characters $\phi_n: z \mapsto z^n$ are orthonormal in this space, since
%
\[ \langle \phi_n, \phi_m \rangle = \frac{1}{N} \sum_{k = 1}^N \omega^{k(n-m)} \]
%
If $n = m$, we may sum up to find $\langle \phi_n, \phi_m \rangle = 1$. Otherwise we use a standard summation formula to find
%
\[ \sum_{k = 1}^N \omega^{k(n-m)} = \omega^{n-m} \frac{\omega^{N(n-m)} - 1}{\omega^{n-m} -1} \]
%
Since $\omega^{N(n-m)} = 1$, we conclude the sum is zero. This implies that the $\phi_n$ are orthonormal, hence linearly independent. Since $V$ is $N$ dimensional, this implies that the family of characters forms an orthogonal basic for the space. Thus, for any function $f: [N] \to \mathbf{C}$, we have, if we set $\widehat{f}(m) = \langle f, \phi_m \rangle$, then
%
\[ f(n) = \sum_{m = 1}^N \langle f, \phi_m \rangle \phi_m(n) = \sum_{m = 1}^N \widehat{f}(m) e(mn/N) \]
%
This calculation can essentially be applied to an arbitrary finite abelian group to obtain an expansion in terms of Fourier coefficients.

\section{An Arbitrary Finite Abelian Group}

It should be easy to guess how we proceed for a general finite abelian group. Given some group $G$, we study the character group $\Gamma(G)$, and how $\Gamma(G)$ represents general functions from $G$ to $\mathbf{C}$. We shall let $V$ be the space of all such functions from $G$ to $\mathbf{C}$, and on it we define the inner product
%
\[ \langle f, g \rangle = \frac{1}{|G|} \sum_{a \in G} f(a) \overline{g(a)} \]
%
If there's any justice in the world, these characters would also form an orthonormal basis.

\begin{theorem}
    The set $\Gamma(G)$ of characters is an orthonormal set.
\end{theorem}
\begin{proof}
    If $e$ is a character of $G$, then $|e(a)| = 1$ for each $a$, and so
    %
    \[ \langle e, e \rangle = \frac{1}{|G|} \sum_{a \in G} |e(a)| = 1 \]
    %
    If $e \neq 1$ is a non-trivial character, then $\sum_{a \in G} e(a) = 0$. To see this, note that for any $b \in G$, the map $a \mapsto ba$ is a bijection of $G$, and so
    %
    \[ e(b) \sum_{a \in G} e(a) = \sum_{a \in G} e(ba) = \sum_{a \in G} e(a) \]
    %
    Implying either $e(b) = 1$, or $\sum_{a \in G} e(a) = 0$. If $e_1 \neq e_2$ are two characters, then
    %
    \[ \langle e_1, e_2 \rangle = \frac{1}{|G|} \sum_{a \in G} \frac{e_1(a)}{e_2(a)} = 0 \]
    %
    since $e_1/e_2$ is a nontrivial character.
\end{proof}

Because elements of $\Gamma(G)$ are orthonormal, they are linearly independent over the space of functions on $G$, and we obtain a bound $|\Gamma(G)| \leq |G|$. All that remains is to show equality. This can be shown very simply by applying the structure theorem for finite abelian groups. First, note it is true for all cyclic groups. Second, note that if it is true for two groups $G$ and $H$, it is true for $G \times H$, because
%
\[ \Gamma(G \times H) \cong \Gamma(G) \times \Gamma(H) \]
%
since a finite abelian group is a finite product of cyclic groups, this proves the theorem. This seems almost like sweeping the algebra of the situation under the rug, however, so we will prove the statement only using elementary linear algebra. What's more, these linear algebraic techniques generalize to the theory of unitary representations in harmonic analysis over infinite groups.

\begin{theorem}
    Let $\{ T_1, \dots, T_n \}$ be a family of commuting unitary matrices. Then there is a basis $v_1, \dots, v_m \in \mathbf{C}^m$ which are eigenvectors for each $T_i$.
\end{theorem}
\begin{proof}
    For $n = 1$, the theorem is the standard spectral theorem. For induction, suppose that the $T_1, \dots, T_{k-1}$ are simultaneously diagonalizable. Write
    %
    \[ \mathbf{C}^m = V_{\lambda_1} \oplus \dots \oplus V_{\lambda_l} \]
    %
    where $\lambda_i$ are the eigenvalues of $T_k$, and $V_{\lambda_i}$ are the corresponding eigenspaces. Then if $v \in V_{\lambda_i}$, and $j < k$,
    %
    \[ T_k T_j v = T_j T_k v = \lambda_i T_j v \]
    %
    so $T_j(V_{\lambda_i}) = V_{\lambda_i}$. Now on each $V_{\lambda_i}$, we may apply the induction hypotheis to diagonalize the $T_1, \dots, T_{k-1}$. Putting this together, we simultaneously diagonalize $T_1, \dots, T_k$.
\end{proof}

This theorem enables us to prove the character theory in a much simpler manner. Let $V$ be the space of complex valued functions on $G$, and define, for $a \in G$, the map $(T_a f)(b) = f(ab)$. $V$ has an orthonormal basic consisting of the $\chi_a(b) = N [a = b]$, for $a \in G$. In this basis, we comcpute $T_a \chi_b = \chi_{ba^{-1}}$, hence $T_a$ is a permutation matrix with respect to this basis, hence unitary. The operators $T_a$ commute, since $T_aT_b = T_{ab} = T_{ba} = T_b T_a$. Hence these operators can be simultaneously diagonalized. That is, there is a family $e_1, \dots, e_n \in V$ and $\lambda_{an} \in \mathbf{T}$ such that for each $a \in G$, $T_a e_n = \lambda_{an} f_n$. We may assume $e_n(1) = 1$ for each $n$ by normalizing. Then, for any $a \in G$, we have $f_n(a) = f_n(a \cdot 1) = \lambda_{an} f_n(1) = \lambda_{an}$, so for any $b \in G$, $f_n(ab) = \lambda_{an} f_n(b) = f_n(a) f_n(b)$. This shows each $f_n$ is a character, completing the proof. We summarize our discussion in the following theorem.

\begin{theorem}
    Let $G$ be a finite abelian group. Then $\Gamma(G) \cong G$, and forms an orthonormal basis for the space of complex valued functions on $G$. For any function $f: G \to \mathbf{C}$,
    %
    \[ f(a) = \sum_{e \in \Gamma(G)} \langle f, e \rangle\ e(a) = \sum_{e \in \Gamma(G)} \hat{f}(e) e(a)\ \ \ \ \ \langle f, g \rangle = \frac{1}{|G|} \sum_{a \in G} f(a) \overline{g(a)} \]
    %
    In this context, we also have Parseval's theorem
    %
    \[ \| f(a) \|^2 = \sum_{e \in \hat{G}} |\widehat{f}(e)|^2\ \ \ \ \ \langle f, g \rangle = \sum_{e \in \hat{G}} \widehat{f}(e) \overline{\widehat{g}(e)} \]
\end{theorem}

\section{Convolutions}

There is a version of convolutions for finite functions, which is analogous to the convolutions on $\RR$. Given two functions $f,g$ on $G$, we define a function $f * g$ on $G$ by setting
%
\[ (f * g)(a) = \frac{1}{|G|} \sum_{b \in G} f(b) g(b^{-1} a) \]
%
The mapping $b \mapsto ab^{-1}$ is a bijection of $G$, and so we also have
%
\[ (f * g)(a) = \frac{1}{|G|} \sum_{b \in G} f(ab^{-1}) g(b) = (g * f)(a) \]
%
For $e \in \Gamma(G)$,
%
\begin{align*}
    \widehat{f * g}(e) &= \frac{1}{|G|} \sum_{a \in G} (f*g)(a) \overline{e(a)}\\
    &= \frac{1}{|G|^2} \sum_{a,b \in G} f(ab) g(b^{-1}) \overline{e(a)}
\end{align*}
%
The bijection $a \mapsto ab^{-1}$ shows that
%
\begin{align*}
    \widehat{f*g}(e) &= \frac{1}{|G|^2} \sum_{a,b} f(a) g(b^{-1}) \overline{e(a)} \overline{e(b^{-1})}\\
    &= \frac{1}{|G|} \left( \sum_a f(a) \overline{e(a)} \right) \frac{1}{|G|} \left( \sum_b g(b) \overline{e(b)} \right)\\
    &= \widehat{f}(e) \widehat{g}(e)
\end{align*}
%
In the finite case we do not need approximations to the identity, for we have an identity for convolution. Define $D: G \to \mathbf{C}$ by
%
\[ D(a) = \sum_{e \in \Gamma(G)} e(a) \]
%
We claim that $D(a) = |G|$ if $a = 1$, and $D(a) = 0$ otherwise. Note that since $|G| = |\Gamma(G)|$, the character space of $\Gamma(G)$ is isomorphic to $G$. Indeed, for each $a \in G$, we have the maps $\widehat{a}: e \mapsto e(a)$, which is a character of $\Gamma(G)$. Suppose $e(a) = 1$ for all characters $e$. Then $e(a) = e(1)$ for all characters $e$, and for any function $f: G \to \mathbf{C}$, we have $f(a) = f(1)$, implying $a = 1$. Thus we obtain $|G|$ distinct maps $\widehat{a}$, which therefore form the space of all characters. It therefore follows from a previous argument that if $a \neq 1$, then
%
\[ \sum_{e \in \Gamma(G)} e(a) = 0 \]
%
Now $f * D = f$, because
%
\[ \widehat{D}(e) = \frac{1}{|G|} \sum_{a \in G} D(a) \overline{e(a)} = \overline{e}(1) = 1 \]
%
$D$ is essentially the finite dimensional version of the Dirac delta function, since it has unit mass, and acts as the identity in convolution.

\section{The Fast Fourier Transform}

The main use of the fourier series on $\mu_n$ in applied mathematics is to approximate the Fourier transform on $\mathbf{T}$, where we need to compute integrals explicitly. If we have a function $f \in L^1(\mathbf{T})$, then $f$ may be approximated in $L^1(\mathbf{T})$ by step functions of the form
%
\[ f_n(t) = \sum_{k = 1}^{n} a_k \mathbf{I}(x \in (2 \pi (k-1) / n, 2 \pi k / n)) \]
%
And then $\widehat{f_n} \to \widehat{f}$ uniformly. The Fourier transform of $f_n$ is the same as the Fourier transform of the corresponding function $k \mapsto a_k$ on $\mathbf{Z}_n$, and thus we can approximate the Fourier transform on $\mathbf{T}$ by a discrete computation on $\mathbf{Z}_n$. Looking at the formula in the definition of the discrete transform, we find that we can compute the Fourier coefficients of a function $f: \mathbf{Z}_n \to \mathbf{C}$ in $O(n^2)$ addition and multiplication operations. It turns out that there is a much better method of computation which employs a divide and conquer approach, which works when $n$ is a power of 2, reducing the calculation to $O(n \log n)$ multiplications. Before this process was discovered, calculation of Fourier transforms was seen as a computation to avoid wherever possible.

To see this, consider a particular division in the group $\mathbf{Z}_{2n}$. Given $f: \mathbf{Z}_{2n} \to \mathbf{C}$, define two functions $g,h: \mathbf{Z}_n \to \mathbf{C}$, defined by $g(k) = f(2k)$, and $h(k) = f(2k + 1)$. Then $g$ and $h$ encode all the information in $f$, and if $\nu = e(\pi/n)$ is the canonical generator of $\mathbf{Z}_{2n}$, we have
%
\[ \hat{f}(m) = \frac{\hat{g}(m) + \hat{h}(m) \nu^m}{2} \]
%
Because
%
\begin{align*}
    \frac{1}{2n} \sum_{k = 1}^{n} \left( g(k) \omega^{-km} + h(m) \omega^{-km} \nu^m \right) &= \frac{1}{2n} \sum_{k = 1}^n f(2k) \nu^{-2km} + f(2k + 1) \nu^{-(2k+1)m}\\
    &= \frac{1}{2n} \sum_{k = 1}^{2n} f(k) \nu^{-km}
\end{align*}
%
This is essentially a discrete analogue of the Poission summation formula, which we will generalize later when we study the harmonic analysis of abelian groups. If $H(m)$ is the number of operations needed to calculate the Fourier transform of a function on $\mu_{2^n}$ using the above recursive formula, then the above relation tells us $H(2m) = 2H(m) + 3 (2m)$. If $G(n) = H(2^n)$, then $G(n) = 2G(n-1) + 3 2^n$, and $G(0) = 1$, and it follows that
%
\[ G(n) = 2^n + 3 \sum_{k = 1}^n 2^{k} 2^{n-k} = 2^n(1 + 3n) \]
%
Hence for $m = 2^n$, we have $H(m) = m(1 + 3 \log (m)) = O(m \log m)$. Similar techniques show that one can compute the inverse Fourier transform in $O(m \log m)$ operations (essentially by swapping the root $\nu$ with $\nu^{-1}$).

\section{Dirichlet's Theorem}

We now apply the theory of Fourier series on finite abelian groups to prove Dirichlet's theorem.

\begin{theorem}
    If $m$ and $n$ are relatively prime, then the set
    %
    \[ \{ m + kn : k \in \mathbf{N} \} \]
    %
    contains infinitely many prime numbers.
\end{theorem}

An exploration of this requries the Riemann-Zeta function, defined by
%
\[ \zeta(s) = \sum_{n = 1}^\infty \frac{1}{n^s} \]
%
The function is defined on $(1,\infty)$, since for $s > 1$ the map $t \mapsto 1/t^s$ is decreasing, and so
%
\[ \sum_{n = 1}^\infty \frac{1}{n^s} \leq 1 + \int_{1}^\infty \frac{1}{t^s} = 1 + \lim_{n \to \infty} \frac{1}{s-1} \left[1 - 1/n^{s-1} \right] = 1 + \frac{1}{s-1} \]
%
The series converges uniformly on $[1+\varepsilon, N]$ for any $\varepsilon > 0$, so $\zeta$ is continuous on $(1,\infty)$. As $t \to 1$, $\zeta(t) \to \infty$, because $n^s \to n$ for each $n$, and if for a fixed $M$ we make $s$ close enough to $1$ such that $|n/n^s - 1|<  1/2$ for $1 \leq n \leq M$, then
%
\[ \sum_{n = 1}^\infty \frac{1}{n^s} \geq \sum_{n = 1}^M \frac{1}{n^s} = \sum_{n = 1}^M \frac{1}{n} \frac{n}{n^s} \geq \frac{1}{2} \sum_{n = 1}^M \frac{1}{n} \]
%
Letting $M \to \infty$, we obtain that $\sum_{n = 1}^\infty \frac{1}{n^s} \to \infty$ as $s \to 1$.

The Riemann-Zeta function is very good at giving us information about the prime integers, because it encodes much of the information about the prime numbers.

\begin{theorem}
    For any $s > 1$,
    %
    \[ \zeta(s) = \prod_{p\ \text{prime}} \frac{1}{1 - p^s} \]
\end{theorem}
\begin{proof}
    The general idea is this -- we may write
    %
    \[ \prod_{p\ \text{prime}} \frac{1}{1 - p^s} = \prod_{p\ \text{prime}} (1 + 1/p^{s} + 1/p^{2s} + \dots) \]
    %
    If we expand this product out formally, enumating the primes to be $p_1, p_2, \dots$, we find
    %
    \[ \prod_{p \leq n} (1 + 1/p^s + 1/p^{2s} + \dots) = \sum_{n_1, n_2, \dots = 0}^\infty \frac{1}{p_1^{n_1}} \]
\end{proof}









\chapter{Complex Methods}

In this chapter, we illustrate the intimate connection between the Fourier transform on the real line, and complex analysis. We have already seen some aspects of this for Fourier analysis on the Torus, with the connection between power series of analytic functions on the unit disk. The main theme is that if $f$ is a function initially defined on the real line, then the problem of extending the function to be analytic on a neighbourhood of this line is connected to to the Fourier transform of $f$ decaying very rapidly (for instance, exponential decay).

\section{Fourier Transforms of Holomorphic Functions}

For each $a > 0$, let $S_a = \{ x + iy: |y| < a \}$ denote the horizontal strip of width $2a$. The next theorem says that functions extendable to be holomorphic on the strip have exponential Fourier decay.

\begin{theorem}
    Let $f: S_a \to \CC$ be holomorphic, integrable on each horizontal line in the strip, such that $f(x + iy) \to 0$ as $|x| \to \infty$. Then if $\widehat{f}$ is the Fourier transform of the restriction of $f$ to the real line, then for each $b < a$,
    %
    \[ |\widehat{f}(\xi)| \lesssim_b e^{-2 \pi b |\xi|}. \]
\end{theorem}
\begin{proof}
    For any $b < a$, $R$, and $\xi > 0$, consider the contour $\gamma_R$ on the rectangle with corners $-R$, $R$, $-R-ib$, and $R-ib$. As $R \to \infty$, the integral along the vertical lines of the rectangle tends to zero as $R \to \infty$, so we conclude that
    %
    \begin{align*}
        \int_{-\infty}^\infty f(x)e^{-2\pi i x \xi}\; dx &= \int_{-\infty}^\infty f(x-ib)e^{- 2 \pi i (x - ib) \xi}\; dx\\
        &= e^{-2 \pi i b \xi} \int_{-\infty}^\infty f(x-ib) e^{- 2 \pi i \xi x}\; dx = e^{-2 \pi i b \xi} \widehat{f_b}(\xi)
    \end{align*}
    %
    where $f_b(x) = f(x - ib)$. But $|\widehat{f_b}(\xi)| \leq \| f_b \|_{L^\infty(\RR)} \lesssim_b 1$, which implies that
    %
    \[ |\widehat{f}(\xi)| \lesssim_b e^{-2 \pi i b \xi}. \]
    %
    A similar estimate when $\xi < 0$ completes the argument.
\end{proof}

It follows that $\widehat{f}$ has exponential decay if $f$ satisfies the hypothesis of the theorem. Thus we can always apply the inverse Fourier transform to conclude
%
\[ f(x) = \int_{-\infty}^\infty \widehat{f}(\xi) e^{2 \pi i \xi x}\; d\xi. \]
%
Conversely, if $f$ is \emph{any} integrable function with $|\widehat{f}(\xi)| \lesssim e^{-2 \pi a |\xi|}$, then $\widehat{f}$ is integrable so the Fourier inversion formula holds. If we define
%
\[ f(x + iy) = \int_{-\infty}^\infty \widehat{f}(\xi) e^{-2 \pi \xi y} e^{2 \pi i \xi x}\; d\xi, \]
%
then this gives a holomorphic extension of $f$ which is well defined on $S_a$.

Pushing this result to an extreme leads to the Paley-Wiener theorem, which gives precise conditions when a function has a compactly supported Fourier transform.

\begin{theorem}
    A function $f: \RR \to \CC$ is bounded, integrable, and continuous. Then $f$ extends to an entire function on the complex plane, such that for all $z$,
    %
    \[ |f(z)| \lesssim e^{2 \pi M |z|}, \]
    %
    if and only if $\widehat{f}$ is supported on $[-M,M]$.
\end{theorem}
\begin{proof}
    If $\widehat{f}$ is supported on $[-M,M]$, then the Fourier inversion formula comes into play, telling us that for all $x \in \RR$,
    %
    \[ f(x) = \int_{-\infty}^\infty \widehat{f}(\xi) e^{2 \pi i \xi x}\; d\xi = \int_{-M}^M \widehat{f}(\xi) e^{2 \pi i \xi x}\; d\xi. \]
    %
    But then we can clearly extend $f$ to an entire function by defining
    %
    \[ f(z) = \int_{-M}^M \widehat{f}(\xi) e^{2 \pi i \xi z}\; d\xi, \]
    %
    and then $|f(z)| \leq e^{2 \pi i M |z|} \| \widehat{f} \|_{L^1[-M,M]} \lesssim e^{2 \pi i M |z|}$.

    Conversely, suppose $f$ is an entire function such that for all $z \in \CC$,
    %
    \[ |f(z)| \leq A g(x) e^{2 \pi M |y|}, \]
    %
    where $g \geq 0$ is integrable on $\RR$. We also assume that $f(x + iy) \to 0$ uniformly as $x \to -\infty$, independently of $y$. Then a contour shift down guarantees that for any $y$,
    %
    \begin{align*}
        \widehat{f}(\xi) &= \int_{-\infty}^\infty f(x) e^{-2 \pi i \xi x}\; dx\\
        &= \int_{-\infty}^\infty f(x - iy) e^{-2 \pi i \xi (x - iy)}\; dx\\
        &\leq A e^{2 \pi M y - 2 \pi \xi y} \int_{-\infty}^\infty g(x) \; dx \lesssim e^{2 \pi (M y - \xi y)}.
    \end{align*}
    %
    If $\xi > M$, then taking $y \to \infty$ shows $\widehat{f}(\xi) = 0$. A contour shift up instead gives $\widehat{f}(\xi) = 0$ if $\xi < -M$. Thus the proof is completed in this case.

    Now suppose the weaker condition
    %
    \[ |f(z)| \leq A e^{2 \pi M |y|}. \]    
    %
    For each $\varepsilon > 0$, let
    %
    \[ f_\varepsilon(z) = \frac{f(z)}{(1 - i\varepsilon z)^2}. \]
    %
    Then $f_\varepsilon$ is analytic in the lower half plane. Moreover,
    %
    \[ |f_\varepsilon(x + iy)| \lesssim_\varepsilon \frac{A e^{2\pi M |y|}}{1 + x^2}. \]
    %
    Thus we can apply the previous shifting techniques to show that $\widehat{f_\varepsilon}(\xi) = 0$ for $\xi > M$. For $x \in \RR$, we have $|f_\varepsilon(x)| \leq |f(x)|$, and since $f_\varepsilon \to f$ pointwise as $\varepsilon \to 0$, we can apply the dominated convergence theorem to imply $\widehat{f_\varepsilon}(\xi) \to \widehat{f}(\xi)$ for each $\xi$. In particular, we find $\widehat{f}(\xi) = 0$ for $\xi > M$. A similar technique with the family of functions
    %
    \[ f_\varepsilon(z) = \frac{f(z)}{(1 + i\varepsilon z)^2}, \] 
    %
    show that $\widehat{f}(\xi) = 0$ for $\xi < -M$.

    Finally, it suffices to show that the condition
    %
    \[ |f(z)| \lesssim e^{2\pi M |z|} \]
    %
    implies $|f(x + iy)| \lesssim e^{2 \pi M |y|}$. To prove this, we can apply a version of the Phragm\'{e}n-Lindel\"{o}f on the quandrant $\{ x + iy: x, y > 0 \}$. Let $g(z) = f(z) e^{-2 \pi i M y}$. Then we have
    %
    \[ |g(x)| = |f(x)| \leq \| f \|_{L^\infty(\RR)}, \]
    %
    and
    %
    \[ |g(iy)| = |f(iy)| e^{-2 \pi i M y} \leq A. \]
    %
    Since $g$ has at most exponential growth on the quadrant, we can apply the Phragm\'{e}n-Lindel\"{o}f to conclude $|g(z)| \leq \max(A, \| f \|_{L^\infty(\RR)})$ for all $z$ on the quandrant. A similar argument works for the other quadrants. Thus we conclude that for all $z \in \CC$
    %
    \[ |f(z)| \leq \max(A, \| f \|_{L^\infty(\RR)}) e^{2 \pi i M |y|}, \]
    %
    and so we can apply the previous cases to conclude that $\widehat{f}$ is supported on $[-M,M]$.
\end{proof}

\begin{remark}
    The Paley-Wiener theorem has several variants. For instance, if $f$ is continuous, integrable, and $\widehat{f}$ is integrable, and we further assume that $\widehat{f}(\xi) = 0$ for all $\xi < 0$, then for $z = x + iy$, we can define
    %
    \[ f(z) = \int_0^\infty \widehat{f}(\xi) e^{2 \pi i \xi z} = \int_0^\infty \widehat{f}(\xi) e^{- 2 \pi \xi y} e^{2 \pi i \xi x} \]
    %
    to extend $f$ to an analytic function in the upper half-plane, i.e. for $y > 0$, which is also continuous and bounded for $y \geq0$. Conversely, similar techniques to those above enable us to show that if $f$ is continuous, integrable, $\widehat{f}$ is integrable, and we can extend $f$ to an analytic function on the open upper half plane, which is continuous and bounded on the closed half plane, then contour shifting shows that $\widehat{f}(\xi) = 0$ for $\xi < 0$.
\end{remark}

\section{Classical Theorems by Contours}

We now prove some classical theorems of Fourier analysis using techniques of harmonic analysis, given that the functions we study have holomorphic extensions to tubes.

\begin{theorem}
    Let $f: S_b \to \CC$ be holomorphic. Then for any $x \in \RR$,
    %
    \[ f(x) = \int_{-\infty}^\infty \widehat{f}(\xi) e^{2 \pi i \xi x}\; dx, \]
    %
    where $\widehat{f}$ is the Fourier transform of $f$ restricted to the real-axis.
\end{theorem}
\begin{proof}
    As in the last theorem, the sign of $\xi$ matters. We write
    %
    \[ \int_{-\infty}^\infty \widehat{f}(\xi) e^{-2 \pi i \xi x} = \int_0^\infty \widehat{f}(\xi) e^{- 2 \pi i \xi x} + \widehat{f}(-\xi) e^{2 \pi i \xi x}. \]
    %
    Now if $b < a$, we can apply a contour integral argument to conclude that
    %
    \begin{align*}
        \widehat{f}(\xi) &= \int_{-\infty}^\infty f(x - ib) e^{-2 \pi i \xi (x - ib)}\; dx\\
        &= \int_{-\infty}^\infty f(x + ib) e^{2 \pi i \xi (x + ib)}\; dx.
    \end{align*}
    %
    Thus by Fubini's theorem, for each $x_0 \in \RR$,
    %
    \begin{align*}
        \int_0^\infty \widehat{f}(\xi) e^{2 \pi i \xi x_0} &= \int_0^\infty \int_{-\infty}^\infty f(x - ib) e^{2 \pi i \xi [x_0 - (x - ib)]}\; dx\; d\xi\\
        &= \int_{-\infty}^\infty f(x - ib) \left( \int_0^\infty e^{2 \pi i \xi [x_0 - (x - ib)]}\; d\xi \right)\; dx\\
        &= \frac{1}{2\pi i} \int_{-\infty}^\infty \frac{f(x - ib)}{(x - ib) - x_0}\; dx.
    \end{align*}
    %
    Similarily, another application of Fubini's theorem implies
    %
    \begin{align*}
        \int_0^\infty \widehat{f}(-\xi) e^{-2 \pi i \xi x_0}\; d\xi &= \int_0^\infty \int_{-\infty}^\infty f(x + ib) e^{-2 \pi i \xi [x_0 - (x + ib)]}\; dx\; d\xi\\
        &= \int_{-\infty}^\infty f(x + ib) \int_0^\infty e^{-2 \pi i \xi [x_0 - (x + ib)]}\; d\xi\; dx\\
        &= \frac{-1}{2 \pi i} \int_{-\infty}^\infty \frac{f(x + ib)}{[(x + ib) - x_0]}\; dx.
    \end{align*}
    %
    In particular, we conclude that
    %
    \[ \int \widehat{f}(\xi) e^{2 \pi i \xi x_0} = \frac{1}{2\pi i} \int_\gamma \frac{f(z)}{z - x_0}, \]
    %
    where $\gamma$ is the path traces over the two horizontal strips $x + ib$ and $x - ib$. Approximating this integral by rectangles, and then apply Cauchy's theorem, we find this value is equal to $f(x)$.
\end{proof}

We can also prove the Poisson summation formula.

\begin{theorem}
    Let $f: S_a \to \CC$ be holomorphic. Then
    %
    \[ \sum_{n \in \ZZ} f(n) = \sum_{n \in \ZZ} \widehat{f}(n), \]
    %
    where $\widehat{f}$ is the Fourier transform of $f$ restricted to the real line.
\end{theorem}
\begin{proof}
    The function
    %
    \[ \frac{f(z)}{e^{2 \pi i z} - 1} \]
    %
    is meromorphic, with simple poles on $\ZZ$, with reside equal to $f(n)$ at each $n \in \ZZ$. If we apply the Residue theorem to a curve $\gamma_N$ travelling around the rectangle connecting the points $N+1/2-ib$, $N+1/2+ib$, $-N-1/2+ib$, and $-N-1/2-ib$, then we conclude
    %
    \[ \sum_{|n| \leq N} f(n) = \int_{\gamma_N} \frac{f(z)}{e^{2 \pi i z} - 1}\; dz. \]
    %
    These values converge to $\sum_{n \in \ZZ} f(n)$ as $N \to \infty$. But this means that
    %
    \[ \sum_n f(n) = \int_\gamma \frac{f(z)}{e^{2 \pi i z} - 1}\; dz, \]
    %
    where $\gamma$ is the two horizontal strips at $b$ and $-b$. Now we use the expansion
    %
    \[ \frac{1}{z - 1} = \sum_{n = 1}^\infty z^{-n}, \]
    %
    for $|z| > 1$, to conclude
    %
    \begin{align*}
        \int_{-\infty}^\infty \frac{f(x - ib)}{e^{2 \pi i (x - ib)} - 1}\; dx &= \int_{-\infty}^\infty \sum_{n = 1}^\infty \frac{f(x - ib)}{e^{2 \pi n i (x - ib)}}\; dx\\
        &= \sum_{n = 1}^\infty \int_{-\infty}^\infty f(x - ib) e^{-2 \pi n i (x - ib)}\; dx = \sum_{n = 1}^\infty \widehat{f}(n),
    \end{align*}
    %
    where we have performed a contour shift at the end. Similarily, we use the expansion
    %
    \[ \frac{1}{z - 1} = - \sum_{n = 0}^\infty z^n, \]
    %
    to conclude that
    %
    \begin{align*}
        - \int_{-\infty}^\infty \frac{f(x + ib)}{e^{2 \pi i (x + ib)} - 1} &= \int_{-\infty}^\infty \sum_{n = 0}^\infty f(x + ib) e^{2 \pi i (x + ib)}\; dx\\
        &= \sum_{n = 0}^\infty \widehat{f}(-n).
    \end{align*}
    %
    Combining these two calculations completes the proof.
\end{proof}

\section{The Laplace Transform}

We now look at things from the dual perspective. Instead of looking at whether a function can be extended to a holomorphic function, we look at whether the Fourier transform can be extended to a holomorphic function. For a function $x: \RR \to \RR$, this gives rise to the \emph{Laplace transform}
%
\[ X(z) = \int_{-\infty}^\infty x(t) e^{- z t}\; dt, \]
%
also denoted by $(\mathcal{L}x)(z)$. For $\xi \in \RR$, $X(i\xi) = \widehat{x}(\xi)$ operates as the usual Fourier transform (slightly rescaled from the version in our notes). But the Laplace transform can also be extended to not-necessarily integrable functions. Given $x$, we can define $X(z)$ for any $z = x + iy$ such that
%
\[ \int e^{-xt} |f(t)|\; dt < \infty. \]
%
It is simple to see this forms a vertical tube in the complex plane, called the \emph{region of convergence} for the Laplace transform. For a particular vertical tube $I \subset \CC$, we let $\mathcal{E}(I)$ be the collection of all functions $x$ whose region of convergence for the Dirichlet transform contains $I$.

\begin{example}
    Let
    %
    \[ H(x) = \begin{cases} 0 &: x < 0, \\ 1/2 &: x = 0, \\ 1 &: x > 0. \end{cases} \]
    %
    The function $H$ is called the \emph{Heavyside Step Function}. It's region of convergence consists of the right-most half plane, i.e. all $\omega + i\xi$, where $\omega > 0$. And if $z = \omega + i\xi$, we calculate that
    %
    \[ \mathcal{L}(H)(z) = \int_0^\infty e^{- z t}\; dt = z^{-1}. \]
    %
    We note that even though the integral formula does not define the Laplace transform of $H$ in the right-most half plane, we \emph{can} analytically continue $\mathcal{L}(H)$ to a meromorphic function on the entire complex plane.
\end{example}

\begin{example}
    Similarily, an integration by parts shows that for $z = \omega + i\xi$ with $\omega > 0$, we have
    %
    \[ \mathcal{L}(tH)(z) = \int_0^\infty t e^{-zt} = \int_0^\infty \frac{e^{-zt}}{z} = z^{-2}. \]
    %
    Against, $\mathcal{L}(tH)$ extends to a meromorphic function on the entire complex plane.
\end{example}

\begin{comment}
    The Laplace transform is useful because it connects the study of the Fourier transform of a function to the study of certain complex analytic functions. For simplicity, we work with functions on the half-line, which eliminates some symmetry at the cost of a more simple theory. For a function $f: [0,\infty) \to \CC$, and $z \in \CC$, we study the integral transform
%
\[ (\mathcal{L} f)(z) = \int_{-\infty}^\infty f(t) e^{-zt}\; dt. \]
%
In some senses, the Laplace transform is a more general version of the Fourier transform. Indeed, we find
%
\[ (\mathcal{L} f)(\omega + i \xi) = \widehat{f e^{- \omega t}}(\xi). \]
%
Thus the Laplace transform of $Lf$ at a particular value $\omega$ measures a weighted frequency representation of $f$. A major advantage is that $Lf$ is defined as the integral of $f$ against a holomorphic function, and in particular, is often a holomorphic function, which enables us to use techniques of complex analysis.

We fix $f \in L^1(\RR)$, andse $f$ is supported on $[-N,\infty)$ for some large $N$. Then for any $\omega \geq 0$,
%
\[ \int_{-\infty}^\infty |f(t)| e^{-\omega t} < \infty. \]
%
Thus the integral
%
\[ \int f(t) e^{-zt}\; dt \]
%
is defined as an absolutely convergent integral for all $z = \omega + i\xi$ with $\omega \geq 0$. Thus we can define the Laplace transform $(\mathcal{L} f)(z)$ for all $z$ in the closed right half-plane. If $\gamma$ is a closed curve in the open right half-plane, and $f \in L^1(\RR)$, then Fubini's theorem implies that
%
\begin{align*}
    \int_\gamma \mathcal{L} f\; dz &= \int_0^1 (\mathcal{L} f)(\gamma(s)) \gamma'(s)\; ds\\
    &= \int_0^1 \int_0^\infty f(t) e^{- \gamma(s) t} \gamma'(s)\; dt\; ds\\
    &= \int_0^\infty f(t) \left( \int_\gamma e^{-zt}\; dz \right)\; dt = \int_0^\infty 0\; dt = 0.
\end{align*}
%
Thus Morera's theorem implies $\mathcal{L} f$ is analytic in the open right half-plane. The Dominated convergence theorem also implies $\mathcal{L} f$ is continuous in the closed half-plane. If we also assume that $f$ is compactly supported, then the Laplace $\mathcal{L} f$ is defined everywhere, and is an entire function.
\end{comment}

\begin{comment}
We often study functions supported on $[0,\infty)$, in which case it suffices to analyze the `one sided' Laplace transform
%
\[ (\mathcal{L} f)(\omega + i\xi) = \int_0^\infty f(x) e^{-2 \pi (\omega + i \xi)t}\; dt. \]
%
The reason for this is quite simple. The Laplace transform is often used to analyze certain convolution operators $Tf = f * g$. One views the function $f(t)$ as a certain signal, with amplitudes varying over a time period. Most often, $T$ is an operator which is computed `online'; we think of feeding in the signal $f(t)$ in real time, and then produce $(Tf)(t)$ at the same time. For example, the operator $Tf = f * H$, where $H$ is the heavyside step function, is defined so that
%
\[ (Tf)(t) = \int_{-\infty}^t f(s)\; ds, \]
%
so $(Tf)(t)$ can be produced given only knowledge of $f$ up to time $t$. In general, $(Tf)(t)$ depends only on $f$ up to time $t$ if and only if $g$ is supported on $[0,\infty)$. As expected, we will show $\mathcal{L}(f * g) = \mathcal{L}f \cdot \mathcal{L}g$ so in many senses it suffices to analyze the Laplace transform of a function defined on a half line.


To rigorously study the Laplace transform, we look at a nice family of functions for which the transform is particularly well behaved. For each $\alpha \in \RR$, we let $\mathcal{E}_\alpha$ be the collection of all functions $f$ supported on $[0,\infty)$ such that $f e^{- \alpha t} \in L^1(\RR^d)$. Then if $\omega \geq \alpha$, and $\xi \in \RR$, $(\mathcal{L}f)(\omega + i\xi)$ is well defined by the integral formula. Moreover, for $\omega > \alpha$, $\mathcal{L} f$ is actually an \emph{analytic} function, with
%
\[ (\mathcal{L}f)'(\omega + i \xi) = - 2\pi \cdot \mathcal{L}(tf)(\omega + i\xi). \]
%
This can be established by a simple approximation argument. The dominated convergence theorem also implies that $\mathcal{L}f$ is continuous on the closed half plane defined by $\omega \geq \alpha$.
\end{comment}

What distinguishes the Laplace transform from the Fourier transform is the ability to use techniques of complex analysis. If $x$ has region of convergence $I$, then $X$ is continuous on $I$, and analytic on $I^\circ$. We can even calculate an explicit formula for the derivative As expected from the Fourier transform of the derivative, if $y(t) = tx(t)$, and $Y$ is the Laplace transform of $y$, then $X'(z) = -Y(z)$. One can verify this quite simply by taking limits of the derivatives of the analytic integrals
%
\[ \int_{-N}^N x(t) e^{-zt}\; dt, \]
%
as $N \to \infty$. Like the Fourier transform, the Laplace transform is symmetric under modulation, translation, and polynomial multiplication:
%
\begin{itemize}
    \item If $w \in \CC$, and $x$ is a function, set $y(t) = e^{wt} x(t)$. Then if $z$ is in the region of convergence for $x$, $z - w$ is in the region of convergence for $y$, and $X(z) = Y(z-w)$.

    \item If $x$ has region of convergence $I$, then the region of convergence for $y(t) = tx(t)$ contains $I^\circ$, and $Y(z) = -X(z)$.

    \item If $x$ has region of convergence $I$, $t_0 \in \RR$, and we set $y(t) = x(t + t_0)$, then $y$ has region of convergence $I$, and $Y(z) = e^{zt_0} X(z)$.

    \item For a function $x$, define
    %
    \[ (\Delta_s x)(t) = \frac{x(t + s) - x(t)}{s}. \]
    %
    If $\omega$ is fixed, if
    %
    \[ \lim_{s \to 0} \int |(\Delta_s x)(t) - x'(t)| e^{-\omega t}\; dt = 0, \]
    %
    if $y(t) = x'(t)$, and if $z = \xi + i \omega$ for some $\xi \in \RR$, then $Y(z) = z X(z)$.

    In particular, this is true if $x$ is supported on $[-N,\infty)$ for some $N$, has a continuous derivative $x'$, and there is $\omega_0 < \omega$ such that
    %
    \[ \lim_{t \to \infty} x(t) e^{-\omega_0 t} = \lim_{t \to \infty} x'(t) e^{-\omega_0 t} = 0. \]
    %
%    It will be useful to consider functions $f$ with $f(t) = 0$ for $t < 0$, such that $f$ is continuously differentiable for $t > 0$, since such functions can be used to solve ordinary differential equations, but such that
    %
%    \[ f(0+) = \lim_{t \to 0^+} f(t) \]
    %
%    exists and is finite. Then $f'$ is defined everywhere but the origin, and an integration by parts tells us that
    %
%    \[ (\mathcal{L} f')(z) = z \cdot (\mathcal{L} f)(z) - f(0+). \]
    %
%    More generally, $(\mathcal{L} f^{(n)})(z) = z^n \cdot (\mathcal{L} f)(z)$
\end{itemize}

\begin{remark}
    It will be interesting for us to consider functions $x$ supported on $[-N,\infty)$ which have a piecewise continuous derivative $x'$ except at finitely many points $t_1, \dots, t_N$, such that the left and right-hand limits exist at each $t_i$. For each $i \in \{ 1, \dots, N \}$, we let
    %
    \[ A_i = x(t_i+) - x(t_i-) \quad\text{and}\quad B_i = x'(t_i+) - x'(t_i-). \]
    %
    If $y(t) = x'(t)$, we calculate a relation between the Laplace transforms of $X$ and $Y$ at $z = \omega + i\xi$ such that there exists $\omega_0 < \omega$ such that
    %
    \[ \lim_{t \to \infty} x(t) e^{-\omega_0 t} = \lim_{t \to \infty} x'(t) e^{-\omega_0 t} = 0. \]
    %
    We consider the function
    %
    \[ x_1(t) = x(t) - \sum_{i = 1}^N A_i H(t - t_i) - \sum_{i = 1}^N B_i (t - t_i) H(t - t_i). \] 
    %
    Then $x_1$ is continuous everywhere, and moreover, has a continuous derivative. We have
    %
    \[ x_1'(t) = x'(t) - \sum_{i = 1}^N B_i H(t - t_i). \]
    %
    Thus if $\omega > 0$, and $z = \omega + i \xi$, if $y_1(t) = x_1'(t)$, we find
    %
    \[ Y_1(z) = z X_1(z). \]
    %
    Now
    %
    \[ Y_1(z) = Y(z) - \sum_{i = 1}^N \frac{B_i e^{-i z t_i}}{iz} \]
    %
    and
    %
    \[ X_1(z) = X(z) - \sum_{i = 1}^N \frac{A_i e^{-i z t_i}}{iz} + \sum_{i = 1}^N \frac{B_i e^{-i z t_i}}{z^2}. \]
    %
    Thus, rearranging, we conclude
    %
    \[ Y(z) = z X(z) - \sum_{i = 1}^N A_i e^{-i z t_i} \]
    %
    We can carry this through recursively to higher order derivatives. For each $k$, we set $A^k_i = f^{(k)}(t_i+) - f^{(k)}(t_i-)$. Then if $y(t) = f^{(n)}(t)$, then
    %
    \[ Y(z) = z^n X(z) - \sum_{k = 0}^{n-1} \sum_{i = 1}^N z^{n-1-k} A^k_i e^{-izt_i}. \]
    %
    This is very useful when wants to solve differential equations without involving distribution theory, provided the solutions to those differential equations do not grow faster than exponentially.
\end{remark}

\begin{example}
    Suppose we wish to find a formula for the unique real-valued function $x: [0,\infty) \to \RR$ such that $x''(t) - x'(t) - 6x(t) = 5e^{3t}$ for $t \geq 0$, such that $x(0) = 6$ and $x'(0) = 1$. Such a function increases at most exponentially, since it is linear, so we may take the Laplace transform of each sides to conclude that if $X$ is the Laplace transform of $x$, then
    %
    \[ \mathcal{L}(x'')(z) = z^2 X(z) - 6z - 1 \quad\text{and}\quad \mathcal{L}(x')(z) = z X(z) - 6. \]
    %
    Thus we conclude
    %
    \[ [z^2 X(z) - 6z - 1] - [zX(z) - 6] - (6X) = \frac{5}{z - 3}. \]
    %
    Thus
    %
    \[ X(z) = \frac{(3z - 4)(2z - 5)}{(z - 3)^2(z+2)} = \frac{3.6}{z + 2} + \frac{2.4}{z - 3} + \frac{1}{(z - 3)^2}. \]
    %
    But this implies that for $t \geq 0$, $x(t) = 3.6 e^{-2t} + 2.4 e^{3t} + te^{3t}$. In particular, we note that the pole of $X$ determines the large scale behaviour of $X$, i.e. for large $t$, and for any $\varepsilon > 0$,
    %
    \[ e^{(3 - \varepsilon)t} \lesssim_\varepsilon x(t) \lesssim_\varepsilon e^{(3 + \varepsilon)t}. \]
    %
    In the next section, we generalize this situation to give asymptotics of functions whose Laplace transforms extend to meromorphic functions on the complex plane.
\end{example}

\section{Asymptotics via the Laplace Transform}

For simplicity, in this chapter we study integrable functions $x: [0,\infty) \to \RR$, whose Laplace transform is thus well defined on the closed, right half-plane. If the Fourier transform of $x$ is integrable, then we can apply the inversion formula to conclude that for each $t \in \RR$,
%
\[ x(t) = \int_{-\infty}^\infty X(i\xi) e^{i \xi t}\; d\xi. \]
%
Now suppose that $X$ can be analytically continued to a holomorphic function $X(\omega + i\xi)$ for all $\omega \geq -\varepsilon$ which is continuous at the boundary, such that, uniformly for $\omega \in [-\varepsilon,0]$,
%
\[ \lim_{|\xi| \to \infty} X(\omega + i\xi) = 0. \]
%
Then a contour shift argument implies that for each $t$,
%
\[ x(t) = \lim_{R \to \infty} \int_{-R}^R X(-\varepsilon + i\xi) e^{(-\varepsilon + i\xi) t}\; d\xi = e^{-\varepsilon t} \lim_{R \to \infty} \int_{-R}^R X(-\varepsilon + i\xi) e^{i \xi t}\; d\xi. \]

For simplicity, we study functions supported on $[0,\infty)$. The region of convergence for such functions then takes the form of a half plane. For a given $a \in \RR$, we let $\mathcal{E}_a$ be the set of functions whose region of convergence contains $\omega + i\xi$ for all $\omega > a$.

\begin{theorem}
    Suppose $x: [0,\infty) \to \RR$ is a continuous function such that some $\omega$,
    %
    \[ \int |x(t)| e^{-\omega t}\; dt < \infty. \]
    %

\end{theorem}
\begin{proof}
    Since $|X(u + iv)| \to 0$ uniformly as $v \to \infty$, we can shift the Fourier inversion formula
    %
    \[ x(t) = \lim_{R \to \infty} \frac{1}{2\pi} \int_{-R}^R X(\omega + i\xi) e^{(\omega + i\xi)t}\; d\xi \]
    %
    (where the $2\pi$ comes up from our rescaling of the Fourier transform) to conclude that
    %
    \[ x(t) = \lim_{R \to \infty} \frac{1}{2\pi} \]
    %
    \[ X(z) = \lim_{} \]
\end{proof}


















\chapter{Applications}

\section{The Wirtinger Inequality on an Interval}

\begin{theorem}
    Given $f \in C^1[-\pi,\pi]$ with $\int_{-\pi}^\pi f(t) dt = 0$,
    %
    \[ \int_{-\pi}^\pi |f(t)|^2 \leq \int_{-\pi}^\pi |f'(t)|^2 \]
\end{theorem}
\begin{proof}
    Consider the fourier series
    %
    \[ f(t) \sim \sum a_n e_n(t)\ \ \ \ \ f'(t) \sim \sum in a_n e_n(t) \]
    %
    Then $a_0 = 0$, and so
    %
    \[ \int_{-\pi}^\pi |f(t)|^2\ dt = 2 \pi \sum |a_n|^2 \leq 2 \pi \sum n^2 |a_n|^2 = \int_{-\pi}^\pi |f'(t)|^2\ dt \]
    %
    equality holds here if and only if $a_i = 0$ for $i > 1$, in which case we find
    %
    \[ f(t) = A e_n(t) + \overline{A} e_n(-t) = B \cos(t) + C \sin(t) \]
    %
    for some constants $A \in \mathbf{C}$, $B,C \in \RR$.
\end{proof}

\begin{corollary}
    Given $f \in C^1[a,b]$ with $\int_a^b f(t)\ dt = 0$,
    %
    \[ \int_a^b |f(t)|^2 dt \leq \left(\frac{b-a}{\pi}\right)^2 \int_a^b |f'(t)|^2\ dt \]
\end{corollary}

\section{Energy Preservation in the String equation}

Solutions to the string equation are

If $u(t,x)$

\section{Harmonic Functions} 

The study of a function $f$ defined on the real line can often be understood by extending it's definition holomorphically to the complex plane. Here we will extend this tool, establishing that a large family of functions $f$ defined on $\RR^n$ can be understood by looking at a {\it harmonic} function on the upper half plane $\mathbf{H}^{n+1}$, which approximates $f$ at it's boundary. This is a form of the Dirichlet problem, which asks, given a domain and a function on the domain's boundary, to find a function harmonic on the interior of the domain which `agrees' with the function on the boundary, in one of several senses. As we saw in our study of harmonic functions on the disk in the study of Fourier series, we can study such harmonic functions by convolving $f$ with an appropriate approximation to the identity which makes the function harmonic in the plane. In this case, we shall use the Poisson kernel for the upper half plane.

\begin{theorem}
    If $f \in L^p(\RR^n)$, for $1 \leq p \leq \infty$, and $u(x,y) = (f * P_y)(x)$, where
    %
    \[ P_y(x) = \frac{\Gamma((n+1)/2)}{\pi^{(n+1)/2}} \frac{1}{(1 + |x|^2)^{(n+1)/2}} \]
    %
    then $u$ is harmonic in the upper half plane, $u(x,y) \to f(x)$ for almost every $x$, and $u(\cdot,y)$ converges to $f$ in $L^p$ as $y \to 0$, with $\| u(\cdot,y) \|_{L^p(\RR^n)} \leq \| f \|_{L^p(\RR^n)}$. If, instead, $f$ is a continuous and bounded function, then $u(\cdot,y)$ converges to $f$ locally uniformly as $y \to 0$.
\end{theorem}
\begin{proof}
    The almost everywhere convergence and convergence in norm follow from the fact that $P_y$ is an approximation to the identity. The fact that $u$ is harmonic follows because
    %
    \[ u_{xx}(x,y) = (f * P_y'')(x)\ \ \ \ \ u_{yy} = (f * ) \]
\end{proof}















%% The following is a directive for TeXShop to indicate the main file
%%!TEX root = HarmonicAnalysis.tex

\part{Calderon-Zygmund Theory}

Here, we try and describe the more modern approaches to real-variable harmonic analysis, as developed by the \emph{Calderon-Zygmund school} in the 1960s and 1970s. Almost all of the problems we consider can be phrased as showing some operator is bounded as a map between functions spaces. Given some function $f$ lying in a space $V$, we have an associated function $Tf$ lying in some space $W$. The main goal of the techniques in this part of the book attempt to understand how quantitative control on certain properties of $f$ imply quantitative control on properties of $Tf$. In particular, given some quantity $A(f)$ associated with each $f \in V$, and a quantity $B(g)$ defined for all $g \in W$, our goal is to understand whether a general bound $B(Tf) \lesssim A(f)$ is possible for all functions $f \in V$, i.e. whether these exists a universal constant $C > 0$ such that $B(Tf) \leq C \cdot A(f)$ for all $f \in V$.

A core technique we employ here is the method of \emph{decomposition}. We write $f = \sum_k f_k$, where the function $f_k$ have particular properties, perhaps being concentrated in a particular region of space, or having a Fourier transform concentrated in a particular region. These concentration properties often simplify the analysis of the operator $T$, enabling us to obtain bounds $B(Tf_k) \lesssim A(f_k)$ for each $n$. Provided that the operator $T$, and the quantities $A$ and $B$ are `stable under addition', we can then obtain the bound $B(Tf) \leq A(f)$ by `summing' up the related quantities. The stability of $A$ and $B$ is often obtained by assuming these quantities are \emph{norms} on their respective function spaces, i.e. that there exists norms $\| \cdot \|_V$ and $\| \cdot \|_W$ such that $A(f) = \| f \|_V$ for each $f \in V$ and $B(g) = \| g \|_W$ for each $g \in W$. The stability of $T$ under addition is obtained by assuming linearity, or at least sub-linearity, in the sense that for each $f_1, f_2 \in V$,
%
\[ \| T(f_1 + f_2) \|_W \leq \| T f_1 \|_W + \| Tf_2 \|_W. \]
%
We can then use the triangle inequality to conclude that
%
\[ \| Tf \|_W \leq \sum_k \| Tf_k \|_W \lesssim \sum_k \| f_k \|_V. \]
%
Thus if $\sum_k \| f_k \|_V \lesssim \| f \|_V$, our argument is complete. This will be true, for instance, if there exists $\varepsilon > 0$ such that $\| f_k \|_V \lesssim 2^{- \varepsilon k} \| f \|_V$. This can often be obtained if we employ a \emph{dyadic decomposition technique}. For such decompositions, it is also possible to generalize are technique not only to norms, but also to \emph{quasinorms}, i.e. maps $\| \cdot \|$ which are homogeneous and satisfy a \emph{quasi-triangle inequality} $\| v + w \| \lesssim \| v \| + \| w \|$.

\begin{lemma}
    Suppose $\| \cdot \|_V$ is a quasi-norm on a vector space $V$, and under the topology induced by $\| \cdot \|_V$, we can write $f = \sum_{k = 1}^\infty f_k$, where there is $\varepsilon > 0$ and $C > 0$ such that for each $n$, $\| f_k \|_V \leq C \cdot 2^{-\varepsilon k}$. Then $\| f \|_V \lesssim_\varepsilon C$.
\end{lemma}

\begin{remark}
	Thus if $T$ is sublinear and we have $\| Tf_k \|_W \lesssim \| f_k \|_V$ and $\| f_k \|_V \lesssim 2^{- \varepsilon k} \| f \|_V$, we conclude $\| Tf_k \|_W \lesssim 2^{-\varepsilon k} \| f \|_V$, and then by sublinearity and the lemma applied to $\| \cdot \|_W$, we conclude
	%
	\[ \| Tf \|_W \leq \| \sum_k Tf_k \|_W \lesssim_\varepsilon \| f \|_V. \]
	%
	A slight modification of the proof below even gives this claim provided $T$ is \emph{quasi sublinear}, in the sense that for all $f_1, f_2 \in V$, $\| T(f_1 + f_2) \|_W \lesssim \| Tf_1 \|_V + \| Tf_2 \|_V$ for all $f_1, f_2 \in V$. However, such operators occur so rarely in practice that it isn't worth concentrating on them.
\end{remark}

\begin{proof}
	Pick $A > 0$ such that $\| f_1 + f_2 \|_V \leq A \cdot (\| f_1 \|_V + \| f_2 \|_V)$ for all $f_1$ and $f_2$. If $A < 2^{\varepsilon}$, we can write apply the quasitriangle inequality iteratively to conclude
    %
    \begin{align*}
        \| f \| &\leq C \cdot \sum_{k = 1}^\infty A^k \| f_k \|_V \leq C \cdot \left( \sum_{k = 1}^\infty (A 2^{-\varepsilon})^k \right) \leq C \cdot \left( \frac{1}{1 - A 2^{-\varepsilon}} \right) \lesssim_\varepsilon C.
    \end{align*}
    %
    In general, fix $N$, and write $f = f^1 + \dots + f^N$, where $f^m = \sum_{k = 0}^\infty f_{m + Nk}$. Then $\| f_{m + Nk} \|_V \leq C \cdot 2^{- N \varepsilon k}$, and if $N$ is chosen large enough that $A < 2^{N \varepsilon}$, we can apply the previous case to conclude that $\| f^m \|_V \lesssim_\varepsilon C$. Then we can apply the quasi-triangle inequality to conclude that $\| f \| \lesssim_\varepsilon C$.
\end{proof}

We can even apply the method of decomposition in the presence of suitably large polynomial decay.

\begin{lemma}
    Suppose $\| \cdot \|_V$ is a quasinorm on a function space $V$. Then there exists $t$ such that for all $s > t$, if $f = \sum_{k = 1}^\infty f_k$, and if $\| f_k \|_V \leq C \cdot k^{-s}$, for $s > t$, then $\| f \|_V \lesssim_s C$.
\end{lemma}
\begin{proof}
    As in the previous lemma, pick $A > 0$ such that $\| f_1 + f_2 \|_V \leq A (\| f_1 \|_V + \| f_2 \|_V)$ for all $f_1,f_2 \in V$. We perform a decomposition of dyadic type, writing $f = \sum_{m = 0}^\infty f^m$, where
    %
    \[ f^m = \sum_{k = 2^m}^{2^{m+1} - 1} f_k. \]
    %
    By splitting up the sum into a binary tree, we can ensure that
    %
    \[ \| f^m \|_V \lesssim A^{m+1} \sum_{k = 2^m}^{2^{m+1} - 1} \| f_k \|_V \leq C \cdot A^{m+1} \sum_{k = 2^m}^{2^{m+1} - 1} k^{-s} \lesssim C (A 2^{1-s})^m. \]
    %
    If $s > 1 + \lg(A)$, the previous lemma applies that $\| f \|_V \lesssim C$.
\end{proof}

In this part of the notes, we define the various classes of quasi-norms we will study, describe the general methods which make up the Calderon-Zygmund theory, and find applications to geometric measure theory, complex analysis, partial differential equations, and analytic number theory.





\chapter{Monotone Rearrangement Invariant Norms}

In this chapter, we discuss common families of \emph{monotone, rearrangement invariant quasinorms} that occur in harmonic analysis. The general framework is as follows. For each function $f$, we associate it's \emph{distribution function} $F: [0,\infty) \to [0,\infty]$ given by $F(t) = |\{ x : |f(x)| > t \}|$. A \emph{rearrangement invariant space} is a subspace $V$ of the collection of measurable complex-valued functions on some measure space $X$, equipped with a quasi-norm $\| \cdot \|$, satisfying the following two properties:
%
\begin{itemize}
    \item \emph{Monotonicity}: If $|f(x)| \leq |g(x)|$ for all $x \in X$, then $\| f \| \leq \| g \|$.

    \item \emph{Rearrangement-Invariance}: If $f$ and $g$ have the same distribution function, then $\| f \| = \| g \|$.
\end{itemize}
%
A monotone rearrangement-invariant norm essentially provides a way of quantifying the height and width of functions on $X$. It has no interest in the `shape' of the objects studied, because of the property of rearrangement invariance. In a particular problem, one picks the norm best emphasizing a particular family of features useful in the problem.

There are two very useful classes of functions useful for testing the behaviour of translation invariant norms:
%
\begin{itemize}
    \item The \emph{indicator functions} $\mathbf{I}_E(x) = \mathbf{I}(x \in E)$, for a measurable set $E$.
    \item The \emph{simple functions} $f = \sum_{i = 1}^n a_i \mathbf{I}_{E_i}$, for disjoint sets $E_i$.
\end{itemize}
%
The class of all simple functions forms a vector space, and for almost all the monotone rearrangement invariant norm we consider in this section, this vector space will form a dense subspace of the class of all functions. This means that when we want to study how an operator transforms the height and width of functions, the behaviour of the operator on simple functions often reflects the behaviour of an arbitrary function.

\section{The $L^p$ norms}

For $p \in (0,\infty)$, we define the $L^p$ norm on measurable function on a measure space $X$ by
%
\[ \| f \|_p = \left( \int |f(x)|^p\; dx \right)^{1/p}. \]
%
For $p = \infty$, we define
%
\[ \| f \|_\infty = \min \left\{ t \geq 0: |f(x)| \leq t\ \text{almost surely} \right\}. \]
%
These are the most fundamental monotone, rearrangement invariant norms. The space of functions $f$ with $\| f \|_p < \infty$ is denoted by $L^p(X)$. The most important spaces to consider here are the space $L^1(X)$, consisting of absolutely square integrable functions, $L^\infty(X)$, consisting of almost-everywhere bounded functions, and $L^2(X)$, consisting of square integrable functions. The main motivation for the introduction of the other $L^p$ spaces is that much of the quantitative theory for $p \in \{ 1, \infty \}$ is rather trivial, in the sense that it is easy to see when certain operators are bounded on these spaces, or unbounded.

As $p$ increases, the $L^p$ norm of a particular function $f$ gives more control over the height of the function $f$, and weaker control on values where $f$ is particular small. At one extreme, $L^\infty(X)$ only has control over the height of a function, and no control over it's width. Conversely, one can think of $L^0(X)$ as being the space of functions with finite support, though no natural norm exists on this space of functions solely classifying width. After all, such a quantity couldn't be homogenous, since the width of $f$ and $\alpha f$ are the same for each $\alpha \neq 0$. Thus the space $L^0(X)$ isn't so interesting to us from a quantitative perspective.

\begin{example}
  If $f(x) = |x|^{-s}$ for $x \in \RR^d$ and $s > 0$, then integration by radial coordinates shows that
  %
  \[ \int_{\varepsilon \leq |x| \leq M} \frac{1}{|x|^{s p}}\; dx \approx \int_\varepsilon^M r^{d-1 - ps}\; dr = \frac{M^{d - p s} - \varepsilon^{d - p s}}{d - p s}. \]
  %
  This quantity remains finite as $\varepsilon \to 0$ if and only if $d > p s$, and finite as we let $M \to \infty$ if and only if $d < p s$. Thus if $p < d/s$, $f$ is \emph{locally} in $L^p$, in the sense that $f \in L^p(B)$ for every bounded $B \in \RR^d$. The class of functions for which this condition holds is denoted $L^p_{\text{loc}}(X)$. Conversely, if $p > d/s$, then for every domain $B$ separated from the origin, $f \in L^p(B)$. For $p = d/s$, the function $f$ fails to be $L^p(\RR^d)$, but only `by a logarithm', in the sense that
  %
  \[ \int_{\varepsilon \leq |x| \leq M} \frac{1}{|x|^{s p}}\; dx \approx \int_\varepsilon^M \frac{dr}{r} = \log(M/\varepsilon). \]
  %
  We will later find `weaker' versions of the $L^p$ norm, and $f$ will have finite version of these norms.
\end{example}

The last example shows that, roughly speaking, control on the $L^p$ norm of a function for large values of $p$ prevents the formation of higher order singularities, and control of the norm for small values of $p$ ensures that functions have large decay at infinity.

\begin{example}
  If $s = A \chi_E$, and we set $H = |A|$ and $W = |E|$, then $\| s \|_p = W^{1/p} H$. As $p \to \infty$, the value of $\| s \|_p$ depends more and more on $H$, and less on $W$, and in fact $\lim_{p \to \infty} \| s \|_p = H$. If $s = \sum A_n \chi_{E_n}$, and $|A_m|$ is the largest constant from all other values $A_n$, then as $p$ becomes large, $|A_m|^p$ overwhelms all other terms. We calculate that as $p \to \infty$,
  %
  \[ \| s \|_p = \left( \sum |E_n| |A_n|^p \right)^{1/p} = |A_m|^p (|E_m| + o(1))^{1/p} = |A_m| (1 + o(1)). \]
  %
  This implies $\| s \|_p \to |A_m|$ as $p \to \infty$. But as $p \to 0$, $\lim_{p \to 0} \| f \|_p$ does not in general exist, even for step functions with finite support. Nonetheless, we can conclude that $\lim_{p \to 0} \| s \|_p^p = \sum |E_n|$, which is the measure of the support of $s$.
\end{example}

As $p \to \infty$, the width of a function is disregarded completely by the $L^p$ norm, motivating the definition of \emph{the $L^\infty$ norm}; Given a measurable $f$, we define $\| f \|_\infty$ to be the smallest number such that $|f| \leq \| f \|_\infty$ almost surely. We then define $L^\infty(X)$ to be the space of measurable functions $f$ for which $\| f \|_\infty < \infty$. We have already shown $\| s \|_p \to \| s \|_\infty$ if $s$ is a simple function, and the density of such functions gives a general result.

\begin{theorem}
    Let $p \in (0,\infty)$. If $f \in L^p(X) \cap L^\infty(X)$, then
    %
    \[ \lim_{t \to \infty} \| f \|_t = \| f \|_\infty. \]
\end{theorem}
\begin{proof}
    Without loss of generality, assume $p \geq 1$. Consider the norm $\| \cdot \|$ on $L^p(X) \cap L^\infty(X)$ given by
    %
    \[ \| f \| = \| f \|_p + \| f \|_\infty. \]
    %
    Then $L^p(X) \cap L^\infty(X)$ is complete with respect to this metric. For each $t \in [p,\infty)$, define $T_t(f) = \| f \|_t$. Then the functions $\{ T_t \}$ are uniformly bounded in the norm $\| \cdot \|$, since if $p = \theta t$, then
    %
    \[ |T_t(f)| = \| f \|_t \leq \| f \|_p^\theta \| f \|_\infty^{1-\theta} \leq \| f \|^\theta \| f \|^{1-\theta} = \| f \|. \]
    %
    For any $\varepsilon > 0$, we can find a step function $s$ with $\| s - f \|_p, \| s - f \|_\infty \leq \varepsilon$. This means that for all $t \in (p,\infty)$,$\| s - f \|_t \leq \varepsilon$. And so
    %
    \begin{align*}
        \Big| T_t(f) - \| f \|_\infty \Big| &\leq |T_t(f) - T_t(s)| + |T_t(s) - \| s \|_\infty| + |\| s \|_\infty - \| f \|_\infty| \leq 2\varepsilon + o(1).
    \end{align*}
    %
    Taking $\varepsilon \to 0$ gives the result.
\end{proof}

Abusing notation, we define $\| f \|_0^0 = | \text{supp} f | = | \{ x: f(x) \neq 0 \} |$, and let $L^0(X)$ be the space of functions with finite support. We know that for any simple function $s$, $\| s \|_p^p \to \| s \|_0^0$ as $p \to 0$. If $f \in L^0(X) \cap L^p(X)$ for some $p \in (0,\infty)$, then the monotone and dominated convergence theorems implies that
%
\[ \| f \|_0^0 = \int \mathbf{I}(f(x) \neq 0) = \int \left( \lim_{t \to 0} |f(x)|^t \right)\; dx = \lim_{t \to 0} \int |f(x)|^t\; dx = \lim_{t \to 0} \| f \|_t^t. \]
%
Thus the space $L^0(X)$ lies at the opposite end of the spectrum to $L^\infty$.

The fact that $\| f \|_0^0$ is a norm taken to the `power of zero' implies that many nice norm properties of the $L^p$ spaces fail to hold for $L^0(X)$. For instance, homogeneity no longer holds; in fact, for each $\alpha \neq 0$,
%
\[ \| \alpha f \|_0^0 = \| f \|_0^0. \]
%
It does, however, satisfy the triangle inequality $\| f + g \|_0^0 \leq \| f \|_0^0 + \| g \|_0^0$, which follows from a union bound on the supports of the functions.

\begin{example}
  Let $p < q$, and suppose $f \in L^p(X) \cap L^q(X)$. For any $r \in (p,q)$, the $L^r$ norm emphasizes the height of $f$ less than the $L^q$ norm, and emphasizes the width of $f$ less than the $L^p$ norm. In particular, we find that for any $\lambda \geq 0$,
  %
  \begin{align*}
    \| f \|_r^r = \int_{\RR} |f(x)|^r\; dx &= \int_{|f(x)| \leq 1} |f(x)|^r\; dx + \int_{|f(x)| > 1} |f(x)|^r\; dx\\
    &\leq \int_{|f(x)| \leq 1} |f(x)|^p\; dx + \int_{|f(x)| > 1} |f(x)|^q\; dx\\
    &\leq \| f \|_p^p + \| f \|_q^q < \infty.
  \end{align*}
  %
  In particular, this shows $f \in L^r(X)$.
\end{example}

\begin{remark}
    The bound obtained in the last example can be improved by using scaling symmetries. For any $A > 0$,
    %
    \[ \| f \|_r^r = \frac{\| Af \|_r^r}{A^r} \leq \frac{\| Af \|_p^p + \| Af \|_q^q}{A^r} \leq \frac{A^p \| f \|_p^p + A^q \| f \|_q^q}{A^r}. \]
    %
    If $1/r = \theta/p + (1 - \theta)/q$, and we set $A = \| f \|_q^{q/(p-q)} / \| f \|_p^{p/(p-q)}$, then the above inequality implies $\| f \|_r \leq 2 \| f \|_p^\theta \| f \|_q^{1 - \theta}$, which is a homogenous equality. The constant 2 can be removed in the equation using the {\it tensor power trick}. If we consider the function on $X^n$ defined by $f^{\otimes n}(x_1, \dots, x_n) = f(x_1) \dots f(x_n)$, then $\| f^{\otimes n} \|_r = \| f \|_r^n$, and so
    %
    \[ \| f \|_r = \| f^{\otimes n} \|_r^{1/n} \leq \left( 2 \| f^{\otimes n} \|_p^\theta \| f^{\otimes n} \|_q^{1-\theta} \right)^{1/n} = 2^{1/n} \| f \|_p^\theta \| g \|_q^{1-\theta}. \]
    %
    We can then take $n \to \infty$ to conclude that $\| f \|_r \leq \| f \|_p^\theta \| f \|_q^{1-\theta}$.
\end{remark}

The argument in the last remark is an instance of \emph{real interpolation}; In order to conclude some fact about a function which lies `between' two other functions we know how to deal with, we split the function up into two parts lying in the other spaces, deal with them separately, and then put them back together to get some equality. One can then apply various symmetry considerations (homogeneity and the tensor power trick being two examples) to eliminate extraneous constants. We now also show how to prove this inequality using convexity, which illustrates another core technique. In the next theorem, $1/\infty = 0$.

\begin{theorem}[H\"{o}lder]
  If $0 < p,q \leq \infty$ and $1/p + 1/q = 1/r$, $\| f g \|_r \leq \| f \|_p \| g \|_q$.
\end{theorem}
\begin{proof}
  The case where $p$ or $q$ is $\infty$ is left as an exercise to the reader. In the other case, by moving around exponents, we may simplify to the case where $r = 1$. The theorem depends on the log convexity inequality, such that for $A,B \geq 0$ and $0 \leq \theta \leq 1$, $A^\theta B^{1 - \theta} \leq \theta A + (1 - \theta) B$. But since the logarithm is concave, we calculate
  %
  \[ \log(A^\theta B^{1 - \theta}) = \theta \log A + (1 - \theta) \log B \leq \log(\theta A + (1 - \theta) B), \]
  %
  and we can then exponentiate. To prove H\"{o}lder's inequality, by scaling $f$ and $g$, which is fine by homogeneity, we may assume that $\| f \|_p = \| g \|_q = 1$. Then we calculate
  %
  \begin{align*}
    \| f g \|_1 &= \int |f(x)| |g(x)| = \int |f(x)|^{p/p} |g(x)|^{q/q}\\
    &\leq \int \frac{|f(x)|^p}{p} + \frac{|g(x)|^q}{q} = \frac{1}{p} + \frac{1}{q} = 1 = \| f \|_p \| g \|_q.
  \end{align*}
  %
  If $p = \infty$, $q = 1$, then the inequality is trivial, since we have the pointwise inequality $|f(x) g(x)| \leq \| f \|_\infty |g(x)|$ almost everywhere, which we can then integrate.
\end{proof}

\begin{remark}
  Note that $A^\theta B^{1-\theta} \leq \theta A + (1 - \theta) B$ is an \emph{equality} if and only if $A = B$, or $\theta \in \{ 0, 1 \}$. In particular, following through the proof above shows that if $\| f \|_p = \| g \|_q = 1$, we must have $|f(x)|^{1/p} = |g(x)|^{1/q}$ almost everywhere. In general, this means H\"{o}lder's inequality is sharp if and only if $|f(x)|^{1/p}$ is a constant multiple of $|g(x)|^{1/q}$.
\end{remark}

The next inequality is known as the \emph{triangle inequality}.

\begin{corollary} \label{lptriangleinequality}
  Given $f$,$g$, and $p \geq 1$, $\| f + g \|_p \leq \| f \|_p + \| g \|_p$.
\end{corollary}
\begin{proof}
  The inequality when $p = 1$ is obtained by integrating the inequality $|f(x) + g(x)| \leq |f(x)| + |g(x)|$, and the case $p = \infty$ is equally trivial. When $1 < p < \infty$, by scaling we can assume that $\| f \|_p + \| g \|_p = 1$. Then we can apply H\"{o}lder's inequality combined with the $p = 1$ case to conclude
  %
  \begin{align*}
    \int |f(x) + g(x)|^p &\leq \int |f(x)| |f(x) + g(x)|^{p-1} + |g(x)| |f(x) + g(x)|^{p-1}\\
    &\leq \| f \|_p \| (f + g)^{p-1} \|_q + \| g \|_p \| (f + g)^{p-1} \|_q = \| f + g \|_{p}^{p-1}
  \end{align*}
  %
  Thus $\| f + g \|_p^p \leq \| f + g \|_p^{p-1}$, and simplifying gives $\| f + g \|_p \leq 1$.
\end{proof}

\begin{remark}
  Suppose $\| f + g \|_p = \| f \| + \| g \|_p$. Following through the proof given above shows that both applications of H\"{o}lder's inequality must be sharp. And this is true if and only if $|f(x)|^p$ and $|g(x)|^p$ are scalar multiples of $|f(x) + g(x)|^p$ almost everywhere. But this means $|f(x)|$ and $|g(x)|$ are scalar multiples of $|f(x) + g(x)|$. If $|f(x)| = A|f(x) + g(x)|$ and $|g(x)| = B|f(x) + g(x)|$. If $g \neq 0$, this implies there is $C$ such that $|f(x)| = C |g(x)|$ for some $C > 0$. Thus we can write $f(x) = C e^{i \theta(x)} g(x)$, and we must have
  %
  \[ \| f + g \|_p^p = \int |1 + C e^{i \theta(x)}|^p |g(x)|^p = (1 + C)^p \int |g(x)|^p \]
  %
  so $|1 + Ce^{i \theta(x)}| = |1 + C|$ almost everywhere but this can only be true if $e^{i \theta(x)} = 1$ almost everywhere, so $f = C g$. Thus the triangle inequality is only sharp is $f$ and $g$ are positive scalar multiples of one another.
\end{remark}

This discussion leads to a useful heuristic: Unless $f$ and $g$ are `aligned' in a certain way, the triangle inequality is rarely sharp. For instance, if $f$ and $g$ have disjoint support, we calculate that
%
\[ \| f + g \|_p = \left( \| f \|_p^p + \| g \|_p^p \right)^{1/p} \]
%
For $p > 1$, this is always sharper than the triangle inequality.

If $p < 1$, then the proof of Corollary \ref{lptriangleinequality} no longer works, and in fact, is no longer true. In fact, if $f$ and $g$ are non-negative functions, then we actually have the \emph{anti} triangle inequality
%
\[ \| f + g \|_p \geq \| f \|_p + \| g \|_p, \]
%
as proved in the next theorem.

\begin{theorem}
    If $p \geq 1$, then for any functions $f_1, \dots, f_N \geq 0$,
    %
    \begin{equation} \label{triangleInequality} ( \| f_1 \|_p^p + \dots + \| f_N \|_p^p )^{1/p} \leq \| f_1 + \dots + f_N \|_p \leq \| f_1 \|_p + \dots + \| f_N \|_p. \end{equation}
    %
    If $p \leq 1$, then the inequality reverses, i.e. for any positive functions $f_1, \dots, f_N$,
    %
    \begin{equation} \label{antiTriangleInequality} \| f_1 \|_p + \dots + \| f_N \|_p \leq \| f_1 + \dots + f_N \|_p \leq (\| f_1 \|_p^p + \dots + \| f_N \|_p^p)^{1/p} \end{equation}
\end{theorem}
\begin{proof}
    The upper bound in \eqref{triangleInequality} is just obtained by applying the triangle inequality iteratively. To obtain the lower bound, we note that for $A_1, \dots, A_N \geq 0$,
    %
    \[ (A_1 + \dots + A_N)^p \geq A_1^p + \dots + A_N^p, \]
    %
    One can prove this from induction from the inequality $(A_1 + A_2)^p \geq A_1^p + A_2^p$, which holds when $A_2 = 0$, and the derivative of the left hand side is greater than the right hand side for all $A_2 \geq 0$. But then setting $A_k = f_k$ and then integrating gives
    %
    \[ \| f_1 + \dots + f_N \|_p^p \geq \| f_1 \|_p^p + \dots + \| f_N \|_p^p. \]
    %
    Now assume $0 < p < 1$. We begin by proving the lower bound in \ref{antiTriangleInequality}. We can assume $N = 2$, and $\| f_1 \|_p + \| f_2 \|_p = 1$, and then it suffices to show $\| f_1 + f_2 \|_p \geq 1$. For any $\theta \in (0,1)$, and $A,B \geq 0$, concavity implies
    %
    \[ (A + B)^p = (\theta (A/\theta) + (1 - \theta) (B/(1-\theta)))^p \geq \theta^{1-p} A^p + (1 - \theta)^{1-p} B^p. \]
    %
    Thus setting $A = f_1(x)$, $B = f_2(x)$, and $\theta = \| f_1 \|_p$, so that $1 - \theta = \| f_2 \|_p$, and then integrating, we find
    %
    \[ \| f_1 + f_2 \|_p^p \geq \theta + (1 - \theta) = 1. \]
    %
    On the other hand, the inequality $(A_1 + \dots + A_N)^p \leq A_1^p + \dots + A_N^p$, which holds for $A_1, \dots, A_N \geq 0$, can be applied with $f_k = A_k$ and integrated to yield
    %
    \[ \| f_1 + \dots + f_N \|_p^p \leq \| f_1 \|_p^p + \dots + \| f_N \|_p^p. \qedhere \]
\end{proof}

Thus the triangle inequality is not satisfied for the $L^p$ norms when $p < 1$. This is one of the deficiencies which leads the $L^p$ theories for $0 < p < 1$ to be rather deficient when compared to the case with $p \geq 1$. One way to fix this is to use the theory of Hardy spaces. We note that for $p < 1$, we do have a \emph{quasi} triangle inequality.

\begin{theorem} \label{quasitriangleinequalitylp}
    For $f_1, \dots, f_N \in L^p(X)$, with $0 < p < 1$,
    %
    \[ \| f_1 + \dots + f_N \|_p \leq N^{1/p - 1} (\| f_1 \|_p + \dots + \| f_N \|_p). \]
\end{theorem}
\begin{proof}
    By H\"{o}lder's inequality applied to sums,
    %
    \[ \| f_1 + \dots + f_N \|_p \leq (\| f \|_p^p + \dots + \| f_N \|_p^p)^{1/p} \leq N^{1/p - 1} (\| f_1 \|_p + \dots + \| f_N \|_p). \qedhere \]
\end{proof}

This result is sharp, i.e. if we take a disjoint family of sets $\{ E_1, E_2, \dots \}$ with $|E_i| = 1$ for each $i$, and then set $f_i = \mathbf{I}_{E_i}$, then the inequality is sharp for each $N$.

\begin{remark}
    When $p < 1$, the space $L^p(X)$ is \emph{not} normable. To see why, we look at the topological features of $L^p(X)$. Fix $\varepsilon > 0$, and let $C$ be a convex set containing all functions $f$ with $\| f \|_p < \varepsilon$. Thus, in particular, $C$ contains all step functions $H \mathbf{I}_E$ where $H |E|^{1/p} < \varepsilon$. But if we now find a countable sequence of disjoint sets $\{ E_k \}$, each with positive measure, and for each $k$, define $H_k = (\varepsilon/2) |E_k|^{-1/p}$, then for any $N$, the function
    %
    \[ f_N = (H_1/N) \mathbf{I}_{E_1} + \dots + (H_N/N) \mathbf{I}_{E_N} \]
    %
    lies in $C$, and
    %
    \[ \| f_N \|_p = (1/N) (H_1^p |E_1| + \dots + H_N^p |E_N|)^{1/p} = (\varepsilon/2) N^{1/p - 1} \]
    %
    as $N \to \infty$, the $L^p$ norm of $f_N$ becomes unbounded. In particular, this means that we have proven that every bounded convex subset of $L^p(X)$ has empty interior, and a norm space certainly does not have this property.
\end{remark}

As we have mentioned, as $p \to \infty$, the $L^p$ norm excludes functions with large peaks, or large height, and as $p \to 0$, the $L^p$ norm excludes functions with large tails, or large width. They form a continuously changing family of functions as $p$ ranges over the positive numbers. In general, there is no inclusion of $L^p(X)$ in $L^q(X)$ for any $p,q$, except in two circumstances which occur often enough to be mentioned.

\begin{example}
  If $X$ is a finite measure space, and $0 < p \leq q \leq \infty$, $L^p(X) \subset L^q(X)$. H\"{o}lder's inequality implies $\| f \|_p = \| f \chi_X \|_p \leq \| f \|_q |X|^{1/p-1/q}$. Taking $q \to \infty$, we conclude $\| f \|_p \leq | X |^{1/p} \| f \|_\infty$. One can best remember the constants here by the formula
  %
  \[ \left( \fint |f(x)|^p \right)^{1/p} \leq \left( \fint |f(x)|^q \right)^{1/q}. \]
  %
  In particular, when $X$ is a probability space, the $L^p$ norms are increasing.
\end{example}

\begin{example}
  On the other hand, suppose the measure space is {\it granular}, in the sense that there is $\varepsilon > 0$ such that either $|E| = 0$ or $|E| \geq \varepsilon$ for any measurable set $E$. Then $L^q(X) \subset L^p(X)$ for $0 < p \leq q \leq \infty$. First we check the $q = \infty$ case, which follows by the trivial estimate
  %
  \[ \int |f(x)|^p \geq \varepsilon \| f \|_\infty, \]
  %
  so $\| f \|_\infty \leq \| f \|_p \varepsilon^{-1/p}$. But then applying log convexity, if $p \leq q < \infty$, we can write $1/q = \theta/p$ for $0 < \theta \leq 1$, and then log convexity shows
  %
  \[ \| f \|_q = \| f \|_p^\theta \| f \|_\infty^{1-\theta} \leq \varepsilon^{-(1 - \theta)/p} \| f \|_p = \varepsilon^{-1/p - 1/q} \| f \|_p. \]
  %
  If $\varepsilon = 1$, which occurs if $X = \ZZ$, then the $L^p$ norms are decreasing in $p$. This gives the best way to remember the constants involved, since the measure $\mu(E) = |E|/\varepsilon$ is one granular, and so
  %
  \[ \left( \frac{1}{\varepsilon} \int |f(x)|^q\; dx \right)^{1/q} \leq \left( \frac{1}{\varepsilon} \int |f(x)|^p\; dx \right)^{1/p}. \]
\end{example}

%\begin{example}
%  Controlling additional properties of the function offers similar properties as for control on the measure space. If $|f(x)| \leq M$ for almost all $x$, then for $p \leq q$,
  %
%  \[ \| f \|_q \leq \| f \|_p^{p/q} M^{1 - p/q}. \]
  %
%  Conversely, if $|f(x)| \geq M$ whenever $f(x) \neq 0$, then
  %
%  \[ \| f \|_p \leq \| f \|_q^{q/p} M^{1-q/p}. \]
  %

%\end{example}

\begin{remark}
  We can often use such results in spaces which are not granular by coarsening the sigma algebra. For instance, the Lebesgue measure is $\varepsilon^d$ granular over the sigma algebra generated by the length $\varepsilon$ cubes whose corner's lie on the lattice $(\ZZ/\varepsilon)^d$, and if a function is measurable with respect to such a $\sigma$ algebra we call the function $\varepsilon$ granular.
\end{remark}

\begin{remark}
  If we let $X = \{ 1, \dots, N \}$, then $X$ is both finite and granular, so all $L^p$ norms are comparable. In particular, if $p \leq q$,
  %
  \[ \| f \|_q \leq \| f \|_p \leq N^{1/p - 1/q} \| f \|_q. \]
  %
  The left hand side of this inequality becomes sharp when $f$ is concentrated at a single point, i.e. $f(n) = \mathbf{I}(n = 1)$. On the other hand, the left hand side becomes sharp when $f$ is constant, i.e. $f(n) = 1$ for all $n$.
\end{remark}

\begin{example}
    We can obtain similar $L^p$ bounds by controlling the functions $f$ involved, rather than the measure space. For instance, if $|f(x)| \leq M$, and $p \leq q$, then then $\| f \|_q \leq \| f \|_p^{p/q} M^{1 - p/q}$, which follows by log convexity. On the other hand, if $|f(x)| \geq M$ on the support of $f$, then $\| f \|_p \leq \| f \|_q^{q/p} M^{1-q/p}$.
\end{example}

\begin{theorem}
  If $p_\theta$ lies between $p_0$ and $p_1$, then
  %
  \[ L^{p_0}(X) \cap L^{p_1}(X) \subset L^{p_\theta}(X) \subset L^{p_0}(X) + L^{p_1}(X) \]
\end{theorem}
\begin{proof}
  If $\| f \|_{p_0}, \| f \|_{p_1} < \infty$, then for any $p_\theta$ between $p_0$ and $p_1$,
  %
  \[ \| f \chi_{|f| \leq 1} \|_{p_\theta}^{p_\theta} = \int_{|f| \leq 1} |f|^{p_\theta} \leq \int_{|f| \leq 1} |f|^{p_0} < \infty \]
  \[ \| f \chi_{|f| > 1} \|_{p_\theta}^{p_\theta} = \int_{|f| > 1} |f|^{p_\theta} \leq \int_{|f| > 1} |f|^{p_1} < \infty \]
  %
  Applying the triangle inequality, we conclude that $\| f \|_{p_\theta} < \infty$. In the case where $p_1 = \infty$, then $f \chi_{|f| > 1}$ is bounded, and must have finite support if $p_0 < \infty$, which shows this integral is bounded. Note the inequalities above show that we can split any function with finite $L^{p_\theta}$ norm into the sum of a function with finite $L^{p_0}$ norm and another with finite $L^{p_1}$ norm.
\end{proof}

\begin{remark}
  This theorem is important in the study of interpolation theory, because if we have two linear operators $T_{p_0}$ defined on $L^{p_0}(X)$ and $T_{p_1}$ on $L^{p_1}(X)$, and they agree on $L^{p_0}(X) \cap L^{p_1}(X)$, then there is a unique linear operator $T_{p_\theta}$ on $L^{p_\theta}(X)$ which agrees with these two functions, and we can consider the boundedness of such a function with respect to the $L^{p_\theta}$ norms.
\end{remark}

The last property of the $L^p$ norms we want to focus on is the principle of \emph{duality}. Given any values of $p$ and $q$ with $1/p + 1/q = 1$, H\"{o}lder's inequality implies that if $f \in L^p(X)$ and $g \in L^q(X)$, then $fg \in L^1(X)$. In particular, for each function $g \in L^q(X)$, the map
%
\[ \lambda: f \mapsto \int f(x)g(x)\; dx \]
%
is a linear functional on $L^p(X)$. H\"{o}lder's inequality implies that $\| \lambda \| \leq \| g \|_q$. But this is actually an \emph{equality}. In particular, if $1 < p < \infty$, one can show these are \emph{all} linear functionals. For $p \in \{ 1, \infty \}$, the dual space of $L^p(X)$ is more subtle. But, since in harmonic analysis we concentrate on quantitative bounds, the following theorem often suffices as a replacement.

\begin{theorem}
    If $1 \leq p < \infty$, and $f \in L^p(X)$, then
    %
    \[ \| f \|_p = \sup \left\{ \int f(x)g(x) : \| g \|_q = 1 \right\}. \]
    %
    If the underlying measure space is $\sigma$ finite, then this claim also holds for $p = \infty$.
\end{theorem}
\begin{proof}
    Suppose that $1 \leq p < \infty$. Given $f$, we define
    %
    \[ g(x) = \frac{1}{\| f \|_p^{p-1}} \text{sgn}(f(x)) |f(x)|^{p-1}. \]
    %
    If $\| f \|_p < \infty$, then
    %
    \[ \| g \|_q^q = \frac{1}{\| f \|_p^{pq - q}} \int |f(x)|^{pq-q} = \frac{1}{\| f \|_p^p} \| f \|_p^p = 1, \]
    %
    and
    %
    \[ \int f(x) g(x) = \frac{1}{\| f \|_p^{p-1}} \int |f(x)|^p = \| f \|_p. \]
    %
    On the other hand, suppose $\| f \|_p = \infty$. Then there exists a sequence of step functions $s_1 \leq s_2 \leq \dots \to |f|$. Each $s_k$ lies in $L^p(X)$, but the monotone convergence theorem implies that $\| s_k \|_p \to \infty$. For each $k$, find a function $g_k \geq 0$ with $\| g_k \|_q = 1$, and $\int g_k(x) s_k(x) \geq \| s_k \|_p / 2$. Then
    %
    \[ \int g_k(x) \text{sgn}(f(x)) f(x) = \int g_k(x) |f(x)| \geq \int g_k(x) s_k(x) \geq \| s_k \|_p / 2 \to \infty, \]
    %
    this completes the proof in this case.

    Now we take the case $p = \infty$. Given any $f$, fix $\varepsilon > 0$. Then we can find a set $E$ with $0 < |E| < \infty$ such that $|f(x)| \geq \| f \|_\infty - \varepsilon$ for $x \in E$. If $g(x) = \text{sgn}(f(x)) \mathbf{I}_E / |E|$, then $\| g \|_1 = 1$, and
    %
    \[ \int f(x) g(x) = \frac{1}{|E|} \int_E |f(x)| \geq \| f \|_\infty - \varepsilon. \]
    %
    Taking $\varepsilon \to 0$ completes the claim.
\end{proof}

\section{Decreasing Rearrangements}

 The properties of a functions distribution are best reflected quite simply in the \emph{distribution function} of the function $f$, i.e. the function $F: [0,\infty) \to [0,\infty)$ given by $F(t) = |\{ x : |f(x)| > t \}|$, and any rearrangement invariant norm on $f$ should be a function of $F$. The function $F$ is right-continuous and decreasing, but has a jump discontinuity whenever $\{ x : |f(x)| = t \}$ is a set of positive measure. We denote distributions of functions $g$ and $h$ by $G$ and $H$.

\begin{lemma}
  Given a function $f$ and $g$, $\alpha \in \mathbf{C}$, and $t,s > 0$, then
  %
  \begin{itemize}
    \item If $|g| \leq |f|$, then $G \leq F$.
    \item If $g = \alpha f$, then $G(t) = F(t/|\alpha|)$.
    \item If $h = f + g$, then $H(t+s) \leq F(t) + G(s)$.
    \item If $h = fg$, then $H(ts) \leq F(t) + G(s)$.
  \end{itemize}
\end{lemma}
\begin{proof}
    The first point follows because $\{ x : |g(x)| > t \} \subset \{ x : |f(x)| > t \}$, and the second because $\{ x : |\alpha f(x)| > t \} = \{ x : |f(x)| > t/|\alpha| \}$. The third point follows because if $|f(x) + g(x)| \geq t + s$, then either $|f(x)| \geq t$ or $|g(x)| \geq s$. Finally, if $|f(x) g(x)| \geq ts$, then $|f(x)| \geq t$ or $|g(x)| \geq s$.
\end{proof}

We can simplify the study of the distribution of $f$ even more by defining the \emph{decreasing rearrangement} of $f$, a decreasing function $f^*: [0,\infty) \to [0,\infty)$ such that $f^*(s)$ is the \emph{smallest} number $t$ such that $F(t) \leq s$. Effectively, $f^*(s)$ is the inverse of $F$:
%
\begin{itemize}
    \item If there is a unique $t$ with $F(t) = s$, then $f^*(s) = t$.
    \item If there are multiple values $t$ with $F(t) = s$, let $f^*(s)$ be the \emph{smallest} such value.
    \item If there are no values $t$ with $F(t) = s$, then we pick the first value $t$ with $F(t) < s$.
\end{itemize}
%
We find
%
\[ \{ s : f^*(s) > t \} = \{ s : s < F(t) \} = [0,F(t)), \]
%
which has measure $F(t)$. This is the most important property of $f^*$; it is a decreasing function on the line which has the same distribution as the function $|f|$. It is also the unique such function which is right continuous. Thus our intuition when analyzing monotone, rearrangement invariant norms is not harmed if we focus on right continuous decreasing functions.

\begin{theorem}
    The function $f^*$ is right continuous.
\end{theorem}
\begin{proof}
    We note that $F(t) > s$ if and only if $t < f^*(s)$. Since $f^*$ is decreasing, for any $s \geq 0$, we automatically have $f^*(s^+) \leq f^*(s)$. If $f^*(s^+) < f^*(s)$, then
    %
    \[ s < F \left( f^*(s^+) \right) \leq F(f^*(s)) \leq s, \]
    %
    which gives a contradiction, so $f^*(s) = f^*(s^+)$.
\end{proof}

\begin{remark}
    We have a jump discontinuity at a point $s$ wherever $F$ is flat, and $f^*$ is flat wherever $F$ has a jump discontinuity.
\end{remark}

In particular, when understanding intuition about monotone rearrangement invariant norms, one is allowed to focus on non-increasing, right continuous functions on $(0,\infty)$. For instance, this means that these norms do not care about the number of singularities that a function has, since all these singularities `pile up' in the decreasing rearrangement.

\section{Weak Norms}

The weak $L^p$ norms are obtained as a slight `refinement' of the $L^p$ norms.

\begin{theorem}
  If $\phi$ is an increasing, differentiable function on the real line with $\phi(0) = 0$, then
  %
  \[ \int_X \phi(|f(x)|) = \int_0^\infty \phi'(t) F(t)\; dt \]
\end{theorem}
\begin{proof}
  An application of Fubini's theorem is all that is needed to show
  %
  \begin{align*}
    \int_X \phi(|f(x)|)\; dx &= \int_X \int_0^{|f(x)|} \phi'(t)\; dt\; dx\\
    &= \int_0^\infty \phi'(t) \int_{|f(x)| > t}\; dx\; du\\
    &= \int_0^\infty \phi'(t) F(t)\; dt. \qedhere
  \end{align*}
\end{proof}

As a special case we find
%
\[ \| f \|_p = \left( p \int_0^\infty F(t) t^p \frac{dt}{t} \right)^{1/p}. \]
%
For this to be true, $F(t)$ must tend to zero `logarithmically faster' than $1/t^p$. Indeed, we find
%
\[ F(t) = |\{ |f|^p > t^p \}| \leq \frac{1}{t^p} \int |f|^p = \frac{\| f \|_p^p}{t^p}, \]
%
a fact known as \emph{Chebyshev's inequality}. But a bound $F(t) \lesssim 1/t^p$ might be true even if $f \not \in L^p(\RR^d)$. This leads to the \emph{weak $L^p$ norm}, denoted by $\| f \|_{p,\infty}$, which is defined to be the smallest value $A$ such that $F(t) \leq (A/t)^p$ for all $t$. We let $L^{p,\infty}(X)$ denote the space of all functions $f$ for which $\| f \|_{p,\infty} < \infty$. By Chebyshev's inequality, $\| f \|_{p,\infty} \leq \| f \|_p$ for any function $f$. The reason that the value $A$ occurs within the brackets is so that the norm is homogenous; if $g = \alpha f$, and $\| f \|_{p,\infty} = A$, then
%
\[ G(t) = F(t/|\alpha|) \leq \left( \frac{A |\alpha|}{t} \right)^p, \]
%
so $\| \alpha f \|_{p,\infty} = |\alpha| \| f \|_p$. The weak norms do not satisfy a triangle inequality, but they do satisfy a quasitriangle inequality. This can be proven quite simply from the property that if $f = f_1 + \dots + f_N$, and $\alpha_1, \dots, \alpha_N \in [0,1]$ satisfy $\alpha_1 + \dots + \alpha_N = 1$, then
%
\[ F(t) = F_1(\alpha_1 t) + \dots + F_N(\alpha_N t). \]
%
Thus if $f = g + h$, then
%
\[ F(t) \leq G(t/2) + H(t/2) \leq \frac{\| g \|_{p,\infty}^p + \| h \|_{p,\infty}^p}{t^p} \lesssim_p \left( \frac{\| g \|_{p,\infty} + \| h \|_{p,\infty}}{t} \right)^p. \]
%
Thus $\| f + g \|_{p,\infty} \lesssim \| f \|_{p,\infty} + \| g \|_{p,\infty}$. We can measure the degree to which the weak $L^p$ norm fails to be a norm by determining how much the triangle inequality fails for the sum of $N$ functions, instead of just one function.

\begin{theorem}[Stein-Weiss Inequality]
  Let $f_1, \dots, f_N$ be functions. If $p > 1$, then
  %
  \[ \| f_1 + \dots + f_N \|_{p,\infty} \lesssim_p \| f_1 \|_{p,\infty} + \dots + \| f_N \|_{p,\infty}. \]
  %
  If $p = 1$, then
  %
  \[ \| f_1 + \dots + f_N \|_{1,\infty} \lesssim \log N \left[ \| f_1 \|_{1,\infty} + \dots + \| f_N \|_{1,\infty} \right]. \]
  %
  If $0 < p < 1$, then
  %
  \[ \| f_1 + \dots + f_N \|_{p,\infty} \lesssim_p \left( \| f_1 \|_{p,\infty}^p + \dots + \| f_N \|_{p,\infty}^{1/p} \right)^{1/p} \]
\end{theorem}
\begin{proof}
    Begin with the case $p \geq 1$. Without loss of generality, assume $\| f_1 \|_{p,\infty} + \dots + \| f_N \|_{p,\infty} = 1$. Fix $t > 0$. For each $k \in [1,N]$, define
    %
    \[ g_k(x) = \begin{cases} f_k(x) &: |f_k(x)| \geq t/2, \\ 0 &: \text{otherwise}, \end{cases} \]
    %
    and
    %
    \[ h_k(x) = \begin{cases} f_k(x) &: |f_k(x)| \leq \| f_k \|_{p,\infty} \cdot (t/2), \\ 0 &: \text{otherwise}. \end{cases} \]
    %
    Also define $j_k = f_k - g_k - h_k$. Then write $f = f_1 + \dots + f_N$, $g = g_1 + \dots + g_N$, $h = h_1 + \dots + h_N$, and $j = j_1 + \dots + j_N$. Note that $\| h \|_\infty \leq t/2$, so
    %
    \[ \{ x : |f(x)| \geq t \} \subset \{ x : |g(x)| \geq t/4 \} \cup \{ x : |j(x)| \geq t/4 \}. \]
    %
    Each $g_k$ is supported on a set of measure at most $\| f_k \|_{p,\infty}^p \cdot (2/t)^p$. We conclude that $g$ is supported on a set of measure at most
    %
    \[ (2/t)^p \sum_{k = 1}^N \| f_k \|_{p,\infty}^p \leq (2/t)^p. \]
    %
    If $p > 1$, then the measure of $\{ x : |j(x)| \geq t/4 \}$ is bounded by
    %
    \begin{align*}
        \frac{4}{t} \int |j(x)|\; dx &\leq \frac{4}{t} \sum_{k = 1}^N \int |j_k(x)|\\
        &= \frac{4}{t} \sum_{k = 1}^N \int_{\| f_k \|_{p,\infty} (t/2)}^{t/2} \frac{\| j_k \|_{p,\infty}^p}{s^p}\; ds\\
        &= \frac{2^{p+1}}{p-1} \frac{1}{t^p} \sum_{k = 1}^N \| j_k \|_{p,\infty}^p \left( \frac{1}{\| f_k \|_{p,\infty}^{p-1}} - 1 \right) \\
        &\leq \frac{2^{p+1}}{p-1} \frac{1}{t^p} \sum_{k = 1}^N \| f_k \|_{p,\infty}^p \left( \frac{1}{\| f_k \|_{p,\infty}^{p-1 }} - 1 \right)\\
        &\leq \frac{2^{p+1}}{p-1} \frac{1}{t^p}.
    \end{align*}
    %
    Thus in total, we conclude the measure of $\{ x: |f(x)| \geq t \}$ is at most
    %
    \[ \frac{2^p}{t^p} + \frac{2^{p+1}}{p - 1} \frac{1}{t^p} \lesssim_p \frac{1}{t^p}. \]
    %
    If $p = 1$, then the measure of $\{ x : |j(x)| \geq t/4 \}$ is bounded
    %
    \begin{align*}
        (4/t) \int |j(x)|\; dx &\leq (4/t) \sum_{k = 1}^N \int |j_k(x)|\\
        &= (4/t) \sum_{k = 1}^N \int_{\| f_k \|_{1,\infty} (t/2)}^{t/2} \frac{\| j_k \|_{1,\infty}}{s}\; ds\\
        &= (4/t) \sum_{k = 1}^N \| f_k \|_{1,\infty} \log(1/\| f_k \|_{1,\infty}).
    \end{align*}
    %
    Now the maximum of $x_1 \log(1/x_1) + \dots + x_N \log(1/x_N)$, subject to the constraint that $x_1 + \dots + x_N = 1$, is maximized by taking $x_k = 1/N$ for all $N$, which gives a maximal bound of $\log(N)$. In particular, we find that
    %
    \[ (2/t) \sum_{k = 1}^N \| f_k \|_{1,\infty} \log(1/\| f_k \|_{1,\infty}) \leq (2 \log N)/t. \]
    %
    Thus in total, we conclude the measure of $\{ x: |f(x)| \geq t \}$ is at most
    %
    \[ 2(1 + \log N)/t \lesssim \log N / t. \]
    %
    If $p < 1$, we may assume without loss of generality that
    %
    \[ \| f_1 \|_{p,\infty}^p + \dots + \| f_N \|_{p,\infty}^p = 1. \]
    %
    Then, we perform the same decomposition as before, with functions $\{ g_k \}$, $\{ h_k \}$, and $\{ j_k \}$, defined the same as before, except that
    %
    \[ h_k(x) = \begin{cases} f_k(x) &: |f_k(x)| \leq \| f_k \|_{p,\infty}^p \cdot (t/2), \\ 0 &: \text{otherwise}. \end{cases} \]
    %
    The function $g_k$ has support at most $\| f_k \|_{p,\infty}^p \cdot (2/t)^p$, and thus $g$ has total support
    %
    \[ \sum \| f_k \|_{p,\infty}^p (2/t)^p = (2/t)^p. \]
    %
    The measure of $\{ x : |j(x)| \geq t/4 \}$ is bounded by
    %
    \begin{align*}
      \frac{4}{t} \int |j(x)|\; dx &\leq \frac{4}{t} \sum_{k = 1}^N \int_{\| f_k \|_{p,\infty}^p (t/2)}^{t/2} \frac{\| f_k \|_{p,\infty}^p}{s^p}\; ds\\
      &\leq \frac{2^{p+1}}{t^p} \frac{1}{1 - p} \sum_{k = 1}^N \| f_k \|_{p,\infty}^{p + p(1-p)}\\
      &= \frac{2^{p+1}}{t^p} \frac{1}{1 - p} \max \| f_k \|_{p,\infty}^{p(1-p)} \lesssim_p \frac{1}{t^p},
    \end{align*}
    %
    Combining the two bounds gives that $\| f_1 + \dots + f_N \|_{p,\infty} \lesssim_p 1$.
\end{proof}

\begin{remark}
  For $p = 1$, compare this \emph{logarithmic} failure to be a norm with the \emph{polynomial} failure to be a norm found in the norms $\| \cdot \|_p$, when $p < 1$, in Theorem \ref{quasitriangleinequalitylp}.
\end{remark}

For $p = 1$, the Stein-Weiss inequality is asymptotically tight in $N$.

\begin{example}
  Let $X = \RR$. For each $k$, let
  %
  \[ f_k(x) = \frac{1}{|x - k|}. \]
  %
  Then $\| f_k \|_{1,\infty} \lesssim 1$ is bounded independantly of $k$. If $|x| \leq N$, there are integers $k_1, \dots, k_N > 0$ such that $|x - k_i| \leq 2i$, so
  %
  \[ f(x) \geq \sum_{i = 1}^N \frac{1}{|x - k_i|} \geq \sum_{i = 1}^N \frac{1}{2i} \gtrsim \log(N). \]
  %
  Thus $\| f \|_{1,\infty} \gtrsim N \log N \gtrsim \log N \sum \| f_k \|_{1,\infty}$.
\end{example}

The weak $L^p$ norms provide another monotone translation invariant norm, and it oftens comes up when finer tuning is needed in certain interpolation arguments, especially when dealing with maximal functions.

\begin{example}
  If $f = H \mathbf{I}_E$, with $|E| = W$, then
  %
  \[ F(t) = W \cdot \mathbf{I}_{[0,H)}. \]
  %
  Thus
  %
  \[ \| f \|_{p,\infty} = \left( \sup_{0 \leq t < H} W t^p \right)^{1/p} = W^{1/p} H^p = \| f \|_p. \]
  %
  If $f = H_1 \mathbf{I}_{E_1} + H_2 \mathbf{I}_{E_2}$, with $|E_1| = W_1$ and $|E_2| = W_2$, with $H_1 \leq H_2$, then
  %
  \[ F(t) = \begin{cases} W_1 + W_2 &: t < H_1, \\ W_2 &: t < H_2, \\ 0 &: \text{otherwise.} \end{cases} \]
  %
  Thus
  %
  \[ \| f \|_{p,\infty} = \left( \max((W_1 + W_2) H_1^p, W_2 H_2^p) \right)^{1/p} = \max((W_1 + W_2)^{1/p} H_1, W_2^{1/p} H_2). \]
\end{example}

\begin{example}
    The function $f(x) = 1/|x|^s$ does not lie in any $L^p(\RR^d)$, but lies in $L^{p,\infty}$ precisely when $p = d/s$, since
    %
    \[ \left| \{ 1/|x|^{ps} > t \} \right| = \left| \left\{ |x| \leq \frac{1}{t^{1/ps}} \right\} \right|\ \propto_d\ \frac{1}{t^{d/ps}}. \]
\end{example}

Before we move on, we consider a form of duality for the weak norm, at least when $p > 1$.

\begin{theorem}
	If $p > 1$, and $X$ is $\sigma$ finite, then
	%
	\[ \| f \|_{p,\infty} \sim_p \sup_{|E| < \infty} \frac{1}{|E|^{1-1/p}} \int_E |f(x)|\; dx \]
\end{theorem}
\begin{proof}
	Suppose $\| f \|_{p,\infty} < \infty$. If we write $f = \sum f_k$, where $f_k = \mathbf{I}_{F_k} f$, and $F_k = \{ x: 2^{k-1} < |f(x)| \leq 2^k \}$, then $|F_k| \leq \| f \|_{p,\infty}^p 2^{-kp}$. Thus
	%
	\[ \left| \int_E |f_k(x)| \right| \leq 2^k \| f \|_{p,\infty}^p 2^{-kp} = \| f \|_{p,\infty}^p 2^{k(1-p)}. \]
	%
	Fix some integer $n$. Then
	%
	\begin{align*}
		\int_E |f(x)|\; dx &\leq \sum_{k = -\infty}^{n-1} \int_E |f_k(x)|\; dx + \sum_{k = n}^\infty \int_E |f_k(x)|\; dx\\
		&\leq |E| 2^{n-1} + \| f \|_{p,\infty}^p \sum_{k = n}^\infty 2^{k(1-p)}\\
		&\lesssim_p |E| 2^n + \| f \|_{p,\infty}^p 2^{-k(1-p)}.
	\end{align*}
	%
	If we let $2^n \sim \| f \|_{p,\infty} |E|^{1/p}$, then we conclude
	%
	\[ \int_E |f(x)|\; dx \lesssim_p |E|^{1 - 1/p} \| f \|_{p,\infty}. \]
	%
	Conversely, write
	%
	\[ A = \sup_{|E| < \infty} \frac{1}{|E|^{1-1/p}} \int_E |f(x)|\; dx/ \]

	%
	If $G_t = \{ x: |f(x)| \geq t \}$, then
	%
	\[ |G_t| \leq \frac{1}{t} \int_{G_t} |f(x)|\; dx \leq \frac{A |G_t|^{1 - 1/p}}{t}, \]
	%
	so
	%
	\[ |G_t| \leq \frac{A^p}{t}, \]
	%
	which gives $\| f \|_{p,\infty} \leq A$.
\end{proof}

For $p \leq 1$, the spaces $L^{p,\infty}(X)$ are not normable, as seen by the tightness of the Stein-Weiss inequality. Nonetheless, we still have a certain `duality' property, that is often useful in the analysis of operators on these spaces. Most useful is it's application when $p = 1$.

\begin{theorem} \label{weakdualitytheorem}
  Let $0 < p < \infty$, and let $f \in L^{p,\infty}(X)$, and let $\alpha \in (0,1)$. Then the following are equivalent:
  %
  \begin{itemize}
    \item $\| f \|_{p,\infty} \lesssim_{\alpha,p} A$.

    \item For any set $E \subset X$ with finite measure, there is $E' \subset E$ with $|E'| \geq \alpha |E|$ such that
    %
    \[ \int_{E'} |f(x)|\; dx \lesssim_{\alpha,p} A |E'|^{1 - 1/p}. \]
  \end{itemize}
\end{theorem}
\begin{proof}
  By homogeneity, assume $\| f \|_{p,\infty} \leq 1$, so that if $F$ is the distribution of $f$, $F(t) \leq 1/t^p$. If $|E| = (1-\alpha)^{-1} / t_0^p$, and we set
  %
  \[ E' = \{ x: |f(x)| \leq t_0 \}, \]
  %
  then
  %
  \[ |E'| \geq |E| - F(t_0) = \frac{(1 - \alpha)^{-1} - 1}{t_0^p} = \alpha |E|, \]
  %
  and
  %
  \[ \int_{E'} |f(x)| \leq t_0 |E'| \lesssim_\alpha |E'|^{1-1/p}. \]
  %
  Conversely, suppose Property (2) holds. For each $k$, set
  %
  \[ E_k = \{ x: 2^k \leq |f(x)| < 2^{k+1} \}. \]
  %
  Then there exists $E_k'$ with $|E_k'| \geq \alpha |E_k|$ and
  %
  \[ \int_{E_k'} |f(x)|\; dx \leq |E_k'|^{1 - 1/p} \]
  %
  On the other hand,
  %
  \[ \int_{E_k'} |f(x)|\; dx \geq 2^k |E_k'|. \]
  %
  Rearranging this equation gives $|E_k'| \leq 2^{-pk}$, and so $|E_k| \lesssim_\alpha 2^{-pk}$. But this means
  %
  \[ F(2^N) = \sum_{k = N}^\infty |E_k| \lesssim_{\alpha,p} 2^{-Np}, \]
  %
  and this implies $\| f \|_{p,\infty} \lesssim_{\alpha,p} 1$.
\end{proof}

\section{Lorentz Spaces}

Recall that we can write
%
\[ \| f \|_p = \left( p \int_0^\infty F(t) t^p \frac{dt}{t} \right)^{1/p}. \]
%
Thus $F(t) t^p$ is integrable with respect to the Haar measure on $\RR^+$. But if we change the integrality condition to the condition that $F(t) t^p \in L^q(\RR^+)$ for some $0 < q \leq \infty$, we obtain a different integrability condition, giving rise to a monotone, translation-invariant norm. Thus leads us to the definition of the \emph{Lorentz norms}. For each $0 < p,q < \infty$, we define the Lorentz norm
%
\[ \| f \|_{p,q} = p^{1/q} \| t F^{1/p} \|_{L^q(\RR^+)} \]
%
The \emph{Lorentz space} $L^{p,q}(X)$ as the space of functions $f$ with $\| f \|_{p,q} < \infty$. We can define the norm in terms of $f^*$ as well.

\begin{lemma}
  For any measurable $f: X \to \RR$, $\| f(t) \|_{p,q} = \| s^{1/p} f^*(s) \|_{L^q(\RR^+)}$.
\end{lemma}
\begin{proof}
  First, assume $f^*$ has non-vanishing derivative on $(0,\infty)$, and that $f$ is bounded, with finite support. An integration by parts gives
  %
  \[ \| f \|_{p,q} = p^{1/q} \left( \int_0^\infty t^{q-1} F(t)^{q/p}\; dt \right)^{1/q} = \left( \int_0^\infty t^q F(t)^{q/p - 1} (-F'(t))\; dt \right)^{1/q}. \]
  %
  If we set $s = F(t)$, then $f^*(s) = t$, and $ds = F'(t) dt$, and so
  %
  \[ \left( \int_0^\infty t^q F(t)^{q/p - 1} F'(t)\; dt \right)^{1/q} = \left( \int_0^\infty f^*(s)^q s^{q/p - 1} ds \right)^{1/q} = \| s^{1/p} f^* \|_{L^q(\RR^+)}. \]
  %
  This gives the result in this case. The general result can then be obtained by applying the monotone convergence theorem to an arbitrary $f^*$ with respect to a family of smooth functions.
\end{proof}

The definition of the Lorentz space may seem confusing, but we really only require various special cases in most applications. Aside from the weak $L^p$ norms $\| \cdot \|_{p,\infty}$ and the $L^p$ norms $\| \cdot \|_p = \| \cdot \|_{p,p}$, the $L^{p,1}$ norms and $L^{p,2}$ norms also occur, the first, because of the connection with integrability, and the second because we may apply orthogonality techniques. As $q \to 0$, the norms $\| \cdot \|_{p,q}$ give stronger control over the function $f$.

\begin{theorem}
    For $q < r$, $\| f \|_{p,r} \lesssim_{p,q,r} \| f \|_{p,q}$.
\end{theorem}
\begin{proof}
    First we treat the case $r = \infty$. We have
    %
    \begin{align*}
        s_0^{1/p} f^*(s_0) &= \left( (p/q) \int_0^{s_0} [s^{1/p} f^*(s_0)]^q \frac{ds}{s} \right)^{1/q}\\
        &\leq \left( (p/q) \int_0^{s_0} [s^{1/p} f^*(s)]^q \frac{ds}{s} \right)\\
        &\leq (p/q)^{1/q} \| f \|_{p,q}.
    \end{align*}
    %
    When $r < \infty$, we can interpolate, calculating
    %
    \begin{align*}
      \| f \|_{p,r} &= \left( \int_0^\infty [s^{1/p} f^*(s)]^r \frac{ds}{s} \right)^{1/r}\\
    &\leq \| f \|_{p,\infty}^{1 - q/r} \| f \|_{p,q}^{q/r} \leq (p/q)^{p(1/q - 1/r)} \| f \|_{p,q}. \qedhere
    \end{align*}
\end{proof}

The fact that multiplying a function by a constant dilates the distribution implies that the Lorentz norm is homogeneous. We do not have a triangle inequality for the Lorentz norms, but we have a quasi triangle inequality.

\begin{theorem}
	For each $p,q > 0$, $\| f_1 + f_2 \|_{p,q} \lesssim_{p,q} \| f_1 \|_p + \| f_2 \|_q$.
\end{theorem}
\begin{proof}
    We calculate that if $g = f_1 + f_2$,
    %
    \begin{align*}
    \| g \|_{p,q} &= \left( q \int_0^\infty \left[t G(t)^{1/p} \right]^q \frac{dt}{t} \right)^{1/q}\\
    &\leq \left( q \int_0^\infty \left[ t (F_1(t/2) + F_2(t/2))^{1/p} \right]^q \frac{dt}{t} \right)^{1/q}\\
    &\lesssim \left( q \int_0^\infty \left[ t \left( F_1(t) + F_2(t) \right)^{1/p} \right]^q \frac{dt}{t} \right)^{1/q}\\
    &\lesssim_p \left( q \int_0^\infty t^q \left( F_1(t)^{q/p} + F_2(t)^{q/p} \right) \frac{dt}{t} \right)^{1/q}\\
    &\lesssim_q  \left( q \int_0^\infty t^q F_1(t)^{q/p} \frac{dt}{t} \right)^{1/q} +  \left( q \int_0^\infty t^q F_2(t)^{q/p} \frac{dt}{t} \right)^{1/q}\\
    &= \| f_1 \|_{p,q} + \| f_2 \|_{p,q}. \qedhere
  \end{align*}
\end{proof}

\section{Dyadic Layer Cake Decompositions}

An important trick to utilizing Lorentz norms is by utilizing a dyadic layer cake decomposition. The dyadic layer cake decompositions enable us to understand a function by breaking it up into parts upon which we can control the height or width of a function. We say $f$ is a \emph{sub step function} with height $H$ and width $W$ if $f$ is supported on a set $E$ with $|E| \leq W$, and $|f(x)| \leq H$. A \emph{quasi step function} with height $H$ and width $W$ if $f$ is supported on a set $E$ with $|E| \sim W$ and on $E$, $|f(x)| \sim H$.

\begin{remark}
  It might seem that sub step functions of height $H$ and width $W$ can take on a great many different behaviours, rather than that of a step function with height $H$ and width $W$. However, from the point of view of monotone, translation invariant norms, this isn't so. This is because using the binary expansion of real numbers, for every sub-step function $f$ of height $H$ and width $W$, we can find sets $\{ E_k \}$ such that
  %
  \[ f(x) = H \sum_{k = 1}^\infty 2^{-k} \mathbf{I}_{E_k}, \]
  %
  where $|E_k| = 1$. Thus bounds on step functions that are stable under addition tend to automatically imply bounds on substep functions.
\end{remark}

We start by discussing the \emph{vertical dyadic layer cake decomposition}. We define, for each $k \in \ZZ$,
%
\[ f_k(x) = f(x) \mathbf{I}(2^{k-1} < |f(x)| \leq 2^k) \]
%
Then we set $f = \sum f_k$. Each $f_k$ is a quasi step function with height $2^k$ and width $F(2^{k-1}) - F(2^k)$. We can also perform a \emph{horizontal layer cake decomposition}. If we define $H_k = f^*(2^k)$, and set
%
\[ f_k(x) = f(x) \mathbf{I}(H_{k-1} < |f(x)| \leq H_k), \]
%
then $f_k$ is a substep function with height $H_k$ and width $2^k$. These decompositions are best visualized with respect to the representation $f^*$ of $f$, in which case the decomposition occurs over particular intervals.

\begin{theorem}
    The following values $A_1, \dots, A_4$ are all comparable up to absolute constant depending only on $p$ and $q$:
    %
    \begin{enumerate}
        \item \label{onebound} $\| f \|_{p,q} \leq A_1$.

        \item \label{twobound} We can write $f = \sum_{k \in \ZZ} f_k$, where $f_k$ is a quasi-step function with height $2^k$ and width $W_k$, and
        %
        \[ \left( \sum_{k \in \ZZ} \left[ 2^k W_k^{1/p} \right]^q \right)^{1/q} \leq A_2. \]

        \item \label{threebound} We can write $f = \sum_{k \in \ZZ} f_k$, where $f_k$ is a sub-step function with height $2^k$ and width $W_k$, and
        %
        \[ \left( \sum_{k \in \ZZ} \left[2^{k} W_k^{1/p} \right]^q \right)^{1/q} \leq A_3. \]

        \item \label{fourbound} We can write $f(x) = \sum_{k \in \ZZ} f_k$, where $f_k$ is a sub-step function with width $2^k$ and height $H_k$, where $\{ H_k \}$ is a decreasing family of functions, and
        %
        \[ \left( \sum_{k \in \ZZ} \left[H_k 2^{k/p} \right]^q \right)^{1/q} \leq A_4. \]
    \end{enumerate}
\end{theorem}
\begin{proof}
    It is obvious that we can always select $A_3 \leq A_2$. Next, we bound $A_2$ in terms of $A_1$ by performing a vertical layer cake decomposition on $f$. If we write $f = \sum_{k \in \ZZ} f_k$, then $f_k$ is supported on a set with measure $W_k = F(2^{k-1}) - F(2^k) \leq F(2^{k-1})$, and so
    %
    \begin{align*}
        \sum_{k \in \ZZ} [2^k W_k^{1/p}]^q &\leq \sum_{k \in \ZZ} [2^k F(2^{k-1})^{1/p}]^q\\
        &\lesssim_q \sum_{k \in \ZZ} [2^{k-1} F(2^k)^{1/p}]^q\\
        &\lesssim \sum_{k \in \ZZ} \int_{2^{k-1}}^{2^k} [tF(t)^{1/p}]^q\; \frac{dt}{t} \lesssim_q \| f \|_{p,q}^q \leq A_1^q.
    \end{align*}
    %
    Thus $A_2 \lesssim_q A_1$. Next, we bound $A_4$ in terms of $A_1$. Perform a horizontal layer cake decomposition, writing $f = \sum f_k$, where $f_k$ is supported on a set with measure $W_k \leq 2^k$, and $H_{k+1} \leq |f_k(x)| \leq H_k$. Then a telescoping sum shows
    %
    \begin{align*}
        H_k 2^{k/p} &= \left( \sum_{m = 0}^\infty (H_{k+m}^q - H_{k+m+1}^q) 2^{kq /p} \right)^{1/q}\\
        &\lesssim_q \left( \sum_{m = 0}^\infty \int_{H_{k+m+1}}^{H_{k+m}} [t 2^{k/p}]^q \frac{dt}{t} \right)^{1/q}\\
        &\leq \left( \sum_{m = 0}^\infty \int_{H_{k+m+1}}^{H_{k+m}} [t F(t)^{1/p}]^q \frac{dt}{t} \right)^{1/q}
    \end{align*}
    %
    Thus
    %
    \[ \left( \sum_{k \in \ZZ} [H_k 2^{k/p}]^q \right)^{1/q} \leq \left( \int_0^\infty [t F(t)^{1/p}]^q \frac{dt}{t} \right)^{1/q} \lesssim_q A_1. \]
    %
    Thus $A_4 \lesssim_q A_1$. It remains to bound $A_1$ by $A_4$ and $A_3$. Given $A_3$, we can write $|f(x)| \leq \sum 2^k \mathbf{I}_{E_k}$, where $|E_k| \leq W_k$. We then find
    %
    \[ F(2^k) \leq \sum_{m = 1}^\infty W_{k+m}. \]
    %
    Thus
    %
    \[ \int_{2^{k-1}}^{2^k} [t F(t)^{1/p}]^q \frac{dt}{t} \lesssim \left[ 2^k \left(\sum_{m = 0}^\infty W_k \right)^{1/p} \right]^q. \]
    %
    Thus if $q \leq p$,
    %
    \begin{align*}
        \| f \|_{p,q} &\lesssim_q \left( \sum_{k \in \ZZ} \left[2^k \left( \sum_{m = 0}^\infty W_{k+m} \right)^{1/p} \right]^q \right)^{1/q}\\
        &\leq \left( \sum_{k \in \ZZ} \sum_{m = 0}^\infty \left[ 2^k W_{k+m}^{1/p} \right]^q \right)^{1/q}\\
        &\leq \left( \sum_{m = 0}^\infty 2^{-qm} \sum_{k \in \ZZ} \left[ 2^{k+m} W_{k+m}^{1/p} \right]^q \right)^{1/q}\\
        &\leq \left( A_3^q \sum_{m = 0}^\infty 2^{-mq} \right)^{1/q} \lesssim_q A_3.
    \end{align*}
    %
    If $q \geq p$, we can employ the triangle inequality for $l^{q/p}$ to write
    %
    \begin{align*}
        \| f \|_{p,q} &\lesssim_q \left( \sum_{k \in \ZZ} \left[2^k \left( \sum_{m = 0}^\infty W_{k + m}  \right)^{1/p} \right]^q \right)^{1/q}\\
        &\leq \left( \sum_{m = 0}^\infty \left( \sum_{k \in \ZZ} 2^{kq} W_{k+m}^{q/p} \right)^{p/q} \right)^{1/p}\\
        &\leq \left( A_3^p \sum_{m = 0}^\infty 2^{-mq} \right)^{1/p} \lesssim_{p,q} A_3.
    \end{align*}
    %
    The bound of $A_1$ in terms of $A_4$ involves the same `shifting' technique, and is left to the reader.
\end{proof}

\begin{remark}
    Heuristically, the theorem above says that if $f = \sum_{k \in \ZZ} f_k$, where $f_k$ is a quasi-step function with width $H_k$ and width $W_k$, and if either $\{ H_k \}$ and $\{ W_k \}$ grow faster than powers of two, then
    %
    \[ \| f \|_{p,q} \sim_{p,q} \left( \sum_{k \in \ZZ} \left[ H_k W_k^{1/p} \right]^q \right)^{1/q}. \]
    %
    Thus the $L^{p,q}$ norm has little interaction between elements of the sum when the sum occurs over dyadically different heights or width. This is one reason why we view the $q$ parameter as a `logarithmic' correction of the $L^p$ norm. In particular, if we can write $f = f_1 + \dots + f_N$, and $q_1 < q_2$, then the last equation, combined with a $l^{q_1}$ to $l^{q_2}$ norm bound, gives
    %
    \[ \left( \sum_{k \in \ZZ} \left[ H_k W_k^{1/p} \right]^{q_1} \right)^{1/q_1} \leq N^{1/q_1 - 1/q_2} \left( \sum_{k \in \ZZ} \left[ H_k W_k^{1/p} \right]^{q_2} \right)^{1/q_2} \]
    %
    This implies
    %
    \[ \| f \|_{p,q_2} \lesssim_{p,q_1,q_2} \| f \|_{p,q_1} \lesssim_{p,q_1,q_2} N^{1/q_1 - 1/q_2} \| f \|_{p,q_2}. \]
    %
    In particular, this occurs if there exists a constant $C$ such that $C \leq |f(x)| \leq C \cdot 2^N$ for all $x$. On the other hand, if we vary the $p$ parameter, we find that for $p_1 < p_2$,
    %
    \[ \left( \sum_{k \in \ZZ} \left[ H_k W_k^{1/p_1} \right]^q \right)^{1/q} \leq \max(W_k)^{1/p_1 - 1/p_2} \left( \sum_{k \in \ZZ} \left[H_k W_k^{1/p_2} \right]^q \right)^{1/q}, \]
    \[ \left( \sum_{k \in \ZZ} \left[ H_k W_k^{1/p_2} \right]^q \right)^{1/q} \leq \left( \frac{1}{\min(W_k)} \right)^{1/p_1 - 1/p_2} \left( \sum_{k \in \ZZ} \left[ H_k W_k^{1/p_2} \right]^q \right)^{1/q}. \]
    %
    which gives
    %
    \[ \min(W_k)^{1/p_1 - 1/p_2} \| f \|_{p_2,q} \lesssim_{p_1,p_2,q} \| f \|_{p_1,q} \lesssim_{p_1,p_2,q} \max(W_k)^{1/p_1 - 1/p_2} \| f \|_{p_2,q}. \]
    %
    Both of these inequalities can be tight. Because of the dyadic decomposition of $f$, we find $\max(W_k) \geq 2^N \min(W_k)$, so these two norms can differ by at least $2^{N(1/p_1 - 1/p_2)}$, and at \emph{most} if the $f_k$ occur over consecutive dyadic values, which is \emph{exponential} in $N$. Conversely, if the heights change dyadically, we find that
    % q = q'p_2/p-1
    \begin{align*}
        \left( \sum_{k \in \ZZ} \left[ H_k W_k^{1/p_2} \right]^q \right)^{1/q} &\leq \left( \sum_{k \in \ZZ} \left[ H_k W_k^{1/p_2} \right]^{qp_2/p_1} \right)^{(p_1/p_2)/q}\\
        &\leq \max(H_k)^{1 - p_1/p_2} \left( \sum_{k \in \ZZ} \left[ H_k W_k^{1/p_1} \right]^q \right)^{(p_1/p_2)/q}
    \end{align*}
    %
    \begin{align*}
        \left( \sum_{k \in \ZZ} \left[ H_k W_k^{1/p_1} \right]^q \right)^{1/q} &\lessapprox \left( \sum_{k \in \ZZ} \left[ H_k W_k^{1/p_1} \right]^{qp_1/p_2} \right)^{(p_2/p_1)/q}\\
        &\leq \left( \frac{1}{\min(H_k)} \right)^{p_2/p_1 - 1} \left( \sum_{k \in \ZZ} \left[ H_k W_k^{1/p_2} \right]^q \right)^{(p_2/p_1)/q}
    \end{align*}
    %
    where $\lessapprox$ denotes a factor ignoring polynomial powers of $N$ occuring from the estimate. Thus
    %
    \[ \min(H_k)^{p_2 - p_1} \| f \|_{p_1,q}^{p_1} \lessapprox_{p_1,p_2,q} \| f \|_{p_2,q}^{p_2} \lesssim_{p_1,p_2,q} \max(H_k)^{p_2-p_1} \| f \|_{p_1,q}^{p_1} \]
    %
    again, these inequalities can be both tight, and $\max(H_k) \geq 2^N \min(H_k)$, with equality if the quasi step functions from which $f$ is composed occur consecutively dyadically.
\end{remark}

\begin{example}
    Consider the function $f(x) = |x|^{-s}$. For each $k$, let
    %
    \[ E_k = \{ x : 2^{-(k+1)/s} \leq |x| < 2^{-k/s} \} \]
    %
    and define $f_k = f \mathbf{I}_{E_k}$. Then $f_k$ is a quasi-step function with height $2^k$, and width $1/2^{dk/s}$. We conclude that if $p = d/s$, and $q < \infty$,
    %
    \[ \| f \|_{p,q} \sim_{p,q,d} \left( \sum_{k = -\infty}^\infty 2^{qk(1 - d/ps)} \right)^{1/q} = \infty. \]
    %
    Thus the function $f$ lies exclusively in $L^{p,\infty}(\RR^d)$.
\end{example}

A simple consequence of the layer cake decomposition is H\"{o}lder's inequality for Lorentz spaces.

\begin{theorem}
    If $0 < p_1,p_2,p < \infty$ and $0 < q_1,q_2,q < \infty$ with
    %
    \[ 1/p = 1/p_1 + 1/p_2 \quad \text{and} \quad 1/q = 1/q_1 + 1/q_2, \]
    %
    then
    %
    \[ \| f g \|_{p,q} \lesssim_{p_1,p_2,q_1,q_2} \| f \|_{p_1,q_1} \| g \|_{p_2,q_2}. \]
\end{theorem}
\begin{proof}
    Without loss of generality, assume $\| f \|_{p_1,q_1} = \| g \|_{p_2, q_2} = 1$. Perform horizontal layer cake decompositions of $f$ and $g$, writing $|f| \leq \sum_{k \in \ZZ} H_k \mathbf{I}_{E_k}$ and $|g| \leq \sum_{k \in \ZZ} H_k' \mathbf{I}_{F_k}$, where $|E_k|, |F_k| \leq 2^k$. Then
    %
    \[ |fg| \leq \sum_{k,k' \in \ZZ} H_k H_k' \mathbf{I}_{E_k \cap F_{k'}} \]
    %
    For each fixed $k$, $|E_{k + m} \cap F_m| \leq 2^m$, and so
    %
    \begin{align*}
        \left\| \sum_{m \in \ZZ} H_{k + m} H_m' \mathbf{I}_{E_{k+m} \cap F_m} \right\|_{p,q} &\lesssim_{p,q} \left( \sum_{m \in \ZZ} [H_{k+m} H_m' 2^{m/p}]^q \right)^{1/q}\\
        &= \left( \sum_{m \in \ZZ} \left[ (H_{k+m} 2^{m/p_1}) (H_m 2^{m/p_2}) \right]^q \right)^{1/q}\\
        &\leq \left( \sum_{m \in \ZZ} [H_{k+m} 2^{m/p_1} ]^{q_1} \right)^{1/q_1} \left( \sum_{m \in \ZZ} [H_m' 2^{m/p_2}]^{q_2} \right)^{1/q_2}\\
        &\lesssim_{p,q,p_1,q_1,p_2,q_2} 2^{-k/p_1}\\
    \end{align*}
    %
    Summing over $k > 0$ gives that
    %
    \[ \left\| \sum_{k \geq 0} \sum_{m \in \ZZ} H_{k+m} H_m' \mathbf{I}_{E_{k+m} \cap F_m} \right\| \lesssim_{p,q,p_1,q_1,p_2,q_2} 1 \]
    %
    By the quasitriangle inequality, it now suffices to obtain a bound
    %
    \[ \left\| \sum_{k < 0} \sum_{m \in \ZZ} H_{k+m} H_m' \mathbf{I}_{E_{k+m} \cap F_m} \right\|_{p,q}. \]
    %
    This is done similarily, but using the bound $|E_{k+m} \cap F_m| \leq 2^{k+m}$ instead of the other bound.
\end{proof}

\begin{corollary}
    If $p > 1$ and $q > 0$, $L^{p,q}(X) \subset L^1_{\text{loc}}(X)$.
\end{corollary}
\begin{proof}
    Let $E$ have finite measure and let $f \in L^{p,q}(X)$. Then the H\"{o}lder's inequality for Lorentz spaces shows
    % 1 = 1/p + 1/p_2 = 1/q + 1/q_2
    %
    \[ \| f \|_{L^1(E)} = \| \mathbf{I}_E f \|_{L^1(X)} \lesssim_{p,q} |E|^{1 - 1/p} \| f \|_{p,q} < \infty. \qedhere \]
\end{proof}

Finally, we consider the duality of the $L^{p,q}$ norms. If $1 < p < \infty$, and $1 < q < \infty$, then $L^{p,q}(X)^* = L^{p',q'}(X)$. When $q = 1$ or $q = \infty$, things are more complex, but the following theorem often suffices. When $p = 1$, things get more tricky, so we leave this case out.

\begin{theorem}
    Let $1 < p < \infty$ and $1 \leq q < \infty$. Then if $f \in L^{p,q}(X)$,
    %
    \[ \| f \|_{p,q} \sim \sup \left\{ \int fg : \| g \|_{p',q'} \leq 1 \right\}. \]
\end{theorem}
\begin{proof}
    Without loss of generality, we may assume $\| f \|_{p,q} = 1$. We may perform a vertical layer cake decomposition, writing $f = \sum_{k \in \ZZ} f_k$, where $2^{k-1} \leq |f_k(x)| \leq 2^k$, is supported on a set with width $W_k$, and
    %
    \[ \left( (2^k W_k^{1/p})^q \right) \sim_{p,q} 1. \]
    %
    Define $a_k = 2^k W_k^{1/p}$, and set $g = \sum_{k \in \ZZ} g_k$, where $g_k(x) = a_k^{q-p} \text{sgn}(f_k(x)) |f_k(x)|^{p-1}$. Then
    %
    \begin{align*}
        \int f(x) g(x) &= \sum_{k \in \ZZ} \int f_k(x) g_k(x) = \sum_{k \in \ZZ} a_k^{q-p} \int |f_k(x)|^p\\
        &\gtrsim_p \sum_{k \in \ZZ} a_k^{q-p} W_k 2^{kp} = \sum_{k \in \ZZ} a_k^q \gtrsim_{p,q} 1.
    \end{align*}
    %
    We therefore need to show that $\| g \|_{p',q'} \lesssim 1$. We note $|g_k(x)| \lesssim a_k^{q-p} 2^{kp}$, and has width $W_k$. The gives a decomposition of $g$, but neither the height nor the widths necessarily in powers of two. Still, we can fix this since the heights increase exponentially; define
    %
    \[ H_k = \sup_{l \geq 0} a_{k-l}^{q-p} 2^{kp} 2^{-lp/2}. \]
    %
    Then $|g_k(x)| \lesssim_{p,q} H_k$, and $H_{k+1} \geq 2^{p/2} H_k$. In particular, if we pick $m$ such that $2^{mp/2} \geq 1$, then for any $l \leq m$, the sequence $H_{km + l}$, as $k$ ranges over values, increases dyadically, and so by the quasitriangle inequality for the $L^{p',q'}$ norm, and then the triangle inequality in $l^q$, we find
    % W_k = a_k^p/ 2^{kp}
    \begin{align*}
        \| g \|_{p',q'} &\lesssim_{m,p,q} \left( \sum [H_k W_k^{1/p'}]^{q'} \right)^{1/q'}\\
        &\lesssim \left( \sum_{k \in \ZZ} \left[ \left( \sup_{l \geq 0} a_{k-l}^{q-p} 2^{kp} 2^{-lp/2} \right) (a_k 2^{-k})^{p-1} \right]^{q'} \right)^{1/q'}\\
        &\lesssim_p \left( \sum_{k \in \ZZ} \left[ a_k^{p-1} \sum_{l = 0}^\infty a_{k-l}^{q-p} 2^{-lp/2} \right]^{q'} \right)^{1/q'}\\
        &\lesssim \sum_{l = 0}^\infty 2^{-lp/2} \left( \sum_{k \in \ZZ} \left[ a_k^{p-1} a_{k-l}^{q-p} \right]^{q'} \right)^{1/q'}.
    \end{align*}
    %
    Applying's H\"{o}lder's inequality shows
    %
    \begin{align*}
        \left( \sum_{k \in \ZZ} \left[ a_k^{p-1} a_{k-l}^{q-p} \right]^{q'} \right)^{1/q'} &\leq  \left( \sum_{k \in \ZZ} a_k^q \right)^{(p-1)/q} \left( \sum_{k \in \ZZ} a_{k-l}^q \right)^{(q-p)/q}\\
        &\lesssim_{p,q} \| f \|_{p,q}^{q-1} \lesssim_{p,q} 1. \qedhere
    \end{align*}
\end{proof}

\begin{remark}
    This technique shows that if $f = \sum f_k$, where $f_k$ is a quasi-step function with measure $W_k$ and height $2^{ck}$, then we can find $m$ such that $cm > 1$, and then consider the $m$ functions $f^1, \dots, f^m$, where $f_i = \sum f_{km + i}$. Then the functions $f_{km + i}$ have heights which are separated by powers of two, and so the quasi-triangle inequality implies
    %
    \begin{align*}
        \| f \|_{p,q} &\lesssim_m \sum_{i = 1}^m \| f^i \|_{p,q}\\
        &\lesssim_{p,q} \sum_{i = 1}^m \left( \sum \left[ H_{km + i} W_{km + i}^{1/p} \right]^q \right)^{1/q}\\
        &\lesssim_m \left( \sum \left[ H_k W_k^{1/p} \right]^q \right)^{1/q}
    \end{align*}
    %
    On the other hand,
    %
    \begin{align*}
        \| f \|_{p,q} &\gtrsim \max_{1 \leq i \leq m} \| f^i \|_{p,q}\\
        &\sim \max_{1 \leq i \leq m} \left( \sum \left[ H_{km + i} W_{km + i}^{1/p} \right]^q \right)^{1/q}\\
        &\gtrsim_m \left( \sum \left[ H_k W_k^{1/p} \right]^q \right)^{1/q}.
    \end{align*}
    %
    Thus the dyadic layer cake decomposition still works in this setting.
\end{remark}

We remark that if $1 < p < \infty$ and $1 \leq q \leq \infty$, then for each $f \in L^{p,q}$, the value
%
\[ \sup \left\{ \int fg : \| g \|_{p',q'} \leq 1 \right\} \]
%
gives a norm on $L^{p,q}(X)$ which is comparable with the $L^{p,q}$ norm. In particular, this implies that for $p > 1$ and $q \geq 1$,
%
\[ \| f_1 + \dots + f_N \|_{p,q} \lesssim_{p,q} \| f_1 \|_{p,q} + \dots + \| f_N \|_{p,q}, \]
%
so that the triangle inequality has constants independent of $N$. We can also use a layer cake decomposition to get a version of the Stein-Weiss inequality for Lorentz norms.

\begin{theorem}
	For each $1 < q < \infty$, there is $\alpha(q) > 0$ such that for any functions $f_1, \dots, f_N$,
	%
	\[ \| f_1 + \dots + f_N \|_{1,q} \lesssim (\log N)^{\alpha(q)} \left( \| f_1 \|_{1,q} + \dots + \| f_N \|_{1,q} \right). \]
\end{theorem}
\begin{proof}
	For values $A$ and $B$ in this argument, we write $A \lessapprox B$ if there exists $\alpha$ such that $A \lesssim (\log N)^\alpha B$. Given $f_1, \dots, f_N$, write $f_i = \sum_{j = -\infty}^\infty f_{ij}$, where $f_{ij}$ has width $W_{ij}$ and height $2^j$. If we assume, without loss of generality, that $\| f_1 \|_{1,q} + \dots + \| f_N \|_{1,q} = 1$, then
	%
	\[ \sum_{i = 1}^N \left( \sum_{j = -\infty}^\infty (2^j W_{ij})^q \right)^{1/q} \lesssim_q 1 \]
	%
	Thus we want to show $\| f_1 + \dots + f_N \|_{1,q} \lessapprox_q 1$. Our first goal is to upper bound the measure of the set
	%
	\[ E = \{ x: 2^{k-1} < |f_1(x) + \dots + f_N(x)| \leq 2^k  \} \]
	%
	The measure of the set $E$ is upper bounded by the measure of the set
	%
	\[ E' = \left\{ x: 2^{k-2} < \left|\sum_{j = k - \lg(N)}^k f_{1j}(x) + \dots + f_{Nj}(x) \right| \leq 2^{k+1} \right\} \]
	%
	Applying the usual Stein-Weiss inequality, we have
	%
	\[ \left\| \sum_{i = 1}^N \sum_{j = k - \lg N}^k f_{ij} \right\|_{1,\infty} \lessapprox \sum_{i = 1}^N \sum_{j = k - \lg N}^k \| f_{ij} \|_{1,\infty} \lesssim \sum_{i = 1}^N \sum_{j = k - \lg N}^k \| f_{ij} \|_{1,\infty} \lesssim_q \sum_{i = 1}^N \sum_{j = k - \lg N}^k W_{ij} 2^j \]
	%
	Thus we conclude
	%
	\[ |E'| \lessapprox_q 2^{-k} \sum_{i = 1}^N \sum_{j = k - \lg N}^k W_{ij} 2^j \]
	%
	This implies that
	%
	\[ \| f_1 + \dots + f_N \|_{1,q} \lessapprox_q \left( \sum_{k = -\infty}^\infty \left( \sum_{i = 1}^N \sum_{j = k - \lg N}^k W_{ij} 2^j \right)^q \right)^{1/q}. \]
	%
	Applying Minkowski's inequality, we conclude
	%
	\begin{align*}
		\left( \sum_{k = -\infty}^\infty \left( \sum_{i = 1}^N \sum_{j = k - \lg N}^k W_{ij} 2^j \right)^q \right)^{1/q} &\lesssim \sum_{i = 1}^N \left( \sum_{k = -\infty}^\infty \left( \sum_{j = k - \lg N}^k W_{ij 2^j} \right)^q \right)^{1/q}\\
		&\lessapprox \sum_{i = 1}^N \left( \sum_{k = -\infty}^\infty \sum_{j = k - \lg N}^k W_{ij}^q 2^{qj} \right)^{1/q}\\
		&\lessapprox \sum_{i = 1}^N \left( \sum_{j = -\infty}^\infty W_{ij}^q 2^{qj} \right)^{1/q} \lesssim 1. \qedhere
	\end{align*}
\end{proof}

\begin{comment}

\section{Normability of the Lorentz Spaces}

Though the Lorentz norms do not satisfy the triangle inequality, the space $L^{p,q}(X)$ is still a `Banach-able' space when $p > 1$, and $q \geq 1$. First off, the standard proof shows the norm gives a complete quasimetric, since a Cauchy sequence in the $L^{p,q}$ norm converges to a function almost everywhere, which is easily verified to have finite $L^{p,q}$ norm. The easiest way to define a norm is to through the decreasing rearrangement.

\begin{lemma}
    For any measurable set $E$,
    %
    \[ \int_E |f(x)|\; dx \leq \int_0^{|E|} f^*(t)\; dt. \]
    %
    and
    %
    \[ \int_{\{ |f(x)| > t \}} |f(x)|\; dx = \int_0^{F(t)} f^*(t)\; dt. \]
\end{lemma}
\begin{proof}
    If $g \leq f$, $g^* \leq f^*$. Thus $(\chi_E f)^* \leq f^*$, so
    %
    \[ \int_E |f(x)|\; dx = \int \chi_E |f(x)| = \int_0^\infty (\chi_E f)^*(t)\; dt = \int_0^{|E|} (\chi_E f)^*(t)\; dt \leq \int_0^{|E|} f^*(t)\; dt. \]
    %
    On the other hand, $(\chi_E f)^* = f^*$ when $E = \{ |f(x)| > t \}$, which gives the second equality.
\end{proof}

For a function $f$ and $t > 0$, we define a family of averages
%
\[ m(t) = \frac{1}{t} \int_0^t f^*(t)\; dt. \]
%
For any fixed $t > 0$, the map $f \mapsto m(t)$ is a norm. Provided our measure space is non-atomic, we have
%
\[ m(t) = \sup_{|E| \leq t} \int_E |f(x)|\; d\mu. \]
%
We define
%
\[ \vvvert f \vvvert_{p,q} = \left( \frac{q}{p} \int_0^\infty [t^{1/p} m(t)]^q \frac{dt}{t} \right)^{1/q} \]
and
%
\[ \vvvert f \vvvert_{p,\infty} = \sup t^{1/p} m(t). \]
%
For $q \geq 1$, each of these functions is a norm, simply because the function $m$ is a norm. On the other hand, since $f^*$ is decreasing, $f^*(t) \leq m(t)$ for all $t$, which shows $\vvvert f \vvvert_{p,q} \geq \| f \|_{p,q}$. If $p = 1$ and $q < \infty$, if $\vvvert f \vvvert_{1,q} < \infty$, then $f = 0$, so these norms are effectively useless. If $q = \infty$, then
%
\[ \vvvert f \vvvert_{1,\infty} = \| f^* \|_{L^1[0,\infty)} = \| f \|_1, \]
%
and therefore doesn't measure the correct norm. But in all other cases, i.e. for $p > 1$ and $q \geq 1$, the norm is comparable to the $L^{p,q}$ norm.

\begin{theorem}
    If $p > 1$,
    %
    \[ \vvvert f \vvvert_{p,q} \leq \frac{p}{p-1} \| f \|_{p,q}. \]
\end{theorem}
\begin{proof}
    We utilize a \emph{Hardy's inequality} technique, which shows that the $L^p$ norm of the averages of a function are comparable to the $L^p$ norm of the function. Applying Minkowski's integral inequality, we conclude that
    %
    \begin{align*}
        \left( \frac{q}{p} \int_0^\infty [t^{1/p} m(t)]^q \frac{dt}{t} \right)^{1/q} &= \left( \frac{q}{p} \int_0^\infty \left( \int_0^1 t^{1/p} f^*(ts)\; ds \right)^q\; \frac{dt}{t} \right)^{1/q}\\
        &\leq \int_0^1 \left( \int_0^\infty \frac{q}{p} (t^{1/p} f^*(ts))^q\; \frac{dt}{t} \right)^{1/q}\; ds\\
        &\leq \left( \int_0^1 s^{- 1/p}\; ds \right) \left( \frac{q}{p} \int_0^\infty (t^{1/p} f^*(t))^q\; \frac{dt}{t} \right)^{1/q}\\
        &\leq \frac{1}{1 - 1/p} \| f \|_{p,q} = \frac{p}{p - 1} \| f \|_{p,q}.
    \end{align*}
    %
    For $q = \infty$, and $t > 0$, we have
    %
    \begin{align*}
        t^{1/p} m(t) &= t^{1/p - 1} \int_0^t f^*(t)\\
        &\leq (\sup_{s > 0} s^{1/p} f^*(s)) t^{1/p - 1} \int_0^t t^{-1/p}\\
        &= \frac{1}{1 - 1/p} \| f \|_{p,\infty} = \frac{p}{p-1} \| f \|_{p,\infty}.
    \end{align*}
    %
    since $t$ was arbitrary, this gives the required bound.
\end{proof}

\end{comment}

\section{Mixed Norm Spaces}

Given two measure spaces $X$ and $Y$, we can form the product measure space $X \times Y$. If we have a norm space $V$ of functions on $X$, with norm $\| \cdot \|_V$ and a norm space $W$ of functions on $Y$, with norm $\| \cdot \|_W$, we can consider a `product norm'; for each function $f$ on $X \times Y$, we can consider the function $y \mapsto \| f(\cdot,y) \|_V$, and take the norm of this function over $Y$, i.e. $\| \| f(\cdot,y) \|_V \|_W$. The most important case of this process is where we fix $0 < p,q \leq \infty$, and consider
%
\[ \| f \|_{L^p(X) L^q(Y)} = \left( \int \left( \int |f(x,y)|^p \; dx \right)^{q/p}\; dy \right)^{1/q}. \]
%
Similarly, we can define $\| f \|_{L^q(Y) L^p(X)}$. We have a duality theory here; for each $1 \leq p,q < \infty$ and any $f$ with $\| f \|_{L^p(X) L^q(Y)} < \infty$, the standard $L^p$ and $L^q$ duality gives
%
\[ \| f \|_{L^p(X) L^q(Y)} = \sup \left\{ \int_{X \times Y} f(x,y) h(x,y)\; dx\;dy : \| h \|_{L^{p^*}(X) L^{q^*}(Y)} \leq 1 \right\}. \]
%
It is often important to interchange norms, and we find the biggest quantity obtained by interchanging norms is always obtained with the largest exponents on the inside.

\begin{theorem}
	If $q \geq p \geq 1$, $\| f \|_{L^p(X) L^q(Y)} \leq \| f \|_{L^q(Y) L^p(X)}$.
\end{theorem}
\begin{proof}
  If $p = q$, then the Fubini-Tonelli theorem implies that
  %
  \[ \| f \|_{L^p(X) L^q(Y)} = \| f \|_{L^q(Y) L^p(X)}. \]
  %
  If $p = 1$, then this result is precisely the Minkowski inequality. We now apply complex interpolation to obtain the result in general. In fact, a simple variation of the proof of Riesz-Thorin using the duality established above gives the result.
\end{proof}

Two special cases are that pointwise maxima dominate individual maxima
%
\[ \sup_n \| f_n \|_{L^p(X)} \leq \left\| \sup_n f_n \right\|_{L^p(X)} \]
%
and that we have the triangle inequality
%
\[ \left\| \sum_n f_n \right\|_{L^p(X)} \leq \sum_n \| f_n \|_{L^p(X)} \]
%
for $p \geq 1$.

It turns out that if $q > p$ and $\| f \|_{L^p(X) L^q(Y)} = \| f \|_{L^q(Y) L^p(X)}$, then $|f|$ is a tensor product. Thus switching mixed norms is likely only efficient if the functions we are working with are close to tensor products.

\begin{theorem}
  Suppose $q > p$, $f$ is a function on $X \times Y$, and
  %
  \[ \| f \|_{L^p(X) L^q(Y)} = \| f \|_{L^q(Y) L^p(X)} < \infty. \]
  %
  Then there exists $f_1(x)$ and $f_2(y)$ such that for any $x \in X$ and $y \in Y$, $|f(x,y)| = |f_1(x)| |f_2(y)|$.
\end{theorem}
\begin{proof}
  Expanding this equation out, we conclude
  %
  \[ \left( \int_Y \left( \int_X |f(x,y)|^p\; dx \right)^{q/p}\; dy \right)^{1/q} = \left( \int_X \left( \int_Y |f(x,y)|^q\; dy \right)^{p/q}\; dx \right)^{1/p}. \]
  %
  Setting $g(x,y) = |f(x,y)|^p$, we see that Minkowski's integral inequality is tight for $g$, i.e.
  %
  \[ \left( \int_Y \left( \int_X |g(x,y)|\; dx \right)^{q/p}\; dy \right)^{p/q} = \left( \int_X \left( \int_Y |g(x,y)|^{q/p}\; dy \right)^{p/q}\; dx \right). \]
  %
  Thus it suffices to show that to show the theorem for $p = 1$ and $q > 1$. Recall the standard proof of Minkowski's inequality, i.e. that by H\"{o}lder's inequality
  %
  \begin{align*}
    \int_Y \left( \int_X |f(x,y)|\; dx \right)^p\; dy &= \int_X \left[ \int_Y |f(x_1,y)| \left( \int_X |f(x_2,y)|\; dx_2 \right)^{p-1}\; dy \right]\; dx_1\\
    &\leq \int_X \left[ \left( \int_Y |f(x_1,y)|^p\; dy \right)^{1/p} \left( \int_Y \left( \int_X |f(x_2,y)|\; dx_2 \right)^{(p-1)p^*}\; dy \right)^{1/p^*} \right]\; dx_1 \\
    &= \left[ \int_X \left( \int_Y |f(x_1,y)|^p\; dy \right)^{1/p}\; dx_1 \right] \left[ \int_Y \left( \int_X |f(x_2,y)|\; dx_2 \right)^p \right]^{1/p^*}.
  \end{align*}
  %
  and rearranging gives Minkowski's inequality. If this inequality is tight, then our application of H\"{o}lder's inequality is tight for almost every $x_1 \in X$. Since $\int |f(x_2,y)|\; dx_2 \neq 0$ for all $y$ unless $f = 0$, it follows that there exists $\lambda(x_1)$ for almost every $x_1 \in X$ such that for almost every $y \in Y$,
  %
  \[ |f(x_1,y)|^p = |\lambda(x_1)| \left( \int_X |f(x_2,y)|\; dx_2 \right)^{p^*(p-1)} = |\lambda(x_1)| \left( \int_X |f(x_2,y)|\; dx_2 \right)^p. \]
  %
  Setting $f_1(x) = |\lambda(x)|^{1/p}$ and $f_2(y) = \int_X |f(x,y)|\; dx$ thus completes the proof.
\end{proof}

TODO: Show that if $q < p$ and $\| f \|_{L^p(X) L^q(Y)} = \| f \|_{L^p(Y) L^q(X)} < \infty$, then $|f|$ is a tensor product. Thus interchanging norms is only a good idea if we think the worst case example in a problem is a tensor-product like function.

\section{Orlicz Spaces}

To develop the class of Orlicz spaces, we note that if $\| f \|_p \leq 1$, and we set $\Phi(t) = t^p$, then
%
\[ \int \Phi \left( |f(x)| \right)\; dx = 1. \]
%
More generally, given any function $\Phi: [0,\infty) \to [0,\infty)$, we might ask if we can define a norm $\| \cdot \|_\Phi$ such that if $\| f \|_\Phi \leq 1$, then
%
\[ \int \Phi \left( |f(x)| \right)\; dx = 1. \]
%
Since a norm would be homogenous, this would imply that if $\| f \|_\Phi \leq A$, then
%
\[ \int \Phi \left( \frac{|f(x)|}{A} \right)\; dx \leq 1. \]
%
If we want these norms to be monotone, we might ask that if $A < B$, then
%
\[ \int \Phi \left( \frac{|f(x)|}{B} \right)\; dx \leq \int \Phi \left( \frac{|f(x)|}{A} \right), \]
%
and the standard way to ensure this is to ask the $\Phi$ is an increasing function. To deal with the property that $\| 0 \| = 0$, we set $\Phi(0) = 0$. In order for $\| \cdot \|_\Phi$ to be a norm, the set of functions $\{ f : \| f \|_\Phi \leq 1 \}$ needs to be convex, and the standard way to obtain this is to assume that $\Phi$ is convex.

In short, we consider an increasing, convex function $\Phi$ with $\Phi(0) = 0$. We then define
%
\[ \| f \|_\Phi = \inf \left\{ A > 0 : \int \Phi \left( \frac{|f(x)|}{A} \right)\; dx \leq 1 \right\}. \]
%
This function is a norm on the space of all $f$ with $\| f \|_\Phi < \infty$. It is easy to verify that $\| f \|_\Phi = 0$ if and only if $f = 0$ almost everywhere, and that $\| \alpha f \|_\Phi = |\alpha| \| f \|_\Phi$. To justify the triangle inequality, we note that if
%
\[ \int \Phi \left( \frac{|f(x)|}{A} \right) \leq 1 \quad\text{and} \quad \int \Phi \left( \frac{|f(x)|}{B} \right) \leq 1, \]
%
then applying convexity gives
%
\begin{align*}
    \int \Phi \left( \frac{|f(x) + g(x)|}{A + B} \right) &\leq \int \Phi \left( \frac{|f(x)| + |g(x)|}{A + B} \right)\\
    &\leq \int \left( \frac{A}{A + B} \right) \Phi \left( \frac{|f(x)|}{A} \right) + \left( \frac{B}{A + B} \right) \Phi \left( \frac{|g(x)|}{B} \right) \leq 1.
\end{align*}
%
Thus we obtain the triangle inequality.

The spaces $L^p(X)$ for $p \in [1,\infty)$ are Orlicz spaces with $\Phi(t) = t^p$. The space $L^\infty(X)$ is not really an Orlicz space, but it can be considered as the Orlicz function with respect to the `convex' function
%
\[ \Phi(t) = \begin{cases} \infty & t > 1, \\ t & t \leq 1. \end{cases} \]
%
More interesting examples of Orlicz spaces include
%
\begin{itemize}
    \item $L \log L$, given by the Orlicz norm induced by $\Phi(t) = t \log(2 + t)$.
    \item $e^L$, defined with respect to $\Phi(t) = e^t - 1$.
    \item $e^{L^2}$, defined with respect to $\Phi(t) = e^{t^2} - 1$.
\end{itemize}
%
One should not think too hard about the constants in the functions defined above, which are included to make $\Phi(0) = 0$. When we are dealing with a finite measure space, they are irrelevant.

\begin{lemma}
  If $\Phi(x) \lesssim \Psi(x)$ for all $x$, then $\| f \|_{\Phi(L)} \lesssim \| f \|_{\Psi(L)}$. If $X$ is finite, and $\Phi(x) \lesssim \Psi(x)$ for sufficiently large $x$, then $\| f \|_{\Phi(L)} \lesssim \| f \|_{\Psi(L)}$.
\end{lemma}
\begin{proof}
  The first proposition is easy, and we now deal with the finite case. We note that the condition implies that for each $\varepsilon > 0$, there exists $C_\varepsilon$ such that $\Phi(x) \leq C_\varepsilon \Psi(x)$ if $|x| \geq \varepsilon$. Assume that $\| f \|_{\Psi(L)} \leq 1$, so that
  %
  \[ \int \Psi(|f(x)|)\; dx \leq 1. \]
  %
  Then convexity implies that for each $A > 0$,
  %
  \[ \int \Psi \left( \frac{|f(x)|}{A} \right) \leq \frac{1}{A}. \]
  %
  Thus
  %
  \begin{align*}
    \int \Phi\left( \frac{|f(x)|}{A} \right)\; dx &\leq \Phi(\varepsilon) |X| + C_\varepsilon \int \Psi \left( \frac{|f(x)|}{A} \right)\\
    &\lesssim \Phi(\varepsilon) |X| + \frac{C_\varepsilon}{A}.
  \end{align*}
  %
  If $\Phi(\varepsilon) \leq 2/|X|$, and $A \geq 2C_\varepsilon$, then we conclude that
  %
  \[ \int \Phi\left( \frac{|f(x)|}{A} \right)\; dx \leq 1. \]
  %
  Thus $\| f \|_{\Phi(L)} \lesssim 1$.
\end{proof}

The Orlicz spaces satisfy an interesting duality relation. Given a function $\Phi$, which we assume is \emph{superlinear}, in the sense that $\Phi(x)/x \to \infty$ as $x \to \infty$, define it's \emph{Young dual}, for each $y \in [0,\infty)$, by
%
\[ \Psi(y) = \sup \{ xy - \Phi(x) : x \in [0,\infty) \}. \]
%
Then $\Psi$ is the smallest function such that $\Phi(x) + \Psi(y) \geq xy$ for each $x,y$. This quantity is finite for each $y$ because $\Phi$ is superlinear; for each $y \geq 0$, there exists $x(y)$ such that $\Phi(x(y)) \geq xy$, and thus the maximum of $xy - \Phi(x)$ is attained for $x \leq x(y)$. In particular, since $\Phi$ is continuous, the supremum is actually attained. Conversely, for each $x_0 \in [0,\infty)$, convexity implies there exists a largest $y$ such that the line $y(x - x_0) + f(x_0) \leq f(x)$ for all $x \in [0,\infty)$. This means that $\Psi(y) = x_0y - x_0$.

We note also that $\Psi(0) = 0$, and $\Psi$ is increasing. Most importantly, the function is convex. Given any $y,z \in [0,\infty)$, and any $x \in [0,\infty)$,
%
\begin{align*}
  x (\alpha y + (1 - \alpha) z) - \Phi(x) &\leq \alpha(xy - \Phi(x)) + (1 - \alpha)(xz - \Phi(x))\\
  &\leq \alpha \Psi(y) + (1 - \alpha) \Psi(z).
\end{align*}
%
Taking infimum over all $x$ gives convexity. The function $\Psi$ is also superlinear, since for any $x \in [0,\infty)$,
%
\[ \lim_{y \to \infty} \frac{\Psi(y)}{y} \geq \lim_{y \to \infty} \frac{xy - \Phi(x)}{y} = x. \]
%
In particular, we can consider the Young dual of $\Psi$.

\begin{lemma}
  If $\Psi$ is the Young dual of $\Phi$, then $\Phi$ is the Young dual of $\Psi$.
\end{lemma}
\begin{proof}
  $\Pi$ is the smallest function such that $\Pi(x) + \Psi(y) \geq xy$. Since $\Phi(x) + \Psi(y) \geq xy$ for each $x$ and $y$, we conclude that $\Pi(x) \leq \Phi(x)$ for each $x$. For each $x$, there exists $y$ such that $\Psi(y) = yx - \Phi(x)$. But this means that $\Phi(x) = yx - \Psi(y) \leq \Pi(x)$.
\end{proof}

Given the Orlicz space $\Phi(L)$ for superlinear $\Phi$, we can consider the Orlicz space $\Psi(L)$, where $\Psi$ is the Young dual of $\Phi$. The inequality $xy \leq \Phi(x) + \Psi(y)$, then
%
\[ |f(x) g(x)| \leq \Phi(|f(x)|) + \Psi(|g(x)|), \]
%
so if $\| f \|_{\Phi(L)}, \| g \|_{\Psi(L)} \leq 1$, then
%
\[ \left| \int f(x) g(x) \right| \leq \int |f(x)| |g(x)| \leq \int \Phi(|f(x)|) + \int \Psi(|g(x)|) \leq 2. \]
%
Thus in general, we have
%
\[ \left| \int f(x) g(x) \right| \leq 2 \| f \|_{\Phi(L)} \| g \|_{\Psi(L)}, \]
%
a form of H\"{o}lder's inequality. The duality between convex functions extends to a duality between the Orlicz spaces.

\begin{theorem}
  For any superlinear $\Phi$ with Young dual $\Psi$,
  %
  \[ \| f \|_{\Phi(L)} \sim \sup \left\{ \int fg : \| g \|_{\Psi(L)} \leq 1 \right\}. \]
\end{theorem}
\begin{proof}
  Without loss of generality, assume $\| f \|_{\Phi(L)} = 1$. The version of H\"{o}lder's inequality proved above shows that
  %
  \[ \| f \|_{\Phi(L)} \lesssim 1. \]
  %
  Conversely, for each $x$, we can find $g(x)$ such that $f(x) g(x) = \Phi(|f(x)|) + \Psi(|g(x)|$. Provided $\| g \|_{\Psi(L)} < \infty$, we have
  %
  \[ \int fg = \int \Phi(|f(x)|) + \int \Psi(|g(x)|) \geq 1 + \| g \|_{\Psi(L)}. \]
  %
  Assuming $f \in L^\infty(X)$, we may choose $g \in L^\infty(X)$. For such a choice of function, $\| g \|_{\psi(L)} < \infty$, which implies the result. Taking an approximation argument then gives the result in general.
\end{proof}

Let us now consider some examples of duality.

\begin{example}
  If $\Phi(x) = x^p$, for $p \geq 1$, and $1 = 1/p + 1/q$, then it's Young dual $\Psi$ satisfies
  % q = p/(p-1)
  \begin{align*}
    \Psi(y) &= \sup_{x \geq 0} xy - x^p = y^{1 + q/p} / p^{q/p} - y^q / p^q = y^q [p^{-q/p} - p^{-q}].
  \end{align*}
  %
  Thus the Young dual corresponds, up to a constant, to the conjugate dual in the $L^p$ spaces.
\end{example}

\begin{example}
  Suppose $X$ has finite measure. If $\Phi(t) = e^t - 1$, then it's dual satisfies, for large $y$,
  %
  \begin{align*}
    \Psi(y) &= \sup_{x \geq 0} xy - (e^x - 1)\\
    &= y \log y - (y - 1) \sim y \log y.
  \end{align*}
  %
  This is comparable to $y \log (y + 2)$ for large $y$. Thus $L \log L$ is dual to $e^L$.
\end{example}

\begin{example}
  Suppose $X$ has finite measure. If $\Phi(x) = e^{x^2} - 1$, then for $y \geq 2$,
  %
  \begin{align*}
    \Psi(y) &= \sup_{x \geq 0} xy - (e^{x^2} - 1) \sim y \log(y/2)^{1/2}.
  \end{align*}
  %
  Thus the dual of $e^{L^2}$ is the space $L (\log L)^{1/2}$.
\end{example}

There is a generalization of both the Lorentz spaces and the Orlicz spaces, known as the Lorentz-Orlicz spaces, but these come up so rarely in analysis that we do not dwell on these norms.
















\chapter{Interpolation Theory}

One of the most fundamental tools in the `hard style' of mathematical analysis, involving explicit quantitative estimates on quantities that arises in basic methods of mathematics, is the theory of interpolation. The main goal of interpolation is to take two estimates, and blend them together to form a family of intermediate estimates. Often each estimate will focus on one component of the problem at hand (an estimate in terms of the decay of the function at $\infty$, an estimate involving the growth of the derivative, or the low frequency the function is, etc). By interpolating, we can optimize and obtain an estimate which simultaneously takes into account multiple features of the function. As should be expected, our main focus will be on the \emph{interpolation of operators}.

\section{Convex Interpolation}

The most basic way to interpolate is using the notion of convexity. Given two inequalities $A_0 \leq B_0$ and $A_1 \leq B_1$, for any parameter $0 \leq \theta \leq 1$, if we define the additive weighted averages $A_\theta = (1 - \theta) A_0 + \theta A_1$ and $B_\theta = (1 - \theta) B_0 + \theta B_1$, then we conclude $A_\theta \leq B_\theta$ for all $\theta$. Similarily, we can consider the weighted multiplicative averages $A_\theta = A_0^{1 - \theta} A_1^\theta$ and $B_\theta = B_0^{1 - \theta}B_1^\theta$, in which case we still have $A_\theta \leq B_\theta$. Note that the additive averages are obtained by taking the unique linear function between two values, and the multiplicative averages are obtained by taking the unique log-linear function between two values. In particular, if $A_\theta$ is defined to be any convex function, then $A_\theta \leq (1 - \theta) A_0 + \theta A_1$, and if $B_\theta$ is logarithmically convex, so that $\log B_\theta$ is convex, then $B_\theta \leq B_0^{1 - \theta} B_1^\theta$. Thus convexity provides us with a more general way of interpolating estimates, which is what makes this property so useful in analysis, enabling us to simplify estimates.

\begin{example}
    For a fixed, measurable function $f$, the map $p \mapsto \| f \|_p$ is a log convex function. This statement is precisely H\"{o}lder's inequality, since the inequality
    %
    \[ \| f \|_{\theta p + (1 - \theta) q} \leq \| f \|_p^\theta \| f \|_{q}^{1-\theta} \]
    %
    says
    %
    \[ \| |f|^{\theta p} |f|^{(1 - \theta) q} \|_1^{1/(\theta p + (1 - \theta) q)} \leq \| f^{\theta p} \|_{1/\theta}^{\theta} \| f^{(1-\theta)q} \|_{1/(1-\theta)}^{1-\theta} \]
    %
    which is precisely H\"{o}lder's inequality. Note this implies that if $p_0 < p_\theta < p_1$, then $L^{p_0}(X) \cap L^{p_1}(X) \subset L^{p_\theta}(X)$.
\end{example}

\begin{example}
    The weak $L^p$ norm is log convex, because if $F(t) \leq A_0^{p_0}/t^{p_0}$, and $F(t) \leq A_1^{p_1}/t^{p_1}$, then we can apply scalar interpolation to conclude that if $p_\theta = (1 - \alpha) p_0 + \alpha p_1$,
    %
    \[ F(t) \leq \frac{A_0^{(1 - \alpha) p_0}A_1^{\alpha p_1}}{t^{(1 - \alpha)p_0 + \alpha p_1}} = \frac{A_\theta^{p_\theta}}{t^{p_\theta}} \]
    %
    where $p_\theta$ is the harmonic weighted average between $p_0$ and $p_1$, and $A_\theta$ the geometric weighted average. Using this argument, interpolating slightly to the left and right of $p_\theta$, we can conclude that if $p_0 < p_\theta < p_1$, then $L^{p_0,\infty}(X) \cap L^{p_1,\infty}(X) \subset L^{p_\theta}(X)$.
\end{example}

\section{Complex Interpolation}

Another major technique to perform an interpolation is to utilize the theory of complex analytic functions to obtain estimates. The core idea of this technique is to exploit the maximum principle, which says that bounding an analytic function at its boundary enables one to obtain bounds everywhere in the domain of the function. The next result, known as Lindel\"{o}f's theorem, is one of the fundamental examples of the application of complex analysis.

\begin{theorem}[The Three Lines Lemma]
    If $f$ is a holomorphic function on the strip $S = \{ z : \text{Re}(z) \in [a,b] \}$ and there exists constants $A,B,\delta > 0$ such that for all $z \in S$,
    %
    \[ |f(z)| \leq Ae^{Be^{(\pi - \delta)|z|}}. \]
    %
    Then the function $M: [a,b] \to [0,\infty]$ given by
    %
    \[ M(s) = \sup_{s \in \RR} |f(s + it)| \]
    %
    is log convex on $[a,b]$.
\end{theorem}
\begin{proof}
    By a change of variables, we can assume that $a = 0$, and $b = 1$, and we need only show that if there are $A_0, A_1 > 0$ such that
    %
    \[ |f(it)| \leq A_0 \quad\text{and}\quad |f(1 + it)| \leq A_1 \quad \text{for all $t \in \RR$}, \]
    %
    then for any $s \in [a,b]$ and $t \in \RR$,
    %
    \[ |f(s + it)| \leq A_0^{1 - s} A_1^s. \]
    %
    By replacing $f(z)$ with the function $A_0^{1-z} A_1^z f(z)$, we may assume without loss of generality that $A_0 = A_1 = 1$, and we must show that $\| f \|_{L^\infty(S)} \leq 1$. If $|f(s + it)| \to 0$ as $|t| \to \infty$, then for large $N$, we can conclude that $|f(s + it)| \leq 1$ for $s \in [a,b]$ and $|t| \geq N$. But then the maximum principle entails that $|f(s + it)| \leq 1$ for $s \in [a,b]$ and $|t| \leq N$, which completes the proof in this case. In the general case, for each $\varepsilon > 0$, define
    %
    \[ u_\varepsilon(z) = \exp(- 2 \varepsilon \sin((\pi - \varepsilon) z + \varepsilon/2)). \]
    %
    Then if $z = s + it$,
    %
    \[ |u_\varepsilon(z)| = \exp(- \varepsilon [e^{(\pi - \varepsilon) t} + e^{-(\pi - \varepsilon) t}] \sin((\pi - \varepsilon) s + \varepsilon/2)), \]
    %
    So, in particular, $|u_\varepsilon(z)| \leq 1$, and there exists a constant $C$ such that if $z \in S$,
    %
    \[ |u_\varepsilon(z)| \leq e^{- C \varepsilon^2 e^{(\pi - \varepsilon) |z|}} \]
    %
    Note that if $\varepsilon < \delta$, then as $|\text{Im}(z)| \to \infty$,
    %
    \[ |f(z) u_\varepsilon(z)| \leq A e^{B e^{(\pi - \delta) |z|} - C \varepsilon^2 e^{(\pi - \varepsilon) |z|} } \to 0. \]
    %
    Applying the previous case to the function $|f(z) u_\varepsilon(z)|$, we conclude that for any $\varepsilon > 0$,
    %
    \[ |f(z)| \leq \frac{1}{|u_\varepsilon(z)|}. \]
    %
    Thus
    %
    \[ |f(z)| \leq \lim_{\varepsilon \to 0} \frac{1}{|u_\varepsilon(z)|} = 1, \]
    %
    which completes the proof.
\end{proof}

\begin{remark}
    The function $e^{-ie^{\pi i s}}$ shows that the assumption of the three lines lemma is essentially tight. In particular, this means there is no family of holomorphic functions $g_\varepsilon$ which decays faster than double exponentially, and pointwise approximates the identity as $\varepsilon \to 0$.
\end{remark}

\begin{remark}
    Similar variants can be used to show that if $f$ is a holomorphic function on an annulus, then the supremum over circles centered around the origin is log convex in the radius of the circle (a result often referred to as the three circles lemma).
\end{remark}

\begin{example}
    Here we show how we can use the three lines lemma to prove that the $L^p$ norms are log convex. If $f = \sum a_n \chi_{E_n}$ is a simple function, then the function
    %
    \[ g(s) = \int |f|^s = \sum |a_n|^s |E_n| \]
    %
    is analytic in $s$, and satisfies the growth condition of the three lines lemma because each term of the sum is exponential in growth. Since $|g(s)| \leq |g(\sigma)|$, the three lines lemma implies that $g$ is log convex on the real line. By normalizing the function $f$ and the underlying measure, given $p_0$, $p_1$, we may assume $\| f \|_{p_0} = \| f \|_{p_1} = 1$, and it suffices to prove that $\| f \|_{p_\theta} \leq 1$ for all $p_\theta \in [p_0, p_1]$. But the log convexity of $g$ guarantees this is true, since $|g(p)| = \| f \|_p^p$. A standard limiting argument then gives the inequality for all functions $f$.
\end{example}

\begin{example}
    Let $f$ be a holmomorphic function on a strip $S = \{ z : \text{Re}(z) \in [a,b] \}$, such that if $z = a + it$, or $z = b + it$, for some $t \in \RR$,
    %
    \[ |f(z)| \leq C_1 (1 + |z|)^\alpha. \]
    %
    Then there exists a constant $C'$ such that for any $z \in S$,
    %
    \[ |f(z)| \leq C_2 (1 + |z|)^\alpha. \]
\end{example}
\begin{proof}
    The function
    %
    \[ g(z) = \frac{f(z)}{(1 + z)^\alpha} \]
    %
    is holomorphic on $S$, and if $z = a + it$ or $z = b + it$,
    %
    \[ |g(z)| \leq \frac{C_1 (1 + |z|)^\alpha}{|1 + z|^\alpha} \lesssim 1. \]
    %
    Thus the three lines lemma implies that $|g(z)| \lesssim 1$ for all $z \in S$, so
    %
    \[ |f(z)| \lesssim |1 + z|^\alpha \lesssim (1 + |z|)^\alpha. \qedhere \]
\end{proof}

\section{Interpolation of Operators}

A major part of modern harmonic analysis is the study of operators, i.e. maps from function spaces to other function spaces. We are primarily interested in studying \emph{linear operators}, i.e. operators $T$ such that $T(f + g) = T(f) + T(g)$, and $T(\alpha f) = \alpha T(f)$, and also \emph{sublinear operators}, such that $|T(\alpha f)| = |\alpha| |T(f)|$ and $|T(f + g)| \leq |Tf| + |Tg|$. Even if we focus on linear operators, it is still of interest to study sublinear operators because one can study the \emph{uniform boundedness} of a family of operators $\{ T_k \}$ by means of the function $T^*(f)(x) = \max (T_k f)(x)$. This is the method of \emph{maximal functions}. Another important example are the $l^p$ sums
%
\[ (S^p f)(x) = \left( \sum |T_k(x)|^p \right). \]
%
These two examples are specific examples where we have a family of operators $\{ T_y \}$, indexed by a measure space $Y$, and we define an operator $S$ by taking $Sf$ to be the norm of $\{ T_y f \}$ in the variable $y$.

Here we address the most basic case of operator interpolation. As we vary $p$, the $L^p$ norms provide different ways of measuring the height and width of functions. Let us consider a simple example. Suppose that for an operator $T$, we have a bound
%
\[ \| Tf \|_{L^1(Y)} \leq \| f \|_{L^1(X)} \quad\text{and}\quad \| Tf \|_{L^\infty(Y)} \leq \| f \|_{L^\infty(X)}. \]
%
The first inequality shows that the width of $Tf$ is controlled by the width of $f$, and the second inequality says the height of $Tf$ is controlled by the height of $f$. If we take a function $f \in L^p(X)$, for some $p \in (1,\infty)$, then we have some control over the height of $f$, and some control of the width. In particular, this means we might expect some control over the width and height of $Tf$, i.e. for each $p$, a bound
%
\[ \| Tf \|_{L^p(Y)} \leq \| f \|_{L^p(X)}. \]
%
This is the idea of interpolation on the $L^p(X)$ spaces.

\section{Complex Interpolation of Operators}

The first theorem we give is the Riesz-Thorin theorem, which utilizes complex interpolation to give such a result. In the next theorem, we work with a linear operator $T$ which maps simple functions $f$ on a measure space $X$ to functions on a measure space $Y$. For the purposes of applying duality, we make the mild assumption that for each simple function $g$,
%
\[ \int |(Tf)(y)| |g(y)|\; dy < \infty. \]
%
Our goal is to obtain $L^p$ bounds on the function $T$. The Hahn-Banach theorem then guarantees that $T$ has a unique extension to a map defined on all $L^p$ functions.

\begin{theorem}[Riesz-Thorin]
    Let $p_0,p_1 \in (0,\infty]$ and $q_0,q_1 \in [1,\infty]$. Suppose that
    %
    \[ \| Tf \|_{L^{q_0}(Y)} \leq A_0 \| f \|_{L^{p_0}(X)} \quad \text{and} \| Tf \|_{L^{q_1}(Y)} \leq A_1 \| f \|_{L^{p_1}(X)}.  \]
    %
    Then for any $\theta \in (0,1)$, if
    %
    \[ 1/p_\theta = (1 - \theta)/p_0 + \theta/p_1 \quad\text{and}\quad 1/q_\theta = (1 - \theta)/q_0 + \theta/q_1, \]
    %
    then
    %
    \[ \| Tf \|_{L^{q_\theta}(Y)} \leq A_\theta \| f \|_{L^{p_\theta}(X)}, \]
    %
    where $A_\theta = A_0^{1 - \theta} A_1^\theta$.
\end{theorem}
\begin{proof}
    If $p_0 = p_1$, the proof follows by the log convexity of the $L^p$ norms of a function. Thus we may assume $p_0 \neq p_1$, so $p_\theta$ is finite in any case of interest. By normalizing the measures on both spaces, we may assume $A_0 = A_1 = 1$. By duality and homogeneity, it suffices to show that for any two simple functions $f$ and $g$ such that $\| f \|_{q_\theta} = \| g \|_{q_\theta^*} = 1$,
    %
    \[ \left| \int_Y (Tf) g\; dy \right| \leq 1. \]
    %
    Our challenge is to make this inequality complex analytic so we can apply the three lines lemma. We write $f = F_0^{1 - \theta} F_1^\theta a$, where $F_0$ and $F_1$ are non-negative simple functions with $\| F_0 \|_{L^{p_0}(X)} = \| F_1 \|_{L^{p_1}(X)} = 1$, and $a$ is a simple function with $|a(x)| = 1$. Similarily, we can write $g = G_0^{1-\theta} G_1^\theta b$. We now write
    %
    \[ H(s) = \int_Y T(F_0^{1 - s} F_1^s a) G_0^{1-s} G_1^s b\; dy. \]
    %
    Since all functions involved here are simple, $H(s)$ is a linear combination of positive numbers taken to the power of $1-s$ or $s$, and is therefore obviously an entire function in $s$. Now for all $t \in \RR$, we have
    %
    \[ \| F_0^{1-it} F_1^{it} a \|_{L^{p_0}(X)} = \| F_0 \|_{L^{p_0}(X)} = 1, \]
    \[ \| G_0^{1-it} G_1^{it} b \|_{L^{q_0}(Y)} = \| G_0 \|_{L^{q_0}(X)} = 1. \]
    %
    Therefore
    %
    \begin{align*}
      |H(it)| &= \left| \int T(F_0^{1 - it} F_1^{it} a) G_0^{1-it} G_1^{it} b\; dy \right| \leq 1.
    \end{align*}
    %
    Similarily, $|H(1 + it)| \leq 1$ for all $t \in \RR$. An application of Lindel\"{o}f's theorem implies $|H(s)| \leq 1$ for all $s$. Setting $s = \theta$ completes the argument.
\end{proof}

If, for each $p,q$, we let $F(1/p,1/q)$ to be the operator norm of a linear operator $T$ viewed as a map from $L^p(X)$ to $L^q(Y)$, then the Riesz-Thorin theorem says that $F$ is a log-convex function. In particular, the set of $(1/p,1/q)$ such that $T$ is bounded as a map from $L^p(X)$ to $L^q(Y)$ forms a convex set. If this is true, we often say $T$ is of \emph{strong type} $(p,q)$.

\begin{example}
  For any two integrable functions $f,g \in L^1(\RR^d)$, we can define an integrable function $f * g \in L^1(\RR^d)$ almost everywhere by the integral formula
  %
  \[ (f * g)(x) = \int f(y) g(x-y)\; dy. \]
  %
  If $f \in L^1(\RR^d)$ and $g \in L^p(\RR^d) \cap L^1(\RR^d)$, for some $p \geq 1$, then Minkowski's integral inequality implies
  %
  \begin{align*}
      \| f * g \|_p &= \left( \int |(f * g)(x)|^p\; dx \right)^{1/p} \leq \int \left( \int |f(y)g(x-y)|^p dx\; \right)^{1/p} dy\\
      &= \int |f(y)| \| g \|_{L^p(\RR^d)} = \| f \|_{L^1(\RR^d)} \| g \|_{L^p(\RR^d)}.
  \end{align*}
  %
  H\"{o}lder's inequality implies that if $f \in L^p(\RR^d)$ and $g \in L^q(\RR^d)$, where $p$ and $q$ are conjugates of one another, then
  %
  \begin{align*}
    \left| \int f(y) g(x-y)\; dy \right| \leq \int |f(y-x)| |g(x)| \leq \| f \|_{L^p(\RR^d)} \| g \|_{L^q(\RR^d)}.
  \end{align*}
    %
    Thus we have the bound
    %
    \[ \| f * g \|_{L^\infty(\RR^d)} \leq \| f \|_{L^p(\RR^d)} \| g \|_{L^q(\RR^d)}. \]
    %
    Now that these mostly trivial results have been proved, we can apply convolution. For each $f \in L^1(\RR^d) \cap L^p(\RR^d)$, we have a convolution operator $T: L^1(\RR^d) \to L^1(\RR^d)$ defined by $Tg = f * g$. We know that $T$ is of strong type $(1,p)$, and of type $(q,\infty)$, where $q$ is the harmonic conjugate of $p$, and $T$ has operator norm $1$ with respect to each of these types. But the Riesz Thorin theorem then implies that if $1/r = \theta + (1 - \theta)/q$, then $T$ is bounded as a map from $L^r(\RR^d)$ to $L^{p/\theta}(\RR^d)$ with operator norm one. Reparameterizing gives \emph{Young's convolution inequality}. Note that we never really used anything about $\RR^d$ here other than it's translational structure, and as such Young's inequality continues to apply in the theory of any modular locally compact group. In particular, the Haar measure $\mu$ on such a group is only defined up to a scalar multiple, and if we swap $\mu$ with $\alpha \mu$, for some $\alpha > 0$, then Young's inequality for this measure implies
    %
    \[ \lambda^{1 + 1/r} \| f * g \|_r = \lambda^{1/p + 1/q} \| f \|_p \| g \|_p \]
    %
    which is a good way of remembering that we must have $1 + 1/r = 1/p + 1/q$.
\end{example}

\begin{example}
Let $X$ be a measure space with $\sigma$ algebra $\Sigma_0$, and let $\Sigma \subset \Sigma_0$ be a $\sigma$ finite sub $\sigma$ algebra. Then $L^2(X,\Sigma)$ is a closed subspace of $L^2(X,\Sigma_0)$, and so there is an orthogonal projection operator $\EE(\cdot|\Sigma): L^2(X,\Sigma_0) \to L^2(X,\Sigma)$, which we call the \emph{conditional expectation operator}. The properties of the projection operator imply that for any $f,g \in L^2(X, \Sigma_0)$,
%
\[ \int \EE(f|\Sigma) \overline{g} = \int f \overline{g} = \int \EE(f|\Sigma) \overline{\EE(g|\Sigma)}. \]
%
If $g \in L^2(X,\Sigma)$, then
%
\[ \int \EE(f|\Sigma) \overline{g} = \int f \overline{g}. \]
%
This gives a full description of $\EE(f|\Sigma)$. In particular, if $u \in L^\infty(X,\Sigma_0)$, then for each $g \in L^2(X,\Sigma)$
%
\[ \int \EE(uf|\Sigma) \overline{g} = \int f [u\overline{g}] = \int u \EE(f|\Sigma) \overline{g}. \]
%
Since this is true for all $g \in L^2(X,\Sigma)$, we find $\EE(uf|\Sigma) = u \EE(f|\Sigma)$. Moreover, if $0 \leq f \leq g$, then $\EE(f|\Sigma) \leq \EE(g|\Sigma)$. This is easy to see because if $f \geq 0$, and $F = \{ x : \EE(f|\Sigma) < 0 \}$, then if $|F| \neq 0$,
%
\[ 0 > \int \EE(f|\Sigma) \mathbf{I}_F = \int f \mathbf{I}_F \geq 0. \]
%
Thus $|F| = 0$, and so $\EE(f|\Sigma) \geq 0$ almost everywhere.

Like all other orthogonal projection operators, conditional expectation is a contraction in the $L^2$ norm, i.e. $\| \mathbf{E}(f|\Sigma) \|_{L^2(X)} \leq \| f \|_{L^2(X)}$. We now use interpolation to show that conditional expectation is strong $(p,p)$, for all $1 \leq p \leq \infty$. It suffices to prove the operator is strong $(1,1)$ and strong $(\infty,\infty)$. So suppose $f \in L^2(X,\Sigma_0) \cap L^\infty(X,\Sigma_0)$. If $|E| < \infty$, then $\mathbf{I}_E \in L^2(X)$, so
%
\[ |\EE(f|\Sigma)| \mathbf{I}_E = |\EE(\mathbf{I}_E f | \Sigma)| \leq \EE(\mathbf{I}_E |f| | \Sigma) \leq \| f \|_\infty \mathbf{E}(\mathbf{I}_E|\Sigma) = \| f \|_\infty \mathbf{I}_E. \]
%
Since $\Sigma$ is a sigma finite sigma algebra, we can take $E \to \infty$ to conclude $\| \EE(f|\Sigma) \|_\infty \leq \| f \|_\infty$. The case $(1,1)$ can be obtained by duality, since conditional expectation is self adjoint, or directly, since if $f \in L^1(X,\Sigma_0) \cap L^2(X,\Sigma_0)$, then for any set $E \in \Sigma$ with $|E| < \infty$,
%
\[ \int |\EE(f|\Sigma)| \mathbf{I}_E \leq \int \EE(|f||\Sigma) \mathbf{I}_E = \int_E |f| \mathbf{I}_E \leq \| f \|_1. \]
%
Since $\Sigma$ is $\sigma$ finite, we can take $E \to \infty$ to conclude $\| \EE(f|\Sigma) \|_1 \leq \| f \|_1$. Thus the Riesz interpolation theorem implies that for each $1 \leq p \leq \infty$, $\| \EE(f|\Sigma) \|_p \leq \| f \|_p$.

Since $L^2(X,\Sigma_0)$ is dense in $L^p(X,\Sigma_0)$ for all $1 \leq p < \infty$, there is a unique extension of the conditional expectation operator from $L^p(X,\Sigma_0)$ to $L^p(X,\Sigma_0)$. For $p = \infty$, there are infinitely many extensions of the conditional expectation operator from $L^\infty(X,\Sigma_0)$ to $L^\infty(X,\Sigma_0)$. However, there is a \emph{unique} extension such that for each $f \in L^2(\Sigma_0)$ and $g \in L^\infty(\Sigma)$, $\EE(fg|\Sigma) = g \EE(f|\Sigma)$. This is because for any $E \in \Sigma$ with $|E| < \infty$, $\EE(f \mathbf{I}_E | \Sigma) = \mathbf{I}_E \EE(f|\Sigma)$ is uniquely defined since $f \mathbf{I}_E \in L^2(\Sigma_0)$, and taking $E \to \infty$ by $\sigma$ finiteness.

A simple consequence of the uniform boundedness of these operators on the various $L^p$ spaces is that if $\Sigma_1, \Sigma_2, \dots$ are a family of $\sigma$ algebras, and $\Sigma_\infty$ is the smallest $\sigma$ algebra containing all sets in $\bigcup_{i = 1}^\infty \Sigma_i$, then for each $1 \leq p < \infty$, and for each $f \in L^p(\Sigma_0)$, $\lim_{i \to \infty} \EE(f|\Sigma_i) = \EE(f|\Sigma_\infty)$. This is because the operators $\{ \EE(\cdot|\Sigma_i) \}$ are uniformly bounded. The limit equation holds for any simple function $f$ composed of sets in $\bigcup_{i = 1}^\infty \Sigma_i$, and a $\sigma$ algebra argument can then be used to show this family of simple functions is dense in $L^p(\Sigma_0)$.
\end{example}

It was an important observation of Elias-Stein that complex interpolation can be used not only with a single operator $T$, but with an `analytic family' of operators $\{ T_s \}$, one for each $s$, such that for each pair of simple functions $f$ and $g$, the function
%
\[ \int (T_s f)(y) g(y) \]
%
is analytic in $s$. Thus bounds on $T_{0+it}$ and $T_{1 + it}$ imply intermediary bounds on all other operators, provided that we still have at most doubly exponential growth. The next theorem gives an example application.

\begin{theorem}[Stein-Weiss Interpolation Theorem]
  Let $T$ be a linear operator, and let $w_0, w_1: X \to [0,\infty)$ and $v_0, v_1 : Y \to [0,\infty)$ be weights which are integrable on every finite-measure set. Suppose that
  %
  \[ \| Tf \|_{L^{q_0}(X,v_0)} \leq A_0 \| f \|_{L^{p_0}(X,w_0)}\quad\text{and}\quad \| Tf \|_{L^{q_1}(X,v_1)} \leq A_1 \| f \|_{L^{p_1}(X,w_0)}. \]
  %
  Then for any $\theta \in (0,1)$,
  %
  \[ \| Tf \|_{L^{q_\theta}(X,v_\theta)} \leq A_\theta \| f \|_{L^{p_\theta}(X,w_\theta)}, \]
  %
  where $w_\theta = w_0^{1-\theta} w_\theta$ and $v_\theta = v_0^{1-\theta} v_1^\theta$.
\end{theorem}
\begin{proof}
  Fix a simple function $f$ with $\| f \|_{L^{p_\theta}(X,w_\theta)}$. We begin with some simplifying assumptions. A monotone convergence argument, replacing $w_i(t)$ with
  %
  \[ w_i'(y) = \begin{cases} w_i(y) &: \varepsilon \leq w_i(t) \leq 1/\varepsilon, \\ 0 &: \text{otherwise}, \end{cases} \]
  %
  and then taking $\varepsilon \to 0$, enables us to assume without loss of generality that $w_0$ and $w_1$ are both bounded from below and bounded from above. Truncating the support of $Tf$ enables us to assume that $Y$ has finite measure. Since $f$ has finite support, we may also assume without loss of generality that $X$ has finite support, and by applying the dominated convergence theorem we may replace the weights $v_i$ with
  %
  \[ v_i'(x) = \begin{cases} v_i(x) &: \varepsilon \leq v_i(x) \leq 1/\varepsilon, \\ 0 &: \text{otherwise}, \end{cases} \]
  %
  and then take $\varepsilon \to 0$. Thus we can assume that the $v_i$ are bounded from above and below. Restricting to the support of $X$, we can also assume $X$ has finite measure.

  For each $s$, consider the operator $T_s$ defined by
  %
  \[ T_s f = w_0^{\frac{1-s}{q_0}} w_1^{\frac{s}{q_1}} T \left( f v_0^{- \frac{1-s}{p_0}} v_1^{-\frac{s}{p_1}} \right). \]
  %
  The fact that all functions involved are simple means that the family of operators $\{ T_s \}$ is analytic. Now for all $t \in \RR$
  %
  \[ \| T_{it} f \|_{L^{q_0}(Y)} = \| T f \|_{L^{q_0}(Y,w_0)} \leq A_0 \| f v_0^{-1/p_0} \|_{L^{p_0}(X,v_0)} = A_0 \| f \|_{L^{p_0}(X)}. \]
  %
  For similar reasons, $\| T_{1 + it} f \|_{L^{q_1}(Y)} \leq A_1 \| f \|_{L^{p_0}(X,v_0)}$. Thus the Stein variant of the Riesz-Thorin theorem implies that
  %
  \[ \| T_\theta f \|_{L^{q_\theta}(Y)} \leq A_\theta \| f \|_{L^{p_\theta}(X)}. \]
  %
  But this, of course, is equivalent to the bound we set out to prove.
\end{proof}

\section{Real Interpolation of Operators}

Now we consider the case of real interpolation. One advantage of real interpolation is that it can be applied to sublinear as well as linear operators, and requires weaker endpoint estimates that the complex case. A disadvantage is that, usually, the operator under study cannot vary, and we lose out on obtaining explicit bounds.

A strong advantage to using real interpolation is that the criteria for showing boundedness at the endpoints can be reduced considerably. Let us give names for the boundedness we will want to understand for a particular operator $T$.
%
\begin{itemize}
  \item We say $T$ is \emph{strong type} $(p,q)$ if $\| Tf \|_{L^q(Y)} \lesssim \| f \|_{L^p(X)}$.

  \item We say $T$ is \emph{weak type} $(p,q)$ if $\| Tf \|_{L^{q,\infty}(Y)} \lesssim \| f \|_{L^p(X)}$.

  \item We say $T$ is \emph{restricted strong type} $(p,q)$ if we have a bound
  %
  \[ \| Tf \|_{L^q(Y)} \lesssim HW^{1/p} \]
  %
  for any sub-step functions with height $H$ and width $W$. Equivalently, for any set $E$,
  %
  \[ \| T(\mathbf{I}_E) \|_{L^q(Y)} \lesssim |E|^{1/p}. \]

  \item We say $T$ is \emph{restricted weak type} $(p,q)$ if we have a bound
  %
  \[ \| Tf \|_{L^{q,\infty}(Y)} \lesssim HW^{1/p} \]
  %
  for all sub-step functions with height $H$ and width $W$. Equivalently, for any set $E$,
  %
  \[ \| T(\mathbf{I}_E) \|_{L^{q,\infty}(Y)} \lesssim |E|^{1/p}. \]
\end{itemize}
%
An important tool for us will be to utilize duality to make our interpolation argument `bilinear'. Let us summarize this tool in a lemma. Proving the lemma is a simple application of Theorem \ref{weakdualitytheorem}.

\begin{lemma}
  Let $0 < p < \infty$ and $0 < q < \infty$. Then an operator $T$ is restricted weak-type $(p,q)$ if and only if for any finite measure sets $E \subset X$ and $F \subset Y$, there is $F' \subset Y$ with $|F'| \geq \alpha |F|$ such that
  %
  \[ \int_{F'} |T(\mathbf{I}_E)| \lesssim_\alpha |E|^{1/p} |F|^{1-1/q}. \]
\end{lemma}

Scalar interpoation leads to a simple version of real interpolation, which we employ as a subroutine to obtain a much more powerful real interpolation principle.

\begin{lemma}
  Let $0 < p_0,p_1 < \infty$, $0 < q_0,q_1 < \infty$. If $T$ is restricted weak type $(p_0,q_0)$ and $(p_1,q_1)$, then $T$ is restricted weak type $(p_\theta,q_\theta)$ for all $\theta \in (0,1)$.
\end{lemma}
\begin{proof}
  By assumption, if $E \subset X$ and $F \subset Y$, then there is $F_0, F_1 \subset Y$ with $|F_i| \geq (3/4)|F|$ such that
  %
  \[ \int_{F_i} |T(\mathbf{I}_E)| \lesssim |E|^{1/p_i} |F_i|^{1 - 1/q_i}. \]
  %
  If we let $F_\theta = F_0 \cap F_1$, then $|F_\theta| \geq |F|/2$, and for each $i$,
  %
  \[ \int_{F_\theta} |T(\mathbf{I}_E)| \lesssim |E|^{1/p_i} |F_\theta|^{1 - 1/q_i}. \]
  %
  Scalar interpolation implies
  %
  \[ \int_{F_\theta} |T(\mathbf{I}_E)| \lesssim |E|^{1/p_\theta} |F_\theta|^{1 - 1/q_\theta}, \]
  %
  and thus we have shown
  %
  \[ \| T(\mathbf{I}_E) \|_{q_\theta,\infty} \lesssim |E|^{1/p_\theta}. \]
  %
  This is sufficient to show $T$ is restricted weak type $(p_\theta,q_\theta)$.
\end{proof}

\begin{theorem}[Marcinkiewicz Interpolation Theorem]
  Let $0 < p_0,p_1 < \infty$, $0 < q_0,q_1 < \infty$, and suppose $T$ is restricted weak type $(p_i,q_i)$, with constant $A_i$, for each $i$. Then, for any $\theta \in (0,1)$, if $q_\theta > 1$, then for any $0 < r < \infty$, then
  %
  \[ \| Tf \|_{L^{q_\theta,r}(Y)} \lesssim A_\theta \| f \|_{L^{p_\theta,r}(X)}, \]
  %
  with implicit constants depending on $p_0, p_1, q_0$, and $q_1$.
\end{theorem}
\begin{proof}
  By scaling $T$, and the measures on $X$ and $Y$, we may assume that $\| f \|_{L^{p_\theta,r}(X)} \leq 1$, and that $T$ is restricted type $(p_i,q_i)$ with constant $1$, so that for any step function $f$ with height $H$ and width $W$,
  %
  \[ \| Tf \|_{L^{q_i,\infty}(Y)} \leq \| f \|_{L^{p_i}(X)}. \]
  %
  By duality, using the fact that $q_\theta > 1$, it suffices to show that for any simple function $g$ with $\| g \|_{L^{q_\theta',r'}(Y)} = 1$,
  %
  \[ \int |Tf| |g| \leq 1. \]
  %
  Using the previous lemma, we can `adjust' the values $q_0,q_1$ so that we can assume $q_0,q_1 > 1$. We can perform a horizontal layer decomposition, writing
  %
  \[ f = \sum_{i = -\infty}^\infty f_i, \quad\text{and}\quad g = \sum_{i = -\infty}^\infty g_i, \]
  %
  where $f_i$ and $g_i$ are sub-step functions with width $2^i$ and heights $H_i$ and $H_i'$ respectively, and if we write $A_i = H_i 2^{i/p_\theta}$, and $B_i = H_i' 2^{i/q_\theta}$, then
  %
  \[ \| A \|_{l^r(\ZZ)}, \| B \|_{l^{r'}(\ZZ)} \lesssim 1. \]
  %
  Applying the restricted weak type inequalities, we know for each $i$ and $j$,
  %
  \[ \int |Tf_i| |g_j| \lesssim H_i H_j \min_{k \in \{0,1\}} \left[ 2^{i/p_k + j(1 - 1/q_k)} \right]. \]

  Applying sublinearity (noting that really, the decomposition of $f$ and $g$ is finite, since both functions are simple). Thus
  %
  \begin{align*}
    \int |Tf| |g| &\leq \sum_{i,j} \int |Tf_i| |g_j|\\
    &\lesssim \sum_{i,j} H_i H_j' \min_{k \in \{0,1\}} \left[ 2^{i/p_k + j(1 - 1/q_k)} \right]\\
    &\lesssim \sum_{i,j} A_i B_j \min_{k \in \{ 0, 1 \}} \left[ 2^{i(1/p_k - 1/p_\theta) + j(1/q_\theta - 1/q_k)} \right].
  \end{align*}
  %
  If $i(1/p_k - 1/p_\theta) + j(1/q_\theta - 1/q_k) = \varepsilon(i + \lambda j)$, where $\varepsilon = (1/p_k - 1/p_\theta)$. We then have
  %
  \[ \sum_{i,j} A_i B_j \min_{k \in \{ 0, 1 \}} \left[ 2^{i(1/p_k - 1/p_\theta) + j(1/q_\theta - 1/q_k)} \right] \sim \sum_{k = -\infty}^\infty \min(2^{\varepsilon k}, 2^{-\varepsilon k}) \sum_i A_i B_{k - \lfloor i/\lambda \rfloor}. \]
  %
  Applying H\"{o}lder's inequality,
  %
  \begin{align*}
    \sum_i A_i B_{k - \lfloor i/\lambda \rfloor} &\leq \| A \|_{l^r(\ZZ)} \left( \sum_i |B_{k - \lfloor i/\lambda \rfloor}|^{r'} \right)^{1/r'}\\
    &\lesssim \lambda^{1/r'} \| A \|_{l^r(\ZZ)} \| B \|_{l^{r'}(\ZZ)} \lesssim 1.
  \end{align*}
  %
  Thus we conclude that
  %
  \begin{align*}
    \sum_{k = -\infty}^\infty \min(2^{\varepsilon k}, 2^{-\varepsilon k}) \sum_i A_i B_{k - \lfloor i/\lambda \rfloor} &\lesssim \sum_{k = -\infty}^\infty \min(2^{\varepsilon k}, 2^{-\varepsilon k}) \lesssim_\varepsilon 1. \qedhere
  \end{align*}
\end{proof}

There are many variants of the real interpolation method, but the general technique almost always remains the same: incorporate duality, decompose inputs, often dyadically, bound these decompositions, and then sum up.










\chapter{Differentiation and Averages}

This chapter is about exploring the behaviour of basic averaging operators. A classical example, given a function $f \in L^1_{\text{loc}}(\RR)$, are the averaging operators
%
\[ A_\delta f(x) = \frac{1}{2\delta} \int_{x-\delta}^{x+\delta} f(y)\; dy. \]
%
If $f \in C(\RR)$, then for each $x \in \RR$, $\lim_{\delta \to 0} A_\delta f(x) = f(x)$. This fact is fundamentally connected to differentiation under the integral sign; if we define the function
%
\[ F(x) = \int_0^x f(y)\; dy \]
%
then for each $x \in \RR$,
%
\[ F'(x) = \lim_{h \to 0} \frac{F(x+h) - F(x)}{h} = \lim_{h \to 0} \frac{1}{h} \int_x^{x+h} f(y)\; dy = f(x). \]
%
Our main goal will be study whether pointwise convergence of the averages $A_\delta f$ hold for a more general family of functions or equivalently, studying whether a kind of fundamental theorem of calculus holds for a more general family of measurable functions, which are not necessarily continuous.


The classical family of averaging operators are defined for $\delta > 0$, $f \in L^1_{\text{loc}}(\RR^d)$, and $x \in \RR^d$ by setting
%
\[ A_\delta f(x) = \frac{1}{|B(x,\delta)|} \int_{B(x,\delta)} f(y)\; dy, \]
%
where $B(x,\delta)$ is the ball of radius $\delta$ centred at $x$. A simple application of Schur's lemma shows that $\| A_\delta f \|_{L^p(\RR^d)} \leq \| f \|_{L^p(\RR^d)}$ for all $1 \leq p \leq \infty$, uniformly in $p$. This uniform bound in $\delta$ is strong enough, together with the density of compactly supported continuous functions is enough to conclude that for any $f \in L^p(\RR^d)$, for $1 \leq p < \infty$, $A_\delta f$ converges to $f$ in $L^p$ norm. This implies that for any $f \in L^p(\RR^d)$, there exists a sequence $\delta_i$ converging to zero such that $A_{\delta_i} f$ converges to $f$ pointwise almost everywhere. In this chapter, we would like to show $A_\delta f$ converges to $f$ pointwise almost everywhere \emph{without taking a subsequence of values $\delta_i$}.

Hardy and Littlewood introduced a powerful technique to study such pointwise convergence problems, known as the \emph{method of maximal functions}. For each $f \in L^1_{\text{loc}}(X)$, we define
%
\[ Mf(x) = \sup_{\delta > 0} A_\delta |f|(x) = \sup_{\delta > 0} \frac{1}{|B(x,\delta)|} \int_{B(x,\delta)} |f(y)|\; dy. \]
%
The next theorem indicates why obtaining bounds on a maximal operator gives pointwise convergence results.

\begin{theorem}
  Let $V$ be a quasinorm space, let $0 < q < \infty$, and consider a family of bounded operators $T_t: V \to L^{q,\infty}(X)$. Then we can define the pointwise maximal operator
  %
  \[ T_* f(x) = \sup_t |T_t f(x)|. \]
  %
  Suppose that for every $f \in L^p(X)$,
  %
  \[ \| T_* f \|_{L^{q,\infty}(X)} \lesssim \| f \|_V. \]
  %
  Then for any bounded operator $S: V \to L^{q,\infty}(X)$,
  %
  \[ \{ x \in V : \lim_{t \to \infty} T_t x(y) = Sx(y)\; \text{for a.e $y$} \} \]
  %
  is closed in $V$.
\end{theorem}
\begin{proof}
  Fix a sequence $\{ u_n \}$ in $V$ converging to $u \in V$, and suppose for each $n$,
  %
  \[ \lim_{t \to \infty} (T_t u_n)(x) = Su_n(x) \]
  %
  holds for almost every $x \in X$. For each $\lambda > 0$, we find
  %
  \begin{align*}
    |\{ x \in X: &\limsup_{t \to \infty} |T_t u(x) - Su(x)| > \lambda \}|\\
    &\leq |\{ x \in X: \limsup_t |T_t(u - u_n)(x) - S(u - u_n)(x)| > \lambda \}|\\
    &\leq |\{ x \in X : |T_*(u - u_n)(x)| > \lambda/2 \}| + | \{ x: |S(u - u_n)(x)| > \lambda/2 \} |\\
    &\lesssim_{p,q} \frac{\| u - u_n \|_V^q}{\lambda^q} + \frac{\| u - u_n \|_V^p}{\lambda^p}.
  \end{align*}
  %
  as $n \to \infty$, this quantity tends to zero. Thus for all $\lambda > 0$,
  %
  \[ |\{ x: \limsup_{t \to \infty} |T_t u(x) - Su(x)| > \lambda \}| = 0 \]
  %
  Taking $\lambda \to 0$ gives that $\limsup_t |T_t u(x) - Su(x)| = 0$ for almost every $x \in X$. But this means precisely that $T_tu(x) \to Su(x)$ for almost every $x \in X$.
\end{proof}

Taking $t = \delta$, $T_t = A_\delta$, and $S$ the identity map, the theorem above implies that one way to obtain almost everywhere convergence for the averages we consider is via bounding the maximal operator $M$. Thus we consider a bound of the form
%
\[ \left\| \sup_{\delta > 0} A_\delta f \right\|_{L^{q,\infty}(\RR^d)} \lesssim \| f \|_V \]
%
for an appropriate norm $\| \cdot \|_V$ and $0 < q < \infty$. We have already obtained a bound
%
\[ \sup_{\delta > 0} \| A_\delta f \|_{L^{q,\infty}(\RR^d)} \leq \sup_{\delta > 0} \| A_\delta f \|_{L^q(\RR^d)} \leq \| f \|_{L^q(\RR^d)} \]
%
but moving the supremum inside the $L^q$ norm is nontrivial. One way to think about the difference between the two bounds is that the latter uniformly controls the height and width of the functions $A_\delta f$, whereas the former inequality shows that the main contribution to the height and widths of the functions $A_\delta f$ are uniformly supported in similar regions of space.

\section{Covering Methods}

The bound $\| Mf \|_{L^\infty(\RR^d)} \leq \| f \|_{L^\infty(\RR^d)}$ from a direct calculation. Thus there are trivial techniques of bounding the height of the function $Mf$ in terms of the height of the function $f$. The difficult part is obtaining control of the width of $Mf$ in terms of the width of $f$. This can only be obtained up to a certain degree, because unless $f = 0$, $Mf$ is non-vanishing on the entirety of $\RR^d$ so the width of $f$ `explodes'. A slightly more technical calculation shows that we cannot even have a bound of the form $\| Mf \|_{L^1(\RR^d)} \lesssim \| f \|_{L^1(\RR^d)}$. In fact, $\| Mf \|_{L^1(\RR^d)} = \infty$ for any nonzero $f \in L^1(\RR^d)$.

\begin{example}
  Fix $f \in L^1(\RR^d)$. By rescaling, we may assume without loss of generality that $\| f \|_{L^1(\RR^d)} = 2$. Then, for suitably large $R \geq 1$,
  %
  \[ \int_{B_R(0)} |f(x)|\; dx \geq 1. \]
  %
  For each $x \in \mathbf{R}^d$, $B_R(0) \subset B_{|x|+R}(x)$ and so
  %
  \[ Mf(x) \geq \fint_{B_{|x|+R}(x)} |f(y)|\; dy \gtrsim \frac{1}{(|x| + R)^d} \gtrsim \frac{1}{|x|^d} \]
  %
  But this means that
  %
  \[ \int_{\RR^d} |Mf(x)| \gtrsim \int_{\RR^d} \frac{1}{|x|^d} = \infty. \]
  %
  If we are more careful, we can even find examples of $f \in L^1(\RR^d)$ such that $Mf$ is not even locally integrable. If $f(x) = 1/|x| \log|x|^2$, then the fact that for $x \geq 0$
  %
  \begin{align*}
    \frac{1}{2h} \int_{x-h}^{x+h} \frac{dy}{|y| \log |y|^2} &= \frac{1}{2h} \left( \frac{1}{\log(x-h)} - \frac{1}{\log(x+h)} \right)\\
    &= \frac{1}{2x \log x} + O \left( \frac{h}{\log x} \right)
  \end{align*}
  %
  implies that
  %
  \[ Mf(x) \geq \frac{1}{2x \log x}. \]
  %
  Thus $Mf$ isn't integrable about the origin. Note however, that $Mf$ is on the \emph{border} of integrability, which hints at the fact that we have a weak type $(1,1)$ bound.
\end{example}

The last example shows that $|Mf(x)| \gtrsim |x|^{-d}$. Note, however, that $|x|^{-d}$ is only \emph{barely} nonintegrable. We will also show that $Mf$ is barely nonintegrable by obtaining a bound
%
\[ \| Mf \|_{L^{1,\infty}(\RR^d)} \lesssim_d \| f \|_{L^1(\RR^d)}. \]
%
Interpolation thus shows that $\| Mf \|_{L^p(\RR^d)} \lesssim_{d,p} \| f \|_{L^p(\RR^d)}$ for all $1 < p \leq \infty$. The standard real-variable technique of obtaining this bound is geometric, applying a covering argument. To obtain the weak-type bound, we must show that the set
%
\[ E_\lambda = \{ x \in \RR^d : |Mf(x)| > \lambda \} \]
%
is small. If $|Mf(x)| > \lambda$, there is a ball $B$ around $x$ such that
%
\[ \int_B |f(y)|\; dy > \lambda |B|. \]
%
Clearly $B \subset E_\lambda$. If we could find a large family of \emph{disjoint balls} $B_1,\dots,B_N$ such that this inequality held, such that $\sum |B_i| \gtrsim_d |E_\lambda|$, then we would conclude that
%
\[ \| f \|_{L^1(\RR^d)} \geq \sum_{i = 1}^N \int_{B_i} |f(y)|\; dy > \lambda \sum_{i = 1}^N |B_i| \gtrsim_d \lambda |E_\lambda| \]
%
which would show $|E_\lambda| \lesssim_d \| f \|_{L^1(\RR^d)} / \lambda$, which would show $\| Mf \|_{L^{1,\infty}(\RR^d)} \lesssim_d \| f \|_{L^1(\RR^d)}$. This intuition is true, and the process through which we obtain the family of disjoint balls $B_1,\dots,B_N$ is through the \emph{Vitali covering lemma}.

This particular technique has been shown to generalize to a wide variety of situations including the maximal ball average. All that is really required for the basic theory is a basic `covering type argument' that holds in a great many situations. In particular, we can generalize this argument to a \emph{space of homogenous type}. We consider a locally compact topological space $X$ together with a nonzero Radon measure. For each $x \in X$ and $\delta > 0$, we fix an open, precompact set $B(x,\delta)$, which we assume to be monotonically increasing in $\delta$. The fundamental property we require of these sets is that there is $c > 0$ such that for any $x \in X$ and $\delta > 0$, if we set
%
\[ B^*(x,\delta) = \bigcup \{ B(x',\delta): B(x,\delta) \cap B(x',\delta) \neq \emptyset \}, \]
%
then $|B^*(x,\delta)| \leq c |B(x,\delta)|$. In the case of balls in $\RR^d$, $B^*(x,\delta) \subset B^*(x,3\delta)$, and so $|B^*(x,\delta)| \leq 3^d |B(x,\delta)|$, so $c = 3^d$. More generally, if we are working in any metric space $X$, where $B(x,\delta)$ are the balls of radius $\delta$ in this metric space, and our measure satisfies a \emph{doubling condition}
%
\[ |B(x,3\delta)| \lesssim |B(x,\delta)| \]
%
for all $x \in X$ and $\delta > 0$, then our assumption holds. We also assume the following two technical assumptions
%
\begin{itemize}
  \item For any $x \in X$,
  %
  \[ \bigcap_{\delta > 0} \overline{B}(x,\delta) = \{ x \} \quad\text{and}\quad \bigcup_{\delta > 0} B(x,\delta) = X \]

  \item For any open set $U \subset X$ and $\delta > 0$, the function
  %
  \[ x \mapsto |B(x,\delta) \cap U| \]
  %
  is a continuous function of $x$.
\end{itemize}
%
These are fairly easily verifiable in any particular instance. It follows from these technical assumptions that $|B(x,\delta)| > 0$ for each $x \in X$ and $\delta > 0$, and moreover, for each $\delta > 0$, and $f \in L^1_{\text{loc}}(X)$, the averaged function $A_\delta f$ given by setting
%
\[ A_\delta f(x) = \frac{1}{|B(x,\delta)|} \int_{B(x,\delta)} f(y)\; dy, \]
%
is measurable.

\begin{lemma}
  If $f \in L_1^{\text{loc}}(X)$, then $A_\delta f$ is a measurable function.
\end{lemma}
\begin{proof}
  If $f = a_1 \mathbf{I}_{U_1} + \dots + a_N \mathbf{I}_{U_N}$ is a simple function, where $U_1,\dots,U_N$ are open sets, then
  %
  \[ A_\delta f(x) = a_1 \frac{|B(x,\delta) \cap U_1|}{|B(x,\delta)|} + \dots + a_N \frac{|B(x,\delta) \cap U_N|}{|B(x,\delta)|} \]
  %
  is a continuous function by our technical assumptions. Next, if $f \geq 0$ is a step function, then there exists a monotonically decreasing family of simple functions $\{ f_n \}$ such that $f_n \to f$ pointwise, then the monotone convergence theorem implies that $A_\delta f_n \to A_\delta f$ pointwise, so $A_\delta f$ is measurable. Finally, decomposing any measurable function into the difference of non-negative measurable functions and then considering pointwise limits of step functions completes the proof.
\end{proof}

It also follows from our technical assumptions that for any open set $U$ containing $x$, there exists $\delta_0$ such that for $\delta \leq \delta_0$, $\overline{B(x,\delta)} \subset U$. It follows that for any $f \in C(X)$ and $x \in X$,
%
\begin{equation} \label{pointwiseaverageconvergence}
  \lim_{\delta \to 0} A_\delta f(x) = f(x).
\end{equation}
%
If $Mf = \sup_{\delta > 0} A_\delta f$, then we will show
%
\[ \| Mf \|_{L^{1,\infty}(X)} \lesssim_c \| f \|_{L^1(X)}. \]
%
In particular, this shows that for any $f \in L^1(X)$,
%
\[ \lim_{\delta \to 0} A_\delta f(x) = f(x) \]
%
for almost every $x \in X$. Since this result is a \emph{local result}, it is easy to verify that the result also holds for any $f \in L^1_{\text{loc}}(X)$, i.e. it also holds for any $f \in L^p(X)$ for $1 \leq p \leq \infty$.

\begin{lemma}[Vitali Covering Lemma]
    If $B_1, \dots, B_n$ is a finite collection of balls in $X$, then there is a disjoint subcollection $B_{i_1}, \dots, B_{i_M}$ such that
    %
    \[ \left| \bigcup_{i = 1}^N B_i \right| \leq c \sum_{j = 1}^M |B_{i_j}|. \]
\end{lemma}
\begin{proof}
  Consider the following greedy selection procedure. Let $B_{i_1}$ be the ball in our collection of maximal radius. Given that we have selected $B_{i_1},\dots,B_{i_k}$, let $B_{i_{k+1}}$ be the ball of largest radius not intersecting previous balls selected if possible. Continue doing this until we cannot select any further balls. If $B_j$ is any ball not chosen by this procedure, it must intersect a ball with radius at least as big as $B_j$ itself. But this means that
  %
  \[ \bigcup_{i = 1}^N B_i \subset \bigcup_{j = 1}^M B^*_{i_j}. \]
  %
  Thus
  %
  \[ \left| \bigcup_{i = 1}^N B_i \right| \leq \sum_{j = 1}^M |B^*_{i_j}| \leq c \sum_{j = 1}^M |B_{i_j}|. \qedhere \]
\end{proof}

We have already indicated our proof strategy for proving a weak type bound for the maximal operator, but let us now do things more rigorously.

\begin{theorem}
  For any $f \in L^1(X)$,
  %
  \[ \| Mf \|_{L^{1,\infty}(X)} \leq c \| f \|_{L^1(X)}. \]
\end{theorem}
\begin{proof}
  Set
  %
  \[ E_\lambda = \{ x \in \mathbf{R}^d: Mf(x) > \lambda \}. \]
  %
  Fix a compact subset $K$ of $E_\lambda$ of finite measure. Then $K$ is covered by finitely many balls $B_1,\dots,B_N$ such that on each ball $B_i$,
  %
  \[ \int_{B_i} |f(y)|\; dy > \lambda |B_i|. \]
  %
  Using the Vitali lemma, extract a disjoint subfamily $B_{i_1},\dots, B_{i_M}$ with
  %
  \[ \left| \sum_{j = 1}^M B_{i_j} \right| \leq c \sum_{j = 1}^M |B_{i_j}|. \]
  %
  Then
  %
  \[ \| f \|_{L^1(X)} > \lambda \sum_{j = 1}^M |B_{j_i}| \geq \frac{\lambda}{c} \left| \bigcup_{j = 1}^M B_{j_i} \right| \geq \frac{\lambda |K|}{c}. \]
  %
  Rearranging gives
  %
  \[ |K| \leq \frac{c \| f \|_{L^1(X)}}{\lambda}. \]
  %
  Since $K$ was arbitrary, inner regularity gives
  %
  \[ |E_\lambda| \leq \frac{c \| f \|_{L^1(X)}}{\lambda}. \]
  %
  Since $\lambda$ was arbitrary, the proof is complete.
\end{proof}

\begin{remark}
  The same covering-type argument also gives the boundedness of the \emph{uncentred} Hardy-Littlewood maximal function
  %
  \[ M'f(x) = \sup_{x \in B} \frac{1}{|B|} \int_B |f(y)|\; dy \]
  %
  where $B$ ranges over all balls $\{ B(x',\delta) : x' \in X, \delta > 0 \}$. Since the supremum is over more balls, we have $Mf(x) \leq M'f(x)$ for each $x \in X$. If we assume a stronger condition than we did previously, that if $B(x',\delta) \cap B(x,\delta) \neq \emptyset$, then $B(x',\delta) \subset B(x,c\delta)$ (so that $B^*(x,\delta) \subset B(x,c\delta)$), and that $|B(x,c\delta)| \leq c' |B(x,\delta)|$, then we also find $M' f(x) \leq c' Mf(x)$. Thus in these situations, $M$ and $M'$ are roughly equivalent operators. We shall find these assumptions are also useful for generalizing the Calderon-Zygmund type decompositions that come up in the real-variable analysis of singular integrals.
\end{remark}

\begin{remark}
  We can exploit the ordering of the real line to show that for any family of intervals $\{ I_\alpha \}$ covering a compact set $K$, there is a subcover $I_1,\dots, I_N$ such that any point in $\RR$ is contained in at most two of the intervals. A modification of the argument above shows this gives the slightly better bound $\| Mf \|_{L^{1,\infty}(\RR)} \leq 2 \| f \|_{L^1(\RR)}$, rather than the bound $\| Mf \|_{L^{1,\infty}(\RR)} \leq 3 \| f \|_{L^1(\RR)}$.
\end{remark}

\newpage

TODO  The fundamental properties we require of our balls are that there exists two universal constants $c_1,c_2 > 1$ such that
%
\begin{enumerate}
  \item[(i)] For any $x_1,x_2 \in X$ and $\delta > 0$, if $B(x_1,\delta) \cap B(x_2,\delta) \neq \emptyset$, then
  %
  \[ B(x_2,\delta) \subset B(x_1,c_1 \delta). \]

  \item[(ii)] For any $x \in X$ and $\delta > 0$, $|B(x,c_1\delta)| \leq c_2 |B(x,\delta)|$.
\end{enumerate}
%
To avoid technical complications, we assume two further assumptions that are true in almost all reasonable examples under consideration:
%
\begin{enumerate}
  \item[(iii)] For any $x \in X$,
  %
  \[ \bigcap_{\delta > 0} \overline{B}(x,\delta) = \{ x \} \quad\text{and}\quad \bigcup_{\delta > 0} B(x,\delta) = X \]

  \item[(iv)] For any open set $U \subset X$ and $\delta > 0$, the function
  %
  \[ x \mapsto |B(x,\delta) \cap U| \]
  %
  is a continuous function of $x$.
\end{enumerate}

\newpage





\section{Dyadic Methods and Calderon-Zygmund Decomposition}

There are many different techniques for showing the boundedness of the maximal operator. Let us consider some \emph{dyadic methods} for proving the inequality. Recall that the set of dyadic cubes is
%
\[ \{ Q_{n,k} : n \in \ZZ, k \in 2^n \ZZ^d \} \]
%
where $Q_{n,k}$ is the cube $[k_1, k_1 + 2^n] \times \dots \times [k_d, k_d + 2^n]$. We note that dyadic cubes nest within one another much more easily than balls do (cubes are either nested or disjoint). In particular, if $Q_1,\dots,Q_N$ is any collection of dyadic cubes, there exists an almost disjoint subcollection $Q_{i_1}, \dots, Q_{i_k}$ with $Q_{i_1} \cup \dots \cup Q_{i_k} = Q_1 \cup \dots \cup Q_N$. In particular, this operates as a Vitali-type covering lemma with a constant independant of $d$, so if we define the \emph{dyadic} Hardy-Littlewood maximal operator
%
\[ M_\Delta f(x) = \sup_{x \in Q} \frac{1}{|Q|} \int_Q |f(y)|\; dy \]
%
then we easily obtain the bound $\| M_\Delta f \|_{L^{1,\infty}(\RR^d)} \leq \| f \|_{L^1(\RR^d)}$, with no implicit constant depending on $d$. The bound $\| M_\Delta f \|_{L^\infty(\RR^d)} \leq \| f \|_{L^\infty(\RR^d)}$ is easy, so interpolation gives $\| M_\Delta f \|_{L^p(\RR^d)} \lesssim_p \| f \|_{L^p(\RR^d)}$ for all $1 < p \leq \infty$, with a constant now \emph{independant of dimension}.

If $Q$ is a dyadic cube, then it is contained in a ball $B$ with $|Q| \lesssim_d B$. It follows that for any function $f$ and $x \in \RR^d$,
%
\[ M_\Delta f(x) \lesssim_d Mf(x). \]
%
Thus bounds on $M$ automatically give bounds on $M_\Delta$. The opposite pointwise inequality is unfortunately, \emph{not true}. For instance, if $f$ is the indicator function on $[0,1]$. Then $M_\Delta f$ is supported on $[0,1]$, but $Mf$ is positive on the entirety of $\RR$. To reduce the study of $M$ to the study of $M_\Delta$, we must instead rely on the \emph{$1/3$ translation trick} of Michael Christ.

\begin{lemma}
  Let $I \subset [0,1]$ be an interval. Then there exists an interval $J$, which is either a dyadic interval, or a dyadic interval shifted by $1/3$, such that $I \subset J$ and $|J| \lesssim |I|$.
\end{lemma}
\begin{proof}
  Let $I = [a,b]$. Perform a binary expansion of $a$ and $b$, writing
  %
  \[ a = 0.a_1a_2 \dots \quad\text{and}\quad b = b_1 b_2 \dots. \]
  %
  Let $n$ be the first value where $a_n \neq b_n$. Then $a_n = 0$ and $b_n = 1$. Then $[a,b]$ is contained in the dyadic interval
  %
  \[ Q_1 = \left[ 0.a_1 \dots a_{n-1}, 0.a_1\dots a_{n-1} + 1/2^{n-1} \right] \]
  %
  which has length $1/2^{n-1}$. Find $0 \leq i < \infty$ such that
  %
  \[ a = 0.a_1 \dots a_{n-1} 0 1^i 0 \dots \]
  %
  and $0 \leq j < \infty$ such that
  %
  \[ b = 0.a_1 \dots a_{n-1} 1 0^j 1. \]
  %
  If no such $j$ exists, then $b = 0.a_1 \dots a_{n-1} 1$, and so $[a,b]$ is contained in the rational interval
  %
  \[ Q_2 = \left[ 0.a_1 \dots a_{n-1} 0 1^i, 0.a_1 \dots a_{n-1} 0 1^i + 1/2^{n+i} \right] \]
  %
  and $b - a \geq 1/2^{n+i+1}$, so $|Q_2| \leq 2(b - a)$. Now if $i \leq 5$ or $j \leq 5$, then $b - a \geq 1/2^{n+5}$, so $|Q_1| \leq 2^5(b-a)$. On the other hand, if $i \geq 5$ and $j \geq 5$, we find $b - a \geq 1/2^{n+\min(i,j)}$. Then we can find a dyadic interval $Q_3$ and $2 \leq r \leq 5$ such that
  %
  \[ 1/3 + Q_3 = \left[ 0.a_1 \dots a_{n-1} 0 1^{\min(i,j)-r} 1 0 1 0 \dots, 0.a_1 \dots a_{n-1} 0 1^{i-r} 1 0 1 0 \dots + 1/2^{n+\min(i,j)-r}  \right] \]
  %
  and so $1/3 + Q_3$ contains $[a,b]$ and $|Q_3| = 1/2^{n+\min(i,j)-r} \leq 2^5 (b - a)$.
\end{proof}

It follows that for each $x \in \RR^d$, and any function $f$,
%
\[ Mf(x) \lesssim_d (M_\Delta f)(x) + (M_\Delta \text{Trans}_{1/3} f)(x). \]
%
Since the $L^p$ norms are translation invariant, this implies that the dyadic maximal operator and the maximal operator satisfy equivalent bounds, with operator norms differing by a constant depending on $n$. Since we independently obtained bounds on $M_\Delta$, this section provides an alternate proof to the boundedness of $M$.

There is an alternate way to view the operator $M_\Delta$. For each integer $n$, we let $\mathcal{B}(n)$ denote the family of all sidelength $1/2^n$ dyadic cubes. Thus $\mathcal{B}(n)$ gives a decomposition of $\RR^d$ into an almost disjoint union of cubes. If we define the conditional expecation operators
%
\[ E_n f(x) = \sum_{Q \in \mathcal{B}(n)} \left( \frac{1}{|Q|} \fint_Q f \right) \cdot \mathbf{I}_Q \]
%
then $M_\Delta f = \sup_{n \in \ZZ} E_n f$. In particular, it is easy to see from the bounds on $M_\Delta$ that for any $f \in L^1_{\text{loc}}(\RR^d)$, $\lim_{n \to \infty} E_n f(x) = f(x)$ holds for almost every $x \in \RR^d$. It is simple to conclude from this result a very useful technique, known as the \emph{Calder\'{o}n-Zygmund decomposition}.

\begin{theorem}
  Given $f \in L^1(\RR^d)$ and $\lambda > 0$, we can write $f = g + b$, where $\| g \|_{L^\infty(\RR^d)} \lesssim_d \lambda$, and there is an almost disjoint family of dyadic cubes $\{ Q_i \}$ such that $g$ is supported on $\bigcup_i Q_i$,
  %
  \[ \sum_i |Q_i| \leq \frac{\| f \|_{L^1(\RR^d)}}{\lambda}, \]
  %
  and for each $i$,
  %
  \[ \int_{Q_i} f(y)\; dy = 0. \]
  %
  We also have $\| g \|_{L^1(\RR^d)}, \| b \|_{L^1(\RR^d)} \lesssim \| f \|_{L^1(\RR^d)}$.
\end{theorem}
\begin{proof}
  Write $E = \{ x: M_\Delta f(x) > \lambda \}$. By the dyadic Hardy-Littlewood maximal inequality,
  %
  \[ |E| \leq \frac{\| f \|_{L^1(\RR^d)}}{\lambda}. \]
  %
  Because $f$ is integrable, $E \neq \mathbf{R}^d$. Thus we can write $E$ as the almost disjoint union of dyadic cubes $\{ Q_i \}$, such that for each $i$,
  %
  \[ \int_{Q_i} |f(x)|\; dx > \lambda |Q_i|, \]
  %
  and also, if $R_i$ is the parent cube of $Q_i$,
  %
  \[ \int_{R_i} |f(x)|\; dx \leq \lambda |R_i|. \]
  %
  This can be done by a greedy strategy, taking the union of dyadic cubes of largest sidelength contained in $E$. This means
  %
  \[ \int_{Q_i} |f(x)|\; dx \leq \int_{R_i} |f(x)|\; dx \leq \lambda |R_i| \leq 2^d \lambda |Q_i|. \]
  %
  Define
  %
  \[ g(x) = \begin{cases} f(x) &: x \not \in E, \\ \frac{1}{|Q_i|} \int_{Q_i} f(x)\; dx &: x \in Q_i\ \text{for some $i$}. \end{cases} \]
  %
  For almost every $x \in E^c$, $|f(x)| \leq \lambda$, since $E_n f(x) \leq \lambda$ for each $n$, and $E_n f(x) \to f(x)$ as $n \to \infty$ for almost every $x$. Conversely, if $x \in Q_i$ for some $i$, then
  %
  \[ \left| \frac{1}{|Q_i|} \int_{Q_i} f(x)\; dx \right| \leq \frac{1}{|Q_i|} \int_{Q_i} |f(x)|\; dx \leq 2^d \lambda. \]
  %
  Thus $\| g \|_{L^\infty(\RR^d)} \lesssim_d \lambda$. If we define $b = f - g$, then $b$ is supported on $\bigcup Q_i = E$, and for each $i$,
  %
  \[ \int_{Q_i} b(x)\; dx = \int_{Q_i} \left( f(x) - \frac{1}{|Q_i|} \int_{Q_i} f(y)\; dy \right)\; dx = 0. \qedhere \]
\end{proof}

\section{Lebesgue Density Theorem}

If $E$ is a measurable subset of $\mathbf{R}^d$, and $x \in \mathbf{R}^d$, we say $x$ is a point of \emph{Lebesgue density} of $E$, or has \emph{full metric density} if
%
\[ \lim_{\substack{|B| \to 0\\x \in B}} \frac{|B \cap E|}{|B|} = 1 \]
%
This means that for every $\alpha < 1$, for suitably small balls, we conclude that $|B \cap E| \geq \alpha |B|$, so $E$ asymptotically contains as large a fraction of the local points around $x$ as is possible. Since $\chi_E \in L^1_{\text{loc}}(\mathbf{R}^d)$, we can apply the Lebesgue differentiation theorem to immediately obtain an interesting result.

\begin{theorem}[Lebesgue Density Theorem]
    If $E$ is a measurable subset, then almost every point in $E$ is a point of Lebesgue density, and almost every point in $E$ is not a point of Lebesgue density.
\end{theorem}

The fact that a point is a point of Lebesgue density implies the existence of large sets of rigid patterns in $E$. Note that if $|B \cap E|, |B \cap F| \geq \alpha |B|$, then a union bound gives $|B \cap E \cap F| \geq (2 \alpha - 1)|B|$. As $\alpha \to 1$, $2\alpha - 1 \to 1$, so if $x$ is a point of Lebesgue density for $E$ and $F$, then $x$ is a point of Lebesgue density of $E \cap F$. If $0$ is a point of Lebesgue density for $E$, then $0$ is a point of Lebesgue density for $\alpha E$ for any $\alpha \neq 0$, and so for any nonzero $\alpha_1, \dots, \alpha_n$, $0$ is a point of Lebesgue density for $\alpha_1 E \cap \dots \cap \alpha_n E$, which in particular implies the existence of $x_i \to 0$ with $x_i \in \alpha_1 E \cap \dots \cap \alpha_n E$, hence $\alpha_1^{-1} x_i, \dots, \alpha_n^{-1} x_i \in E$. This is very difficult to prove without the existence of the Lebesgue density theorem, because of the discreteness of the equations. Applying these results with $\alpha_1 = 1$, $\alpha_2 = 1/2$, $\alpha_2 = 1/3$, and so on, we conclude that sets of positive measure in $\mathbf{R}$ contain infinitely many arbitrarily long arithmetic progressions.

If $f$ is locally integrable, the \emph{Lebesgue set} of $f$ consists of all points $x \in \mathbf{R}^d$ such that $f(x)$ is finite and
%
\[ \lim_{\substack{|B| \to 0\\x \in B}} \frac{1}{|B|} \int |f(y) - f(x)|\ dy = 0 \]
%
If $f$ is continuous at $x$, it is obvious to see that $x$ is in the Lebesgue set of $f$, and if $x$ is in the Lebesgue set of $f$, then the averages of $f$ on balls around $x$ coverge to $f(x)$.

\begin{theorem}
    If $f \in L^1_{\text{loc}}(\mathbf{R}^d)$, almost every point is in the Lebesgue set of $f$.
\end{theorem}
\begin{proof}
    For each rational number $p$, the function $|f - p|$ is measurable, so that there is a set $E_p$ of measure zero such that for $x \in E_p^c$,
    %
    \[ \lim_{\substack{|B| \to 0\\x \in B}} \frac{1}{|B|} \int_B |f(y) - p|\ dy \to |f(x) - p| \]
    %
    Taking unions, we conclude that $E = \bigcup E_p$ is a set of measure zero. Suppose $x \in E^c$, and $f(x)$ is finite. For any $\varepsilon$, there is a rational $p$ such that $|f(x) - p| < \varepsilon$, and we know the equation above holds, so
    %
    \begin{align*}
        \lim_{\substack{|B| \to 0\\x \in B}} &\frac{1}{|B|} \int_B |f(y) - f(x)|\ dy\\
        &\leq \limsup_{\substack{|B| \to 0\\x \in B}} \frac{1}{|B|} \int_B |f(y) - p| + |p - f(x)|\ dy \leq 2\varepsilon
    \end{align*}
    %
    we can then let $\varepsilon \to 0$. Since $f(x)$ is finite for almost all $x$ when $f$ is locally integrable, this completes the proof.
\end{proof}

It is interesting to note that if $f = g$ almost everywhere, then the set of points $x$ where the averages of $f$ on balls around $x$ converges is the same as the set of points $x$ where the averages of $f$ on balls around $x$ converges, and so we can in some sense define a `universal' function $h$ from the equivalence class of these functions such that the averages of $h$ on balls around $x$ always converge to $h(x)$ when the limit exists. However, this isn't often done, because it doesn't really help in the analysis of integrable functions. We note, however, that the Lebesgue set of a function does depend on the function chosen from the equivalence class. However, the Lebesgue set of the universal function constructed above is the largest of any function in the equivalence class, which is sometimes taken as the canonical Lebesgue set of the class. Alternatively, this version of the Lebesgue set can be taken as the points $x$ such that there exists $a_x$ with
%
\[ \lim_{\substack{|B| \to 0 \\ x \in B}} \fint_B |f(y) - a_x| = 0 \]
%
Then it is clear that the Lebesgue set of two functions agree if they are equal almost everywhere.

\section{Generalizing The Differentiation Theorem}

The boundedness of the maximal function we considered earlier depends very little on the fact that the sets we are averaging over are balls. In fact, there are only very few properties of $\RR^d$ that we used. To begin with, we can generalize the family of sets we use. A family of sets $U_\alpha$ universally containing a point $x$ is said to \emph{shrink regularly} to $x$, or has \emph{bounded eccentricity} at $x$, if $\inf |U_\alpha| = 0$, and there is a constant $c > 0$ such that for each $U_\alpha$, there is a ball $B$ with $x \in B$, $U_\alpha \subset B$, and $|U_\alpha| \geq c |B|$. Thus $U_\alpha$ contains a large percentage of certain balls $B$ around $x$. In particular, if we define
%
\[ M_U(f) = \sup_{U_\alpha} \fint_{U_\alpha} |f(x)|\; dx \]
%
Then
%
\[ M_U(f) = \sup_{U_\alpha} \fint_{U_\alpha} |f(x)|\; dx \leq c^{-1} \sup_B \fint_B |f(x)|\; dx = c^{-1} (Mf)(x) \]
%
In particular, $M_U \lesssim M$, which implies $M_U$ satisfies the same bounds that the Hardy-Littlewood maximal function satisfies. We therefore conclude that for any locally integrable $f$,
%
\[ \lim_{\substack{U_\alpha \to 0\\x \in U_\alpha}} \fint_{U_\alpha} f(y)\;dy = f(x) \]
%
Thus the differentiation theorem easily generalizes to averages over any sets which don't differ too much from a ball.

\begin{example}
    The set of all open cubes in $\mathbf{R}^d$ containing $x$ shrinks regularly to $x$, because if a cube $U$ centered at $y$ with side lengths $r$ contains $x$, then using the existence of a constant $C$ such that for all $x,y \in \mathbf{R}^d$,
    %
    \[ \| x - y \|_\infty \leq C \| x - y \|_2 \]
    %
    we conclude that the cube is contained within a ball $B$ of radius $2Cr$, and since $|U| = r^d$, and $|B|$ is proportional to $(2Cr)^d$ up to a constant, so that $U$ has bounded eccentricity.
\end{example}

\begin{example}
    The set of all rectangles in $\mathbf{R}^d$ containing $x$ does {\it not} shrink regularly, because we can let the rectangle have one large side length while keeping all other side lengths relatively small, and then a ball containing this rectangle must be incredibly large.
\end{example}

\begin{theorem}
    If $f$ is locally integrable on $\mathbf{R}^d$, and $\{ U_\alpha \}$ shrinks regularly to $x$, then for every point $x$ in the Lebesgue set of $f$,
    %
    \[ \lim_{|U_\alpha| \to 0} \frac{1}{|U_\alpha|} \int_{U_\alpha} f(y)\ dy = f(x) \]
\end{theorem}
\begin{proof}
    We just calculate that for every $x$ in the Lebesgue set of $f$,
    %
    \[ \lim_{|U_\alpha| \to 0} \frac{1}{|U_\alpha|} \int_{U_\alpha} |f(y) - f(x)|\ dy = 0 \]
    %
    This follows because if $U_\alpha \subset B_\alpha$, with $|U_\alpha| \geq C|B_\alpha|$, then
    %
    \[ \frac{1}{|U_\alpha|} \int_{U_\alpha} |f(y) - f(x)|\ dy \leq \frac{1}{C|B_\alpha|} \int_{B_\alpha} |f(y) - f(x)|\ dy \]
    %
    and since $|U_\alpha| \to 0$, $|B_\alpha| \to 0$, giving us the result.
\end{proof}

We can also consider more general ambient spaces than $\RR^d$, which will enable us to obtain maximal type bounds like
%
\[ Mf(n) = \sup_{N > 0} \frac{1}{N} \sum_{m = 1}^N |f(n + m)|. \]
%
for functions $f$ on $\ZZ$, and
%
\[ Mf(x) = \sup_{r > 0} \frac{1}{2r} \int_{-r}^r |f(x + t)|\; dt \]
%
for functions $f$ on $\TT$. TODO FINISH THIS (STEIN'S BOOK?)

One consequence of the integer-domain maximal inequality is a pointwise convergence result in ergodic theory. We recall that a \emph{measure preserving system} is a probability space $X$ together with a measure preserving transformation $T: X \to X$.

\begin{theorem}
  Let $X$ and $T$ form a measure preserving transformation. Then for all $f \in L^1(X)$ and almost every $x \in X$, the limit
  %
  \[ \lim_{N \to \infty} \frac{1}{N} \sum_{n = 1}^N T^n f(x) \]
  %
  exists.
\end{theorem}
\begin{proof}
  Fix $N_0 > 0$ and $f \in L^1(X)$, and define a measurable function $F$ on $X \times [2N_0]$ by defining
  %
  \[ F(x,n) = T^n f(x). \]
  %
  Let
  %
  \[ MF(x,n) = \sup_{1 \leq N \leq N_0} \frac{1}{N} \sum_{m = 1}^N T^{n+m} f(x). \]
  %
  Then the integer-valued maximal inequality implies that
  %
  \[ \| MF \|_{l^{1,\infty}[N_0]} \lesssim \| F \|_{l^1[2N_0]} \]
  %
  and integrating in $X$, that
  %
  \[ \| MF \|_{l^{1,\infty}[N_0] L^1(X)} \lesssim \| F \|_{l^1[2N_0] L^1(X)} = \| F \|_{L^1(X \times [2N_0])} = 2N_0 \| f \|_{L^1(X)}. \]
  %
  TODO FINISH THIS.
\end{proof}

\section{Approximations to the Identity}

We now switch to the study of how we can approximate functions by convolutions of concentrated functions around the origin. In this section we define the various classes of such functions which give convergence results, to various degrees of strength. We say a family $K_\alpha \in L^1(\mathbf{R}^d)$ is a \emph{good kernel} if it is bounded in the $L^1$ norm, for every $\alpha$,
    %
    \[ \int K_\alpha(x)\ dx = 1 \]
    %
    and if for every $\delta > 0$, as $\alpha \to \infty$,
    %
    \[ \int_{|x| \geq \delta} |K_\alpha(x)|\ dx \to 0 \]
    %
It requires only basic analysis to verify good kernel convergence.

\begin{theorem}
    If $K_\alpha$ is a good kernel, then for any absolutely integrable function $f$, $f * K_\alpha \to f$ in the $L^1$ norm, and $(f * K_\alpha)(x) \to f(x)$ for every $x$ which is a point of continuity of $f$.
\end{theorem}
\begin{proof}
    Note that
    %
    \begin{align*}
        \| (f * K_\alpha) - f \|_1 &= \int |(f * K_\alpha)(x) - f(x)|\ dx\\
        &= \int \left| \int K_\alpha(y) [f(x - y) - f(x)]\ dy \right|\ dx\\
        &\leq \int |K_\alpha(y)| \| T_y f - f \|_1\ dy
    \end{align*}
    %
    where $(T_y f)(x) = f(x - y)$. We know that $\| T_y f - f \|_1 \to 0$ as $y \to 0$. Thus, for each $\varepsilon$, we can pick $\delta$ such that if $|y| < \delta$, $\| T_y f - f \|_1 \leq \varepsilon$, and if we pick $\alpha$ large enough that $\int_{|y| \geq \delta} |K_\alpha(y)|\ dy \leq \varepsilon$, and then
    %
    \[ \| (f * K_\alpha) - f \|_1 \leq \varepsilon \int_{|y| < \delta} |K_\alpha(y)|\ dy + 2 \| f \|_1 \int_{|y| \geq \delta} |K_\alpha(y)|\ dy \leq \varepsilon[\| K_\alpha \|_1 + 2 \| f \|_1] \]
    %
    Since $\| K_\alpha \|_1$ is universally bounded over $\alpha$, we can let $\varepsilon \to 0$ to obtain convergence. If $x$ is a fixed point of continuity, and for a given $\varepsilon > 0$, we pick $\delta > 0$ with $|f(y) - f(x)| \leq \varepsilon$ for $|y - x| < \delta$, then
    %
    \begin{align*}
        |(f * K_\alpha)(x) - f(x)| &= \left| \int_{-\infty}^\infty f(y) K_\alpha(x - y)\ dy - f(x) \right|\\
        &= \left| \int_{-\infty}^\infty [f(y) - f(x)] K_\alpha(x-y)\ dy \right|\\
        &= \left| \int_{-\delta}^\delta [f(y) - f(x)] K_\alpha(x-y)\ dy \right|\\
        &\ \ \ \ \ + \left| \int_{|y| \geq \delta} [f(y) - f(x)] K_\alpha(x - y)\ dy \right|\\
        &\leq \varepsilon \| K_\alpha \|_1 + [\| f \|_1 + f(x)] \int_{|y| \geq \delta} |K_\alpha(y)|\ dy
    \end{align*}
    %
    If $\| K_\alpha \|_1 \leq M$ for all $\alpha$, and we choose $\alpha$ large enough that $\int_{|y| \geq \delta} |K_\alpha(y)| \leq \varepsilon$, then we conclude the value about is bounded by $\varepsilon [M + \| f \|_1 + f(x)]$, and we can then let $\varepsilon \to 0$.
\end{proof}

To obtain almost sure pointwise convergence of $f * K_\alpha$ to $f$, we must place stronger conditions on our family. We say a family $K_\delta \in L^1(\mathbf{R}^d)$, is an \emph{approximation to the identity} if $\int K_\delta = 1$, and
%
\[ |K_\delta(x)| \lesssim \frac{\delta}{|x|^{d+1}}\ \ \ \ |K_\delta(x)| \lesssim \frac{1}{\delta^d} \]
%
where the constant bound is independent of $x$ and $\delta$. These assumptions are stronger than being a good kernel, because if $K_\delta$ is an approximation to the identity, then
%
\[ \int_{|x| \geq \varepsilon} |K_\delta(x)| \leq \int_\varepsilon^\infty \int_{S^{d-1}} \frac{C \delta}{r}\ d\sigma dr = C \delta |S^{n-1}| \int_\varepsilon^\infty \frac{dr}{r} \leq \frac{C \delta |S^{n-1}|}{\varepsilon} \]
%
which converges to zero as $\delta \to 0$. Combined with
%
\[ \int_{|x| < \varepsilon} |K_\delta(x)| \leq C \int_0^\varepsilon \int_{S^{d-1}} \frac{r^{d-1}}{\delta^d} d\sigma dr = \frac{C \varepsilon^d |S^{n-1}|}{d \delta^d} \]
%
This calculation also implies
%
\begin{align*}
    \| K_\delta \|_1 &\leq C |S^{n-1}| \left[ \frac{\delta}{\varepsilon} + \frac{\varepsilon^d}{\delta^d} \right]
\end{align*}
%
Setting $\varepsilon = \delta$ optimizes this value, and gives a bound
%
\[ \| K_\delta \|_1 \leq 2C |S^{n-1}| \]
%
So an approximation to the identity is a stronger version of a good kernel.

\begin{example}
    If $\varphi$ is a bounded function in $\mathbf{R}^d$ supported on the closed ball of radius one with $\int \varphi(x)\ dx = 1$, then $K_\delta(x) = \delta^{-d} \varphi(\delta^{-1}x)$ is an approximation to the identity, because by a change of variables, we calculate
    %
    \[ \int_{\mathbf{R}^d} \frac{\varphi(\delta^{-1}x)}{\delta^d} = \int_{\mathbf{R}^d} \varphi(x) = 1 \]
    %
    Because $\varphi$ is bounded, we find
    %
    \[ |K_\delta(x)| \leq \frac{\| \varphi \|_\infty}{\delta^d} \]
    %
    Now $K_\delta$ is supported on a disk of radius $\delta$, this bound also shows
    %
    \[ |K_\delta(x)| \leq \frac{\delta \| \varphi \|_\infty}{|x|^{d+1}} \]
    %
    and so $K_\delta$ is an approximation to the identity. If $\varphi$ is an arbitrary integrable function, then $K_\delta$ will only be a good kernel.
\end{example}

\begin{example}
    The Poisson kernel in the upper half plane is given by
    %
    \[ P_y(x) = \frac{1}{\pi} \frac{y}{x^2 + y^2} \]
    %
    where $x \in \mathbf{R}$, and $y > 0$. It is easy to see that
    %
    \[ P_y(x) = y^{-1} P_1(xy^{-1}) \]
    %
    And
    %
    \[ \int \frac{1}{1 + x^2} = \arctan(\infty) - \arctan(-\infty) = \pi \]
    %
    We easily obtain the bounds
    %
    \[ |P_y(x)| \leq \frac{\| P_1 \|_\infty}{y}\ \ \ \ \ |P_y(x)| \leq \frac{y}{\pi |x|^2} \]
    %
    so the Poisson kernel is an approximation to the identity.
\end{example}

\begin{example}
    The heat kernel in $\mathbf{R}^d$ is defined by
    %
    \[ H_t(x) = \frac{e^{-|x|^2/4t}}{(4 \pi t)^{d/2}} \]
    %
    where $\delta = t^{1/2} > 0$. Then $H_t(x) = \delta^{-d} H_1(x\delta^{-1})$, and
    %
    \[ \int e^{-|x|^2/4} = \frac{1}{2^d} \int e^{-|x|^2} = \frac{|S^{n-1}|}{2^d} \int_0^\infty r^{d-1} e^{-r^2} dr \]
    %
%    By induction, we can prove that if $d$ is odd, then the antiderivative of $r^de^{-r^2}$ is equal to $P_d(r)e^{-r^2}$, where the coefficients of $P_d$ are nonzero only when the coefficient index is even. This follows because the chain rule gives
    %
%    \[ \int re^{-r^2} = -e^{-r^2}/2 \]
    %
%    and an integration by parts gives
    %
%    \[ \int r^{d+2}e^{-r^2} = r^2P_d(r)e^{-r^2} - 2 \int rP_d(r)e^{-r^2} \]
    %
%    Thus
    %
%    \[ \int r^3e^{-r^2} = (-r^2/2)e^{-r^2} + \int re^{-r^2} = (-1/2)(r^2 + 1) e^{-r^2} \]
%    \[ \int r^5e^{-r^2} = -(1/2) r^2(r^2 + 1)e^{-r^2} + \int r(r^2 + 1) e^{-r^2} = (-1/2)[r^4 + 2r^2 + 2] \]
%    \[ \int r^7e^{-r^2} = -(1/2) r^2[r^4 + 2r^2 + 2]e^{-r^2} + \int (r^5 + 2r^3 + 2r) e^{-r^2} = (-1/2) [r^6 + 3r^4 + 6r^2 + ] e^{-r^2} \]
\end{example}

\begin{example}
    The Poisson kernel for the disk is
    %
    \[ \frac{P_r(x)}{2 \pi} = \begin{cases} \frac{1}{2\pi} \frac{1 - r^2}{1 - 2r \cos x + r^2} &: |x| \leq \pi \\ 0 &: |x| > \pi \end{cases} \]
    %
    where $0 < r < 1$, and $\delta = 1-r$.
\end{example}

\begin{example}
    The F\'{e}jer kernel is
    %
    \[ \frac{F_N(x)}{2 \pi} = \begin{cases} \frac{1}{2 \pi N} \frac{\sin^2(Nx/2)}{\sin^2(x/2)} \end{cases} \]
    %
    where $\delta = 1/N$.
\end{example}

As $\delta \to 0$, we may think of the $K_\delta$ as `tending to the unit mass' Dirac delta function $\delta$ at the origin. $\delta$ may be given a precise meaning, either in the theory of Lebesgue-Stieltjes measures or as a `generalized function', but we don't need it to discuss the actual convergence results of the functions $K_\delta$.

\begin{theorem}
    If $\{ K_\delta \}$ is an approximation to the identity, and $f$ is integrable on $L^1(\mathbf{R}^d)$, then $(f * K_\delta)(x) \to f(x)$ for every $x$ in the Lebesgue set of $f$, and $f * K_\delta$ converges to $f$ in the $L^1$ norm.
\end{theorem}
\begin{proof}
    We rely on the fact that if $x$ is in the Lebesgue set, then the function
    %
    \[ A(r) = \frac{1}{r^d} \int_{|y| \leq r} |f(x-y) - f(x)|\ dy \]
    %
    is a bounded continuous function of $r > 0$, converging to $0$ as $r \to 0$. This means that if $\Delta(y) = |f(x-y) - f(x)| |K_\delta(y)|$, then
    %
    \[ \int \Delta(y)\ dy = \int_{|y| \leq \delta} \Delta(y) + \sum_{k = 0}^\infty \int_{2^k \delta \leq |y| \leq 2^{k+1} \delta} \Delta(y) \]
    %
    The first term is easily upper bounded by $CA(\delta)$, and the $k$'th term of the sum by $C'2^{-k}A(2^{k+1}\delta) \leq C''2^{-k}$ for constants $C',C''$ that do not depend on $\delta$. Letting $\delta \to 0$ gives us the convergence result.
\end{proof}

\section{Differentiability of Measurable Functions}

We now switch our object of study to finding a condition on a measurable function $f$ which guarantees differentiability almost everywhere, such that the derivative is absolutely integrable, and
%
\[ f(b) - f(a) = \int_a^b f'(t)\ dt \]
%
holds almost everywhere. One way we can solve our problem is to fix our attention to functions $f$ obtained by indefinite integrals. The results we have established guarantee that this theorem holds. But this leads to the extended problem of considering ways to characterize the properties of functions that arise from these indefinite integrals. We shall find that if $f$ has {\it bounded variation}, then most of these problems are answered.

If $f$ is a complex valued function on $[a,b]$, and $P$ is a partition, we can consider it's variation on a partition $P = a \leq t_0 < \dots < t_n \leq b$ to be
%
\[ V(f,P) = \sum_{k = 1}^n |f(t_k) - f(t_{k-1})| \]
%
we say $f$ has \emph{bounded variation} if there is a constant $M$ such that for any partition $P$, $V(f,P) \leq M$. This implies that, since the net $P \mapsto V(f,P)$ is increasing, the net converges to a value $V(f) = V(f,a,b)$, the \emph{total variation} of $f$ on $[a,b]$. The problem of variation is very connected to the problem of the {\it rectifiability of curves}. If $x: [a,b] \to \mathbf{R}^d$ parameterizes a continuous curve in the plane, then, for a given partition $P = a \leq t_0 \leq \dots \leq t_n$, we can consider an approximate length
%
\[ L_P(x) = \sum_{k = 1}^n |x(t_i) - x(t_{i-1})| \]
%
If $x$ has a reasonable notion of length, then the straight lines between $x(t_{i-1})$ and $x(t_i)$ should be shorter than the length of $x$ between $t_{i-1}$ and $t_i$. It therefore makes sense to define the \emph{length} of $x$ as
%
\[ L(x) = \sup L_P(x) \]
%
The triangle inequality implies that the map $P \mapsto L_P(x)$ is an increasing net, so $L$ is also the limit of the meshes as they become finer and finer. If $L(x) < \infty$, we say $x$ is a \emph{rectifiable curve}. One problem is to determine what analytic conditions one must place on $x$ in order to guarantee regularity, and what further conditions guarantee that, if $x_i$ is differentiable almost everywhere,
%
\[ L(x) = \int_a^b \sqrt{x_1'(t)^2 + \dots + x_n'(t)^2}\ dt \]
%
Considering rectifiable curves leads directly to the notion of a function with bounded variation.

\begin{theorem}
    A curve $x$ is rectifiable iff each $x_i$ has bounded variation.
\end{theorem}
\begin{proof}
    We can find a universal constants $A,B > 0$ such that for any $x,y \in \mathbf{R}^d$,
    %
    \[ A \sum |x_i - y_i| \leq |x-y| \leq B \sum |x_i - y_i| \]
    %
    This means that if $P$ is a partition of $[a,b]$, then
    %
    \[ A \sum_{ij} |x_j(t_i) - x_j(t_{i-1})| \leq \sum |x(t_i) - x(t_{i-1})| \leq B \sum_{ij} |x_j(t_i) - x_j(t_{i-1})| \]
    %
    So $A \sum V(x_i,P) \leq L_P(x) \leq B \sum V(x_i,P)$ gives the required result.
\end{proof}

\begin{example}
    If $f$ is a real-valued, monotonic, increasing function on $[a,b]$, then $f$ has bounded variation, and one can verify that $V(f) = f(b) - f(a)$.
\end{example}

\begin{example}
    If $f$ is differentiable at every point, and $f'$ is bounded, then $f$ has bounded variation. The mean value theorem implies that if $|f'| \leq M$, then for all $x,y \in [a,b]$,
    %
    \[ |f(x) - f(y)| \leq M |x-y| \]
    %
    This implies that $V(f,P) \leq M(b-a)$ for all partitions $P$.
\end{example}

\begin{example}
    Consider the functions $f$ defined on $[0,1]$ with
    %
    \[ f(x) = \begin{cases} x^a \sin(x^{-b}) &: 0 < x \leq 1 \\ 0 &: x = 0 \end{cases} \]
    %
    Then $f$ has bounded variation on $[0,1]$ if and only if $a > b$. The function oscillates from increasing to decreasing on numbers of the form $x = (n \pi)^{-1/b}$, so the total variation is described as
    %
    \begin{align*}
      V(f) &= 1 + \sum_{n = 1}^\infty (n \pi)^{-a/b} + ((n+1) \pi)^{-a/b}
    \end{align*}
    %
    This sum is finite precisely when $a/b > 1$. Thus functions of bounded variation cannot oscillate too widely, too often.
\end{example}

The next result is a decomposition theorem for bounded variation functions into bounded increasing and decreasing functions. We define the \emph{positive variation} of a real valued function $f$ on $[a,b]$ to be
%
\[ P(f,a,b) = \sup_P \sum_{f(t_i) \geq f(t_{i-1})} f(t_i) - f(t_{i-1}) \]
%
The \emph{negative variation} is
%
\[ N(f,a,b) = \sup_P \sum_{f(t_i) \leq f(t_{i-1})} -[f(t_i) - f(t_{i-1})] \]
%
Note that for each partition $P$, the sums of the two values above add up to the variation with respect to the partition.

\begin{lemma}
    If $f$ is real-valued and has bounded variation on $[a,b]$, then for all $a \leq x \leq b$,
    %
    \[ f(x) - f(a) = P(f,a,x) - N(f,a,x) \]
    %
    \[ V(f) = P(f,a,b) + N(f,a,b) \]
\end{lemma}
\begin{proof}
    Given $\varepsilon$, there exists a partition $a = t_0 < \dots < t_n = x$ such that
    %
    \[ \left| P(f,a,x) - \sum_{f(t_i) \geq f(t_{i-1})} f(t_i) - f(t_{i-1}) \right| < \varepsilon \]
    \[ \left| N(f,a,x) + \sum_{f(t_i) \leq f(t_{i-1})} f(t_i) - f(t_{i-1}) \right| < \varepsilon \]
    %
    It follows that
    %
    \[ |f(x) - f(a) - [P(f,a,x) - N(f,a,x)]| < 2 \varepsilon \]
    %
    and we can then take $\varepsilon \to 0$. The second identity follows the same way.
\end{proof}

A real function $f$ on $[a,b]$ has bounded variation if and only if $f$ is the difference of two increasing bounded functions, because if $f$ has bounded variation, then
%
\[ f(x) = [f(a) + P(f,a,x)] - N(f,a,x) \]
%
is the difference of two bounded increasing functions. On the other hand, the difference of two bounded increasing functions is clearly of bounded variation. A complex function has bounded variation if and only if it is the linear combination of four increasing functions in each direction.

\begin{theorem}
    If $f$ is a continuous function of bounded variation, then
    %
    \[ x \mapsto V(f,a,x) \ \ \ \ \ x \mapsto V(x,b) \]
    %
    are continuous functions.
\end{theorem}
\begin{proof}
    $V(f,a,x)$ is an increasing functin of $x$, so for continuity on the left it suffices to prove that for each $x$ and $\varepsilon$, there is $x_1 < x$ such that $V(f,a,x_1) \geq V(f,a,x) - \varepsilon$. If we consider a partition
    %
    \[ P = \{ a = t_0 <  \dots < t_n = x \} \]
    %
    where $|V(f,P) - V(f,a,x)| \leq \varepsilon$, then by continuity of $f$ at $x$, there is $t_{n-1} < x_1 < x$ with $|f(x) - f(x_1)| < \varepsilon$, and then if we modify $P$ to obtain $Q$ by swapping $t_n$ with $x_1$, we find
    %
    \begin{align*}
        V(f,a,x_1) \geq V(f,Q) &= V(f,P) - |f(x) - f(t_{n-1})| + |f(x_1) - f(t_{n-1})|\\
        &\geq V(f,P) - \varepsilon \geq V(f,a,x) - \varepsilon
    \end{align*}
    %
    A similar argument gives continuity on the right, and the continuity as the left bound of the interval changes.
\end{proof}

To obtain the differentiation theorem for functions of bounded variation, we require a lemma of F. Riesz.

\begin{lemma}[Rising Sun lemma]
    If $f$ is real-valued and continuous on $\mathbf{R}$, and $E$ is the set of points $x$ where there exists $h > 0$ such that $f(x+h) > f(x)$, then, provided $E$ is non-empty, it must be open, and can be written as a union of disjoint intervals $(a_n,b_n)$, where $f(b_n) = f(a_n)$. If $f$ is continuous on $[a,b]$, then $E$ is still an open subset of $[a,b]$, and can be written as the disjoint union of countably many intervals, with $f(b_n) = f(a_n)$ except if $a_n = a$, where we can only conclude $f(a_n) \leq f(b_n)$.
\end{lemma}
\begin{proof}
  The openness is clear, and the fact that $E$ can be broken into disjoint intervals follows because of the characterization of open sets in $\mathbf{R}$. If
  %
  \[ E = \bigcup (a_n,b_n) \]
  %
  Then $f(a_n + h) \leq f(a_n)$ and $f(b_n + h) \leq f(b_i)$, implying in particular that $f(b_n) \leq f(a_n)$, If $f(b_n) < f(a_n)$, then choose $f(b_n) < c < f(a_n)$. The intermediate value theorem implies there is $x$ with $f(x) = c$, and we may choose the largest $x \in [a_n,b_n]$ for which this is true. Then since $x \in (a_n,b_n)$, there is $y \in (x,b_i)$ with $f(x) < f(y)$, and by the intermediate value theorem, since $f(b_n) < f(x) < f(y)$, there must be $x' \in (y,b_n)$ with $f(x') = c$, contradicting that $x$ was chosen maximally. The proof for closed intervals operates on the same principles.
\end{proof}

\begin{theorem}
    If $f$ is increasing and continuous on $[a,b]$, then $f$ is differentiable almost everywhere. That is,
    %
    \[ f'(x) = \lim_{h \to 0} \frac{f(x+h) - f(x)}{h} \]
    %
    exists for almost every $x \in [a,b]$, $f'$ is measurable, and
    %
    \[ \int_a^b f'(x) \leq f(b) - f(a) \]
    %
    In particular, if $f$ is bounded on $\mathbf{R}$, then $f'$ is integrable on $\mathbf{R}$.
\end{theorem}
\begin{proof}the theorem in the
    It suffices to assume that $f$ is increasing, and we shall start by proving case assuming $f$ is continuous. We define
    %
    \[ \Delta_h f (x) = \frac{f(x+h) - f(x)}{h} \]
    %
    and the four {\it Dini derivatives}
    %
    \[ D_+ f(x) = \liminf_{h \downarrow 0} \Delta_h f(x)\ \ \ \ \ D^+ f(x) = \limsup_{h \downarrow 0} \Delta_h f(x) \]
    \[ D_- f(x) = \liminf_{h \uparrow 0} \Delta_h f(x)\ \ \ \ \ D^- f(x) = \limsup_{h \uparrow 0} \Delta_h f(x) \]
    %
    Clearly, $D_+ f \leq D^+ f$ and $D_- f \leq D^- f$, It suffices to show $D^+ f(x) < \infty$ for almost every $x$, and $D^+ f(x) \leq D_- f(x)$ for almost every $x$, because if we consider the function $g(x) = -g(-x)$, then we obtain $D^- f(x) \leq D_+ f(x)$ for almost every $x$, so
    %
    \[ D^+ f (x) \leq D_- f(x) \leq D^- f(x) \leq D_+ f(x) \leq D^+ f(x) < \infty \]
    %
    for almost every $x$, implying all values are equal, and that the derivative exists at $x$.

    For a fixed $\gamma > 0$, consider $E_\gamma = \{ x: D^+ f (x) > \gamma \}$. Since each $\Delta_h f$ is continuous, the supremum of the $\Delta_h f$ over any index set is lower semicontinuous, and since
    %
    \[ D^+ f(x) = \lim_{h \to 0} \sup_{0 \leq s \leq h} \Delta_h f (x + s) \]
    %
    can be expressed as the countable limit of these lower semicontinuous functions, $D^+ f$ is measurable, hence $E_\gamma$ is measurable. Now consider the shifted function $g(x) = f(x) - \gamma x$. If $\bigcup (a_i,b_i)$ is the set obtainable from $g$ from the rising sun lemma, then $E_\gamma \subset \bigcup (a_i, b_i)$, for if $D^+ f(x) > \gamma$, then there is $h > 0$ arbitrarily small with $\Delta_h f(x) > \gamma$, hence $f(x + h) - f(x) > \gamma h$, hence $g(x+h) > g(x)$. We know that $g(a_k) \leq g(b_k)$, so $f(b_k) - f(a_k) \geq \gamma(b_k - a_k)$, so
    %
    \[ |E_\gamma| \leq \sum (b_k - a_k) \leq \frac{1}{\gamma} \sum f(b_k) - f(a_k) \leq \frac{f(b) - f(a)}{\gamma} \]
    %
    Thus $|E_\gamma| \to 0$ as $\gamma \downarrow 0$, implying $D^+ f(x) = \infty$ only on a set of measure zero.

    Now for two real numbers $r < R$, we will now show
    %
    \[ E = \{ a \leq x \leq b : D^+ f(x) > R\ \ \ D_-f(x) < r \} \]
    %
    is a set of measure zero. Letting $r$ and $R$ range over all rational numbers establishes that $D^+ f(x) \leq D_-f(x)$ almost surely. We assume $|E| > 0$ and derive a contradiction. By regularity, we may consider an open subset $U$ in $[a,b]$ containing $E$ such that $|U| < |E| (R/r)$. We can write $U$ as the union of disjoint intervals $I_n$. For a fixed $I_N$, apply the rising sun lemma to the function $rx - f(-x)$ on the interval $-I_N$, yielding a union of intervals $(a_n,b_n)$. If we now apply the rising sun lemma to the function $f(x) - Rx$ on $(a_n, b_n)$, we get intervals $(a_{nm}, b_{nm})$, whose union we denote $U_N$. Then
    %
    \[ R(b_{nm} - a_{nm}) \leq f(b_{nm}) - f(a_{nm})\ \ \ \ \ f(b_n) - f(a_n) \leq r(b_n - a_n) \]
    %
    then, because $f$ is increasing,
    %
    \begin{align*}
      |U_N| &= \sum_{nm} (b_{nm} - a_{nm}) \leq \frac{1}{R} \sum_{nm} (f(b_{nm}) - f(a_{nm}))\\
      &\leq \frac{1}{R} \sum f(b_n) - f(a_n) \leq \frac{r}{R} \sum_n (b_n - a_n) \leq \frac{r}{R} |I_N|
    \end{align*}
    %
    Now $E \cap I_N$ is contained in $U_N$, because if $x \in E \cap I_N$, then $D^+ f(x) > R$ and $D_- f(x) < r$, so we can sum in $N$ to conclude that
    %
    \[ |E| \leq \sum \frac{r}{R} |I_N| = \frac{r}{R} |U_N| < |E| \]
    %
    a contradiction proving the claim.
\end{proof}

\begin{corollary}
  If $f$ is increasing and continuous, then $f'$ is measurable, non-negative, and
  %
  \[ \int_a^b f'(x)\; dx \leq f(b) - f(a) \]
\end{corollary}
\begin{proof}
  The fact the $f'$ is measurable and non-negative results from the fact that the functions $g_n(x) = \Delta_{1/n} f(x)$ are non-negative and continuous, and $g_n \to f'$ almost surely. We know
  %
  \begin{align*}
    \int_a^b f'(x) &\leq \liminf_{n \to \infty} \int_a^b g_n(x) = \liminf_{n \to \infty} n \int_a^b [f(x + 1/n) - f(x)]\\
    &= \liminf_{n \to \infty} n \left[ \int_b^{b+1/n} f(x) - \int_a^{a + 1/n} f(x) \right] = f(b) - f(a)
  \end{align*}
  %
  where the last equality follows because $f$ is continuous.
\end{proof}

Even for increasing continuous functions, the inequality in the theorem above need not be an equality, as the next example shows, so we need something stronger to obtain our differentiation theorem.

\begin{example}
  The Cantor-Lebesgue function is a continuous increasing function $f$ from $[0,1]$ to itself, with $f(0) = 0$, and $f(1) = 1$, but with $f'(x) = 0$ almost everywhere. This means
  %
  \[ \int_0^1 f'(x) = 0 < 1 = f(1) - f(0) \]
  %
  so we cannot obtain equality in general. To construct $f$, consider the Cantor set $C = \bigcap C_k$, where $C_k$ is the disjoint union of $2^k$ closed intervals. Set $f_0 = 0$, and $f_1(0) = 0$, $f_1(x) = 1/2$ on $[1/3,2/3]$, $f_1(1) = 1$, and $f$ linear between $[0,1/3]$ and $[2/3,1]$. Similarily, set $f_2(0) = 0$, $f_2(x) = 1/4$ on $[1/9, 2/9]$, $f_2(x) = 1/2$ on $[1/3,2/3]$, $f_2(x) = 3/4$ on $[7/9,8/9]$, and $f_2(1) = 1$. The functions $f_i$ are increasing and cauchy in the uniform norm, so they converge to a continuous function $f$ called the \emph{Cantor function}. $f$ is constant on each interval in the complement of the cantor set, so $f'(x) = 0$ almost everywhere.
\end{example}

To obtain equality in the integral formula, we require additional conditions on our increasing functions, provided by absolute continuity.

\section{Absolute Continuity}

A function $f: [a,b] \to \mathbf{R}$ is \emph{absolutely continuous} if for every $\varepsilon > 0$, there is $\delta > 0$ such that whenever $(a_1, b_1), \dots, (a_n,b_n)$ are disjoint intervals with $\sum (b_i - a_i) < \delta$, $\sum |f(b_i) - f(a_i)| < \varepsilon$. Thus the function should be `essentially constant' over every set of zero measure. It is easy to see from this that absolutely continuous functions must be uniformly continuous, and have bounded variation. Thus $f$ has a decomposition into the difference of two continuous increasing functions, and one can see quite easily that these functions are also absolutely continuous. Most promising to us, if $f$ is a function defined by $f(x) = \int_a^x g(t)\ dt$, where $g \in L^1[a,b]$, then $f$ is absolutely continuous. This shows that absolute continuity is necessary in order to hope for the integral formula
%
\[ \int_a^b f'(x)\ dx = f(b) - f(a) \]
%
The Cantor function is {\it not} absolutely continuous, since it is constant except on the Cantor set, and we can cover the Cantor set by intervals with total length $(2/3)^n$ for each $n$. Thus it is impossible for the Cantor function to satisfy the fundamental theorem of calculus.

\begin{theorem}
  If $g \in L^1(\mathbf{R})$, and
  %
  \[ f(x) = \int_a^x g(t)\; dt \]
  %
  then $f$ is absolutely continuous.
\end{theorem}
\begin{proof}
  Fix $\varepsilon > 0$. We claim that there is $\delta$ such that if $|E| < \delta$, then $\int_E |g| < \varepsilon$. Otherwise there are sets $E_n$ with $|E_{n+1}| \leq |E_n|/3$ and with $\int_{E_n} |g| \geq \varepsilon$. Thus if we define the sets $E_m' = E_m - \bigcup_{n > m} E_n$ then the $E_m'$ and we have $|E_m| \sim |E_m|'$. Since $g$ is integrable, we must have $\sum \int_{E_n'} |g| < \infty$, so we conclude that as $N \to \infty$,
  %
  \[ \sum_{n \geq N} \int_{E_n'} |g| \to 0 \]
  %
  Yet for any $N$,
  %
  \[ \sum_{n \geq N} \int_{E_n'} |g| = \int_{E_N} |g| \geq \varepsilon \]
  %
  which is an impossibility. Thus such a $\delta$ exists for every $\varepsilon$, and so if we have disjoint intervals $(a_n,b_n)$ with $\sum (b_n - a_n) < \delta$, then
  %
  \[ \sum |f(b_n) - f(a_n)| = \sum \left| \int_{a_n}^{b_n} g(t) \right| \leq \sum \int_{a_n}^{b_n} |g| = \int_{\bigcup (a_n,b_n)} |g| < \varepsilon \]
  %
  which shows the function is absolutely continuous.
\end{proof}

To prove the differentiation theorem, we require a covering estimate not unlike that used to prove the Lebesgue differentiation theorem. We say a collection of balls is a \emph{Vitali covering} of a set $E$ if for every $x \in E$ and every $\eta > 0$, there is a ball $B$ in the cover containing $x$ with $|B| < \eta$. Thus every point is covered by an arbitrary small ball.

\begin{lemma}
    If $E$ is a set of finite measure, and $\{ B_\alpha \}$ is a Vitali covering of $E$, then for any $\delta > 0$, we can find finitely many disjoint balls $B_1, \dots, B_n$ in the covering such that
    %
    \[ \left| \bigcup B_i \right| = \sum |B_i| \geq |E| - \delta \]
\end{lemma}
\begin{proof}
    Without loss of generality, assume $\delta \leq |E|$. By inner regularity, pick a compact set $K \subset E$ with $|K| \geq |E| - \delta/2$. Then $K$ is covered by finitely many balls of radius less than $\eta$ in the covering $\{ B_\alpha \}$, and the elementary Vitali covering lemma gives a disjoint subcollection of balls $B_1, \dots, B_{n_0}$ with
    %
    \[ |K| \leq \left| \bigcup B_\alpha \right| \leq 3^d \sum |B_k| \]
    %
    so $\sum |B_k| \geq 3^{-d} |K|$. If $\sum |B_k| \geq |K| - \delta/2$, we're done. Otherwise, define $E_1 = K - \bigcup \overline{B_k}$. Then
    %
    \[ |E_1| \geq |K| - \sum |\overline{B_k}| = |K| - \sum |B_k| > \delta/2 \]
    %
    If we pick a compact set $K_1 \subset E_1$ with $|K_1| \geq \delta/2$, then if we remove all sets in the Vitali covering which intersect $B_1, \dots, B_{n_0}$, then we still obtain a Vitali covering for $K_1$, and we can repeat the argument above to find a disjoint collection of open sets $B_1^1, \dots, B_{n_1}^1$ with $\sum |B_k^1| \geq 3^{-d} |K_1|$. Then $\sum |B_k| + \sum |B^1_k| \geq 2 (3^{-d} \delta)$. If $\sum |B_k| + \sum |B^1_k| < |K| - \delta/2$, we repeat the same process, finding a disjoint family for $K_2 \subset E_2$, where $\smash{E_2 = K_1 - \bigcup \overline{B^1_k}}$. If this process repeats itself $k$ times, then we obtain a family of open sets with total measure greater than or equal to $k (3^{-d} \delta)$. But then if we eventually have $k \geq (|E| - \delta) 3^d/ \delta$, then the family of open sets satisfies the requirements of the theorem.
\end{proof}

\begin{corollary}
    We can arrange the choice of balls such that
    %
    \[ \left| E - \bigcup B_i \right| < 2\delta \]
\end{corollary}
\begin{proof}
    Let $E \subset U$, where $U$ is an open set with $|U - E| < \delta$. In the algorithm above, we may consider only balls in the Vitali covering as contained within $U$. But then
    %
    \[ \left| E - \bigcup B_i \right| \leq |U| - \sum |B_i| = |U| - \bigcup E_i \leq \delta + |E| - \sum |B_i| \leq 2\delta \]
    %
    and this gives the required bound.
\end{proof}

\begin{theorem}
    If $f: [a,b] \to \mathbf{R}$ is absolutely continuous, then $f'$ exists almost everywhere, and if $f'(x) = 0$ almost surely, then $f$ is constant.
\end{theorem}
\begin{proof}
    It suffices to prove that $f(a) = f(b)$, since we can then apply the theorem on any subinterval. Let $E = \{ x \in (a,b): f'(x) = 0 \}$. Then $|E| = b - a$. Fix $\varepsilon > 0$. Since for each $x \in E$, we have
    %
    \[ \lim_{h \to 0} \frac{f(x+h) - f(x)}{h} = 0 \]
    %
    This implies that the family of intervals $(x,y)$ such that the inequality $|f(y) - f(x)| \leq \varepsilon (y-x)$ holds forms a Vitali covering of $E$, and we may therefore select a family of disjoint intervals $I_i = (x_i,y_i)$ with
    %
    \[ \sum |I_i| \geq |E| - \delta = (b - a) - \delta \]
    %
    But $|f(y_i) - f(x_i)| \leq \varepsilon (y_i - x_i)$, so we conclude
    %
    \[ \sum |f(y_i) - f(x_i)| \leq \varepsilon (b - a) \]
    %
    The complement of $I_i$ is a union of intervals $J_i = (x_i',y_i')$ of total length $\leq \delta$. Applying the absolute continuity of $f$, we conclude
    %
    \[ \sum |f(y_i') - f(x_i')| \leq \varepsilon \]
    %
    so applying the triangle inequality,
    %
    \[ |f(b) - f(a)| \leq \sum |f(y_i') - f(x_i')| + \sum |f(y_i) - f(x_i)| \leq \varepsilon(b - a + 1) \]
    %
    We can then let $\varepsilon \to 0$ to obtain equality.
\end{proof}

\begin{theorem}
    Suppose $f$ is absolutely continuous on $[a,b]$. Then $f'$ exists almost every and is integrable, and
    %
    \[ f(b) - f(a) = \int_a^b f'(y)\ dy \]
    %
    so the fundamental theorem of calculus holds everywhere. Conversely, if $f \in L^1[a,b]$, then there is an absolutely continuous function $g$ with $g' = f$ almost everywhere.
\end{theorem}
\begin{proof}
    Since $f$ is absolutely continuous, we can write $f$ as the difference of two continuous increasing functions on $[a,b]$, and this easily implies $f$ is differentiable almost everywhere and is integrable on $[a,b]$. If $g(x) = \int_a^x f'(x)$, then $g$ is absolutely continuous, hence $g - f$ is also absolutely continuous. But we know that $(g - f)' = g' - f' = 0$ almost everywhere, so the last theorem implies that $g$ differs from $f$ by a constant. Since $g(a) = 0$, $g(x) = f(x) - f(a)$. The converse was proved exactly in our understanding of differentiating integrals.
\end{proof}

We now dwell slightly longer on the properties of absolutely continuous functions, which enables us to generalize other properties of integrals found in the calculus. We begin by noting that it is easy to verify that if $f$ and $g$ are absolutely continuous functions, then $fg$ is also absolutely continuous. We know $f'$, $g'$, and $(fg)'$ exist almost everywhere. But when all three exist simultaneously, the product rule gives $(fg)' = f'g + fg'$. The absolute continuity implies that
%
\[ \int_a^b f'g + fg' = \int_a^b (fg)' = f(b)g(b) - f(a)g(a) \]
%
Thus one can integrate a pair of absolutely continuous functions by parts. Next, we shall show that monotone absolutely continuous functions are precisely those we can use to change variables. One important thing to note is that even if $f$ is a continuous function, and $g$ is measurable, $g \circ f$ need not be measurable. The easy reason to see this is that the inverse image of every open set in $g$ is measurable, so in order to guarantee $g \circ f$ is measurable we need the inverse image of every measurable set under $f$ be measurable.

\begin{example}
  Consider the function $f(x) = \int_0^x \chi_E(x)\; dx$, where $E$ is a thick Cantor set. Then $f$ is absolutely continuous, strictly increasing on $[0,1]$, and maps $E$ to a set of measure zero. This is because $E = \lim E_n$, where $E_n$ is a family of intervals with $|E_n| \downarrow |E|$. Then $f(E_n)$ has total length $|E_n - E|$, so as $n \to \infty$, we see $\lim f(E_n) = f(E)$ has measure zero.  This means that $f(X)$ is measurable for any subset $X$ of $E$, and in particular, if $X$ is non-measurable, then $f^{-1}(f(X))$ cannot be measurable, even though $f(X)$ is measurable. Note that $f$ is strictly increasing even though it's derivatives vanish on a set of positive measure.
\end{example}

The next lemmas will show that even though $g \circ f$ may not be measurable, this does not really bother us too much when changing variables.

\begin{lemma}
  If $f$ is absolutely continuous, then it maps sets of measure zero to sets of measure zero.
\end{lemma}
\begin{proof}
  Let $E$ be a set of measure zero. Then for each $\delta > 0$, $E$ is coverable by a family of open intervals with total length $\delta$. But if $\delta$ is taken small enough, this means that $f(E)$ is coverable by a family of open intervals with total length bounded by $\varepsilon$, for any $\varepsilon$.
\end{proof}

This property of absolutely continuous functions is independant of the properties of the Euclidean domain as it's domain, and is used in the generalization of absolute continuity to more general domains, or even to measures. If $f$ is absolutely continuous, then the image of every interval is an interval, and since $f(\bigcup K_n) = \bigcup f(K_n)$, this implies that the image of a $F_\sigma$ set is measurable. But since every measurable set of $\mathbf{R}$ differs from a $F_\sigma$ set by a set of measure zero, the image of every Lebesgue measurable set is Lebesgue measurable. The reverse is almost true.

\begin{lemma}
  If $f$ is absolutely continuous, and $E$ measurable, then the set
  %
  \[ f^{-1}(E) \cap \{ x : f'(x) > 0 \} \]
  %
  is measurable.
\end{lemma}
\begin{proof}
  If $E$ is an open set, then
  %
  \[ |E| = \int_{f^{-1}(E)} f'(x)\; dx \]
  %
  It suffices to prove this when $E$ is an interval, and then this is just the theorem of differentiation for absolutely continuous functions. But then applying the dominated convergence theorem shows that this equation remains true if $E$ is an $G_\delta$ set. Furthermore, this means the theorem is true if $E$ is a closed set, and so by applying the monotone convergence theorem, the theorem is true if $E$ is an $F_\sigma$ set. But if $E$ is an arbitrary measurable set, then for every $\varepsilon$ there are $F_\sigma$ and $G_\delta$ sets $K \subset E \subset U$ with $|U - K| = 0$. But
  %
  \[ \alpha|f^{-1}(U - K) \cap \{ f' \geq \alpha \}| \leq \int_{f^{-1}(U-K)} f'(x)\; dx = |U - K| = 0 \]
  %
  Thus $f^{-1}(U-K) \cap \{ f' \geq \alpha \}$ is a set of measure zero, and so in particular by completeness, every set contained in this set is measurable, in particular $f^{-1}(U - E) \cap \{ f' \geq \alpha \}$ is measurable. But now this means
  %
  \[ \{ f' \geq \alpha \} - f^{-1}(U-E) \cap \{ f' \geq \alpha \} = f^{-1}(E) \cap \{ f' \geq \alpha \} \]
  %
  is measurable. Taking $\alpha \downarrow 0$ completes the proof.
\end{proof}

Because of this, even though $g \circ f$ is not necessarily measurable, $(g \circ f) f'$ is always measurable if $f$ is absolutely continuous. Thus the expression $\int (g \circ f) f'$ makes sense, and thus we can always interpret the change of variables formula.

\begin{theorem}
  If $f$ is absolutely continuous, and $g$ is integrable, then
  %
  \[ \int g(f(x)) f'(x)\; dx = \int g(y)\; dy \]
\end{theorem}
\begin{proof}
  Using the notation in the last proof, if $E$ is measurable, then
  %
  \[ |K| = \int_{f^{-1}(K)} f'(x)\; dx \leq \int_{f^{-1}(E)} f'(x)\; dx \leq \int_{f^{-1}(U)} f'(x)\; dx = |U| \]
  %
  and $|U| = |K| = |E|$, so that for any measurable set $E$,
  %
  \[ |E| = \int_{f^{-1}(E)} f'(x)\; dx \]
  %
  This imples the theorem we need to prove is true whenever $g$ is the characteristic function of any measurable set. But then by linearity, it is true for any simple function. By monotone convergence, it is then true for any non-negative function, and then by partitioning $g$ into the sum of simple functions, we obtain the theorem in general.
\end{proof}





\section{Differentiability of Jump Functions}

We now consider the differentiability of not necessarily continuous monotonic functions. Set $f$ to be an increasing function on $[a,b]$, which we may assume to be bounded.  Then the left and right limits of $f$ exist at every point, which we will denote by $f(x-)$ and $f(x+)$. Of course, we have $f(x-) \leq f(x) \leq f(x+)$. If there is a discontinuity, this means we are forced to have a `jump discontinuity' where $f$ skips an interval. This implies that $f$ can only have countably many such discontinuities, because a family of disjoint intervals on $\mathbf{R}$ is at most countable. Now define the jump function $\Delta f(x) = f(x^+) - f(x-)$, with $\theta(x) \in [0,1]$ defined such that $f(x_n) = f(x_n-) + \theta(x) \Delta f(x)$. If we define the functions
%
\[ j_y(x) = \begin{cases} 0 & : x < y \\ \theta(y) & : x = y \\ 1 & x > y \end{cases} \]
%
then we can define the \emph{jump function} associated with $f$ by
%
\[ J(x) = \sum_x \Delta f(x) j_n(x) \]
%
Since $f$ is bounded on $[a,b]$, we make the final observation that
%
\[ \sum_{x \in [a,b]} \Delta f(x) \leq f(b) - f(a) < \infty \]
%
so the series defining $J$ converges absolutely and uniformly.

\begin{lemma}
    If $f$ is increasing and bounded on $[a,b]$, then $J$ is discontinuous precisely at the values $x$ with $\Delta f(x) \neq 0$ with $\Delta J(x) = \Delta f(x)$. The function $f - J$ is continuous and increasing.
\end{lemma}
\begin{proof}
    If $x$ is a continuity point of $f$, then $j_y$ is continuous at $x$, and hence, because $\sum_y \Delta f(y) j_y(x) \to J(x)$ uniformly, so we conclude that $J$ is continuous at $x$. On the other hand, for each $y$, $j_y(y-) = 0$ and $j_y(y+) = 1$, and if we label the points of discontinuity of $f$ by $x_1, x_2, \dots$, then
    %
    \[ J(x) = \sum_{i = 1}^k \Delta f(x_i) j_{x_i} + \sum_{i = k+1}^\infty \Delta f(x_i) j_{x_i} \]
    %
    The right hand partial sums are continuous at $x_k$, whereas the left hand sum has a jump discontinuity of the same order as $f$ at $x_k$, we conclude $J$ also has this discontinuity. But this means that
    %
    \[ (f - J)(x_k+) - (f - J)(x_k-) = 0 \]
    %
    so $f - J$ is continuous at every point. $f - J$ is increasing because of the inequality
    %
    \[ J(y) - J(x) \leq \sum_{x < x_n \leq y} \alpha_n \leq f(y) - f(x) \]
    %
    which follows because $J$ is just the sum of jump discontinuities, and the right hand side because $f$ can decrease and increase outside of the jump discontinuities.
\end{proof}

Since $f - J$ is continuous and increasing, it is differentiable almost everywhere. It therefore remains to analyze the differentiability of the jump function $J$.

\begin{theorem}
    $J'$ exists and vanishes almost everywhere.
\end{theorem}
\begin{proof}
    Fix $\varepsilon > 0$, and consider
    %
    \[ E = \left\{ x \in [a,b]: \limsup_{h \to 0} \frac{J(x + h) - J(x)}{h} > \varepsilon \right\} \]
    %
    Then $E$ is measurable, because we can take the $\limsup$ over rational numbers because $J$ is increasing. We want to show it has measure zero. Suppose $\delta = |E|$. Consider $\eta > 0$ to be chosen later, and find $n$ such that $\sum_{k = n}^\infty \alpha_k < \eta$. Write
    %
    \[ J_0(x) = \sum_{n > N} \alpha_n j_n \]
    %
    Then $J_0(b) - J_0(a) < \eta$. Now $E$ differs from the set
    %
    \[ E' = \left\{ x \in [a,b]: \limsup_{h \to 0} \frac{J_0(x + h) - J_0(x)}{h} > \varepsilon \right\} \]
    %
    by finitely many points. Using inner regularity, find a compact set $K \subset E'$ with $|K| \geq \delta/2$. For each $x \in K$, we can find intervals $(\alpha_x, \beta_x)$ upon which $J_0(\beta_x) - J_0(\alpha_x) \geq \varepsilon |\beta_x - \alpha_x|$. But applying the elementary Vitali covering lemma, we can find a disjoint family of such intervals with $\sum (\beta_{x_i} - \alpha_{x_i}) \geq |K|/3 \geq \delta/6$. But now we find
    %
    \[ J_0(b) - J_0(a) \geq \sum J_0(\beta_{x_i}) - J_0(\alpha_{x_i}) \geq \varepsilon \delta/6 \]
    %
    This means $\delta \leq 6 \eta/\varepsilon$, and by letting $\eta \to 0$, we can conclude $\delta = 0$.
\end{proof}

This concludes our argument that {\it every} function of bounded variation has a derivative almost everywhere, because every such function can be uniquely written (up to a shift in the range of the functions) as the sum of a continuous function and a jump function. If $f$ is a function with bounded variation, then the function
%
\[ F(x) = \int_0^x f'(x) \]
%
is absolutely continuous, and $f - F$ is a continuous function with derivative zero almost everywhere. The fact that this decomposition is unique up to a shift as well (which can easily be seen in the case of an increasing function, from which the general case follows) leads us to refer to this as the \emph{Lebesgue decomposition} of a function of bounded variation on the real line.

\section{Rectifiable Curves}

We now consider the validity of the length formula
%
\[ L = \int_a^b (x'(t)^2 + y'(t)^2)^{1/2}\ dt \]
%
where $L$ is the length of the curve parameterized by $(x,y)$ on $[a,b]$. We cannot always expect this formula to hold, because if $x$ and $y$ are both the Cantor devil staircase function, then the formula above gives a length of zero, whereas we know the curve traces a line between $0$ and $1$, hence has length at least $\sqrt{2}$.

\begin{theorem}
    If a curve is parameterized by absolutely continuous functions $x$ and $y$ on $[a,b]$, then the curve is rectifiable, and has length
    %
    \[ \int_a^b (x'(t) + y'(t))^{1/2}\ dt \]
\end{theorem}
\begin{proof}
  This proof can be reworded as saying if $f$ is complex-valued and absolutely continuous, then it's total variation can be expressed as
  %
  \[ V(f,a,b) = \int_a^b |f'(t)|\; dt \]
  %
  If $P = \{ a \leq t_1 < \dots < t_n \leq b \}$ is a partition, then
  %
  \[ \sum |f(t_{n+1}) - f(t_n)| = \sum \left| \int_{t_n}^{t_{n+1}} f'(t)\; dt \right| \leq \sum \int_{t_n}^{t_{n+1}} |f'(t)|\; dt \leq \int_a^b |f'(t)|\; dt \]
  %
  so $V(f,a,b) \leq \int_a^b |f'(t)|\; dt$. To prove the converse inequality, fix $\varepsilon > 0$, and find a step function $g$ with $f' = g + h$, with $\| h \|_1 \leq \varepsilon$. If $G(x) = \int_a^x g(t)\; dt$ and $H(x) = \int_a^x h(t)\; dt$, then $F = G + H$, and $V(f,a,b) \geq V(G,a,b) - V(H,a,b) \geq V(G,a,b) - \varepsilon$, and if we partition $[a,b]$ into $a = t_0 < \dots < t_N$, where $G$ is constant on each $(t_n, t_{n+1})$, then
  %
  \begin{align*}
    V(G,a,b) &\geq \sum |G(t_n) - G(t_{n-1})| = \sum \left| \int_{t_{n-1}}^{t_n} g(t)\; dt \right|\\
    &= \sum \int_{t_{n-1}}^{t_n} |g(t)|\; dt = \int_a^b |g(t)|\; dt \geq \| f' \|_1 - \varepsilon
  \end{align*}
  %
  Letting $\varepsilon \to 0$ now gives the result.
\end{proof}

It is interesting to note that any rectifiable curve has a special {\it parameterization by arclength}, i.e. a parameterization $(x(t), y(t))$ such that if $L$ is the length function associated to the parameterization, then $L(A,B) = B - A$. This is obtainable by inverting the length function.

\begin{theorem}
  If $z = (x,y)$ is a parameterization of a rectifiable curve by arclength, then $x$ and $y$ are absolutely continuous, and $|z'| = 1$ almost everywhere.
\end{theorem}
\begin{proof}
  For any $s < t$,
  %
  \[ t - s = L(s,t) = V(f,s,t) \geq |z(t) - z(u)| \]
  %
  so it follows immediately that $|z|$ is an absolutely continuous function, and $|z'| \leq 1$ almost surely. But now we know that
  %
  \[ \int_a^b |z'(t)| = b - a \]
  %
  and this equality can now only hold if $|z'(t)| = 1$ almost surely.
\end{proof}

\section{Bounded Variation in Higher Dimensions}

Since the higher dimensional Euclidean domains do not have an ordering, it is impossible to define their length by partitioning their domain, and the meaning of a jump discontinuity is no longer clear. However, there are properties equivalent to having bounded variation which are more extendable to higher dimensions.

\begin{theorem}
  The following properties of $f: \mathbf{R} \to \mathbf{R}$ are equivalent, for some fixed finite constant $A$.
  %
  \begin{itemize}
    \item $f$ can be modified on a set of measure zero so that it has bounded variation not exceeding $A$.
    \item $\int |f(x+h) - f(x)| \leq A|h|$ for all $h \in \mathbf{R}$.
    \item For any $C^1$ function $\varphi$ with compact support, $\left| \int f(x) \varphi'(x) \right| \leq A \| \varphi \|_\infty$.
  \end{itemize}
\end{theorem}
\begin{proof}
  If $V(f) = A$, where $A < \infty$, then we can write $f = f^+ - f^-$, where $f^+$ and $f^-$ are both increasing functions, and with $V(f) = V(f^+) + V(f^-)$. It then follows that $|f(x+h) - f(x)| \leq (f^+(x+h) - f^+(x)) + |f^-(x+h) - f^-(x)|$, so it suffices to prove the second property by assuming $f$ is increasing. But then by the monotone convergence theorem, assuming $h > 0$ without loss of generality,
  %
  \[ \int |f(x+h) - f(x)| = \lim_{y \to \infty} \int_{-y}^y f(x+h) - f(x) = \lim_{y \to \infty} \int_y^{y+h} f(x) - \int_{-y-h}^{-y} f(x) \]
  %
  The first term of the limit converges to $hV(f)$, and the second to zero, completing the first part of the theorem. Now assuming the second point, we prove the third point. Then using the second point, we find
  %
  \begin{align*}
    \left|\int f(x) \varphi'(x) \right| &= \left| \lim_{h \to 0} \int f(x) \frac{\varphi(x+h) - \varphi(x)}{h} \right|\\
    &= \left| \lim_{h \to 0} \int \frac{f(x-h) - f(x)}{h} \varphi(x) \right| \leq A \| \varphi \|_\infty
  \end{align*}
  %
  Finally, we consider the third point being true. The set of all partitions with rational points is countable. Suppose that for each rational $P = \{ t_0 < \dots < t_N \}$ there is a set $E_P$ of measure zero for each rational partition $P$ such that
  %
  \[ \sum_{n = 1}^N \sup_{\substack{x \in [t_{n-1},t_n]\\x \not \in E_P}} f(x) - \inf_{\substack{x \in [t_{n-1},t_n]\\x \not \in E_P}} f(x) \leq A \]
  %
  Then the union of $E_P$ over all rational $P$ has measure zero. We can modify $f$ on $E_P$ by setting $f(x) = \liminf_{y \to 0} f(x+y)$, and then $V(f,P) \leq A$ for all rational partitions $P$. If $Q$ is now any partition, we can find a rational partition $P$ with $V(f,P) \geq V(f,Q) - \varepsilon$, and so $V(f,P) \leq A - \varepsilon$. Taking $\varepsilon \to 0$ completes the argument. Thus if $f$ cannot be modified to have finite variation $A$, there exists a rational partition $P$ such that for any set $E$ of measure zero,
  %
  \[ \sum_{n = 1}^N \sup_{\substack{x \in [t_{n-1},t_n]\\x \not \in E}} f(x) - \inf_{\substack{x \in [t_{n-1},t_n]\\x \not \in E}} f(x) > A \]
  %
  Thus for any $\varepsilon$, there exists $E_n^+, E_n^- \subset [t_{n-1},t_n]$ of positive measure such that
  %
  \[ \sum_{n = 1}^N \inf_{x \in E_n^+} f(x) - \sup_{x \in E_n^-} f(x) > A \]
  %
  If we consider the polygonal function $\phi$ which
\end{proof}

\section{Minkowski Content}

Given a set $K \in \mathbf{R}^n$, we let $K^\delta$ denote the open set consisting of points $x$ with $d(x,K) < \delta$. The $m$ dimensional \emph{Minkowski content} of $K$ is defined to be
%
\[ \lim_{\delta \to 0} \frac{1}{\alpha(n-m)} \frac{|K^\delta|}{\delta^m} \]
%
where $\alpha(d)$ is the volume of the unit ball in $d$ dimensions. When this limit exists, we denote it by $M^m(K)$. In this section, we mainly discuss the one dimensional Minkowski content in two dimensions, i.e. the values of
%
\[ \lim_{\delta \to 0} \frac{|K^\delta|}{2 \delta^m} \]
%
and it's relation the length of curves. Since we now only care about the one dimensional Minkowski content, we let $M(K)$ denote the one dimension Minkowski content.

\begin{lemma}
  If $\Gamma = \{ z(t): a \leq t \leq b \}$ is a curve, and $\Delta$ is the distance between the endpoints of the curve, then $|\Gamma^\delta| \geq 2 \delta \Delta$.
\end{lemma}
\begin{proof}
  By rotating, we may assume that both endpoints of the curve lie on the $x$ axis, so $z(a) = (A,0)$, $z(b) = (B,0)$ with $A < B$, so $\Delta = B - A$. If $\Delta = 0$, the theorem is obvious. Otherwise, for each point $x \in [A,B]$ there is $t(x)$ such that if $z_1(t(x)) = x$, and so $\Gamma^\delta$ contains $x \times [z_2(t(x)) - \delta, z_2(t(x)) + \delta]$, which has length $2 \delta$. Thus by Fubini's theorem,
  %
  \[ |\Gamma^\delta| = \int_{-\infty}^\infty \int_{-\infty}^\infty \chi_{\Gamma^\delta}(x,y)\; dx \;dy \geq \int_A^B 2 \delta = 2 \delta \Delta \]
  %
  so the theorem is proved.
\end{proof}

\begin{theorem}
  If $\Gamma = \{ z(t): a \leq t \leq b \}$ is a quasi-simple curve (simple except at finitely many points), then the Minkowski content of $\Gamma$ exists if and only if $\Gamma$ is rectifiable, and in this case $M^1(\Gamma)$ is the length of the curve $L$.
\end{theorem}
\begin{proof}
  To prove the theorem, we consider the upper and lower Minkowski contents
  %
  \[ M^*(\Gamma) = \limsup_{\delta\to 0} \frac{|\Gamma|^\delta}{\alpha(n-1) \delta}\ \ \ \ M_*(\Gamma) = \liminf_{\delta \to 0} \frac{|\Gamma|^\delta}{\alpha(n-1) \delta} \]
  %
  First, we prove that $M^*(\Gamma) \leq L$. Consider a partition $P$ of $[a,b]$, and let $L_P$ be the length of the polygonal approximation to the curve. By refining the partition, we may assume that $\Gamma$ is simple, with the repeated points at the boundaries of the intervals. For each interval $I_n$ in the partition, we select a closed subinterval $J_n = [t_n,u_n]$ such that $\Gamma$ is simple on $\bigcup J_n$, and
  %
  \[ \sum |z(u_n) - z(t_n)| \geq L_P - \varepsilon \]
  %
  Since the intervals $J_n$ are disjoint, for suitably small $\delta$ the sets $J_n^\delta$ are disjoint. Applying the previous lemma, we conclude that
  %
  \[ |\Gamma^\delta| \geq \sum |J_n^\delta| \geq 2 \delta \sum |z(u_n) - z(t_n)| = 2 \delta (L_p - \varepsilon) \]
  %
  First, by letting $\varepsilon \to 0$ and then $\delta \to 0$, we conclude that $M_*(\Gamma) \geq \lim_P L_P$. In particular, this shows that if $\Gamma$ has Minkowski content one, then the curve is rectifiable. Conversely, we consider the functions
  %
  \[ F_n(s) = \sup_{0 < |h| < 1/n} \left| \frac{z(s+h) - z(s)}{h} - z'(s) \right| \]
  %
  Because $z$ is continuous, this supremum can be considered over a countable, dense subset, and so each $F_n$ is measurable. Since $F_n(s) \to 0$ for almost every $s$, we can apply Egorov's theorem to show that this limit is uniform except on a singular set $E$ with $|E| < \varepsilon$, so that for some large $N$, for $s \not \in E$ and $|h| < 1/N$, $|z(s+h) - z(s) - hz'(s)| < \varepsilon h$. We now split the interval $[a,b]$ into consecutive intervals $I_1, \dots, I_{M+1}$, with each interval but $I_{M+1}$ having length $1/N$. We let $\Gamma_n$ denote the section of the curve travelled along the interval $I_n$. Thus $|\Gamma^\delta| \leq \sum |\Gamma_n^\delta|$. If an interval $I_n$ contains an element of $E^c$, we say $I_n$ is a `good' interval. Then we can pick an element $x_n \in I_n$ for which for any $x \in I_n$,
  %
  \[ |z(x) - z(x_n) - (x - x_n) z'(x_n)| < \varepsilon |x - x_n| < \varepsilon / N \]
  %
  Thus $\Gamma_n$ is covered by a $\varepsilon / N$ thickening of a length $1/N$ line $J_n$ in $\mathbf{R}^2$ through $z(x_n)$ with slope $z'(x_n)$. Thus if $\varepsilon \leq 1$, we conclude
  %
  \begin{align*}
    |\Gamma_n^\delta| &\leq J_n^{\varepsilon/N + \delta} \leq (1/N + 2\varepsilon/N + 2 \delta)(2\varepsilon/N + 2\delta)\\
    \leq 2 \delta/N + O(\delta \varepsilon / N + \delta^2 + \varepsilon/N^2)
  \end{align*}
  %
  Since $M \leq NL$, if we take the sum of $|\Gamma_n^\delta|$ over all `good' intervals we obtain an upper bound of
  %
  \[ NL \left( 2 \delta/N + O(\delta \varepsilon / N + \delta^2 + \varepsilon/N^2) \right) = 2 \delta L + O(\delta \varepsilon + \delta^2 N + \varepsilon/N) \]
  %
  On the other hand, if $I_n$ is contained within $E$, or if $n = M+1$, we say $I_n$ is a bad interval. Since $E$ has total measure bounded by $\varepsilon$, there can be at most $\varepsilon N + 1$ bad intervals. On these intervals we use the crude estimate $|z(t) - z(u)| \leq |t-u|$ (true because $z$ is an arclength parameterization) to show $\Gamma_n$ is contained in a rectangle with sidelengths $1/N$, so we obtain that $|\Gamma_n^\delta| \leq (1/N + 2\delta)^2 = O(1/N^2 + \delta^2)$. Thus the sum of $|\Gamma_n^\delta|$ over the `bad intervals' is bounded by
  %
  \[ O(\varepsilon/N + 1/N^2 + \varepsilon N \delta^2 + \delta^2) \]
  %
  In particular, the sum of the two bounds gives
  %
  \[ |\Gamma^\delta| \leq 2 \delta L + O(\delta \varepsilon + \delta^2 N + \varepsilon/N + 1/N^2) \]
  %
  Or
  %
  \[ \frac{|\Gamma^\delta|}{2 \delta} \leq L + O(\varepsilon + \delta N + \varepsilon/N + 1/N^2) \]
  %
  If we choose $N \geq 1/\delta$, we get that
  %
  \[ \frac{|\Gamma^\delta|}{2 \delta} \leq L + O(\varepsilon + \delta N + \delta \varepsilon) = L + O(\varepsilon + \delta N) \]
  %
  Letting $\delta \downarrow 0$, we conclude that $M^*(\Gamma) \leq L + O(\varepsilon)$, and we can then let $\varepsilon \downarrow 0$ to conclude $M^*(\Gamma) \leq L$. This completes the proof that if $\Gamma$ is rectifiable, then $\Gamma$ has one dimensional Minkowski content, and $M(\Gamma) = L$.
\end{proof}

If $\Gamma$ is rectifiable, it is parameterizable by a Lipschitz map (the arclength parameterization). If we instead consider a curve parameterizable by a map $z$ which is Lipschitz of order $\alpha$, which may no longer be absolutely continuous, but still has a decay very similar to the Minkowski dimension decay.

\begin{theorem}
  If $z$ is a planar curve which is Lipschitz of order $\alpha > 1/2$, then it's trace $\Gamma$ satisfies $|\Gamma^\delta| = O(\delta^{2-1/\alpha})$.
\end{theorem}
\begin{proof}
  Since $|z(t) - z(s)| \leq |t - s|^\alpha$, we can cover $z$ by $O(N)$ radius $1/N^\alpha$ balls, so $|\Gamma| \lesssim N^{1-2\alpha}$, and so $|\Gamma^\delta| \lesssim N (1/N^\alpha + \delta)^2$. Setting $N = \delta^{-1/\alpha} + O(1)$ gives $|\Gamma^\delta| \lesssim \delta^{2-\alpha - 1/\alpha}$.
\end{proof}

\section{The Isoperimetric Inequality}

We now use our Minkowski content techniques to prove the isoperimetric inequality, which asks us to find the region in the plane with largest area whose boundary has a bounded length $L$. We suppose $\Omega$ is a bounded region of the plane, whose boundary $\partial \Omega$ is a rectifiable curve with length $L$. In particular, we shall find the region with the largest area whose boundary has a fixed length are balls. A key inequality used in the proof is the Brun Minkowski inequality, which lowers bounds the measure of $A+B$ in terms of $A$ and $B$. If we hope for an estimate $|A+B|^\alpha \gtrsim |A|^\alpha + |B|^\alpha$, then taking $B = \alpha A$, where $A$ is convex and, for which $A + \alpha A = (1 + \alpha)A$, we find $(1 + \alpha)^{d\alpha} \gtrsim (1 + \alpha^{d\alpha})$. Thus $\alpha \geq 1/d$.

\begin{lemma}
  If $A$, $B$, and $A+B$ are measurable, $|A + B|^{1/d} \geq |A|^{1/d} + |B|^{1/d}$.
\end{lemma}
\begin{proof}
  Suppose first that $A$ and $B$ are rectangles with side lengths $x_n$ and $y_n$. Then the Minkowski inequality becomes
  %
  \[ \left( \prod (x_n + y_n) \right)^{1/d} \geq \left( \prod x_n \right)^{1/d} + \left( \prod y_n \right)^{1/d} \]
  %
  Replacing $x_n$ with $\lambda_n x_n$ and $y_n$ with $\lambda_n y_n$, we find that we may assume $x_n + y_n = 1$, and so we must prove that for any $x_n \leq 1$,
  %
  \[ \left( \prod x_n \right)^{1/d} + \left( \prod (1 - x_n) \right)^{1/d} \leq 1 \]
  %
  But this inequality is an immediate consequence of the arithmetic geometric mean inequality. Thus the case is proved. Next, we suppose $A$ and $B$ are unions of disjoint closed rectangles, and we prove the inequality by induction on the number of rectangles. Without loss of generality, by symmetry in $A$ and $B$, we may assume that $A$ has at least two rectangles $R_1$ and $R_2$. Since the inequality is translation invariant separately in $A$ and $B$, and $R_1$ and $R_2$ is disjoint, hence separated by a coordinate axis, we may assume there exists an index $j$ such that every element $x$ of $R_1$ has $x_1 < 0$ and every element $x$ of $R_2$ has $x_1 > 0$. Let $A^+ = A \cap \{ x_j \leq 0 \}$ and $A^- = A \cap \{ x_j \geq 0\}$. Next, we translate $B$ such that if $B^{\pm}$ are defined similarily, then
  %
  \[ \frac{|B^{\pm}|}{|B|} = \frac{|A^{\pm}|}{|A|} \]
  %
  Note that $A+B$ contains the union of $A^+ + B^+$ and $A^- + B^-$, and this union is disjoint. Thus by induction,
  %
  \begin{align*}
    |A+B| &\geq |A^+ + B^+| + |A^- + B^-|\\
    &\geq (|A^+|^{1/d} + |B^+|^{1/d})^d + (|A^-|^{1/d} + |B^-|^{1/d})^d\\
    &= |A^+| \left( 1 + \left( \frac{|B|^+}{|A|^+} \right)^{1/d} \right)^d + |A^-| \left( 1 + \left( \frac{|B|^-}{|A|^-} \right) \right)^d\\
    &= (|A|^{1/d} + |B|^{1/d})^d
  \end{align*}
  %
  Thus the proof is completed for unions of rectangles. The proof then passes to open sets by approximating open sets by closed rectangles contained within. Then we can pass to where $A$ and $B$ are compact sets, since then $A+B$ is compact, and so if we consider the open thickenings $A^\varepsilon$, $B^\varepsilon$, and $(A+B)^\varepsilon$, then
  %
  \[ |A| = \lim |A^\varepsilon|\ \ \ |B| = \lim |B^\varepsilon|\ \ \ |A + B| = \lim |(A + B)^\varepsilon| \]
  %
  and $(A+B)^\varepsilon \subset A^\varepsilon + B^\varepsilon \subset (A + B)^{2\varepsilon}$. Finally, we can use inner regularity to obtain the theorem in full.
\end{proof}

\begin{theorem}
  For any region $\Omega$, $4 \pi |\Omega| \leq L^2$.
\end{theorem}
\begin{proof}
  For $\delta > 0$, consider
  %
  \[ \Omega_+(\delta) = \{ x: d(x,\Omega) < \delta \}\ \ \ \ \Omega_-(\delta) = \{ x : d(x,\Omega^c) \geq \delta \} \]
  %
  Then we have a disjoint union $\Omega_+(\delta) = \Omega_-(\delta) + \Gamma^\delta$, where $\Gamma$ is the boundary curve of $\Omega$. Furthermore, $\Omega_+(\delta)$ contains $\Omega + B_\delta$, and $\Omega$ contains $\Omega_-(\delta) + B_\delta$. Applying the Brun Minkowski inequality, we conclude
  %
  \[ |\Omega_+(\delta)| \geq (|\Omega|^{1/2} + \pi^{1/2} \delta)^2 \geq |\Omega| + 2 \pi^{1/2} \delta |\Omega|^{1/2} \]
  \[ |\Omega| \geq (|\Omega_-(\delta)|^{1/2} + \pi^{1/2} \delta)^2 \geq |\Omega_-(\delta)| + 2 \pi^{1/2} \delta |\Omega_-(\delta)|^{1/2} \]
  %
  But
  %
  \[ |\Gamma^\delta| = |\Omega_+(\delta)| - |\Omega_-(\delta)| \geq 2 \pi^{1/2} \delta \left( |\Omega|^{1/2} + |\Omega_-(\delta)|^{1/2} \right) \]
  %
  Dividing by $2\delta$ and letting $\delta \to 0$, we conclude $L \geq 2 \pi^{1/2} |\Omega|^{1/2}$. This is precisely the inequality we need.
\end{proof}

Using some Fourier analysis, we can prove that the only smooth curves which make this inequality tight are circles. Indeed, if a closed $C^1$ curve $\Gamma = \{ z(t): a \leq t \leq b \}$ is given, then Green's theorem implies the area of its interior is given by
%
\[ \frac{1}{2} \left| \int_\Gamma x\; dy - y\; dx \right| = \frac{1}{2} \left| \int_a^b x(t) y'(t) - y(t) x'(t) \right| \]
%
We then take a Fourier series in $x$ and $y$.

\begin{theorem}
  The only curves $\Gamma$ with rectifiable boundary such that $A = \pi (L/2)^2$ are circles.
\end{theorem}
\begin{proof}
By normalizing, we may assume $z$ is an arcline parameterization, and $\Gamma$ has length $2\pi$, so $z:[0,2\pi] \to \mathbf{R}^2$, and $z$ is absolutely continuous. If $x(t) \sim \sum a_n e^{nit}$ and $y(t) \sim \sum b_n e^{int}$, then $x'(t) \sim \sum i n a_n e^{n i t}$ and $y(t) \sim \sum i n b_n e^{nit}$. Parseval's equality implies
%
\[ \int_0^{2\pi} x(t) y'(t) - y(t) x'(t) = 2 \pi i \sum n (b_n \overline{a_n} - a_n \overline{b_n}) \]
%
Thus the area of the curve is precisely
%
\[ \pi \left| \sum n (b_n \overline{a_n} - a_n \overline{b_n}) \right| \leq \pi \sum 2n|b_na_n| \leq \pi \sum |n|(|a_n|^2 + |b_n|^2) \]
%
On the other hand, the length constraint implies that, since $|z'(t)| = 1$,
%
\[ 1 = \frac{1}{2\pi} \int_0^{2\pi} x'(t)^2 + y'(t)^2 = \sum |n|^2(|a_n|^2 + |b_n|^2) \]
%
If $A = \pi$, then
%
\[ \sum |n| (|a_n|^2 + |b_n|^2) \geq 1 = \sum |n|^2 (|a_n|^2 + |b_n|^2) \]
%
This means we cannot have $|n| < |n|^2$ whenever $a_n$ or $b_n$ is nonzero. Thus the Fourier support of $x$ and $y$ is precisely $\{ -1, 0, 1 \}$. Since $x$ is real valued, $a_1 = \overline{a_{-1}} = a$, $b_1 = \overline{b_{-1}}$. We thus have $2(|a_1|^2 + |b_1|^2) = 1$, and since we must have $a$ a scalar multiple of $b$ so the Cauchy Schwarz inequality application becomes an equality, we must have $|a_1| = |b_1| = 1/2$. If $a_1 = e^{i\alpha}/2$ and $b_1 = e^{i\beta}/2$, the fact that $1 = 2|a_1\overline{b_1} - \overline{a_1}b_1|$ implies $|\sin(\alpha - \beta)| = 1$, hence $\alpha - \beta = k \pi /2$, where $k$ is an odd integer. Thus $x(s) = \cos(\alpha + s)$, and $y(s) = \cos(\beta + s)$, which parameterizes a circle.
\end{proof}





\begin{theorem}
    If $\phi, \psi \in C_c^\infty(\RR^n)$, then $\Lambda * (\phi * \psi) = (\Lambda * \phi) * \psi$.
\end{theorem}
\begin{proof}
  Let $K$ be a compact set containing the supports of $\phi$ and $\psi$. It is simple to verify that for each $x \in \RR^d$,
    %
    \[ (\phi * \psi)^*(x) = \int \phi^*(x + y) \psi(y)\; dy = \int (T_y \phi^*)(x) \psi(y)\; dy \]
    %
    since the map $y \mapsto (T_y \phi)^* \psi(y)$ is continuous, and vanishes out of the compact set $K$, we can consider the $C_c^\infty(K)$ valued integral
    %
    \[ (\phi * \psi)^* = \int_K \psi^*(y) T_y \phi^*\; ds \]
    %
    This means precisely that
    %
    \begin{align*}
        (\Lambda * (\phi * \psi))(0) &= \Lambda((\phi * \psi)^*) = \int_K \psi^*(y) \Lambda(T_y \phi^*)\; dy\\
        &= \int_K \psi^*(y) (\Lambda * \phi)(y)\; dy = ((\Lambda * \phi) * \psi)(0)
    \end{align*}
    %
    The commutativity in general results from applying the commutativity of the translation operators.
\end{proof}

A net $\{ \phi_\alpha \}$ is known as an {\it approximate identity} in the space of distributions if $\Lambda * \phi_\alpha \to \Lambda$ weakly as $\alpha \to \infty$, for every distribution $\Lambda$, and an approximate identity in the space of test functions if $\psi * \phi_\alpha \to \psi$ in $C_c^\infty(\RR^n)$.

\begin{theorem}
    If $\phi_\alpha$ is a family of non-negative functions in $C_c^\infty(\RR^n)$ which are eventually supported on every neighbourhood of the origin, and integrate to one, then $\phi_\alpha$ is an approximation to the identity in the space of test functions and in the space of distributions.
\end{theorem}
\begin{proof}
    It is easy to verify that if $f$ is a continuous function, then $f * \phi_\delta$ converges locally uniformly to $f$ as $\delta \to 0$. But now we calculate that if $f \in C_c^\infty(\RR^n)$, then $D^\alpha(f * \phi_\delta) = (D^\alpha f) * \phi_\delta$ converges locally uniformly to $D^\alpha \phi$, which gives that $f * \phi$ converges to $f$ in $C_c^\infty(\RR^n)$. Now if $\Lambda$ is a distribution, and $\psi$ is a test function, then continuity gives
    %
    \begin{align*}
        \Lambda(\psi^*) &= \lim_{\delta \to 0} \Lambda(\phi_\delta * \psi) = \lim_{\delta \to 0} (\Lambda * (\phi_\delta * \psi))(0)\\
        &= \lim_{\delta \to 0} ((\Lambda * \phi_\delta) * \psi)(0) = \lim_{\delta \to 0} (\Lambda * \phi_\delta)(\psi^*)
    \end{align*}
    %
    and $\psi$ was arbitrary.
\end{proof}

If $\Lambda$ is a distribution on $\RR^n$, then the map $\phi \mapsto \Lambda * \phi$ is a linear transformation from $C_c^\infty(\RR^n)$ into $C^\infty(\RR^n)$, which commutes with translations. It is also continuous. To see this, we consider a fixed compact $K$, and consider the map from $C_c^\infty(K)$ to $C^\infty(\RR^n)$. We can apply the closed graph theorem to prove continuity, so we assume the existence of $\phi_1, \phi_2, \dots$ converging to $\phi$ in $C_c^\infty(K)$ and $\Lambda * \phi_1, \Lambda * \phi_2, \dots$ converges to $f$. It suffices to show $f = \Lambda * \phi$. But we calculate that for each $x \in \RR^d$,
%
\[ f(x) = \lim (\Lambda * \phi_n)(x) = \lim \Lambda(T_x \phi^*_n) = \Lambda (\lim T_x \phi^*_n) = \Lambda(T_x \phi^*) = (\Lambda * \phi)(x). \]
%
Here we have used the fact that $T_x \phi_n^*$ converges to $T_x \phi^*$ in $C_c^\infty(\RR^n)$. Suprisingly, the converse is also true.

\begin{theorem}
    If $L: C_c^\infty(\RR^n) \to C^\infty(\RR^n)$ and commutes with translations, then there is a distribution $\Lambda$ such that $L(\phi) = \Lambda * \phi$.
\end{theorem}
\begin{proof}
    If $L(\phi) = \Lambda * \phi$, then we would have
    %
    \[ \Lambda(\phi) = (\Lambda * \phi^*)(0) = L(\phi^*)(0) \]
    %
    and we take this as the definition of $\Lambda$ for an arbitrary operator $L$. Indeed, it then follows that $\Lambda$ is continuous because all the operations here are continuous, and because $L$ commutes with translations, we conclude
    %
    \[ (\Lambda * \phi)(x) = \Lambda(T_x \phi^*) = L(T_{-x} \phi)(0) = L(\phi)(x) \]
    %
    which gives the theorem.
\end{proof}

We now move onto the case where a distribution $\Lambda$ has compact support. Then $\Lambda$ extends to a continuous functional on $C^\infty(\RR^n)$, and we can define the convolution $\Lambda * \phi$ if $\phi \in C^\infty(\RR^n)$. The same techniques as before verify that translations and derivatives are carried into the convolution.

\begin{theorem}
    If $\phi$ and $\Lambda$ have compact support, then $\Lambda * \phi$ has compact support.
\end{theorem}
\begin{proof}
    Let $\phi$ and $\Lambda$ be supported on $K$. Then $(\Lambda * \phi)(x) = \Lambda(T_x \phi^*)$. Since $T_x \phi^*$ is supported on $x - K$, for $x$ large enough $x-K$ is disjoint from $K$, and so $\Lambda * \phi$ vanishes outside of $K + K$.
\end{proof}

\begin{theorem}
    If $\Lambda$ and $\psi$ have compact support, and $\phi \in C^\infty(\RR^n)$, then
    %
    \[ \Lambda * (\phi * \psi) = (\Lambda * \phi) * \psi = (\Lambda * \psi) * \phi \]
\end{theorem}
\begin{proof}
    Let $\Lambda$ and $\psi$ be supported on some balanced compact set $K$. Let $V$ be a bounded, balanced open set containing $K$. If $\phi_0$ is a function with compact support equal to $\phi$ on $V + K$, then for $x \in V$,
    %
    \[ (\phi * \psi)(x) = \int \phi(x - y) \psi(y)\; dy = \int \phi_0(x - y) \psi(y)\; dy = (\phi_0 * \psi)(x) \]
    %
    Thus
    %
    \[ (\Lambda * (\phi * \psi))(0) = (\Lambda * (\phi_0 * \psi))(0) = ((\Lambda * \psi) * \phi_0)(0) \]
    %
    But $\Lambda * \psi$ is supported on $K + K$, so $((\Lambda * \psi) * \phi_0)(0) = ((\Lambda * \psi) * \phi)(0)$. Now we also calculate
    %
    \[ (\Lambda * (\phi * \psi))(0) = ((\Lambda * \phi_0) * \psi)(0) = ((\Lambda * \phi) * \psi)(0) \int (\Lambda * \phi_0)(-y) \psi(y) \]
    %
    where the last fact follows because $\Lambda * \phi_0$ agrees with $\Lambda * \phi$ on $K$. The general fact follows by applying the translation operators.
\end{proof}

Now we come to the grand finale, defining the convolution of two distributions. Given two distributions $\Lambda$ and $\Psi$, one of which has compact support, we define the linear operator
%
\[ L(\phi) = \Lambda * (\Psi * \phi) \]
%
Then $L$ commutes with translations, and is continuous, because if we have $\phi_1, \phi_2, \dots$ converging to $\phi$ in $C_c^\infty(K)$, then $\Psi * \phi_n$ converges to $\Psi * \phi$ in $C^\infty(\RR^n)$. If $\Psi$ is supported on a compact support $C$, then the $\Psi * \phi_n$ have common compact support $C + K$, and actually converge in $C_c^\infty(C + K)$, hence $\Lambda * (\Psi * \phi_n)$ converges to $\Lambda * (\Psi * \phi)$. Conversely, if $\Lambda$ has compact support, then $\Psi * \phi_n$ converges in $C^\infty(\RR^n)$, which implies $\Lambda * (\Psi * \phi_n)$ converges to $\Lambda * (\Psi * \phi)$ in $C^\infty(\RR^n)$. Thus $L$ corresponds to a distribution, and we define this distribution to be $\Lambda * \Psi$.

\begin{theorem}
    If $\Lambda$ and $\Psi$ are distributions, one of which has compact support, then $\Lambda * \Psi = \Psi * \Lambda$. Let $S_\Lambda$ and $S_\Psi$, and $S_{\Lambda * \Psi}$ denote the supports of $\Lambda$, $\Psi$, and $\Lambda * \Psi$. Then $\Lambda * \Psi = \Psi * \Lambda$, and $S_{\Lambda * \Psi} \subset S_\Lambda + S_\Psi$.
\end{theorem}
\begin{proof}
    We calculate that for any two test functions $\phi$ and $\psi$,
    %
    \[ (\Lambda * \Psi) * (\phi * \psi) = \Lambda * (\Psi * (\phi * \psi)) = \Lambda * ((\Psi * \phi) * \psi) \]
    %
    If $\Lambda$ has compact support, then
    %
    \[ \Lambda * ((\Psi * \phi) * \psi) = (\Lambda * \psi) * (\Psi * \phi) \]
    %
    Conversely, if $\Psi$ has compact support, then
    %
    \[ \Lambda * ((\Psi * \phi) * \psi) = \Lambda * (\psi * (\Psi * \phi)) = (\Lambda * \psi) * (\Psi * \phi) \]
    %
    We also calculate
    %
    \begin{align*}
        \Psi * ((\Lambda * \phi) * \psi) &= \Psi * (\Lambda * (\phi * \psi)) = \Psi * (\Lambda * (\psi * \phi))\\
        &= \Psi * ((\Lambda * \psi) * \phi) = (\Psi * \phi) * (\Lambda * \psi)
    \end{align*}
    %
    But since convolution is commutative, we have
    %
    \[ ((\Lambda * (\Psi * \phi)) * \psi) = \Lambda * ((\Psi * \phi) * \psi) = \Psi * ((\Lambda * \phi) * \psi) = (\Psi * (\Lambda * \phi)) * \psi \]
    %
    Since $\psi$ was arbitrary, we conclude
    %
    \[ (\Lambda * \Psi) * \phi = \Lambda * (\Psi * \phi) = \Psi * (\Lambda * \phi) = (\Psi * \Lambda) * \phi \]
    %
    and now since $\phi$ was arbitrary, we conclude $\Lambda * \Psi = \Psi * \Lambda$. Now we know convolution is commuatative, we may assume $S_\Psi$ is compact. The support of $\Psi * \phi^*$ lies in $S_\Psi - S_\phi$. But this means that if $S_\phi - S_\Psi$ is disjoint from $S_\Lambda$, which means exactly that $S_\phi$ is disjoint from $S_\Lambda + S_\Psi$, then
    %
    \[ (\Lambda * \Psi)(\phi) = (\Lambda * (\Psi * \phi))(0) = 0 \]
    %
    and this gives the support of $\Lambda * \Psi$.
\end{proof}

This means that the convolution of two distributions with compact support also has compact support. This means that if we have three distributions $\Lambda, \Psi$, and $\Phi$, two of which have compact support, then the distributions $\Lambda * (\Psi * \Phi)$ and $(\Lambda * \Psi) * \Phi$ are well defined, so convolution is associative and commutative. We calculate that for any test function $\phi$,
%
\[ (\Lambda * (\Psi * \Phi)) * \phi = \Lambda * (\Psi * (\Phi * \phi)) \]
\[ ((\Lambda * \Psi) * \Phi) * \phi = (\Lambda * \Psi) * (\Phi * \phi) \]
%
If $\Phi$ has compact support, then $\Phi * \phi$ has compact support, and so we can move $(\Lambda * \Psi)$ into the equation to prove equality. If $\Phi$ does not have compact support, then $\Lambda$ and $\Psi$ have compact support, and
%
\[ \Lambda * (\Psi * \Phi) = \Lambda * (\Phi * \Psi) \]
%
and we can apply the previous case to obtain that this is equal to $(\Lambda * \Phi) * \Psi$. Repeatedly applying the previous case brings this to what we want.

\begin{theorem}
    If $\Lambda$ and $\Psi$ are distributions, one of which having compact support, then
    %
    \[ D^\alpha(\Lambda * \Psi) = (D^\alpha \Lambda) * \Psi = \Lambda * (D^\alpha \Psi). \]
\end{theorem}
\begin{proof}
    The Dirac delta function $\delta$ satisfies
    %
    \[ (\delta * \phi)(x) = \int \phi(y) \delta(x-y)\; dy = \phi(x) \]
    %
    so $\delta * \phi = \phi$. Now $D^\alpha \delta$ is also supported at $x$, since
    %
    \[ (D^\alpha \delta)(\phi) = (-1)^{|\alpha|} \int \delta(x) (D^\alpha \phi)(x)\; dx = (-1)^{|\alpha|} (D^\alpha \phi)(0) \]
    %
    which means that for any distribution $\Lambda$, then $(D^\alpha \delta) * \Lambda$ has compact support,
    %
    \[ (((D^\alpha \delta) * \Lambda) * \phi)(0) = (D^\alpha \delta)((\Lambda * \phi)^*) = (-1)^{|\alpha|} D^\alpha (\Lambda * \phi)^* = ((D^\alpha \Lambda) * \phi)(0) \]
    %
    which verifies that $(D^\alpha \delta) * \Lambda = \delta * (D^\alpha \Lambda)$. But now we find
    %
    \[ D^\alpha(\Lambda * \Psi) = (D^\alpha \delta) * \Lambda * \Psi = ((D^\alpha \delta) * \Lambda) * \Psi = D^\alpha \Lambda * \Psi \]
    \[ D^\alpha(\Lambda * \Psi) = D^\alpha(\Psi * \Lambda) = (D^\alpha \Psi) * \Lambda = \Lambda * (D^\alpha \Psi) \]
    %
    which verifies the theorem in general.
\end{proof}




\chapter{Singular Integral Operators}




\chapter{Fourier Multiplier Operators}

Our aim in this chapter is to study the boundedness of \emph{Fourier multiplier operators}. Given a function $m: \RR^d \to \CC$, known as a \emph{symbol}, we want to associate a multiplier operator $m(D)$ which when applied to a function $f: \RR^d \to CC$ should be formally given by the equation
%
\[ (m(D)f)(x) = \int_{\RR^d} m(\xi) \widehat{f}(\xi) e^{2 \pi i \xi \cdot x}\; d\xi. \]
%
In maximum generality, if $m$ is a tempered distribution on $\RR^d$ we can consider the continuous operator $m(D): \mathcal{S}(\RR^d) \to \mathcal{S}(\RR^d)$. But often times $m$ will be much more regular, which we would hope can be exploited to give stronger continuity statements. Taking the Fourier transform shows that if $\widehat{K} = m$, then $m(D) f = K * f$. Thus Fourier multiplier operators are the same as convolution operators by tempered distributions.I n any case, the map $m \mapsto m(D)$ gives an injective \emph{algebra homomorphism} from the family of all tempered distributions to the family of continuous operators on $\mathcal{S}(\RR^d)$. The main goal, of course, is to determine what properties of the symbol or it's Fourier transform imply boundedness properties of the operator $T$.

\begin{remark}
  In engineering these operators are known as \emph{filters}, and occur in a variety of contexts. Due to the presence of error the regularity of these operators are of utmost importance. The function $m$ is known as the \emph{system-transfer function}, \emph{optical-transfer function}, or \emph{frequency response}, depending on the context, and the function $K$ is known as the \emph{point-spread function}.
\end{remark}

\begin{example}
  Over $\RR$, we consider the Fourier multiplier
  %
  \[ m(\xi) = - i \text{sgn}(\xi). \]
  %
  Then $m(D)$ is the Hilbert transform.
\end{example}

\begin{example}
  In $\RR^d$, we consider the Fourier multiplier
  %
  \[ m_R(\xi) = \mathbf{I}(|\xi| \leq R). \]
  %
  The operator $m_R(D)$ is known as the \emph{ball multiplier operator}. More generally, given any compact set $S$ we can consider the Fourier multiplier $\mathbf{I}_S(D)$. In the engineering literature these multipliers are called \emph{ideal low pass filters}.
\end{example}

\begin{example}
  In this chapter, it is natural to renormalize the differentiation operators $D^\alpha: \mathcal{S}(\RR^d) \to \mathcal{S}(\RR^d)$ so that for $f \in \mathcal{S}(\RR^d)$,
  %
  \[ \widehat{D^\alpha f} = \xi^\alpha \widehat{f}. \]
  %
  In particular, this implies that if $m(\xi) = \xi_i^\alpha$, then $m(D) = D^\alpha$. More generally, if $m(\xi) = \sum_{|\alpha| \leq k} c_\alpha \xi^\alpha$, then
  %
  \[ m(D) = \sum_{|\alpha| \leq k} c_\alpha D^\alpha. \]
  %
  Thus the family of Fourier multiplier operators contains all constant coefficient differential operators.
\end{example}

Fourier multiplier operators have been essential to us in the classical theory. In particular, we have used Fourier multiplier operators to prove a great many results; the convolution operator by the Poisson kernel is a Fourier multiplier given by the symbol $e^{-|x|}$, and the heat kernel is a Fourier multiplier with symbol $e^{- \pi |x|^2}$. This is no coincidence. It is a general heuristic that any well-behaved translation invariant operator is given by convolution with an appropriate function.

For instance, we have already seen in the chapter on distributions that any translation invariant continuous linear operator $T: C_c^\infty(\RR^d) \to C^\infty(\RR^d)$ is given by convolution with a distribution. If the distribution is tempered, we can take the Fourier transform to conclude that the operator is a Fourier multiplier operator. In fact, if $1 \leq p,q \leq \infty$ and $T$ satisfies a bound of the form
%
\[ \| Tf \|_{L^q(\RR^d)} \lesssim \| f \|_{L^p(\RR^d)} \]
%
for any $f \in \mathcal{S}(\RR^d)$, then $T$ is a Fourier multiplier operator. To prove this, we rely on a Sobolev-type regularity result.

\begin{lemma}
  Suppose $1 \leq p,q \leq \infty$. If $f \in L^p(\RR^d)$ has a strong derivative $D^\alpha f$ in $L^p(\RR^d)$ for all $|\alpha| \leq d+1$, then $f \in C(\RR^d)$, and
    %
    \[ |f(0)| \lesssim_{d,p} \sum_{|\alpha| \leq d + 1} \| D^\alpha f \|_{L^p(\RR^d)}. \]
\end{lemma}
\begin{proof}
    Let us first suppose $p = 1$. Then
    %
    \begin{align*}
      |\widehat{f}(x)| &\lesssim \frac{\sum_{|\alpha| \leq d+1} |x^\alpha \widehat{f}(x)|}{(1 + |x|)^{d+1}}\\
      &\lesssim \frac{\sum_{|\alpha| \leq d+1} \| D^\alpha f \|_{L^1(\RR^d)}}{(1 + |x|)^{d+1}}
    \end{align*}
    %
    Since $1/(1 + |x|)^{d+1} \in L^1(\RR^d)$, we conclude that $\widehat{f} \in L^1(\RR^d)$, and
    %
    \[ \| \widehat{f} \|_{L^1(\RR^d)} \lesssim \sum_{|\alpha| \leq d+1} \| D^\alpha f \|_{L^1(\RR^d)}. \]
    %
    It follows by the Fourier inversion formula that $f \in C(\RR^d)$, and moreover,
    %
    \[ \| f \|_{L^\infty(\RR^d)} \leq \sum_{|\alpha| \leq d+1} \| D^\alpha f \|_{L^1(\RR^d)}, \]
    %
    which completes the proof for $p = 1$.

    For $p > 1$, any compactly supported bump function $\phi$, and any multi-index $\alpha$ with $|\alpha| \leq d+1$,
    %
    \[ \| D^\alpha(\phi f) \|_{L^1(\RR^d)} \leq \sum_{\beta \leq \alpha} \| D^\beta \phi \cdot D^{\alpha - \beta} f \|_{L^1(\RR^d)} \lesssim_\phi \sum_{\beta \leq d+1} \| D^\beta f \|_{L^p(\RR^d)}. \]
    %
    It follows from the previous case that $\phi f \in C(\RR)$, and
    %
    \[ \phi(0) f(0) \lesssim_\phi \sum_{\beta \leq d+1} \| D^\beta f \|_{L^p(\RR^d)}. \]
    %
    Since $\phi$ was arbitrary, we conclude $f \in C(\RR)$, and that
    %
    \[ f(0) \lesssim \sum_{\beta \leq d+1} \| D^\beta f \|_{L^p(\RR^d)}. \qedhere \]
\end{proof}

\begin{theorem}
  Suppose $1 \leq p,q \leq \infty$, and $T: \mathcal{S}(\RR^d) \to L^q(\RR^d)$ is a linear map commuting with translations and satisfies
  %
  \[ \| Tf \|_{L^q(\RR^d)} \leq \| f \|_{L^p(\RR^d)} \]
  %
  for all $f \in \mathcal{S}(\RR^d)$. Then $T$ is a Fourier multiplier operator.
\end{theorem}
\begin{proof}
  For any $f \in \mathcal{S}(\RR^d)$, $Tf \in W^{q,n}(\RR^d)$ for any $n > 0$. To see this, we note that for any $h > 0$ and $k \in \{ 1, \dots, d \}$, and
  %
  \[ (\Delta_h f)(x) = \frac{f(x + he_k) - f(x)}{h}. \]
  %
  Then $\Delta_h(T f) = T(\Delta_h f)$ because $T$ is translation invariant. Since $f$ is a Schwartz function, $\Delta_h f$ converges to $D^k f$ in $L^p(\RR^d)$. Thus by continuity of $f$, $Tf$ has a strong derivative $T(D^k f)$ in $L^q(\RR^d)$. Induction shows $Tf$ has strong derivatives of all orders. The last lemma shows that $Tf \in C(\RR^d)$, and
  %
  \begin{align*}
    |Tf(0)| &\lesssim \sum_{|\alpha| \leq n+1} \| D^\alpha(Tf) \|_{L^q(\RR^d)}\\
    &= \sum_{|\alpha| \leq n+1} \| T(D^\alpha f) \|_{L^q(\RR^d)}\\
    &\lesssim \sum_{|\alpha| \leq n+1} \| D^\alpha f \|_{L^q(\RR^d)}.
  \end{align*}
  %
  The map $f \mapsto Tf(0)$ is thus continuous on $\mathcal{S}(\RR^d)$, and therefore defines a tempered distribution $\Lambda$. Translation invariance shows that $Tf = \Lambda * f$, and setting $m = \widehat{\Lambda}$ completes the proof.
\end{proof}

\begin{remark}
    It therefore follows that if $T: \mathcal{S}(\RR^d) \to L^q(\RR^d)$ is a linear operator commuting a translation satisfying a bound
    %
    \[ \| Tf \|_{L^q(\RR^d)} \lesssim \| f \|_{L^p(\RR^d)}, \]
    %
    then $Tf \in C^\infty(\RR^d)$ and is slowly increasing, as is all of it's derivatives.
\end{remark}

We now wish to know what conditions on $m$ guarantee bounds of the form
%
\[ \| m(D) f \|_{L^q(\RR^d)} \lesssim \| f \|_{L^p(\RR^d)}. \]
%
for all $f \in \mathcal{S}(\RR^d)$. Littlewood's principle tells us that the only interesting case occur with `the larger exponent on the left'.

\begin{theorem}
  Fix $1 \leq q < p \leq \infty$, and suppose $m \in \mathcal{S}(\RR^d)'$ with
  %
  \[ \| m(D) f \|_{L^q(\RR^d)} \lesssim \| f \|_{L^p(\RR^d)} \]
  %
  for all $f \in \mathcal{S}(\RR^d)$. Then $m = 0$.
\end{theorem}
\begin{proof}
  Suppose $m \neq 0$. Then there is $f_0 \in \mathcal{S}(\RR^d)$ with $m(D) f \neq 0$. Thus $m(D) f_0$ lies in $C^\infty(\RR^d) \cap L^q(\RR^d)$. Fix a large integer $N$ and pick $x_1,\dots,x_N \in \RR^d$ separated far enough apart that
  %
  \[ \| \sum_{n = 1}^N \text{Trans}_{x_n} f_0 \|_{L^p(\RR^d)} \gtrsim N^{1/p} \| f_0 \|_{L^p(\RR^d)} \]
  %
  and
  %
  \[ \| \sum_{n = 1}^N \text{Trans}_{x_n} m(D) f_0 \|_{L^q(\RR^d)} \sim N^{1/q} \| m(D) f_0 \|_{L^q(\RR^d)} \lesssim N^{1/q} \| f_0 \|_{L^p(\RR^d)}. \]
  %
  Translation invariance of convolution shows $N^{1/q} \lesssim N^{1/p}$, which is impossible for suitably large $N$. Thus $m = 0$.
\end{proof}

In general, a characterization of the tempered distributions which give bounded convolution operators is unknown except in a few very particular situations. For each $1 \leq p \leq q \leq \infty$, we let $\| m \|_{M^{p,q}(\RR^d)}$ denote the operator norm of the multiplier operator $m(D)$ from $L^p(\RR^d)$ to $L^q(\RR^d)$, i.e. the smallest quantity such that
%
\[ \| m(D) f \|_{L^q(\RR^d)} \leq \| m \|_{M^{p,q}(\RR^d)} \| f \|_{L^p(\RR^d)} \]
%
for all $f \in \mathcal{S}(\RR^d)$. We let $M^{p,q}(\RR^d)$ be the set of tempered distributions for which the bound is finite. For simplicity, we also let $M^p(\RR^d)$ denote $M^{p,p}(\RR^d)$. By symmetries of the Fourier transform, it is easy to check that translations, modulations, and dilations all preserve the $M^{p,q}$. Thus we have a complete set of affine symmetries, as well as a modulation symmetry.

\begin{example}
  A Fourier multiplier operator $T$ corresponding to a tempered distribution $m$ has a bound
  %
  \[ \| m(D)f \|_{L^2(\RR^d)} \lesssim \| f \|_{L^2(\RR^d)} \]
  %
  if and only if $m \in L^\infty(\RR^d)$, and then $\| m \|_{M^{2,2}(\RR^d)} = \| m \|_{L^\infty(\RR^d)}$. To see this, let
  %
  \[ \Phi(x) = e^{- \pi |x|^2} \]
  %
  be the Gaussian distribution. Then
  %
  \[ \widehat{m(D) \Phi} = \Phi \cdot m. \]
  %
  Since $\Phi \in L^2(\RR^d)$, $m(D) \Phi \in L^2(\RR^d)$, and so $\Phi \cdot m \in L^2(\RR^d)$. But then we conclude that
  %
  \[ m = \frac{\widehat{m(D) \Phi}}{\Phi}. \]
  %
  Thus $m \in L^1_{\text{loc}}(\RR^d)$. But then the result is obvious.
\end{example}

For any tempered distribution $m$ and $f,g \in \mathcal{S}(\RR^d)$,
%
\begin{align*}
  \langle m(D) f, g \rangle &= \langle m \widehat{f}, \widehat{g} \rangle
  &= \langle \widehat{f}, m^* \widehat{g} \rangle
  &= \langle f, m^*(D) g \rangle.
\end{align*}
%
Thus we have an adjoint relation $m(D)^* = m^*(D)$, which gives a natural duality theory for Fourier multiplier operators.

\begin{theorem}
  For any $1 \leq p,q \leq \infty$ and any tempered distribution $m$,
  %
  \[ \| m \|_{M^{p,q}(\RR^d)} = \| m \|_{M^{q^*,p^*}(\RR^d)}. \]
\end{theorem}
\begin{proof}
  Using the adjoint relation, if
  %
  \[ \| m(D) f \|_{L^q(\RR^d)} \lesssim \| f \|_{L^p(\RR^d)} \]
  %
  then
  %
  \[ \| m^*(D) f \|_{L^{p^*}(\RR^d)} \lesssim \| f \|_{L^{q^*}(\RR^d)} \]
  %
  But it is easy to calculate that if we set $[\text{Ref} u](x) = u(-x)$, then for any $x \in \RR^d$,
  %
  \[ [m^*(D) f](x) = [m(D)(\text{Ref}f^*)(-x)]^* \]
  %
  and so $\| m^*(D) f \|_{L^{p^*}(\RR^d)} = \| m(D) f \|_{L^{p^*}(\RR^d)}$.
\end{proof}

In particular, if $1 \leq p \leq \infty$ and $m \in M^p(\RR^d)$, then also $m \in M^{p^*,p^*}(\RR^d)$ and so Riesz-interpolation implies $m \in M^{2,2}(\RR^d)$. Thus if we are studying $L^p$ to $L^p$ boundedness for any $1 \leq p \leq \infty$, we may restrict our attention to bounded Fourier multipliers.

\begin{example}
  The only remaining space which is completely understood is the space $M^{1,1}(\RR^d) = M^{\infty,\infty}(\RR^d)$; in this case, a tempered distribution is included if and only if the distribution is the Fourier transform of a finite Borel measure, and moreover, if $\mu \in M(\RR^d)$ is a finite Borel measure, then $\| \widehat{\mu} \|_{M^{1,1}(\RR^d)} = \| \mu \|_{TV(\RR^d)}$. If $\{ \Phi_\delta : \delta > 0 \}$ is the Gauss kernel, set
  %
  \[ m_\delta(x) = e^{- \delta |x|^2} m(\xi) \]
  %
  Then by assumption of $L^1$ boundedness, and the fact that the Fourier transform of $e^{-\delta |x|^2}$ is a constant multiple of $\Phi_\delta$, we conclude that for all $\delta > 0$,
  %
  \[ \| \widecheck{m_\delta} \|_{L^1(\RR^d)} \lesssim 1. \]
  %
  Thus $\{ \widecheck{m_\delta} \}$ are uniformly bounded in $L^1(\RR^d)$, so by Banach Alaoglu theorem, combined with the fact that $L^1(\RR^d)$ embeds in $M(\RR^d)$, which is the dual of $C_0(\RR^d)$, we conclude there is a subsequence $\{ \delta_k \}$ converging to zero such that $\widecheck{m_{\delta_k}}$ converges weakly to some finite Borel measure $\mu$. But this implies that $m_{\delta_k}$ converges weakly to $\widehat{\mu}$, which implies $m = \widehat{\mu}$.
\end{example}

For $1 < p < 2$ and $2 < p < \infty$, characterizing $M^p(\RR^d)$ is a much more subtle task, if not impossible. For instance, it remains an open question to determine for which values of $p$ and $\delta$ for which the multiplier
%
\[ m^\delta(\xi) = \max((1 - |\xi|^2)^\delta,0) \]
%
lies in $M^p(\RR^d)$, a problem known as the \emph{Bochner-Riesz conjecture}.

The difficulty here is that $m^\delta$ is singular on the boundary of the unit sphere, which is a large, curved set. However, mathematicians have developed criteria which implies boundedness of various operators. The most fundamental occurs if the multiplier $m$ has no singularities. For instance, if $m \in \mathcal{S}(\RR^d)$, then $\widecheck{m} \in L^1(\RR^d)$, so $m \in M^{1,1}(\RR^d)$, and thus in $M^p(\RR^d)$ for all $1 \leq p \leq \infty$. Similarily, if $m$ is a bump function adapted to $L(\Omega)$ for a fixed boundary domain $\Omega$, then $\| m \|_{M^p(\RR^d)} \lesssim_{d,\Omega} 1$ for all $1 \leq p \leq \infty$.

If the multiplier $m$ is only singular on a smaller set, we can also do better. For instance, if the Hilbert transform satisfies the bounds
%
\[ \| Hf \|_{L^p(\RR^d)} \lesssim_p \| f \|_{L^p(\RR^d)} \]
%
for all $f \in \mathcal{S}(\RR^d)$ and $1 < p < \infty$. It therefore follows that for $1 < p < \infty$ and any (possibly unbounded interval) $I$,
%
\[ \| \mathbf{I}_{I} \|_{M^p(\RR^d)}, \| \mathbf{I}_{I} \|_{M^p(\RR^d)} \lesssim_p 1. \]
%
Now suppose $m \in L^\infty(\RR^d)$ has \emph{bounded variation}, which means the quantity
%
\[ V(m) = \sup_{\xi_1 < \dots < \xi_N} \sum_{i = 1}^{N-1} |m(\xi_{i+1}) - m(\xi_i)|. \]
%
is finite. Then $m$ has countably many discontinuities, and the variation prevents too much nonsmoothness.

\begin{theorem}
  Suppose $m \in L^\infty(\RR)$ has finite variation. Then for each $1 < p < \infty$,
  %
  \[ \| m \|_{M^p(\RR)} \lesssim_p \| m \|_{L^\infty(\RR)} + V(m) \]
\end{theorem}
\begin{proof}
  For each $n$, pick $\xi_1,\dots,\xi_{N_n}$ such that
  %
  \[ \sum_{i = 1}^{N-1} |m(\xi_{i+1}) - m(\xi_i)| \geq V(m) - 1/n. \]
  %
  If we define
  %
  \[ m_n = m(\xi_1) \mathbf{I}_{(-\infty,\xi_1)} + \sum_{i = 1}^{N-1} m(\xi_i) \mathbf{I}_{(\xi_i,\xi_{i+1})} + m(\xi_N) \mathbf{I}_{(\xi_N,\infty)} \]
  %
  then $m - m_n$ is a finite signed Borel measure with $\| m - m_n \|_{M(\RR)} \leq 1/n$. Thus
  %
  \[ \| m \|_{M^p(\RR)} \leq \limsup_{n \to \infty} \| m_n \|_{M^p(\RR)}. \]
  %
  Now we can rewrite
  %
  \[ m_n(\xi) = m(\xi_1) \mathbf{I}_{(-\infty,\xi_1)} + \sum_{i = 1}^{N-1} [m(\xi_i) - m(\xi_{i+1})] \mathbf{I}_{(\xi_1,\xi_i)}(\xi) + m(\xi_N) \mathbf{I}_{(\xi_N,\infty)}. \]
  %
  Thus we find that for $1 < p < \infty$,
  %
  \[ \| m_n \|_{M^p(\RR)} \lesssim_p |m(\xi_1)| + \sum_{i = 1}^{N-1} |m(\xi_i) - m(\xi_{i+1})| + |m(\xi_N)| \leq \| m \|_{L^\infty(\RR)} + V(m). \]
  %
  But this means that $\| m \|_{M^p(\RR)} \lesssim_p \| m \|_{L^\infty(\RR)} + V(m)$.
\end{proof}

The theory of Fourier multipliers gets more complicated as we increase the dimension of the ambient space we are working in. De Leeuw's theorem shows slices of continuous $d+1$ dimensional multipliers are bounded by the original mutiplier.

\begin{theorem}
  Let $m \in C(\RR^{d+1})$. For each $\xi_0 \in \RR$ define $m_0 \in C(\RR^d)$ by setting
  %
  \[ m_0(\xi) = m(\xi,\xi_0). \]
  %
  Then for any $1 \leq p \leq \infty$, $\| m_0 \|_{M^p(\RR^d)} \leq \| m \|_{M^p(\RR^{d+1})}$.
\end{theorem}
\begin{proof}
  Without loss of generality, assume $\xi_0 = 0$. For $\lambda > 0$ set
  %
  \[ L(\xi_1,\dots,\xi_d) = (\xi_1,\dots,\xi_{d-1},\xi_d/\lambda). \]
  %
  Then
  %
  \[ \| m \circ L_\lambda \|_{M^p(\RR^d)} = \| m \|_{M^p(\RR^d)}. \]
  %
  Take $\lambda \to \infty$. Since $m$ is continuous, $m \circ L_\lambda$ converges to $m \circ L_\infty$ pointwise as $\lambda \to \infty$, where $L_\infty(\xi_1,\dots,\xi_d) = (\xi_1,\dots,\xi_{d-1},0)$. On the other hand,
  %
  \[ \| m \circ L_\infty \|_{M^p(\RR^d)} = \| m_0 \|_{M^p(\RR^d)}. \]
  %
  Thus it suffices to show that
  %
  \[ \| m \circ L_\infty \|_{M^p(\RR^d)} \leq \limsup_{\lambda \to \infty} \| m \circ L_\lambda \|_{M^p(\RR^d)}. \]
  %
  But to do this it suffices to use a weak convergence argument; for any $f,g \in \mathcal{S}(\RR^{d+1})$, we just note that dominated convergence shows that
  %
  \[ \lim_{\lambda \to \infty} |\langle (m \circ L_\lambda)(D) f, g \rangle| = \langle (m \circ L_\infty)(D) f, g \rangle. \qedhere \]
\end{proof}

The theorem of H\"{o}rmander-Mikhlin gives another instance of this phenomenon, giving $L^p$ bounds to Fourier multipliers which decay smoothly and rapidly away from the origin.

\begin{theorem}
  Let $m \in L^\infty(\RR^d)$ and suppose there exists an integer $n > d/2$ such that for any $\beta \in C_c^\infty(\RR^d - \{ 0 \})$ and any multi-index $\alpha$ with $|\alpha| \leq n$,
  %
  \[ \| D^\alpha( (\text{Dil}_{1/\lambda} \beta) \cdot m ) \|_{L^2(\RR^d)} \lesssim_\beta \lambda^{d/2-|\alpha|} \]
  %
  Then for any $1 < p < \infty$ and $f \in \mathcal{S}(\RR^d)$,
  %
  \[ \| m(D) f \|_{L^p(\RR^d)} \lesssim_p \| f \|_{L^p(\RR^d)}. \]
\end{theorem}

\begin{remark}
  The assumptions of the theorem hold for $m \in C^\infty(\RR^d - \{ 0 \})$ any multi-index $\alpha$ and any $\xi \neq 0$, $|D^\alpha m(\xi)| \lesssim_\alpha |\xi|^{-\alpha}$. It then follows that
  %
  \[ D^\alpha((\text{Dil}_{1/\lambda} \beta) \cdot m) = \sum_{\gamma \leq \alpha} \lambda^{-|\gamma|} \cdot (\text{Dil}_{1/\lambda} (D^\gamma \beta)) \cdot D^{\alpha - \gamma} m \]
  %
  Now rescaling shows
  %
  \begin{align*}
    \| \lambda^{-|\gamma|} (\text{Dil}_{1/\lambda} (D^\gamma \beta)) \cdot (D^{\alpha - \gamma} m) \|_{L^2(\RR^d)} \lesssim_{\beta,n} \lambda^{d/2-|\alpha|},
  \end{align*}
  %
  and summing up implies $m(D)$ is a H\"{o}rmander-Mikhlin operator. In particular, this is true if $m \in C^\infty(\RR^d - \{ 0 \})$ is homogenous of degree zero.
\end{remark}

\begin{proof}
  s
\end{proof}











\chapter{Sobolev Spaces}

Let $\Omega$ be an open subset of $\RR^d$. A natural problem when studying smooth functions $\phi \in C_c^\infty(\Omega)$ is to obtain estimates on the partial derivatives of $\phi$. For instance, one can consider the norms
%
\[ \| \phi \|_{C^n(\Omega)} = \max_{|\alpha| \leq n} \| D^\alpha f \|_{L^\infty(\Omega)}. \]
%
The space $C_c^\infty(\Omega)$ is not complete with respect to this norm, but it's completion is the space $C^n_b(\Omega)$ of $n$ times bounded continuously differentiable functions on $\Omega$, which still consists of regular functions. Unfortunately, such estimates are only encountered in the most trivial situations. As in the non-smooth case, one can often get much better estimates using the $L^p$ norms of the derivatives, i.e. considering the norms
%
\[ \| \phi \|_{W^{n,p}(\Omega)} = \left( \sum_{|\alpha| \leq p} \| D^\alpha \phi \|_{L^p(\Omega)}^p \right)^{1/p}. \]
%
As might be expected, $C_c^\infty(\Omega)$ is not complete with respect to the $W^{n,p}(\Omega)$ norm. However, it's completion cannot be identified with a family of $n$ times differentiable functions. Instead, to obtain a satisfactory picture of the compoetion under this norm, a Banach space we will denote by $W^{n,p}(\Omega)$, we must take a distribution approach.

For each multi-index $\alpha$, if $f$ and $f_\alpha$ are locally integrable functions on $\Omega$, we say $f_\alpha$ is a weak derivative for $f$ if for any $\phi \in C_c^\infty(\Omega)$,
%
\[ \int_\Omega f_\alpha(x) \phi(x)\; dx = (-1)^{|\alpha|} \int_\Omega f(x) \phi_\alpha(x)\; dx. \]
%
In other words, this is the same as the derivative of $f$ viewed as a distribution on $\Omega$. We define $W^{n,p}$ to be the space of all functions $f \in L^p(\Omega)$ such that for each $|\alpha| \leq n$, a weak derivative $f_\alpha$ exists and is an element of $L^p(\Omega)$. We then define
%
\[ \| f \|_{W^{n,p}(\Omega)} = \left( \sum_{|\alpha| \leq n} \| f_\alpha \|_{L^p(\Omega)} \right)^{1/p}. \]
%
Where this sum is treated as a maximum in the case $p = \infty$. Later on we will be able to show this space is a complete Banach space.

\begin{example}
  Let $B$ be the open unit ball in $\RR^d$, and let $u(x) = |x|^{-s}$, where $s < n-1$. For which $p$ is $u \in W^{1,p}(B)$? We calculate by an integration by parts that if $\phi \in C_c^\infty(B)$, we fix $\varepsilon > 0$ and write
  %
  \[ \int_B \phi_i(x) u(x)\; dx = \int_{|x| \leq \varepsilon} \phi_i(x) u(x) + \int_{\varepsilon < |x| \leq 1} \phi_i(x) u(x). \]
  %
  The integral on the $\varepsilon$ ball is neglible since $s < n$. Since $u$ is smooth away from the origin, it's distributional derivative agrees with it's standard derivative, which is
  %
  \[ u_i(x) = \frac{- \alpha x_i}{|x|^{s + 2}}. \]
  %
  Thus $|u_i| \lesssim 1/|x|^{s + 1}$. An integration by parts gives
  % in the $i$'th direction, and we calculate $\nabla u(x) = -\alpha x |x|^{-\alpha-2}$. Thus an integration by parts gives
  %
  \[ \int_{\varepsilon < |x| \leq 1} \phi_i(x) u(x) = \int_{|x| = \varepsilon} \phi(x) u(x) \nu_i\ dS + \int_{\varepsilon < |x| \leq 1} \frac{s \phi(x) x_i}{|x|^{s + 2}}\; dx, \]
  %
  where $\nu_i$ is the normal vector to the sphere pointing inward. Since $s < n-1$, the surface integral tends to zero as $\varepsilon \to 0$. Thus the weak derivative of $u$ is equal to the standard derivative. Consequently, $u \in W^{1,p}(B)$ if $s < n/p - 1$.
\end{example}

\begin{example}
  If $\{ r_k \}$ is a countable, dense subset of $B$, then we can define
  %
  \[ u(x) = \sum_{k = 1}^\infty \frac{|x - r_k|^{-s}}{2^k} \]
  %
  Then $u \in W^{1,p}(B)$ if $0 < \alpha < n/p - 1$, yet $u$ has a dense family of singularities, and thus does not behave like any differentiable function we would think of.
\end{example}

\begin{theorem}
  For each $k \in \mathbf{N}$ and $1 \leq p \leq \infty$, $W^{k,p}(\Omega)$ is a Banach space.
\end{theorem}
\begin{proof}
  It is easy to verify that $\| \cdot \|_{W^{k,p}}$ is a norm on $W^{k,p}(\Omega)$. Let $\{ u_n \}$ be a Cauchy sequence in $W^{k,p}(\Omega)$. In particular, this means that $\{ D^\alpha u_n \}$ is a Cauchy sequence in $L^p(\Omega)$ for each multi-index $\alpha$ with $|\alpha| \leq k$. In particular, these are functions $v_\alpha$ such that $D^\alpha u_n$ converges to $v_\alpha$ in the $L^p$ norm for each $\alpha$. Thus it suffices to prove that if $v = \lim u_n$, then $D^\alpha v = v_\alpha$ for each $\alpha$. But this follows because the H\"{o}lder inequality implies that for each fixed $\phi \in C_c^\infty(\Omega)$,
  %
  \begin{align*}
    (-1)^{|\alpha|} \int \phi_\alpha(x) v(x)\; dx &= \lim_{n \to \infty} (-1)^{|\alpha|} \phi_\alpha u_n(x)\; dx\\
    &= \lim_{n \to \infty} \int \phi(x) (D^\alpha u_n)(x)\; dx\\
    &= \int \phi(x) v_\alpha(x)\; dx.
  \end{align*}
  %
  Thus $W^{k,p}(\Omega)$ is complete.
\end{proof}

\section{Smoothing}

It is often useful to be able to approximate elements of $W^{k,p}(\Omega)$ by elements of $C^\infty(\Omega)$. This is mostly possible. If $u \in W^{k,p}(\Omega)$, and $\{ \eta_\varepsilon \}$ is a family of smooth mollifiers, then, viewing $u$ as a function on $\RR^n$ supported on $\Omega$, we can consider the convolution $u^\varepsilon = u * \eta_\varepsilon$, i.e. the function defined by setting
%
\[ u^\varepsilon(x) = \int_\Omega u(x - y) \eta_\varepsilon(y)\; dy. \]
%
This is just normal convolution, where we identify the function $u$ with the function $u \mathbf{I}_\Omega$ on $\RR^d$. Then $u^\varepsilon$ is a smooth function on $\RR^d$ supported on a $\varepsilon$ thickening of $\Omega$. However, $u^\varepsilon$ does not necessarily converge to $u$ in $W^{k,p}(\Omega)$ as $\varepsilon \to 0$, since the behaviour of the convolution can cause issues at the boundary of $\Omega$, where the distributional derivative $D^\alpha(u \mathbf{I}_\Omega)$ does not behave like a locally integrable function. This is the only problem, however.

\begin{theorem}
  If $U \Subset \Omega$, then $\lim_{\varepsilon \to 0} \| u^\varepsilon - u \|_{L^p(U)} = 0$.
\end{theorem}
\begin{proof}
  For each $\varepsilon > 0$, let $U^\varepsilon = \{ x \in \Omega: d(x,\partial \Omega) > \varepsilon \}$. If $x \in \Omega^\varepsilon$, then
  %
  \[ ((D^\alpha u) * \eta_\varepsilon)(x) = (u_\alpha \mathbf{I}_\Omega * \eta_\varepsilon)(x), \]
  %
  since the convolution only depends on the behaviour of $D^\alpha u$ on a $\varepsilon$ ball around $x$, which is contained in the interior of $\Omega$. We can apply standard results about mollifiers to conclude that $u_\alpha \mathbf{I}_\Omega * \eta_\varepsilon$ converges to $u_\alpha \mathbf{I}_\Omega$ in $L^p(\RR^d)$ as $\varepsilon \to 0$. Since $U \Subset \Omega$, we have $U \subset U^\varepsilon$ for small enough $\varepsilon$, and so $(D^\alpha u) * \eta_\varepsilon$ converges to $u_\alpha$ in $L^p(U)$ as $\varepsilon \to 0$. Since this is true for each $\alpha$ with $|\alpha| \leq k$, we obtain the result.
\end{proof}

If we are a little more careful, then we can fully approximate elements of $W^{k,p}(\Omega)$ by smooth functions on $U$.

\begin{theorem}
  $C^\infty_c(\Omega) \cap W^{k,p}(\Omega)$ is dense in $W^{k,p}(\Omega)$.
\end{theorem}
\begin{proof}
  Consider a family of open sets $\{ V_n \}$ such that $V_n \Subset \Omega$ for each $n$, and $U = \bigcup V_n$. Then we can consider a smooth partition of unity $\{ \xi_n \}$ subordinate to the cover $\{ V_n \}$. For each $u \in W^{k,p}(\Omega)$, we can write $u = \sum_n u \xi_n$. In particular, this means that for each $\varepsilon > 0$, there is $N$ such that $\| \sum_{n = N+1}^\infty u \xi_n \|_{W^{k,p}(\Omega)} \leq \varepsilon$. For each $n \in \{ 1, \dots, N \}$, we can find $\delta_n$ small enough that the $\delta_n$ thickening of $V_n$ is compactly contained in $\Omega$. If $\varepsilon_n$ is small enough, we find $(u \xi_n)^{\varepsilon_n}$ is supported on the $\delta_n$ thickening of $V_n$, and $\| (u \xi_n)^{\varepsilon_n} - u \xi_n \|_{W^{k,p}(V_n)} \leq \varepsilon / N$. But we then find
  %
  \begin{align*}
    \| u - \sum_{n = 1}^N (u \xi_n)^{\varepsilon_n} \|_{W^{k,p}(\Omega)} \leq \varepsilon + \sum_{n = 1}^N \| u \xi_n - (u \xi_n)^{\varepsilon_n} \|_{W^{k,p}(\Omega)} \leq 2\varepsilon.
  \end{align*}
  %
  Thus $C_c^\infty(\Omega)$ is dense in $W^{k,p}(\Omega)$.
\end{proof}

Approximation by elements of $C^\infty(\overline{\Omega})$ requires some more care, and additional assumptions on the behaviour of $\partial \Omega$.











\chapter{Basics of Kernel Operators}

We now consider a general family of operators, which can be seen as the infinite dimensional analogue of matrix multiplication. We fix two measure spaces $X$ and $Y$, and consider a function $K: X \times Y \to \CC$, which we call a \emph{kernel}. From this kernel, we obtain an induced operator $T_K$ taking functions on $X$ to functions on $Y$, given, heuristically at least, by the integral formula
%
\[ (T_K f)(y) = \int_X K(x,y) f(x)\; dx. \]
%
Our goal is to relate control on the kernel $K$ to the boundedness of the operator $T_K$ with respect to various norms.

\begin{example}
  Let $X = Y = \RR^d$, equipped with the Lebesgue measure. If we set $K(x,\xi) = e^{2 \pi i \xi \cdot x}$, then using this function as a kernel we can obtain an integral operator
  %
  \[ (T_K f)(\xi) = \int f(x) e^{2 \pi i \xi \cdot x}\; dx. \]
  %
  In the standard theory of Fourier analysis, we find that if $f \in L^1(\RR)$, then for any $\xi$ the integral
  %
  \[ \int f(x) e^{2 \pi i \xi \cdot x} \]
  %
  converges absolutely, and is thus well-defined in the sense of a Lebesgue integral. Moreover, for any $f \in L^1(\RR)$,
  %
  \[ \| T_K f \|_{L^\infty(\RR)} \leq \| f \|_{L^1(\RR)}. \]
  %
  We also know from the classical Hausdorff-Young inequality that if $1 \leq p \leq 2$, then for any $f \in L^1(\RR) \cap L^p(\RR)$,
  %
  \[ \| T_K f \|_{L^{p^*}(\RR)} \leq \| f \|_{L^p(\RR)}. \]
  %
  In particular, this means that there exists a unique extension of $T_K$ to a bounded operator from $L^p(\RR)$ to $L^{p^*}(\RR)$; note, however, that for a general element $f \in L^p(\RR)$, the integral formula
  %
  \[ \int f(x) e^{2 \pi i \xi \cdot x}\; dx \]
  %
  is \emph{not well-defined} in the Lebesgue sense. Thus we can only heuristically view the integral formula as defining the integral operator.
\end{example}

\begin{example}
  Let $X = \{ 1, \dots, N \}$ and $Y = \{ 1, \dots, M \}$, each equipped with the counting measure. Then each kernel $K$ corresponds to an $M \times N$ matrix $A$, with $A_{ij} = K(j,i)$. For any $f: X \to Y$ we can define a vector $v \in \RR^N$ by setting $v_i = f(i)$, and then
  %
  \[ (T_K f)(m) = \sum_{n = 1}^N f(n) K(n,m) = \sum_{n = 1}^N A_{mn} v_n = (Av)_m. \]
  %
  Thus with respect to the standard basis, $T_K$ is just given by matrix multiplication by $A$.
\end{example}

It turns out that if we map \emph{from} $L^1(X)$, or \emph{into} $L^\infty(Y)$, then the conditions on $K$ determining boundedness are trivial to determine for \emph{most} norms. This is one motivation for introduction the intermediate $L^p$ norms, since these norms enable us to extract more features out of the kernel operator $K$.

Without even qualitative knowledge of the kernel $K$ besides it's measurability, it is difficult to know for which functions $f$ the operator $T_K f$ is well-defined, even if $f$ is simple. A natural trick here is to introduce the sublinear analogue of the kernel operator, i.e. the operator $S_K$ defined by setting
%
\[ (S_K f)(y) = \int_X |K(x,y)| |f(x)|; dx \]
%
The flexibility of the theory of non-negative integrals means this operator is well defined for \emph{any} measurable $f$ (though it may take on infinite values). Moreover, if we are to interpret $(T_K f)(y)$ in the Lebesgue sense, then it is necessary and sufficient that $(S_K f)(y) < \infty$.

\begin{theorem}
  Fix $q \geq 1$, and suppose $X$ and $Y$ are $\sigma$ finite. Then the smallest coefficient $C > 0$ such that for any $f \in L^1(X)$,
  %
  \[ \| S_K f \|_{L^q(Y)} \leq C \| f \|_{L^1(X)} \]
  %
  is equal to $\| K \|_{L^q(Y) L^\infty(X)}$. In particular, if $\| K \|_{L^q(Y) L^\infty(X)} < \infty$, then for each $f \in L^1(X)$, $(T_K f)(y)$ is well-defined in the Lebesgue sense for almost every $y \in Y$, and the operator norm of $T_K$ from $L^1(X)$ to $L^q(Y)$ is equal to $\| K \|_{L^q(Y) L^\infty(X)}$.
\end{theorem}
\begin{proof}
  We calculate by Minkowski's inequality that
  %
  \begin{align*}
    \| S_K f \|_{L^q(Y)} &= \| K f \|_{L^1(X) L^q(Y)}\\
    &\leq \| Kf \|_{L^q(Y) L^1(X)}\\
    &= \int \left( \int |K(x,y)|^q\; dy \right)^{1/q} |f(x)|\; dx\\
    &\leq \| K \|_{L^q(Y)L^\infty(X)} \| f \|_{L^1(X)}.
  \end{align*}
  %
  If $\| K \|_{L^q(Y) L^\infty(X)} < \infty$, then $(S_K f)(y) < \infty$ for almost every $y \in Y$, which implies $(T_K f)(y)$ is well-defined for almost every $y \in Y$. Since $(T_K f)(y) \leq (S_K f)(y)$ for such $y$, we conclude that
  %
  \[ \| T_K f \|_{L^q(Y)} \leq \| K \|_{L^q(Y) L^\infty(X)} \| f \|_{L^1(X)}. \]
  %
  Let us now show this constant is tight. By an approximation argument I leave to the end of the discussion, we may assume that we can write
  %
  \[ K = \sum_{i = 1}^N \sum_{j = 1}^M a_{ij} \mathbf{I}_{E_i \times F_j} \]
  %
  where $E_1,\dots,E_N$ and $F_1,\dots,F_N$ are disjoint finite measure sets. Then there exists $i \in \{ 1, \dots, N \}$ such that for each $x \in E_i$,
  %
  \[ \left( \int |K(x,y)|^q\; dy \right)^{1/q} = \left( \sum_{j = 1}^M |a_{ij}|^q |F_j| \right)^{1/q} = \| K \|_{L^q(Y) L^\infty(X)}. \]
  %
  If $f = \mathbf{I}_{E_i}$, then $\| f \|_{L^1(X)} = |E_i|$, and $T_K f = \sum_{j = 1}^M a_{ij} \mathbf{I}_{F_j}$, so
  %
  \[ \| T_K f \|_{L^q(Y)} = \left( \sum_{j = 1}^M |a_{ij}|^q |F_j| \right)^{1/q} = \| K \|_{L^q(Y)L^\infty(X)} \| f \|_{L^1(X)}. \]
  %
  Thus we conclude that for a certain `dense' family of $K$, $T_K$ is tight. Let us now complete the argument to prove the result in general.


  By a simple approximation argument in $\| \cdot \|_{L^q(Y) L^\infty(X)}$, using the fact that $X$ and $Y$ are $\sigma$ finite, we may assume that $K$ is supported on a product of finite measure subsets of $X$ and $Y$, so without loss of generality we can assume $X$ and $Y$ have finite measure.
\end{proof}

\begin{theorem}
  Fix $q \geq 1$. If $\| K \|_{L^q(Y) L^\infty(X)} < \infty$, then $S_K$ is bounded as an operator from $L^1(X)$ to $L^q(Y)$, with operator norm bounded above by $\| K \|_{L^q(Y) L^\infty(X)}$, with equality if $X$ and $Y$ are $\sigma$ finite. Correspondingly, for each $f \in L^1(X)$, we have
  %
  \[ \int K(x,y) f(x)\; dx < \infty\ \text{for almost every $y$}, \]
  %
  and $\| T_K f \|_{L^q(Y)} \leq \| K \|_{L^q(Y) L^\infty(X)} \| f \|_{L^1(X)}$.
\end{theorem}
\begin{proof}
  Applying Minkowski's inequality, we conclude that
  %
  \begin{align*}
    \| S_K f \|_{L^q(Y)} &= \left( \left( \int |f(x)| |K(x,y)|\; dx \right)^q \right)^{1/q}\\
    &\leq \int \left( \int |f(x)|^q |K(x,y)|^q\; dy \right)^{1/q}\; dx\\
    &\leq \int |f(x)| \| K \|_{L^q(Y)}(x)\; dx\\
    &\leq \| f \|_{L^1(X)} \| K \|_{L^q(Y) L^\infty(X)}.
  \end{align*}
  %
  To show tightness, consider the first case where $K$ can be written as
  %
  \[ \sum_{i = 1}^N \sum_{j = 1}^M a_{ij} \mathbf{I}_{E_i \times F_j}, \]
  %
  where $E_1, \dots, E_N$ are disjoint, finite measure sets in $X$, and $F_1, \dots, F_M$ are disjoint, finite measure sets in $Y$. Then there exists $i \in \{ 1, \dots, N \}$ such that for each $x \in E_i$,
  %
  \[ \left( \int |K(x,y)|^q\; dy \right)^{1/q} = \left( \sum_{j = 1}^M |a_{ij}|^q |F_j| \right)^{1/q} = \| K \|_{L^q(Y) L^\infty(X)}. \]
  %
  If $f = \mathbf{I}_{E_i}$, then $\| f \|_{L^1(X)} = |E_i|$, and
  %
  \begin{align*}
    \left( \left( \int |K(x,y) f(x)|\; dx \right)^q dy \right)^{1/q} &= \left( \sum_{j = 1}^M |F_j| |a_{ij}|^q |E_i|^q \right)^{1/q}\\
    &= \| f \|_{L^1(X)} \| K \|_{L^q(Y) L^\infty(X)}.
  \end{align*}
  %
  Thus $f$ is an extremizer for $S_K$.

  To show this inequality is tight. Let us first consider the case where $q < \infty$. By a monotone convergence result if $X$ and $Y$ are $\sigma$ finite, we may assume that $X$ and $Y$ have finite measure. It then follows that for each $\varepsilon > 0$, there are functions $u_1, \dots, u_n \in L^1(X)$ and $v_1, \dots, v_n \in L^1(Y)$ such that $\| K - u_1 \otimes v_1 - \dots - u_n \otimes v_n \|_{L^1(X \times Y)} < \varepsilon$.
\end{proof}

\begin{lemma}
  BLAH
\end{lemma}
\begin{proof}
  Let $\Pi$ be the family of all sets $E \times F \subset X \times Y$, where $E$ is a measurable subset of $X$, and $F$ is a measurable subset of $Y$. Then $\Pi$ is a $\pi$ system, in the sense that if $E_1 \times F_1, E_2 \times F_2 \in \Pi$, then $(E_1 \times F_1) \cap (E_2 \times F_2) = (E_1 \cap E_2) \times (F_1 \cap F_2) \in \Pi$. Now let
  %
  \[ \Delta = \left\{ G \subset X \times Y: \left( \begin{array}{c} \text{for all $\varepsilon > 0$, there are simple $u_1, \dots, u_n$} \\ \text{on $X$ and $v_1,\dots, v_n$ on $Y$ such that} \\ \| \mathbf{I}_G - \sum u_i \otimes v_i \|_{L^q(Y) L^\infty(X)} < \varepsilon \end{array} \right) \right\}. \]
  %
  Our goal is to show that $\Delta$ is a $\lambda$ system. It is easy to see that $\Delta$ contains $\Pi$, so by the $\pi$-$\lambda$ theorem it follows that $\Delta$ contains all measurable subsets of $X \times Y$. Thus it suffices to show $\Delta$ is closed under complements and countable unions of disjoint sets. The complement property follows easily since $\mathbf{I}_{G^c} = 1 - \mathbf{I}_G$ and $1 = \mathbf{I}_X \otimes \mathbf{I}_Y$ is a tensor product. Next, if $G_1, G_2, \dots$ are a disjoint family of sets in $\Delta$, then for each $\varepsilon > 0$, and for each $k$ we can find $u_{k1}, \dots, u_{kN_k}$ and $v_{k1}, \dots, v_{kN_k}$ such that
  %
  \[ \left\| \mathbf{I}_{G_k} - \sum_{i = 1}^{N_k} u_{ki} \otimes v_{ki} \right\|_{L^q(Y) L^\infty(X)} < \varepsilon / 2^k. \]
  %
  \[ \| \mathbf{I}_{G_k} \|_{L^q(Y) L^\infty(X)} = \sup_{x \in X} |G_k(x)|^{1/q} \]
  %
  By monotone convergence, if $G = \bigcup G_k$, then for each fixed $x$,
  %
  \[ \lim_{N \to \infty} \int \mathbf{I}_G(x,y) - \sum_{k = 1}^N \mathbf{I}_{G_k}(x,y)\; dx \]
\end{proof}










\chapter{Riemann Theory of Trigonometric Series}

Using the techniques of measure theory, we can actually prove that the Fourier series is essentially the unique way of representing a function on any part of its domain as a trigonometric series.

\begin{lemma}
  For any sequence $u_n$ and set $E$ of finite measure,
  %
  \[ \lim_{n \to \infty} \int_E \cos^2(nx + u_n)\; dx = |E|/2 \]
\end{lemma}
\begin{proof}
  We have
  %
  \[ \cos^2(nx + u_n) = \frac{1 + \cos(2nx + 2u_n)}{2} = \frac{1}{2} + \frac{\cos(2nx) \cos(2u_n) - \sin(2nx) \sin(2u_n)}{2} \]
  %
  Since $\cos(2u_n)$ and $\sin(2u_n)$ are bounded, we have $\int \chi_E(x) \cos(2nx)$ and $\int \chi_E(x) \sin(2nx) \to 0$ as $n \to \infty$, and the same is true for the latter component of the sum since $\cos(2u_n)$ and $\sin(2u_n)$ are bounded, we conclude that
  %
  \[ \int_E \cos^2(nx + u_n) = \int \chi_E(x) \cos^2(nx + u_n) = |E|/2 \]
  %
  completing the proof.
\end{proof}

\begin{theorem}[Cantor-Lebesgue Theorem]
  If, for some pair of sequences $a_0, a_1, \dots$ and $b_0, b_1, \dots$ are chosen such that
  %
  \[ \sum_{n = 0}^\infty a_n \cos(2 \pi nx) + b_n \sin(2 \pi nx) \]
  %
  converges on a set of positive measure in $[0,1]$, then $a_n, b_n \to 0$.
\end{theorem}
\begin{proof}
  Let $E$ be the set of points upon which the trigonometric series converges. We write $a_n \cos(2 \pi n x) + b_n \sin(2 \pi n x) = r_n \cos(nx + c_n)$. The result of the theorem is then precisely that $r_n \to 0$. If this is not true, then we must have $\cos(nx + c_n) \to 0$ for every $x \in E$. In particular, the dominated convergence theorem implies that
  %
  \[ \lim_{n \to \infty} \int_E \cos(nx + c_n)^2\; dx = 0 \]
  %
  Yet we know this tends to $|E|/2$ as $n \to \infty$, which is a contradiction.
\end{proof}

TODO: EXPAND ON THIS FACT.






\section{Convergence in $L^p$ and the Hilbert Transform}

We now move onto a more 20th century viewpoint on Fourier series, namely, those to do with operator theory. Under this viewpoint, the properties of convergence are captured under the boundedness of certain operators on function spaces, allowing us to use the modern theory of functional analysis to it's full extent on our problems. However, unlike in most of basic functional analysis, where we assume all operators we encounter are bounded to begin with, in harmonic analysis we more often than not are given an operator defined only on a subset of spaces, and must prove the continuity of such an operator to show it is well defined on all of space. We will illustrate this concept through the theory of the circular Hilbert transform, and its relation to the norm convergence of Fourier series.

A \emph{Fourier multiplier} is a linear transform $T$ associated with a given sequence of scalars $\lambda_n$, for $n \in \ZZ$. It is defined for any trigonometric polynomial $f = \sum_{|n| \leq N} c_n e_n$ as $Tf = \sum_{|n| \leq N} \lambda_n c_n e_n$. The trigonometric polynomials are dense in $L^p(\mathbf{T})$, for each $p < \infty$. An important problem is determining whether $T$ is therefore figuring out whether the operator can be extended to a {\it continuous operator} on the entirety of $L^p$. Because the trigonometric polynomials are dense in $L^p$, in the light of the Hahn Banach theorem it suffices to prove an inequality of the form $\| Tf \| \lesssim \| f \|$. Here are some examples of Fourier operators we have already seen.

\begin{example}
    The truncation operator $S_N$ is the transform associated with the scalars $\lambda_n = [|n| \leq N]$. The truncation is continuous, since for any integrable function $f$, the Fourier coefficients are uniformly bounded by $\| f \|_1$, so $\| S_N f \|_1 \leq N \| f \|_1$. Similarily, the F\'{e}jer truncation $\sigma_N$ associated to the multipliers $\lambda_N = [|n| \leq N](1 - |n|/N)$ is continuous on all integrable functions. These operators are easy to extend precisely because the nonzero multipliers have finite support.
\end{example}

\begin{example}
    In the case of the Abel sum, $A_r$, associated with $\lambda_n = r^{|n|}$, $A_r$ extends in a continuous way to all integrable functions, since
    %
    \[ |A_r f| = \left| \sum r^{|n|} \widehat{f}(n) e_n(t) \right| \leq \| f \|_1 \sum r^{|n|} = \| f \|_1 \left( 1 + \frac{2}{1 - r} \right) \]
    %
    Thus the map is bounded.
\end{example}

To understand whether the truncations $S_N f$ of $f$ converge to $f$ in the $L^p$ norms, rather than pointwise, we turn to the analysis of an operator which is the core of the divergence issue, known as the \emph{Hilbert transform}. It is a Fourier multiplier operator $H$ associated with the coeficients
%
\[ \lambda_n = \frac{\text{sgn}(n)}{i} = \begin{cases} +1/i & n > 0 \\ 0 & n = 0 \\ -1/i & n < 0 \end{cases} \]
%
Because
%
\[ [|n| \leq N] = \frac{\text{sgn}(n + N) - \text{sgn}(n-N)}{2} + \frac{[n = N] + [n = -N]}{2} \]
%
we conclude
%
\[ S_n f = \frac{i \left( e_{-n} H(e_n f) - e_n H(e_{-n} f) \right)}{2} + \frac{\widehat{f}(n) e_n + \widehat{f}(-n) e_{-n}}{2} \]
%
Since the operators $f \mapsto \widehat{f}(n) e_n$ are bounded in all the $L^p$ spaces since they are continuous in $L^1(\mathbf{T})$, we conclude that the operators $S_n$ are uniformly bounded as endomorphisms on $L^p(\mathbf{T})$ provided that $H$ is bounded as an operator from $L^p(\mathbf{T})$ to $L^q(\mathbf{T})$. Since $S_n f$ converges to $f$ in $L^p$ whenever $f$ is a trigonometric polynomial, this would establish that $S_n f$ converges to $f$ in the $L^p$ norm for any function $f$ in $L^p(\mathbf{T})$. Later on, as a special case of the Hilbert transform on the real line, we will be able to prove that $H$ is a bounded operator on $L^p(\mathbf{T})$ for all $1 < p < \infty$, and as a result, we find that $S_N f \to f$ in $L^p$ for all such $p$. Unfortunately, $H$ is not bounded from $L^1(\mathbf{T})$ to itself, and correspondingly, $S_N f$ does not necessarily converge to $f$ in the $L^1$ norm for all integrable $f$.

For now, we explore some more ideas in how we can analyze the Hilbert transform via convolution, the dual of Fourier multipliers. The fact that $\smash{\widehat{f * g} = \widehat{f} \widehat{g}}$ implies that if their is an integrable function $g$ whose Fourier coefficients corresponds to the multipliers of an operator $T$, then $f * g = Tf$ for any trigonometric polynomial $f$, and by the continuity of convolution, this is the unique extension of the Fourier multiplier operator. In the theory of distributions, one generalizes the family of objects one can take the Fourier series from integrable functions to a more general family of objects, such that every sequence of Fourier coefficients is the Fourier series of some {\it distribution}. One can take the convolution of any such distribution $\Lambda$ with a $C^\infty$ function $f$, and so one finds that $\Lambda * f = Tf$ for any trigonometric polynomial $f$. There is a theorem saying that {\it all} continuous translation invariant operators from $L^p(\mathbf{T})$ to $L^q(\mathbf{T})$ are given by convolution with a Fourier multiplier operator. In practice, we just compute the convolution kernel which defines the Fourier multiplier, but it is certainly a satisfying reason to justify the study of Fourier multipliers. For instance, a natural question is to ask which Fourier multipliers result in bounded operations in space.

\begin{theorem}
    A Fourier multiplier is bounded from $L^2(\mathbf{T})$ to itself if and only if the coefficients are bounded.
\end{theorem}
\begin{proof}
    If a Fourier multiplier is given by $\lambda_n$, then for some trigonometric polynomial $f$,
    %
    \[ \| Tf \|_2^2 = \sum \left|\widehat{Tf}(n) \right|^2 = \sum |\lambda_n|^2 \left| \widehat{f}(n) \right|^2 \]
    %
    If the $\lambda_n$ are bounded, then we can obtain from this formula the bound
    %
    \[ \| Tf \|_2^2 \leq \max |\lambda_n| \| f \|_2^2 \]
    %
    Conversely, if $Tf$ is bounded, then
    %
    \[ |\lambda_n^2| = \| T(e_n) \|_2^2 \leq \| T \|^2 \]
    %
    so the $\lambda_n$ are bounded.
\end{proof}

\begin{corollary}
    The Hilbert transform is a bounded endomorphism on $L^2(\mathbf{T})$. Note that we already know that $S_N f \to f$ in the $L^2$ norm.
\end{corollary}

The terms of the Hilbert transform cannot be considered the Fourier coefficients of any integrable function. Indeed, they don't vanish as $n \to \infty$. Nonetheless, we can use Abel summation to treat the Hilbert transform as convolution with an appropriate operator. For $0 < r < 1$, consider, for $z = e^{it}$,
%
\[ K_r(z) = \sum_{n \in \ZZ} \frac{\text{sgn}(n)}{i} r^{|n|} z^n = K * P_r \]
%
Since we know the Hilbert transform is continuous in $L^2(\mathbf{T})$, we can conclude that, in particular, for any $C^\infty$ function $f$,
%
\[ H f = \lim_{r \to 1} K * (P_r * f) = \lim_{r \to 1} (K * P_r) * f = \lim_{r \to 1} K_r * f \]
%
So it suffices to determine the limit of the $K_r$. We find that
%
\begin{align*}
    \sum_{n = 1}^\infty \frac{(rz)^n - (r \overline{z})^n}{i} &= \frac{r}{i} \left( \frac{1}{\overline{z} - r} - \frac{1}{z - r} \right) = \frac{r}{i} \frac{z - \overline{z}}{|z|^2 - 2r \text{Re}(z) + r^2}\\
    &= \frac{2r \sin(t)}{1 - 2r \cos(t) + r^2} = \frac{4r \sin(t/2) \cos(t/2)}{(1 - r)^2 + 4r \sin^2(t/2)}\\
    &= \cot(t/2) + O \left( \frac{(1 - r)^2}{t^3} \right)
\end{align*}
%
Thus $K_r(t)$ tends to $\cot(t/2)$ locally uniformly away from the origin. But
%
\[ K_r(t) = \frac{4r \sin(t/2) \cos(t/2)}{(1 - r)^2 + 4r\sin^2(t/2)} = O \left( \frac{t}{(1 - r)^2} \right) \]
%
If $f$ is any $C^\infty$ function on $\mathbf{T}$, then
%
\[ \left| \int_{|t| \geq \varepsilon} [K_r(t) - \cot(t/2)] f(t) \right| \lesssim (1 - r)^2 \| f \|_\infty \int_{|t| \geq \varepsilon} \frac{dt}{|t|^3} \lesssim \frac{(1 - r)^2 \| f \|_\infty}{\varepsilon^2} \]
%
\begin{align*}
    \left| \int_{|t| < \varepsilon} K_r(t) f(t)\; dt \right| &\leq \int_0^\varepsilon |K_r(t)||f(t) - f(-t)|\\
    &\lesssim \int_0^\varepsilon |tK_r(t)||f'(0)| \lesssim \frac{|f'(0)|}{(1 - r)^2} \int_0^\varepsilon t^2 \lesssim \| f' \|_\infty \frac{\varepsilon^3}{(1 - r)^2}
\end{align*}
%
\[ \left| \int_{|t| < \varepsilon} \cot(t/2) f(t)\; dt \right| \lesssim \int_0^\varepsilon \frac{|f(t) - f(-t)|}{t} \lesssim \varepsilon f'(0) \]
%
Thus
%
\[ \left| \int K_r(t) f(t)\; dt - \int \cot(t/2) f(t)\; dt \right| \lesssim \frac{(1 - r)^2}{\varepsilon^2} \| f \|_\infty + \left( \frac{\varepsilon^3}{(1 - r)^2} + \varepsilon \right) \| f' \|_\infty \]
%
Choosing $\varepsilon = (1 - r)^\alpha$ for some $2/3 < \alpha < 1$ shows that for sufficiently smooth $f$,
%
\[ (Hf)(x) = \lim_{r \to 1} \int \cot(t/2) f(x - t)\; dt \]


\section{A Divergent Fourier Series}

Analysis was built to analyze continuous functions, so we would hope the method of fourier expansion would work for all continuous functions. Unfortunately, this is not so. The behaviour of the Dirichlet kernel away from the origin already tells us that the convergence of Fourier series is subtle. We shall take advantage of this to construct a continuous function with divergent fourier series at a point.

To start with, we shall consider the series
%
\[ f(t) \sim \sum_{n \neq 0} \frac{e_n(t)}{n} \]
%
where $f$ is an odd function equaling $i(\pi - t)$ for $t \in (0,\pi]$. Such a function is nice to use, because its Fourier representation is simple, yet very close to diverging. Indeed, if we break the series into the pair
%
\[ \sum_{n = 1}^\infty  \frac{e_n(t)}{n}\ \ \ \ \ \ \ \ \ \ \sum_{n = -\infty}^{-1} \frac{e_n(t)}{n} \]
%
Then these series no longer are the Fourier representations of a Riemann integrable function. For instance, if $g(t) \sim \sum_{n = 1}^\infty \frac{e_n(t)}{n}$, then the Abel means

$A_r(f)(t) = $

\section{Conjugate Fourier Series}

When $f$ is a real-valued integrable function, then $\overline{\widehat{f}(-n)} = \widehat{f}(n)$. Thus we formally calculate that
%
\[ \sum_{n = -\infty}^\infty \widehat{f}(n) e_n(t) = \text{Re} \left( \widehat{f}(0) + 2\sum_{n = 1}^\infty \widehat{f}(n) e_n(t) \right) \]
%
This series defines an analytic function in the interior of the unit circle since the coefficients are bounded. Thus the sum is a harmonic function in the interior of the unit circle. The imaginary part of this sum is
%
\[ \text{Im} \left( \widehat{f}(0) + 2\sum_{n = 1}^\infty \widehat{f}(n) e_n(t) \right) = \Re \left( -i \sum_{n = -\infty}^\infty \text{sgn}(n) \widehat{f}(n) e_n(t) \right) \]
%
The right hand side is known as the conjugate series to the Fourier series $\widehat{f}(n)$. It is closely related to the study of a function $\tilde{f}$ known as the {\it conjugate function}.







\chapter{Oscillatory Integrals}

The goal of the theory of oscillatory integrals is to obtain estimates of integrals with highly oscillatory integrands, where standard techniques such as taking in absollute values, or various spatial decomposition strategies, fail completely to give tight estimates. A typical oscillatory integral is of the form
%
\[ I(\lambda) = \int e^{\lambda i \phi(x)} \psi(x)\; dx, \]
%
where $\phi$ and $\psi$ are scalar valued functions, known as the \emph{phase} and \emph{amplitude} functions. The value $\lambda$ is a parameter measuring the degree of oscillation. As $\lambda$ increases, oscillation increases, which implies more cancellation should occur on average, hence we should expect $I(\lambda)$ to decay as $\lambda \to \infty$. One of the main problems in the study of oscillatory integrals is to measure the asymptotic decay more precisely.

\begin{example}
    The most basic example of an oscillatory integral is the Fourier transform, where for each function $f \in L^1(\RR)$, and each $\xi \in \RR$, we consider the quantity
    %
    \[ \widehat{f}(\xi) = \int_{-\infty}^\infty e(-\xi x) f(x)\; dx. \]
    %
    Thus $f$ plays the role of the amplitude, the phase function is $\phi(x) = x$, and $\xi$ takes the role of $\lambda$. The basic theory of the Fourier transform hints that we can obtain decay in this integral as $\xi \to \infty$ by exploiting the smoothness of the function $f$.
\end{example}

There are two main tools to estimate oscillatory integrals. The first, the method of steepest descent, uses complex analysis to shift the integral to a domain where less oscillation occurs, so that standard estimation strategies can be exmployed. However, this method seems to have limited applicability to oscillatory integrals over multivariable domains. The second method, known as the method of stationary phase, states that if $\phi$ is smooth, and $\nabla \phi$ has an isolated family of zeroes, then the oscillatory integral asymptotics can be localized to regions around the values $x_0$ with $\nabla \phi(x_0) = 0$. Heuristically, each zero $x_0$ contributes $\psi(x_0) e ( \lambda \phi(x_0) )$, times the volume of the region around $x_0$ where $\phi$ deviates by $O(1/\lambda)$ to the overall asymptotics.

\section{One Dimensional Theory}

Let us begin with a simple example of an oscillatory integral, i.e.
%
\[ I(\lambda) = \int_J e^{i \lambda \phi(x)}\; dx, \]
%
where $J$ is a closed interval, and $\phi: J \to \RR$ is Borel measurable. Taking in absolute values shows that $|I(\lambda)| \leq |J|$ for all $\lambda$. If $\phi$ is constant, then $I(\lambda) = |J| e^{i \lambda \phi}$, so in this case the estimate is sharp. But if $\phi$ varies, we expect $I(\lambda)$ to decay as $\lambda \to \infty$. For instance, the Esse\'{e}n concentration inequality shows that if we are to expect \emph{average} decay in the integral $I$ over a range of $\lambda$, then $\phi$ must not be concentrated around any point.

\begin{theorem}[Esse\'{e}n Concentration Inequality]
  Let $\phi: J \to \RR$ be Borel measurable, and for each $\lambda \in \RR$, set
  %
  \[ I(\lambda) = \int_J e^{i \lambda \phi(x)}\; dx. \]
  %
  Then for any $\varepsilon > 0$,
  %
  \[ \sup_{\phi_0 \in \RR} |\{ x \in [0,1]: |\phi(x) - \phi_0| \leq \varepsilon \}| \lesssim \varepsilon \int_0^{1/\varepsilon} |I(\lambda)|\; d\lambda, \]
  %
  where the implicit constant is independant of $\phi$.
\end{theorem}
\begin{proof}
  By rescaling, we may assume that $J = [0,1]$. Moreover, for any choice of $\phi_0$, we may replace $\phi$ with $\phi - \phi_0$, reducing the analysis to the case where $\phi_0 = 0$. Similarily, replacing $\phi$ with $\phi/\varepsilon$ reduces us to the situation where $\varepsilon = 1$. Thus we must show
  %
  \[ |\{ x \in [0,1]: |\phi(x)| \leq 1 \}| \lesssim \int_0^1 |I(\lambda)|\; d\lambda, \]
  %
  where the implicit constant is independant of the function $\phi$. If $\psi$ is an integrable function supported on $[0,1]$, then Fubini's theorem implies
  %
  \begin{align*}
    \int_0^1 \psi(\lambda) I(\lambda)\; d\lambda &= \int_0^1 \int_0^1 \psi(\lambda) e^{\lambda i \phi(x)}\; d\lambda\; dx\\
    &= \int_0^1 \widehat{\psi}(- \phi(x) / 2 \pi)\; dx.
  \end{align*}
  %
  In particular, this means that
  %
  \[ \left| \int_0^1 \widehat{\psi}(- \phi(x) / 2\pi)\; dx \right| \leq \| \psi \|_{L^\infty[0,1]} \int_0^1 |I(\lambda)|\; d\lambda. \]
  %
  If we choose a bounded function $\psi$ such that $\widehat{\psi}$ is non-negative, and bounded below on $[-2\pi,2\pi]$, then
  %
  \[ \left| \int_0^1 \widehat{\psi}(- \phi(x) / 2 \pi)\; dx \right| \gtrsim |\{ x \in [0,1]: |\phi(x)| \leq 1 \}|, \]
  %
  and so the claim follows easily.
\end{proof}

Thus if large cancellation happens in $I(\lambda)$ for the average $\lambda$, this automatically implies that $\phi$ cannot be concentrated around any particular point.  Conversely, we want to show that if $\phi$ varies significantly, then $I$ exhibits cancellation as $\lambda \to \infty$. The condition that $\phi'$ is bounded below is not sufficient to guarantee cancellation independant of the function $\phi$, as the next example shows, if the integrand oscillated at a wavelength $1/\lambda$.

\begin{example}
  Fix $\lambda_0 \in \ZZ$, and let $\phi(x) = 2 \pi x + f(\lambda_0 x) / \lambda_0$, where $f$ is smooth and 1-periodic, $\| f' \|_{L^\infty(\RR)} \leq \pi$, and
  %
  \[ \int_0^1 e^{2 \pi i x + i f(x)}\; dx \neq 0. \]
  %
  Then for each $x \in \RR$, $\pi \leq |\phi'(x)| \leq 3\pi$, and in particular, is bounded independently of $\lambda_0$. Since $\phi(x + 1/\lambda_0) = \phi(x) + 2 \pi / \lambda_0$, we find $e^{i \lambda_0 \phi(x)}$ is $1/\lambda_0$ periodic. In particular, this means
  %
  \[ I(\lambda_0) = \int_0^1 e^{\lambda_0 i \phi(x)} = \int_0^1 e^{2 \pi i x + i f(x)}\; dx. \]
  %
  which is comparable to 1, independantly of $\lambda_0$.
\end{example}

Controlling $\phi''$ in addition to $\phi'$, however, is sufficient.

\begin{theorem}
  Let $\phi: J \to \RR$ be smooth, and suppose there exists constants $A,B > 0$ with $|\phi'(x)| \geq A$ and $|\phi''(x)| \leq B$ for all $x \in J$. Then for all $\lambda > 0$, we find
  %
  \[ |I(\lambda)| \lesssim \frac{1}{\lambda} \left( \frac{1}{A} + \frac{B}{A^2} |J| \right). \]
\end{theorem}
\begin{proof}
  A dimensional analysis shows that the inequality is invariant under rescalings in $x$ and $\lambda$, so we may assume that $J = [0,1]$, and $\lambda = 1$. An integration by parts shows that
  %
  \begin{align*}
    \int_0^1 e^{i \phi(x)}\; dx &= \int_0^1 \frac{1}{i \phi'(x)} \frac{d}{dx} \left( e^{i \phi(x)} \right)\; dx\\
    &= \left( \frac{e^{i \phi(1)}}{i \phi'(1)} - \frac{e^{i \phi(0)}}{i \phi'(0)} \right) - \int_0^1 \frac{d}{dx} \left( \frac{1}{\phi'(x)} \right) e^{i \phi(x)}.
  \end{align*}
  %
  Now
  %
  \[ \frac{d}{dx} \left( \frac{1}{\phi'(x)} \right) = - \frac{\phi''(x)}{\phi'(x)^2}, \]
  %
  so taking in absolute values completes the proof.
\end{proof}

One can keep applying absolute values to obtain further bounds in terms of higher order derivatives of $\phi$. For instance, another integration by parts shows that if there is $A,B,C > 0$ such that for $x \in J$, if $\phi'(x) \geq A$, $\phi''(x) \leq B$, and $\phi'''(x) \leq C$, then
%
\[ |I(\lambda)| \lesssim \frac{1}{\lambda} \left( \frac{1}{A} \right) + \frac{1}{\lambda^2} \left( \frac{B}{A^3} + \frac{C}{A^3} |J| + \frac{B^2}{A^4} |J| \right). \]
%
One can keep taking in absolute values, but the $1/\lambda$ decay will still remain. This is to be expected, for instance, if $\phi(x) = x$ and $J = [0,1]$ then
%
\[ \limsup_{\lambda \to \infty} |I(\lambda) \cdot \lambda| = 2, \]
%
so we cannot obtain any better decay than $1/\lambda$ here.

Another option is to not require control on the second derivative of the phase, but instead to assume that $\phi'$ is monotone, which prevents the kind of oscillation present in our counterexample.

\begin{lemma}[Van der Corput]
  Let $\phi: \RR \to \RR$ be a smooth phase such that $|\phi'(x)| \geq A$ for all $x \in J$, and $\phi'$ is monotone. Then for all $\lambda > 0$ we have
  %
  \[ |I(\lambda)| \lesssim \frac{1}{A \lambda}, \]
  %
  where the implicit constant is independent of $J$.
\end{lemma}
\begin{proof}
  The same integration by parts as before shows that if $J = [a,b]$,
  %
  \begin{align*}
    \int_J e^{\lambda i \phi(x)}\; dx &= \left( \frac{e^{i \phi(b)}}{\lambda i \phi'(b)} - \frac{e^{i \phi(a)}}{i \phi'(a)} \right) + \frac{1}{i\lambda} \int_J \frac{d}{dx} \left( \frac{1}{\phi'(x)} \right) e^{i \phi(x)}\; dx.
  \end{align*}
  %
  The two endpoints are $O(1/A \lambda)$. For the second sum, we perform a simple trick. Since $\phi'$ is monotone, so too is $1/\phi'$, so in particular, it's derivative has a constant sign. Thus by the fundamental theorem of calculus,
  %
  \begin{align*}
    \left| \int_J \frac{d}{dx} \left( \frac{1}{\phi'(x)} \right) e^{i \phi(x)}\; dx \right| &\leq \int_J \left| \frac{d}{dx} \left( \frac{1}{\phi'(x)} \right)  \right|\; dx\\
    &= \left| \int_J \frac{d}{dx} \left( \frac{1}{\phi'(x)} \right) \right|\; dx\\
    &= \frac{1}{\phi'(b)} - \frac{1}{\phi'(a)}.
  \end{align*}
  %
  Combining these inequalities completes the proof.
\end{proof}

Since the Van der Corput bound does not depend on $|J|$, it can be easily iterated to give a theorem about higher derivatives of a function $\phi$.

\begin{lemma}
  Let $\phi: \RR \to \RR$ be smooth, and suppose there is some $k \geq 2$ such that $|\phi^{(k)}(x)| \geq A$ for all $x \in J$. Then for all $\lambda > 0$, we find
  %
  \[ |I(\lambda)| \lesssim_k \frac{1}{(A \lambda)^{1/k}}, \]
  %
  where the implicit constant is independant of $J$.
\end{lemma}
\begin{proof}
  We perform an induction on $k$, the case $k = 1$ already proven. By scale invariance, we may assume $\lambda = 1$. Now $\phi^{(k-1)}$ is monotone, so for each $\alpha > 0$, outside an interval of length at most $O(\alpha/A)$, $|\phi^{(k-1)}(x)| \geq \alpha$. Thus applying the trivial bound in the excess region, and the case $k - 1$ on the other intervals, we conclude
  %
  \[ |I(\lambda)| \lesssim_k \frac{\alpha}{A} + \alpha^{-1/(k-1)} \]
  %
  Optimizing over $\alpha$, we find $|I(\lambda)| \lesssim_k A^{-1/k}$.
\end{proof}

Let us now consider a one dimensional oscillatory integral with a varying amplitude $\psi$, i.e.
%
\[ I(\lambda) = \int_{-\infty}^\infty e^{i \lambda \phi(x)} \psi(x)\; dx. \]
%
The Van der Corput lemma also applies here.

\begin{lemma}
  Fix $k \geq 1$. Suppose $\psi$ is supported on $[a,b]$, and suppose $|\phi^{(k)}(x)| \geq A$ for all $x \in [a,b]$, with $\phi'$ monotone if $k = 1$. Then
  %
  \[ |I(\lambda)| \lesssim_k \frac{\| \psi \|_{L^\infty(\RR)} + \| \psi' \|_{L^1(\RR)}}{(A \lambda)^{1/k}}. \]
\end{lemma}
\begin{proof}
  Fix $c_0 \in [a,b]$, and define
  %
  \[ I_0(x) = \int_{c_0}^x e^{i \lambda \phi(t)}\; dt. \]
  %
  The standard Van-der Corput lemma implies that for all $x$,
  %
  \[ |I_0(x)| \lesssim_k \frac{1}{(A \lambda)^{1/k}}. \]
  %
  An integration by parts gives that for any $a < b$,
  %
  \begin{align*}
    \int_a^b \psi(x) e^{i \lambda \phi(x)}\; dx &= \int_a^b \psi(x) I_0'(x)\; dx\\
    &= [\psi(b) I_0(b) - \psi(a) I_0(a)] - \int_a^b \psi'(x) I_0(x)\; dx.
  \end{align*}
  %
  Now
  %
  \[ |\psi(b) I_0(b) - \psi(a) I_0(a)| \lesssim \frac{\| \psi \|_{L^\infty(\RR)}}{(A \lambda)^{1/k}} \]
  %
  and
  %
  \[ \left| \int_a^b \psi'(x) I_0(x)\; dx \right| \lesssim_k \frac{\| \psi' \|_{L^1(\RR)}}{(A \lambda)^{1/k}}. \]
  %
  Putting these two estimates together completes the proof.
\end{proof}

If $\psi$ is smooth and compactly supported, integration by parts is very successful because there are no boundary terms.

\begin{theorem}
    If $\phi$ and $\psi$ are smooth, with $\psi$ compactly supported, and $\phi'(x) \neq 0$ for all $x$ in the support of $\psi$, then for all $N > 0$,
    %
    \[ I(\lambda) \lesssim_N 1/\lambda^N, \]
    %
    where the implicit constants depend on the functions $\phi$ and $\psi$.
\end{theorem}
\begin{proof}
  A single integration by parts gives
  %
  \begin{align*} I(\lambda) &= \frac{1}{\lambda} \int \frac{\psi(x)}{i \phi'(x)} \frac{d}{dx} \left( e^{\lambda i \phi(x)} \right)\\
  &= - \frac{1}{i \lambda} \int \frac{d}{dx} \left( \frac{\psi(x)}{\phi'(x)} \right) e^{i \phi(x)}\\
  &= \frac{1}{i \lambda} \int \frac{\phi'(x) \psi'(x) - \psi(x) \phi''(x)}{\phi'(x)^2}.
  \end{align*}
  %
  Further integration by parts give, for each $N$, that
  %
  \[ I(\lambda) = \lambda^{-N} \int \frac{P(x)}{\phi'(x)^{2N}} e^{i \phi(x)}, \]
  %
  where $P(x)$ is a polynomial function in the derivatives of $\phi$ and $\psi$ up to order $N+1$, in particular, with the same support as $\psi$. Thus we can take in absolute values and integrate to conclude $|I(\lambda)| \lesssim_{\psi,N} \lambda^{-N}$.
\end{proof}

\begin{remark}
  We note that the implicit constants in the theorem for a particular $N$ can be upper bounded uniformly, given uniform upper bounds on the measure of the support of $\psi$, upper bounds on the derivatives of $\phi$ and $\psi$ of order up to $N+1$, and lower bounds on $\phi'$ over the support of $\psi$.
\end{remark}

Let us now move onto a `stationary phase', i.e. a phase $\phi$ whose derivative vanishes at a point. The simplest example of such a phase is the integral
%
\[ I(\lambda) = \int_{-\infty}^\infty e^{i \lambda x^2} \psi(x)\; dx. \]
%
Our heuristics tell us $I(\lambda)$ decays on the order of $\lambda^{-1/2}$, which agrees with the asymptotics we now find.

\begin{theorem}
  Let $\psi \in \mathcal{S}(\RR)$ be a Schwartz amplitude. Then for each $N \geq 0$,
  %
  \[ \int_{-\infty}^\infty \psi(x) e^{\lambda i x^2}\; dx = e^{i \pi / 4} \cdot \pi^{1/2} \cdot \sum_{n = 0}^N \frac{i^n \psi^{(2n)}(0)}{4^n \lambda^{n + 1/2}} + O_{N,\psi}(1/\lambda^{N + 3/2}). \]
\end{theorem}
\begin{comment}
\begin{proof}
  Rescaling, it suffices to prove the theorem when $\psi(x) = 1$ whenever $|x| < 1$. Let $\alpha(x)$ be a smooth function with $\alpha(x) = 1$ for $|x| \geq 1/2$, and with $\alpha(x) = 0$ for $|x| < 1/4$. Then for each $k \geq 1$, define
  %
  \[ \beta_k(x) = \alpha(2^k x) - \alpha(2^{k-1} x). \]
  %
  Then $\beta_k$ is supported on $1/2^{k+2} \leq |x| \leq 1/2^k$, and moreover, for each $x \in \RR$,
  %
  \[ \alpha(x) + \sum_{k = 1}^\infty \beta_k(x) = 1. \]
  %
  It is simple to see that
  %
  \begin{align*}
    \int_{-\infty}^\infty \psi(x) e^{\lambda i x^2}\; dx &= \int_{-\infty}^\infty \alpha(x) \psi(x) e^{\lambda i x^2}\; dx\\
    &\ \ \ \ + \sum_{k = 1}^\infty \int_{-\infty}^\infty \beta_k(x) e^{\lambda i x^2}\; dx.
  \end{align*}
  %
  Now $\alpha \psi$ is a compactly supported amplitude supported away from the origin, so for each $N$,
  %
  \[ \left| \int_{-\infty}^\infty \alpha(x) \psi(x) e^{\lambda i x^2}\; dx \right| \lesssim_{\alpha,\psi,N} \lambda^{-N}. \]
  %
  The same argument works for $\beta_1$, and so by rescaling, for each $k$ and $M$,
  %
  \begin{align*}
    \left| \int_{-\infty}^\infty \beta_k(x) e^{\lambda i x^2}\; dx \right| &\lesssim_{\alpha,\psi,M} 2^{(2M-1)k} \lambda^{-M}.
  \end{align*}
  %
  In particular, we may sum the inequality for small $k$, and with $M$ an appropriate multiple of $N$, to conclude
  %
  \[ \sum_{k = 1}^{\lg(\lambda^{1/2-\varepsilon})} \left| \int_{-\infty}^\infty \beta_k(x) e^{\lambda i x^2}\; dx \right| \lesssim_{\alpha,\psi,N,\varepsilon} \lambda^{-N}. \]
  %
  If we set
  %
  \[ \gamma(x) = \sum_{k = \lg(\lambda^{1/2-\varepsilon})}^\infty \beta_k(x), \]
  %
  then $\gamma(x) = 0$ for $|x| \geq 1/\lambda^{3/4}$, and $\gamma(x) = 1$ for $|x| \leq 1/4\lambda^{3/4}$. Rescaling, we have
  %
  \[ \int_{-\infty}^\infty \gamma(x) e^{\lambda i x^2} = \lambda^{-3/4} \int_{-\infty}^\infty \gamma(x \cdot \lambda^{3/4}) e^{ix^2 / \lambda^{1/2}}. \]
\end{proof}


IDEA: Sum up dyadically on intervals $|x| \sim 2^k \lambda^{-1/2}$, for $k = 1$ to $k = \lfloor \log( \lambda^{1/2} \varepsilon) - 2 \rfloor$, then hopefully the oscillatory integral with phase $x^2$ and amplitude $\psi(x) \sum_{k = \lfloor \lg(\lambda^{1/2} \varepsilon) - 2 \rfloor}^\infty \beta_k(x/\lambda^{1/2})$ decays arbitrarily fast in $\lambda$?


Then for each $n$, define $\beta_n(x) = \alpha(x/2^n) - \alpha(x/2^{n-1})$. Thus we have $\alpha(x) + \sum_{k = 1}^\infty \beta_k(x) = 1$ for all $x \in \RR$. Moreover, $\beta_k$ is supported on $[-2^n, -$

\end{comment}
\begin{proof}
  Applying the multiplication formula for the Fourier transform, noting that the distributional Fourier transform of $e^{i \lambda x^2}$ is
  %
  \[ e^{i \pi / 4} (\pi/\lambda)^{1/2} e^{-i \pi^2 \xi^2 / \lambda}, \]
  %
  we conclude that
  %
  \[ I(\lambda) = e^{i \pi / 4} (\pi/\lambda)^{1/2} \int_{-\infty}^\infty e^{-i \pi^2 \xi^2 / \lambda} \widehat{\psi}(\xi)\; d\xi. \]
  %
  Now for any $N$, we can write
  %
  \[ e^{-i \pi^2 \xi^2 / \lambda} = \sum_{n = 0}^N \frac{1}{n!} \left( \frac{-i \pi^2 \xi^2}{\lambda} \right)^n + O_N \left( (\xi^2 / \lambda)^{N+1} \right). \]
  %
  Thus substituting in the Taylor series, and then applying the Fourier inversion formula, we find
  %
  \begin{align*}
    I(\lambda) &= e^{i \pi / 4} (\pi/\lambda)^{1/2} \sum_{n = 0}^N \frac{1}{n!} \int_{-\infty}^\infty \left( \frac{-i \pi^2 \xi^2}{\lambda} \right)^n \widehat{\psi}(\xi)\; d\xi + O_{\psi,N} \left( 1/\lambda^{N+3/2}  \right)\\
    &= e^{i \pi / 4} (\pi/\lambda)^{1/2} \sum_{n = 0}^N \frac{i^n}{4^n n!} \frac{1}{\lambda^n} \int_{-\infty}^\infty (2\pi i \xi)^{2n} \widehat{\psi}(\xi)\; d\xi + O_{\psi,N} \left( 1 / \lambda^{N+3/2} \right)\\
    &= e^{i \pi / 4} (\pi/\lambda)^{1/2} \sum_{n = 0}^N \frac{i^n \psi^{(2n)}(0)}{4^n n!} \frac{1}{\lambda^n} + O_{\psi,N} \left( 1 / \lambda^{N+3/2} \right). \qedhere
  \end{align*}
\end{proof}

\begin{remark}
  The implicit constant can be made independent of $\psi$ given uniform upper bounds on
  %
  \[ \int_{-\infty}^\infty |\widehat{\psi}(\xi)| |\xi|^{2(N+1)}\; d\xi. \]
  %
  In particular, this can be obtained by uniform upper bounds on the support of $\psi$, upper bounds on the magnitude of $\psi$, and upper bounds on the magnitude of the $(2N+4)$th derivative of $\psi$.
\end{remark}

It requires only a simple change of variables to extend this theorem to arbitrary quadratic phases. We say a critical point of a function is \emph{non-degenerate} if the second derivative at that point is nonzero.

\begin{theorem}
  Let $\phi$ be a smooth phase with finitely many non-degenerate critical points, and let $\psi$ be a smooth compactly supported amplitude function. Then there exists a sequence of constants $\{ a_n \}$, depending on the derivatives of $\phi$ and $\psi$ at the critical points, such that for each $N \geq 0$,
  %
  \[ I(\lambda) = \lambda^{-1/2} \sum_{n = 0}^N a_n \lambda^{-n} + O_{\phi,\psi,N} ( 1/\lambda^{N+3/2} ). \]
  %
  In particular, if $\phi$ has a single critical point at some $x_0$, then
  %
  \[ a_0 = \sqrt{ \frac{2\pi}{-i \phi''(x_0)} } \cdot e^{\lambda i \phi(x_0)} \psi(x_0). \]
\end{theorem}
\begin{proof}
  By a partition of unity argument, it suffices to prove this theorem assuming that $\phi$ has only a single stationary point, which by translation we may assume to be at the origin, with $\phi(0) = 0$, and that $\phi(x)$ and $\phi'(x)$ are nonzero for all nonzero $x$ in the support of $\psi$. Moreover, rescaling enables us to assume $\phi''(0) = 2$. We can define a function
  %
  \[ y(x) = \text{sgn}(x) \cdot \phi(x)^{1/2}. \]
  %
  Then $y$ is a smooth function in the support of $\phi$. By the change of variables formula, there exists a smooth, compactly supported function $\psi_0(y)$ such that
  %
  \[ I(\lambda) = \int \psi(x) e^{\lambda i \phi(x)}\; dx = \int \psi_0(y) e^{\lambda i y^2}\; dy. \]
  %
  Thus we can apply the previous theorem to conclude that there exists a sequence of constants $\{ a_n \}$ such that for each $N$,
  %
  \[ I(\lambda) = \lambda^{-1/2} \sum_{n = 0}^N a_n \lambda^{-n} + O_{\phi,\psi,N}(1/\lambda^{N+3/2}). \]
  %
  The existence in this theorem is a \emph{constructive} existence statement. The proof gives an effective algorithm to produce the constants $a_n$ for any particular phase $\phi$. In particular,
  %
  \[ a_0 = e^{i\pi/4} \pi^{1/2} \psi_0(0) = e^{i\pi/4} \pi^{1/2} \left( \frac{\psi(0)}{y'(0)} \right). \]
  %
  Since
  %
  \begin{align*}
    y'(0) &= \lim_{x \to 0} y'(x) = \lim_{x \to 0} \frac{\phi'(x)}{2 \text{sgn}(x) \phi(x)^{1/2}}\\
    &= \frac{1}{2} \lim_{x \to 0} \frac{\phi'(x)}{x} \left( \frac{x^2}{\phi(x)} \right)^{1/2} = \frac{\phi''(0)}{2 \phi''(0)^{1/2}} = \phi''(0)^{1/2}/2 = 2^{1/2}.
  \end{align*}
  %
  Thus $a_0 = 2^{1/2} e^{i\pi/4} \pi^{1/2} \psi(0)$.
\end{proof}

\begin{remark}
  If we incorporate $\lambda$ into the phase, considering the oscillatory integral
  %
  \[ \int e^{i \phi(x)} \psi(x)\; dx, \]
  %
  then if $\phi$ has a nondegenerate stationary point at $x_0$, the last theorem says that
  %
  \[ \int e^{i \phi(x)} \psi(x)\; dx \approx \left( \frac{2\pi}{-i \phi''(x_0)} \right)^{1/2} e^{i \phi(x_0)} \psi(x_0), \]
  %
  where this approximation gets better and better for larger and larger $\lambda$.
\end{remark}

If the phase $\phi$ has a critical point of order greater than two, than the asymptotics of the oscillatory integral get worse. In particular, if $\phi$ has a zero of order $k$, then around this region $\phi$ differs by $1/\lambda$ on an interval of length $1/\lambda^{1/k}$, so we might $I(\lambda)$ to be proportional to $\lambda^{1/k}$. This is precisely what happens, but our proof will not rely on the Fourier transform since the computation of the Fourier transform of $e^{\lambda ix^k}$ is quite difficult to calculate when $k > 2$. The next proof also works for the case $k = 2$, but the proof is different.

\begin{lemma}
  For any non-negative integers $l$ and $k$, there is a positive constant $A_{kl} > 0$ such that for any $\lambda \in \RR$ and $\varepsilon > 0$,
  %
  \[ \int_0^\infty e^{\lambda i x^k} e^{-\varepsilon x^k} x^l\; dx = A_{kl} (\varepsilon - i \lambda)^{-(l+1)/k}, \]
  %
  where the $k$th root is the principal root for non-negative complex numbers.
\end{lemma}
\begin{proof}
  If $z = (\varepsilon - i \lambda)^{1/k} x$, and if $\alpha_N$ is the ray between the origin and the point $N (\varepsilon - i \lambda)^{1/k}$, then
  %
  \[ \int_0^N e^{\lambda i x^k} e^{- \varepsilon x^k} x^l\; dx = (\varepsilon - i \lambda)^{-(l+1)/k} \int_{\alpha_N} e^{-z^k} z^l\; dz. \]
  %
  Let $\theta \in (-\pi/2,0]$ be the argument of $(\varepsilon - i \lambda)^{1/k}$, and set $\beta_N$ to be the arc between $N ( \varepsilon - i \lambda)^{1/k}$ and $N (\varepsilon^2 + \lambda^2)^{1/2}$. Then $\beta_N$ has length $O(N)$, with implicit constant depending on $\lambda$ and $\varepsilon$. Moreover, any point $z$ on $\beta_N$ has modulus $N (\varepsilon^2 + \lambda^2)^{1/2}$ and argument less than or equal to $\theta / k$. But this implies that $\text{Re}(z^k) \geq N^k (\varepsilon^2 + \lambda^2)^{k/2} \cos(\theta)$, and so there exists a constant $c$ depending on $\varepsilon$ and $\lambda$ such that $|e^{-z^k}| \leq e^{c N^k}$. But this means that $|z^l e^{-z^k}| \leq N^l e^{-cN^k}$. Thus taking in absolute values gives that
  %
  \[ \lim_{N \to \infty} \int_{\beta_N} e^{-z^k} z^l\; dz = 0. \]
  %
  In particular, applying Cauchy's theorem, we conclude that
  %
  \[ \lim_{N \to \infty} \int_{\gamma_N} e^{-z^k} z^l\; dz = \int_0^\infty e^{-x^k} x^l\; dx. \]
  %
  If we denote the latter integral by $A_{kl} > 0$, then we have shown that
  %
  \[ \int_0^\infty e^{\lambda i x^k} e^{-\varepsilon x^k} x^l\; dx = A_{kl} \cdot (\varepsilon - i \lambda)^{-(l+1)/k}, \]
  %
  as was required to be shown.
\end{proof}

\begin{remark}
  In particular, this implies that for each $\varepsilon$, there exists constants $A_{kln}$ such that
  %
  \[ \int_0^\infty e^{\lambda i x^k} e^{-x^k} x^l\; dx = \lambda^{-(l+1)/k} \sum_{n = 0}^\infty A_{kln} \lambda^{-n}. \]
  %
  This is obtained by taking the Laurent series of
  %
  \[ (1 - i\lambda)^{-(l+1)/k} = \lambda^{-(l+1)/k} (\lambda^{-1} - i)^{-(l+1)/k}, \]
  %
  which converges absolutely for $\lambda > 1$. In particular, for each $N$ and for each $\lambda$, we conclude
  %
  \[ \int_0^\infty e^{\lambda i x^k} e^{-x^k} x^l\; dx = \lambda^{-(l+1)/k} \sum_{n = 0}^N A_{kln} \lambda^{-n} + O_N \left(1/\lambda^{n + 1 + 1/k} \right). \]
\end{remark}

\begin{lemma}
  If $\eta$ is compactly supported and smooth, then
  %
  \[ \left| \int_{-\infty}^\infty e^{\lambda i x^k} x^l \eta(x)\; dx \right| \lesssim_{l,k,\eta} \lambda^{-(l + 1)/k}. \]
\end{lemma}
\begin{proof}
  Let $\alpha$ be a bump function supported on $[-2,2]$ with $\alpha(x) = 1$ for $|x| \leq 1$. For each $\varepsilon > 0$, write
  %
  \begin{align*}
    \int_{-\infty}^\infty e^{\lambda i x^k} x^l \eta(x)\; dx &= \int_{-\infty}^\infty e^{\lambda i x^k} x^l \eta(x) \alpha(x/\varepsilon)\; dx\\
    &\ \ \ \ + \int_{-\infty}^\infty e^{\lambda i x^k} x^l \eta(x) (1 - \alpha(x/\varepsilon))\; dx,
  \end{align*}
  %
  where we will bound each term and optimize for a small $\varepsilon$. We trivially have
  %
  \[ \left| \int_{-\infty}^\infty e^{\lambda i x^k} x^l \eta(x) \alpha(x/\varepsilon)\; dx \right| \lesssim_\eta \varepsilon^{l+1}, \]
  %
  We apply an integration by parts to the second integral, noting that $e^{\lambda i x^k}$ is a fixed point of the differential operator
  %
  \[ Df = \frac{1}{\lambda i k x^{k-1}} \frac{df}{dx}. \]
  %
  If we consider the differential operator
  %
  \[ D^*g = \frac{d}{dx} \left( \frac{-f}{\lambda i k x^{k-1}} \right) = \left( \frac{i}{\lambda k} \right) \left( \frac{f'(x)}{x^{k-1}} - \frac{(k-1) f(x)}{x^k} \right), \]
  %
  then for any smooth $f$ and compactly supported $g$,
  %
  \[ \int_{-\infty}^\infty (Df)(x) g(x) = \int_{-\infty}^\infty f(x) (D^* g)(x). \]
  %
  In particular,
  %
  \begin{align*}
    \int_{-\infty}^\infty e^{\lambda i x^k} x^l \eta(x) (1 - \alpha(x/\varepsilon))\; dx &= \int_{-\infty}^\infty D^N(e^{\lambda i x^k})\; x^l \eta(x) (1 - \alpha(x/\varepsilon))\; dx\\
    &= \int_{-\infty}^\infty e^{\lambda i x^k}\; (D^*)^N \{ x^l \eta(x) (1 - \alpha(x/\varepsilon)) \}\; dx.
  \end{align*}
  %
  Write $g_N(x) = (D^*)^N \{ x^l \eta(x) (1 - \alpha(x/\varepsilon)) \}$. Since $x^l \eta(x) (1 - \alpha(x/\varepsilon))$ vanishes for $|x| \leq \varepsilon$, so too does $g_N(x)$. For $N \geq l/(k-1)$, and $|x| \geq \varepsilon$, we have
  %
  \[ |g_N(x)| \lesssim_{N,\eta} \lambda^{-N} \varepsilon^{-N} |x|^{l - N(k-1)}, \]
  %
  where the implicit constant depends on upper bounds for the derivatives of $\eta$ of order to $N$. We can thus take in absolute values after integrating by parts to conclude that if $N > (l+1)/(k-1)$, then
  %
  \[ \left| \int_{-\infty}^\infty e^{\lambda i x^k} x^l \eta(x) (1 - \alpha(x/\varepsilon))\; dx \right| \lesssim_{N,\eta} \lambda^{-N} \varepsilon^{l + 1 - Nk} \]
  %
  Thus we can put the two bounds together to conclude that
  %
  \[ \left| \int_{-\infty}^\infty e^{\lambda i x^k} x^l \eta(x)\; dx \right| \lesssim_{N,\eta} \varepsilon^{l+1} + \lambda^{-N} \varepsilon^{l+1-Nk}. \]
  %
  Picking $\varepsilon = \lambda^{-1/k}$ gives
  %
  \[ \left| \int_{-\infty}^\infty e^{\lambda i x^k} x^l \eta(x)\; dx \right| \lesssim_{N,\psi} \lambda^{-(l+1)/k}. \qedhere \]
  %
  But $N$ was chosen depending only on $k$ and $l$, so the implicit constants depend on the correct variables.
%  \[ D^* \{ x^l \eta(x) (1 - \alpha(x/\varepsilon)) \} = c x^{l-k} \eta(x) (1 - \alpha(x/\varepsilon)) + x^{l+1-k} \eta'(x) (1 - \alpha(x/\varepsilon) - \varepsilon^{-1} x^{l+1-k} \eta(x) \alpha'(x/\varepsilon) \]
%  \[ (D^*)^2 \{ x^l \eta(x) (1 - \alpha(x/\varepsilon)) \} = x^{l-2k} (1 - \alpha(x/\varepsilon))
  %  (d/dx) \{ x^{l+1-2k} \eta(x) (1 - \alpha(x/\varepsilon)) + x^{l+2-2k} \eta'(x) (1 - \alpha(x/\varepsilon) - \varepsilon^{-1} x^{l+2-2k} \eta(x) \alpha'(x/\varepsilon) \} \]
\end{proof}

\begin{remark}
  The implicit constants can be bounded uniformly given uniform upper bounds on the magnitude of the derivatives of $\eta$ of order up to
  %
  \[ \lceil (l+1)/(k-1) \rceil, \]
  %
  and upper bounds on the measure of the support of $\eta$.
\end{remark}

We can now prove the asymptotics for the model case $\phi(x) = x^k$.

\begin{theorem}
  Suppose $\psi$ is a smooth compactly supported amplitude, and $\phi$ is a smooth phase with $\phi'(x) \neq 0$ on the support of $\psi$ except at some point $x_0$, where $\phi'(x_0) = \dots = \phi^{(k-1)}(x_0) = 0$, and $\phi^{(k)}(x_0) \neq 0$. Then there is a sequence $\{ a_n \}$ such that for each $N$,
  %
  \[ I(\lambda) = \lambda^{-1/k} \sum_{n = 0}^N a_n \lambda^{-n/k} + O_{\psi,k,N} \left( 1/\lambda^{(N+2)/k} \right). \]
\end{theorem}
\begin{proof}
  Let us begin with the model case $\phi(x) = x^k$. Let $\tilde{\psi}$ be a bump function with $\tilde{\psi}(x) = 1$ for all $x$ with $\psi(x) > 0$. Then
  %
  \[ I(\lambda) = \int_{-\infty}^\infty e^{\lambda i x^k} e^{-x^k} [e^{x^k} \psi(x)] \tilde{\psi}(x)\; dx. \]
  %
  For each $N$, perform a Taylor expansion, writing
  %
  \[ e^{x^k} \psi(x) = \sum_{n = 0}^N a_n x^n + x^{N+1} R_N(x). \]
  %
  Thus if $P_N(x) = \sum_{n = 0}^N a_n x^n$,
  %
  \begin{align*}
    \int_{-\infty}^\infty &e^{\lambda i x^k} e^{-x^k} [e^{x^k} \psi(x)] \tilde{\psi}(x)\; dx\\
    & = \int_{-\infty}^\infty e^{\lambda i x^k} e^{-x^k} P_N(x)\; dx\\
    & + \int_{-\infty}^\infty e^{\lambda i x^k} e^{-x^k} P_N(x) (\tilde{\psi}(x) - 1)\; dx\\
    & + \int_{-\infty}^\infty e^{\lambda i x^k} e^{-x^k} x^{N+1} R_N(x) \tilde{\psi}(x)\; dx.
  \end{align*}
  %
  The first integral can be expanded in the required power series. The second integral, since it is supported away from the origin, is $O_M(\lambda^{-M})$ for any $M > 0$. And in the last lemma we showed the third integral is $O(\lambda^{-(N+2)/k})$, so combining these three terms gives the required result. The general case follows from a change of variables.
\end{proof}

\begin{remark}
  As we saw in the case $k = 2$, if $k$ is even and $n$ is odd then
  %
  \[ \int_{-\infty}^\infty e^{\lambda i x^k} e^{-x^k} x^n = 0. \]
  %
  Thus we can actually improve the asymptotics to the existence of a sequence $\{ a_n \}$ such that
  %
  \[ I(\lambda) = \lambda^{-1/k} \sum_{n = 0}^N a_n \lambda^{-2n/k} + O_{\phi,\psi,N} \left( 1 / \lambda^{(2N + 3)/k} \right). \]
\end{remark}

\begin{comment}

Changing variables then proves the result for general $k$'th order phases.

\begin{lemma}
  Let $\psi$ is a compactly supported and smooth amplitude, and $\phi'(x) \neq 0$ on the support of $\psi$, except at some point $x_0$ where $\phi'(x_0) = \dots = \phi^{(k-1)}(x_0) = 0$, with $\phi^{(k)}(x_0) \neq 0$, then there exists constants $\{ a_n \}$ depending on the derivatives of $\phi$ and $\psi$ at $x_0$, such that for each $N$,
  %
  \[ \int_{-\infty}^\infty e^{\lambda i \phi(x)} \psi(x)\; dx = \lambda^{-1/k} \sum_{n = 0}^N a_n \lambda^{-n/k} + O_{\phi,\psi,N} \left( 1/\lambda^{(N+2)/k} \right). \]
\end{lemma}

\begin{lemma}
  Let $\psi$ is a compactly supported and smooth amplitude, and $\phi'(x) \neq 0$ on the support of $\psi$, except at some point $x_0$ where $\phi'(x_0) = \dots = \phi^{(k-1)}(x_0) = 0$, with $\phi^{(k)}(x_0) \neq 0$, then there exists constants $\{ a_n \}$ depending on the derivatives of $\phi$ and $\psi$ at $x_0$, such that for each $N$,
  %
  \[ \int_{-\infty}^\infty e^{\lambda i \phi(x)} \psi(x)\; dx = \lambda^{-1/k} \sum_{n = 0}^N a_n \lambda^{-n/k} + O_{\phi,\psi,N} \left( 1/\lambda^{(N+2)/k} \right). \]
\end{lemma}
\begin{proof}
  Without loss of generality, we may rescale our integral so that $\phi^{(k)}(x_0) = k!$. Then we can write $\phi(x) = x^k + x^{k+1} R(x)$, where $R(x)$ is a smooth function. A Taylor series approach tells us that
  %
  \[ e^{\lambda i \phi(x)} = e^{\lambda i x^k} \left[ \sum_{n = 0}^M \frac{(\lambda i x^{k+1} R(x))^n}{n!} + Q(x) \right], \]
  %
  where
  %
  \[ Q(x) = \frac{(i\lambda)^{M+1}}{(M+1)!} x^{(k+1)(M+1)} R(x)^{M+1} \int_0^1 e^{\lambda i x^{k+1} R(x) s} (1 - s)^M\; ds \]
  %
  For each $n$, we let
  %
  \[ I_n(\lambda) = \frac{(\lambda i)^n}{n!} \int e^{\lambda i x^k} R(x)^n x^{n(k+1)} \psi(x)\; dx, \]
  %
  and let
  %
  \[ J(\lambda) = \int_{-\infty}^\infty \int_0^1 e^{\lambda i x^k} x^{(k+1)(M+1)} R(x)^{M+1} e^{\lambda i x^{k+1} R(x) s} (1 - s)^M \psi(x)\; ds\; dx. \]
  %
  Then
  %
  \[ I(\lambda) = I_1(\lambda) + \dots + I_M(\lambda) + \frac{(i\lambda)^{M+1}}{(M+1)!} J(\lambda), \]
  %
  and so it suffices to obtain asymptotics on each term separately. From the previous argument, we know there are are constants $\{ a_{nm} \}$ such that for each $M$,
  %
  \[ I_n(\lambda) = \lambda^{-1/k} \sum_{m = 1}^M a_{nm} \lambda^{n-m/k} + O_{\psi,R,k,n} \left( 1/ \lambda^{(M+2)/k} \right). \]
  %
  Moreover, we can write
  %
  \[ I_n(\lambda) = \int e^{\lambda i x^k} x^{n(k+1)} \psi_n(x)\; dx, \]
  %
  where $\psi_n(x) = R(x)^n \psi(x)$ is smooth. But then we know
  %
  \[ |I_n(\lambda)| \lesssim_{n,k,\psi,R} 1/\lambda^{n + (n+1)/k}. \]
  %
  In particular, this means that $a_{nm} = 0$ for $m \leq (2k+1)n$. Thus we conclude $I_n(\lambda)$ can be expanded in positive powers of $\lambda^{1/k}$, up to an error $O(1/\lambda^{(M+2)/k})$. If we split the integrand for $J(\lambda)$ using a bump function into values $x$ with $|x| \leq \varepsilon$ and values $|x| \geq \varepsilon$, bound the former by bringing in absolute values, bound the latter using integration by parts, and then optimizing over $\varepsilon$ yields

  To complete the argument, we show that for sufficiently large $M$, we can treat $J$ as an error term. If we define
  %
  \[ \psi_M(x,s) = \psi(x) (1 - s)^M R(x)^{M+1}, \]
  %
  then
  %
  \[ J(\lambda) = \int_{-\infty}^\infty \int_0^1 e^{\lambda i x^k} x^{(k+1)(M+1)} e^{\lambda i x^{k+1} R(x) s} \psi_M(x,s)\; dx\; ds. \]
  %
  We introduce a bump function $\alpha$ such that $\alpha(x) = 1$ for $|x| \leq 1$, and vanishes outside for $|x| \geq 2$. For $\varepsilon > 0$, we write
  %
  \begin{align*}
    J(\lambda) &= \int_{-\infty}^\infty \int_0^1 e^{\lambda i x^k} x^{(k+1)(M+1)} e^{\lambda i x^{k+1} R(x) s} \psi_M(x,s) \alpha(x/\varepsilon)\; dx\\
    &\ \ \ + \int_{-\infty}^\infty \int_0^1 e^{\lambda i x^k} x^{(k+1)(M+1)} e^{\lambda i x^{k+1} R(x) s} \psi_M(x,s) (1 - \alpha(x/\varepsilon))\; dx.
  \end{align*}
  %
  The first integral is $O_{\psi,R,M}(\varepsilon^{(k+1)(M+1)+1})$. For the second, we employ an integration by parts in $x$. The value $e^{\lambda i x^k}$ is a fixed point of the differential operator $Df = f' / \lambda i x^{k-1}$, whose adjoint is
  %
  \[ D^* g = \frac{d}{dx} \left( \frac{g}{\lambda i x^{k-1}} \right). \]
  %
  For each $L$, there exists constants $c_n$ such that
  %
  \[ (D^*)^L g = \frac{1}{\lambda^L} \sum_{n = 0}^L \frac{c_n g^{(n)}}{x^{Lk - n}}. \]
  %
  If we write
  %
  \[ g_n(\lambda,x,s) = \frac{1}{x^{Lk-n}} \frac{d^n}{dx^n} \left\{ x^{(k+1)(M+1)} e^{\lambda i x^{k+1} R(x) s} \psi(x,s) (1 - \alpha(x/\varepsilon)) \right\}, \]
  %
  then our oscillatory integral is bounded by an implicit constant depending on $L$ and $k$, and
  %
  \[ \sum_{n = 0}^L \frac{1}{\lambda^L} \int_{-\infty}^\infty \int_0^1 e^{\lambda i x^k} g_n(x)\; dx. \]
  %
  Now $g_n$ has compact support depending on $\psi$, and vanishes for $|x| \leq \varepsilon$. For $|x| \geq \varepsilon$, we find that
  % \lambda \lesssim_{k,\phi} 1/\varepsilon
  \[ |g_n(\lambda,x,s)| \lesssim_{L,M,k,\alpha,\psi,n} \sum_{m = 0}^n \lambda^m \varepsilon^{m-n} x^{n-Lk+(k+1)(M+1) + km}. \]
  %
  Thus we conclude that for sufficiently large $L$
  %
  \[ |J(\lambda)| \lesssim_{\psi,R,M,L,k,\alpha} \varepsilon^{(k+1)(M+1) + 1} \left( 1 + \varepsilon^{- Lk} \sum_{m = 0}^L \lambda^{m-L} \varepsilon^{(k+1)m} \right). \]
  %
  TODO.
\end{proof}

\end{comment}

Let us now consider some examples of the method of stationary phase in one dimension.

\begin{example}
  The Bessel function of order $m$, denoted $J_m(r)$, is defined to be the oscillatory integral
  %
  \[ J_m(r) = \frac{1}{2\pi} \int_0^{2\pi} e^{i r \sin(\theta)} e^{-i m \theta} d\theta. \]
  %
  We want to use the method of stationary phase to determine the decay of $J_m(r)$ as $r \to \infty$. The amplitude is $\psi(\theta) = (1/2\pi) e^{-im \theta}$, and the phase is $\phi(\theta) = \sin(\theta)$. We note that the phase $\phi(\theta) = \sin(\theta)$ is stationary when $\theta = \pi/2$ and $\theta = 3\pi/2$, and that these stationary points are nondegenerate. Thus we might expect $|J_m(r)| = O_m(r^{-1/2})$. More precisely, we write $1 = \psi_1 + \psi_2 + \psi_3$, where $\psi_1$ is supported in a small neighbourhood of $\pi/2$, $\psi_2$ in a neighbourhood of $3\pi/2$, and $\psi_3$ is supported away from $\pi/2$ and $3\pi/2$. This oscillatory integral is defined over an integral, but the integrand is periodic, so an integration by parts verifies that since $\psi_3$ is supported away from stationary points, for any $N > 0$,
  %
  \[ \frac{1}{2\pi} \int_0^{2\pi} e^{i r \sin(\theta)} e^{-i m \theta} \psi_3(\theta)\; d\theta = O_N(r^{-N}). \]
  %
  Next, we verify using our formula for the stationary phase that
  %
  \begin{align*}
    \frac{1}{2\pi} \int & e^{i r \sin(\theta)} e^{-i m \theta} \psi_1(\theta)\; d\theta\\
    &= \left( \frac{2\pi}{-i \phi''(\pi/2)} \right)^{1/2} \cdot (1/2\pi) e^{i r \sin(\pi/2)} e^{-i m (\pi/2)} \cdot r^{-1/2} + O_m(r^{-3/2})\\
    &= (2\pi)^{-1/2} e^{i(r - m\pi/2 - \pi/4)} r^{-1/2} + O_m(r^{-3/2}).
  \end{align*}
  %
  Similarily,
  %
  \begin{align*}
    \frac{1}{2\pi} \int & e^{i r \sin(\theta)} e^{-i m \theta} \psi_2(\theta)\; d\theta\\
    &= \left( \frac{2\pi}{-i \phi''(3\pi/2)} \right)^{1/2} \cdot (1/2\pi) e^{i r \sin(3\pi/2)} e^{-i m (3\pi/2)} \cdot r^{-1/2} + O_m(r^{-3/2})\\
    &= (2\pi)^{-1/2} e^{-i(r - m\pi/2 - \pi/4)} r^{-1/2} + O_m(r^{-3/2}).
  \end{align*}
  %
  Summing up the three estimates, we conclude
  %
  \[ J_m(r) = (2/\pi r)^{1/2} \cos(r - m\pi/2 - \pi/4) + O_m(r^{-3/2}). \]
\end{example}

\begin{example}
  Consider the Airy function
  %
  \[ \text{Ai}(x) = \int_{-\infty}^\infty e^{i(x \xi + \xi^3/3)}\; d\xi, \]
  %
  which arises as a solution to the differential equation $y'' = xy$. Again, this integral is not defined absolutely. Nonetheless, for a large $N$, an integration by parts shows that for any finite interval $I$ containing only points $x$ with $|x| \geq N$,
  %
  \[ \int_I e^{i(x \xi + \xi^3/3)}\; d\xi = O(1/N), \]
  %
  where the implicit constant is independant of $I$. Thus we can interpret the integral as
  %
  \[ \lim_{n \to \infty} \int_{a_n}^{b_n} e^{i(x \xi + \xi^3/3)}\; d\xi, \]
  %
  where $\{ a_n \}$ and $\{ b_n \}$ are any sequences with $a_n \to -\infty$, $b_n \to \infty$.

  Now consider the phase $\phi(\xi) = x \xi + \xi^3/3$. Then $\phi'(\xi) = x + \xi^2$. When $x$ is negative, there are two stationary points. Thus we can rescale the integral, writing $\nu = x^{-1/2} \xi$, so that
  %
  \[ \text{Ai}(-x) = x^{1/2} \int_{-\infty}^\infty e^{i x^{3/2}(\nu^3/3 - \nu)}\; d\nu. \]
  %
  If we write $\phi_0(\nu) = \nu^3/3 - \nu$, then $\phi_0$ has two stationary points, at $\nu = \pm 1$. These stationary points are non-degenerate, so if we write $1 = \psi_1 + \psi_2 + \psi_3 + \psi_4$, where $\psi_1$ equal to one in a neighbourhood of $1$, $\psi_2$ equal to one in a neighbourhood of $-1$, and $\psi_3$ is supported in the region between $-1$ and $1$, and $\psi_4$ vanishes in all such regions, then we decompose $\text{Ai}(-x)$ as $I_1 + I_2 + I_3 + I_4$. Now the principle of stationary phase tells us that
  %
  \[ I_1 = \pi^{1/2} x^{-1/4} e^{i \pi/4} e^{-2i x^{3/2}/3} + O(x^{-1}) \]
  %
  and
  %
  \[ I_2 = \pi^{1/2} x^{-1/4} e^{-i\pi/4} e^{2i x^{3/2}/3} + O(x^{-1}). \]
  %
  Moreover, $I_3 = O_N(x^{-N})$ for all $N \geq 0$. It remains to show $I_4 = O(x^{-1})$. Indeed, an integration by parts shows that
  %
  \begin{align*}
    I_4 &= x^{1/2} \int_{-\infty}^\infty e^{i x^{3/2} \phi_0(\nu)} \psi_4(\nu)\; d\nu\\
    &= \frac{i}{x} \int_{-\infty}^\infty e^{i x^{3/2} \phi_0(\nu)} \frac{d}{d\nu} \left( \frac{\psi_4(\nu)}{\nu^2 - 1} \right)\; d\nu.
  \end{align*}
  %
  Taking in absolute values shows $|I_4| \lesssim 1/x$. Thus as $x \to \infty$,
  %
  \[ \text{Ai}(-x) = 2 \pi^{1/2} x^{-1/4} \cos((2/3) x^{3/2} - \pi/4) + O(1/x), \]
  %
  which gives the first order asymptotics of the integral.

  On the other hand, let us consider large positive $x$. Then the phase $\phi$ has no critical points, and we therefore expect very fast decay. To achieve this decay, we employ a contour shift, replacing the oscillatory integral with a different oscillatory integral which \emph{has} a stationary point, so we can obtain asymptotics here. If we write $\phi(z) = xz + z^3/3$, then $\phi'(z) = 0$ when $z = \pm i x^{1/2}$. A simple contour shift argument to the line $\RR + i x^{1/2}$ gives
  %
  \begin{align*}
    \text{Ai}(x) &= \int_{-\infty}^\infty e^{i \phi(\xi + ix^{1/2})}\; d\xi\\
    &= e^{-(2/3)x^{3/2}} \int_{-\infty}^\infty e^{-\xi^2 x^{1/2}} e^{i \xi^3/3}\; d\xi.
  \end{align*}
  %
  We have
  %
  \[ \int_{-\infty}^\infty e^{-\xi^2 x^{1/2}} e^{i \xi^3/3}\; d\xi \approx x^{-1/4} \int_{-\infty}^\infty e^{-\xi^2} e^{i x^{-3/4} \xi^3/3}\; d\xi. \]
  %
  Now a Taylor series shows
  %
  \[ e^{i x^{-3/4} \xi^3/3} = 1 + O(x^{-3/4} \xi^3/3), \]
  %
  so, plugging in, we conclude
  %
  \[ \text{Ai}(x) = \pi^{1/2} x^{-1/4} e^{-(2/3) x^{3/2}} + O(x^{-3/4} e^{-(2/3) x^{3/2}}). \]
  %
  Thus Airy's function decreases exponentially as $x \to \infty$.
\end{example}

\begin{example}
  Let us consider the integral quantities
  %
  \[ \int_0^1 e^{i x \xi} e^{i/x} x^{-\gamma}\; dx \]
  %
  where to avoid technicalities we assume $0 \leq \gamma < 2$. These integral quantities are not defined absolutely, so we actually interpret this integral as
  %
  \[ \lim_{\varepsilon \to 0} \int_0^1 e^{i x \xi} e^{i/x} x^{-\gamma}\; dx \]
  %
  If we write $\phi(x) = x \xi + 1/x$, then
  %
  \[ \int_0^1 e^{i x \xi} e^{i/x} x^{-\gamma}\; dx = \int_0^1 e^{i\phi(x)} x^{-\gamma}\; dx. \]
  %
  For $0 < \varepsilon_1 < \varepsilon_2 < \varepsilon$, since $\phi'(x) = \xi - 1/x^2$, an easy integration by parts shows that for $\varepsilon \leq \xi^{-1/2}/2$,
  %
  \begin{equation} \label{riemannsingularityibp}
  \begin{aligned}
    \int_{\varepsilon_1}^{\varepsilon_2} e^{i\phi(x)} x^{-\gamma}\; dx &= \frac{1}{i \xi} \int_{\varepsilon_1}^{\varepsilon_2} \frac{d}{dx} \left( e^{i \phi(x)} \right) \frac{x^{2-\gamma}}{x^2 - 1/\xi}\; dx\\
    &= \frac{-1}{i \xi} \int_{\varepsilon_1}^{\varepsilon_2} e^{i \phi(x)} \frac{d}{dx} \left( \frac{x^{2-\gamma}}{x^2 - 1/\xi} \right) + O(\varepsilon^{2-\gamma})\\
    &= O(\varepsilon^{2-\gamma}),
  \end{aligned}
  \end{equation}
  %
  where the constant is independent of $\xi$. This implies the limit we study certainly exist. We wish to prove an asymptotic formula for this integral as $\xi \to \infty$. If we write $\phi(x) = x \xi + 1/x$, then
  %
  \[ \int_0^1 e^{i x \xi} e^{i/x} x^{-\gamma}\; dx = \int_0^1 e^{i \phi(x)} x^{-\gamma}. \]
  %
  Since $\phi$ has a nondegenerate stationary point when $x = \xi^{-1/2}$, our heuristics might suggest that if the phase and amplitude were smooth at the origin, then as $\gamma \to \infty$,
  %
  \begin{align*}
    \int_0^1 e^{i\phi(x)} x^{-\gamma} &\approx \left( \frac{2\pi}{-i \phi''(\xi^{-1/2})} \right)^{1/2} e^{i\phi(\xi^{-1/2})} \xi^{\gamma/2}\\
    &= \pi^{1/2} e^{i(2 \xi^{1/2} + \pi/4)} \xi^{\gamma/2 - 3/4}.
  \end{align*}
  %
  We shall show that these heuristics continue to hold, up to an error of $O(\xi^{\gamma/2 - 1})$.

  In an attempt to isolate the critical point, we split the interval $[0,1]$ into three parts, $[0,0.5 \xi^{-1/2}]$, $[0.5 \xi^{-1/2},1.5 \xi^{-1/2}]$, and $[1.5 \xi^{-1/2},1]$, obtaining three integrals $I_1$, $I_2$, and $I_3$. The calculation \eqref{riemannsingularityibp} shows that $|I_1| \lesssim \xi^{\gamma/2 - 1}$, and thus is neglible to our asymptotic formula. To obtain a bound on $I_3$, we use the Van der Corput lemma, noting that $\phi'(x) = \xi - 1/x^2$ is monotone, and $|\phi'(x)| \gtrsim \xi$ for $x \geq 1.5 \xi^{-1/2}$. Thus we find $|I_1| \lesssim \xi^{-1}$, and thus is also neglible to our formula. Thus we are left with the trick part of calculating $I_2$ accurately. It will easiest to do this by renormalizing the integral, i.e. writing $y = \xi^{1/2} x$, and calculating
  %
  \[ I_2 = \int_{0.5 \xi^{-1/2}}^{1.5 \xi^{-1/2}} e^{i \phi(x)} x^{-\gamma}\; dx = \xi^{\gamma/2 - 1/2} \int_{0.5}^{1.5} e^{i \xi^{1/2}(y + 1/y)} y^{-\gamma}\; dy. \]
  %
  We consider a smooth amplitude function $\psi(x)$ supported on the interior of $[0.5,1.5]$. Then since $y + 1/y$ is stationary at $y = 1$, but non-degenerate, we can write
  %
  \[ \int e^{i \xi^{1/2}(y + 1/y)} y^{-\gamma} \psi(y)\; dy = \xi^{-1/4} \pi^{1/2} e^{i(2\xi^{1/2} + 1/4)} + O(\xi^{-1/2}), \]
  %
  from which we obtain our main term. On the other hand, we can apply the Van der Corput lemma to show that
  %
  \[ \int_{0.5}^{1.5} e^{i \xi^{1/2}(y + 1/y)} y^{-\gamma} (1 - \psi(y))\; dy = \int e^{i \xi^{1/2}(y + 1/y)} y^{-\gamma} \psi(y)\; dy = O(\xi^{-1/2}). \]
  %
  Combining all these estimates gives the theorem.

  On the other hand, consider the integral
  %
  \[ I(\xi) = \int_0^1 e^{-i \xi x} e^{i/x} x^{-\gamma}\; dx = \int_0^1 e^{i \phi(x)} x^{-\gamma}, \]
  %
  where $\phi(x) = 1/x - \xi x$ is the phase. Then the phase has no critical points so we can assume that we can large decay for large $\xi$. We decompose the integral onto the intervals $[0,\xi^{-1/2}]$ and $[\xi^{-1/2},1]$, inducing the two quantities $I_1$ and $I_2$. Now applying the Van der Corput lemma to $I_2$ with $|\phi'(x)| = |1/x^2 + \xi| \geq \xi$ for $x \geq 0$, gives $|I_2| \lesssim \xi^{\gamma/2 - 1}$. On the other hand, renormalizing with $y = \xi^{1/2} x$, we have
  %
  \[ I_1 = \xi^{\gamma/2 - 1/2} \int_0^1 e^{i \xi^{1/2} (1/y - y)} y^{-\gamma}\; dy. \]
  %
  For each $n$, we note that for the phase $\phi_0(x) = 1/y - y$, for $1/2^{n+1} \leq y \leq 1/2^n$, we have $|\phi_0'(x)| \gtrsim 4^n$. Thus we can apply the Van der Corput lemma to conclude
  %
  \[ \left| \int_{1/2^{n+1}}^{1/2^n} e^{i \phi_0(x)} y^{-\gamma}\; dy \right| \lesssim \frac{2^{\gamma n}}{4^n \xi^{1/2}}. \]
  %
  Summing up over all $n \geq 0$, we conclude $|I_1| \lesssim \xi^{\gamma/2 - 1}$. Thus $|I(\xi)| \lesssim \xi^{\gamma/2 - 1}$.

  One way to interpret this asymptotic formula is through a \emph{Riemann singularity}, i.e. a tempered distribution $\Lambda$ supported on the half-life $x \geq 0$, that agrees with the oscillatory function $e^{i/x} x^{-\gamma}$ for small $x$, but is compactly supported and smooth away from the origin. We consider the case $0 \leq \gamma < 2$ for simplicity. Thus for Schwartz $f \in \mathcal{S}(\RR)$, we have
  %
  \[ \Lambda(f) = \lim_{\varepsilon \to 0^+} \int_\varepsilon^\infty f(x) e^{i/x} x^{-\gamma} \psi(x)\; dx, \]
  %
  where $\psi$ is smooth and compactly supported, and equals one in a neighbourhood of the origin. An easy integration by parts shows that for a fixed Schwartz $f$, and for $0 < \varepsilon_1 < \varepsilon_2 < \varepsilon$,
  %
  \[ \left| \int_{\varepsilon_1}^{\varepsilon_2} f(x) e^{i/x} x^{-\gamma}\; dx \right| = O\left(\varepsilon^{2-\gamma} \right), \]
  %
  where the implicit constants depend on upper bounds for $f$ and $f'$ in a neighbourhood of the origin. Thus we find $\Lambda(f)$ is well defined, and moreover, $\Lambda$ is a distribution of order one. Since $\Lambda$ is compactly supported, the Paley-Weiner theorem implies that $\widehat{\Lambda}$ is a distribution represented by a locally integrable function, and
  %
  \[ \widehat{\Lambda}(\xi) = \int_0^\infty e^{i/x} x^{-\gamma} \psi(x) e^{-2 \pi \xi i x}\; dx. \]
  %
  Our asymptotics under some small modifications tell us that if $\xi$ is large, then
  %
  \[ \widehat{\Lambda}(-\xi) = 2^{\gamma/2 - 3/4} \pi^{\gamma/2-1/2} e^{i(2^{3/2} \pi^{1/2} \xi^{1/2} + \pi/4)} \xi^{\gamma/2 - 3/4} + O(\xi^{\gamma/2 - 1}). \]
  %
  On the other hand,
  %
  \[ \widehat{\Lambda}(\xi) = O(\xi^{\gamma/2 - 1}), \]
  %
  so the Fourier transform of $\Lambda$ decays much faster to the right than to the left.
\end{example}

\section{Stationary Phase in Multiple Variables}

When we move from a single variable oscillatory integral to a multivariable oscillatory integrals. Thus we consider the oscillatory integral
%
\[ I(\lambda) = \int_{\RR^d} \psi(x) e^{\lambda i \phi(x)}\; dx. \]
%
for large $\lambda$. The method of stationary phase becomes significantly more complicated in this setting because the stationary points of the phase function are no longer necessarily isolated. In certain basic situations, however, we can obtain simple results.

\begin{theorem}
  Let $\phi$ and $\psi$ be smooth functions on $\RR^d$, with $\psi$ compactly supported. If $\nabla \phi$ is nowhere vanishing on the support of $\psi$, then for each $N$, $|I(\lambda)| \lesssim_N \lambda^{-N}$ for all $N$.
\end{theorem}

\begin{proof}
    Set $a = (\nabla \phi)/|\nabla \phi|^2$. Note that $\nabla e^{\lambda i \phi(x)} = (i\lambda) e^{\lambda i \phi(x)} \nabla \phi(x)$ is an eigenfunction of the differential operator $D$ defined such that
    %
    \[ Df(x) = \frac{a \cdot \nabla f}{i \lambda}. \]
    %
    The adjoint operator of $D$ is the operator $D^*$ defined by setting
    %
    \[ D^*f(x) = \frac{\nabla \cdot (af)}{-i\lambda}, \]
    %
    i.e. for any smooth $f$ and $g$, with one of these functions compactly supported,
    %
    \[ \int Df(x) g(x)\; dx = \int f(x) (D^*g)(x)\; dx. \]
    %
    Thus
    %
    \[ I(\lambda) = \int D^N(e^{i \lambda \phi(x)}) \psi(x)\; dx = \int e^{i \lambda \phi(x)} ((D^*)^N\psi)(x)\; dx. \]
    %
    Taking absolute values in the last integral gives that
    %
    \[ |I(\lambda)| \leq \int |(D^*)^N \psi(x)|\; dx \lesssim_{\phi,\psi,N} \frac{1}{\lambda^N}. \qedhere \]
\end{proof}

\begin{remark}
  The implicit constants for a fixed $N$ can be uniformly bounded given a uniform lower bound on $|\nabla \phi|$, and upper bounds on the derivatives of $\phi$ up to order $N+1$, on $\psi$ up to order $N$, and on the measure of the support of $\psi$.
\end{remark}

A tensorization argument establishes the result for a quadratic phase.

\begin{theorem}
  Let $A$ be an invertible $d \times d$ matrix, fix $x_0 \in \RR^d$, and consider the phase $\phi(x) = A(x - x_0) \cdot (x - x_0)$. Then for any compactly supported smooth amplitude $\psi$, there exists constants $\{ a_n \}$ depending only on the derivatives of $\psi$ at the origin, such that for each $N$,
  %
  \[ I(\lambda) = \lambda^{-d/2} \sum_{n = 0}^N a_n \lambda^{-n} + O_N(1/\lambda^{N+d/2 +1}). \]
  %
  Moreover,
  %
  \[ a_0 = \frac{(2\pi)^{d/2} \psi(x_0)}{(-i \mu_1)^{1/2} \dots (-i \mu_d)}, \]
  %
  where $\mu_1, \dots, \mu_d$ are the eigenvalues of $A$.
\end{theorem}
\begin{proof}
  Suppose first that $\psi$ is a tensor product of $d$ compactly supported functions in $\RR$. The constant $a_0$ is invariant under affine changes of coordinates. Thus we may assume that $A$ is a diagonal matrix. But then the oscillatory integral splits into the product of single variable integrals, to which we can apply our one-dimensional asymptotics. Since the asymptotics here depend only on the support of $\psi$, and upper bounds on the magnitude of $\psi$ on derivatives up to order $2N + 4$. A density argument then shows the argument generalizes to any smooth $\psi$, with implicit constants depending on upper bounds on the measure of the support of $\psi$, and upper bounds on the derivative of $\psi$ of order up to $2N + (d + 4)$.
\end{proof}

Morse's theorem says that if $x_0$ is a non-degenerate critical point of a smooth function $\phi$, then there exists a coordinate system around $x_0$ and $a_1, \dots, a_d \in \{ \pm 1 \}$ such that, in this coordinate system,
%
\[ \phi(x_0 + t) = a_1 t_1^2 + \dots + a_d t_d^2. \]
%
In one dimension, the same is true if $x_0$ has a higher order critical point, but this does not generalize to higher dimensions, which reflects the lack of as nice a theory in this case. But in the case of functions with finitely many non-degenerate critical points, we can obtain nice asymptotics. Applying Morse's theorem gives the following theorem.

\begin{theorem}
  Let $\phi$ and $\psi$ be smooth functions, with $\psi$ compactly supported. Suppose $\phi$ has only finitely many critical points on the support of $\psi$, each of which being nondegenerate. Then there exists constants $\{ a_n \}$ depending only on finitely many derivatives of $\Phi$ and $\psi$ at the points $x_1, \dots, x_n$, such that for each $N$,
  %
  \[ I(\lambda) = \lambda^{-d/2} \sum_{n = 0}^N a_n \lambda^{-n} + O_N(1/\lambda^{N+d/2 +1}). \]
  %
  Moreover, if the critical points of $\psi$ are $x_1, \dots, x_m$, then
  %
  \[ a_0 = \sum_{k = 1}^m \frac{(2\pi)^{d/2} \psi(x_k)}{\prod_{l = 1}^m (-i \mu_l(x_k))^{1/2}}, \]
  %
  where $\mu_1(x_k), \dots, \mu_d(x_k)$ are the eigenvalues of the Hessian of $\phi$ at $x_k$.
\end{theorem}

\section{Surface Carried Measures}

Let us consider oscillatory integrals on a `curved' version of Euclidean space. One most basic example is the Fourier transform of the surface measure of the sphere, i.e.
%
\[ \widehat{\sigma}(\xi) = \int_{S^{d-1}} e^{-2 \pi i \xi x} d\sigma(x). \]
%
Studying the decay of this surface measure is of much interest to many problems in analysis. One can reduce the study of this Fourier transform to the study of Bessel functions, to which we have already developed an asymptotic theory.

\begin{theorem}
  If $\sigma$ is the surface measure on the sphere $S^{d-1}$, then
  %
  \[ \widehat{\sigma}(\xi) = \frac{2\pi \cdot J_{d/2 - 1}(2 \pi |\xi|)}{|\xi|^{d/2 - 1}}. \]
  %
  In particular,
  %
  \[ \widehat{\sigma}(\xi) = \frac{2 \cos(2\pi |\xi| - (d/2 - 1)(\pi/2) - \pi/4)}{|\xi|^{(d-1)/2}} + O_d(1/|\xi|^{(d+1)/2}). \]
\end{theorem}
\begin{proof}
  Since $\sigma$ is rotationally symmetric, so too is $\widehat{\sigma}$. In particular, we can apply Fubini's theorem to conclude that if $V_{d-2}$ is the surface area of the unit sphere in $\RR^{d-2}$, then
  %
  \begin{align*}
    \widehat{\sigma}(\xi) &= \int_{S^{d-1}} e^{-2 \pi |\xi| x_1} d\sigma(x)\\
    &= V_{d-2} \int_{-1}^1 e^{-2 \pi |\xi| t} (1 - t^2)^{d/2-1} dt.
  \end{align*}
  %
  Setting $r = 2 \pi |\xi|$ completes the argument.
\end{proof}

Since the multivariate stationary phase approach is essentially `coordinate independant', we can also generalize the approach to manifolds. If $M$ is a $d$ dimensional Riemmannian manifold, and $\phi$ and $\psi$ are complex-valued functions on the manifold, we can consider the oscillatory integral
%
\[ I(\lambda) = \int_M e^{\lambda i \phi(x)} \psi(x) d\sigma(x), \]
%
where $\sigma$ is the surface measure induced by the metric on $M$. If $\phi$ and $\psi$ are compactly supported, then this integral is well defined in the Lebesgue sense.

\begin{theorem}
  Suppose that $\psi$ is a compactly supported smooth amplitude on a Riemannian manifold $M$, $\phi$ is a smooth phase, and $\nabla \phi$ vanishes on at most finitely many points $x_1, \dots, x_m$ on the support of $\psi$, upon each of which the Hessian $H\phi$ is non-degenerate at each point. Then there exists constants $\{ a_n \}$ such that for each $N$,
  %
  \[ I(\lambda) = \lambda^{-d/2} \sum_{n = 0}^N a_n \lambda^{-n} + O(1/\lambda^{N + d/2 + 1}). \]
  %
  Moreover,
  %
  \[ a_0 = \sum_{k = 1}^m \frac{(2\pi)^{d/2} \psi(x_k)}{\prod_{l = 1}^m (-i \mu_l(x_k))^{1/2}}, \]
  %
  where $\mu_1, \dots, \mu_d$ are the eigenvalues of the Hessian $H\phi$.
\end{theorem}

The theorem is proved by a simple partition of unity approach which reduces to the Euclidean case. It has the following important corollary.

\begin{theorem}
  If a surface $\Sigma$ is a smooth submanifold of $\RR^{d+1}$, and has non-vanishing Gauss curvature, and if $\psi$ is a smooth, compactly supported function on $\Sigma$, then
    %
    \[ |\widehat{\psi \sigma}(\xi)| \lesssim_{\psi,\sigma} \frac{1}{|\xi|^{d/2}}, \]
    %
    where $\sigma$ is the surface measure of $\Sigma$.
\end{theorem}
\begin{proof}
  For each $\xi \in S^d$,
  %
  \[ I_\xi(\lambda) = \int_M e^{\lambda i \phi_\xi(x)} \psi(x)\; d\sigma, \]
  %
  where $\phi_\xi(x) = -2 \pi i \xi \cdot x$. The derivatives of $\phi_\xi$ of order $\leq N$ on $M$ are $O_N(1)$, independently of $\xi$. Similarily, $H_M \phi_\xi$ is uniformly non-degenerate, in the sense that the operator norm of $(H_M \phi_\xi)^{-1}(x)$ is $O(1)$, independently of $\xi$. Working with $\Sigma$ as a local graph, and then applying the curvature condition on $\Sigma$ implies that for each $\xi \in S^d$, $\phi_\xi$ has $O(1)$ stationary points on the support of $\psi$. There also exists a constant $r$ such that if $x$ does not lie in any ball of radius $r$ around a stationary point, then $|\nabla_M \phi_\xi| \gtrsim 1$. Moreover, the Hessian $H_M \phi_\xi$ is uniformly non-degenerate in the radius $r$ balls around the critical point, independently of $\xi$. Thus we can apply the last result to conclude
  %
  \[ I_\xi(\lambda) \lesssim \lambda^{-d/2}, \]
  %
  where the implicit constant is independent of $\xi$, because all the required derivatives are uniformly bounded.
\begin{comment}

  Working locally, since the support of $\mu$ is precompact, we may consider a finite partition of unity $\{ \psi_\alpha \}$ with respect to an open family of sets $\{ U_\alpha \}$ covering $U$, such that for each $\alpha$, there exists a transformation $T$ composed of a rotation, translation, and dilation such that if $B$ is the open unit ball, then there is a smooth function $u: B \to \RR$ with $\nabla u(0) = 0$, such that
  %
  \[ T(U_\alpha) = \left\{ (x,u(x)): x \in B \right\}. \]
  %
  The fact that $\Sigma$ has non-vanishing Gauss curvature implies that the Hessian of $u$ is non-zero at each point, so in particular we may assume that $\nabla u$ does not equal to zero anywhere else on $B$. The asymptotics of the Fourier transform of $\mu \psi_\alpha$ is the same as the Fourier transform of the measure $T_*(\mu \psi_\alpha)$, so we may assume without loss of generality that $U_\alpha$ takes the form $T(U_\alpha)$. Then
  %
  \[ d\sigma = (1 + |\nabla u|^2)^{1/2}\; dx^1 \dots dx^d, \]
  %
  which can be incorporated into $\psi_\alpha$. Thus it suffices to show that for each $\xi = (\xi_0, \xi_1) \in \RR^d$,
  %
  \[ \left| \int_B e^{-2 \pi i(\xi_0 \cdot x + \xi_1 u(x))} \psi(x)\; dx \right| \lesssim_{\psi,u} \frac{1}{(|\xi_0|^2 + \xi_1^2)^{d/4}}. \]
  %
  Let us fix $\xi \in S^d$, and consider the oscillatory integral
  %
  \[ I_\xi(\lambda) = \int_B e^{\lambda i \phi_\xi(x)} \psi(x)\; dx, \]
  %
  where
  %
  \[ \phi_\xi(x) = -2\pi \left( \xi_0 \cdot x + \xi_1 u(x) \right). \]
  %
  Note that
  %
  \[ \nabla \phi_\xi(x) = -2\pi(\xi_0 + \xi_1 \nabla u(x)) \]
  %
  The inverse function theorem combined with our curvature condition tells us that if we choose our neighbourhoods $U_\alpha$ small enough, the map $x \mapsto \nabla u(x)$ is a smooth diffeomorphisms on $B$. In particular, for each $\xi$, there is at most one $x \in B$ such that $\phi_\xi$ is stationary at $x$. Even without the curvature condition, we can conclude that for each $x$, there is at most one $\xi \in S^d$, up to a negation, such that $\phi_\xi$ is stationary at $x$.

  Because of the curvature condition, one can verify that for each $\xi \in \RR^d$, there are $O(1)$ stationary points for the phase $\phi_\xi$. Moreover, if

  In particular, we conclude that for each $\xi$, there are $O(1)$ stationary points for the phase $\phi(\cdot | \xi)$ on the support of $\psi$.

  The curvature condition implies that, if we have chosen are neighbourhoods small enough, for each $\xi$ there is at most one $x \in B$ such $\phi(\cdot|\xi)$ is stationary at $x$.

  Then $\nabla_x \phi(x|\xi) = -2\pi( \xi_0 + \xi_1 \nabla u(x) )$. In particular, $\phi(\cdot|(0,1))$ has a stationary point at $0$. If we consider $D_x[\nabla_x \phi(x|\xi)] = -2\pi  \xi_1 H_x u(x)$, which is nonsingular by assumption in a neighbourhood of $(0,1)$. Thus there exists a relatively open set $V_\alpha \subset S^d$


  Applying the implicit function theorem, having chosen our sets $\psi_\alpha$ small enough, there exists a relatively open set $V_\alpha \subset S^d$ containing $(0,1)$ such that for each $\xi = (\xi_0,\xi_1) \in V_\alpha$, there is a unique $x(\xi) \in B$ such that $\nabla_x \phi(x(\xi)|\xi) = 0$

  such that for each $\xi \in \RR^d$ and $\alpha$, there is at most one point $x \in U_\alpha$ such that $\xi$ is orthogonal to $T_x(M)$. Moreover, we can find a relatively open subset $V_\alpha$ of $S^{d-1}$ such that for each $\xi \in V_\alpha$, there exists a unique $x \in U_\alpha$ such that $\xi$ is orthogonal to $T_x(M)$.

  we can write
  %
  \[ \widehat{\mu}(\xi) = \sum_\alpha \int e^{-2 \pi i \xi \cdot x} \psi_\alpha(x)\; d\mu. \]
  %


  $\Sigma$ is the smooth graph of a function $f: B \to \RR$, where $B$ is the closed unit ball in $\RR^d$, i.e.
  %
  \[ \Sigma = \{ (x,t) : x \in B, t = f(x) \}. \]
  %
  Then $d\sigma = (1 + |\nabla f|^2)^{1/2} dx^1 \dots dx^d$, and the fact that $\Sigma$ has non-vanishing Gauss curvature implies that the Hessian of $f$ is non-degenerate. It thus suffices to obtain a bound of the form
  %
  \[ \int e^{i (\xi \cdot x) + \eta f(x)} \psi(x)\; dx \lesssim_{\psi} \frac{1}{\sqrt{|\xi|^2 + \eta^2}}. \]
  %
  where $\psi$ is a compactly supported, smooth function on $B$. Fix $(\xi_0, \eta_0)$ such that $\xi_0^2 + \eta_0^2 = 1$, and consider the oscillatory integral
  %
  \[ I(\lambda) = \int_B e^{\lambda i (\xi_0 \cdot x) + \eta_0 f(x)} \psi(x)\; dx \]
  %
  This is an oscillatory integral with phase $\phi(x;\xi_0,\eta_0) = (\xi_0 \cdot x) + \eta_0 f(x)$.

  We are considering the oscillatory integral
  %
  \[ \int e^{-2 \pi i \xi x} d\Sigma(x). \]
  %
  If $\phi(x) = x \cdot \xi$. The nondegeneracy of the Hessian of $\phi$ on the support of $\mu$ corresponds precisely to the nonvanishing of the curvatures of $\Sigma$ on $\mu$. Then we may find a family of
\end{comment}
\end{proof}

If $\Omega$ is a bounded open subset of $\RR^{d+1}$ whose boundary is a Riemannian manifold with non-zero Gaussian curvature at each point, then it's Fourier transform has decay one order better than the Fourier transform of it's boundary.

\begin{corollary}
  If $\Omega$ is a bounded open subset of $\RR^d$ whose boundary is a Riemannian manifold $\Sigma$ with non-zero Gaussian curvature at each point. If $I_\Omega$ is the indicator function on $\Omega$, then
  %
  \[ |\widehat{I_\Omega}(\xi)| \lesssim_\Omega |\xi|^{-d/2}. \]
\end{corollary}
\begin{proof}
  We have
  %
  \[ \widehat{I_\Omega}(\xi) = \int_\Omega e^{-2 \pi i \xi \cdot x}\; dx \]
  % u div(V) dV = u (V . n) dS - Grad(u) . V dV
  % As long as V is smooth and div(V) is fine
  % If u = e^{-2 \pi i \xi . x}, Grad(u) = (-2\pi i \xi) e^{-2 \pi i \xi . x}
  % If V(x) = x_1
  %
  Then we can apply Stoke's theorem for each $1 \leq k \leq d+1$ to conclude
  %
  \[ \int_\Omega e^{-2 \pi i \xi \cdot x}\; dx = \frac{(-1)^k}{2 \pi i \xi_k} \int_\Sigma e^{-2 \pi i \xi \cdot x} (dx^1 \wedge \dots \wedge \widehat{dx^k} \wedge \dots \wedge dx^n). \]
  %
  For each $k$, there is a smooth function $\psi_k$ such that
  %
  \[ dx^1 \wedge \dots \wedge \widehat{dx^k} \wedge \dots \wedge dx^n = \psi_k d\sigma. \]
  %
  Thus applying the last case, we find
  %
  \[ \left| \frac{(-1)^k}{2 \pi i \xi_k} \int_\Sigma e^{-2 \pi i \xi \cdot x} (dx^1 \wedge \dots \wedge \widehat{dx^k} \wedge \dots \wedge dx^n) \right| \lesssim \xi_k^{-1} |\xi|^{-(d-1)/2}. \]
  %
  At each point $\xi$, if we choose $\xi_k$ with the largest value, then $|\xi_k| \sim |\xi|$, so
  %
  \[ \left| \int_\Omega e^{-2 \pi i \xi \cdot x}\; dx \right| \lesssim |\xi|^{-(d+1)/2}. \qedhere \]
\end{proof}

The fact that curved surfaces have Fourier decay has many consequences in harmonic analysis.

\begin{example}
  If $M$ is a hypersurface in $\RR^d$, and $\psi$ is a smooth, compactly supported function on $M$, and $f$ is a smooth, compactly supported function on $\RR^d$, we can define a function $Af$ on $\RR^d$ by defining
  %
  \[ (Af)(y) = \int_M f(y - x) \psi(x)\; d\sigma(x). \]
  %
  We note that $Af$ is really the convolution of $f$ with $\psi \sigma$. Thus
  %
  \[ \widehat{Af}(\xi) = \widehat{f}(\xi) \widehat{\psi d\sigma}(\xi). \]
  %
  For each multi-index $\alpha$, the derivative $(Af)_\alpha$ is equal to
  %
  \[ \int_M f_\alpha(y-x) \psi(x)\; d\sigma(x) = f_\alpha * (\psi \sigma). \]
  %
  In particular, we have
  %
  \[ \widehat{(Af)_\alpha} = (2 \pi i \xi)^\alpha \widehat{f}(\xi) \widehat{\psi \sigma}(\xi). \]
  %
  Since we have shown
  %
  \[ |\widehat{\psi \sigma}(\xi)| \lesssim |\xi|^{-(d-1)/2}, \]
  %
  we conclude that if $|\alpha| \leq k$, where $k = (d-1)/2$,
  %
  \[ \| (Af)_\alpha \|_{L^2(\RR^d)} \lesssim \| f \|_{L^2(\RR^d)}. \]
  %
  In particular, this implies that $A$ extends to a unique bounded operator from $L^2(\RR^d)$ to $L^2_k(\RR^d)$, i.e. to a map such that for each $f \in L^2(\RR^d)$, $Af$ is a square integrable function which has square integrable weak derivatives of all orders less than or equal to $k$, and moreover, $\| (Af)_\alpha \|_{L^2(\RR^d)} \lesssim \| f \|_{L^2(\RR^d)}$ for all $|\alpha| \leq k$. Thus the operator $A$ is `smoothening', in a certain sense.

  The operator $A$ is obviously bounded from $L^1(\RR^d)$ to $L^1(\RR^d)$ and $L^\infty(\RR^d)$ to $L^\infty(\RR^d)$, purely from the fact that $\psi \sigma$ is a finite measure. Using curvature and some analytic interpolation, we will now also show that $A$ is bounded from $L^p(\RR^d)$ to $L^q(\RR^d)$, where $p = (d+1)/d$, and $q = d+1$. Interpolation thus yields a number of intermediate estimates. The trick here is to obtain an $(L^1,L^\infty)$ bound for an `improved' version of $A$, and an $(L^2,L^2)$ bound for a `worsened' version of $A$. Interpolating between these two results gives a bound for precisely $A$. It suffices to prove this bound `locally' on $M$, since we can then sum up these bounds, so we may assume that $M$ is given as the graph of some function, i.e. there exists $u$ such that
  %
  \[ M = \{ (x,u(x)):  \} \]


  For each $s$, we write $A_sf = K_s * f$, where
  %
  \[ K_s(x) = \gamma_s |x_d - \phi(x')|_+^{s-1} \psi_0(x). \]
  %
  Here $\gamma_s = s \dots (s + N) e^{s^2}$, where $N$ is some large parameter to be fixed in a moment. The $e^{s^2}$ parameter is to mitigate the growth of $\gamma_s$ as $|\text{Im}(s)| \to \infty$, which allows us to interpolate. The quantity $|u|_+^{s-1}$ is equal to $u^{s-1}$ where $u > 0$, and is equal to 0 when $u \leq 0$. And $\psi_0(x) = \psi(x) (1 + |\nabla_{x'} \phi(x')|^2)^{1/2}$.
\end{example}





\section{Restriction Theorems}

If $f \in L^p(\RR^n)$, then the Hausdorff Young theorem says that $\widehat{f}$ is a function in $L^q(\RR^n)$, where $q$ is the dual of $p$. If $f \in L^1(\RR^n)$, then $\smash{\widehat{f}}$ is actually continuous, so you can meaningfully discuss the behaviour of the Fourier transform when restricted to low dimensional hypersurfaces, for instance, on a sphere of a fixed radius. However, in general $\widehat{f}$ will only be defined almost everywhere, and so it is unclear whether one can form a well defined restriction of the Fourier transform.

The general situation is as follows. If $\mu$ is a measure carried on a compact surface $M$, for a fixed $p$, does there exist an estimate
%
\[ \| \widehat{f} \|_{L^q(M,\mu)} \lesssim \| f \|_{L^p(\RR^n)} \]
%
for Schwartz functions $f$. If this is true, we can apply a density argument to show that the restriction operator $\smash{R(f) = \widehat{f}|_M}$ uniquely extends to a well defined continuous linear operator from $L^p(\RR^n)$ to $L^q(M,\mu)$.

We begin by determining a duality result to the restriction calculation. Assuming our functions are suitably regular, we calculate
%
\begin{align*}
    \int_M (Rf)(\xi) \overline{g(\xi)}\; d\mu(\xi) &= \int_M \left( \int_{\RR^n} f(x) e(- \xi \cdot x)\; dx \right) \overline{g(\xi)}\; d\mu(\xi)\\
    &= \int_{\RR^n} f(x) \overline{\int_M g(\xi) e(\xi \cdot x)\; d\mu(\xi)}\; dx
\end{align*}
%
which implies the formal adjoint of the map $R$ is the \emph{extension operator}
%
\[ (R^* f)(x) = \int_M e(\xi \cdot x) f(\xi) d\mu(\xi) \]
%
which extends a function in frequency space supported on $M$ to a function on the entirety of phase space. By duality properties, $R$ is continuous as an operator from $L^p(\RR^n)$ to $L^q(M,\mu)$ if and only if $R^*$ is continuous as an operator from $L^{q^*}(M,\mu)$ to $L^{p^*}(\RR^n)$. We also calculate
%
\[ ((R^* R)f)(x) = \int_{\RR^n} \left( \int_M e(\xi \cdot (x-y))\; d\mu(\xi) \right) f(y)\; dy = \left( f * \widecheck{\mu} \right)(x) \]
%
So if $R$ is $(p,2)$ continuous, $R^*$ is $(2,p^*)$ continuous, and so $R^*R$ is $(p,p^*)$ continuous. Conversely, if we know that $R^*R$ is $(p,p^*)$ continuous, then we find that for $f \in L^p(\RR^n)$, H\"{o}lder's inequality implies
%
\[ \| Rf \|_{L^2(M,\mu)}^2 = (Rf,Rf)_M = ((R^* R)f, f)_{\RR^d} \leq \| R^*R \|_{p \to p^*} \| f \|_p^2 \]
%
and so we conclude that $\| R \|_{p \to 2} \leq \sqrt{\| R^* R\|_{p \to p^*}}$.

We now prove that $R$ is $(2n+2/n+3, 2)$ continuous, assuming that $M$ has non-zero Gaussian curvature at each point. The previous paragram implies that it suffices to show that it is enough to show that $R^*R$ is $(p,p^*)$ continuous, where $p = (2n+2)/(n+3)$ and $p^* = (2n+2)/(n-1)$. Since
%
\[ (R^*R)(f) = f * \widecheck{\mu} \]
%
We shall verify this using Stein's interpolation theorem. Consider the family of kernels $k_s$, where $\smash{k_s = \widecheck{K_s}}$, and $K_s = \gamma_s |x_n - \varphi(x')|^{s-1}_+ \varphi_0(x)$, where $\gamma_s = s(s+1) \dots (s + N) e^{s^2}$







\chapter{Bellman Function Methods}

It is interesting to ask whether we can obtain bounds of the form
%
\[ \| M f \|_{L^p(\RR^d)} \lesssim_{d,p} \| f \|_{L^p(\RR^d)} \]
%
without employing any interpolation techniques. This is possible, though nontrivial. We begin with a Bellman function approach, which works best in the dyadic scheme, i.e. proving bounds on $M_\Delta$.

The idea here is to perform an \emph{induction on scales}, i.e. to induct on the complexity of the function $f$. For a fixed $f \in L^p(\RR^d)$, our goal is to obtain bounds of the form
%
\[ \left( \int |M_\Delta f(x)|^p\; dx \right)^{1/p} \lesssim \left( \int |f(x)|^p \right)^{1/p} \]
%
where the implicit constant is independent of $p$.

We begin by applying some monotone convergence arguments to simplify our analysis. For each $x \in \RR^d$, $|M_\Delta f(x)| = \lim_{m \to -\infty} |M_{\geq m} f(x)|$, where $M_{\geq m}$ is the operator giving a maximal average over all dyadic cubes containing a point with sidelength exceeding $2^m$, and the limit is monotone increasing. It follows that for any $f \in L^p(\RR^d)$,
%
\[ \| M_\Delta f \|_{L^p(\RR^d)} = \lim_{m \to -\infty} \| M_{\geq m} f \|_{L^p(\RR^d)}. \]
%
Thus if we can obtain a bound
%
\[ \| M_{\geq m} f \|_{L^p(\RR^d)} \lesssim \| f \|_{L^p(\RR^d)} \]
%
with a bound independant of $m$, we would obtain the required bound on $M_\Delta$. But if we could obtain a bound
%
\[ \| M_{\geq 0} f \|_{L^p(\RR^d)} \lesssim \| f \|_{L^p(\RR^d)} \]
%
for all $f \in L^p(\RR^d)$, then a rescaling argument, using the fact that
%
\[ M_{\geq m} f = \text{Dil}_{1/2^d} M_{\geq 0} \text{Dil}_{2^d} f \]
%
shows that we in fact have
%
\begin{align*}
  \| M_{\geq m} f \|_{L^p(\RR^d)} &= 2^{d/p} \| M_{\geq 0} \text{Dil}_{2^d} f \|_{L^p(\RR^d)}\\
  &\lesssim 2^{d/p} \| \text{Dil}_{2^d} f \|_{L^p(\RR^d)} = \| f \|_{L^p(\RR^d)}.
\end{align*}
%
Thus we need only concentrate on the operator $M_{\geq 0}$. Finally, we note we can \emph{localize} our estimates. Given a function $f$ supported on a dyadic cube $Q$ with sidelength $2^n$, and given $x \not \in Q$, then there exists a smallest value $m_x > n$ such that $x$ is contained in a dyadic cube with sidelength $2^{m_x}$ which also contains $Q$. It then follows that
%
\[ (M_{\geq 0} f)(x) = \frac{\int_Q |f(y)|\; dy}{2^{dm_x}} = \frac{\| f \|_{L^1(Q)}}{2^{dm_x}} \]
%
For each $m > n$, if we set $E_m = \{ x \in \RR^d: m_x = m \}$, then $E_m$ is contained in a dyadic cube of sidelength $2^m$, so $|E_m| \leq 2^{dm}$. Thus we have
%
\begin{align*}
  \| M_{\geq 0} f \|_{L^p(Q^c)} &= \left( \sum_{m = n+1}^\infty \| M_{\geq 0} f \|_{L^p(E_m)}^p \right)^{1/p}\\
  &\leq \left( \sum_{m = n+1}^\infty \left( \| f \|_{L^1(Q)}^p / 2^{dpm} \right) 2^{dm} \right)^{1/p}\\
  &\lesssim_{d,p} \| f \|_{L^1(Q)} 2^{dn(1/p - 1)} = \| f \|_{L^1(Q)} |Q|^{1/p-1} \leq \| f \|_{L^p(Q)}.
\end{align*}
%
Thus, if we obtained the bound $\| M_{\geq 0} f \|_{L^p(Q)} \lesssim \| f \|_{L^p(Q)}$, then we would find
%
\begin{align*}
  \| M_{\geq 0} f \|_{L^p(\RR^d)} &\leq \| M_{\geq 0} f \|_{L^p(Q)} + \| M_{\geq 0} f \|_{L^p(Q^c)} \lesssim \| f \|_{L^p(Q)}.
\end{align*}
%
Thus if $f$ is supported on a dyadic cube $Q$, it suffices to estimate $M_{\geq 0} f$ on the support of $f$. But by a final monotone convergence argument, it suffices to bound such functions, since given any $n$ we can write $[-2^n,2^n]$ as the almost disjoint union of $2^d$ sidelength $2^d$ dyadic cubes $Q_{n,1},\dots,Q_{n,2^d}$. For any $f \in L^p(\RR^d)$, we consider a pointwise limit $f = \lim_{n \to \infty} f_{n,1} + \dots + f_{n,2^d}$, where $f_{n,i}$ is equal to $f$ restricted to $Q_{n,i}$, and the limit is monotone. We also have
%
\[ M_{\geq 0} f = \lim_{n \to \infty} M_{\geq 0} f_{n,1} + \dots + M_{\geq 0} f_{n,2^d}. \]
%
where the limit is pointwise and monotone, so
%
\begin{align*}
  \| M_{\geq 0} f \|_{L^p(\RR^d)} &= \lim_{n \to \infty} \| M_{\geq 0} f_{n,1} + \dots + M_{\geq 0} f_{n,2^d} \|_{L^p(\RR^d)}\\
  &\lesssim \lim_{n \to \infty} \| f_{n,1} \|_{L^p(\RR^d)} + \dots + \| f_{n,2^d} \|_{L^p(\RR^d)} \lesssim 2^d \| f \|_{L^p(\RR^d)}.
\end{align*}
%
Thus, after a technical reduction argument, we now show that we only have to establish a bound
%
\[ \| M_{\geq 0} f \|_{L^p(Q)} \lesssim \| f \|_{L^p(Q)}, \]
%
where $f \in L^p(Q)$, $Q$ is a dyadic cube with sidelength $\geq 1$, and the implicit constant is independant of $Q$.

To carry out the inequality, we perform an \emph{induction on scales}. For each $n \geq 0$, we let $C(n)$ denote the optimal constant such that for any function $f \in L^p(\RR^d)$ supported on a dyadic cube $Q$ of sidelength $2^n$,
%
\[ \| M_{\geq 0} f \|_{L^p(Q)} \leq C(n) \cdot \| f \|_{L^p(Q)}. \]
%
If $n = 0$, the problem is trivial, since if $Q$ is dyadic with sidelength $1$ and $x \in Q$, then
%
\[ M_{\geq 0} f = \fint_Q |f(y)|\; dy \]
%
so $\| M_{\geq 0} f \|_{L^p(Q)} = \| f \|_{L^1(Q)}$, and $C(0) = 1$. Our goal is to show that $\sup_{n \geq 0} C(n) < \infty$. Given $f$ supported on a cube $Q$ with sidelength $2^n$, the cube has $2^d$ children $Q_1,\dots,Q_{2^d}$ with sidelength $2^{n-1}$. If we decompose $f = f_1 + \dots + f_{2^d}$ onto these cubes, then by induction we know that
%
\[ \| M_{\geq 0} f_i \|_{L^p(Q_i)} \leq C(n-1) \| f_i \|_{L^p(Q_i)}. \]
%
Now for $x \in Q_i$,
%
\[ (M_{\geq 0} f)(x) = \max \left(M_{\geq 0} f_i(x), \fint_Q |f(y)|\; dy \right). \]
%
Thus if $A = \fint_Q |f(y)|\; dy$, then
%
\begin{align*}
  \| M_{\geq 0} f \|_{L^p(Q)} &= \left( \| M_{\geq 0} f \|_{L^p(Q_1)}^p + \dots + \| M_{\geq 0} f \|_{L^p(Q_{2^d})}^p \right)^{1/p}\\
  &= \left( \| \max(M_{\geq 0} f_1, A) \|_{L^p(Q_1)}^p + \dots + \| \max(M_{\geq 0} f_{2^d}, A) \|_{L^p(Q_{2^d})}^p \right)^{1/p}
\end{align*}
%
The bound $\max(M_{\geq 0} f_i, A) \leq M_{\geq 0} f_i + A$ gives
%
\[ \| M_{\geq 0} f \|_{L^p(Q)} \leq C(n-1) \| f \|_{L^p(Q)} + 2^{d/p} |Q|^{1/p} A = (C(n-1) + 2^{d/p}) \| f \|_{L^p(Q)}. \]
%
This gives $C(n) \leq C(n-1) + 2^{d/p}$, which is not enough to obtain a uniform bound. The idea here is to include more information in our induction hypothesis which gives control on $\max(M_{\geq 0} f_i, A)$. Since $Q$ contains points not in $Q_i$, we need to treat $A$ as an arbitrary quantity in our hypothesis.

To do this, we introduce \emph{cost functions}. For each $A,B,D > 0$ and any integer $n \geq 0$, we let $V_n(A,B,D)$ be the optimal constant such that
%
\[ \| \max(M_{\geq 0} f, A)^p \|_{L^p(Q)} \leq V_n(A,B,D) \]
%
For any function $f$ supported on a dyadic cube $Q$ with sidelength $2^n$, with
%
\[ \| f \|_{L^1(Q)} = B\quad\text{and}\quad \| f \|_{L^p(Q)} = D. \]
%
Our goal will be to show $V_n(A,B,D) \lesssim_p 2^{-dn/p} A + D$ which completes the proof. The role of $B$ is subtle, but will soon become apparan. Of course, we have $\| f \|_{L^1(Q)} \leq 2^{dn(1-1/p)} \| f \|_{L^p(Q)}$, so we have $V_n(A,B,D) = -\infty$ unless $B \leq 2^{dn(1 - 1/p)} D$.

The recursive inequality gives an inequality for the values $V_n(A,B,D)$. TODO: COMPLETE THIS PROOF.

\chapter{$TT^*$ Arguments}

The method of $TT^*$ arguments enables us to obtain bounds on an operator $T$ by exploiting cancellation between an operator and it's adjoint. However, this approach only works when establishing $L^2$ estimates (or at least where one side of an inequality has a norm induced by an inner product). By monotocity, it suffices to consider maximal operators of the form $\max(A_{r_1} f, \dots, A_{r_N} f)$ (provided the implicit constants are independant of $N$),  and by linearization, it suffices to show that for any measurable function $r: \RR^d \to \{ r_1, \dots, r_N \}$,
%
\[ \left( \int |A_{r(x)} f(x)|^p\; dx \right)^{1/p} \lesssim \| f \|_{L^p(\RR^d)} \]
%
where the implicit constant is independant of the function $r$. Thus we consider the linearized operator $M_r: L^2(\RR^d) \to L^2(\RR^d)$ obtained by setting
%
\[ M_r f(y) = (A_{r(y)} f)(y). \]
%
We see easily that $M_r$ is a kernel operator with kernel
%
\[ K(x,y) = \frac{1}{|B_{r(y)}(y)|} \mathbf{I}(|x - y| \leq r(y)). \]
%
Thus
%
\[ M_r^* g(x) = \int_{\RR^d} \frac{\mathbf{I}(|x - y| \leq r(y))}{|B_{r(y)}(y)|} g(y)\; dy, \]
%
and so one can verify that
%
\begin{align*}
  |(M_r M_r^* f)(y)| &= \left| \int_{\RR^d} \int_{\RR^d} \frac{\mathbf{I}(|z - x| \leq r(z)) \mathbf{I}(|y - x| \leq r(y))}{|B(z,r(z))| |B(y,r(y))|} f(z)\; dz\; dx \right|\\
  &= \left| \int_{\RR^d} \int_{\RR^d} \frac{\mathbf{I}(|z - x| \leq r(z)) \mathbf{I}(|y - x| \leq r(y))}{|B(z,r(z))| |B(y,r(y))|} f(z)\; dx\; dz \right|.
\end{align*}
%
For a fixed $z$, the integrand in $x$ vanishes unless $|z - y| \leq r(y) + r(z)$, and in this case we find
%
\[ \left| \int_{\RR^d} \frac{\mathbf{I}(|z - x| \leq r(z)) \mathbf{I}(|y - x| \leq r(y))}{|B(z,r(z))| |B(y,r(y))|} \right| \lesssim_d \frac{1}{\max(r(y)^d, r(z)^d)}. \]
%
Thus we can write
%
\begin{align*}
  |(M_r M_r^* f)(y)| &\lesssim_d \int_{\RR^d} \left( \int_{\substack{|z - y| \leq r(y) + r(z)\\r(y) \leq r(z)}} \frac{|f(z)|}{r(z)^d} + \int_{\substack{|z - y| \leq r(y) + r(z)\\r(y) \geq r(z)}} \frac{|f(z)|}{r(y)^d}\; dx \right)\\
  &\lesssim_d M_{2r}|f|(y) + M_{2r}^* |f|(y).
\end{align*}
%
But we verify by rescaling that $\| M_{2r} \| = \| M_{2r}^* \| = \| M_r \|$, so
%
\[ \| M_r \|^2 = \| M_r M_r^* \| \lesssim_d \| M_r \|. \]
%
But this means that $\| M_r \| \lesssim_d 1$, which gives the bound that we required. Thus we find that the Hardy-Littlewood maximal function is bounded from $L^2(\RR^d)$ to $L^2(\RR^d)$.






\chapter{Maximal Averages Over Curves}

\section{Averages over a Parabola}

Given any measurable function $f: \RR^2 \to \CC$ we can consider the maximal average
%
\[ (Mf)(x,y) = \sup_{\varepsilon > 0} \frac{1}{2\varepsilon} \int_{-\varepsilon}^\varepsilon|f(x+t,y+t)|\; dt. \]
%
Thus $Mf$ gives a maximal average over parabolas. Our goal is to show $\| Mf \|_{L^p(\RR^d)} \lesssim_p \| f \|_{L^p(\RR^d)}$ for $1 < p < \infty$.

It will be convenient to look at the operator
%
\[ \tilde{M} f(x,y) = \sup_{\varepsilon > 0} \frac{1}{2\varepsilon} \int_{\varepsilon/2}^{\varepsilon} |f(x+t,y+t^2)|\; dt. \]
%
A dyadic decomposition shows that $L^p$ bounds for $\tilde{M}$ imply $L^p$ bounds for $M$.

For each $k \in \ZZ$, let $\tilde{M_k} f(x,y) = 2^{-k} \int_{2^k}^{2^{k+1}} f(x+t,y+t^2)\; dt$. Rescaling shows that
%
\[ \| \tilde{M}_k \|_{L^p(\RR^2) \to L^p(\RR^2)} = \| \tilde{M}_0 \|_{L^p(\RR^2) \to L^p(\RR^2)} \]
%
so it suffices to focus on $\tilde{M}_0$. The operator is translation invariant and therefore has a Fourier multiplier
%
\[ \tilde{m}(\xi,\eta) = \int_1^2 e^{2 \pi i (\xi t + \eta t^2)}\; dt. \]
%
Note that $\tilde{m}$ is defined by an oscillatory integral with phase $\phi(t) = \xi t + \eta t^2$. We note that $\phi'(t) = \xi + 2 \eta t$, so Van der Corput's lemma implies that for $|\xi| \geq 10|\eta|$,
%
\[ |\tilde{m}(\xi,\eta)| \lesssim \frac{1}{|\xi|}. \]
%
Similarily, $\phi''(t) = 2 \eta$, so we find
%
\[ |\tilde{m}(\xi,\eta)| \lesssim \frac{1}{|\eta|^{1/2}}. \]
%
If $f \in L^2(\RR^2)$ and $\widehat{f}$ is supported on the region
%
\[ E_0 = \{ (\xi,\eta) : |\eta| \geq 1\ \text{or}\ |\xi| \leq 1\ \text{and} |\eta| \geq 10 \} \]
%
then $\| \tilde{m} \|_{L^\infty(E_0)} \lesssim 1$ and so
%
\[ \| \tilde{M}_0 f \|_{L^2(\RR^2)} = \| \tilde{m} \widehat{f} \|_{L^2(\RR^2)} \lesssim \| \widehat{f} \|_{L^2(\RR^2)} = \| f \|_{L^2(\RR^2)}. \]
%
On the other hand, we can decompose $\RR^2 - E_0$ into

suppose $\widehat{f}$ is supported on the region
%
\[ E_1 = \{ (\xi,\eta) : |\xi| \leq 1\ \text{and}\ |\eta| \leq 10 \}. \]
%
Then the uncertainty principle implies that $f$ is roughly constant on scales $|\Delta x| \leq 1$ and $|\Delta y| \leq 1/10$, which should imply good bounds for the maximal average. More precisely, $\widehat{f}$ is supported on the ellipsoid
%
\[ \left\{ (\xi,\eta) \in \RR^2 : \xi^2/2 + \eta^2/20 \leq 1 \right\}. \]
%
Thus the uncertainty principle implies that $f$ is roughly constant on scales $|\Delta x|^2 \leq 1/2$ and $|\Delta y|^2 \leq 1/20$,
%
\[ s \]
%
\[ \phi(x) = \frac{1}{( 1 + 2 x^2 + 20 y^2 )^N} \]


\part{Abstract Harmonic Analysis}

The main property of spaces where Fourier analysis applies is symmetry -- for a function $\RR$, we can translate and negate. On $\RR^n$ we have not only translational symmetry but also rotational symmetry. It turns out that we can apply Fourier analysis to any `space with symmetry'. That is, functions on an Abelian group. We shall begin with the study of finite abelian groups, where convergence questions disappear, and with it much of the analytical questions involved in the theory. We then proceed to generalize to a study of infinite abelian groups with topological structure.







\chapter{Topological Groups}

In abstract harmonic analysis, the main subject matter is the {\bf topological group}, a group $G$ equipped with a topology which makes the operation of multiplication and inversion continuous. In the mid 20th century, it was realized that basic Fourier analysis could be generalized to a large class of groups. The nicest generalization occurs over the locally compact groups, which simplifies the theory considerably.

\begin{example}
    There are a few groups we should keep in mind for intuition in the general topological group.
    %
    \begin{itemize}
        \item The classical groups $\RR^n$ and $\mathbf{T}^n$, from which Fourier analysis originated.
        \item The group $\mu$ of roots of unity, rational numbers $\mathbf{Q}$, and cyclic groups $\mathbf{Z}_n$.
        \item The matrix subgroups of the general linear group $GL(n)$.
        \item The product $\mathbf{T}^\omega$ of Torii, occurring in the study of Dirichlet series.
        \item The product $\mathbf{Z}_2^\omega$, which occurs in probability theory, and other contexts.
        \item The field of $p$-adic numbers $\mathbf{Q}_p$, which are the completion of $\mathbf{Q}$ with respect to the absolute value $|p^{-m} q|_p = p^m$.
    \end{itemize}
\end{example}

\section{Basic Results}

The topological structure of a topological group naturally possesses large amounts of symmetry, simplifying the spatial structure. For any topological group, the maps
%
\[ x \mapsto gx\ \ \ \ \ \ \ \ \ \ x \mapsto xg\ \ \ \ \ \ \ \ \ \ x \mapsto x^{-1} \]
%
are homeomorphisms. Thus if $U$ is a neighbourhood of $x$, then $gU$ is a neighbourhood of $gx$, $Ug$ a neighbourhood of $xg$, and $U^{-1}$ a neighbourhood of $x^{-1}$, and as we vary $U$ through all neighbourhoods of $x$, we obtain all neighbourhoods of the other points. Understanding the topological structure at any point reduces to studying the neighbourhoods of the identity element of the group.

In topological group theory it is even more important than in basic group theory to discuss set multiplication. If $U$ and $V$ are subsets of a group, then we define
%
\[ U^{-1} = \{ x^{-1} : x \in U \}\ \ \ \ \ \ \ \ UV = \{ xy: x \in U, y \in V \} \]
%
We let $V^2 = VV$, $V^3 = VVV$, and so on.

\begin{theorem}
    Let $U$ and $V$ be subsets of a topological group.
    %
    \begin{enumerate}
        \item[(i)] If $U$ is open, then $UV$ is open.
        \item[(ii)] If $U$ is compact, and $V$ closed, then $UV$ is closed.
        \item[(iii)] If $U$ and $V$ are connected, $UV$ is connected.
        \item[(iv)] If $U$ and $V$ are compact, then $UV$ is compact.
    \end{enumerate}
\end{theorem}
\begin{proof}
    To see that (i) holds, we see that
    %
    \[ UV = \bigcup_{x \in V} Ux \]
    %
    and each $Ux$ is open. To see (ii), suppose $u_i v_i \to x$. Since $U$ is compact, there is a subnet $u_{i_k}$ converging to $y$. Then $y \in U$, and we find
    %
    \[ v_{i_k} = u_{i_k}^{-1} ( u_{i_k} v_{i_k} ) \to y^{-1} x \]
    %
    Thus $y^{-1} x \in V$, and so $x = y y^{-1} x \in UV$. (iii) follows immediately from the continuity of multiplication, and the fact that $U \times V$ is connected, and (iv) follows from similar reasoning.
\end{proof}

\begin{example}
    If $U$ is merely closed, then (ii) need not hold. For instance, in $\RR$, take $U = \alpha \mathbf{Z}$, and $V = \mathbf{Z}$, where $\alpha$ is an irrational number. Then $U + V = \alpha \mathbf{Z} + \mathbf{Z}$ is dense in $\RR$, and is hence not closed.
\end{example}

There are useful ways we can construct neighbourhoods under the group operations, which we list below.

\begin{lemma}
    Let $U$ be a neighbourhood of the identity. Then
    %
    \begin{itemize}
        \item[(1)] There is an open $V$ such that $V^2 \subset U$.
        \item[(2)] There is an open $V$ such that $V^{-1} \subset U$.
        \item[(3)] For any $x \in U$, there is an open $V$ such that $xV \subset U$.
        \item[(4)] For any $x$, there is an open $V$ such that $xVx^{-1} \subset U$.
    \end{itemize}
\end{lemma}
\begin{proof}
    (1) follows simply from the continuity of multiplication, and (2) from the continuity of inversion. (3) is verified because $x^{-1}U$ is a neighbourhood of the origin, so if $V = x^{-1}U$, then $xV = U \subset U$. Finally (4) follows in a manner analogously to (3) because $x^{-1}Ux$ contains the origin.
\end{proof}

If $\mathcal{U}$ is an open basis at the origin, then it is only a slight generalization to show that for any of the above situations, we can always select $V \in \mathcal{U}$. Conversely, suppose that $\mathcal{V}$ is a family of subsets of a (not yet topological) group $G$ containing $e$ such that (1), (2), (3), and (4) hold. Then the family $\mathcal{V}' = \{ xV : V \in \mathcal{V}, x \in G \}$ forms a subbasis for a topology on $G$ which forms a topological group. If $\mathcal{V}$ also has the base property, then $\mathcal{V}'$ is a basis.

\begin{theorem}
    If $K$ and $C$ are disjoint, $K$ is compact, and $C$ is closed, then there is a neighbourhood $V$ of the origin for which $KV$ and $CV$ is disjoint. If $G$ is locally compact, then we can select $V$ such that $KV$ is precompact.
\end{theorem}
\begin{proof}
    For each $x \in K$, $C^c$ is an open neighbourhood containing $x$, so by applying the last lemma recursively we find that there is a symmetric neighbourhood $V_x$ such that $x V_x^4 \subset C^c$. Since $K$ is compact, finitely many of the $xV_x$ cover $K$. If we then let $V$ be the open set obtained by intersecting the finite subfamily of the $V_x$, then $KV$ is disjoint from $CV$.
\end{proof}

Taking $K$ to be a point, we find that any open neighbourhood of a point contains a closed neighbourhood. Provided points are closed, we can set $C$ to be a point as well.

\begin{corollary}
    Every Kolmogorov topological group is Hausdorff.
\end{corollary}

\begin{theorem}
    For any set $A \subset G$,
    %
    \[ \overline{A} = \bigcap_V AV \]
    %
    Where $V$ ranges over the set of neighbourhoods of the origin.
\end{theorem}
\begin{proof}
    If $x \not \in \overline{A}$, then the last theorem guarantees that there is $V$ for which $\overline{A}V$ and $Ax$ are disjoint. We conclude $\bigcap AV \subset \overline{A}$. Conversely, any neighbourhood contains a closed neighbourhood, so that $\overline{A} \subset AV$ for a fixed $V$, and hence $\overline{A} \subset \bigcap AV$.
\end{proof}

\begin{theorem}
    Every open subgroup of $G$ is closed.
\end{theorem}
\begin{proof}
    Let $H$ be an open subgroup of $G$. Then $\overline{H} = \bigcap_V HV$. If $W$ is a neighbourhood of the origin contained in $H$, then we find $\overline{H} \subset HW \subset H$, so $H$ is closed.
\end{proof}

We see that open subgroups of a group therefore correspond to connected components of the group, so that connected groups have no proper open subgroups. This also tells us that a locally compact group is $\sigma$-compact on each of its components, for if $V$ is a pre-compact neighbourhood of the origin, then $V^2, V^3, \dots$ are all precompact, and $\bigcup_{k = 1}^\infty V^k$ is an open subgroup of $G$, which therefore contains the component of $e$, and is $\sigma$-compact. Since the topology of a topological group is homogenous, we can conclude that all components of the group are $\sigma$ compact.

\section{Quotient Groups}

If $G$ is a topological group, and $H$ is a subgroup, then $G/H$ can be given a topological structure in the obvious way. The quotient map is open, because $VH$ is open in $G$ for any open set $V$, and if $H$ is normal, $G/H$ is also a topological group, because multiplication is just induced from the quotient map of $G \times G$ to $G/H \times G/H$, and inversion from $G$ to $G/H$. We should think the quotient structure is pleasant, but if no conditions on $H$ are given, then $G/H$ can have pathological structure. One particular example is the quotient $\mathbf{T}/\mu_\infty$ of the torus modulo the roots of unity, where the quotient is lumpy.

\begin{theorem}
    If $H$ is closed, $G/H$ is Hausdorff.
\end{theorem}
\begin{proof}
    If $x \neq y \in G/H$, then $xHy^{-1}$ is a closed set in $G$, not containing $e$, so we may conclude there is a neighbourhood $V$ for which $V$ and $VxHy^{-1}$ are disjoint, so $VyH$ and $VxH$ are disjoint. This implies that the open sets $V(xH)$ and $V(yH)$ are disjoint in $G/H$.
\end{proof}

\begin{theorem}
    If $G$ is locally compact, $G/H$ is also.
\end{theorem}
\begin{proof}
    If $\{ U_i \}$ is a basis of precompact neighbourhoods at the origin, then $U_iH$ is a family of precompact neighbourhoods of the origin in $G/H$, and is in fact a basis, for if $V$ is any neighbourhood of the origin, there is $U_i \subset \pi^{-1}(V)$, and so $U_iH \subset V$.
\end{proof}

If $G$ is a non-Hausdorff group, then $\overline{\{e\}} \neq \{ e \}$, and $G/\overline{\{e\}}$ is Hausdorff. Thus we can get away with assuming all our topological groups are Hausdorff, because a slight modification in the algebraic structure of the topological group gives us this property.

\section{Uniform Continuity}

An advantage of the real line $\RR$ is that continuity can be explained in a {\it uniform sense}, because we can transport any topological questions about a certain point $x$ to questions about topological structure near the origin via the map $g \mapsto x^{-1}g$. We can then define a uniformly continuous function $f: \RR \to \RR$ to be a function possessing, for every $\varepsilon > 0$, a $\delta > 0$ such that if $|y| < \delta$, $|f(x+y) - f(x)|<\varepsilon$. Instead of having to specify a $\delta$ for every point on the domain, the $\delta$ works uniformly everywhere. The group structure is all we need to talk about these questions.

We say a function $f: G \to H$ between topological groups is (left) uniformly continuous if, for any open neighbourhood $U$ of the origin in $H$, there is a neighbourhood $V$ of the origin in $G$ such that for each $x$, $f(xV) \subset f(x) U$. Right continuity requires $f(Vx) \subset U f(x)$. The requirement of distinguishing between left and right uniformity is important when we study non-commutative groups, for there are certainly left uniform maps which are not right uniform in these groups. If $f: G \to \mathbf{C}$, then left uniform continuity is equivalent to the fact that $\| L_x f - f \|_\infty \to 0$ as $x \to 1$, where $(L_x f)(y) = f(xy)$. Right uniform continuity requires $\| R_x f - f \|_\infty \to 0$, where $(R_x f)(y) = f(yx)$. $R_x$ is a homomorphism, but $L_x$ is what is called an antihomomorphism.

\begin{example}
    Let $G$ be any Hausdorff non-commutative topological group, with sequences $x_i$ and $y_i$ for which $x_i y_i \to e$, $y_i x_i \to z \neq e$. Then the uniform structures on $G$ are not equivalent.
\end{example}

It is hopeless to express uniform continuity in terms of a new topology on $G$, because the topology only gives a local description of continuity, which prevents us from describing things uniformly across the whole group. However, we can express uniform continuity in terms of a new topology on $G \times G$. If $U \subset G$ is an open neighbourhood of the origin, let
%
\[ L_U = \{ (x,y): yx^{-1} \in U \}\ \ \ \ \ R_U = \{ (x,y): x^{-1}y \in U \} \]
%
The family of all $L_U$ (resp. $R_U$) is known as the left (right) uniform structure on $G$, denoted $LU(G)$ and $RU(G)$. Fix a map $f: G \to H$, and consider the map
%
\[ g(x,y) = (f(x), f(y)) \]
%
from $G^2$ to $H^2$. Then $f$ is left (right) uniformly continuous if and only if $g$ is continuous with respect to $LU(G)$ and $LU(H)$ ($RU(G)$ and $RU(H)$). $LU(G)$ and $RU(G)$ are weaker than the product topologies on $G$ and $H$, which reflects the fact that uniform continuity is a strong condition than normal continuity. We can also consider uniform maps with respect to $LU(G)$ and $RU(H)$, and so on and so forth. We can also consider uniform continuity on functions defined on an open subset of a group.

\begin{example}
    Here are a few examples of easily verified continuous maps.
    \begin{itemize}
        \item If the identity map on $G$ is left-right uniformly continuous, then $LU(G) = RU(G)$, and so uniform continuity is invariant of the uniform structure chosen.
        \item Translation maps $x \mapsto axb$, for $a,b \in G$, are left and right uniform.
        \item Inversion is uniformly continuous.
    \end{itemize}
\end{example}

\begin{theorem}
    All continuous maps on compact subsets of topological groups are uniformly continuous.
\end{theorem}
\begin{proof}
    Let $K$ be a compact subset of a group $G$, and let $f:K \to H$ be a continuous map into a topological group. We claim that $f$ is then uniformly continuous. Fix an open neighbourhood $V$ of the origin, and let $V'$ be a symmetric neighbourhood such that $V'^2 \subset V$. For any $x$, there is $U_x$ such that
    %
    \[ f(x)^{-1} f(xU_x) \subset V' \]
    %
    Choose $U'_x$ such that $U'^2_x \subset U_x$. The $xU'_x$ cover $K$, so there is a finite subcover corresponding to sets $U'_{x_1}, \dots, U'_{x_n}$. Let $U = U'_{x_1} \cap \dots \cap U'_{x_n}$. Fix $y \in G$, and suppose $y \in x_k U'_{x_k}$. Then
    %
    \begin{align*}
        f(y)^{-1} f(yU) &= f(y)^{-1} f(x_k) f(x_k)^{-1} f(yU)\\
        &\subset f(y)^{-1} f(x_k) f(x_k)^{-1} f(x_k Ux_k)\\
        &\subset f(y)^{-1} f(x_k) V'\\
        &\subset V'^2 \subset V
    \end{align*}
    %
    So that $f$ is left uniformly continuous. Right uniform continuity is proven in the exact same way.
\end{proof}

\begin{corollary}
    All maps with compact support are uniformly continuous.
\end{corollary}

\begin{corollary}
    Uniform continuity on compact groups is invariant of the uniform structure chosen.
\end{corollary}

\section{Ordered Groups}

In this section we describe a general class of groups which contain both interesting and pathological examples. Let $G$ be a group with an ordering $<$ preserved by the group operations, so that $a < b$ implies both $ag < bg$ and $ga < gb$. We now prove that the order topology gives $G$ the structure of a normal topological group (the normality follows because of general properties of order topologies).

First note, that $a < b$ implies $a^{-1} < b^{-1}$. This results from a simple algebraic trick, because
%
\[  a^{-1} = a^{-1} b b^{-1} > a^{-1} a b^{-1} = b^{-1} \]
%
This implies that the inverse image of an interval $(a,b)$ under inversion is $(b^{-1}, a^{-1})$, hence inversion is continuous.

Now let $e < b < a$. We claim that there is then $e < c$ such that $c^2 < a$. This follows because if $b^2 \geq a$, then $b \geq ab^{-1}$ and so
%
\[ (ab^{-1})^2 = ab^{-1}ab^{-1} \leq ab^{-1}b = a \]
%
Now suppose $a < e < b$. If $\inf \{ y : y > e \} = x > e$, then $(x^{-1}, x) = \{ e \}$, and the topology on $G$ is discrete, hence the continuity of operations is obvious. Otherwise, we may always find $c$ such that $c^2 < b$, $a < c^{-2}$, and then if $c^{-1} < g,h < c$, then
%
\[ a < c^{-2} < gh < c^2 < b \]
%
so multiplication is continuous at every pair $(x,x^{-1})$. In the general case, if $a < gh < b$, then $g^{-1}ah^{-1} < e < g^{-1}bh^{-1}$, so there is $c$ such that if $c^{-1} < g',h' < c$, then $g^{-1}ah^{-1} < g'h' < g^{-1}bh^{-1}$, so $a < gg'h'h < b$. The set of $gg'$, where $c^{-1} < g' < c$, is really just the set of $gc^{-1} < x < gc$, and the set of $h'h$ is really just the set of $c^{-1}h < x < ch$. Thus multiplication is continuous everywhere.

\begin{example}[Dieudonne]
    For any well ordered set $S$, the dictionary ordering on $\RR^S$ induces a linear ordering inducing a topological group structure on the set of maps from $S$ to $\RR$.
\end{example}

Let us study Dieudonne's topological group in more detail. If $S$ is a finite set, or more generally possesses a maximal element $w$, then the topology on $\RR^S$ can be defined such that $f_i \to f$ if eventually $f_i(s) = f(s)$ for all $s < w$ simultaneously, and $f_i(w) \to f(w)$. Thus $\RR^S$ is isomorphic (topologically) to a discrete union of a certain number of copies of $\RR$, one for each tuple in $S - \{ w \}$.

If $S$ has a countable cofinal subset $\{ s_i \}$, the topology is no longer so simple, but $\RR^S$ is still first countable, because the sets
%
\[ U_i = \{ f : (\forall w < s_i: f(w) = 0) \} \]
%
provide a countable neighbourhood basis of the origin.

The strangest properties of $\RR^S$ occur when $S$ has no countable cofinal set. Suppose that $f_i \to f$. We claim that it follows that $f_i = f$ eventually. To prove by contradiction, we assume without loss of generality (by thinning the sequence) that no $f_i$ is equal to $f$. For each $f_i$, find the largest $w_i \in S$ such that for $s < w_i$, $f_i(s) = f(s)$ (since $S$ is well ordered, the set of elements for which $f_i(s) \neq f(s)$ has a minimal element). Then the $w_i$ form a countable cofinal set, because if $v \in S$ is arbitrary, the $f_i$ eventually satisfy $f_i(s) = f(s)$ for $s < v$, hence the corresponding $w_i$ is greater than $v_i$. Hence, if $f_i \to f$ in $\RR^S$, where $S$ does not have a countable cofinal subset, then eventually $f_i = f$. We conclude all countable sets in $\RR^S$ are closed, and this proof easily generalises to show that if $S$ does not have a cofinal set of cardinality $\mathfrak{a}$, then every set of cardinality $\leq \mathfrak{a}$ is closed.

The simple corollary to this proof is that compact subsets are finite. Let $X = f_1, f_2, \dots$ be a denumerable, compact set. Since all subsets of $X$ are compact, we may assume $f_1 < f_2 < \dots$ (or $f_1 > f_2 > \dots$, which does not change the proof in any interesting way). There is certainly $g \in \RR^S$ such that $g < f_1$, and then the sets $(g,f_2), (f_1, f_3), (f_2,f_4), \dots$ form an open cover of $X$ with no finite subcover, hence $X$ cannot be compact. We conclude that the only compact subsets of $\RR^S$ are finite.

Furthermore, the class of open sets is closed under countable intersections. Consider a series of functions
%
\[ f_1 \leq f_2 \leq \dots < h < \dots \leq g_2 \leq g_1 \]
%
Suppose that $f_i \leq k < h < k' \leq g_j$. Then the intersection of the $(f_i, g_i)$ contains an interval $(k,k')$ around $h$, so that the intersection is open near $h$. The only other possiblity is that $f_i \to h$ or $g_i \to h$, which can only occur if $f_i = h$ or $g_i = h$ eventually, in which case we cannot have $f_i < h$, $h < g_i$. We conclude the intersection of countably many intervals is open, because we can always adjust any intersection to an intersection of this form without changing the resulting intersecting set (except if the set is empty, in which case the claim is trivial). The general case results from noting that any open set in an ordered group is a union of intervals.

\section{Topological Groups arising from Normal subgroups}

Let $G$ be a group, and $\mathcal{N}$ a family of normal subgroups closed under intersection. If we interpret $\mathcal{N}$ as a neighbourhood base at the origin, the resulting topology gives $G$ the structure of a totally disconnected topological group, which is Hausdorff if and only if $\bigcap \mathcal{N} = \{ e \}$. First note that $g_i \to g$ if $g_i$ is eventually in $gN$, for every $N \in \mathcal{N}$, which implies $g_i^{-1} \in Ng^{-1} = g^{-1}N$, hence inversion is continuous. Furthermore, if $h_i$ is eventually in $hN$, then $g_ih_i \in gNhN = ghN$, so multiplication is continuous. Finally note that $N^c = \bigcup_{g \neq e} gN$ is open, so that every open set is closed.

\begin{example}
    Consider $\mathcal{N} = \{ \mathbf{Z}, 2\mathbf{Z}, 3\mathbf{Z}, \dots \}$. Then $\mathcal{N}$ induces a Hausdorff topology on $\mathbf{Z}$, such that $g_i \to g$, if and only if $g_i$ is eventually in $g + n \mathbf{Z}$ for all $n$. In this topology, the series $1,2,3,\dots$ converges to zero!
\end{example}

This example gives us a novel proof, due to Furstenburg, that there are infinitely many primes. Suppose that there were only finitely many, $\{ p_1, p_2, \dots, p_n \}$. By the fundamental theorem of arithmetic,
%
\[ \{ -1, 1 \} = (\mathbf{Z} p_1)^c \cap \dots \cap (\mathbf{Z} p_n)^c \]
%
and is therefore an open set. But this is clearly not the case as open sets must contain infinite sequences.

\chapter{The Haar Measure}

One of the reasons that we isolate locally compact groups to study is that they possess an incredibly useful object allowing us to understand functions on the group, and thus the group itself. A {\bf left (right) Haar measure} for a group $G$ is a Radon measure $\mu$ for which $\mu(xE) = \mu(E)$ for any $x \in G$ and measurable $E$ ($\mu(Ex) = \mu(E)$ for all $x$ and $E$). For commutative groups, all left Haar measures are right Haar measures, but in non-commutative groups this need not hold. However, if $\mu$ is a right Haar measure, then $\nu(E) = \mu(E^{-1})$ is a left Haar measure, so there is no loss of generality in focusing our study on left Haar measures.

\begin{example}
    The example of a Haar measure that everyone knows is the Lebesgue measure on $\RR$ (or $\RR^n$). It commutes with translations because it is the measure induced by the linear functional corresponding to Riemann integration on $C_c^+(\RR^n)$. A similar theory of Darboux integration can be applied to linearly ordered groups, leading to the construction of a Haar measure on such a group.
\end{example}

\begin{example}
    If $G$ is a Lie group, consider a $2$-tensor $g_e \in T^2_e(G)$ inducing an inner product at the origin. Then the diffeomorphism $f: a \mapsto b^{-1}a$ allows us to consider $g_b = f^* \lambda \in T^2_b(G)$, and this is easily verified to be an inner product, hence we have a Riemannian metric. The associated Riemannian volume element can be integrated, producing a Haar measure on $G$.
\end{example}

\begin{example}
    If $G$ and $H$ have Haar measures $\mu$ and $\nu$, then $G \times H$ has a Haar measure $\mu \times \nu$, so that the class of topological groups with Haar measures is closed under the product operation. We can even allow infinite products, provided that the groups involved are compact, and the Haar measures are normalized to probability measures. This gives us measures on $F_2^\omega$ and $\mathbf{T}^\omega$, which models the probability of an infinite sequence of coin flips.
\end{example}

\begin{example}
    $dx/x$ is a Haar measure for the multiplicative group of positive real numbers, since
    %
    \[ \int_a^b \frac{1}{x} = \log(b) - \log(a) = \log(cb) - \log(ca) = \int_{ca}^{cb} \frac{1}{x} \]
    %
    If we take the multiplicative group of all non-negative real numbers, the Haar measure becomes $dx/|x|$.
\end{example}

\begin{example}
    $dx dy/(x^2 + y^2)$ is a Haar measure for the multiplicative group of complex numbers, since we have a basis of `arcs' around the origin, and by a change of variables to polar coordinates, we verify the integral is changed by multiplication. Another way to obtain this measure is by noticing that $\mathbf{C}^\times$ is topologically isomorphic to the product of the circle group and the multiplicative group of real numbers, and hence the measure obtained should be the product of these measures. Since
    %
    \[ \frac{dx dy}{x^2 + y^2} = \frac{dr d\theta}{r} \]
    %
    We see that this is just the product of the Haar measure on $\RR^+$, $dr/r$, and the Haar measure on $\mathbf{T}$, $d \theta$.
\end{example}

\begin{example}
    The space $M_n(\RR)$ of all $n$ by $n$ real matrices under addition has a Haar measure $dM$, which is essentially the Lebesgue measure on $\RR^{n^2}$. If we consider the measure on $GL_n(\RR)$, defined by
    %
    \[ \frac{dM}{\text{det}(M)^n} \]
    %
    To see this, note the determinant of the map $M \mapsto NM$ on $M_n(\RR)$ is $\text{det}(N)^n$, because we can view $M_n(\RR)$ as the product of $\RR^n$ $n$ times, multiplication operates on the space componentwise, and the volume of the image of the unit paralelliped in each $\RR^n$ is $\text{det}(N)$. Since the multiplicative group of complex numbers $z = x + iy$ can be identified with the group of matrices of the form
    %
    \[ \begin{pmatrix} x & -y \\ y & x \end{pmatrix} \]
    %
    and the measure on $\mathbf{C} - \{ 0 \}$ then takes the form $dM/\text{det}(M)$. More generally, if $G$ is an open subset of $\RR^n$, and left multiplication acts affinely, $xy = A(x)y + b(x)$, then $dx/|\text{det}(A(x))|$ is a left Haar measure on $G$, where $dx$ is Lebesgue measure.
\end{example}

It turns out that there is a Haar measure on any locally compact group, and what's more, it is unique up to scaling. The construction of the measure involves constructing a positive linear functional $\phi: C_c(G) \to \RR$ such that $\phi(L_x f) = \phi(f)$ for all $x$. The Riesz representation theorem then guarantees the existence of a Radon measure $\mu$ which represents this linear functional, and one then immediately verifies that this measure is a Haar measure.

\begin{theorem}
    Every locally compact group $G$ has a Haar measure.
\end{theorem}
\begin{proof}
    The idea of the proof is fairly simple. If $\mu$ was a Haar measure, $f \in C_c^+(G)$ was fixed, and $\phi \in C_c^+(G)$ was a function supported on a small set, and behaving like a step function, then we could approximate $f$ well by translates of $\phi$,
    %
    \[ f(x) \approx \sum c_i (L_{x_i} \phi) \]
    %
    Hence
    %
    \[ \int f(x) d \mu \approx \sum c_i \int L_{x_i} \phi = \sum c_i \int \phi \]
    %
    If $\int \phi = 1$, then we could approximate $\int f(x) d \mu$ as literal sums of coefficients $c_i$. Since $\mu$ is outer regular, and $\phi$ is supported on neighbourhoods, one can show $\int f(x) d\mu$ is the infinum of $\sum c_i$, over all choices of $c_i > 0$ and $\int \phi \geq 1$, for which $f \leq \sum c_i L_{x_i} \phi$. Without the integral, we cannot measure the size of the functions $\phi$, so we have to normalize by a different factor. We define $(f: \phi)$ to be the infinum of the sums $\sum c_i$, where $f \leq \sum c_i L_{x_i} \phi$ for some $x_i \in G$. We would then have
    %
    \[ \int f d \mu \leq (f: \phi) \int \phi d\mu \]
    %
    If $k$ is fixed with $\int k = 1$, then we would have
    %
    \[ \int f d\mu \leq (f: \phi) (\phi: k) \]
    %
    We cannot change $k$ if we wish to provide a limiting result in $\phi$, so we notice that $(f: g) (g: h) \leq (f:h)$, which allows us to write
    %
    \[ \int f d\mu \leq \frac{(f: \phi)}{(k : \phi)} \]
    %
    Taking the support of $\phi$ to be smaller and smaller, this value should approximate the integral perfectly accurately.

    Define the linear functional
    %
    \[ I_\phi(f) = \frac{(f: \phi)}{(k: \phi)} \]
    %
    Then $I_\phi$ is a sublinear, monotone, function with a functional bound
    %
    \[ (k: f)^{-1} \leq I_\phi(f) \leq (f: k) \]
    %
    Which effectively says that, regardless of how badly we choose $\phi$, the approximation factor $(f:\phi)$ is normalized by the approximation factor $(k:\phi)$ so that the integral is bounded. Now we need only prove that $I_\phi$ approximates a linear functional well enough that we can perform a limiting process to obtain a Haar integral. If $\varepsilon > 0$, and $g \in C_c^+(G)$ with $g = 1$ on $\text{supp}(f_1 + f_2)$, then the functions
    %
    \[ h = f_1 + f_2 + \varepsilon g \]
    %
    \[ h_1 = f_1/h \ \ \ \ \ h_2 = f_2/h \]
    %
    are in $C^+_0(G)$, if we define $h_i(x) = 0$ if $f_i(x) = 0$. This implies that there is a neighbourhood $V$ of $e$ such that if $x \in V$, and $y$ is arbitrary, then
    %
    \[ | h_1(xy) - h_1(y) | \leq \varepsilon\ \ \ \ \ | h_2(xy) - h_2(y) | < \varepsilon \]
    %
    If $\text{supp}(\phi) \subset V$, and $h \leq \sum c_i L_{x_i} \phi$, then
    %
    \[ f_j(x) = h(x) h_j(x) \leq \sum c_i \phi(x_i x) h_j(x) \leq \sum c_i \phi(x_i x) \left[ h_j(x_i^{-1}) + \varepsilon \right] \]
    %
    since we may assume that $x_i x \in \text{supp}(\phi) \subset V$. Then, because $h_1 + h_2 \leq 1$,
    %
    \[ (f_1: \phi) + (f_2 : \phi) \leq \sum c_j [h_1(x_j^{-1}) + \varepsilon] + \sum c_j [h_2(x_j^{-1}) + \varepsilon] \leq \sum c_j [1 + 2 \varepsilon] \]
    %
    Now we find, by taking infinums, that
    %
    \[ I_\phi(f_1) + I_\phi(f_2) \leq I_\phi(h) (1 + 2 \varepsilon) \leq [I_\phi(f_1 + f_2) + \varepsilon I_\phi(g)] [1 + 2 \varepsilon] \]
    %
    Since $g$ is fixed, and we have a bound $I_\phi(g) \leq (g: k)$, we may always find a neighbourhood $V$ (dependant on $f_1$, $f_2$) for any $\varepsilon > 0$ such that
    %
    \[ I_\phi(f_1) + I_\phi(f_2) \leq I_\phi(f_1 + f_2) + \varepsilon \]
    %
    if $\text{supp}(\phi) \subset V$.

    Now we have estimates on how well $I_\phi$ approximates a linear function, so we can apply a limiting process. Consider the product
    %
    \[ X = \prod_{f \in C^+_0(G)} [(k : f)^{-1}, (k: f_0)] \]
    %
    a compact space, by Tychonoff's theorem, consisting of $F: C_c^+(G) \to \RR$ such that $(k : f)^{-1} \leq F(f) \leq (f: k)$. For each neighbourhood $V$ of the identity, let $K(V)$ be the closure of the set of $I_\phi$ such that $\text{supp}(\phi) \subset V$. Then the set of all $K(V)$ has the finite intersection property, so we conclude there is some $I: C_c^+(G) \to \RR$ contained in $\bigcap K(V)$. This means that every neighbourhood of $I$ contains $I_\phi$ with $\text{supp}(\phi) \subset V$, for all $\phi$. This means that if $f_1, f_2 \in C_c^+(G)$, $\varepsilon > 0$, and $V$ is arbitrary, there is $\phi$ with $\text{supp}(\phi) \subset V$, and
    %
    \[ |I(f_1) - I_\phi(f_1)| < \varepsilon\ \ \ |I(f_2) - I_\phi(f_2)| < \varepsilon \]
    \[ |I(f_1 + f_2) - I_\phi(f_1 + f_2)| < \varepsilon \]
    %
    this implies that if $V$ is chosen small enough, then
    %
    \[ |I(f_1 + f_2) - (I(f_1) - I(f_2))| \leq 2 \varepsilon + |I_\phi(f_1 + f_2) - (I_\phi(f_1) + I_\phi(f_2))| < 3 \varepsilon \]
    %
    Taking $\varepsilon \to 0$, we conclude $I$ is linear. Similar limiting arguments show that $I$ is homogenous of degree 1, and commutes with all left translations. We conclude the extension of $I$ to a linear functional on $C_0(G)$ is well defined, and the Radon measure obtained by the Riesz representation theorem is a Haar measure.
\end{proof}

We shall prove that the Haar measure is unique, but first we show an incredibly useful regularity property.

\begin{prop}
    If $U$ is open, and $\mu$ is a Haar measure, then $\mu(U) > 0$. It follows that if $f$ is in $C_c^+(G)$, then $\int f d \mu > 0$.
\end{prop}
\begin{proof}
    If $\mu(U) = 0$, then for any $x_1, \dots, x_n \in G$,
    %
    \[ \mu \left( \bigcup_{i = 1}^n x_i U \right) \leq \sum_{i = 1}^n \mu(x_i U) = 0 \]
    %
    If $K$ is compact, then $K$ can be covered by finitely many translates of $U$, so $\mu(K) = 0$. But then $\mu = 0$ by regularity, a contradiction.
\end{proof}

\begin{theorem}
    Haar measures are unique up to a multiplicative constant.
\end{theorem}
\begin{proof}
    Let $\mu$ and $\nu$ be Haar measures. Fix a compact neighbourhood $V$ of the identity. If $f,g \in C_c^+(G)$, consider the compact sets
    %
    \[ A = \text{supp}(f) V \cup V \text{supp}(f)\ \ \ \ \ B = \text{supp}(g) V \cup V \text{supp}(g) \]
    %
    Then the functions $F_y(x) = f(xy) - f(yx)$ and $G_y(x) = g(xy) - g(yx)$ are supported on $A$ and $B$. There is a neighbourhood $W \subset V$ of the identity such that $\| F_y \|_\infty, \| G_y \|_\infty < \varepsilon$ if $y \in W$. Now find $h \in C_c^+(G)$ with $h(x) = h(x^{-1})$ and $\text{supp}(h) \subset W$ (take $h(x) = k(x) k(x^{-1})$ for some function $k \in C^+_c(G)$ with $\text{supp}(k) \subset W$, and $k = 1$ on a symmetric neighbourhood of the origin). Then
    %
    \begin{align*}
        \left( \int h d\mu \right) \left( \int f d\lambda \right) &= \int h(y) f(x) d\mu(y) d\lambda(x)\\
        &= \int h(y) f(yx) d\mu(y) d\lambda(x)
    \end{align*}
    %
    and
    %
    \begin{align*}
        \left( \int h d\lambda \right) \left( \int f d\mu \right) &= \int h(x) f(y) d\mu(y) d\lambda(x)\\
        &= \int h(y^{-1}x) f(y) d\mu(y) d\lambda(x)\\
        &= \int h(x^{-1}y) f(y) d\mu(y) d\lambda(x)\\
        &= \int h(y) f(xy) d\mu(y) d\lambda(x)
    \end{align*}
    %
    Hence, applying Fubini's theorem,
    %
    \begin{align*}
        \left| \int h d\mu \int f d\lambda - \int h d\lambda \int f d\mu \right| &\leq \int h(y) |F_y(x)| d\mu(y) d\lambda(x)\\
        &\leq \varepsilon \lambda(A) \int h d\mu
    \end{align*}
    %
    In the same way, we find this is also true when $f$ is swapped with $g$, and $A$ with $B$. Dividing this inequalities by $\int h d\mu \int f d\mu$, we find
    %
    \[  \left| \frac{\int f d\lambda}{\int f d\mu} - \frac{\int h d\lambda}{\int h d\mu} \right| \leq \frac{\varepsilon \lambda(A)}{\int f d\mu} \]
    %
    and this inequality holds with $f$ swapped out with $g$, $A$ with $B$. We then combine these inequalities to conclude
    %
    \[ \left| \frac{\int f d\lambda}{\int f d\mu} - \frac{\int g d\lambda}{\int g d\mu} \right| \leq \varepsilon \left[ \frac{\lambda(A)}{\int f d\mu} + \frac{\lambda(B)}{\int g d\mu} \right] \]
    %
    Taking $\varepsilon$ to zero, we find $\lambda(A), \lambda(B)$ remain bounded, and hence
    %
    \[ \frac{\int f d\lambda}{\int f d\mu} = \frac{\int g d\lambda}{\int g d\mu} \]
    %
    Thus there is a cosntant $c > 0$ such that $\int f d\lambda = c \int f d\mu$ for any function $f \in C_c^+(G)$, and we conclude that $\lambda = c \mu$.
\end{proof}

The theorem can also be proven by looking at the translation invariant properties of the derivative $f = d\mu/d\nu$, where $\nu = \mu + \lambda$ (We assume our group is $\sigma$ compact for now). Consider the function $g(x) = f(yx)$. Then
%
\[ \int_A g(x) d\nu = \int_{yA} f(x) d\nu = \mu(yA) = \mu(A) \]
%
so $g$ is derivative, and thus $f = g$ almost everywhere. Our interpretation is that for a fixed $y$, $f(yx) = f(x)$ almost everywhere with respect to $\nu$. Then (applying a discrete version of Fubini's theorem), we find that for almost all $x$ with respect to $\nu$, $f(yx) = f(x)$ holds for almost all $y$. But this implies that there exists an $x$ for which $f(yx) = f(x)$ holds almost everywhere. Thus for any measurable $A$,
%
\[ \mu(A) = \int_A f(y) d\nu(y) = f(x) \nu(A) = f(x) \mu(A) + f(x) \nu(A) \]
%
Now $(1 - f(x)) \mu(A) = f(x) \nu(A)$ for all $A$, implying (since $\mu, \nu \neq 0$), that $f(x) \neq 0,1$, and so
%
\[ \frac{1-f(x)}{f(x)} \mu(A) = \nu(A) \]
%
for all $A$. This shows the uniqueness property for all $\sigma$ compact groups. If $G$ is an arbitrary group with two measures $\mu$ and $\nu$, then there is $c$ such that $\mu = c \nu$ on every component of $G$, and thus on the union of countably many components. If $A$ intersects uncountably many components, then either $\mu(A) = \nu(A) = \infty$, or the intersection of $A$ on each set has positive measure on only countably many components, and in either case we have $\mu(A) = \nu(A)$.

\section{Fubini, Radon Nikodym, and Duality}

Before we continue, we briefly mention that integration theory is particularly nice over locally compact groups, even if we do not have $\sigma$ finiteness. This essentially follows because the component of the identity in $G$ is $\sigma$ compact (take a compact neighbourhood and its iterated multiples), hence all components in $G$ are $\sigma$ compact. The three theorems that break down outside of the $\sigma$ compact domain are Fubini's theorem, the Radon Nikodym theory, and the duality between $L^1(X)$ and $L^\infty(X)$. We show here that all three hold if $X$ is a locally compact topological group.

First, suppose that $f \in L^1(G \times G)$. Then the essential support of $f$ is contained within countably many components of $G \times G$ (which are simply products of components in $G$). Thus $f$ is supported on a $\sigma$ compact subset of $G \times G$ (as a locally compact topological group, each component of $G \times G$ is $\sigma$ compact), and we may apply Fubini's theorem on the countably many components (the countable union of $\sigma$ compact sets is $\sigma$ compact). The functions in $L^p(G)$, for $1 \leq p < \infty$, also vanish outside of a $\sigma$ compact subset (for if $f \in L^p(G)$, $|f|^p \in L^1(G)$ and thus vanishes outside of a $\sigma$ compact set). What's more, all finite sums and products of functions from these sets (in either variable) vanish outside of $\sigma$ compact subsets, so we almost never need to explicitly check the conditions for satisfying Fubini's theorem, and from now on we apply it wantonly.s

Now suppose $\mu$ and $\nu$ are both Radon measures, with $\nu \ll \mu$, and $\nu$ is $\sigma$-finite. By inner regularity, the support of $\nu$ is a $\sigma$ compact set $E$. By inner regularity, $\mu$ restricted to $E$ is $\sigma$ finite, and so we may find a Radon Nikodym derivative on $E$. This derivative can be extended to all of $G$ because $\nu$ vanishes on $G$.

Finally, we note that $L^\infty(X) = L^1(X)^*$ can be made to hold if $X$ is not $\sigma$ finite, but locally compact and Hausdorff, provided we are integrating with respect to a Radon measure $\mu$, and we modify $L^\infty(G)$ slightly. Call a set $E \subset X$ {\bf locally Borel} if $E \cap F$ is Borel whenever $F$ is Borel and $\mu(F) < \infty$. A locally Borel set is {\bf locally null} if $\mu(E \cap F) = 0$ whenever $\mu(F) < \infty$ and $F$ is Borel. We say a property holds {\bf locally almost everywhere} if it is true except on a locally null set. $f: X \to \mathbf{C}$ is {\bf locally measurable} if $f^{-1}(U)$ is locally Borel for every borel set $U \subset \mathbf{C}$. We now define $L^\infty(X)$ to be the space of all functions bounded except on a locally null set, modulo functions that are locally zero. That is, we define a norm
%
\[ \| f \|_\infty = \inf \{ c : |f(x)| \leq c\ \text{locally almost everywhere} \} \]
%
and then $L^\infty(X)$ consists of the functions that have finite norm. It then follows that if $f \in L^\infty(X)$ and $g \in L^1(X)$, then $g$ vanishes outside of a $\sigma$-finite set $Y$, so $fg \in L^1(X)$, and if we let $Y_1 \subset Y_2 \subset \dots \to Y$ be an increasing subsequence such that $\mu(Y_i) < \infty$, then $|f(x)| \leq \| f \|_\infty$ almost everywhere for $x \in Y_i$, and so by the monotone convergence theorem
%
\[ \int |fg| d\mu = \lim_{Y_i \to \infty} \int_{Y_i} |fg| d\mu \leq \| f \|_\infty \int_{Y_i} |g| d\mu \leq \| f \|_\infty \| g \|_1 \]
%
Thus the map $g \mapsto \int fg d\mu$ is a well defined, continuous linear functional with norm $\| f \|_\infty$. That $L^1(X)^* = L^\infty(X)$ follows from the decomposibility of the Carath\'{e}odory extension of $\mu$, a fact we leave to the general measure theorists.

\section{Unimodularity}

We have thus defined a left invariant measure, but make sure to note that such a function is not right invariant. We call a group who's left Haar measure is also right invariant {\bf unimodular}. Obviously all abelian groups are unimodular.

Given a fixed $y$, the measure $\mu_y(A) = \mu(Ay)$ is a new Haar measure on the space, hence there is a constant $\Delta(y) > 0$ depending only on $y$ such that $\mu(Ay) = \Delta(y) \mu(A)$ for all measurable $A$. Since $\mu(Axy) = \Delta(y) \mu(Ay) = \Delta(x) \Delta(y) \mu(A)$, we find that $\Delta(xy) = \Delta(x) \Delta(y)$, so $\Delta$ is a homomorphism from $G$ to the multiplicative group of real numbers. For any $f \in L^1(\mu)$, we have
%
\[ \int f(xy) d\mu(x) = \Delta(y^{-1}) \int f(x) d\mu(x)  \]
%
If $y_i \to e$, and $f \in C_c(G)$, then $\| R_{y_i} f - f \|_\infty \to 0$, so
%
\[ \Delta(y_i^{-1}) \int f(x) d\mu = \int f(xy_i) d\mu \to \int f(x) d\mu \]
%
Hence $\Delta(y_i^{-1}) \to 1$. This implies $\Delta$, known as the unimodular function, is a continuous homomorphism from $G$ to the real numbers. Note that $\Delta$ is trivial if and only if $G$ is unimodular.

\begin{theorem}
    Any compact group is unimodular.
\end{theorem}
\begin{proof}
    $\Delta: G \to \RR^*$ is a continuous homomorphism, hence $\Delta(G)$ is compact. But the only compact subgroup of $\RR$ is trivial, hence $\Delta$ is trivial.
\end{proof}

Let $G^c$ be the smallest closed subgroup of $G$ containing the commutators $[x,y] = xyx^{-1}y^{-1}$. It is verified to be a normal subgroup of $G$ by simple algebras.

\begin{theorem}
    If $G/G^c$ is compact, then $G$ is unimodular.
\end{theorem}
\begin{proof}
    $\Delta$ factors through $G/G^c$ since it is abelian. But if $\Delta$ is trivial on $G/G^c$, it must also be trivial on $G$.
\end{proof}

The modular function relates right multiplication to left multiplcation in the group. In particular, if $d \mu$ is a Left Haar measure, then $\Delta^{-1} d\mu$ is a right Haar measure. Hence any right Haar measure is a constant multiple of $\Delta^{-1} d\mu$. Hence the measure $\nu(A) = \mu(A^{-1})$ has a value $c$ such that for any function $f$,
%
\[ \int \frac{f(x)}{\Delta(x)} d\mu(x) = c \int f(x) d\nu(x) = c \int f(x^{-1}) d\mu \]
%
If $c \neq 1$, pick a symmetric neighbourhood $U$ such that for $x \in U$, $|\Delta(x) - 1| \leq \varepsilon |c - 1|$. Then if $f > 0$
%
\[ |c-1|\mu(U) = |c\mu(U^{-1}) - \mu(U)| = \left| \int_U [\Delta(x^{-1}) - 1] d\mu(x) \right| \leq \varepsilon \mu(U) |c-1| \]
%
A contradiction if $\varepsilon < 1$. Thus we have
%
\[ \int f(x^{-1}) d\mu(x) = \int \frac{f(x)}{\Delta(x)} d\mu(x) \]
%
A useful integration trick. When $\Delta$ is unbounded, then it follows that $L^p(\mu)$ and $L^p(\nu)$ do not consist of the same functions. There are two ways of mapping the sets isomorphically onto one another -- the map $f(x) \mapsto f(x^{-1})$, and the map $f(x) \mapsto \Delta(x)^{1/p} f(x)$.

From now on, we assume a left invariant Haar measure is fixed over an entire group. Since a Haar measure is uniquely determined up to a constant, this is no loss of generality, and we might as well denote our integration factors $d\mu(x)$ and $d\mu(y)$ as $dx$ and $dy$, where it is assumed that this integration is over the Lebesgue measure.

\section{Convolution}

If $G$ is a topological group, then $C(G)$ does not contain enough algebraic structure to identify $G$ -- for instance, if $G$ is a discrete group, then $C(G)$ is defined solely by the cardinality of $G$. The algebras we wish to study over $G$ is the space $M(G)$ of all complex valued Radon measures over $G$ and the space $L^1(G)$ of integrable functions with respect to the Haar measure, because here we can place a Banach algebra structure with an involution. We note that $L^1(G)$ can be isometrically identified as the space of all measures $\mu \in M(G)$ which are absolutely continuous with respect to the Haar measure. Given $\mu, \nu \in M(G)$, we define the convolution measure
%
\[ \int \phi d(\mu * \nu) = \int \phi(xy) d\mu(x) d\nu(y) \]
%
The measure is well defined, for if $\phi \in C_c^+(X)$ is supported on a compact set $K$, then
%
\begin{align*}
    \left| \int \phi(xy) d\mu(x) d\nu(y) \right| &\leq \int_G \int_G \phi(xy) d|\mu|(x) d|\nu|(y)\\
    &\leq \| \mu \| \| \nu \| \| \phi \|_\infty
\end{align*}
%
This defines an operation on $M(G)$ which is associative, since, by applying the associativity of $G$ and Fubini's theorem.
%
\begin{align*}
    \int \phi d((\mu * \nu) * \lambda) &= \int \int \phi(xz) d(\mu * \nu)(x) d\lambda(z)\\
    &= \int \int \int \phi((xy)z) d\mu(x) d\nu(y) d\lambda(z)\\
    &= \int \int \int \phi(x(yz)) d\mu(x) d\nu(y) d\lambda(z)\\
    &= \int \int \phi(xz) d\mu(x) d(\nu * \lambda)(z)\\
    &= \int \phi d(\mu * (\nu * \lambda))
\end{align*}
%
Thus we begin to see how the structure of $G$ gives us structure on $M(G)$. Another example is that convolution is commutative if and only if $G$ is commutative. We have the estimate $\| \mu * \nu \| \leq \| \mu \| \| \nu \|$, because of the bound we placed on the integrals above. $M(G)$ is therefore an involutive Banach algebra, which has a unit, the dirac delta measure at the identity.

As a remark, we note that involutive Banach algebras have nowhere as near a nice of a theory than that of $C^*$ algebras. $M(G)$ cannot be renormed to be a $C^*$ algebra, since every weakly convergent Cauchy sequence converges, which is impossible in a $C^*$ algebra, except in the finite dimensional case.

A {\bf discrete measure} on $G$ is a measure in $M(G)$ which vanishes outside a countable set of points, and the set of all such measures is denoted $M_d(G)$. A {\bf continuous measure} on $G$ is a measure $\mu$ such that $\mu(\{x\}) = 0$ for all $x \in G$. We then have a decomposition $M(G) = M_d(G) \oplus M_c(G)$, for if $\mu$ is any measure, then $\mu(\{x\}) \neq 0$ for at most countably many points $x$, for
%
\[ \| \mu \| \geq \sum_{x \in G} |\mu|(x) \]
%
This gives rise to a discrete measure $\nu$, and $\mu - \nu$ is continuous. If we had another decomposition, $\mu = \psi + \phi$, then $\mu(\{x\}) = \psi(\{x\}) = \nu(\{x\})$, so $\psi = \nu$ by discreteness, and we then conclude $\phi = \mu - \nu$. $M_c(G)$ is actually a closed subspace of $M(G)$, since if $\mu_i \to \mu$, and $\mu_i \in M_c(G)$, and $\| \mu_i - \mu \| < \varepsilon$, then for any $x \in G$,
%
\[ \varepsilon > \| \mu - \mu_i \| \geq |(\mu_i - \mu)(\{x\})| = |\mu(\{ x \})| \]
%
Letting $\varepsilon \to 0$ shows continuity.

The convolution on $M(G)$ gives rise to a convolution on $L^1(G)$, where
%
\[ (f*g)(x) = \int f(y) g(y^{-1}x) dy \]
%
which satisfies $\| f*g \|_1 \leq \| f \|_1 \| g \|_1$. This is induced by the identification of $f$ with $f(x) dx$, because then
%
\begin{align*}
    \int \phi (f(x) dx * g(x) dx) &= \int \int \phi(yx) f(y) g(x) dy dx\\
    &= \int \phi(y) \left( \int f(y) g(y^{-1}x) dx \right) dy
\end{align*}
%
Hence $f d\mu * g d\mu = (f * g) d\mu$. What's more,
%
\[ \| f \|_1 = \| f d\mu \| \]
%
If $\nu \in M(G)$, then we can still define $\nu * f \in L^1(G)$
%
\[ (\nu * f)(x) = \int f(y^{-1}x) d\mu(y) \]
%
which holds since
%
\[ \int \phi d(\nu * f \mu) = \int \phi(yx) f(x) d\nu(y) d\mu(x) = \int \phi(x) f(y^{-1}x) d\nu(y) d\mu(x) \]
%
If $G$ is unimodular, then we also find
%
\[ \int \phi d(f \mu * \nu) = \int \phi(yx) f(y) d\mu(y) d\nu(x) = \int \phi(x) f(y) d\mu(y) d\nu(y^{-1}x) \]
%
So we let $f * \mu(x) = \int f(y) d\mu(y^{-1}x)$.

\begin{theorem}
    $L^1(G)$ and $M_c(G)$ are closed ideals in $M(G)$, and $M_d(G)$ is a closed subalgebra.
\end{theorem}
\begin{proof}
    If $\mu_i \to \mu$, and each $\mu_i$ is discrete, the $\mu$ is discrete, because there is a countable set $K$ such that all $\mu_i$ are equal to zero outside of $K$, so $\mu$ must also vanish outside of $K$ (here we have used the fact that $M(G)$ is a Banach space, so that we need only consider sequences). Thus $M_d(G)$ is closed, and is easily verified to be subalgebra, essentially because $\delta_x * \delta_y = \delta_{xy}$. If $\mu_i \to \mu$, then $\mu_i(\{x\}) \to \mu(\{x\})$, so that $M_c(G)$ is closed in $M(G)$. If $\nu$ is an arbitrary measure, and $\mu$ is continuous, then
    %
    \[ (\mu * \nu)(\{ x \}) = \int_G \mu(\{ y \}) d\nu(y^{-1}x) = 0 \]
    \[ (\nu * \mu)(\{ x \}) = \int_G \mu(\{ y \}) d\nu(xy^{-1}) = 0 \]
    %
    so $M_c(G)$ is an ideal. Finally, we verify $L^1(G)$ is closed, because it is complete, and if $\nu \in M(G)$ is arbitrary, and if $U$ has null Haar measure, then
    %
    \[ (f dx * \nu)(U) = \int \chi_{U}(xy) f(x) dx\ d\nu(y) = \int_G \int_{y^{-1}U} f(x) dx d\nu(y) = 0 \]
    \[ (\nu * f dx)(U) = \int \chi_U(xy) d\nu(x) f(y) dy = \int_G \int_{Ux^{-1}} f(y) dy d\nu(x) = 0 \]
    %
    So $L^1(G)$ is a two-sided ideal.
\end{proof}

If we wish to integrate by right multiplication instead of left multiplication, we find by the substitution $y \mapsto xy$ that
%
\begin{align*}
    (f*g)(x) &= \int f(y) g(y^{-1}x) dy\\
    &= \int \int f(xy) g(y^{-1}) dy\\
    &= \int \int \frac{f(xy^{-1}) g(y)}{\Delta(y)} dy
\end{align*}
%
Observe that
%
\[ f*g = \int f(y) L_{y^{-1}} g\ dy \]
%
which can be interpreted as a vector valued integral, since for $\phi \in L^\infty(\mu)$,
%
\[ \int (f*g)(x) \phi(x) dx = \int f(y) g(y^{-1}x) \phi(x) dx dy \]
%
so we can see convolution as a generalized `averaging' of translate of $g$ with respect to the values of $f$. If $G$ is commutative, this is the same as the averaging of translates of $f$, but not in the noncommutative case. It then easily follows from operator computations $L_z (f*g) = (L_z f) * g$, and $R_z (f*g) = f * (R_zg)$, or from the fact that
%
\[ (f*g)(zx) = \int f(y) g(y^{-1}zx) dy = \int f(zy) g(y^{-1}x) dy = [(L_z f) * g](x) \]
\[ (f*g)(xz) = \int f(y) g(y^{-1}xz) dy = [f * (R_z g)](x) \]
%
Convolution can also be applied to the other $L^p$ spaces, but we have to be a bit more careful with our integration.

\begin{theorem}
    If $f \in L^1(G)$ and $g \in L^p(G)$, then $f*g$ is defined for almost all $x$, $f*g \in L^p(G)$, and $\| f*g\|_p \leq \|f \| \| g \|_p$. If $G$ is unimodular, then the same results hold for $g*f$, or if $G$ is not unimodular and $f$ has compact support.
\end{theorem}
\begin{proof}
    We use Minkowski's inequality to find
    %
    \begin{align*}
        \| f*g \|_p &= \left( \int \left| \int f(y) |g(y^{-1}x) dy \right|^{p} dx \right)^{1/p}\\
        &\leq \int |f(y)| \left( \int |g(y^{-1}x)|^p dx \right)^{1/p} dy\\
        &= \| f \|_1 \| g \|_p
    \end{align*}

    If $G$ is unimodular, then
    %
    \[ \| g*f \|_p = \left( \int \left| \int g(xy^{-1}) f(y) dy \right|^{p} dx \right)^{1/p} \]
    %
    and we may apply the same trick as used before.

    If $f$ has compact support $K$, then $1/\Delta$ is bounded above by $M > 0$ on $K$ and
    %
    \begin{align*}
        \| g * f \|_p &= \left( \int \left| \int \frac{ g(xy^{-1}) f(y)}{\Delta(y)} dy \right|^{p} dx \right)^{1/p}\\
        &\leq \int \left( \int \left| \frac{g(xy^{-1}) f(y)}{\Delta(y)} \right|^p dx \right)^{1/p} dy\\
        &= \| g \|_p \int_K \frac{|f(y)|}{\Delta(y)} d \mu(y)\\
        &\leq M \| g \|_p \| f \|_1
    \end{align*}
    %
    which shows that $g*f$ is defined almost everywhere.
\end{proof}

\begin{theorem}
    If $G$ is unimodular, $f \in L^p(G)$, $g \in L^q(G)$, and $p = q^*$, then $f*g \in C_0(G)$ and $\| f * g \|_\infty \leq \| f \|_p \| g \|_q$.
\end{theorem}
\begin{proof}
    First, note that
    %
    \begin{align*}
        |(f*g)(x)| &\leq \int |f(y)| |g(y^{-1}x)| dy\\
        &\leq \| f \|_p \left( \int |g(y^{-1}x)|^q dy \right)^{1/q}\\
        &= \| f \|_p \| g \|_q
    \end{align*}
    %
    For each $x$ and $y$, applying H\"{o}lder's inequality, we find
    %
    \begin{align*}
        |(f*g)(x) - (f*g)(y)| &\leq \int |f(z)| |g(z^{-1}x) - g(z^{-1}y)| dz\\
        &\leq \| f \|_p \left( \int |g(z^{-1}x) - g(z^{-1}y)|^q dz \right)^{1/q}\\
        &= \| f \|_p \left( \int |g(z) - g(zx^{-1}y)|^q dz \right)^{1/q}\\
        &= \| f \|_p \| g - R_{x^{-1}y} g \|_q
    \end{align*}
    %
    Thus to prove continuity (and in fact uniform continuity), we need only prove that $\| g - R_x g \|_q \to 0$ for $q \neq \infty$ as $x \to \infty$ or $x \to 0$. This is the content of the next lemma.
\end{proof}

We now show that the map $x \mapsto L_x$ is a continuous operation from $G$ to the weak $*$ topology on the $L_p$ spaces, for $p \neq \infty$. It is easily verified that translation is not continuous on $L_\infty$, by taking a suitable bumpy function.

\begin{theorem}
    If $p \neq \infty$, then $\| g - R_x g \|_p \to 0$ and $\| g - L_x g \|_p \to 0$ as $x \to 0$.
\end{theorem}
\begin{proof}
    If $g \in C_c(G)$, then one verifies the theorem by using left and right uniform continuity. In general, we let $g_i \in C_c(G)$ be a sequence of functions converging to $g$ in the $L_p$ norm, and we then find
    %
    \[ \| g - L_x g \|_p \leq \| g - g_i \|_p + \| g_i - L_x g_i \|_p + \| L_x (g_i - g) \|_p = 2 \| g - g_i \|_p + \| g_i - L_x g_i \|_p \]
    %
    Taking $i$ large enough, $x$ small enough, we find $\| g - L_x g \|_p \to 0$. The only problem for right translation is the appearance of the modular function
    %
    \begin{align*}
        \| R_x (g - g_i) \|_p = \frac{\| g - g_i \|_p}{\Delta(x)^{1/p}}
    \end{align*}
    %
    If we assume our $x$ values range only over a compact neighbourhood $K$ of the origin, we find that $\Delta(x)$ is bounded below, and hence $\| R_x (g - g_i) \|_p \to 0$, which effectively removes the problems in the proof.
\end{proof}

Since the map is linear, we have verified that the map $x \mapsto L_x f$ is uniformly continuous in $L^p$ for each $f \in L^p$. In the case where $p = \infty$, the same theorem cannot hold, but we have even better conditions that do not even require unimodularity.

\begin{theorem}
    If $f \in L^1(G)$ and $g \in L^\infty(G)$, then $f*g$ is left uniformly continuous, and $g*f$ is right uniformly continuous.
\end{theorem}
\begin{proof}
    We have
    %
    \[ \| L_z (f*g) - (f*g) \|_\infty = \| (L_z f - f) * g \|_\infty \leq \| L_z f - f \|_1 \| g \|_\infty \]
    %
    \[ \| R_z (g*f) - (g*f) \|_\infty = \| g * (R_z f - f) \|_\infty \leq \| g \|_\infty \| R_z f - f \|_1 \]
    %
    and both integrals converge to zero as $z \to 1$.
\end{proof}

The passage from $M(G)$ to $L^1(G)$ removes an identity from the Banach algebra in question (except if $G$ is discrete), but there is always a way to approximate an identity.

\begin{theorem}
    For each neighbourhood $U$ of the origin, pick a function $f_U \in (L^1)^+(G)$, with $\int \phi_U = 1$, $\text{supp}(f_U) \subset U$. Then if $g$ is any function in $L^p(G)$,
    %
    \[ \| f_U * g - g \|_p \to 0 \]
    %
    where we assume $g$ is left uniformly continuous if $p = \infty$, and if $f_U$ is viewed as a net with neighbourhoods ordered by inclusion. If in addition $f_U(x) = f_U(x^{-1})$, then $\| g * f_U - g \|_p \to 0$, where $g$ is right uniformly continuous for $p = \infty$.
\end{theorem}
\begin{proof}
    Let us first prove the theorem for $p \neq \infty$. If $g \in C_c(G)$ is supported on a compact $K$, and if $U$ is small enough that $|g(y^{-1}x) - g(x)| < \varepsilon$ for $y \in U$, then because $\int_U f_U(y) = 1$, and by applying Minkowski's inequality, we find
    %
    \begin{align*}
        \| f_U * g - g \|_p &= \left( \int \left| \int f_U(y) [g(y^{-1}x) - g(x)] dy \right|^p dx \right)^{1/p} \\
        &\leq \int f_U(y) \left( \int |g(y^{-1}x) - g(x)|^p dx \right)^{1/p} dy\\
        &\leq 2 \mu(K)\varepsilon \int f_U(y) dy \leq 2 \mu(K)\varepsilon
    \end{align*}
    %
    Results are then found for all of $L^p$ by taking limits. If $g$ is left uniformly continuous, then we may find $U$ such that $|g(y^{-1}x) - g(x)| < \varepsilon$ for $y \in U$ then
    %
    \[ |(f_U * g - g)(x)| = \left| \int f_U(y) [g(y^{-1}x) - g(x)] \right| \leq \varepsilon \]
    %
    For right convolution, we find that for $g \in C_c(G)$, where $|g(xy) - g(x)| < \varepsilon$ for $y \in U$, then
    %
    \begin{align*}
        \| g * f_U - g \|_p &= \left( \int \left| \int g(y) f_U(y^{-1}x) - g(x) dy \right|^p dx \right)^{1/p}\\
        &= \left( \int \left| \int [g(xy) - g(x)] f_U(y) dy \right|^p dx \right)^{1/p}\\
        &\leq \int \left( \int |g(xy) - g(x)|^p dx \right)^{1/p} f_U(y) dy\\
        &\leq \mu(K) \varepsilon \int f_U(y) (1 + \Delta(y)) dy\\
        &= \mu(K) \varepsilon + \mu(K) \varepsilon \int f_U(y) \Delta(y) dy
    \end{align*}
    %
    We may always choose $U$ small enough that $\Delta(y) < \varepsilon$ for $y \in U$, so we obtain a complete estimate $\mu(K) (\varepsilon + \varepsilon^2)$. If $g$ is right uniformly continuous, then choosing $U$ for which $|g(xy) - g(x)| < \varepsilon$, then
    %
    \[ |(g * f_U - g)(x)| = \left| \int [g(xy) - g(x)] f_U(y) dy \right| \leq \varepsilon \]
    %
    We will always assume from hereon out that the approximate identities in $L^1(G)$ are of this form.
\end{proof}

We have already obtained enough information to characterize the closed ideals of $L^1(G)$.

\begin{theorem}
    If $V$ is a closed subspace of $L^1(G)$, then $V$ is a left ideal if and only if it is closed under left translations, and a right ideal if and only if it is closed under right translations.
\end{theorem}
\begin{proof}
    If $V$ is a closed left ideal, and $f_U$ is an approximate identity at the origin, then for any $g$,
    %
    \[ \| (L_z f_U) * g - L_z g \|_1 = \| L_z (f_U * g - g) \|_1 = \| f_U * g - g \| \to 0 \]
    %
    so $L_z g \in V$. Conversely, if $V$ is closed under left translations, $g \in L^1(G)$, and $f \in V$, then
    %
    \[ g * f = \int g(y) L_{y^{-1}} f dy \]
    %
    which is in the closed linear space of the translates of $f$. Right translation is verified very similarily.
\end{proof}

\section{The Riesz Thorin Theorem}

We finalize our basic discussion by looking at convolutions of functions in $L^p * L^q$. Certainly $L^p * L^1 \subset L^p$, and $L^p * L^q \subset L^\infty$ for $q = p^*$. To prove general results, we require a foundational interpolation result.
%
\begin{theorem}
    For any $0 < \theta < 1$, and $0 < p,q \leq \infty$. If we define
    %
    \[ 1/r_\theta = (1-\theta)/p + \theta/q \]
    %
    to be the inverse interpolation of the two numbers. Then
    %
    \[ \| f \|_{r_\theta} \leq \| f \|_p^{1-\theta} \| f \|_q^\theta \]
\end{theorem}
\begin{proof}
    We apply H\"{o}lder's inequality to find
    %
    \[ \| f \|_{r_\theta} \leq \| f \|_{p/(1 - \theta)} \| f \|_{q/\theta} = \left( \int |f|^{p/(1 - \theta)} \right)^{(1-  \theta)/p} \left( \int |f|^{q/\theta} \right)^{\theta/q} \]
    %
    so it suffices to prove $\| f \|_{p/(1-\theta)} \leq \| f \|_p^{1-\theta}$, $\| f \|_{q/\theta} \leq \| f \|_q^\theta$.

    The map $x \mapsto x^p$ is concave for $0 < p < 1$, so we may apply Jensen's inequality in reverse to conclude
    %
    \[ \left( \int |f|^{p/(1 - \theta)} \right)^{(1-  \theta)/p} \leq \left( \int |f|^p \right)^{1/p} \]
\end{proof}

The Riesz Thorin interpolation theorem then implies $L^p * L^q \subset L^r$, for $p^{-1} + q^{-1} = 1 + r^{-1}$. However, these estimates only guarantee $L^1(G)$ is closed under convolution. If $G$ is compact, then $L_p(G)$ is closed under convolution for all $p$ (TODO). The $L_p$ conjecture says that this is true if and only if $G$ is compact. This was only resolved in 1990.

\section{Homogenous Spaces and Haar Measures}

The natural way for a locally compact topological group $G$ to act on a locally compact Hausdorff space $X$ is via a representation of $G$ in the homeomorphisms of $X$. We assume the action is transitive on $X$. The standard example are the action of $G$ on $G/H$, where $H$ is a closed subspace. These are effectively all examples, because if we fix $x \in X$, then the map $y \mapsto yx$ induces a continuous bijection from $G/H$ to $X$, where $H$ is the set of all $y$ for which $yx = x$. If $G$ is a $\sigma$ compact space, then this map is a homeomorphism.

\begin{theorem}
    If a $\sigma$ compact topological group $G$ has a transitive topological action on $X$, and $x \in X$, then the continuous bijection from $G/G_x$ to $X$ is a homeomorphism.
\end{theorem}
\begin{proof}
    It suffices to show that the map $\phi: G \to X$ is open, and we need only verify this for the neighbourhood basis of compact neighbourhoods $V$ of the origin by properties of the action. $G$ is covered by countably many translates $y_1V, y_2V, \dots$, and since each $\phi(y_kV) = y_k \phi(V)$ is closed (compactness), we conclude that $y_k \phi(V)$ has non-empty interior for some $y_k$, and hence $\phi(V)$ has a non-empty interior point $\phi(y_0)$. But then for any $y \in V$, $y$ is in the interior of $\phi(y V y_0^{-1}) \subset \phi(VV y_0^{-1})$, so if we fix a compact $U$, and find $V$ with $V^3 \subset U$, we have shown $\phi(U)$ is open in $X$.
\end{proof}

We shall say a space $X$ is homogenous if it is homeomorphic to $G/H$ for some group action of $G$ over $X$. The $H$ depends on our choice of basepoint $x$, but only up to conjugation, for if if we switch to a new basepoint $y$, and $c$ maps $x$ to $y$, then $ax = x$ holds if and only if $cac^{-1}y = y$. The question here is to determine whether we have a $G$-invariant measure on $X$. This is certainly not always possible. If we had a measure on $\RR$ invariant under the affine maps $ax + b$, then it would be equal to the Haar measure by uniqueness, but the Haar measure is not invariant under dilation $x \mapsto ax$.

Let $G$ and $H$ have left Haar measures $\mu$ and $\nu$ respectively, denote the projection of $G$ onto $G/H$ as $\pi: G \to G/H$, and let $\Delta_G$ and $\Delta_H$ be the respective modular functions. Define a map $P: C_c(G) \to C_c(G/H)$ by
%
\[ (Pf)(Hx) = \int_H f(xy) d\nu(y) = \int_H  \]
%
this is well defined by the invariance properties of $\nu$. $Pf$ is obviously continuous, and $\text{supp}(Pf) \subset \pi(\text{supp}(f))$. Moreover, if $\phi \in C(G/H)$ we have
%
\[ P((\phi \circ \pi) \cdot f)(Hx) = \phi(xH) \int_H f(xy) d\nu(y) \]
%
so $P((\phi \circ \pi) \cdot f) = \phi P(f)$.

\begin{lemma}
    If $E$ is a compact subset of $G/H$, there is a compact $K \subset G$ with $\pi(K) = E$.
\end{lemma}
\begin{proof}
    Let $V$ be a compact neighbourhood of the origin, and cover $E$ by finitely many translates of $\pi(V)$. We conclude that $\pi^{-1}(E)$ is covered by finitely many of the translates, and taking the intersections of these translates with $\pi^{-1}(E)$ gives us the desired $K$.
\end{proof}

\begin{lemma}
    A compact $F \subset G/H$ gives rise to a function $f \geq 0$ in $C_c(G)$ such that $Pf = 1$ on $E$.
\end{lemma}
\begin{proof}
    Let $E$ be a compact neighbourhood containing $F$, and if $\pi(K) = E$, there is a function $g \in C_c(G)$ with $g > 0$ on $K$, and $\phi \in C_c(G/H)$ is supported on $E$ and $\phi(x) = 1$ for $x \in F$, let
    %
    \[ f = \frac{\phi \circ \pi}{P g \circ \pi} g \]
    %
    Hence
    %
    \[ Pf = \frac{\phi}{Pg} Pg = \phi \]
\end{proof}

\begin{lemma}
    If $\phi \in C_c(G/H)$, there is $f \in C_c(G)$ with $Pf = \phi$, and $\pi(\text{supp} f) = \text{supp}(\phi)$, and also $f \geq 0$ if $\phi \geq 0$.
\end{lemma}
\begin{proof}
    There exists $g \geq 0$ in $C_c(G/H)$ with $Pg = 1$ on $\text{supp}(\phi)$, and then $f = (\phi \circ \pi) g$ satisfies the properties of the theorem.
\end{proof}

We can now provide conditions on the existence of a measure on $G/H$.

\begin{theorem}
    There is a $G$ invariant measure $\psi$ on $G/H$ if and only if $\Delta_G = \Delta_H$ when restricted to $H$. In this case, the measure is unique up to a common factor, and if the factor is chosen, we have
    %
    \[ \int_G f d\mu = \int_{G/H} Pf d\psi = \int_{G/H} \int_H f(xy) d\nu(y) d\psi(xH) \]
\end{theorem}
\begin{proof}
    Suppose $\psi$ existed. Then $f \mapsto \int Pf d \psi$ is a non-zero left invariant positive linear functional on $G/H$, so $\int Pf d\psi = c \int f d\mu$ for some $c > 0$. Since $P(C_c(G)) = C_c(G/H)$, we find that $\psi$ is determined up to a constant factor. We then compute, for $y \in H$,
    %
    \begin{align*}
        \Delta_G(y) \int f(x) d\mu(x) &= \int f(xy^{-1}) d\mu(x)\\
        &= \int_{G/H} \int_H f(xzy^{-1}) d\nu(z) d\psi(xH)\\
        &= \Delta_H(y) \int_{G/H} \int_H f(xz) d\nu(z) d\psi(xH)\\
        &= \Delta_H(y) \int f(x) d\mu(x)
    \end{align*}
    %
    Hence $\Delta_G = \Delta_H$. Conversely, suppose $\Delta_G = \Delta_H$. First, we claim if $f \in C_c(G)$ and $Pf = 0$, then $\int f d\mu = 0$. Indeed if $P\phi = 1$ on $\pi(\text{supp} f)$ then
    %
    \[ 0 = Pf(xH) = \int_H f(xy) d\nu(y) = \Delta_G(y^{-1}) \int_H f(xy^{-1}) d\nu(y) \]
    %
    so
    %
    \begin{align*}
        0 &= \int_G \int_H \Delta_G(y^{-1}) \phi(x) f(xy^{-1}) d\nu(y) d\mu(x)\\
        &= \int_H \int_G \phi(xy) f(x) d\mu(x) d\nu(y)\\
        &= \int_G P\phi(xH) f(x) d\mu(x)\\
        &= \int_G f(x) d\mu(x)
    \end{align*}
    %
    This implies that if $Pf = Pg$, then $\int_G f = \int_G g$. Thus the map $Pf \mapsto \int_G f$ is a well defined $G$ invariant positive linear functional on $C_c(G/H)$, and we obtain a Radon measure from the Riesz representation theorem.
\end{proof}

If $H$ is compact, then $\Delta_G$ and $\Delta_H$ are both continuous homomorphisms from $H$ to $\RR^+$, so $\Delta_G$ and $\Delta_H$ are both trivial, and we conclude a $G$ invariant measure exists on $G/H$.

\section{Function Spaces In Harmonic Analysis}

There are a couple other function spaces that are interesting in Harmonic analysis. We define $\text{AP}(G)$ to be the set of all almost periodic functions, functions $f \in L^\infty(G)$ such that $\{ L_x f : x \in G \}$ is relatively compact in $L^\infty(G)$. If this is true, then $\{ R_x f : x \in G \}$ is also relatively compact, a rather deep theorem. If we define $\text{WAP}(G)$ to be the space of weakly almost periodic functions (the translates are relatively compact in the weak topology). It is a deep fact that $\text{WAP}(G)$ contains $C_0(G)$, but $\text{AP}(G)$ can be quite small. The reason these function spaces are almost periodic is that in the real dimensional case, $\text{AP}(\RR)$ is just the closure of the set of all trigonometric polynomials.

\chapter{The Character Space}

Let $G$ be a locally compact group. A character on $G$ is a {\it continuous} homomorphism from $G$ to $\mathbf{T}$. The space of all characters of a group will be denoted $\Gamma(G)$.

\begin{example}
    Determining the characters of $\mathbf{T}$ involves much of classical Fourier analysis. Let $f: \mathbf{T} \to \mathbf{T}$ be an arbitrary continuous character. For each $w \in \mathbf{T}$, consider the function $g(z) = f(zw) = f(z)f(w)$. We know the Fourier series acts nicely under translation, telling us that
    %
    \[ \hat{g}(n) = w^n \hat{f}(n) \]
    %
    Conversely, since $g(z) = f(z)f(w)$,
    %
    \[ \hat{g}(n) = f(w) \hat{f}(n) \]
    %
    Thus $(w^n - f(w)) \hat{f}(n) = 0$ for all $w \in \mathbf{T}$, $n \in \mathbf{Z}$. Fixing $n$, we either have $f(w) = w^n$ for all $w$, or $\hat{f}(n) = 0$. This implies that if $f \neq 0$, then $f$ is just a power map for some $n \in \mathbf{Z}$.
\end{example}

\begin{example}
    The characters of $\RR$ are of the form $t \mapsto e(t\xi)$, for $\xi \in \RR$. To see this, let $e: \RR \to \mathbf{T}$ be an arbitrary character. Define
    %
    \[ F(x) = \int_0^x e(t) dt \]
    %
    Then $F'(x) = e(x)$. Since $e(0) = 1$, for suitably small $\delta$ we have
    %
    \[ F(\delta) = \int_0^\delta e(t) dt = c > 0 \]
    %
    and then it follows that
    %
    \[ F(x + \delta) - F(x) = \int_x^{x + \delta} e(t) dt = \int_0^\delta e(x + t) dt = c e(x) \]
    %
    As a function of $x$, $F$ is differentiable, and by the fundamental theorem of calculus,
    %
    \[ \frac{dF(x + \delta) - F(x)}{dt} = F'(x + \delta) - F'(x) = e(x + \delta) - e(x) \]
    %
    This implies the right side of the above equation is differentiable, and so
    %
    \[ ce'(x) = e(x + \delta) - e(x) = e(x) [e(\delta) - 1] \]
    %
    Implying $e'(x) = A e(x)$ for some $A \in \mathbf{C}$, so $e(x) = e^{Ax}$. We require that $e(x) \in \mathbf{T}$ for all $x$, so $A = \xi i$ for some $\xi \in \RR$.
\end{example}

\begin{example}
    Consider the group $\RR^+$ of positive real numbers under multiplication. The map $x \mapsto \log x$ is an isomorphism from $\RR^+$ and $\RR$, so that every character on $\RR^+$ is of the form $e(s \log x) = x^{is}$, for some $s \in \RR$. The character group is then $\RR$, since $x^{is} x^{is'} = x^{i(s + s')}$.
\end{example}

There is a connection between characters on $G$ and characters on $L^1(G)$ that is invaluable to the generalization of Fourier analysis to arbitrary groups.

\begin{theorem}
    For any character $\phi: G \to \mathbf{C}$, the map
    %
    \[ \varphi(f) = \int \frac{f(x)}{\phi(x)} dx \]
    %
    is a non-zero character on the convolution algebra $L^1(G)$, and all characters arise this way.
\end{theorem}
\begin{proof}
    The induced map is certainly linear, and
    %
    \begin{align*}
        \varphi(f * g) &= \int \int \frac{f(y) g(y^{-1}x)}{\phi(x)} dy dx\\
        &= \int \int \frac{f(y) g(x)}{\phi(y) \phi(x)} dy dx\\
        &= \int \frac{f(y)}{\phi(y)} dy \int \frac{g(x)}{\phi(x)} dx
    \end{align*}
    %
    Since $\phi$ is continuous, there is a compact subset $K$ of $G$ where $\phi > \varepsilon$ for some $\varepsilon > 0$, and we may then choose a positive $f$ supported on $K$ in such a way that $\varphi(f)$ is non-zero.

    The converse results from applying the duality theory of the $L^p$ spaces. Any character on $L^1(G)$ is a linear functional, hence is of the form
    %
    \[ f \mapsto \int f(x) \phi(x) dx \]
    %
    for some $\phi \in L^\infty(G)$. Now
    %
    \begin{align*}
        \int \int f(y) g(x) \phi(yx) dy dx &= \int \int f(y) g(y^{-1}x) \phi(x) dy dx\\
        &= \int f(x) \phi(x) dx \int g(y) \phi(y) dy\\
        &= \int f(x) g(y) \phi(x) \phi(y) dx dy
    \end{align*}
    %
    Since this holds for all functions $f$ and $g$ in $L^1(G)$, we must have $\phi(yx) = \phi(x) \phi(y)$ almost everywhere. Also
    %
    \begin{align*}
        \int \varphi(f) g(y) \phi(y) dy &= \varphi(f * g)\\
        &= \int \int g(y) f(y^{-1}x) \phi(x) dy dx\\
        &= \int \int (L_{y^{-1}} f)(x) g(y) \phi(x) dy dx\\
        &= \int \varphi(L_{y^{-1}} f) g(y) dy
    \end{align*}
    %
    which implies $\varphi(f) \phi(y) = \varphi(L_{y^{-1}} f)$ almost everywhere. Since the map $\varphi(L_{y^{-1}} f)/\varphi(f)$ is a uniformly continuous function of $y$, $\phi$ is continuous almost everywhere, and we might as well assume $\phi$ is continuous. We then conclude $\phi(xy) = \phi(x) \phi(y)$. Since $\| \phi \|_\infty = 1$ (this is the norm of any character operator on $L^1(G)$), we find $\phi$ maps into $\mathbf{T}$, for if $\| \phi(x) \| < 1$ for any particular $x$, $\| \phi(x^{-1}) \| > 1$.
\end{proof}

Thus there is a one-to-one correspondence with $\Gamma(G)$ and $\Gamma(L^1(G))$, which implies a connection with the Gelfand theory and the character theory of locally compact groups. This also gives us a locally compact topological structure on $\Gamma(G)$, induced by the Gelfand representation on $\Gamma(L^1(G))$. A sequence $\phi_i \to \phi$ if and only if
%
\[ \int \frac{f(x)}{\phi_i(x)} dx \to \int \frac{f(x)}{\phi(x)} dx \]
%
for all functions $f \in L^1(G)$. This actually makes the map
%
\[ (f,\phi) \mapsto \int \frac{f(x)}{\phi(x)} dx \]
%
a jointly continuous map, because as we verified in the proof above,
%
\[ \widehat{f}(\phi) \phi(y) = \widehat{L_y f}(\phi) \]
%
And the map $y \mapsto L_y f$ is a continuous map into $L^1(G)$. If $K \subset G$ and $C \subset \Gamma(G)$ are compact, this allows us to find open sets in $G$ and $\Gamma(G)$ of the form
%
\[ \{ \gamma : \| 1 - \gamma(x) \| < \varepsilon\ \text{for all}\ x \in K \}\ \ \ \ \ \{ x : \| 1 - \gamma(x) \| < \varepsilon\ \text{for all}\ \gamma \in C \} \]
%
And these sets actually form a base for the topology on $\Gamma(G)$.

\begin{theorem}
    If $G$ is discrete, $\Gamma(G)$ is compact, and if $G$ is compact, $\Gamma(G)$ is discrete.
\end{theorem}
\begin{proof}
    If $G$ is discrete, then $L^1(G)$ contains an identity, so $\Gamma(G) = \Gamma(L^1(G))$ is compact. Conversely, if $G$ is compact, then it contains the constant $1$ function, and
    %
    \[ \widehat{1}(\phi) = \int \frac{dx}{\phi(x)} \]
    %
    And
    %
    \[ \frac{1}{\phi(y)} \widehat{1}(\phi) = \int \frac{dx}{\phi(yx)} = \int \frac{dx}{\phi(x)} = \hat{1}(\phi) \]
    %
    So either $\phi(y) = 1$ for all $y$, and it is then verified by calculation that $\widehat{1}(\phi) = 1$, or $\widehat{1}(\phi) = 0$. Since $\widehat{1}$ is continuous, the trivial character must be an open set by itself, and hence $\Gamma(G)$ is discrete.
\end{proof}

Given a function $f \in L^1(G)$, we may take the Gelfand transform, obtaining a function on $C_0(\Gamma(L^1(G)))$. The identification then gives us a function on $C_0(\Gamma(G))$, if we give $\Gamma(G)$ the topology induced by the correspondence (which also makes $\Gamma(G)$ into a topological group). The formula is
%
\[ \widehat{f}(\phi) = \phi(f) = \int \frac{f(x)}{\phi(x)} \]
%
This gives us the classical correspondence between $L^1(\mathbf{T})$ and $C_0(\mathbf{Z})$, and $L^1(\RR)$ and $C_0(\RR)$, which is just the Fourier transform. Thus we see the Gelfand representation as a natural generalization of the Fourier transform. We shall also denote the Fourier transform by $\mathcal{F}$, especially when we try and understand it's properties as an operator. Gelfand's theory (and some basic computation) tells us instantly that

\begin{itemize}
    \item $\widehat{f * g} = \widehat{f} \widehat{g}$ (The transform is a homomorphism).
    \item $\mathcal{F}$ is norm decreasing and therefore continuous: $\| \widehat{f} \|_\infty \leq \| f \|_1$.
    \item If $G$ is unimodular, and $\gamma \in \Gamma(G)$, then $(f * \gamma)(x) = \gamma(x) \widehat{f}(\gamma)$.
\end{itemize}

Whenever we integrate a function with respect to the Haar measure, there is a natural generalization of the concept to the space of all measures on $G$. Thus, for $\mu \in M(G)$, we define
%
\[ \widehat{\mu}(\phi) = \int \frac{dx}{\phi(x)} \]
%
which we call the {\bf Fourier-Stieltjes transform} on $G$. It is essentially an extension of the Gelfand representation on $L^1(G)$ to $M(G)$. Each $\widehat{\mu}$ is a bounded, uniformly continuous function on $\Gamma(G)$, because the transform is still contracting, i.e.
%
\[ \left| \int \frac{d\mu(x)}{\phi(x)} dx \right| \leq \| \mu \| \]
%
It is uniformly continuous, because
%
\[ (L_{\nu} \widehat{\mu} - \widehat{\mu})(\phi) = \int \frac{1 - \nu(x)}{\nu(x) \phi(x)} d\mu(x)  \]
%
The regularity of $\mu$ implies that there is a compact set $K$ such that $|\mu|(K^c) < \varepsilon$. If $\nu_i \to 0$, then eventually we must have $|\nu_i(x) - 1| < \varepsilon$ for all $x \in K$, and then
%
\[ |(L_{\nu} \widehat{\mu} - \widehat{\mu})(\phi)| \leq 2|\mu|(K^c) + \varepsilon \| \mu \| \leq \varepsilon(2 + \|\mu\|) \]
%
Which implies uniform continuity.

Let us consider why it is natural to generalize operators on $L^1(G)$ to $M(G)$. The first reason is due to the intuition of physicists; most of classical Fourier analysis emerged from physical considerations, and it is in this field that $L^1(G)$ is often confused with $M(G)$. Take, for instance, the determination of the electric charge at a point in space. To determine this experimentally, we take the ratio of the charge over some region in space to the volume of the region, and then we limit the size of the region to zero. This is the historical way to obtain the density of a measure with respect to the Lebesgue measure, so that the function we obtain can be integrated to find the charge over a region. However, it is more natural to avoid taking limits, and to just think of charge as an element of $M(\RR^3)$. If we consider a finite number of discrete charges, then we obtain a discrete measure, whose density with respect to the Lebesgue measure does not exist. This doesn't prevent physicists from trying, so they think of the density obtained as shooting off to infinity at points. Essentially, we obtain the Dirac Delta function as a `generalized function'. This is fine for intuition, but things seem to get less intuitive when we consider the charge on a subsurface of $\RR^3$, where the `density' is `dirac'-esque near the function, where as measure theoretically we just obtain a density with respect to the two-dimensional Hausdorff measure on the surface. Thus, when physicists discuss quantities as functions, they are really thinking of measures, and trying to take densities, where really they may not exist.

There is a more austere explanation, which results from the fact that, with respect to integration, $L^1(G)$ is essentially equivalent to $M(G)$. Notice that if $\mu_i \to \mu$ in the weak-$*$ topology, then $\widehat{\mu_i} \to \widehat{\mu}$ pointwise, because
%
\[ \int \frac{d\mu_i(x)}{\phi(x)} \to \int \frac{d\mu(x)}{\phi(x)} \]
%
(This makes sense, because weak-$*$ convergence is essentially pointwise convergence in $M(G)$). Thus the Fourier-Stietjes transform is continuous with respect to these topologies. It is the unique continuous extension of the Fourier transform, because

\begin{theorem}
    $L^1(G)$ is weak-$*$ dense in $M(G)$.
\end{theorem}
\begin{proof}
    First, note that the Dirac delta function can be weak-$*$ approximated by elements of $L^1(G)$, since we have an approximate identity in the space.

    First, note that if $\mu_i \to \mu$, then $\mu_i * \nu \to \mu * \nu$, because
    %
    \[ \int f d(\mu_i * \nu) = \int \int f(xy) d\mu_i(x) d\nu(y) \]
    %
    The functions $y \mapsto \int f(xy) d\mu_i(x)$ converge pointwise to $\int f(xy) d\mu(y)$. Since
    %
    \[ \left| \int f(xy) d\mu_i(x) \right| \leq \| f \|_1 \| \mu_i \| \]

    If $i$ is taken large enough that
\end{proof}

If $\phi_\alpha \to \phi$, in the sense that $\phi_\alpha(x) \to \phi(x)$ for all $x \in G$, then, because $\| \phi_\alpha(x) \| = 1$ for all $x$, we can apply the dominated convergence theorem on any compact subset $K$ of $G$ to conclude
%
\[ \int_K \frac{d\mu(x)}{\phi_\alpha(x)} \to \int_K \frac{d\mu(x)}{\phi(x)} \]

It is immediately verified to be a map into $L^1(\Gamma(G))$, because
%
\[ \int \left| \int \frac{d\mu(x)}{\phi(x)} \right| d\phi \leq \int \int \| \mu \| \]

The formula above immediately suggests a generalization to a transform on $M(G)$. For $\nu \in M(G)$, we define
%
\[ \mathcal{F}(\nu)(\phi) = \int \frac{d \nu}{\phi} \]
%
If $\mathcal{G}: L^1(G) \to C_0(\Gamma(G))$ is the Gelfand transform, then the transform induces a map $\mathcal{G}^* : M(\Gamma(G)) \to L^\infty(G)$.

The duality in class-ical Fourier analysis is shown through the inversion formulas. That is, we have inversion functions
%-00
\[ \mathcal{F}^{-1}(\{ a_k \}) = \sum a_k e_k(t)\ \ \ \ \ \mathcal{F}^{-1}(f)(x) = \int f(t) e(x t) \]
%
which reverses the fourier transform on $\mathbf{T}$ and $\RR$ respectively, on a certain subclass of $L^1$. One of the challenges of Harmonic analysis is trying to find where this holds for the general class of measurable functions.

The first problem is to determine surjectivity. We denote by $A(G)$ the space of all continuous functions which can be represented as the fourier transform of some function in $L^1(G)$. It is to even determine $A(\mathbf{T})$, the most basic example. $A(G)$ always separates the points of $\Gamma(G)$, by Gelfand theory, and if $G$ is unimdoular, then it is closed under conjugation. If we let $g(x) = \overline{f(x^{-1})}$, we find
%
\[ \mathcal{F}(g)(\phi) = \int \frac{g(x)}{\phi(x)} dx = \overline{ \int \frac{f(x^{-1})}{\phi(x^{-1})} dx } = \int \frac{f(x)}{\phi(x)} dx = \overline{\mathcal{F}(f)(\phi)} \]
%
so that by the Stone Weirstrass theorem $A(G)$ is dense in $C_0(\Gamma(L^1(G)))$.

\chapter{Banach Algebra Techniques}

In the mid 20th century, it was realized that much of the analytic information about a topological group can be captured in various $C^*$ algebras related to the group. For instance, consider the Gelfand space of $L^1(\mathbf{Z})$ is $\mathbf{T}$, which represents the fact that one can represent functions over $\mathbf{T}$ as sequences of numbers. Similarily, we find the characters of $L^1(\RR)$ are the maps $f \mapsto \widehat{f}(x)$, so that the Gelfand space of $\RR$ is $\RR$, and the Gelfand transform is the Fourier transform on this space. For a general $G$, we may hope to find a generalized Fourier transform by understanding the Gelfand transform on $L^1(G)$. We can also generalize results by extending our understanding to the class $M(G)$ of regular, Borel measures on $G$.

\chapter{Vector Spaces}

If $\mathbf{K}$ is a closed, multiplicative subgroup of the complex numbers, then $\mathbf{K}$ is also a locally compact abelian group, and we can therefore understand $\mathbf{K}$ by looking at its dual group $\mathbf{K}^*$. The map $\langle x,y \rangle = xy$ is bilinear, in the set that it is a homomorphism in the variable $y$ for each fixed $x$, and a homomorphism in the variable $x$ for each $y$.

If $\mathbf{K}$ is a subfield of the complex numbers, then $\mathbf{K}$ is also an abelian group under addition, and we can consider the dual group $\mathbf{K}^*$. The inner product $\langle x, y \rangle = xy$ gives a continuous bilinear map $\mathbf{K} \times \mathbf{K} \to \mathbf{C}$, and therefore we can define $x^* \in \mathbf{K}^*$ by $x^*(y) = \langle x,y \rangle$. If $x^*(y) = xy = 0$ for all $y$, then in particular $x^*(1) = x$, so $x = 0$. This means that the homomorphism $\mathbf{K} \to \mathbf{K}^*$ is injective.

\chapter{Interpolation of Besov and Sobolev spaces}

An important class of operators arise as singular integrals, that is, they arise as convolution operators $T$ given by $T(f) = f * K$, where $K$ is an appropriate distribution. Taking Fourier transforms, these operators can also be defined by $\widehat{T(f)} = \widehat{f} \widehat{K}$. The function $\widehat{K}$ is known as a {\bf Fourier multiplier}, because it operates by multiplication on the frequencies of the function $f$. We say $\widehat{K}$ is a {\bf Fourier multiplier on $L^p(\RR^n)$} if $T$ is a bounded map from $S(\RR^n)$ to $L^p(\RR^n)$, under the $L^p$ norms. Such maps clearly extend uniquely to maps from $L^p(\RR^n)$ to $L^p(\RR^n)$, and so we can think of $T$ as operating by convolution on the space of $L^p$ functions. We will denote the space of all Fourier multipliers on $L^p$ by $M_p$. We define the $L^p$ norm on these distributions $K$, denoted $\| K \|_p$, to be the operator norm of the associated operator $T$.

\begin{example}
    Consider the space $M_\infty$. If $K$ is a distribution in $M_\infty$, then $\| K \|_\infty < \infty$, and since convolution commutes with translations, in the sense that $f_h * K = (f * K)_h$, then
    %
    \[ \| K \|_\infty = \sup_{f \in L^\infty(\RR^n)} \frac{|(f * K)(0)|}{\| f \|_\infty} \]
    %
    But then the map $f \mapsto (f * K)(0)$ is a bounded operator on the space of bounded continuous functions, and so the Riesz representation says there is a bounded Radon measure $\mu$ such that
    %
    \[ (f * K)(0) = \int f(-y)\; d\mu(y) \]
    %
    But now we know
    %
    \[ (f * K)(x) = (f_{-x} * K)(0) = \int f(x - y) d\mu(y) = (f * \mu)(x) \]
    %
    Thus $M_\infty$ is really just the space of all bounded Radon measures, and
    %
    \[ \| K \|_\infty = \sup_{f \in L^\infty(\RR^n)} \frac{\left| \int f(y)\; d\mu(y) \right|}{\| f \|_\infty} = \| \mu \|_1 \]
    %
    so $M_\infty$ even has the same norm as the space of all bounded Radon measures. Note that it becomes a Banach algebra under convolution of distributions, since the convolution of two bounded Radon measures is a bounded Radon measure.
\end{example}

\begin{theorem}
    For any $1 \leq p \leq \infty$, and $q = p^*$, then $M_p = M_q$.
\end{theorem}
\begin{proof}
    Let $f \in L^p$, and $g \in L^q$, then H\"{o}lder's inequality gives
    %
    \[ |(K * f * g)(0)| \leq \| K * f \|_p \| g \|_q \leq \| K \|_p \| f \|_p \| g \|_p \]
    %
    Thus $K * g \in L_q$, and that $K \in M_q$ with $\| K \|_q \leq \| K \|_p$. By symmetry, we find $\| K \|_p = \| K \|_q$.
\end{proof}

\begin{example}
    Consider $M_2$. If $K$ is a distribution with $\| f * K \|_2 \leq A \| f \|_2$, then Parsevel's inequality implies that
    %
    \[ \| \widehat{f} \widehat{K} \|_2 = \| f * K \|_2 \leq A \| f \|_2 = A \| \widehat{f} \|_2 \]
    %
    so for each $\widehat{f}$, TODO: PROVE THAT THIS IS REALLY JUST THE SPACE $L^\infty(\RR^n)$, with the supremum norm. Note that this is also a Banach algebra under pointwise multiplication.
\end{example}

Using the Riesz-Thorin interpolation theorem, we find that if $1/p = (1 - \theta)/p_0 + \theta/p_1$, then $\| K \|_p \leq \| K \|_{p_0}^{1 - \theta} \| K \|_{p_1}^\theta$, when $K$ lies in the three spaces. In particular, $\| K \|_p$ is a decreasing function of $p$ for $1 \leq p \leq 2$, so we find $M_1 \subset M_p \subset M_q \subset M_2$ for $1 \leq p < q \leq 2$. In particular, all Fourier multipliers can be viewed as Fourier multipliers with respect to bounded, measurable functions on $L^\infty$. Riesz interpolation shows that each $M_p$ is a Banach algebra under multiplication in the frequency domain, or convolution in the spatial domain.

\begin{theorem}
    Let $T: \RR^n \to \RR^m$ be a surjective affine transformation. Then the endomorphism $T^*$ on $M_p(\RR^n)$ defined by $(T^* f)(\xi) = f(T(\xi))$ is an isometry, and if $T$ is a bijection, so too is $T^*$.
\end{theorem}
\begin{proof}
    TODO
\end{proof}

The next theorem is the main tool to prove results about Sobolev and Besov space. Note that it assumes $1 < p < \infty$, and cannot be applied for $p = 1$ or $p = \infty$. The proof relies on two lemmas, the first of which is used frequently later, and the second is used universally in modern harmonic analysis.

\begin{lemma}
    There exists a Schwartz function $\varphi$ on $\RR^n$ which is supported on the annulus
    %
    \[ \{ \xi: 1/2 \leq |\xi| \leq 2 \} \]
    %
    is positive for $1/2 < |\xi| < 2$, and satisfies
    %
    \[ \sum_{k = -\infty}^\infty \varphi(2^{-k} \xi) = 1 \]
    %
    for all $\xi \neq 0$.
\end{lemma}

\begin{lemma}[Calderon-Zygmund Decomposition]
    Let $f \in L^1(\RR^n)$, and $\sigma > 0$. Then there are pairwise almost disjoint cubes $I_1, I_2, \dots$ with edges parallel to the coordinate axis and
    %
    \[ \sigma < \frac{1}{|I_n|} \int_{I_n} |f(x)|\; dx \leq 2^n \sigma \]
    %
    and with $|f(x)| \leq \sigma$ for almost all $x$ outside these cubes.
\end{lemma}

\begin{theorem}[The Mihlin Multiplier Theorem]
    Let $m$ be a bounded function on $\RR^n$ which is smooth except possibly at the origin, such that
    %
    \[ \sup_{\substack{\xi \in \RR^n\\|\alpha| \leq L}} |\xi|^{|\alpha|} |(D^\alpha m)(x)| < \infty \]
    %
    Then $m$ is an $L^p$ Fourier multiplier for $1 < p < \infty$.
\end{theorem}

\section{Besov Spaces}

Recall the Schwarz function $\varphi$ used to prove the Mihlin multiplier theorem. We now define functions $\varphi_k$ such that
%
\[ \widehat{\varphi_n}(\xi) = \varphi(2^{-n} \xi)\ \ \ \ \ \ \ \ \widehat{\psi}(\xi) = 1 - \sum_{n = 1}^\infty \varphi(2^{-n} \xi) \]
%
Thus $\varphi_n$ essentially covers the annulus $2^{n-1} \leq |\xi| \leq 2^{n+1}$, and the function $\psi$ covers the remaining low frequency parts covered in the frequency ball of radius 2. We have
%
\[ \varphi_n(\xi) = \widecheck{\varphi_{2^{-n}}}(\xi) = 2^{dn} \widecheck{\varphi}(2^n \xi) \]
%
Given $s \in \RR$, and $1 \leq p, q \leq \infty$, we write
%
\[ \| f \|_{pq}^s = \| \psi * f \|_p + \left( \sum_{n = 1}^\infty (2^{sn} \| \varphi_k * f \|_p)^q \right)^{1/q} \]
%
The convolution $\varphi_n * f$ essentially captures the portion of $f$ whose frequencies lie in the annulus $2^{n-1} \leq |\xi| \leq 2^{n+1}$

\section{Proof of The Projection Result}

As with Marstrand's projection theorem, we require an energy integral variant. Rather than considering the Riesz kernel on $\RR^n$, we consider the kernel on balls
%
\[ K_\alpha(x) = \frac{\chi_{B(0,R)}(x)}{|x|^\alpha} \]
%
where $R$ is a fixed radius. If $\alpha < \beta$, and $\mu$ is measure supported on a $\beta$ dimensional subset of $\RR^n$, then $\mu * K_\alpha \in L^\infty(\RR^d)$ because $\mu$ cancels out the singular part of $K_\alpha$. Assuming $\beta < d$, we conclude $\mu * K_\alpha \in L^1(\RR^d)$. Applying interpolation (TODO: Which interpolation), we conclude that $\nu * K_\rho$ 

\chapter{The Cap Set Problem}

The cap set problem comes out of additive combinatorics, whose goal is to understand additive structure in some abelian group, typically the integers. For instance, we can think of a set $A$  as being roughly closed under addition if $|A+A| = O(|A|)$. Over rings, we can study the interplay between additive and multiplicative structure. For instance, one conjecture of Erd\"{o}s and Szemer\'{e}di says that if $A$ is a finite subset of real numbers, then $\text{max}(|A+A|,|A \cdot A|) \gtrsim |A|^{1+c}$ for some positive $c \in (0,1)$. The best known $c$ so far is $c \sim 1/3$, though it is conjectured that we can take $c$ arbitrarily close to $1$. This can be seen as a discrete version of the results of Bourgain and Edgar-Miller on the Hausdorff dimensions of Borel subrings.

\begin{theorem}[Van Der Waerden - 1927]
    For any positive integes $r$ and $k$, there is $N$ such that if the integers in $[1,N]$ are given an $r$ coloring, then there is a monochromatic $k$ term arithmetic progression.
\end{theorem}

The coloring itself is not so important, more just the partitioning. We just pidgeonhole, using the density of $k$ term arithmetic progressions. This problem suggests the Ramsey type problem of determining the largest set $A$ of the integers $[1,N]$ which does not contain $k$ term arithmetic progressions. Behrend's theorem says we can choose $A$ to be on the order of $N\exp(-c \sqrt{\log N})$.

\begin{theorem}[Roth - 1956] If $A$ is a set of integers in $[1,N]$ which is free of three term arithmetic progressions, then $|A| = O(N/\log \log N)$.
\end{theorem}

Szemer\'{e}di proved that if $A$ is free of $k$ term arithmetic progressions, $|A| = o(N)$. If Erd\"{o}s Turan, if $\sum_{x \in X} 1/x$ diverges, then $X$ contains arbitrarily long arithmetic progressions. For now, we'll restrict our attention to three term arithmetic progressions. Heath and Brown showed that three term arithmetic progresisons are $O(N/(\log N)^c)$ for some constant $c$. In 2016, the best known bound was given by Bloom, given $O(N(\log \log N)^4/\log N)$.

One way we can simplify our problem is to note that avoiding three term arithmetic progressions is a local issue, so we can embed $[1,N]$ in $\mathbf{Z}/M\mathbf{Z}$ for suitably large $M$, and we lose none of the problems we had over the integers. A heuristic is that it is easier to solve these kind of problems in $\mathbf{F}_p^n$, where $p$ is small and $n$ is large, which should behave like $\{ 1, \dots, p^n \}$. This leads naturally to the cap set problem.

\begin{theorem}[Cap Set Problem]
    What is the largest subset of $\mathbf{F}_3^n$ containing no three term arithmetic progressions?
\end{theorem}

We look at $\mathbf{F}_3$ because it is the smallest case where three term arithmetic progressions become important.

\begin{theorem}[Meschulam - 1995]
    Let $A \subset \mathbf{F}_3^n$ be a cap set. Then $|A| = O(3^n/n)$. This is analogous to a $N/\log N$ case over the integers, giving evidence that the finite field case is easier.
\end{theorem}

In 2012, Bateman and Katz showed $|A| = O(3^n/n^{1 + \varepsilon})$ for some $c > 0$. This was a difficult proof. In 2016, there was a more significant breakthrough, which gave an easy proof using the polynomial method of an exponentially small bound of $c^n$, where $c < 4$, over $\mathbf{Z}/4\mathbf{Z}$, and a week later Ellenberg-Gijswijt used this argument in the $\mathbf{F}_3$ case to prove that if $A$ is a capset in $\mathbf{F}_3$,then $|A| = O(c^n)$, for $c = 2.7551\dots$.

The idea of the polynomial method is to take combinatorial information about some set, encode it as some algebraic structura information, and then apply the theory of polynomials to encode this algebraic information and use it to limit and enable certain properties to occur.

If $V$ is the space of polynomials of degree $d$ vanishing on a set $A$, then we know $\dim V \geq \dim \mathcal{P}_d - |A|$. This gives a lower bound on the size of $A$, whereas we want a lower bound. To get an upper bound, we take $|A|^c$ instead, which shows
%
\[ \dim V \geq \dim \mathcal{P}_d + |A| - 3^n \]
%
whichs gives $|A| \leq 3^n + \dim V - \dim \mathcal{P}_d$. Now using linear algebra, we can find a polynomial $P$ vanishing on $A^c$ with support of cardinality greater than or equal to $\dim V$, hence
%
\[ |A| \leq 3^n - \dim \mathcal{P}_d + \max |\text{supp}(P)| \]
%
It follows that $A$ is a cap set if and only if $x + y = 2z$, or $x + y + z = 0$ holds if and only if $x = y = z$. This is an algebraic property which says directly that $A$ has no nontrivial three term arithmetic progressions. Thus for any $a_1, \dots, a_m \in A$, $P(-a_i-a_j) = 0$ when $i \neq j$. Equivalently, this means $P(-a_i-a_j) \neq 0$ when $i = j$. This suggests we consider the $|A|$ by $|A|$ matrix $M$ with $M_{ij} = P(-a_i-a_j)$. This is a diagonal matrix, with $M_{ii} = P(a_i)$. Thus the rank of this matrix is the dimension of the support of $P$, so it suffices to upper bound the rank of $M$. The key observation, where we now explicitly employ the fact that $P$ is a polynomial, is that $P(-x-y)$ is a polynomial in $2n$ variables $x,y \in \mathbf{F}_3^n$,












%% The following is a directive for TeXShop to indicate the main file
%%!TEX root = HarmonicAnalysis.tex

\part{Decoupling}

Decoupling Theory is an in depth study of how `interference patterns' can show up when combined waves with frequency supports in disjoint regions of space. The geometry of these regions effects how much constructive interference can happen. Of course decoupling theory is essential to studying many dispersive partial differential equations, but also has surprising applications in number theory as well, as well as other areas of harmonic analysis, such as restriction theory. The theory of decoupling was initiated by Wolff in the early 2000s, and Laba-Wolff, Laba-Pramanik, Pramanik-Seeger, and Garrigos and Seeger in the 2000s. But the theory was brought to the forefront of harmonic analysis by Bourgain, Demeter, and Guth in the mid 2010s.








\chapter{The General Framework}

In any norm space $X$, given $x_1, \dots, x_N \in X$, one can apply the Cauchy-Schwartz inequality to obtain the estimate
%
\[ \| x_1 + \dots + x_N \|_X \leq \| x_1 \|_X + \dots + \| x_N \|_X. \]
%
Such a result is often sharp for general $x_1, \dots, x_N$. For instance, when $X = L^1(\RR^d)$, and the $x_1, \dots, x_N$ are functions with disjoint supports, but with equal $L^1$ norm. Iff $\| x_i \|_X \sim 1$ for each $i$, then we conclude that
%
\[ \| x_1 + \dots + x_N \|_X \lesssim N. \]
%
However, if the $x_1, \dots, x_n$ are `uncorrelated', then one can often expect this result to be substantially improved. For instance, if $X$ is a Hilbert space, and if $x_1, \dots, x_N$ are pairwise orthogonal, Bessel's inequality allows us to conclude that
%
\[ \| x_1 + \dots + x_N \|_X \leq \left( \| x_1 \|_X^2 + \dots + \| x_N \|_X^2 \right)^{1/2}. \]
%
In this case, if $\| x_i \|_X \sim 1$, then
%
\[ \| x_1 + \dots + x_N \|_X \lesssim N^{1/2}. \]
%
Thus we obtain a significant `square root cancellation' in $N$. For instance, in $L^2(\RR^d)$, this occurs if $x_1, \dots, x_N$ have disjoint supports, or perhaps more interestingly in the context of Fourier analysis, if their Fourier transforms have disjoint supports.

The most basic case of square root cancellation occurs if we replace the property of orthogonality with \emph{almost orthogonality}. If $x_1, \dots, x_N$ are a family of vectors in a Hilbert space $H$, then
%
\begin{align*}
  \| x_1 + \dots + x_N \|_H^2 &= \sum \| x_i \|_H^2 + 2 \sum_{1 \leq i < j \leq N} \langle x_i, x_j \rangle.
\end{align*}
%
Thus, in particular, if each $x_i$ is orthogonal to all but at most $M$ of the other vectors in this family, then we conclude that
%
\[ \| x_1 + \dots + x_N \|_H^2 \leq \sum_i \| x_i \|_H^2 + 2M \sum_i \| x_i \|_H^2 \leq (2M + 1) \sum \| x_i \|_H^2. \]
%
Thus, provided that $M$ is much smaller than $N$, we also obtain square root cancellation in this setting.

We are interested in determining what causes `square root cancellation' in more general norm spaces than just Hilbert. The theory of \emph{almost orthogonality} studies this phenomena in Hilbert spaces, but we are interested in studying when this phenomenon in other norm spaces. Informally, we say $x_1, \dots, x_N$ satisfies a \emph{decoupling inequality} in a norm space $X$ if for all $\varepsilon > 0$, we have
%
\[ \| x_1 + \dots + x_N \|_X \lesssim_\varepsilon N^\varepsilon \left( \| x_1 \|_X^2 + \dots + \| x_N \|_X^2 \right)^{1/2}. \]
%
Thus decoupling theory is the study of when certain values can be correlated with one another, in general norm spaces. Of particular importance in harmonic analysis is the determination of the properties the Fourier transform of a family of functions must have to enable us to obtain decoupling phenomena.

\begin{remark}
  We are primarily interested in studying decoupling in $L^p(\Omega)$. However, in this case we should only expect decoupling to occur when $p \geq 2$. Functions in $L^p$ spaces are `most orthogonal' when their supports are disjoint. If a family of functions $\{ f_i \}$ have disjoint supports, then we actually have
  %
  \[ \| f_1 + \dots + f_N \|_{L^p(\Omega)} = \left( \| f_1 \|_{L^p(\Omega)}^p + \dots + \| f_N \|_{L^p(\Omega)}^p \right)^{1/p}. \]
  %
  For $p < 2$, we have a sharp inequality
  %
  \[ \left( \| f_1 \|_{L^p(\Omega)}^p + \dots + \| f_N \|_{L^p(\Omega)}^p \right)^{1/p} \leq N^{1/p - 1/2} \left( \| f_1 \|_{L^p(\Omega)}^2 + \dots + \| f_N \|_{L^2(\Omega)}^2 \right)^{1/2}, \]
  %
  so we do not have square root cancelation in this setting.
\end{remark}

The most basic case where we can obtain decoupling in $L^p(\Omega)$ for $p > 2$ is when $p$ is an even integer. As a basic example, say a family of functions $f_1, \dots, f_N \in L^4(\Omega)$ are \emph{biorthogonal} if $\{ f_i f_j : i < j \}$ forms an orthogonal family in $L^2(\Omega)$.

\begin{theorem}
  If $f_1, \dots, f_N$ are a biorthogonal family, then
  %
  \[ \| f_1 + \dots + f_N \|_{L^4(\Omega)} \lesssim \left( \| f_1 \|_{L^4(\Omega)}^2 + \dots + \| f_N \|_{L^4(\Omega)}^2 \right)^{1/2}. \]
\end{theorem}
\begin{proof}
  We write
  %
  \begin{align*}
    \left\| f_1 + \dots + f_N \right\|_{L^4(\Omega)}^2 &= \left\| (f_1 + \dots + f_N)^2 \right\|_{L^2(\Omega)}\\
    &= \left\| \sum_{1 \leq i,j \leq N} f_i f_j \right\|_{L^2(\Omega)}\\
    &\lesssim \sum_{i = 1}^N \| f_i^2 \|_{L^2(\Omega)} + \left\| \sum_{1 \leq i < j \leq N} f_i f_j \right\|_{L^2(\Omega)}
  \end{align*}
  %
  Applying Bessel's inequality, we conclude that
  %
  \begin{align*}
    \left\| \sum_{1 \leq i < j \leq N} f_i f_j \right\|_{L^2(\Omega)} &= \left( \sum_{1 \leq i < j \leq N} \| f_i f_j \|_{L^2(\Omega)}^2 \right)^{1/2}\\
    &= \left\| \sum_{i = 1}^N |f_i|^2 \right\|_{L^2(\Omega)} \lesssim \sum_{i = 1}^N \| f_i^2 \|_{L^2(\Omega)}.
  \end{align*}
  %
  Combining these calculations, noticing that $\| f_i^2 \|_{L^2(\Omega)} = \| f_i \|_{L^4(\Omega)}^2$, and taking in square roots completes the claim.
\end{proof}

\begin{remark}
  The calculation above shows a decoupling inequality still holds if the family $f_i f_j$ is almost biorthogonal, e.g. if each of the elements of the family $\{ f_i f_j \}$ are orthogonal to all but $O_\varepsilon(N^\varepsilon)$ of the other elements of the family.
\end{remark}

\begin{remark}
  Similarily, if $f_1, \dots, f_N \in L^6(\Omega)$ are chosen to be \emph{triorthogonal}, in the sense that the family of functions $\{ f_i f_j f_k \}$ are orthogonal to one another, one can obtain a decoupling inequality in the $L^6$ norm.
\end{remark}

We will be most interested in studying families of functions $f_1, \dots, f_N$ with disjoint Fourier supports in $L^p(\RR^d)$, where $p \geq 2$. It is then certainly true that
%
\[ \| f_1 + \dots + f_N \|_{L^2(\RR^d)} \leq \left( \| f_1 \|_{L^2(\RR^d)}^2 + \dots + \| f_N \|_{L^2(\RR^d)}^2 \right)^{1/2}. \]
%
However, for $p > 2$ having disjoint Fourier supports does not immediately imply decoupling, because constructive interference can still ocur in the $L^p$ norm in a way not detected in the $L^2$ norm, and we require addition features of the family in order to guarantee that this constructive interference does not occur.

\begin{theorem}
  If $f_1, \dots, f_N \in L^p(\RR^d)$, have disjoint Fourier support, and $2 \leq p \leq \infty$, then
  %
  \[ \| f_1 + \dots + f_N \|_{L^p(\RR^d)} \leq N^{1/2 - 1/p} \left( \| f_1 \|_{L^p(\RR^d)}^2 + \dots + \| f_N \|_{L^p(\RR^d)}^2 \right)^{1/2}. \]
\end{theorem}
\begin{proof}
  For $p = \infty$, we have the trivial inequality
  %
  \begin{align*}
    \| f_1 + \dots + f_N \|_{L^\infty(\RR^d)} &\leq \| f_1 \|_{L^\infty(\RR^d)} + \dots + \| f_N \|_{L^\infty(\RR^d)}\\
    &\leq N^{1/2} \left( \| f_1 \|_{L^\infty(\RR^d)}^2 + \dots + \| f_N \|_{L^\infty(\RR^d)}^2 \right)^{1/2}.
  \end{align*}
  %
  By a density argument (applying a cutoff in frequency space), we may assume without loss of generality that the functions $\{ f_i \}$ are Schwartz. But then by orthogonality of the Fourier transform implies the functions themselves are orthogonal, so we have
  %
  \[ \| f_1 + \dots + f_N \|_{L^2(\RR^d)} \leq \left( \| f_1 \|_{L^2(\RR^d)}^2 + \dots + \| f_N \|_{L^2(\RR^d)}^2 \right)^{1/2}. \]
  %
  We can then interpolate with the case $p = \infty$.
\end{proof}

In general, this result is optimal, as the next result shows one can have significant constructive interference for $p > 2$ if our functions are spaced about on a periodic family of frequencies.

\begin{example}
  Fix $\phi \in \mathcal{S}(\RR^d)$ with $\phi(0) = 1$, and with Fourier support in $[0,1]$. For each $k \in \{ 1, \dots, N \}$, set $f_k = e^{2 \pi i k x} \phi(x)$. Then $f_k$ has Fourier support in $[2k,2k+1]$, and $f_k(0) = 1$. The uncertainty principle thus implies that the functions $f_k$ are all roughly constant at a scale $O(1/N)$, which implies that in a ball of radius $O(1/N)$ around the origin, we have $f_1(x) + \dots + f_n(x) \approx 1$. Thus we conclude that
  %
  \[ \| f_1 + \dots + f_N \|_{L^p(\RR)} \gtrsim N^{1 - 1/p}. \]
  %
  On the other hand, we have
  %
  \[ \left( \| f_1 \|_{L^p(\RR)}^2 + \dots + \| f_N \|_{L^p(\RR)}^2 \right)^{1/2} \lesssim N^{1/2}, \]
  %
  Thus
  %
  \[ \| f_1 + \dots + f_N \|_{L^p(\RR)} \gtrsim N^{1/2 - 1/p} \left( \| f_1 \|_{L^p(\RR)}^2 + \dots + \| f_N \|_{L^p(\RR)}^2 \right)^{1/2}, \]
  %
  which shows our result is tight up to constants.
\end{example}

In the face of this result, we are interested in knowning, for a given family $\mathcal{S}$ of disjoint sets in $\RR^d$, what the smallest constant $\text{Dec}(S,p)$ is for which it is true that if $f_1, \dots, f_N$ have Fourier support on distinct regions $S_1, \dots, S_N \in \mathcal{S}$, we have
%
\[ \| f_1 + \dots + f_N \|_{L^p(\RR^d)} \leq \text{Dec}(\mathcal{S},p) \left( \| f_1 \|_{L^p(\RR^d)}^2 + \dots + \| f_N \|_{L^p(\RR^d)}^2 \right)^{1/2}. \]
%
Pure orthogonality gives $\text{Dec}(\mathcal{S},2) = 1$. The triangle inequality gives $\text{Dec}(\mathcal{S},\infty) \leq \#(S)^{1/2}$, and this is actually sharp: we have $\text{Dec}(\mathcal{S},\infty) $

If each element of $\mathcal{S}$ contains a ball of radius $\delta$, then by summing up modulated bump functions on these balls, we can find functions such that $\text{Dec}(\mathcal{S},\infty) \gtrsim \delta^d \#(S)^{1/2}$. Thus non-trivial decoupling in the framework we are considering is impossible in $L^\infty(\RR^d)$, i.e. $\text{Dec}(\mathcal{S},\infty) \sim_\delta \#(S)^{1/2}$. In between, we can interpolate to get $\text{Dec}(\mathcal{S},p) \leq \#(S)^{1/2 - 1/p}$.

Such a result depends significantly on the geometric structure of the regions in $\mathcal{S}$. The techniques we will use (e.g. induction on scales) imply the need for the `$\varepsilon$ loss' given by the $N^\varepsilon$ factor above. Below is a positive result for a particular family $\mathcal{S}$, easily proved using the biorthogonality arguments established above.

\begin{theorem}
  Suppose $\mathcal{S}$ is a family of sets in $\RR^d$ such that for any four sets $S_1,S_2,S_3,S_4 \in \mathcal{S}$ with $\{ S_1, S_2 \} \neq \{ S_3, S_4 \}$, the sets $S_1 + S_2$ are disjoint from $S_3 + S_4$. Then if distinct sets $S_1, \dots, S_N \in \mathcal{S}$ are selected from $\mathcal{S}$, and $f_1, \dots, f_N$ are a family of Schwartz functions in $\RR^d$ such that $f_i$ has Fourier support in $S_i$ for each $i$, we find`'
  %
  \[ \| f_1 + \dots + f_N \|_{L^4(\Omega)} \lesssim \left( \| f_1 \|_{L^4(\Omega)}^2 + \dots + \| f_N \|_{L^4(\Omega)}^2 \right)^{1/2}. \]
\end{theorem}

\begin{remark}
  We say a set of integers $A \subset \{ 1, \dots N \}$ is a \emph{Sidon set} if there does not exist a nontrivial solution to the equation $a_1 + a_2 = a_3 + a_4$. If $A$ is Sidon, then $\mathcal{S} = \{ [2k,2k+1]: k \in A \}$ satisfies the constraints of the result above, and so we obtain that if $\{ f_k: k \in A \}$ are a family of Schwartz functions such that $f_k$ has Fourier support in $[2k,2k+1]$, then
  %
  \[ \| \sum_{k \in A} f_k \|_{L^4(\RR)} \lesssim \left( \sum_{k \in A} \| f_k \|_{L_4(\RR)}^2 \right)^{1/2}. \]
  %
  On the other hand, a variant of the example above shows that for any Sidon set $A$, there is a family of functions $\{ f_k : k \in A \}$ with $f_k$ having Fourier support on $[2k,2k+1]$, and with
  %
  \[ \left\| \sum_{k \in A} f_k \right\|_{L^4(\RR)} \gtrsim \frac{\#(A)^{1/2}}{N^{1/4}} \left( \sum_{k \in A} \| f_k \|_{L^4(\RR)}^2 \right)^{1/2}. \]
  %
  Combining this inequality with the decoupling inequality, we obtain an interesting number theory result: any Sidon set $A$, must satisfy the bound $\#(A) \lesssim N^{1/2}$. More generally, we can extend this result to show that any set $A \subset \{ 0, \dots, N-1 \}$ having no nontrivial solutions to the equation
  %
  \[ x_1 + \dots + x_m = y_1 + \dots + y_m \]
  %
  should satisfy $\#(A) \lesssim N^{1/m}$.
\end{remark}

Another example family of sets $\mathcal{S}$ where Decoupling occurs occurs in Littlewood-Paley theory.

\begin{theorem}
  Let $\mathcal{S}$ be the collection of all boxes in $\RR^d$ of the form
  %
  \[ I_k = I_{\pm k_1} \times \dots \times I_{\pm k_d} \]
  %
  such that $I_{+ k_1} = [2^{k_1}, 2^{k_1 + 1}]$ and $I_{-k_1} = [-2^{k_1+1}, -2^{k_1}]$. Littlewood-Paley theory implies that if $S_1, \dots, S_N \in \mathcal{S}$ and $f_1, \dots, f_N$ are Schwartz functions with $f_i$ having Fourier support on $S_i$ for each $i$, then for each $2 \leq p < \infty$,
  %
  \[ \| f_1 + \dots + f_N \|_{L^p(\RR^d)} \sim_{p,d} \| f_i \|_{L^p(\RR^d) l^2_N} \lesssim \| f_i \|_{l^2_N L^p(\RR^d)}. \]
  %
  This is precisely a decoupling inequality. More generally, whenever we have a reverse square function estimate
  %
  \[ \| f_1 + \dots + f_N \|_{L^p(\RR^d)} \lesssim \| f_i \|_{L^p_x l^2_N}, \]
  %
  we get a decoupling inequality by changing norms.
\end{theorem}

We say a set $\omega \subset \RR^d$ is an \emph{almost rectangular box} with sidelengths $L_1,\dots,L_d$ if there is a rectangular box $R$ with sidelengths $L_1, \dots, L_d$ centered at the origin and $\theta \in \RR^d$ such that $\theta + C^{-1} R \subset \theta \subset \theta + C \cdot R$ for some universal constant $C$. As a special case, we have \emph{almost cubes} as well. For each $R \geq 1$, partition $[-1,1]^{d-1}$ into $\sim R^{(d-1)/2}$ almost rectangular boxes of sidelength $R^{-1/2}$. If we consider the paraboloid $\mathbf{P}^{n-1} = \{ \xi, |\xi|^2 \}$, then we obtain a partition of $N(\mathbf{P}^{n-1} \cap [0,1]^d, R^{-1})$ by a family $\Theta(1/R)$ of $R^{-1/2} \times R^{-1}$ almost rectangular boxes, by setting
%
\[ \Theta(1/R) = \{ (\omega \times \RR) \cap N(\mathbf{P}^{n-1} \cap [0,1]^d ) \} \]
%
It is conjectured that for $p = 2n/(n-1)$, and for Schwartz functions $f_1, \dots, f_N$ with Fourier support on distinct elements of $\Theta(1/R)$, we have
%
\[ \| f_1 + \dots + f_N \|_{L^p(\RR^d)} \lessapprox_R \| f_i \|_{L^p(\RR^d) l^2_N}, \]
%
This reverse square function estimates is strong. In particular, it implies the restriction conjecture. But it also implies the decoupling estimate
%
\[ \| f_1 + \dots + f_N \|_{L^p(\RR^d)} \lesssim_\varepsilon R^\varepsilon \left( \| f_1 \|_{L^p(\RR^d)} + \dots + \| f_N \|_{L^p(\RR^d)} \right)^{1/2} \].
%
However, unlike the restriction conjecture, this problem is closed: we have a proof of this decoupling estimate not only when $p = 2d/(d-1)$, but even when $2 \leq p \leq 2(d+1)/(d-1)$.

\begin{comment}
\begin{example}
  TODO: Move this. When $p < 2$, one does not usually expect to find decoupling inequalities in $L^p(\Omega)$. For instance, for any family of disjoint measurable sets $E_1, \dots, E_N \in \Omega$, each with non-negative measure, one can find $f_1, \dots, f_N \in L^p(\Omega)$, with $f_i$ supported on $E_i$ for each $i$ such that
  %
  \begin{align*}
    \| f_1 + \dots + f_N \|_{L^p(\Omega)} &= \left( \| f_1 \|_{L^p(\Omega)}^p + \dots + \| f_N \|_{L^p(\Omega)}^p \right)^{1/p}\\
    &\geq N^{1/p - 1/2} \left( \| f_1 \|_{L^p(\Omega)}^2 + \dots + \| f_N \|_{L^p(\Omega)}^2 \right)^{1/2}.
  \end{align*}
  %
  The idea of this is simple; we just choose a family of scalars $A_1, \dots, A_N$ such that
  %
  \[ (A_1^p + \dots + A_N^p)^{1/p} = N^{1/p - 1/2} (A_1^2 + \dots + A_N^2)^{1/2}. \]
  %
  Given functions $f_1, \dots, f_N$ such that $f_i$ is supported in $E_i$ for each $i$, we need only rescale each function such that $\| f_i \|_{L^p(\Omega)} = A_i$ for each $i$. Similarily, if $U_1, \dots, U_N$ are disjoint open sets in $\RR^d$, we can find Schwartz functions $f_1, \dots, f_N$, such that $f_i$ has Fourier support in $U_i$ for each $i$, such that
  %
  \[ \| f_1 + \dots + f_N \|_{L^p(\Omega)} \gtrsim N^{1/p - 1/2} \left( \| f_1 \|_{L^p(\Omega)}^2 + \dots + \| f_N \|_{L^p(\Omega)}^2 \right)^{1/2}, \]
  %
  where the implict constant is independant of $N$, and $U_1, \dots, U_N$. The idea here is to begin with Schwarz functions $f_1, \dots, f_N$ such that $f_i$ has Fourier suppport in $U_i$, and then replace these Schwarz functions with translations such that the masses of the $f_i$ are essentially disjoint from one another, which only modulates the Fourier transform and so does not affect the Fourier support of the functions. Rescaling then gives the result.
\end{example}
\end{comment}

\section{Localized Estimates}

Suppose $f_1, \dots, f_N$ are Schwartz functions in $\RR^d$ with disjoint Fourier supports, and $\Omega \subset \RR^d$. A natural question to ask is when one should expect
%
\[ \| f_1 + \dots + f_N \|_{L^2(\Omega)}^2 \lesssim \| f_1 \|_{L^2(\Omega)}^2 + \dots + \| f_N \|_{L^2(\Omega)}^2. \]
%
If we consider the bump function counterexample constructed from earlier, and let $\Omega = \{ x \in \RR: |x| \lesssim 1/N \}$, then $\| f_1 + \dots + f_N \|_{L^2(\Omega)} \gtrsim N$, whereas $\| f_k \|_{L^2(\Omega)}^2 \lesssim 1/N$ so $\| f_1 \|_{L^2(\Omega)}^2 + \dots + \| f_N \|_{L^2(\Omega)}^2 \lesssim 1$, which means such a result cannot be obtained. However, we shall find that such a result holds if $\Omega$ is large enough, depending on the supports of $f_1, \dots, f_N$, and if we allow weighted estimates.

Let us begin with the case in one dimension. Given an interval $I$ with centre $x_0$, and length $R$, we consider the weight function
%
\[ w_I(x) = \left( 1 + \frac{|x - x_0|}{R} \right)^{-M} \]
%
It is a useful heuristic that if $f$ has Fourier support in $I$, then $f$ is `locally constant' on intervals of length $1/|I|$.

In $\RR^d$, given a ball $B$ with centre $x_0$ and radius $R$, we consider the weight function
%
\[ w_B(x) = \left( 1 + \frac{|x - x_0|}{R} \right)^{-M}, \]
%
where $M$ is a large integer. Then
%
\[ \int w_B(x)\; dx \]

TODO FINISH THIS

\section{Vinogradov Systems}

One long standing number theoretic conjecture that has been particularly amenable to the use of decoupling techniques is \emph{Vinogradov's Conjecture}. Let $J_{s,k}(N)$ denote the number of solutions to the \emph{system} of equations
%
\[ x_1^i + \dots + x_s^i = y_1^i + \dots + y_s^i \]
%
for $1 \leq i \leq k$, where $x_1,\dots,x_s,y_1,\dots,y_s \in \{ 1, \dots, N \}$. Our primary interest is determining the asymptotic growth in this quantity as $N \to \infty$. Setting $y_i = x_i$ gives a family of $N^s$ solutions. Thus $J_{s,k}(N) \geq N^s$. On the other hand, dyadic pidgeonholing implies that there exists some family of integer tuples $I \subset [1,N] \times \dots \times [1,N^k]$ and some integer $M$ with $N^s = M \#(I)$ such that for each tuple $(r_1,\dots,r_k) \in I$, there are $\Omega(M)$ tuples of values $(x_1,\dots,x_s)$ such that for each $1 \leq i \leq k$,
%
\[ x_1^i + \dots + x_s^i = r_i. \]
%
For any $i$, $x_1^i + x_s^i$ takes on values between $1$ and $N^i$. Since $1 \leq \#(I) \leq N^{k(k+1)/2}$, we have $N^{s - k(k+1)/2} \lesssim M \lesssim N^s$, and so
%
\[ J_{s,k}(N) \gtrsim \#(I) M^2 \gtrsim N^{2s - k(k+1)/2}, \]
%
a bound that is much tighter if the number of variables is large, i.e. $s \geq k(k+1)/2$. Vinogradov's mean value conjecture is that
%
\[ J_{s,k}(N) \lesssim_{s,k,\varepsilon} N^\varepsilon ( N^{2s-k(k+1)/2} + N^s ), \]
%
for any $\varepsilon > 0$. One result of decoupling is a proof of this conjecture.

\begin{remark}
  Another approach to the Vinogradov mean value conjecture is to use more number theoretic techniques, for instance, efficient congruencing. There has even been some feedback from this alternate approach in the field of decoupling, e.g. using efficient congruencing to obtain decoupling inequalities (see Guo-Li-Yung-Zorin Kranich, 2021).
\end{remark}

Similar types of number theoretic problems had been studied using harmonic analysis techniques. Classically, Hardy and Littlewood studied the quantity $\text{HL}_{k,s}(N)$, which counts the number of solutions to the \emph{single equation}
%
\[ x_1^k + \dots + x_s^k = y_1^k + \dots + y_s^k \]
%
with $x_1,\dots,x_s,y_1,\dots,y_s \in \{ 1, \dots, N \}$. To study this quantity, Hardy and Littlewood introduced the function
%
\[ f_{k,N}(x) = \sum_{n = 1}^N e^{2 \pi i n^k x}. \]
%
Orthogonality then implies that
%
\[ \text{HL}_{k,s}(N) = \| h_{k,N} \|_{L^{2s}[0,1]}. \]
%
Bounding the nature of the function $h_{k,N}$ then involves Hardy and Littlewood's circle method. On the other hand, for Vinogradov's conjecture we are lead to study a higher dimensional function
%
\[ j_{k,N}(x) = \sum_{n = 1}^N e^{2 \pi i (n x_1 + n^2 x_2 + \dots + n^k x_k)}, \]
%
in which case
%
\[ J_{k,s}(N) = \| j_{k,N} \|_{L^{2s}[0,1]^k}. \]
%
Though in higher dimensions, Vinogradov initially expected his result to be \emph{easier} than Hardy and Littlewood's problem, because of the \emph{rescalable} nature of the problem, which we will get to later. Indeed, decoupling has lead to a solution of Vinogradov's conjecture for almost all parameters, but a complete solution to the conjectured bound
%
\[ \text{HL}_{k,s}(N) \lesssim_\varepsilon N^\varepsilon ( N^s + N^{2s - k} ) \]
%
has not yet been resolved.

To do our analysis, it will be helpful to rescale our quantities. Thus we let
%
\[ f_{k,N}(x) = \sum_{n = 1}^N e^{2 \pi i ( (n/N) x_1 + \dots + (n/N)^k x_k )}. \]
%
The function $f_{k,N}$ is then $N$-periodic, and so Vinogradov's conjecture will follow from showing that
%
\[ \| f_{k,N} \|_{L^{2s}[0,N^k]^k} = N^{k^2/2s} \| j_{k,N} \|_{L^{2s}[0,1]^k} \]
%
satisfies a bound
%
\[ \| f_{k,N} \|_{L^{2s}[0,N^k]^k} \lesssim_\varepsilon N^{k^2/2s + \varepsilon} (N^s + N^{2s - k}). \]
%
This is what we will address using the theory of decoupling.

From the perspective of decoupling, it will be more handy to study the quantities
%
\[ \tilde{j}_{k,N}(x) = \sum_{n = 1}^N e^{2 \pi i (n x_1 + \dots + n^k x_k)} \phi(x), \]
%
and
%
\[ \tilde{f}_{k,N}(x) = \sum_{n = 1}^N e^{2 \pi i ((n/N) x_1 + \dots + (n/N)^k x_k)} \phi(x_1 / N, \dots, x_k / N^k), \]
%
where $\phi \in \mathcal{S}(\RR^d)$ has Fourier support in $|\xi| \leq 1/10$ and $\phi(x) \geq 1$ in a neighborhood of the origin. Bounding the $L^{2s}$ norm of $\tilde{j}_{k,N}$ and $\tilde{f_{k,N}}$ is essentially equivalent to bounding $j_{k,N}$ since we can upper bound $j_{k,N}$ and $f_{k,N}$ pointwise by $O(1)$ sums of this form on the domain upon which we are taking the $L^{2s}$ norm. Now if we let
%
\[ f_i = e^{2 \pi i ((n/N) x_1 + \dots + (n/N)^k x_k)} \phi(x_1 / N, \dots, x_k / N^k), \]
%
then $\widehat{f_i}$ is supported on a rectangle with sidelengths $O(N) \times \dots \times O(N^k)$ centered at the point $(n/N, \dots, (n/N)^k)$ of the moment curve. The $l^2$ decoupling inequality for the moment curve thus implies that
%
\[ \| f \|_{L^{2s}(\RR^d)} \lesssim_\varepsilon N^{1/2 + \varepsilon}, \]
%
which proves Vinogradov's mean value result for $s \geq k^3 / (2k^2 - 1)$, i.e. for $s \gtrsim k$, which, up to constants, is the current state of the art for the mean-value theorem.

\end{document}






\section{Homogenous Banach Spaces}

We finish our basic discussion of Fourier summation with a theorem employing the more abstract parts of functional analysis to generalize the convergence properties of good kernels. We saw a Banach space $X$ is a {\bf Homogenous Banach space} if it is a translation invariant subset of $L^1(\mathbf{T})$, with $\| \cdot \|_X \gtrsim \| \cdot \|_1$, and $\| T_t f \|_X = \| f \|_X$ for all $f$ and $T_t$. Finally, we require that $t \mapsto T_t f$ is continuous for each fixed $f \in X$.

\begin{theorem}
	For any good kernel $K_N \in C(\mathbf{T})$, and $f \in X$, $f * K_N \to f$ in $X$.
\end{theorem}
\begin{proof}
	Basic analysis shows that if a map $\phi: [0,2\pi] \to X$ is continuous, then
%
\[ \lim_{N \to \infty} \frac{1}{N} \sum_{m = 1}^N \phi(2\pi m/N) \]
%
exists in the $X$ norm, ala the Riemann integral. We define this to be the `Riemann integral' of $\phi$, i.e.
%
\[ \int_0^{2\pi} \phi(x)\; dx \]
%
This definition obeys all the finite additivity properties of the origin Riemann integral. Further basic approximations prove that if $K_N$ is a good kernel, then
%
\[ \lim \int K_N(x) \phi(x)\; dx = \phi(0) \]
%
If the map $x \mapsto f_x$ is continuous in the $X$ norm, then
%
\[ \lim \int K_N(x) f_x\; dx = f \]
%
But since the $X$ norm upper bounds the $L^1$ norm, $x \mapsto f_x$ is continuous in the $L^1$ norm, and it is easy to see that the $L^1$ value of $\int f_x K_N(x)\; dx$ is $f * K_N$, so we conclude that $f * K_N$ converges to $f$ in the $X$ norm.
\end{proof}

\begin{corollary}
	The trigonometric polynomials contained in $X$ are dense, and for any closed, translation invariant subspace $Y$ of $X$, $f \in Y$ if and only if for every $n$ for which $\widehat{f}(n) \neq 0$, there is $g \in Y$ such that $\widehat{g}(n) \neq 0$.
\end{corollary}
\begin{proof}
	The techniques of the proof above show that if $f \in C(\mathbf{T})$, and $g \in X$, then $f * g \in X$. In particular, this means that the convolutions of any $g \in X$ with an exponential function is contained in $X$. In particular, the trigonometric polynomials spanned by these exponentials must be dense in $X$. The fact about $Y$ is then obvious.
\end{proof}

Examples of homogenous spaces include $C(\mathbf{T})$, with uniform convergence, $C^n(\mathbf{T})$, with uniform convergence of the first $n$ derivatives, and $L^p(\mathbf{T})$, for $p < \infty$. Unfortunately, however, the space $L^\infty(\mathbf{T})$ fails to satisfy the fact that $t \mapsto f_t$ is uniformly continuous, with a counterexample provided by letting $f$ be the characteristic function of an interval. Similarily, the space of $\alpha$ Lipschitz continuous functions for $0 < \alpha < 1$ also does not satisfy this continuity property.

However, if $X$ is any space satisfying the properties of a homogenous Banach space except for the continuity of translation, the space of functions for which translation {\it is} continuous form a closed subspace which is a homogenous Banach space. In the case of $L^\infty(\mathbf{T})$, this subspace is precisely $C(\mathbf{T})$. For the $\alpha$ continuous functions, these functions are precisely those $f$ for which
%
\[ \limsup_{h \to 0} \frac{|f(t+h) - f(t)|}{|h|^\alpha} = 0 \]
%
For $\alpha = 1$, this space is again $C(\mathbf{T})$.

