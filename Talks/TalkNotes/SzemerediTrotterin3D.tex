\documentclass{article}

\usepackage{kpfonts}
\usepackage{amsmath}
\usepackage{amssymb}
\usepackage{amsthm}
\usepackage{esint}
\usepackage{MnSymbol}

\theoremstyle{plain}
\newtheorem{theorem}{Theorem}
\newtheorem{lemma}[theorem]{Lemma}
\newtheorem{corollary}[theorem]{Corollary}
\newtheorem{prop}[theorem]{Proposition}
\newtheorem*{example}{Example}
\newtheorem*{remark}{Remark}

\title{Incidence Theorems in Arbitrary Characteristic}
\author{Jacob Denson}
\date{\today}

\DeclareMathOperator{\RR}{\mathbf{R}}
\DeclareMathOperator{\CC}{\mathbf{C}}
\DeclareMathOperator{\PP}{\mathbf{P}}
\DeclareMathOperator{\LL}{\mathcal{L}}
\DeclareMathOperator{\ord}{\text{ord}}
\DeclareMathOperator{\coker}{\text{coker}}
\DeclareMathOperator{\emphcoker}{\emph{coker}}

\begin{document}

\maketitle

In this lecture, we try and explore the Szemeredi-Trotter theorem from a non-topological perspective, i.e. we try and bound the quantities
%
\[ I(P,L) = \sum_{p \in P} \ord_p(L)\ \ \ \ I(L) = \sum_p \ord_p(L) - 1 \]
%
Recall that in the plane, using topological techniques, such as polynomial partitioning, one is able to conclude that for any set of lines $L$ and points $P$,
%
\[ |I(P,L)| \lesssim |P|^{2/3} |L|^{2/3} + |P| + |L| \]
%
One way to test the dependence of this theorem on topological techniques is to try and prove the conjecture over an arbitrary field $k$, rather than over $\RR$. Indeed, for an arbitrary field there are no precise topological properties one can exploit in a proof of an incidence bound. Using mathematical logic, we can even show this truly does test how powerful `non topological methods' are on the real plane.

\begin{example}
    The {\it completeness} of affine and projective geometry shows that a statement over $\RR$ is provable using only basic methods of affine or projective geometry if and only if that statement remains true over any field $k$. If we want to add `synthetic' topological methods, such as exploiting the fact that a point is `between' two other points, then the completeness in this setting shows that the statment is true when $k$ is any {\it ordered} field if and only if there is a proof of this theorem.
\end{example}

Of course, incidence theorems on finite fields have independant interest and have many applications outside of incidence geometry, so trying to prove incidence bounds over finite fields is a very important problem. Thus we work with incidence bounds in $\PP^2$, over as a general a field $K$ as possible.

To begin with, we recall the Szmeredi-Trotter theorem for bounding incidences in the plane. Using Cauchy-Schwartz together with an $L^2$ bound on incidences, we were able to prove the bound $I(P,L) \leq |L|^{3/4} |M|^{3/4}$. This proof was obtained {\it purely} using incidence methods, and holds over any affine / projective space. We then applied polynomial partitioning to sharpen this bound, but we cannot use this technique over arbitrary fields. And indeed, one cannot sharpen the result, at least in the setting of finite fields.

\begin{example}
    Let $k = \mathbf{F}_p$, with $P$ and $L$ consisting of all projective points and lines. Then $|P| = |L| = p^2 + p+1$, and we count
    %
    \[ I(P,L) = (p+1)(p^2+p+1) = \frac{p+1}{(p^2 + p + 1)^{1/2}} |L|^{3/4} |M|^{3/4} \]
    %
    As $p \to \infty$, $I(P,L) \sim |L|^{3/4} |M|^{3/4}$, so we obtain sharpness.
\end{example}

This means there is no purely affine proof of the Szemeredi-Trotter bound in $\RR^2$. But it remains an open question whether the only counterexamples in the two dimensional finite field setting are {\it only} obtained by taking very large sets, i.e. is there a threshold at which sparse sets of points and lines behave like Euclidean space.

Our goal in this talk is to discuss some results obtained by Koller for three dimensional incidences in arbitrary fields. A key idea here is to utilize the arithmetic genus to upper bound incidences, which is easily calculated using classical results of algebraic geometry.

We begin by discussing a variant of the Guth-Katz bound on $I(L)$, which works over a general class of fields. The proof given here is very close to what we discussed over distinct distances, but we must replace the parts of the proof which involve polynomial partitioning with alternative geometric results. We `almost' succeed to do this.

\begin{theorem}[Theorem 6]
    Let $L$ be a set of projective lines in a field of characteristic $0$, or of characteristic $p$ where $|L| \leq p^2$, and let $c > 0$ be a constant. If no plane or smooth quadric contains more than $c |L|^{1/2}$ lines, then $I(L) \leq (29.1 + 0.5c) |L|^{3/2}$.
\end{theorem}

This theorem differs from the Guth-Katz result in three important ways:
%
\begin{itemize}
    \item Guth-Katz only holds when $K = \RR$.
    \item Their bound counts the cardinality of incidences rather than multiplicity.
    \item Explicit constants are not obtained.
\end{itemize}
%
Nonetheless, we obtain Theorem 6 by working very similarly to Guth-Katz, except replacing applications of polynomial partitioning with a more geometric method.

%\begin{example}
%    NOTE TO DAN: THIS DOESN'T REALLY SHOW SHARPNESS IN THEOREM 6, RIGHT?

%    Consider the grid example, i.e. we let $L$ be the family of $3n^2$ axis parallel lines passing through the $n^3$ points in $\mathbf{Z}^3 \cap [1,n]^3$. For each point $p$ in this lattice, $\text{ord}_p(L) = 3$, and so $I(L) = 2n^3$. The only problem with this example is that two many lines lie in a single plane. But this is easily fixed. If we consider the Veronese-type embedding $(x,y,z) \mapsto (xyz,xy,xz,yz,x,y,z)$, then axis-parallel lines in $\RR^3$ remain straight in the embedding, whereas no hyperplane in $\RR^7$ contains more than two lines. Thus a generic projection from $\RR^7$ to $\RR^3$ gives a family of lines such that any plane contains no more than two lines. Thus the hypothesis of Theorem 1 is satisfied with $c = 1/n$, and so Theorem 1 shows that $I(L) \leq (29.1 + 0.5/n) 3^{3/2} n^3 = 151.2 n^3 + 2.59 n^2$
%\end{example}

\section{Proof of Theorem 6}

Given $L$, standard polynomial counting techniques enable us to find a surface $S$ containing all lines in $L$ with degree $d \leq (6|L|)^{1/2} - 2$. Decompose $S$ into irreducible components $S_1, \dots, S_k$, and let $L_i$ denote the set of lines that are contained in $S_i$, but not in $S_1, \dots, S_{i-1}$.

\begin{lemma}
    \[ I(L) = \underbrace{\sum_{i = 1}^K I(L_i)} + \underbrace{\sum_{i = 2}^k |L_i \cap (L_1 \cup \dots \cup L_{i-1}))}| \]
    %
    The first term is {\it Internal Incidences}, the second {\it External Incidences}.
\end{lemma}
\begin{proof}
    Note that $\{ L_i \}$ is a partition of $L$, and $\ord_p(L) = \sum \ord_p(L_i)$, so
    %
    \begin{align*}
        \sum_{i = 1}^k I(L_i) &= \sum_{i = 1}^k \sum_{p \in L_i} \ord_p(L_i) - 1\\
        &= \sum_{p \in L} \ord_p(L) - |\{ i: p \in L_i \}|\\
        &= I(L) - \sum_{i = 2}^k |L_i \cap (L_1 \cup \dots \cup L_{i-1})|. \qedhere
    \end{align*}
\end{proof}

Bounding external incidences is quite easy. And bounding the internal incidences when the surface is ruled is done analogously to Guth-Katz. Carrying out these calculations gives
%
\[ I(L) \leq 0.5c \cdot |L|^{3/2} + 1.23 \cdot |L|^{3/2} + \sum_{\text{non-ruled}\ S_i} I(L_i) \]
%
Because this is so similar to Guth-Katz, we omit the details. All the calculations to obtain this bound can be done over arbitrary fields. What is important is bounding incidences over non-ruled surfaces, which is where Guth-Katz applied polynomial partitioning. The novel feature of Koll\'{a}r's approach is that this can be completely avoided, provided one takes a recourse in looking at the arithmetic genus of a surface.

\section{Arithmetic Genus}

The multiplicative incidence number is a projective invariant of a family of lines. In particular, it seems plausible that the incidence number is related to other classical projective invariants in geometry. Kollar's observation was that the arithmetic genus of the union of liens is closely related to the incidence number, and furthermore, is not too difficult to calculate.

Recall the following notation:
%
\begin{itemize}
    \item Let $R = k[x_0, \dots, x_d]$.

    \item If $M$ is finitely generated over $R$, the {\it Hilbert function} is $H_M(t) = \dim_k(M_t)$.

    \item $H_M$ agrees with a degree $\leq d$ polynomial $P_M(t)$ for sufficiently large $t$.

%    \item Additivity: If $0 \to M \to N \to L \to 0$ is exact, then $P_N = P_M + P_L$.

    \item If $X$ is a projective variety/subscheme of $\PP^n$ corresponding to some homogenous ideal $I$, then $k[X] = R/I$ is the homogenous coordinate ring, and we let $H_X = H_{R/I}$, $P_X = P_{R/I}$.

    \item $P_X$ describes {\it all} additive invariants of $X$, i.e. invariants satisfying the rank nullity theorem. For instance,
    \begin{itemize}
        \item $\deg(P_X) = \dim(X)$.
        \item The leading coefficient is $\deg X / (\dim X)!$
        \item The constant coefficient is the Euler characteristic $\chi(X,\mathcal{O})$.
    \end{itemize}

    \item If $X$ is a projective curve, then one calls $g(X) = 1 - \chi(X,\mathcal{O})$ the {\it arithmetic genus} of $X$. This agrees with the topological genus if $X$ is a smooth curve over the complex numbers.
\end{itemize}

\begin{lemma}
    If $H = V(f)$ is a projective hypersurface of degree $a$, then
    %
    \[ P_{X \cap H}(t) = P_X(t) - P_X(t-a) \]
\end{lemma}
\begin{proof}
    If $f$ is homogenous of degree $a$, let $H$ be the projective variety corresponding to $f$. If $f$ is not a zero-divisor of $X$, the map $g \mapsto fg$ induces an exact sequence of vector spaces
    %
    \[ 0 \to k[X]_{t-a} \to k[X]_t \to k[X \cap H]_t \to 0 \]
    %
    Thus $P_X(t-a) + P_{X \cap H}(t) = P_X(t)$, for $t \geq a$.
\end{proof}

Consider projective hypersurfaces $H_1, \dots, H_{n-1}$, with degrees $a_1, \dots, a_{n-1}$. If $C = H_1 \cap \dots \cap H_{n-1}$ is one-dimensional, we say it is a {\it complete intersection curve}.

\begin{lemma}
    $g(C) = 1 + \frac{\sum a_i - (n + 1)}{2} \prod_{i = 1}^{n-1} a_i$.
\end{lemma}
\begin{proof}
     Since
    %
    \[ P_{\PP^n}(t) = {{t+n} \choose {n}} \]
    %
    Then using the last lemma recursively, we conclude
    %
    \[ P_C(t) = \left( \prod_{i = 1}^{n-1} a_i \right) \cdot t - \frac{\sum a_i - (n + 1)}{2} \prod_{i = 1}^{n-1} a_i. \]
    %
    Thus
    %
    \[ g(C) = 1 - \chi(C) = 1 + \frac{\sum a_i - (n + 1)}{2} \prod_{i = 1}^{n-1} a_i. \qedhere \]
\end{proof}

There are a few technicalities here, related to the fact that the intersections in the last two lemmas are {\it scheme theoretical}, rather than {\it set theoretical}. And the genus can disagree for these two objects.

\begin{example}
    Let $I_1 = (xy-zt)$, and $I_2 = (x^2 + xy - zt)$ be homogenous ideals, generating hyperboloids $H_1 = V(I_1)$ and $H_2 = V(I_2)$ in $\PP^3$. Let $C = H_1 \cap H_2$ be the scheme theoretic intersection. Then $I_1 \oplus I_2 = (x^2, xy - zt)$, and an easy algebraic calculation shows $x \not \in I_1 \oplus I_2$, but $x^2 \in I_1 \oplus I_2$. Thus $R/(I_1 \oplus I_2)$ is not a reduced ring, and the reduction of this ring is $R/I(x,zt)$ with corresponding `reduced curve' $C'$. The last lemma shows $P_C(t) = 4t$, yet $P_{C'}(t) = 2(t+1) = 2t + 2$. Thus $g(C) = 1$, but $g(C') = -1$. Note that both of these differ from the topological genus of the set, which is zero.
\end{example}

 Note, that in this example $g(C') \leq g(C)$. Using standard techniques in scheme cohomology, we can show that for {\it any} complete intersection curve $C$, if $C'$ is a reduced version of $C$, then
 %
 \[ g(C') \leq g(C) = 1 + \frac{\sum a_i - (n+1)}{2} \prod_{i = 1}^{n-1} a_i \]
 %
 We leave the details to the paper, which involves some basic sheaf cohomology. According to the paper, this inequality is intuitive: For families of curves, the arithmetic genus tends to jump up near the singular curves in the family, i.e. for curves where the scheme theoretic intersection disagrees with the set theoretic intersection.

 So why should arithmetic genus tell us about the incidences of a set of lines? Recall that for a degree $d$ curve $C$, $H_C(t) = dt + 1 - g(C)$. Thus among all degree $d$ curves, those with small genus have the most degree $t$ homogenous functions defined on them. If $L$ is a {\it disjoint} union of projective lines $\{ L_i \}$, then a degree $t$ regular function on $L$ is precisely given by degree $t$ functions on each individual line. However, if the lines in $L$ are allowed to intersect, then the degree $t$ regular functions on each line must agree at the intersection points. Thus the dimension of the space of degree $t$ regular functions must decrease, hence the genus increases. Thus there should be a relation between the incidence number and the arithmetic genus.

Now suppose $L$ is formed from $k$ projective lines $L_1, \dots, L_k$. For each line $L_i$, the embedding $e_i: L_i \to L$ induces a graded restriction homomorphism $e_i^*: k[L] \to k[L_i]$ given by $e_i^*(f) = f|_{L_i}$. We can put these together to obtain a map $e^*: k[L] \to \bigoplus k[L_i]$. We let $e^*_t: k[L]_t \to \bigoplus k[L_i]_t$ denote the restriction of $e^*$ to $k[L]_t$.

\begin{lemma}
    For $t \gg 0$, $g(L) = \dim(\emphcoker(e^*_t)) + k-1$.
\end{lemma}
\begin{proof}
    Both $L$ and $L^* = \coprod L_i$ have the same degree, so $P_L$ differs from $P_{L^*}$ by a constant coefficient. If we consider
    %
    \[ 0 \to k[L]_t \to \bigoplus k[L_i]_t \to \coker(e^*_t) \to 0 \]
    %
    The rank nullity theorem then implies that
    %
    \[ \dim k[L]_t + \dim (\coker(e^*_t)) = \sum_{i = 1}^k \dim k[L_i]_t \]
    %
    Thus $\dim(\coker(e^*_t)) = H_L(t) - H_{L^*}(t)$. Note that both $L$ and $L^*$ have the same degree, so $H_L$ differs from $H_{L^*}$ by a constant for sufficiently large $t$. We calculate that $P_{L^*}(t) = k(t+1)$, so $g(L^*) = 1-k$. Thus $g(L) = \dim(\coker(e^*)) + k-1$.
\end{proof}

\begin{example}
    Suppose $L$ is a family of $k$ lines intersecting at the origin in $\mathbf{P}^3$. Since $e$ preserves elements at $\infty$, $e^*$ preserves the affine degree of any element of $k[L]$, i.e. the degree of $f(1,x_1,x_2,x_3)$ is the same as $(e^*_i f)(1,t_0,t_1)$ for each $i$. If we decompose
    %
    \[ k[L]_t = k[L]_{t1} \oplus \dots \oplus k[L]_{tt}\ \ \ k[L_i]_t = k[L_i]_t = k[L_i]_{t1} \oplus \dots \oplus k[L_i]_{tt} \]
    %
    where $k[L]_{ts}$ is the space of homogenous polynomials with projective degree $t$ and affine degree $s$, and $e_{ts}^*: k[L]_{ts} \to \bigoplus k[L_i]_{ts}$ then
    %
    \[ \dim \coker(e_t^*) = \bigoplus_{s = 1}^t \dim ( \coker(e_{ts}^*)) \]
    %
    We use the elementary estimate
    %
    \[ \dim ( \coker(e_{ts}^* )) \geq \sum_{i = 0}^k \dim k[L_i]_{ts} - \dim k[L]_{ts} \geq k -  {s+2 \choose 2} \geq k - (s+2)^2/2 \]
    %
    This estimate is only useful if $i \leq (2k)^{1/2} - 2$. For $s = 0$, we obtain the $\dim(\coker(e_t^*)) \geq k-1$. This will suffice for our purposes, but we can also calculate that
    %
    \begin{align*}
        \dim(\coker(e_t^*)) &\geq \sum_{i = 1}^{\lfloor (2k)^{1/2} - 2 \rfloor} k - (i+2)^2/2 \geq 0.7 \cdot (k-1)^{3/2}\\
    \end{align*}
    %
    In particular, $g(L) \geq 0.7 \cdot (k-1)^{3/2} + (k-1)$, which is more useful in point-incidence type bounds.
\end{example}

We can put this together to get bounds on general $k$ element line incidences $L$. Given any point of incidence $p \in L$, let $L(p)$ denote all lines through $p$. We consider the restriction map $f^*: k[L] \to \bigoplus k[L(p)]$, and $e^*$ factors through $f^*$, and this factor map on each $L(p)$ is precisely the one considered in the last example. Thus
%
\[ \dim(\coker(e^*)) \geq \sum \dim(\coker(e^*_p)) \geq \sum_p (\ord_p(L) - 1) \]
%
%\[ \dim(\coker(e^*)) \geq \sum \dim(\coker(e^*_p)) \geq \sum_p 0.7 \cdot (r(p)-1)^{3/2}. \]
%
Thus $g(L) \geq \sum_p (\ord_p(L) - 1)$.

Now we put together this calculation with the theory of complete intersection curves. Let $L'$ be lines lying on the non-ruled components $S'$ of $S$. If $T$ denotes the surface of degree $\leq 11d - 24$ corresponding to the flecnode polynomial of $S$, then $S \cap T$ contains all the lines $L$. Monge's theorem implies that $T$ contains no component of $S$, so $S \cap T$ is one dimensional, and thus a complete intersection curve. Thus
%
\begin{align*}
    I(L') &= \sum \ord_p(L) - 1 \leq g(L) \leq g(S \cap T)\\
    &\leq 0.5 \cdot (d(11d - 24)(d + (11d - 24) - 2)) \leq 66d^3 \leq 66(6|L'|)^{3/2} \leq 970 |L'|^{3/2} \leq 970 |L|^{3/2}
\end{align*}
%
We therefore find that $I(L) \leq c |L|^{3/2} + 1000 |L|^{3/2}$

%Thus $g(L) \geq 0.7 \cdot (r(p)-1)^{3/2} + k-1$.

\end{document}











\begin{lemma}
    If the $S_i$ are appropriately ordered,
    %
    \[ \sum_{i = 2}^k |L_i \cap (L_1 \cup \dots \cup L_{i-1})| \leq 1.23 |L|^{3/2}. \]
\end{lemma}
\begin{proof}
    Suppose $l \in L_i$. Then $l$ is not contained in $S_1, \dots, S_{i-1}$. Bezout's theorem thus implies that $l \cap S_j$ contains at most $\deg(S_j)$ points if $j < i$. But $l \cap L_j \subset l \cap S_j$, so a union bound gives
    %
    \[ \sum_{i = 2}^k |L_i \cap (L_1 \cup \dots \cup L_{i-1})| \leq \sum_{j < i} |L_i| \deg(S_j) \leq \sum |L_i| \deg(S_j) = d|L| \leq 6^{1/2} |L|^{3/2} \]
    %
    We can do slightly better by reordering the $S_i$. If $|L_i|/\deg(S_i)$ is monotonically decreasing, then $|L_j|/\deg(S_j) \geq |L_i|/\deg(S_i)$
    %
    \[ \sum_{j < i} |L_i| \deg(S_j) \leq \sum_{j < i} \deg(S_i) |L_j| \]
    %
    Note that
    %
    \[ 2 \sum_{j < i} |L_i| \deg(S_j) \leq \sum_{j < i} |L_i| \deg(S_j) + \sum_{j > i} |L_i| \deg(S_j) \leq 6^{1/2} |L|^{3/2} \]
    %
    Thus $\sum_{j < i} |L_i| \deg(S_j) \leq (3/2)^{1/2} |L|^{3/2}$. This can be tight if $|L_i|\deg(S_j) = |L_j| \deg(S_i)$, and $|L_i|\deg(S_i)$ is small in comparison to the overall sum.
\end{proof}

We now bound internal incidences. Just as in the distinct distances problem, we must separate incidences on non-ruled surfaces from incidences on ruled surfaces, and further separate analysis on planes and quadrics from analysis on the other ruled surfaces.

\begin{lemma}
    If $S_i$ is a plane, $I(L_i) \leq 0.5c \cdot |L|^{1/2} |L_i|$.
\end{lemma}
\begin{proof}
    By assumption, $|L_i| \leq c |L|^{1/2}$. Since we are working in a projective context, any two lines in $L_i$ intersect one another. The number of incidences is maximized if any pair of lines intersects in a distinct location. Thus we find $I(L_i) \leq |L_i|^2/2$.
\end{proof}

\begin{lemma}
    If $S_i$ is a singular quadric (a cone), $I(L_i) = |L_i| - 1$.
\end{lemma}
\begin{proof}
    A singular quadric has a vertex, and any straight line on the surface passes through the vertex. This gives the result directly.
\end{proof}

\begin{lemma}
    If $S_i$ is a smooth quadric, $I(L_i) \leq 0.5c \cdot |L_i| |L|^{1/2}$.
\end{lemma}
\begin{proof}
    Since $S_i$ is a smooth quadric, the family of lines on $S$ splits into two rulings. Write $L_i = L_i^0 \cup L_i^1$, where $L_i^0$ is the first ruling, and $L_i^1$ the second ruling, and no lines intersect in the same family. Thus
    %
    \[ I(L_i) \leq |L_i^0| |L_i^1| \leq |L_i|^2/4 \leq 0.5c \cdot |L|^{1/2}. \qedhere \]
\end{proof}

We say a ruled surface is `non-exception' if it isn't a plane, cone, or quadric.

\begin{lemma}
    If $S_i$ is a `non-exceptional' ruled surface, then
    %
    \[ I(L_i) \leq 0.5 \cdot |L_i| \deg(S_i) + 2 |L_i|. \]
\end{lemma}
\begin{proof}
    If $S_i$ is a non-special ruled surface, then there are at most two disjoint lines on $S_i$, disjoint from each other, which intersect more than $\deg(S_i)$ lines, known as special. If we set $L_i = L_i^0 \cup L_i^1$, where $L_i^0$ are the special lines, then
    %
    \[ I(L_i) = I(L_i^0) + I(L_i^1) + |L_i^0 \cap L_i^1| \leq 0 + 0.5 \cdot \deg(S_i) \cdot |L_i^1| + 2|L_i^1|.\qedhere \]
%        &\leq 0.5 \cdot |L_i| \deg(S_i) + 2 |L_i|. \qedhere
\end{proof}

Thus
%
\begin{align*}
    \sum_{S_i\ \text{ruled}} I(L_i) &= \left( \sum_{\text{planes}} + \sum_{\text{smooth quadric}} \right) + \left( \sum_{\text{cones}} + \sum_{\text{non exceptional}} \right)\\
    &\leq \left( \sum 0.5c \cdot |L|^{1/2} |L_i| \right) + \left( 0.5 \cdot \sum |L_i| \deg(S_i) \right)\\
    &\leq \left( 0.5c \cdot |L|^{3/2} \right) + \left( 0.5 d |L| \right)\\
    &\leq 0.5c \cdot |L|^{3/2} + 1.23 \cdot |L|^{3/2}
\end{align*}
%
All these bounds are essentially as expected from the Guth-Katz result. 



\section{Salmon's Construction}

TODO: Define a ruled surface.

\begin{theorem}
    Let $f$ be a degree $d$ polynomial. Then there is a polynomial $\text{Flec}(f)$ of degree no more than $11d - 24$ which vanishes at every point $p \in V(f)$ for which there is a triple tangent for $f$ through $p$.
\end{theorem}

\begin{corollary}
    If $S$ is a surface of degree $d$, such that no irreducible component is ruled, then the surface $T$ with degree at most $11d - 24$ shares no irreducible components with $S$.
\end{corollary}

We say a surface $S$ is {\it ruled} if it contains a line through a generic point.

\begin{corollary}
    If $\text{Flec}(f)$, then 
\end{corollary}

If a line $l$ through $p$ is parameterized as $t \mapsto p + mt$, and $f$ is a polynomial 

In particular, $T$ contains every line that $S$ also contains.

\begin{theorem}[Monge-Salmon-Cayley]
    Let $S \subset \mathbf{CP}^3$ be a surface of degree $d$ without ruled irreducible components. Then there is a surface $T$ of degree at most $11d - 24$ such that $S$ and $T$ do not have common irreducible components and every line on $S$ is contained in $S \cap T$.
\end{theorem}
\begin{proof}
    Consider three homogenous polynomials in $\PP^2$:
    %
    \[ f_1(x) = \sum_{1 \leq i \leq 3} a_ix_i\ \ \ f_2(x) = \sum_{1 \leq i \leq j \leq 3} b_{ij} x_i x_j\ \ \ f_3(x) = \sum_{1 \leq i \leq j \leq k \leq 3} c_{ijk} x_i x_j x_k \]
    %
    %
    We want to understand when $V(f_1) \cap V(f_2) \cap V(f_3)$ is non-empty. If $x \in V(f_1)$, we know
    %
    \[ x_3 = -\frac{a_1x_1+a_2x_2}{a_3}. \]
    %
    Then substituting this into $f_2$ and $f_3$ induces forms of degree two and three in the variables $x_2$ and $x_3$, i.e.
    %
    \[ g_1(x_1,x_2) = f_1(x_1,x_2,(a_1x_1 + a_2x_2)/a_3) = \sum_{1 \leq i \leq j \leq 2} B_{ij} x_ix_j \]
    \[ g_2(x_1,x_2) = f_2(x_1,x_2,(a_1x_1 + a_2x_2)/a_3) = \sum_{1 \leq i \leq j \leq k \leq 2} C_{ijk} x_ix_jx_k \]
    %
    Now consider an irreducible surface $S$ given by an equation $f$. Fix $p \in S$, and 
\end{proof}

\section{Intersections of Surfaces}

Given the two surfaces $S$ and $T$, we now need to study it's intersection, which contains all lines in $S$. We rely on the higher dimensional version of Bezout's theorem: If $H_1, \dots, H_n$ are hypersurfaces with degrees $d_1, \dots, d_n$, then $H_1 \cap \dots \cap H_n$ contains an algebraic curve, or consists of at most $d_1 \dots d_n$ points.

\begin{theorem}
    Let $S,T \subset \mathbf{P}^3$ be two surfaces of degree $a$ and $b$ with no common irreducible components. Let $C = S \cap T$. Then
    %
    \begin{enumerate}
        \item $C$ has at most $ab$ irreducible components.
        \item $\sum_{p \in C} (r(p) - 1) \leq ab(a+b-2)/2$.
        \item $\sum_{p \in C} (r(p) - 1)^{3/2} \leq ab(a+b-2)/\sqrt{2}$.
        \item $\sum \left\{ r(p)(r(p) - 1) : p \in C\ \text{and} p\ \text{is smooth on $S$ or $T$} \right\}$
    \end{enumerate}
\end{theorem}
\begin{proof}
    If $H$ is a generic hyperplane, then $C \cap H = S \cap T \cap H$ consists of $ab$ points. Thus $C$ has at most $ab$ irreducible components (and if this number is achieved, $C$ is a union of lines).
\end{proof}