\documentclass{article}

\usepackage{amsmath}
\usepackage{amssymb}
\usepackage{amsthm}

\theoremstyle{plain}
\newtheorem{theorem}{Theorem}
\newtheorem{lemma}[theorem]{Lemma}
\newtheorem{corollary}[theorem]{Corollary}
\newtheorem{prop}[theorem]{Proposition}

%\theoremstyle{remark}
\newtheorem*{example}{Example}
\newtheorem*{proof*}{Proof}

\theoremstyle{definition}
\newtheorem*{defi}{Definition}
\newenvironment{definition}
    {\begin{samepage}\begin{framed}\begin{defi}}
    {\end{defi}\end{framed}\end{samepage}}

\title{The Fourier Stieltjes Integral}
\author{Jacob Denson}
\date{November 23rd, 2016}

\begin{document}

\maketitle

With the help of the Gelfand representation on $L^1(G)$, we are able to discern a generalized Fourier transform on any compact group, which transforms elements of $L^1(G)$ into functions on the group $\Gamma(G)$ of characters on $G$. The functions in the transformed domain are normally easier to understand than the original functions we intended to study; they are continuous, vanish at infinity, and transform the fundamental operation of convolution into pointwise multiplication. An incredibly important part of the transform operator is that it has an integral representation
%
\[ \widehat{f}(\phi) = \int \frac{f(x)}{\phi(x)} dx \]
%
which makes the operator more tractable, since we can use tools from both operator and integration theory to understand the transformation. We would hope that we might be able to find a nice representation of the character transform on $M(G)$, since the algebraic structure of $M(G)$ is a relatively simple extension of $L^1(G)$. Such a representation, if it exists at all, remains elusive\footnote{A strategy developed in \cite{taylor} indicates a representation of $M(G)^*$ as an algebra of continuous functions over a new group $G'$, but the group $G'$ is normally pathologically complicated, and as such we shall not discuss this representation here.}. We can, however, naively extend the Fourier transform to $M(G)$, since integration is defined on all of $M(G)$, not just $L^1(G)$. The algebra homomorphism obtained is known as the {\it Fourier-Stieltjes transform}, defined for all measures by the formula
%
\[ \widehat{\mu}(\phi) = \int \frac{d\mu(x)}{\phi(x)} \]
%
The topic of this article is to justify why the generalization is a natural generalization of the transform on $L^1(G)$. We shall find that an explanation exists through the perspective of theoretical physics, from which the Fourier transform originally emerged from problems in the study of heat dispersion.

From a practical perspective, the Stieltjes transform is useful because it still has many of the important properties of the Fourier transform. It's still a homomorphism with respect to convolution, because
%
\[ \widehat{\mu * \nu}(\phi) = \int \frac{d(\mu * \nu)(x)}{\phi(x)} = \int \frac{d \mu (x) d \nu(y)}{\phi(xy)} = \int \frac{d \mu(x)}{\phi(x)} \frac{d \nu(y)}{\phi(y)} = \widehat{\mu}(\phi) \widehat{\nu}(\phi) \]
%
Unfortunately, the functions obtained cannot vanish at infinity; $M(G)$ has a unit, unlike $L^1(G)$, so the transform must contain constant functions. However, we still have continuity (and in fact, uniform continuity as well). This is easy to verify for a measure $\mu$ supported on a compact set $K$, because if $L_\psi$ is the left translation operator (by $\psi \in \Gamma(G)$), then we have the inequality
%
\[ | L_\psi \widehat{\mu}(\phi) - \widehat{\mu}(\phi)| = \left| \int_K \frac{1 - \psi(x)}{\psi(x) \phi(x)} d\mu \right| \leq \int_K |1 - \psi| d|\mu| \]
%
The family
%
\[ U_{K,\varepsilon} = \{ \phi \in \Gamma(G) : |1 - \phi(x)| < \varepsilon\ \text{for all}\ x \in K \} \]
%
for $K \subset G$ compact is a neighbourhood basis of the origin in $\Gamma(G)$, and if $\psi$ is constrained to be contained in this neighbourhood then $\| L_\psi \widehat{\mu} - \widehat{\mu} \|_\infty < \varepsilon \| \mu \|$. Since we may approximate all finite Radon measures by compactly supported measures, and the Fourier transform is continuous as a map from the topology induced by the variation norm to the topology induced by the uniform norm, the theorem is then proven for all measures by approximation.

Physical considerations brought rise to the Fourier transform, so it is useful to look back to physics to understand why the Stieltjes transform is a good generalization of the Fourier transform. Theoretical physics often defines experimental quantites defined at points (such as the distribution of electric charge and masses across a region) as limiting quantities of averages over regions. In order to verify this is well defined, they essentially use the Lebesgue-differentiation theorem to find a function $f \in L^1(G)$ such that the equation
%
\[ f(x) = \lim_{r \to 0} \frac{1}{v(B_r(x))} \int_{B_r(x)} f(x) dx = \lim_{r \to 0} F_r(x) \]
%
holds around almost every point $x$. However, problems with this strategy occur when one wants to discuss point masses and point charges, which possess properties which the integrals of functions do not possess. For a point mass or charge, we would need to find a distribution function $f$ which satisfies
%
\[ \int_Y f(x) dx = \begin{cases} 1 & y \in Y \\ 0 & y \not \in Y \end{cases} \]
%
Then the formula obtained from the Lebesgue-differentiation method implies that $f(x) = 0$, except possibly at $y$, yet it somehow still integrates to one over the entire space. We would like to define $f$ as the limit of the functions
%
\[ F_r(x) = \frac{\mathbf{I}(y \in B_r(x))}{v(B_r(x)} \]
%
but this limit does not exist by any of the standard analytical procedures.



It is much more elegant to define charge or mass, not in terms of functions, but instead representing the quantity in terms of a measure defined on space. Indeed, this enables us to define the dirac delta function as a discrete measure, which gives us the properties that the point mass functions required earlier. In this context, we see the issue with the Lebesgue differentiation theorem: we are trying to take the Radon-Nikodym derivative of a function which isn't absolutely continuous with respect to the Lebesgue measure. An additional advantage of the measure theoretic approach is the elegance of handling more tricky situations. Physicists may also need to take an electric charge distribution over a hypersurface in three dimensional space, and while geometric intuition about the Dirac delta function begins to break down when `infinities' occur along an entire region of space, measure theorists can just take the Radon Nikodym derivative with respect to the two dimensional Hausdorff measure to obtain a well defined function of study across the surface.

Nonetheless, it is often useful to think of charge distributions as limitations of functions of the form $F_r$ (this is why theoretical physicists really approach the theory from this perspective), and this thought process actually has an interpretation in the general theory of the Stieltjes transform, because the Stieltjes transform on $M(G)$ is essentially obtained from limits of the Fourier transform of functions of $L^1(G)$.

\begin{theorem}
    If $f_U$ is one of the canonical approximate identities in $L^1(G)$, then $\widehat{f_U} \to 1$ pointwise.
\end{theorem}
\begin{proof}
    Let $\phi \in \Gamma(G)$ be arbitrary. Then
    %
    \[ \widehat{f_U}(\phi) = \int \frac{f_U(x)}{\phi(x)} dx \]
    %
    Fix some precompact $U$ and $\phi \in \Gamma(G)$. Then, by modifying $\phi$ outside $U$ so it is in $L^1(G)$, we find
    %
    \[ \widehat{f_U}(\phi) = (f_U * \phi)(e) \to \phi(e) = 1 \]
    %
    Since $\phi$ was arbitrary, $\widehat{f_U} \to 1$.
\end{proof}

We could have introduced the Stieltjes transform of an arbitrary measure as the pointwise limit of the Fourier transforms of the functions $f_U * \mu \in L^1(G)$, since $\widehat{f_U * \mu} = \widehat{f_U} * \widehat{\mu} \to \widehat{\mu}$. This essentially brings us back to the situation in physics; if $G = \mathbf{R}^n$, and $f_r$ is the normalized characteristic function of $B_r$ around the origin, then we find
%
\[ (f_r * \mu)(x) = \frac{1}{v(B_r(x))} \int_{B_r(x)} d\mu \]
%
At least as the Fourier transform is concerns, measures really are `limits' of functions in $L^1(G)$. The physicists were right all along!

To obtain further results along this line of thought, we require some further results from the general theory of abstract harmonic analysis, a theorem which is true for the Fourier analysis of any abelian locally compact group.

\begin{theorem}
    The Fourier-Stieltjes transform is injective. That is, if $\mu$ is a measure for which $\widehat{\mu} = 0$, then $\mu = 0$.
\end{theorem}

\begin{theorem}
    $L^1(G)$ is weak-$*$ dense in $M(G)$, when $G$ is abelian.
\end{theorem}
\begin{proof}
    Let $\{ f_U \}$ be an approximate identity for $L^1(G)$. By standard constructions we may assume $\| f_U \| \leq 1$, and since the unit ball of $M(G)$ is weak-$*$ compact, we may let $f_U dx$ converge weakly to some measure $\nu$ by taking a subnet. We have already verified that $\widehat{f_U} \to 1$ pointwise. This implies that if $\mu \in M(G)$ is arbitrary, then $\widehat{f_U * \mu} \to \widehat{\mu}$ pointwise. But $\widehat{f_U * \mu} = \widehat{f_U} \widehat{\mu} \to \widehat{\nu} \widehat{\mu}$ pointwise, so $\widehat{\mu} = \widehat{\nu} \widehat{\mu} = \widehat{\nu * \mu}$, implying by injectivity that $\mu = \nu * \mu$, and so
    %
    \[ \mu = \nu * \mu = \left( \lim f_U \right) * \mu = \lim f_U * \mu \]
    %
    so $f_U$ is an approximate identity on $M(G)$ as well as $L^1(G)$, and the elements $f_U * \mu$ in the above equation can be seen as elements of $L^1(G)$, so that $L^1(G)$ can weak-$*$ approximate all measures in $M(G)$.
\end{proof}

Thus we finally see why the Fourier-Stieltjes transform is a natural extension of the Fourier transform on $L^1(G)$ -- it is the unique continuous extension of the transform, relative to the weak $*$ topology on $M(G)$, and the pointwise topology of $\mathbf{C}^{\Gamma(G)}$. Conversely, this also tells us why the Fourier transform is useful -- it is a specialization of the Stieltjes transform which is actually calculatable, while still representing most of the properties of the Stieltjes transform in the first place. We can study $L^1(G)$ instead of $M(G)$ because, at least with respect to integration theory, the two spaces are the same. This is one of the reasons harmonic analysis is so easy to study when $G$ is abelian -- functions are just more amenable to integration than measures are.

\begin{thebibliography}{9}

\bibitem{folland}
    Gerald Folland,
    \emph{A Course in Abstract Harmonic Analysis},
    CRC Press,
    2015.

\bibitem{follandreal}
    Gerald Folland
    \emph{Real Analysis: Modern Techniques and Applications}
    Wiley,
    1999.

\bibitem{rudin}
    Walter Rudin,
    \emph{Fourier Analysis on Groups},
    Wiley,
    1990.

\bibitem{taylor}
    Joseph L. Taylor,
    \emph{The Structure of Convolution Measure Algebras},\\
    Transactions of the American Mathematical Society,
    1965.

\end{thebibliography}

\end{document}

\begin{lemma}
    If $\mu_\alpha$ weak-$*$ converges to $\mu$, and $\nu \in M(G)$, then $\mu_\alpha * \nu$ weak-$*$ converges to $\mu * \nu$. In other words, convolution is continuous in each variable.
\end{lemma}
\begin{proof}
    If $\mu_\alpha$ converges weak-$*$ to $\mu$, $f \in C_c(G)$, and $\nu$ has compact support then by weak-$*$ convergence, we find
    %
    \[ \int f d(\nu * \mu_\alpha) = \int f(xy) d\nu(x) d\mu_\alpha(y) \to \int f(xy) d\nu(x) d\mu(y) = \int f d(\nu * \mu) \]
    %
    Which completes the needed calculation, provided that the map $y \mapsto \int f(xy) d\nu(x)$ is in $C_c(G)$, which is true by applying uniform continuity, since
    %
    \[ \left| \int [f(xy) - f(xy')] d\nu(x) \right| \leq \| \nu \| \| f(xy) - f(xy') \|_\infty \to 0 \]
    %
    If $\nu$ does not necessarily have compact support, we need only approximate it by functions which do, for
    %
    \begin{align*}
        &\left| \int f d(\nu * \mu_\alpha) - \int f d(\nu * \mu) \right|\\
        &\leq \left| \int f d((\nu - \nu_\beta) * \mu_\alpha) \right|\\
        &\ \ \ \ \ \ + \left| \int f d (\nu_\beta * \mu_\alpha) - \int f d (\nu_\beta * \mu) \right|\\
        &\ \ \ \ \ \ + \left| \int fd((\nu_\beta - \nu) * \mu) \right|\\
        &\leq \| f \|_\infty \| \nu - \nu_\beta \| \left( \| \mu_\alpha \| + \| \mu \| \right) + \left| \int f d (\nu_\beta * \mu_\alpha) - \int f d (\nu_\beta * \mu) \right|
    \end{align*}
    %
    If the $\mu_\alpha$ are bounded, we may first fix $\beta$ such that $\| \nu - \nu_\beta \| < \varepsilon$, and then let $\alpha$ be large enough such that the whole term becomes miniscule.
\end{proof}