\documentclass[12pt, dvipsnames]{report}

\usepackage{amsmath}
\usepackage{algorithm}
%\usepackage{algorithmic}
\usepackage[noend]{algpseudocode}

\usepackage{amsmath}
\usepackage{amssymb}
\usepackage{amsthm}
\usepackage{amsopn}

\usepackage{kpfonts}

\usepackage{graphicx}

% Probably don't need this on notes anymore
%\usepackage{kbordermatrix}

% Standard tool for drawing diagrams.
\usepackage{tikz}
\usepackage{tkz-berge}
\usepackage{tikz-cd}
\usepackage{tkz-graph}

\usepackage{comment}

%
\usepackage{multicol}

%
\usepackage{framed}

%
\usepackage{mathtools}

%
\usepackage{float}

%
\usepackage{subfig}

%
\usepackage{wrapfig}

%
\let\savewideparen\wideparen
\let\wideparen\relax
\usepackage{mathabx}
\let\wideparen\savewideparen

% Used for generating `enlightening quotes'
\usepackage{epigraph}

% Forget what this is used for :P
\usepackage[utf8]{inputenc}

% Used for generating quotes.
\usepackage{csquotes}

% Allows what to generate links inside
% generated pdf files
\usepackage{hyperref}

% Allows one to customize theorem
% environments in mathematical proofs.
\usepackage{thmtools}

% Gives access to a proof
\usepackage{lplfitch}

% I forget what this is for.
\usepackage{accents}

% A package for drawing simple trees,
% as a substitute for unnesacary TIKZ code
\usepackage{qtree}

% Enables sequent calculus proofs
\usepackage{ebproof}

% For braket notation
\usepackage{braket}

% To change line spacing when using mathematical notations which require some height!
\usepackage{setspace}

%\usepackage[dvipsnames]{xcolor}

\usepackage{float}

% For block commenting
\usepackage{comment}




\setlength\epigraphwidth{8cm}

\usetikzlibrary{arrows, petri, topaths, decorations.markings}

% So you can do calculations in coordinate specifications
\usetikzlibrary{calc}
\usetikzlibrary{angles}

\theoremstyle{plain}
\newtheorem{theorem}{Theorem}[chapter]
\newtheorem{axiom}{Axiom}
\newtheorem{lemma}[theorem]{Lemma}
\newtheorem{corollary}[theorem]{Corollary}
\newtheorem{prop}[theorem]{Proposition}
\newtheorem{exercise}{Exercise}[chapter]
\newtheorem{fact}{Fact}[chapter]

\newtheorem*{example}{Example}
\newtheorem*{proof*}{Proof}

\theoremstyle{remark}
\newtheorem*{exposition}{Exposition}
\newtheorem*{remark}{Remark}
\newtheorem*{remarks}{Remarks}

\theoremstyle{definition}
\newtheorem*{defi}{Definition}

\usepackage{hyperref}
\hypersetup{
    colorlinks = true,
    linkcolor = black,
}

\usepackage{textgreek}

\makeatletter
\renewcommand*\env@matrix[1][*\c@MaxMatrixCols c]{%
  \hskip -\arraycolsep
  \let\@ifnextchar\new@ifnextchar
  \array{#1}}
\makeatother

\renewcommand*\contentsname{\hfill Table Of Contents \hfill}

\newcommand{\optionalsection}[1]{\section[* #1]{(Important) #1}}
\newcommand{\deriv}[3]{\left. \frac{\partial #1}{\partial #2} \right|_{#3}} % partial derivative involving numerator and denominator.
\newcommand{\lcm}{\operatorname{lcm}}
\newcommand{\im}{\operatorname{im}}
\newcommand{\bint}{\mathbf{Z}}
\newcommand{\gen}[1]{\langle #1 \rangle}

\newcommand{\End}{\operatorname{End}}
\newcommand{\Mor}{\operatorname{Mor}}
\newcommand{\Id}{\operatorname{id}}
\newcommand{\visspace}{\text{\textvisiblespace}}
\newcommand{\Gal}{\text{Gal}}

\newcommand{\xor}{\oplus}
\newcommand{\ft}{\wedge}
\newcommand{\ift}{\vee}

\newcommand{\prob}{\mathbf{P}}
\newcommand{\expect}{\mathbf{E}}
\DeclareMathOperator{\Var}{\mathbf{V}}
\newcommand{\Ber}{\text{Ber}}
\newcommand{\Bin}{\text{Bin}}

%\newcommand{\widecheck}[1]{{#1}^{\ft}}

\DeclareMathOperator{\diam}{\text{diam}}

\DeclareMathOperator{\QQ}{\mathbf{Q}}
\DeclareMathOperator{\ZZ}{\mathbf{Z}}
\DeclareMathOperator{\RR}{\mathbf{R}}
\DeclareMathOperator{\HH}{\mathbf{H}}
\DeclareMathOperator{\CC}{\mathbf{C}}
\DeclareMathOperator{\AB}{\mathbf{A}}
\DeclareMathOperator{\PP}{\mathbf{P}}
\DeclareMathOperator{\MM}{\mathbf{M}}
\DeclareMathOperator{\VV}{\mathbf{V}}
\DeclareMathOperator{\TT}{\mathbf{T}}
\DeclareMathOperator{\LL}{\mathcal{L}}
\DeclareMathOperator{\EE}{\mathbf{E}}
\DeclareMathOperator{\NN}{\mathbf{N}}
\DeclareMathOperator{\DQ}{\mathcal{Q}}
\DeclareMathOperator{\IA}{\mathfrak{a}}
\DeclareMathOperator{\IB}{\mathfrak{b}}
\DeclareMathOperator{\IC}{\mathfrak{c}}
\DeclareMathOperator{\IP}{\mathfrak{p}}
\DeclareMathOperator{\IQ}{\mathfrak{q}}
\DeclareMathOperator{\IM}{\mathfrak{m}}
\DeclareMathOperator{\IN}{\mathfrak{n}}
\DeclareMathOperator{\IK}{\mathfrak{k}}
\DeclareMathOperator{\ord}{\text{ord}}
\DeclareMathOperator{\Ker}{\textsf{Ker}}
\DeclareMathOperator{\Coker}{\textsf{Coker}}
\DeclareMathOperator{\emphcoker}{\emph{coker}}
\DeclareMathOperator{\pp}{\partial}
\DeclareMathOperator{\tr}{\text{tr}}

\DeclareMathOperator{\supp}{\text{supp}}

\DeclareMathOperator{\codim}{\text{codim}}

\DeclareMathOperator{\minkdim}{\dim_{\mathbf{M}}}
\DeclareMathOperator{\hausdim}{\dim_{\mathbf{H}}}
\DeclareMathOperator{\lowminkdim}{\underline{\dim}_{\mathbf{M}}}
\DeclareMathOperator{\upminkdim}{\overline{\dim}_{\mathbf{M}}}
\DeclareMathOperator{\lhdim}{\underline{\dim}_{\mathbf{M}}}
\DeclareMathOperator{\lmbdim}{\underline{\dim}_{\mathbf{MB}}}
\DeclareMathOperator{\packdim}{\text{dim}_{\mathbf{P}}}
\DeclareMathOperator{\fordim}{\dim_{\mathbf{F}}}

\DeclareMathOperator*{\argmax}{arg\,max}
\DeclareMathOperator*{\argmin}{arg\,min}

\DeclareMathOperator{\ssm}{\smallsetminus}

\title{Functional Analysis}
\author{Jacob Denson}

\begin{document}

\pagenumbering{gobble}
\maketitle
\tableofcontents
\pagenumbering{arabic}


\part{Banach Spaces}

\chapter{Introduction}

Functional analysis is the interlace of algebra and analysis, in which algebraic structures are endowed with topological structure. The approach's utility counts for the rapid growth of applications over the past century, be it in quantum mechanics, statistics, or computing science. Rarely are we concerned with a single object, like a function, a random variable, or a measure, but instead consider large classes of such objects. One technique for handling these classes is to add algebraic structure which elaborates on the natural relation between these objects. Functional analysis provides general tools for this methodology.

\begin{example}
    We rarely analyze a measurable function $f$ in isolation. Instead, we prove theorems about a class of measurable functions defined on the same measurable space. If $f$ and $g$ are measurable, then we may consider their addition $f + g$, their multiplication $fg$, and scaling $\lambda f$ (for $\lambda \in \mathbf{R}$), which are all measurable. Thus the space of measurable functions on a set is a vector space. Similarily, we may consider $C[0,1]$, the space of all continuous, real-valued functions on $[0,1]$.
\end{example}

Functional analysis mostly applies to functions between topological and algebraic structures, because we may add and scale things quite naturally, and we obtain topologies based on how functions converge. The power of functional analysis rests on how strongly the algebraic structure of a space connects to the topology. Stronger relations result in stronger theorems.

Most important spaces work over a field of characteristic zero. For analytic niceties, we also require our spaces to be complete. We therefore begin with real vector spaces, for a complete rational vector space can always be manufactured into a real vector space. Real spaces do not permit much generalization, apart from considering complex spaces. Thus, in this book, $\mathbf{F}$ will stand for either $\mathbf{R}$ or $\mathbf{C}$. Real vector spaces are spaces which can be naturally `scaled'. Complex vector spaces can also be `twisted'. As an example, We may scale functions into the real numbers pointwise, so the space $C([0,1], \mathbf{R})$ of continuous, real valued functions is an $\mathbf{R}$ vector space. On the other hand, functions in $C([0,1], \mathbf{C})$ can be twisted; each function in the space can be visualized as a corkscrew -- the domain is the length, the range is the extremities of the screw. Twisting a corkscrew is natural, so we have a complex vector space.

\chapter{Hilbert Spaces}

The nicest vector space is $\mathbf{R}^n$. It is here that we may measure angles and distance, which give rise to the topological properties of a space. Recall that the inner product on $\mathbf{R}^n$ is naturally related to the measure of angles between vectors. Without an inner product, we can only talk about linear dependance, which is a binary relation, hardly suited to analysis. With an inner product, we obtain an abstract definition of angle, which is a continuous degree of similarity between two vectors.

\begin{definition}
    An {\bf inner product space} is an $\mathbf{R}$ vector space $V$ with a positive-definite inner product $\langle \cdot, \cdot \rangle : V^2 \to \mathbf{R}$: in particular, given any $\lambda, \gamma \in \mathbf{R}$, $v,w, u \in V$,
    %
    \[ \langle v, w \rangle = \langle w, v \rangle\ \ \ \ \ \langle \lambda v + \gamma w, u \rangle = \lambda \langle v, u \rangle + \gamma \langle w, u \rangle \]
    %
    \[ \langle v, v \rangle \geq 0\ \ \ \ \ \ \ \langle v, v \rangle = 0\ \text{if and only if}\ v = 0 \]
\end{definition}

We would also like to consider complex inner product spaces, but we have a problem. If $\langle \cdot, \cdot \rangle$ is defined to be positive definite, then we have
%
\[ 0 \leq \langle i v, i v \rangle = - \langle v, v \rangle \leq 0 \]
%
so $\langle v, v \rangle = 0$ for all $v$, hence $v = 0$. The problem results because we are restricted to considering real values of the product, yet we can vary $v$ by a complex coefficient.

To determine how a `complex inner product' results, let us return to the fundamentals of forming a complex vector space. Let $V$ be a real vector space, and let $J$ be a linear map with $J^2 = -1$. Then $J$ is a way of `twisting' $V$, so we may make $V$ into a complex vector space by defining
%
\[ (\lambda + \gamma i) v = \lambda v + \gamma J(v) \]
%
Thus the twist is just multiplication by $i$. A choice of $J$ can be obtained from any vector space, but it is usually only useful when formed naturally. If we already have a {\it real} inner product $\langle \cdot, \cdot \rangle$, then $J$ should twist ninety degrees -- for any $v \in V$,
%
\[ \langle v, Jv \rangle = 0 \]
%
We would also like this rotation to be uniform, so that
%
\[ \langle Jv, Jw \rangle = \langle v, w \rangle \]
%
Then, since $J^2 = -1$,
%
\[ \langle Jv, w \rangle = \langle Jv, J^2(-w) \rangle = \langle v, -Jw \rangle \]
%
We wish to extend $\langle \cdot, \cdot \rangle$ to a complex `form' $(\cdot, \cdot)$, in the sense that
%
\[ \Re (\cdot, \cdot) = \langle \cdot, \cdot \rangle \]
%
The new form $(\cdot, \cdot)$ cannot be linear in both factors, since
%
\[ \Re (v, iw) = \langle v, iw \rangle = \langle -iv, w \rangle \]
%
The closest thing we could have is that $(\cdot, \cdot)$ is {\bf sequilinear}\footnote{sequi means `one and a half' in greek.}, linear in the first factor, and `anti-linear' in the second factor, in the sense that
%
\[ (v, \lambda w + \gamma u) = \overline{\lambda} (v,w) + \overline{\gamma} (v,u) \]
%
And almost symmetric, in the sense that
%
\[ (v,w) = \overline{(w,v)} \]
%
Then
%
\[ \Im (v,w) = - \Re [i (v, w)] = - \Re (iv, w) = - \langle iv, w \rangle = \langle v, i w \rangle \]
%
So if we have any hope of extending $\langle \cdot, \cdot \rangle$, we must define
%
\[ (v,w) = \langle v, w \rangle + i \langle v, iw \rangle \]
%
With the properties we have displayed, this map is sesquilinear, such that $(v,v) \geq 0$, and $(v,v) = 0$ if and only if $v = 0$. Looking back enlightens us on the new definition. The inner product on a real vector space measures the ratio of projection from one vector to another. When we encounter complex spaces, this ratio of projection becomes complex, so a complex inner-product must be complex valued. The plane spanned by a vector $w$ and $iw$ is the real span of the vectors $w$ and $iw$. If we project onto this plane, it is natural for this projection to be the complex combination of the projection onto $w$ and the projection onto $iw$. This is how the definition arises. The real part of such a product measures the `real' angle between $v$ and $w$, the imaginary part measures a `rotational angle'.

\begin{definition}
    A {\bf Hermitian Product Space} is a complex vector space $V$ equipped with a sesquilinear map $(\cdot, \cdot)$. That is, for any vectors $v,w \in V$, and scalars $\lambda, \gamma \in \mathbf{C}$,
    %
    \[ (\lambda v + \gamma w, u) = \lambda (v, u) + \gamma (w, u) \]
    %
    \[ (v, w) = \overline{(w, v)}\ \ \ \ \ (u, \lambda v + \gamma w) = \overline{\lambda} (u,v) + \overline{\gamma} (u,w) \]
    %
    \[ (v, v) \geq 0\ \ \ \ \ \ \ (v, v) = 0\ \text{if and only if}\ v = 0 \]
\end{definition}

When we talk about inner-product spaces, we are referring to both real-valued inner-product spaces, and complex-valued hermitian-product spaces at the same time.

A norm is a function on a space which tells us how `large' vectors are.
Every inner-product space gives rise to a norm,
%
\[ \]


A Hilbert space is a Banach space most similar to Euclidean space. They have an incredibly structure, and occur in wide applications of functional analysis to mathematics, physics, and computing science.

\begin{definition}
    A {\bf Hilbert space} is a complete inner product space -- That is, a vector space equiped with an inner (hermitian) product such that the corresponding metric structure is complete.
\end{definition}

In this chapter, we shall let $H$ denote a general Hilbert space, and $\langle \cdot, \cdot \rangle$ the inner product space with which the space is equipped.

\begin{theorem}[Cauchy-Schwarz inequality]
    $\langle x, y \rangle \leq \| x \| \| y \|$.
\end{theorem}

The main reason why the subject is called functional analysis is because most of the time we shall be analyzing functions from one space to another, which naturally have an additive and multiplicative structure obtained from the space these functions are defined on.

\begin{definition}
    A {\bf topological vector space} is a vector space over a field (here assumed to be $\mathbf{R}$ or $\mathbf{C}$), endowed with  topology which makes the operations of addition and multiplication continuous.
\end{definition}

Some immediate corollaries of the definition include that

\begin{prop}
    The translation $U + v$ of any open set $U$ by a vector $v$ is open.
\end{prop}
\begin{prop}
    When $v_\alpha \to v$, $w_\alpha \to w$, and $\lambda_\alpha \to \lambda$, $\lambda_\alpha (v_\alpha + w_\alpha) = \lambda ( v + w)$.
\end{prop}
\begin{prop}
    A translated local base at the origin by a vector $v$ is a local base at $v$.
\end{prop}
%
Proposition (1.2) and (1.3) give different characterizations of topological vector spaces. We can define topological vector spaces either in terms of neighbourhood bases at the origin, or in terms of convergent nets obeying the rule denoted in (1.2). A neighbourhood containing the origin will be hereafter known as a {\bf 0-neighbourhood}.
%
\begin{lemma}
    Every 0-neighbourhood $U$ contains a 0-neighbourhood $W$ for which $W + W \subset U$.
\end{lemma}
\begin{proof}
    Since addition is continuous, and $U$ is a 0-neighbourhood, there are neighbourhoods $W$ and $V$ for which $W + V \subset U$. Our problem is solved by picking $W \cap V$ as our neighbourhood.
\end{proof}

\begin{definition}
    A set $B$ is {\bf balanced} if $\lambda B \subset B$ for $|\lambda| < 1$.
\end{definition}

\begin{lemma}
    Every 0-neighbourhood can be shrunk to a balanced neighbourhood.
\end{lemma}
\begin{proof}
    Let $U$ be a neighbourhood of zero. Then there is a scalar neighbourhood $\Lambda$ of zero, and a neighbourhood $V$ of any vector $V$, for which $\Lambda V \subset U$. We may choose $\Lambda$ so it is balanced, and then $\Lambda V$ is balanced.
\end{proof}

\begin{prop}
    If $K$ and $C$ are disjoint subsets of a T1 vector space $V$, where $K$ is compact and $C$ is closed, then 0 has a neighbourhood $V$ such that $K + V$ and $C + V$ are disjoint.
\end{prop}
\begin{proof}
    Fix some point $k \in K$. It suffices to show that there is a neighbourhood $U_k$ for which $k + U_k$ is disjoint from $C + U_k$. If this is true, (1.4) tells us we may pick a subset $W_k$ of $U_k$ for which $W_k + W_k \subset U_k$. Then we may choose a finite subcover $k_1 + W_{k_1}, \dots, k_n + W_{k_n}$ of $K$. If we let $V = \bigcap_{i = 1}^n W_{k_i}$, then we find $K + V$ is disjoint from $C + V$.

    To find this neighbourhood of $k$, we know that we may first pick a neighbourhood $U$ containing $C$, disjoint from $k$. Without loss of generality, by translation, we may assume $U$ is a neighbourhood of 0. Then we may choose $W$ for which $W + W \subset U$ and $-W = W$. It then follows that $k + W$ is disjoint from $W + C$, since if $k + w = w' + c$, $k + w - w' = c$, the left side is contained within $U$.
\end{proof}

\begin{corollary}
    A T1 vector space is Hausdorff.
\end{corollary}

\begin{corollary}
    Every set in a neighbourhood base contains the closure of another neighbourhood.
\end{corollary}
\begin{proof}
    Let 
\end{proof}

\begin{corollary}
    The closure of a set $A$ is the intersection of all $A + V$, where $V$ is a 0-neighbourhood.
\end{corollary}
\begin{proof}

\end{proof}

\begin{prop}
    A locally bounded space is first-countable.
\end{prop}
\begin{proof}
    
\end{proof}

\begin{definition}
Vector spaces should be familiar, but we specify some basic properties which may have been missed in a basic course.
\begin{enumerate}
    \item A set $C$ is {\bf convex} if $tC + (1 - t)C \subset C$ for any $t \in [0,1]$.
    \item A set $B$ is {\bf balanced} if $\lambda B \subset B$ for $| \lambda | \leq 1$.
    \item a set $B$ is {\bf (Von-Neumann) bounded} if, for any neighbourhood $E$ of 0, there is a scalar $\lambda > 0$ such that $B \subset \gamma E$ for every $\gamma > \lambda$.
\end{enumerate}
\end{definition}

\begin{definition}
    There are many additional properties one can ascribe to a topological vector space.
    %
    \begin{enumerate}
        \item A topological vector space is {\bf locally convex} if there is a neighbourhood base of convex sets.
        \item A space is {\bf locally bounded} if there a neighbourhood base of bounded sets.
        \item An {\bf $\mathbf{F}$-space} is a vector space endowed with a complete, invariant metric ($d(v + u, w + u) = d(v,w)$ for any vectors $v,w,u$).
        \item A {\bf Fr\'{e}chet} space is a locally convex $F$-space.
        \item A {\bf normed space} is a vector space endowed with a norm function $\| \cdotp \|: V \to \mathbf{R}+$, such that
        %
        \begin{itemize}
            \item $\| v \| = 0$ if and only if $v = 0$.
            \item $\| v + w \| \leq \| v \| + \| w \|$.
            \item $\| \lambda v \| = | \lambda | \| v \|$.
        \end{itemize}
        %
        We can consider a norm space as an $F$-space by defining a distance function $d(v,w) = \| v - w \|$.
        \item A {\bf Banach space} is a complete normed space.
        \item A space is {\bf normable} if its topology can be induced by a norm.
        \item A space is {\bf Heine-Borel} if every closed-bounded set is compact.
    \end{enumerate}
\end{definition}

\begin{theorem}
    Every locally bounded space is first countable.
\end{theorem}
\begin{proof}
    
\end{proof}

We will be studying a specific subclass of topological vector space.

\begin{definition}
    A norm space is a vector space $V$ endowed with a norm function $\| \cdotp \|: \mathbf{R} \to \mathbf{R}^+$, such that
    %
    \begin{enumerate}
        \item $\| v \| = 0$ if and only if $v = 0$.
        \item $\| \lambda v \| = | \lambda | \| v \|$.
        \item $\| v + w \| \leq \| v \| + \| w \|$.
    \end{enumerate}
    %
    We obtain a distance function on a norm space by defining $d(v,w) = \| v - w \|$.
\end{definition}








\section{Convexity}

A function $f:(a,b) \to \mathbf{R}$ is convex when the line segment between $(a,f(a))$ and $(b,f(b))$ lies above the graph of $f$. The line segment connecting these two points is described by
%
\[ \{ (\lambda a + (1 - \lambda) b, \lambda f(a) + (1 - \lambda) f(b)) : 0 \leq \lambda \leq 1 \} \]
%
and we require that $(\lambda a + (1 - \lambda)b, f(\lambda a + (1 - \lambda)b)$ lies below the corresponding point on the line defined above. This is satisfied exactly when we have a certain inequality:

\begin{definition}
    A function $f:U \to \mathbf{R}$ is {\bf convex} on $(a,b)$ if, for any $a \leq x < y \leq b$, and $0 \leq \lambda \leq 1$, we have
    %
    \begin{equation} \label{convex1} f(\lambda x + (1 - \lambda) y) \leq \lambda f(x) + (1 - \lambda) f(y) \end{equation}
    %
    By rewording the definition, convexity is also satisfied when, for $a \leq x < y < z \leq b$,
    %
    \begin{equation} \label{convex2} \frac{f(y) - f(x)}{y - x} \leq \frac{f(z) - f(y)}{z - y} \end{equation}
    %
    Geometrically, this equation says that the slope of the tangent line from $(x, f(x))$ to $(y, f(y))$ is smaller than the tangent line from $(y, f(y))$ to $(z, f(z))$.
\end{definition}

\begin{lemma}
    A $C^1$ convex function's derivative is non-decreasing on $(a,b)$.
\end{lemma}
\begin{proof}
    Suppose $f$ is a $C^1$ function, and $f'$ is non-decreasing, then consider any $a \leq x < y < z \leq b$. Then $f'(x) \leq f'(y) \leq f'(z)$. Applying the mean value theorem, we conclude there is $t \in (x,y)$, and $u \in (y,z)$, for which
    %
    \begin{equation} \label{convexderivative} f'(t) = \frac{f(y) - f(x)}{y - x}\ \ \ \ \ f'(u) = \frac{f(z) - f(y)}{z - y} \end{equation}
    %
    And since $t < u$, (\ref{convexderivative}) implies $f'(t) \leq f'(u)$, i.e. (\ref{convex2}) is satisfied.

    If $f$ is $C^1$ and convex, then surely $f'$ is non-decreasing. Fix $a \leq x < y \leq b$. In (\ref{convex2})), letting $y$ converge to $x$, we obtain
    %
    \begin{equation} \label{yconvergeleft} f'(x) = \lim_{y \to x^+} \frac{f(y) - f(x)}{y - x} \leq \lim_{y \to x} \frac{f(z) - f(y)}{z - y} = \frac{f(z) - f(x)}{z - x} \end{equation}
    %
    Conversely, letting $y \to z$, we obtain
    %
    \begin{equation} \label{yconvergeright} f'(y) = \lim_{y \to z^-} \frac{f(z) - f(y)}{z - y} \geq \lim_{y \to z} \frac{f(y) - f(x)}{y - x} = \frac{f(z) - f(x)}{z - x} \end{equation}
    %
    And in tandem, (\ref{yconvergeleft}), (\ref{yconvergeright}) and (\ref{convex2}) imply that $f'(x) \leq f'(y)$.
\end{proof}

\begin{example}
    $\exp: \mathbf{R} \to \mathbf{R}$ is convex on $(-\infty, \infty)$, since $\exp'' = \exp > 0$.
\end{example}

\begin{lemma}
    A function is continuous on the open segments where it is convex.
\end{lemma}

The most important inequality in analysis is the triangle inequality, undershadowed by the Schwarz inequality. Almost as important is Jensen's inequality. Despite its importance, the proof is fairly simple and intuitive. Consider the center of mass of an object.  

\begin{theorem}[Jensen's Inequality]
    Let $(\Omega, \mathbf{P})$ be a probability space. If $X \in L^1(\mathbf{P})$, where $a < X< b$, and if $f$ is a real function, convex on $(a,b)$, then
    %
    \begin{equation} \label{jensen} f(\mathbf{E}[X]) \leq \mathbf{E}[f \circ X] \end{equation}
\end{theorem}
\begin{proof}
    Since $a < X < b$, $a < \mathbf{E}[X] < b$. Let
    %
    \[ \beta = \sup_{a < x < \mathbf{E}[X]} \frac{g(\mathbf{E}[X]) - g(x)}{\mathbf{E}[X] - x} \]
    %
    For any $a < x < y$, $g(x) + \beta(\mathbf{E}[X] - x) \geq g(y)$. But also, by the right side of $(\ref{convex2})$, for any $z > y$, $g(z) - \beta(z - \mathbf{E}[X]) \geq g(\mathbf{E}[X])$. For any $\omega \in \Omega$, we may restate these equations as as
    %
    \[ g(f(\omega)) - \beta(f(\omega) - y) \geq g(y) \]
    %
    But then taking expectations, we obtain (\ref{jensen}).
\end{proof}

Jensen's inequality is incredibly useful. To see this, consider some examples.

\begin{example}
    We have seen the exponential function is convex. Hence for any $L_1(\mathbf{P})$, where $\mathbf{P}$ is a probability measure, we have
    %
    \begin{equation} \label{harmonic} exp\left(\int f d\mathbf{P} \right) \leq \int e^f d\mathbf{P} \end{equation}
    %
    Let $\mathbf{P}$ be the uniform measure over a finite set $\{ x_1, x_2, \dots, x_n \}$. Then (\ref{harmonic})) tells us that
    %
    \[ e^{f(x_1)/n}e^{f(x_2)/n} \dots e^{f(x_n)/n} \leq \frac{\sum e^{f(x_i)}}{n} \]
    %
    If we let $x_i$ be real numbers, and $f(x_i) = \log(x_i)$, we conclude
    %
    \[ (x_1x_2 \dots x_n)^{1/n} \leq \frac{\sum x_i}{n} \]
    %
    In other words, the geometric mean is always smaller than the arithmetic mean.
\end{example}

Jensen's inequality implies so many other inequalities in analysis.

\begin{definition}
    Let $1 \leq p \leq \infty$. We say $1 \leq q \leq \infty$ is the {\bf conjugate} of $p$ if $p^{-1} + q^{-1} = 1$, and write $q = p'$.
\end{definition}

\begin{theorem}[H\"{o}lder]
    If $p' = q$, $(\Omega,\mu)$ is a measure space, and $f,g$ are positive measurable functions on $\Omega$, then
    %
    \begin{equation} \label{holder} \| fg \|_1 \leq \| f \|_p \| g \|_q \end{equation}
\end{theorem}
\begin{proof}
    Let $A,B$ be the values on the right hand side of (\ref{holder})). If $A = 0$, then $f = 0$ almost everywhere, so that the theorem is trivial. Symmetry shows that the same is true if $B = 0$, so assume $A, B \neq 0$. Define $F = f/A$, and $G = g/B$. Then
    %
    \[ \left( \int F^p d\mu \right)^{1/p} = A^{-1} \left( \int f^p d\mu \right)^{1/p} = 1\ \ \ \ \ \left( \int G^q d\mu \right)^{1/q} = B^{-1} \left( \int g^q d\mu \right)^{1/q} = 1 \]
    %
    For any $x$, there is a number $s$ such that $e^{s/p} = F(x)$, and $t$ such that $e^{t/q} = G(x)$. By the convexity of the exponential function, $e^{s/p + t/q} \leq e^s p^{-1} + e^t q^{-1}$. Thus $FG \leq p^{-1} F^p + q^{-1} G^q$, and thus by integrating,
    %
    \[ \int FG\ d\mu \leq p^{-1} + q^{-1} = 1 \]
\end{proof}

\begin{corollary}[Minkowski]
    For
    \[ \| f + g \|_p \leq \| f \|_p + \| g \|_p \]
\end{corollary}






\section{The Dual of a Hilbert Space}

\begin{lemma}
    Fix $x,z \in H$. If $\langle x, y \rangle = \langle z, y \rangle$ for all $y \in H$, then $x = z$.
\end{lemma}
\begin{proof}
    For then $\langle x - z, y \rangle = 0$ for all $y$, and in particular, $\langle x - z, x - z \rangle = 0$.
\end{proof}





\chapter{Operator Theory}

\section{Operators on a Hilbert Space}

If $H$ is a Hilbert space, then $H^*$ is a dull object to study, for it is essentially $H$. When one says they study Hilbert spaces, they really study the bounded operators from one space to another. This is the task we begin in this section. Given two Hilbert spaces, we let $B(H,K)$ denote the continuous linear maps from $H$ to $K$.

\begin{lemma}
    If $( \cdot, \cdot )$ is a bounded\footnote{In the sense that there is a `sesquilinear norm' $\|(\cdot, \cdot)\|$, which is the smallest number such that $(x,y) \leq \| x \| \| y \| \|(\cdot, \cdot)\|$.}, sesquilinear form on a Hilbert space, then there is a unique $M \in B(H)$ such that $(x,y) = \langle x, My \rangle$.
\end{lemma}
\begin{proof}
    For each $x$, $(\cdot, x)$ is a bounded linear functional, so there is a unique $Mx \in H$ such that $(y,x) = \langle y, Mx \rangle$. Since
    %
    \[ (z, \lambda x + \gamma y) = \overline{\lambda}(z, x) + \overline{\gamma}(z,y) = \langle z, \lambda Mx + \gamma My \rangle \]
    %
    $M$ is linear, because $M(\lambda x + \gamma y) = \lambda Mx + \gamma My$. $M$ is certainly uniquely defined, and
    %
    \[ \langle x, My \rangle = (x,y) \leq \|(\cdot, \cdot)\| \| x \| \| y \| \]
    %
    So $\| M \| \leq \|(\cdot,\cdot)\|$, but also
    %
    \[ (x,y) = \langle x, My \rangle \leq \| M \| \| x \| \| y \| \]
    %
    so $\|(\cdot, \cdot)\| \leq \| M \|$, and we have equality of norms.
\end{proof}

\begin{theorem}
    Let $H$ and $K$ be Hilbert spaces. If $M \in B(H,K)$, there is a unique map $M^* \in B(K,H)$ such that
    %
    \[ \langle Mx, y \rangle = \langle x, M^*y \rangle \]
    %
    $M^*$ is known as the {\bf adjoint} of $M$. We also have $\| M^* \| = \| M \|$.
\end{theorem}
\begin{proof}
    The map $(x,y) = \langle Mx, y \rangle$ is a sesquilinear form, so there is a unique $M^*$ such that $(x,y) = \langle x, M^*y \rangle$.
\end{proof}

\begin{example}
    If we are working in $\mathbf{R}^n$, then every operator $B(\mathbf{R}^n, \mathbf{R}^m)$ can be identified with a unique matrix in $M_{n,m}(\mathbf{R})$. If $(a_{ij})$ is that operator, then the adjoint is the transpose $(a_{ji}) \in M_{m,n}(\mathbf{R})$. In $\mathbf{C}^n$, the adjoint is the conjugate transpose $(\overline{a_{ji}})$.
\end{example}

\begin{example}
    Given $\phi \in L^\infty(X)$, where $X$ is a $\sigma$-finite measure space, we have an operator $M_\phi \in B(L^2(X))$ defined by $M_\phi(f) = \phi f$. Then
    %
    \[ \langle M_\phi f, g \rangle = \int \phi f \overline{g} = \int f \overline{g \overline{\phi}} = \langle f, \overline{\phi} g \rangle \]
    %
    Thus $M_\phi^* = M_{\overline{\phi}}$.
\end{example}

The adjoint operation $*$ is an antilinear operator from $B(H,K)$ to $B(K,H)$, such that $(ST)^* = T^*S^*$. This is verified by calculation, since
%
\[ \langle ST x, y \rangle = \langle Tx, S^* y \rangle = \langle x, T^*S^* y \rangle \]
%
We also have $\| T^*T \| = \| T \|^2$, since
%
\[ \langle T^*T x, x \rangle = \langle Tx, Tx \rangle = \| T \|^2 \| x \| \]
%
so $\| T^*T \| \leq \| T \|^2$, and this is inequality is attained in the suprema, since if $x_i \in B_H$ is such that $\| T x_i \| \to \| T \|$, then
%
\[ \langle T^*T x_i, x_i \rangle = \| T x_i \|^2 \to \| T \|^2 \]
%
Thus $*$ is what is known as an involution.

We can described some operators from a Hilbert space to itself by their action under the involution. $T$ is a {\bf self adjoint operator} if $T^* = T$. This is something like $T$ being symmetric, but in the infinite dimensional case. $T$ is unitary if $T^* = T^{-1}$. These are analogous to rotations, but in infinite dimensions. $T$ is {\bf normal} if $T^*T = TT^*$. Any operator $M$ can be expressed as $M = T + iS$, where $T$ and $S$ are self adjoint operators, by defining
%
\[ T = \frac{1}{2}(M + M^*)\ \ \ \ \ S = \frac{1}{2i}(M - M^*) \]
%
We shall show this is a unique expression. This is essentially like beginning with $\mathbf{C}$, and identifying $\mathbf{R}$ as those elements invariant under complex conjugation.

\begin{theorem}
    If $T$ is a self-adjoint operator on a complex Hilbert space, then
    %
    \[ \| T \| = \sup \{ \langle Tx, x \rangle: \| x \| \leq 1 \} \]
\end{theorem}
\begin{proof}
    Let $M$ be the supremum. Clearly $M \leq \| T \|$. If $\| x \|, \| y \| \leq 1$,
    %
    \begin{align*}
        \langle T(x \pm y), x \pm y \rangle &= \langle Tx, x \rangle \pm \langle Ty, x \rangle \pm \langle Tx, y \rangle + \langle Ty, y \rangle\\
        &= \langle Tx, x \rangle \pm \overline{\langle Tx, y \rangle} \pm \langle Tx, y \rangle + \langle Ty, y \rangle\\
        &= \langle Tx, x \rangle \pm 2 \Re \langle Tx, y \rangle + \langle Ty, y \rangle
    \end{align*}
    %
    Then, subtracting, we find
    %
    \[ 4\Re \langle Tx, y \rangle = \langle T(x+y),x+y \rangle - \langle T(x - y), x - y \rangle \]
    %
    Thus
    %
    \[ 4 \Re \langle Tx, y \rangle \leq M(\|x + y\|^2 + \|x - y\|^2) = 2M(\|x\|^2 + \|y\|^2) \leq 4M \]
    %
    Let $\lambda$ be such that $\langle Tx, y \rangle = \lambda |\langle Tx, y \rangle|$. If we replace $x$ by $\lambda x$, we find
    %
    \[ |\langle Tx, y \rangle| = \overline{\lambda} \langle Tx, y \rangle = \langle T(\overline{\lambda}x), y \rangle \leq M \]
    %
    Hence
    %
    \[ \| Tx \| = \sup \{ |\langle Tx, y \rangle| : y \in B_H \} \leq M \]
    %
    and therefore $\|T\| \leq M$.
\end{proof}

\begin{corollary}
    $\langle Tx, x \rangle = 0$ if and only if $T = 0$.
\end{corollary}
\begin{corollary}
    If $\langle Tx, x \rangle = \langle Sx, x \rangle$ for each $x \in H$, then $T = S$.
\end{corollary}

This theorem is false for real spaces, for instance, if we take a rotation operator on the plane.

\begin{theorem}
    $T$ is normal if and only if $\| Tx \| = \| T^* x \|$ for each $x$.
\end{theorem}
\begin{proof}
    For then, for each $x$,
    %
    \[ \| Tx \|^2 = \langle Tx, Tx \rangle = \langle T^*Tx, x \rangle \]
    \[ \| T^*x \|^2 = \langle T^*x, T^*x \rangle = \langle TT^*x, x \rangle \]
    %
    which implies $T^*T = TT^*$. The converse repeats backwards through the proof, since the norm of $\| T \|$ is specified by its action through the inner product.
\end{proof}

\begin{theorem}
    Any normal operator $T$ has the following properties.
    %
    \begin{enumerate}
        \item[(a)] $\text{ker}(T) = \text{ker}(T^*)$.
        \item[(b)] $T(H)$ is dense in $H$ if and only if $T$ is injective.
        \item[(c)] $T$ is a surjective isomorphism if and only if $\| Tx \| \geq C \| x \|$ for some $C$.
        \item[(d)] If $Tx = \lambda x$, then $T^*x = \overline{\lambda} x$.
        \item[(e)] If $\lambda$ and $\gamma$ are distinct eigenvalues, then the eigenspaces are orthogonal to one another.
    \end{enumerate}
\end{theorem}
\begin{proof}
    To prove $(a)$ notice that $\| Tx \| = 0$ if and only if $\| T^*x \| = 0$. $(b)$ follows since
    %
    \[ T(H)^\perp = \text{ker}(T^*) = \text{ker}(T) \]
    %
    If $\| Tx \| \geq C \| x \|$, then $T(H)$ is closed, and dense, therefore $T(H) = H$. The converse follows by definition. If $T$ is normal, then $\lambda - T$ is also normal, since $(\lambda - T)^* = \overline{\lambda} - T^*$, so $\text{ker}(\lambda - T) = \text{ker}(\lambda - \overline{\lambda})$. Finally, if $Tx = \lambda x$, and $Ty = \gamma y$, then
    %
    \[ \lambda \langle x, y \rangle = \langle Tx, y \rangle = \langle x, T^*y \rangle = \langle x, \overline{\gamma} y \rangle = \gamma \langle x, y \rangle \]
    %
    so if $\lambda \neq \gamma$, then $\langle x,y \rangle = 0$.
\end{proof}

\begin{theorem}
    If $U \in B(H)$, then the following are equivalent.
    %
    \begin{enumerate}
        \item[(a)] $U$ is unitary.
        \item[(b)] $GL(H) = H$ and $\langle Ux, Uy \rangle = \langle x, y \rangle$.
        \item[(c)] $GL(H) = H$ and $\| Ux \| = \| x \|$, so $U$ is an isometry from $H$ to itself.
    \end{enumerate}
\end{theorem}
\begin{proof}
    If $U$ is unitary, then $U$ is invertible, so $GL(H) = H$, and
    %
    \[ \langle Ux, Uy \rangle = \langle U^*Ux, y \rangle = \langle x, y \rangle \]
    %
    If $(b)$ holds, then
    %
    \[ \| Ux \|^2 = \langle Ux, Ux \rangle = \langle x, x \rangle = \| x \|^2 \]
    %
    If $(c)$ holds, then
    %
    \[ \langle U^*Ux, x \rangle = \langle Ux, Ux \rangle = \|Ux\|^2 = \|x\|^2 = \langle x, x \rangle \]
    %
    Thus $U^*U$ is the identity. Since $U$ is injective and surjective, the open mapping theorem tells us $U$ is invertible, so $U^* = U^{-1}$.
\end{proof}

This shows that unitary operators are the natural automorphisms in the category of Hilbert spaces.

\begin{lemma}
    An operator $T \in B(H)$ in a complex Hilbert space is self adjoint if and only if $\langle Tx, x \rangle \in \mathbf{R}$ for each $x \in H$.
\end{lemma}
\begin{proof}
    If $T$ is self adjoint, then
    %
    \[ \langle Tx, x \rangle = \langle x, Tx \rangle = \overline{\langle Tx, x \rangle} \]
    %
    Conversely, if $\langle Tx, x \rangle \in \mathbf{R}$ for each $x \in H$, then
    %
    \[ \langle Tx, x \rangle = \overline{\langle x, Tx \rangle} = \langle x, Tx \rangle = \langle T^*x, x \rangle \]
    %
    for each $x$, from which we conclude $T = T^*$.
\end{proof}

\begin{theorem}
    If $P$ is a projection, then the following are equivalent:
    %
    \begin{enumerate}
        \item[(a)] $P$ is self-adjoint.
        \item[(b)] $P$ is normal.
        \item[(c)] $P(H) = \text{ker}(P)^\perp$.
        \item[(d)] $\langle Px, x \rangle = \| Px \|^2$.
    \end{enumerate}
\end{theorem}
\begin{proof}
    If $P$ is self-adjoint, then $P$ is trivially normal. If $P$ is normal, then
    %
    \[ P(H) = \text{ker}(P^*)^\perp = \text{ker}(P)^\perp \]
    %
    If (c) holds, then every $x = y + Px$, where $Py = 0$. Then
    %
    \[ \langle P(y + Px), y + Px \rangle = \langle Px, y + Px \rangle = \langle Px, Px \rangle = \| Px \|^2 \]
    %
    If (d) holds, then $\langle Px, x \rangle \in \mathbf{R}$ for each $x$, so $P$ is self-adjoint. If (c) holds, then $\langle Px, x \rangle$
\end{proof}






\section{Compact Operators}

We already know operators on an infinite dimensional Banach space are more tricky than there finite dimensional counterparts. Thus we restrict our attention to operators which do not `spread themselves out' too much. An operator is {\bf compact} if the image of any bounded set is precompact. It suffices to verify that the image of the unit ball is precompact. Given two Banach spaces $X$ and $Y$, we shall let $K(X,Y)$ denote the space of all compact operators from $X$ to $Y$.

\begin{theorem}
    $K(X,Y)$ is a closed subspace of $B(X,Y)$, such that for any $M \in K(X,Y)$, $N \in B(Y,Z)$, and $L \in B(Z',X)$, $NML \in K(Z',Z)$.
\end{theorem}
\begin{proof}
    Suppose $M$ and $N$ are compact. Let $U$ be a bounded susbet of $X$. Then
    %
    \[ (\lambda M + \gamma N)(U) = \lambda M(U) + \gamma N(U) \]
    %
    And the sum of two precompact sets is precompact. Thus $\lambda M + \gamma N$ is precompact. If $U$ is a bounded subset of $Z'$, then $L(U)$ is a bounded subset of $X$, so $ML(U)$ is a precompact subset of $Y$, implying $NML(U)$ is a precompact subset of $Z$, for if every sequence $y_i$ has a convergent subsequence, so must $N(y_i)$.

    To prove $K(X,Y)$ is closed we rely on the fact that a subset of a complete metric space is precompact if and only if, for any $\varepsilon > 0$, the subset can be covered by finitely many balls of radius $\varepsilon$. Now suppose we have $M_1, M_2, \dots \in K(X,Y)$, and $M_i \to N$. We claim $N$ is compact. Let $U$ be a bounded subset of $X$, such that $\| x \| < K$ for each $x \in U$. Then for each $\varepsilon > 0$, there are finitely many balls of radius $\varepsilon$ which cover $M_i(U)$, for each $i$. If $\| M_i - N \| < \varepsilon / K$, then $\| (M_i - N)(x) \| < \varepsilon$ for each $x \in U$, so that if $M_i x$ is contained in some ball with center $y$ of radius $\varepsilon$, then $Nx$ is contained in the ball with center $y$ and radius $2 \varepsilon$. Thus $N(U)$ can be covered with finitely many balls of radius $2\varepsilon$, and is therefore precompact.
\end{proof}
\begin{corollary}
    $K(X)$ is a closed, two-sided ideal of $B(X)$.
\end{corollary}

The best examples of compact operators were found in function spaces, and the best way to verify compactness in these spaces is to use the Arzela-Ascoli criterion. Say a family of functions $\{ g_\alpha : X \to \mathbf{C} \}$ is {\bf equicontinuous} at $x \in X$ if, for any $\varepsilon > 0$, there is an open set $U$ containing $x$ such that $\| g(a) - g(b) \| < \varepsilon$ for $a,b \in U$.

\begin{theorem}[Arzela-Ascoli]
    The precompact subsets of $l^\infty(X)$ are precisely those subsets which are equicontinuous, when $X$ is compact.
\end{theorem}
\begin{proof}
    Suppose $C$ is a precompact subset of $l^\infty(X)$. 
\end{proof}

\begin{example}
    The best examples of compact operators, and the operators for which most of the theory was developed, are integral equations. Given two Hausdorff spaces $A$ and $B$, equipped with respective measures $\mu$ and $\nu$, consider some measurable function $K: A \times B \to \mathbf{C}$, and consider the integral operator $\mathbf{K}$ mapping a measurable function $f$ to a function on $A$ defined by
    %
    \[ \mathbf{K}f(x) = \int_A K(x,y) f(y) d \mu(y) \]
    %
    $K$ is known as the {\bf kernel} of $\mathbf{K}$. We assume that $K$ has bee chosen such that $\mathbf{K}f$ is measurable for each measurable $f$. If $A$ and $B$ are compact, and $K$ is continuous, then $\mathbf{K}$ is a compact operator from $L^1(A)$ to $L^1(B)$.
\end{example}

It is a theorem of Calkin that $K(X)$ is the only closed, two-sided ideal of $B(X)$, when $X$ is a Hilbert space. Thus we are lead to consider the {\bf Calkin Algebra} $B(X)/K(X)$, which is simple, and quite useful in $K$-theory.

A {\bf finite-rank} operator is an operator whose range is finite-dimensional. It is clear that every finite-rank operator is compact. We shall eventually show that the set $F(X)$ of all finite-rank operators is dense in $K(X)$, when $X$ is a Hilbert space.



\newpage

In Linear Algebra, we know we may diagonalize any symmetric matrix. Here we apply our knowledge of Hilbert spaces to extend this to infinite dimensional spaces, for `symmetric' operators which are `almost finite dimensional'.

\begin{theorem}
    If $X$ is a Banach space, and $Y$ is a closed subspace, $M$ is a compact operator on $X$, and $\lambda \neq 0$ is such that
    %
    \[ \inf \{ \| (\lambda - M)y : y \in B_Y \| \} = 0 \]
    %
    then $Y \cap \ker(\lambda - M) \neq 0$.
\end{theorem}
\begin{proof}
    Let $y_i$ be a sequence in $B_Y$ such that $(\lambda - M)y_i \to 0$. Since $M$ is a compact operator, we may assume that $My_i$ converges to some $z$. But then, by continuity,
    %
    \[ \lim_i \lambda y_i = \lim_i \lambda y_i - M y_i + M y_i = \lim_i y_i - M y_i + \lim_i M y_i = 0 + z \]
    %
    Since $\lambda \neq 0$, $y_i \to \lambda^{-1} z$. But then $z \in Y$, and $z = Mz$.
\end{proof}

\begin{corollary}
    Let $M$ be a compact operator. If $\lambda \in \sigma(M)$ is non-zero, then $\dim(\ker(\lambda - M)) < \infty$, $(\lambda - M)X$ is closed and $\text{codim}(\lambda - M) < \infty$, and there is $n$ such that
    %
    \[ \ker[(\lambda - M)^n] = \ker[(\lambda - M)^{n+1}] \]
    %
    \[ (\lambda - M)^n X = (\lambda - M)^{n+1} X \]
\end{corollary}
\begin{proof}
    First note that $\lambda = M$ on $\ker(\lambda - M)$, implying the identity is compact on $\ker(\lambda - M)$, and therefore that $\dim(\ker(\lambda - M)) < \infty$. It follows that $\ker(\lambda - M)$ is closed, and thus there is a subspace $Y$ of $X$ such that $X = \ker(\lambda - M) \oplus Y$. $M$ restricted to $Y$ is injective, and the last theorem implies, since $Y \cap \ker(\lambda - M) = (0)$, that
    %
    \[ C = \inf \{ \| (\lambda - M)y \| : y \in B_Y \} > 0 \]
    %
    Hence $\| (\lambda - M)y \| \geq C \|y\|$ on $Y$, and so $\lambda - M$ is an isomorphism from $Y$ to a subspace of $X$, which is therefore complete and thus closed. Note that
    %
    \[ [(\lambda - M)X]^\perp = \{ \phi \in X^*: \phi \circ (\lambda - M) = 0 \} = \ker( \lambda - M^*) \]
    %
    Since $T$ is compact, so is $T^*$, and because $\lambda$ must also be an eigenvalue of $T^*$, $\text{dim}[((\lambda - M)X)^\perp] < \infty$, so that $(\lambda - M)X$ has finite codimension.
\end{proof}










\part{Operator Theory}





\chapter{Banach Algebras}

In this chapter, we develop the machinery to study the various classes of bounded operators which occur when classifying transformations on a space. To do this, we employ tools from abstract algebra. Recall that an algebra over a field $\mathbf{F}$ is a (not necessarily unital) ring $A$ together with a fixed embedding of $\mathbf{F}$ into $A$, which gives $A$ a vector space structure. A consistant norm attached to an algebra is a nice situation to study operators over a Banach space. What's more, the general theory gives light to many other circumstance  which would not have been apparent were we just analyzing concrete operators. We shall use capital letters, like $M$ and $N$, to denote abstract elements of these algebras, and we denote algebras by capital letters near the beginning of the alphabet, such as $A$ or $B$. A {\bf Banach Algebra} is a Banach space which is also an algebra, and satisfies
%
\begin{equation} \label{algebranorm} \| MN \| \leq \| M \| \| N \|\ \ \ \ \ \| 1 \| = 1 \end{equation}
%
Thus multiplication is a continuous operation. We shall only consider Banach algebras over the complex numbers, because here we can apply the unique properties of holomorphicity and conjugation. Why this is so important will become clear over time.

\begin{example}
    Let $K$ be a compact space. Then the space $C(K)$ of complex-valued continuous functions is a Banach algebra under the uniform convergence norm $\| \cdot \|_\infty$ and with pointwise multiplication. If $K$ consists of $n$ points, then $C(K) \cong \mathbf{C}^n$, which is perhaps the most basic commutative Banach algebra.
\end{example}

\begin{example}
    The space $L_\infty(X)$ of essentially bounded functions on a measure space $X$ is a Banach algebra. More generally, the space $C_b(X)$ of continuous, bounded complex-valued functions on any topological space $X$ is a Banach algebra. If $\mu$ is the measure of $X$, then provided $\mu(X) > 0$, $L_\infty(X)$ has a unit.
\end{example}

\begin{example}
    If $X$ is locally compact, then the space $C_0(X)$ of functions which vanish at infinity form a Banach algebra, except that the space does not always contain an identity. To add an identity, we enlarge the space to $C_c(X)$, the space of eventually constant continuous functions $f$ -- those functions for which there is $\lambda \in \mathbf{C}$ such that $f - \lambda$ vanishes at infinity.
\end{example}

\begin{example}
    If $K$ is a compact neighbourhood in $\mathbf{C}$, then we define $A(K)$ to be the set of continuous functions on $K$ which are analytic in $K^\circ$. Since uniform convergence preserves holomorphicity, $A(K)$ is a closed subalgebra of $C(K)$, and is therefore a Banach algebra. $A(\mathbf{D})$ is known as the disk algebra.
\end{example}

\begin{example}
    Let $G$ be locally compact, with Haar measure $\mu$, and consider the space of functions $L^1(G)$, where multiplication is convolution,
    %
    \[ (f * g)(x) = \int f(y) g(y^{-1}x) d\mu(y) \]
    %
    The space is a Banach algebra, since applying Tonelli's theorem (on each component of $G$, which is $\sigma$-compact) and the translation invariance of the Haar measure, we find $\| f * g \|_1 \leq \| f \|_1 \| g \|_1$. This algebra is commutative when $G$ is a commutative. It does not always possess a unit, but we can enlarge the space so it does. Let $M(G)$ be the space of complex-valued Radon measures on $G$, and define convolution on $M(G)$ by letting
    %
    \[ \int f d (\eta * \nu) = \int \int f(xy) d\eta(x) d\nu(y) \]
    %
    If we identify each $f$ with the measure $f \mu$, such that $\int g d(f \mu) = \int g f d\mu$. In other words, $f \lambda$ is defined by the density equation
    %
    \[ \frac{d(f \mu)}{d \mu} = f \]
    %
    The dirac delta function $\delta$ at the identity is a convolution identity on $M(G)$, so the set of all measures of the form $f \mu + \gamma \delta$ is a one dimensional extension of $L^1(G)$ which now has an identity.
\end{example}

\begin{example}
    A concrete example of $L^1(G)$ is $L^1(\mathbf{Z})$, which can be taken to be the space of integer valued sequences $c = \{ c_n \}$ (where $\sum |c_n| < \infty$), with convolution
    %
    \[ (a * b)_n = \sum_{k \in \mathbf{Z}} a_k b_{n-k} \]
    %
    The dirac delta function here is just the sequence $\delta$ such that
    %
    \[ \delta_k = \begin{cases} 1 & k = 0 \\ 0 & \text{elsewise} \end{cases} \]
    %
    If we define $\delta^1$ to be the sequence valued at one, and zero elsewhere, then we see that $\delta^1$ and its inverse, for any finitely non-zero sequence can be written as the sum of convolutions of $\delta^1$ and $(\delta^1)^{-1}$, so that $L^1(\mathbf{Z})$ is generated by $\delta^1$ and its inverse (a Banach algebra $A$ is generated by $M_1, \dots, M_n$ if $\mathbf{C}[M_1, \dots, M_n]$ is dense in $A$).
\end{example}

Almost all examples above are abelian algebras. One of the prime reasons to study Banach algebras is to study operators on a Banach space, which are almost always non-commutative. In fact, some folks call the study of Banach algebras `non-commutative analysis'.

\begin{example}
    Let $E$ be a Banach space. The space $B(E)$ of all bounded linear operators from $E$ to itself is a Banach algebra with respect to the operator norm. It is a unital algebra, since it possesses the identity operator. The subset $K(E)$ of compact linear operators is a closed (double-sided) ideal of $B(E)$, and so is also a Banach algebra. $K(E)$ is unital if and only if $E$ is finite dimensional.
\end{example}

Really, all that distinguishes a Banach space from a Banach algebra is a continuous multiplication structure, modulo the norm we use to define the topology of the space.

\begin{prop}
    Let $X$ be a Banach space upon with a continuous multiplication structure. Then there is an equivalent norm on $X$ which makes the space into a Banach algebra.
\end{prop}
\begin{proof}
    Embed $X$ in $B(X)$ by defining $\Lambda_M(N) = MN$ (since multiplication on the right is continuous, $\Lambda_M$ truly is in $B(X)$ rather than just being a linear map). It is trivial to verify this is an algebra morphism, and we have the inequality
    %
    \begin{equation} \label{embedinequality} \| M \| = \| \Lambda_M (1) \| \leq \| \Lambda_M \| \| 1 \| \end{equation}
    %
    which implies the map is injective, and its inverse continuous. The closed graph theorem implies the embedding is continuous, provided we can show that if $M_i \to M$, and $\Lambda_{M_i} \to \Lambda$, then $\Lambda = \Lambda_M$. Since multiplication is continuous,
    %
    \[ \Lambda(N) = \lim \Lambda_{M_i}(N) = \lim M_iN = MN \]
    %
    So $\Lambda = \Lambda_M$, and we have continuity. The proof is complete, since in $B(X)$, the inequalities ($\ref{algebranorm}$) are known to hold, except in one special case: if $X = (0)$, then $\| \text{id}_X \| = 0$, but in this case, it is impossible to define an algebra structure on $X$.
\end{proof}

\section{Invertibility in Banach Algebras}

We shall begin Banach algebra theory by analyzing criteria for invertibility, which coincides with the spectral theory of these algebras. For obvious reasons, we restrict our attention to unital algebras. The set of all units in a unital algebra $A$ will be denoted $GL(A)$, and called the general linear group of $A$. The {\bf spectrum} and {\bf resolvent} of an element $M$ of a Banach algebra $A$ are defined respectively as
%
\[ \sigma_{A}(M) = \{ \lambda \in \mathbf{C} : \lambda - M \not \in GL(A) \} \]
%
\[ \rho_{A}(M) = \{ \lambda \in \mathbf{C} : \lambda - M \in GL(A) \} \]
%
The resolvent is the complement of the spectrum. When the underlying algebra is canonical, we just use $\sigma(M)$ and $\rho(M)$ to denote the spectrum. One way to view the addition of a complex number to an operator as an `infinitisimal shift' in the effects of the operator. The spectrum tells us in which directions the operator goes bad under infinitisimal changes, and we shall find that knowledge of the spectrum will allow us to perturb operators more finely. A large number of theorems are based on relating these perturbations to the original operator.

\begin{example}
    Let $X$ be a space, and consider $f \in C_b(X)$. Then $\sigma(f) = \overline{f(X)}$. If $\lambda \in \rho(f)$, then
    %
    \[ (\lambda - f)^{-1}(x) = \frac{1}{\lambda - f(x)} \]
    %
    This function is bounded if and only if $\lambda \not \in \overline{f(X)}$. If $X$ is compact, and $f \in C(X)$, $\sigma(f) = f(X)$.
\end{example}

\begin{example}
    Consider a Banach space $E$. The inverse mapping theorem tells us that $M \in B(E)$ is invertible if and only if it is bijective. If $\dim(E) < \infty$, this is simply the set of injective operators; the spectrum is then exactly the set of eigenvalues of the operator. One can consider eigenvalues in the infinite dimensional case, yet they are almost always a proper subset of the spectra. The collection of eigenvalues is known as the {\bf point spectra}, denoted $\sigma_p(M)$.
\end{example}

The next lemma is incredibly important, and is an extension of the power series formula
%
\[ \frac{1}{1 - z} = \sum_{k = 0}^\infty z^k \]
%
whose conclusion really relies on no properties of the complex numbers, aside from the products and sum all Banach algebras possess.

\begin{lemma}[Neumann Series]
    If $\|M\| \leq 1$, then $1 - M \in GL(A)$, and
    %
    \[ (1 - M)^{-1} = \sum_{k = 0}^\infty M^k \]
    %
    in the sense that the right hand side also converges to a well-defined value.
\end{lemma}
\begin{proof}
    The right side converges absolutely by the comparison test, since $\| M^k \| \leq \| M \|^k$. Because $A$ is Banach, absolute convergence implies convernences, and so we are justified in the manipulation
    %
    \[ (1 - M) \sum_{k = 0}^\infty M^k = \sum_{k = 0}^\infty (1 - M)M^k = \sum_{k = 0}^\infty M^k - M^{k+1} = \lim_{n \to \infty} 1 - M^{n+1} \]
    %
    As $n \to \infty$, $M^{n+1} \to 0$, so the limit above tends to one. To obtain a right side inverse, note $(1 - M)$ commutes with $\sum_{k = 0}^n M^k$ for each $k$, so that, by continuity, $(1 - M)$ commutes with $\sum_{k = 0}^\infty M^k$, and thus $\sum_{k = 0}^\infty M^k$ is also a right sided inverse.
\end{proof}

\begin{corollary}
    If $\| 1 - M \| < 1$, then $M \in GL(A)$, and
    %
    \[ M^{-1} = \sum_{k = 0}^\infty (1 - M)^k \]
\end{corollary}

\begin{corollary}
    $GL(A)$ is an open subset of $A$.
\end{corollary}
\begin{proof}
    If $M \in GL(A)$, and if $\| M - N \| < 1/\| M^{-1} \|$, then 
    %
    \[ \| 1 - M^{-1}N \| \leq \| M^{-1} \| \| M - N \|  < 1 \]
    %
    so $M^{-1}N \in GL(A)$, and thus $N \in GL(A)$.
\end{proof}

\begin{corollary}
    $\sigma(M)$ is a closed and bounded subset of $\mathbf{C}$, and $\rho(M)$ is open.
\end{corollary}
\begin{proof}
    The map $f: \lambda \mapsto \lambda - M$ is a continuous operation, for
    %
    \[ \| (\lambda - M) - (\mu - M) \| = \| \lambda - \mu \| = | \lambda - \mu | \]
    %
    Since $GL(A)$ is open, $f^{-1}(GL(A)) = \rho(M)$ is open, hence $\sigma(M)$ is closed. If $|\lambda| > \|M\|$, then $\| M/\lambda \| < 1$, so $(1 - M/\lambda) \in GL(A)$, which means $\lambda - M$ is also invertible. Thus $\sigma(A)$ is closed and bounded, hence compact.
\end{proof}

We shall see that the spectra of complex algebras are never empty. This is why we mainly study complex algebras, rather than real algebras; there are even finite dimensional real operators with empty spectra.

\begin{example}
    Consider the matrix
    %
    \[ M = \begin{pmatrix} 0 & 1 \\ -1 & 0 \end{pmatrix} \]
    %
    as an element of the Banach algebra $M_2(\mathbf{R})$. The characteristic polynomial is calculated to be $\lambda^2 + 1$, so the matrix has no eigenvalues in the real numbers, and correspondingly, $\sigma(M) = \emptyset$. Over $M_2(\mathbf{C})$, we find $\sigma(M) = \{ i, -i \}$, so the spectrum is non-empty in the complex extension of the Banach algebra.
\end{example}

\begin{lemma}
    Inversion in an operator algebra $A$ is continuous.
\end{lemma}
\begin{proof}
    Let $(M_n)$ be a sequence in $GL(A)$ converging to an invertible element $M$. Then, by continuity, $M_nM^{-1} \to 1$. If $\| 1 - M_n M^{-1} \| < 1/2$, then
    %
    \[ M_n^{-1} M = (M^{-1}M_n)^{-1} = \sum_{k = 0}^\infty (1 - M^{-1}M_n)^k \]
    %
    It follows that
    %
    \begin{align*}
        \| M_n^{-1} \| &\leq \| M^{-1} \| \| M_n^{-1} M \| \leq \| M^{-1} \| \sum_{k = 0}^\infty \| (1 - M^{-1} M_n)^k \|\\
        &\leq \| M^{-1} \| \sum 2^{-k} = 2 \| M^{-1} \|
    \end{align*}
    %
    Finally, we obtain convergence of inverses,
    %
    \[ \| M_n^{-1} - M^{-1} \| = \| M_n^{-1} (M - M_n) M^{-1} \| \leq \| M_n^{-1} \| \| M - M_n \| \| M^{-1} \| \]
    %
    which tends to zero, since the first and third values are bounded, and the second converges to zero. Thus inversion is continuous.
\end{proof}

The invertible elements $GL(A)$ cannot form a closed set of $A$, for $A$ is path-connected, so there is a sequence of invertible elements $x_i$ which converge to a non-invertible element. Fortunately, Banach algebras have a way of telling us when things are about to go wrong; the norm of the operator blows up.

\begin{lemma}
    If $x_i \to x$, each $x_i \in GL(A)$, and $x \not \in GL(A)$, then $\| x_i^{-1} \| \to \infty$.
\end{lemma}
\begin{proof}
    If $\| x_n^{-1} \| \leq M$ for all $n$, then
    %
    \[ \| 1 - x_n^{-1} x \| \leq \| x_n^{-1} \| \| x_n - x \| \leq M \| x_n - x \| \]
    %
    If we choose $n$ large enough that $\| x_n - x \| < 1/M$, then $x_n^{-1} x$ is invertible, implying $x$ is invertible.
\end{proof}

\begin{theorem}
    If $B$ is a closed subalgebra of $A$, and $M \in GL(B)$, then the entire connected component of $M$ in $B \cap GL(A)$ is contained in $GL(B)$.
\end{theorem}
\begin{proof}
    $GL(A)$ is open, so $B \cap GL(A)$ is open in $B$. If $M_n \to M$, with $M_n \in GL(B)$, and $M \in GL(A)$, then $\| M_n^{-1} \| \to \| M^{-1} \| < \infty$, so by the last lemma, $M$ must be invertible in $B$.
\end{proof}

\begin{corollary}
    If $B$ is a closed subalgebra of $A$, then $\sigma_B(M)$ is obtained from $\sigma_A(M)$ by adding certain components of $\rho_A(M)$.
\end{corollary}
\begin{proof}
    $\sigma_B(M)$ is homeomorphic to the set $GL(B) \cap \{ \lambda - M : \lambda \in \mathbf{C} \}$, and $\sigma_A(M)$ homeomorphic to $GL(A) \cap \{ \lambda - M : \lambda \in \mathbf{C} \}$. Any component in $GL(B) \cap \{ \lambda - M : \lambda \in \mathbf{C}$ can be enlarged to a component of $GL(B)$, so if $\lambda \in \sigma_A(M)$, consider the component of $\lambda - M$ in $GL(B)$.
\end{proof}

\begin{corollary}
    If $A \subset B$ are algebras, and $\sigma_B(M) \subset \mathbf{R}$, then $\sigma_A(M) = \sigma_A(M)$.
\end{corollary}
\begin{proof}
    $\sigma_B(M)$ is a bounded subset of $\mathbf{R}$, so $\rho_B(M)$ is connected in $\mathbf{C}$. Hence $\sigma_A(M) = \sigma_B(M)$, since $\sigma_A(M) \neq \mathbf{C}$, and if we added any more points to the spectrum we would have to add the entire component, which would have to be all of $\rho_B(M)$.
\end{proof}

\subsection{Holomorphicity and Resolvent Formalism.}

The fundamental theorem of spectral theory relies on heavy complex analysis, hence its restricted application to Banach algebras over the complex numbers. Define the resolvent of $M$, defined on $\rho(M)$ by
%
\[ R(z; M) = (M - z)^{-1} \]
%
Fixing $M$, we obtain an analytic function into $A$.

\begin{lemma}
    $R$ is analytic, in the sense that
    %
    \[ \lim_{w \to 0} \frac{R(z + w, M) - R(z,M)}{w} \]
    %
    converges to a well defined value. for all $z \in \rho(M)$
\end{lemma}
\begin{proof}
    The proof is a pure computation.
    %
    \begin{align*}
        \lim_{w \to 0} &\frac{R(z + w, M) - R(z,M)}{w}\\
        &= \lim_{w \to 0} \frac{(z + w - M)^{-1} - (z - M)^{-1}}{w}\\
        &= \lim_{w \to 0} (z + w - M)^{-1} \frac{(z - M) - (z + w - M)}{w} (z - M)^{-1}\\
        &= \lim_{w \to 0} -(z + w - M)^{-1} (z - M)^{-1}\\
        &= -(z - M)^{-2}
    \end{align*}
    %
    We apply the continuity of multiplication and inversion.
\end{proof}

In Banach theory, we call such a mapping {\bf strongly analytic}. A {\bf weakly analytic function} is $f: \mathbf{C} \to A$ for which $\langle \phi, f \rangle$ is analytic for any choice of $\phi \in A^*$. It is clear that all strongly analytic functions are weakly analytic, which follows because
%
\[ \lim_{w \to 0} \phi \left( \frac{f(z + w) - f(z)}{w} \right) = \phi \left( \lim_{w \to 0} \frac{f(z + w) - f(z)}{w} \right) \]
%
For fun, we shall also prove that all weakly analytic functions are strongly analytic.

\begin{theorem}
    Every weakly analytic function $f: D \to X$ is strongly analytic.
\end{theorem}
\begin{proof}
    Fix $\phi \in X^*$. Consider a particular contour winding counterclockwise around a point $w$ in the domain, which is at least a unit distance away from $w$ at any point on the contour. If $h,k \in \mathbf{C}$ are small enough that $w + h$ and $w + k$ are contained within the contour, then by the Cauchy integral theorem,
    %
    \begin{align*}
        &\phi\left( \frac{1}{h-k} \left[ \frac{f(w + h) - f(w)}{h} - \frac{f(w + k) - f(w)}{k} \right] \right)\\
        &\ \ \ \ \ =  \phi \left( \frac{f(w)}{hk} + \frac{f(w + h)}{(h - k)h} - \frac{f(w + k)]}{(h - k)k} \right)\\
        &\ \ \ \ \ = \frac{1}{2\pi i} \int_C \frac{\phi[f(z)]\ dz}{[z - (w + h)][z - (w + k)][z - w]}
    \end{align*}
    %
    Find $\delta$ such that if $\| h \| < \delta$, the distance between any point on $C$ and $w + h$ is greater than $1/2$. Then, if $M$ is the length of $C$, and $K$ is the supremum of $f$ on $C$, then
    %
    \[ \left| \frac{1}{2\pi i} \int_C \frac{\phi[f(z)]\ dz}{[z - (w + h)][z - (w + k)][z - w]}\right| \leq \frac{4MK}{2 \pi} \| \phi \| = \frac{2MK}{\pi} \| \phi \| \]
    %
    Applying the Banach Steinhaus theorem (on $X^*$, viewing elements of $X$ as elements of $X^{**}$), we conclude that for all $h,k$ sufficiently small, there exists $K$ such that
    %
    \[ \left| \frac{f(w + h) - f(w)}{h} - \frac{f(w + k) - f(k)}{k} \right| \leq K |h - k| \]
    %
    By the completeness of $X$, the quotients of $h$ and $k$ converge to a well defined quantity as $h - k$ converges to zero.
\end{proof}

The non-emptiness of the spectrum relies on Louiville's theorem, in Banach space form. The reason for this is that the resolvent is an entire function if the spectrum exists.

\begin{theorem}
    A bounded weakly analytic function $f: \mathbf{C} \to X$ is constant.
\end{theorem}
\begin{proof}
    Let $\phi \in X^*$. Then $\phi \circ f$ is an analytic function, and
    %
    \[ | (\phi \circ f)(z) | \leq \| \phi \| \| f \|_\infty \]
    %
    Louiville tells us there is $w \in \mathbf{C}$ such that $\phi \circ f = w$. Fix $x,y \in \mathbf{C}$. Then $\phi[f(x) - f(y)] = 0$ for all $\phi \in X^*$. By the Hahn Banach theorem, since $\phi$ was arbitrary, we must have $f(x) - f(y) = 0$, so $f(x) = f(y)$.
\end{proof}

There is a deep relationship between complex analysis and Banach algebras. We shall return to `holomorphic functional analysis' later.

\begin{theorem}
    The spectrum of a complex Banach algebra is non-empty.
\end{theorem}
\begin{proof}
    Assume $\sigma(M)$ is empty. Then $R(\cdot, M)$ is an analytic, entire function, and tends to zero at infinity, since if $|\lambda| > N \| M \|$,
    %
    \begin{align*}
        \| R(\lambda, M) \| &= \| (\lambda - M)^{-1} \| = \frac{1}{|\lambda|} \left\| \sum_{k = 0}^\infty \frac{M^k}{\lambda^k} \right\| \leq \sum_{k = 0}^\infty \frac{\| M \|^k}{|\lambda|^{k+1}}\\
        &\leq \frac{1}{|\lambda|} \sum_{k = 0}^\infty \frac{1}{N^{k+1}} = \frac{1}{N(N - 1)\|M\|}
    \end{align*}
    %
    This implies $\| R(\cdot, M) \|$ is therefore constant. Since it tends to zero at infinity, $R(z, M) = 0$ for all $z \in \mathbf{C}$. But then
    %
    \[ R(z, M) = (z - M)^{-1} = 0 \]
    %
    and this is clearly impossible for any particular $\lambda$.
\end{proof}

A cute little theorem arises from this property of Banach algebras that will surprisingly find great use in the analysis of abelian Banach algebras.

\begin{corollary}
    Every complex Banach division algebra is isometric to $\mathbf{C}$.
\end{corollary}
\begin{proof}
    In any unital Banach algebra $A$, $\mathbf{C} \cdot 1$ is isometric to $\mathbf{C}$, for
    %
    \[ \| \lambda \cdot 1 \| = |\lambda| \| 1 \| = |\lambda| \]
    %
    Let $A$ be a complex division algebra, and fix $M \in A$. Pick some $\lambda \in \sigma(A)$. Then $\lambda - M \not \in GL(A)$, hence $\lambda - M = 0$, i.e. $M = \lambda$. Thus $A = \mathbf{C} \cdot 1 \cong \mathbf{C}$.
\end{proof}

The real case is much more complicated. There are three real division algebras: $\mathbf{R}$, $\mathbf{C}$, and $\mathbf{Q}$ (the quaternions), and it is much more difficult to show that these are the only three.

\begin{theorem}
    If $A$ is a Banach algebra for which $M$ exists such that
    %
    \[ \| x \| \| y \| \leq M \| x y \| \]
    %
    for all $x,y \in \mathbf{C}$, then $A$ is isometric to $\mathbf{C}$.
\end{theorem}
\begin{proof}
    Let $y \in \partial GL(A)$. Then there are $y_i \to y$ with $y_i \in GL(A)$. Thus $\| y_i^{-1} \| \to \infty$. But since
    %
    \[ \| y_i \| \| y_i^{-1} \| \leq M \]
    %
    we must have $\| y_i \| \to 0$, so $y = 0$. Suppose $z \not \in GL(A)$. Consider the line between $z$ and 1. Consider
    %
    \[ \lambda = \inf \{ \gamma \in [0,1] : \gamma + (1 - \gamma) z \in GL(A) \} \]
    %
    then $\gamma + (1 - \gamma) z \in \partial GL(A)$, so $\gamma + (1 - \gamma) z = 0$, and since $\gamma \neq 1$ (since $GL(A)$ is open), we conclude $z = \gamma/(1-\gamma)$, and since $z$ is not invertible, we must have $z = 0$. Thus $A$ is a division algebra, so Gelfand implies that $A$ is isometric to $\mathbf{C}$.
\end{proof}

\begin{theorem}
    $\sigma$ is `continuous', in the sense that for any open set $U \subset \mathbf{C}$ containing $\sigma(M)$, there is an open neighbourhood $V$ of $M$ such that if $N \in V$, $\sigma(N) \subset U$.
\end{theorem}
\begin{proof}
    The map $\lambda \mapsto \| (\lambda - M)^{-1} \|$ is a bounded continous function in $U^c$, with some bound $K$. If $\| N \| < 1/K$, and $\lambda \in U^c$, then
    %
    \[ \lambda - (M + N) = (\lambda - M)(1 - (\lambda - M)^{-1}N) \]
    %
    is invertible, since $(\lambda - M)$ is invertible, since $\| (\lambda - M)^{-1} N \| < 1$. It follows that $\sigma(M + N) \subset U$.
\end{proof}

\subsection{Spectral Radii}

The {\bf spectral radius} of an element $M \in A$ is defined to be
%
\[ r(M) = \sup \{ |\lambda| : \lambda \in \sigma(M) \} \]
%
It bounds where to look for the spectrum of $M$ occurs. What is amazing is that we can define the spectral radius without any algebraic reference; this is crazy, since if we enlarge our Banach algebra, more elements become invertible, and thus the spectrum shrinks in size. The radius formula says that there still exists points on the boundary of the spectrum.

\begin{lemma}
    Let $M \in A$, $n \in \mathbf{N}$. Then, if $\lambda \in \sigma(M)$, $\lambda^n \in \sigma(M^n)$.
\end{lemma}
\begin{proof}
    Suppose $\lambda \in \sigma(M)$. Then
    %
    \[ \lambda^n - M^n = (\lambda - M) \left(\sum \lambda^{n-1-k} M^k \right) = \left(\sum \lambda^{n-1-k} M^k \right) (\lambda - M) \]
    %
    If $\lambda - M$ was invertible, then $\lambda^n - M^n$ would also be invertible.
\end{proof}

\begin{theorem}[Spectral Radius Theorem]
    \[ r(M) = \lim_{n \to \infty} \| M^n \|^{1/n} \]
\end{theorem}
\begin{proof}
    If $\lambda \in \sigma(M)$, then $\lambda^n \in \sigma(M^n)$, and therefore by Neumann's lemma, $|\lambda|^n \leq \| M^n \|$. We conclude
    %
    \[ |\lambda| \leq \| M^n \|^{1/n} \]
    %
    taking extrema,
    %
    \[ r(M) = \sup_{\lambda \in \sigma(M)} |\lambda| \leq \liminf_{n \to \infty} \|M^n\|^{1/n} \]
    %
    Set $R = r(M)^{-1}$ (which can be $\infty$, if $r(M) = 0$), and $r = \|M\|^{-1}$. Let $\lambda$ be a complex number with modulus less than $R$. Then $1/|\lambda| > r(M)$, so $1 - \lambda M \in GL(A)$. If $\phi \in A^*$, define
    %
    \[ f: \lambda \mapsto \langle \phi, (1 - \lambda M)^{-1} \rangle \]
    %
    Then $f$ is holomorphic in the disk of radius $R$. If $|\lambda| < r$, then $\| \lambda M \| < 1$, $1 - \lambda M \in GL(A)$, and
    %
    \[ \phi \left( (1 - \lambda M)^{-1} \right) = \sum_{k = 0}^\infty \phi( \lambda^k M^k) \]
    %
    power series expansions are unique, hence this expansion should work in the whole disk of radius $R$. But $\phi$ was arbitrary, so the sequence $\lambda^k M^k$ must be bounded, by Banach Steinhaus. If $\lambda$ is fixed, then there is $C$ such that
    %
    \[ |\lambda^n| \|M^n\| \leq C \]
    %
    for all $n$, so
    %
    \[ \|M^n\|^{1/n} \leq \frac{C^{1/n}}{|\lambda|} \]
    %
    Hence
    %
    \[ \limsup_{n \to \infty} \|M^n\|^{1/n} \leq \lim_{n \to \infty} \frac{C^{1/n}}{\lambda} = \frac{1}{\lambda} \]
    %
    Letting $\lambda \to R$, we obtain that $\limsup \|M^k\|^{1/k} \leq r(a)$. We have shown
    %
    \[ \liminf \|M^k\|^{1/k} \geq r(M) \geq \limsup \|M^k\|^{1/k} \]
    %
    from which the theorem follows.
\end{proof}

\begin{corollary}
    The spectral radius of a Banach algebra element is invariant of which Banach algebra the element is in. If $B$ is a Banach subalgebra of $A$, with $M \in B$, then $r_{A}(M) = r_B(a)$.
\end{corollary}

\begin{example}
    We can isometrically embed $A(\mathbf{D})$ in $C(\mathbf{T})$ via the map $f \mapsto f|_\mathbf{T}$, so that we may view $A(\mathbf{D})$ as a closed subspace of $C(\mathbf{T})$. These properties follows simply from the maximum modulus principle. Let $z: \mathbf{T} \to \mathbf{C}$ be the identity map. Then
    %
    \[ \sigma_{A(\mathbf{D})}(z) = \mathbf{D} \supsetneqq \mathbf{T} = \sigma_{C_0(\mathbf{T})}(z|_\mathbf{T}) \]
    %
    while the spectrum are different, the spectral radius is the same.
\end{example}

In our last example, enlarging the spectrum of $f$ from $\mathbf{T}$ to $\mathbf{D}$ ws the only thing that could occur from enlarging the algebra, for we cannot add any more points outside of the unit disk, and adding a single point in the unit disk will add the entire component to the spectrum.

\subsection{Non-Unital Spectrum \& Non-Commutative Topology}

We would like to extend these tools to non-unital algebras, but there is no way to talk about invertibility since there is no unit to invert to! We still need to analyze these algebras, since they naturally occur in analysis. If $X$ is not compact, then $C_0(X)$ does not contain an identity. Though we have noted that there is usually a natural trick for adding an identity, this modifies the structure of the algebra (for instance, the ideals change). Nonetheless, it is psychologically relieving to find that all non-unital Banach algebras can be isometrically embedded into Banach algebras with unit. Given a non-unital algebra $A$, consider
%
\[ A \ltimes \mathbf{C} = \{ (a,\lambda) : a \in A, \lambda \in \mathbf{C} \} \]
%
with a multiplicative structure
%
\[ (M + \lambda)(N + \gamma) = MN + \lambda N + \gamma M + \lambda \gamma \]
%
We have an identity $0 + 1$, and the space considered still satisfies the properties that make it a Banach algebra if we give it the norm
%
\[ \| M + \lambda \| = \| M \| + |\lambda| \]
%
If $M_i + \lambda_i$ is a cauchy sequence, then $M_i$ and $\lambda_i$ are both separately cauchy sequences, and therefore converge to a well defined quantity, which also converges in the abstract norm. We may then define $\sigma_A(M) = \sigma_{A \ltimes \mathbf{C}}(M)$. The fundamental theorem of spectral theory still applies, as does the spectral radius formula.

Given $X$, we may embed $C_0(X)$ in $C(X^*)$, the continuous functions on the one-point compactification of $X$. We then find that $\mathbf{C} \ltimes C_0(X)$ is isometric to $C(X^*)$, so unitization on an arbitrary Banach algebra corresponds to the one-point compactification of the `underlying space', even if there is no underlying space. This is the premise behind the school of non-commutative topology. We shall soon see that all commutative Banach algebras can be seen as subalgebras of continuous functions on some compact space. The school of non-commutative topology contends that the intuition gained from the algebra of continuous functions gives us intuitions about arbitrary Banach algebras, which are seen as spaces of continuous functions over `noncommutative space'.

\subsection{Bounded Approximate Identities}

We may not have an identity in a non-unital Banach algebra, but we may be able to `approximate' an identity in some sense. An {\bf approximate identity} (or {\bf AI}) for an algebra $A$ is a net $\{ M_\alpha \}$ such that for any $N \in A$,
%
\[ \lim M_\alpha N = N\ \ \ \ \ \ \ \ \lim N M_\alpha = N \]
%
Nets for which one of these equations hold are known as {\bf left} or {\bf right} approximate identities. A {\bf bounded approximate identity} ({\bf BLAI} for short) is an approximate identity for which $\sup \| M_\alpha \| < \infty$.

\begin{example}
    Let $X$ be locally compact and Hausdorff. The set of all compact subsets is a directed, exhausting set. Using Urysohn's lemma, find $f_K$, for each compact $K$, such that $f_K |_K = 1$, and $\| f_K \|_\infty \leq 1$. Fix $g \in C_0(X)$. Pick $K$ such that $|g| \leq \varepsilon$ outside of $K$. Then, in $K$, $|f_K g - g| = 0$, and outside of $K$, $|f_K g - g| < 2 \varepsilon$. Thus $\| f_K g - g \|_\infty < 2 \varepsilon$. $\{ f_K \}$ is not cauchy, for if $|f_K| < \varepsilon$ outside of $K'$, then $\| f_K - f_{K'} \|_\infty > 1 - \varepsilon$.
\end{example}

\begin{example}
    Consider the net $P_r(z) = \sum_\mathbf{Z} r^{|n|} z^n$ in the group algebra $L^1(S^1)$, for $0 < r < 1$. Given a continuous function $g$, it is well known that $(P_r * g) \to g$ uniformly. Given $g \in L^1(S^1)$, pick $h \in C(S^1)$ with $\| g - h \|_1 < \varepsilon$. Then if $\| h * P_r - h \|_\infty < \varepsilon$ as well, then
    %
    \begin{align*}
        \| g * P_r - g \|_1 &\leq \|(g - h) * P_r \|_1 + \| h * P_r - h \|_1 + \| h - g \|_1\\
        &\leq \|g - h\|_1 \| P_r \|_1 + \| h * P_r - h \|_\infty + \| h - g \|_1\\
        &\leq 3 \varepsilon
    \end{align*}
    %
    And we see that $g * P_r \to g$ in $L^1$.
\end{example}

\begin{example}
    If $H$ is a Hilbert space, $B(H)$ always has the bounded approximation property, as do the classical sequence spaces $c_0$ and $l_p$. Finding a space without the approximation property was an open problem for more than 40 years. Proposed as a challenge by Stanislaw Mazur in 1936, the solver of this problem was promised a live goose. In 1972, the goose was granted to the swedish mathematician Per Enflo, who solved the problem.
\end{example}

Advanced Banach space theory makes large uses of bounded left approximate identities. A Banach space $E$ has the {\bf approximation property} if there is a net $\{ T_\alpha \}$ of finite rank operators which tend to $\text{id}_E$ uniformly on compact subsets. The space has the {\bf bounded approximation property} if $\| T_\alpha \|$ is bounded. Uniform convergence on compact sets is equivalent to pointwise convergence for bounded operators. Choose some compact $K$, fix $\varepsilon > 0$, and pick $x_1, \dots, x_n$ for which $\{ B_\varepsilon(x_i) \}$ is a cover of $K$. Then, if $x \in K$, there is $x_j$ with $\| x - x_j \| < \varepsilon$, and then
%
\begin{align*}
    \| T_\alpha x - x \| &\leq \| T_\alpha x - T_\alpha x_j \| + \| T_\alpha x_j - x_j \| + \| x_j - x \|\\
    &\leq \varepsilon \sup \|T_\alpha\| + \| T_\alpha x_j - x_j \| + \varepsilon
\end{align*}
%
If we choose $\alpha$ big enough, then this is guaranteed to be less than $3 \varepsilon$. For any Banach space $X$, we denote by $A(X)$ the closure of all finite rank operators in $B(X)$. It is clear than $A(X) \subset K(X)$. The problem of whether $A(X) = K(X)$ is much more subtle.

\begin{example}
    Let $X$ be a Banach space with the bounded approximation property. Let $\{ T_\alpha \}$ be a bounded net. If $S$ is a compact operator, then $S(\overline{B_X})$ is precompact, and
    %
    \[ \| T_\alpha S - S \| = \sup \{ \| T_\alpha Sx - Sx \| : x \in B_X \} \leq \sup \{ \| T_\alpha y - y \| : y \in \overline{S(B_X)} \} \]
    %
    The right side is the supremum over a compact set, and since $\{ T_\alpha \}$ uniformly tends to the identity on compact sets, $\| T_\alpha S - S \| \to 0$. This shows $\{ T_\alpha \}$ is a bounded left approximate identity, and that $A(X) = K(X)$.

    Now suppose $A(X)$ has a bounded left approximate identity $\{ T_\alpha \}$. Without loss of generality, we may assume each $T_\alpha$ is of finite rank, because if we choose, for each $\alpha$, a bounded net $W_{\alpha, \beta}$ of finite rank operators such that $\lim_\beta W_{\alpha, \beta} = T_\alpha$, then $\{ W_{\alpha, \beta} \}$ is a bounded left approximate identity. For $x \in X$, $\phi \in X^*$, let $x \otimes \phi : X \to X$ be defined by $(x \otimes \phi)(y) = \phi(y) x$. Then $x \otimes \phi$ obviously has finite rank. Pick $y$ for which $\langle \phi, y \rangle = 1$. Then
    %
    \[ \| T_\alpha x - x \| = \| T_\alpha (x \otimes \phi)(y) - (x \otimes \phi)(y) \| \leq \| y \| \| T_\alpha (x \otimes \phi) - (x \otimes \phi) \| \to 0 \]
    %
    So the nets converge pointwise, which implies convergence on compact sets. To summarize, $X$ has the bounded approximxation property if and only if $A(X)$ has a bounded left approximate identity if and only if $K(X)$ has a bounded left approximate identity belonging to $A(X)$.
\end{example}

\begin{lemma}
    If a space has the bounded left and right approximation properties then it has the two sided approximation property.
\end{lemma}
\begin{proof}
    Let $\{ M_\alpha \}$ be a left approximation identity, and $\{ N_\beta \}$ a right approximation identity. We contend $\{ M_\alpha + N_\beta - M_\alpha N_\beta \}$ is a two sided approximator. The limits below certainly converge, and the iterated limits must therefore equal the convergent factor, which is
    %
    \[ \lim_\alpha \lim_\beta L(M_\alpha + N_\beta - M_\alpha N_\beta) = \lim_\alpha LM_\alpha + L - LM_\alpha = L \]
    %
    \[ \lim_\beta \lim_\alpha (M_\alpha + N_\beta - M_\alpha N_\beta)L = \lim_\beta L + N_\beta L - N_\beta L = L \]
    %
    So twe have a two sided approximator.
\end{proof}

Before the logician Paul Cohen got into logic, he was a functional analyst who contributed to the theory of approximation identities. We shall prove the theorem he contributed to the field, generalized to work over arbitrary modules. Let $A$ be a Banach algebra. A {\bf left banach $A$-module} is a Banach space $X$, which is a module over $A$, for which for any $M \in A$, $x \in X$,
%
\[ \| Mx \| \leq \| M \| \| x \| \]
%
This makes the module operations continuous. As with Banach algebras, if module operations are continuous, then a Banach space can be made into a Banch $A$-module.

Now if $A$ is a unital Banach algebra, then every $M \in A$ can be trivially written as $NL$, with $NL \in A$ (just let $N = 1$). If we have an approximate identity, then we may write $M$ approximately as $NL$. Suprisingly, Cohen found that we need not approximate this factorization.

\begin{theorem}[Cohen's Factorization Theorem]
    Let $A$ be a Banach algebra with BLAI $\{ E_\alpha \}$, bounded by $K$. If $X$ is a Banach $A$-module, let $x \in \overline{AX}$, and let $\varepsilon > 0$. Then there is $M \in A$ with $\| M \| \leq K$, $y \in \overline{AE}$ with $\| y - x \| < \varepsilon$, for which $x = My$.
\end{theorem}
\begin{proof}
    If $A$ has an identity, then the proof is trivial. Since $\overline{AX}$ is a closed subset of $X$, we might as well assume $\overline{AX} = X$. We may extend $X$ to be an module over $A \ltimes \mathbf{C}$, by defining 
    %
    \[ (M + \lambda) x = Mx + \lambda x \]
    %
    Pick $\lambda > 0$ small enough such that
    %
    \[ 0 < \frac{\lambda}{1 - \lambda} K < 1 \]
    %
    Define a net $\{ L_\alpha \}$ in $A \ltimes \mathbf{C}$ by letting
    %
    \[ L_\alpha = \lambda E_\alpha + (1 - \lambda) = (1 - \lambda) \left( 1 + \frac{\lambda}{1 - \lambda} E_\alpha \right) \]
    %
    Then $L_\alpha N \to N$ for all $N \in A$, so $\{ L_\alpha \}$ is a BLAI, and each $L_\alpha$ is invertible in $A \ltimes \mathbf{C}$ by the choice of $\lambda$, so
    %
    \[ L_\alpha^{-1} = \frac{1}{1 - \lambda} \sum_{k = 0}^\infty \left(- \frac{\lambda}{1 - \lambda} L_\alpha \right)^k \]
    %
    which implies
    %
    \begin{align*}
        \| L_\alpha^{-1} N - N \| &= \| L_\alpha^{-1} N - L_\alpha^{-1} L_\alpha N \| \leq \| L_\alpha^{-1} \| \| N - L_\alpha N \|\\
        &\leq \left( \frac{1}{1 - \lambda} \sum_{k = 0}^\infty \left( \frac{\lambda K}{1 - \lambda} \right)^k \right) \| N - L_\alpha N \| \to 0
    \end{align*}
    %
    Therefore $L_\alpha^{-1}$ is also a BLAI. Fix $x \in X$, and let
    %
    \[ \delta < 1, \frac{\varepsilon}{2 + \|x\|} \]
    %
    We will inductively construct a sequence $\alpha_n$ of indices such that the sequence $\{ L_{\alpha_n} \dots L_{\alpha_1} \}$ and $\{ L_{\alpha_1}^{-1} \dots L_{\alpha_n}^{-1} \}$ are convergent subnets. Choose $\alpha_1$ such that
    %
    \[ \| L_{\alpha_1}^{-1} x - x \| < \delta/2 \]
    %
    If $\alpha_1, \dots, \alpha_n$ has been chosen. There is a unique element $T_n \in A$ such that
    %
    \[  L_{\alpha_n} \dots L_{\alpha_1} = (1 - \lambda)^n + T_n \]
    %
    Pick $\alpha_{n+1}$ such that
    %
    \[ \| L_{\alpha_{n+1}} x - x \| < \frac{\delta}{\| L_{\alpha_1}^{-1} \dots L_{\alpha_n}^{-1} \| 2^{n+2}}\ \ \ \ \ \ \ \ \ \ \| L_{\alpha_{n+1}} T_n - T_n \| < 1/2^n \]
    %
    Then
    %
    \[ \| L_{\alpha_1}^{-1} \dots L_{\alpha_n}^{-1} L_{\alpha_{n+1}}^{-1} x - L_{\alpha_1}^{-1} \dots L_{\alpha_n}^{-1} x \| \leq \| L_{\alpha_1}^{-1} \dots L_{\alpha_n}^{-1} \| \| x - L_{\alpha_{n+1}}^{-1} x \| < \frac{\delta}{2^{n+2}} \]
    %
    This implies $\lim_{n \to \infty} L_{\alpha_1}^{-1} \dots L_{\alpha_n}^{-1} x = y$ exists, and
    %
    \begin{align*}
    \| y - x \| &= \left\| \sum_{n = 0}^\infty (L_{\alpha_1}^{-1} \dots L_{\alpha_{n+1}^{-1}} x - L_{\alpha_1}^{-1} \dots L_{\alpha_n}^{-1} x) \right\|\\
    &\leq \sum_{n = 0}^\infty \| L_{\alpha_1}^{-1} \dots L_{\alpha_{n+1}^{-1}} x - L_{\alpha_1}^{-1} \dots L_{\alpha_n}^{-1} x \| < \sum_{n = 0}^\infty \frac{\delta}{2^{n+2}} < \delta/2
    \end{align*}
    %
    By construction,
    %
    \[ L_{\alpha_{n+1}} (L_{\alpha_n} \dots L_{\alpha_1}) = (1 - \lambda)^n L_{\alpha_{n+1}} + L_{\alpha_{n+1}} T_n \]
    %
    Thus
    %
    \begin{align*}
        \| L_{\alpha_{n+1}} L_{\alpha_n} \dots L_{\alpha_1} - L_{\alpha_n} \dots L_{\alpha_1} \| &= \| (1 - \lambda)^n L_{\alpha_{n+1}} + L_{\alpha_{n+1}} T_n - T_n - (1-  \lambda)^n \|\\
        &\leq (1 - \lambda)^n \| L_{\alpha_{n+1}} - 1 \| + \frac{1}{2^n}
    \end{align*}
    %
    Thus $L_{\alpha_n} \dots L_{\alpha_1}$ converge to some element $M$. Since $A$ is closed in $A \ltimes \mathbf{C}$, $M \in A$, and moreover,
    %
    \[ x = \lim_{n \to \infty} (L_{\alpha_1} \dots L_{\alpha_n}) (L_{\alpha_1}^{-1} \dots L_{\alpha_n}^{-1} x) = My \]
    %
    Also, observe that
    %
    \begin{align*}
        T_{n+1} &= L_{\alpha_{n+1}} \dots L_{\alpha_1} - (1 - \lambda)^{n+1}\\
        &= L_{\alpha_{n+1}} T_n + L_{\alpha_{n+1}} (1 - \lambda)^n - (1 - \lambda)^{n+1}\\
        &= L_{\alpha_{n+1}} T_n + (1 - \lambda)^n \lambda e_{\alpha_{n+1}}
    \end{align*}
    %
    Thus
    %
    \[ \| T_{n+1} - T_n \| = \| (L_{\alpha_{n+1}} T_n - T_n + (1 - \lambda)^n \lambda e_{\alpha_{n+1}} \| \leq 1/2^n + (1 - \lambda)^n \lambda K \]
    %
    Since $M = \lim T_n$,
    %
    \[ \| M \| \leq \| T_1 \| + \left(\sum_{n = 1}^\infty (1 - \lambda)^n \lambda K \right) + \delta \leq \lambda K + (1 - \lambda) K + \delta = K + \delta \]
    %
    Let $M' = \frac{K}{K + \delta} M$, $y' = \frac{K + \delta}{K} y$. Then $M'y' = My = x$, and by the choice of $\delta$ (kept hidden all this time),
    %
    \[ \| x - y' \| \leq \frac{\| Kx - Ky \| + \| \delta y \|}{K} < \delta + \delta \| x \| + \delta^2 \leq \delta(2 + \|x\|) = \varepsilon \]
    %
    And we have verified what was needed.
\end{proof}

\begin{corollary}
    If $A$ is a non-unital Banach algebra with a BLAI, then each $M \in A$ may be written $M = NL$, with $N,L \in A$.
\end{corollary}

\begin{corollary}
    Let $A$ be a Banach algebra with a BLAI, and let $\{ M_n \}$ be a sequence in $A$ converging to 0. Then there is $M \in A$ and $\{ N_m \}$ in $A$ tending to zero such that $M_k = M N_k$.
\end{corollary}
\begin{proof}
    Let $X$ be the set of all sequences in $A$ that converge to zero, with the $\| \cdot \|_\infty$ norm. Then $X$ is a left $A$ module, and $X = \overline{AX}$, since $A$ has a BLAI. Applying Cohen's theorem, we find that we may write
    %
    \[ (M_1, M_2, \dots) = M (N_1, N_2, \dots) \]
    %
    hence $M_i = M N_i$ for each $i$.
\end{proof}





\section{Gelfand Theory}

Gelfand realized that all commutative Banach algebras were really just manifestations of continuous functions over a compact set. The way to find a space representing a given algebra. This has incredibly important repurcussions for all of Banach algebra theory, since all algebras contain commutative subalgebras. The ingenious trick to Gelfand theory is how to find a compact space on which to represent this algebra. We shall first analyze $C(K)$, for compact $K$, and find that points in $K$ can be uniquely defined with maximal ideals in $C(K)$. This means that, in general, if a given algebra is to be represented by continuous functions, the only space we could represent it on would be homeomorphic to the space of maximal ideals, with some topology. In any algebra, we may consider this space, but it is only for abelian algebras where we obtain a full representation.

\subsection{Algebra Homomorphisms and Ideals}

Algebraically, if $A$ is an algebra, the ideals are the objects we may quotient $A$ by. Since $A$ is both a ring and a vector space, an ideal therefore must be a ring ideal and a subspace. This continues to hold in the non-unital cases. We can consider left, right, and two-sided ideals, and we'll denote them by gaudy letters such as $\mathfrak{a}$ and $\mathfrak{b}$. It is an easy consequence of ring theory that a proper left ideal cannot contain any left invertible elements, and a right ideal cannot contain any right invertible elements. It is also easy to verify that the closure of any ideal is an ideal. In the commutative case, the quotient of an algebra by a maximal ideal is a field. Furthermore, every left, right, or double sided ideal can be extended to a maximal ideal of the same type.

For completeness, we include a discussion of quotients of algebras by ideals, which will come very handy in the near future. Given an ideal $\mathfrak{a}$ in an algebra $A$, we can surely form a subalgebra $A/\mathfrak{a}$, which consists of equivalence classes of the form
%
\[ [M] = \{ N : M - N \in \mathfrak{a} \} \]
%
where multiplication is obtained by performing the opeartions in the equivalence classes. This also has a natural Banach space structure, under the quotient norm.
%
\[ \| X \| = \inf \{ \| N \| : N \in X \} \]
%
It remains to be seen that this is a Banach algebra. By definition, for each $M$ and $\varepsilon > 0$, there is $N \in X$ such that
%
\[ \| N \| \leq \| X \| + \varepsilon \]
%
Similarily, for another equivalence class $Y$, there is $L$ such that
%
\[ \| L \| \leq \| Y \| + \varepsilon \]
%
From which it follows that
%
\[ \| XY \| \leq \| NL \| \leq \| N \| \| L \| \leq \| X \| \| Y \| + [\| X \| + \| Y \|] \varepsilon + \varepsilon^2 \]
%
Letting $\varepsilon$ tend to zero, we find
%
\[ \| XY \| \leq \| X \| \| Y \| \]
%
so the quotient structure really is a Banach algebra.

Ideals are naturally connected with homomorphisms $f: A \to B$ between algebras, maps which are both ring homomorphisms and linear maps. Every ideal is the kernel of some homomorphism, and the kernel of every homomorphism is an ideal. Correspondingly, {\it closed ideals} in a Banach algebra correspond to continuous homomorphisms. The most tractable homomorphisms to study are the {\bf characters}, homomorphisms from a ring $A$ to $\mathbf{C}$. In this section, we will deduce their structure.

\begin{lemma}
    If $\| M \| \leq 1$, then $|\phi(M)| \leq 1$ for a character $\phi$. Thus $\| \phi \| \leq 1$.
\end{lemma}
\begin{proof}
    The kernel of every homomorphism cannot be all of $A$, and it is certainly an ideal of $A$. Therefore the kernel cannot contain any invertible elements. If $f(M) = \lambda$, then $\lambda - M \in \ker(M)$, so that $\lambda - M$ is not invertible. This implies, from our discussion of $\sigma(M)$, that $|\lambda| \leq 1$.
\end{proof}

\begin{corollary}
    Every character is a continuous linear map.
\end{corollary}

The next lemma follows because a proper ideal of an algebra cannot contain invertible elements.

\begin{lemma}
    Every maximal ideal of a Banach algebra is closed.
\end{lemma}
\begin{proof}
    Let $\mathfrak{a}$ be a maximal ideal of an algebra $A$. It is easy to show the closure of any ideal is an ideal. It follows that either $\overline{\mathfrak{a}} = \mathfrak{a}$ (so that $\mathfrak{a}$ is closed), or $\mathfrak{a}$ is dense in $A$. Suppose the second option holds. Let $M \in GL(A)$ be chosen. Then there is $M_i \in \mathfrak{a}$ converging to $M$. But then the $M_i$ are eventually invertible, since $GL(A)$ is open, from which we conclude $\mathfrak{a} = A$, a contradiction.
\end{proof}

\begin{corollary}
    If $\mathfrak{a}$ is a maximal ideal in an abelian Banach algebra, then $A/\mathfrak{a}$ is isometric to $\mathbf{C}$.
\end{corollary}
\begin{proof}
    For then $A/\mathfrak{a}$ is a Banach division algebra.
\end{proof}

\begin{lemma}
    The kernel of every character is a maximal ideal. In an abelian algebra, every maximal ideal corresponds to the kernel of some character.
\end{lemma}
\begin{proof}
    We apply the first isomorphism theorem. Let $\phi$ be a character. Then $\phi$ is surjective, for $\phi(\mathbf{C} \cdot 1) = \mathbf{C}$. Thus if $\mathfrak{a} = \ker(f)$, $A/\mathfrak{a} \cong \mathbf{C}$, implying $\mathfrak{a}$ is maximal. If $A$ is commutative, and $\mathfrak{a}$ is a maximal ideal, then $A/\mathfrak{a}$ is a division ring, so we may project $A$ down to $A/\mathfrak{a}$, and then apply an isometry to $\mathbf{C}$.
\end{proof}

\begin{lemma}
    If the kernels of two characters correspond, then the two characters are equal.
\end{lemma}
\begin{proof}
    Suppose $\ker(\phi) = \ker(\psi)$. Fix $x \in A$. Then
    %
    \[ \phi(x - \phi(x)) = \phi(x) - \phi(x) = 0 \]
    %
    so
    %
    \[ \psi(x - \phi(x)) = \psi(x) - \phi(x) = 0 \]
    %
    Thus $\psi(x) = \phi(x)$.
\end{proof}

\begin{corollary}
    In a commutative algebra, every maximal ideal of $A$ is the kernel of a unique complex homomorphism, so there is a one-to-one correspondence between maximal ideals and characters.
\end{corollary}

We denote the set of characters of a Banach algebra by $A$ by $\Phi_A$, known as the {\bf character space} of $A$. Gelfand's theory is the study of $\Phi_A$, and its relation to $A$.

\begin{example}
    Let $K$ be a compact space, and consider the algebra $C(K)$ of continuous functions. Given a closed subset $C$ of $K$, the set
    %
    \[ \{ f \in C(K) : f|_C = 0 \} \]
    %
    is an ideal of $C(K)$. We contend these are all such closed ideals. Let $\mathfrak{a}$ be an arbitrary closed ideal of $C(K)$. It is easy to see that the set
    %
    \[ C = \{ x \in K : (\forall f \in \mathfrak{a}: f(x) = 0) \} \]
    %
    is closed. Let $x_i \to x$. If $f(x_i) = 0$ for all $x_i$, then $f(x_i) \to f(x)$, so $f(x) = 0$. Let $g$ be an arbitrary continuous function with $g|_C = 0$. Fix $\varepsilon > 0$. Let $V = g^{-1}(B_\varepsilon(0))$. For each $y \in K - C$, there is a function $f_y \in \mathfrak{a}$ with $f_y(y) \neq 0$. We may assume $f_y(y) = g(y)$, since $\mathfrak{a}$ is closed under multiplication by a constant. Let $U_y = (f_y - g)^{-1}(B_\varepsilon(0))$. Then $\{ V \} \cup \{ U_y \}$ is an open cover of $K$, and thus has a finite subcover $V, U_{y_1}, \dots, U_{y_n}$. Let $h_V, h_{y_1}, \dots, h_{y_n}$ be a partition of unity subordinate to the cover. Consider the function $g' = \sum h_{y_i} f_{y_i} \in \mathfrak{a}$. For $x \in K - V$,
    %
    \[ |g'(x) - g(x)| = | \sum h_{y_i}(x) (f(x) - f_{y_i}(x)) | \leq \sum h_{y_i}(x) |f(x) - f_{y_i}(x)| < \varepsilon \]
    %
    For $x \in V$,
    %
    \[ |g'(x) - g(x)| = | \sum h_{y_i}(x) (f(x) - f_{y_i}(x)) - h_V(x) f(x) | < \varepsilon + h_V(x) f(x) \leq 2 \varepsilon \]
    %
    Thus $\| g' - g \|_\infty \leq 2 \varepsilon$. It follows that $g$ can be approximated to arbitrary precision by elements of $\mathfrak{a}$, so $g \in \mathfrak{a}$, since $\mathfrak{a}$ is closed. We conclude $\mathfrak{a}$ consists of all functions which vanish on $C$. Thus closed ideals of $C(K)$ are in one-to-one correspondence with closed sets of $K$. From this, it is fairly easy to identify the maximal ideals. If $C \subset D$, then the ideal corresponding to $D$ is a subset of the ideal corresponding to $C$. The closed ideal corresponding to $\emptyset$ is $C(K)$, so the maximal ideals correspond to the smallest, non-empty compact sets; these are just the points. Thus the character space of $C(K)$ is in one-to-one correspondence with $K$.
\end{example}

\begin{lemma}
    If $M \in GL(A)$ then $M \not \in \text{ker}(\phi)$ for all $\phi \in \Phi_A$. The converse holds in commutative algebras.
\end{lemma}
\begin{proof}
    If $M \in GL(A)$, we know $\phi(M) \neq 0$ for all $\phi \in \Phi_A$, for otherwise $\ker(\phi)$ would contain an invertible element, implying $\ker(\phi) = A$. Conversely, in a commutative ring, if $M \not \in GL(A)$, then $(M) \neq A$, so $(M)$ extends to a maximal ideal $\mathfrak{a}$, and the projection $\phi: A \to A/\mathfrak{a}$ contains $M$ in its kernel, so we have an algebra isomorphism from $A/\mathfrak{a}$ to $\mathbf{C}$ by Gelfand-Mazur, which does not enlarge the kernel.
\end{proof}

\begin{corollary} \label{spectralhomomorphism}
    If $\phi(M) = \lambda$ for some character $\phi$, then $\lambda \in \sigma(M)$. The converse holds for commutative algebras.
\end{corollary}

\begin{example}
    Things can go very wrong for noncommutative algebras. Let us classify the ideals of $M_n(\mathbf{C})$. Let $\mathfrak{a}$ be a left ideal of $M_n(\mathbf{C})$, and suppose that $\mathfrak{a}$ contains a non-zero element $M$. It follows from the elementary theory of matrices that there are matrices $N$ and $L$ such that $NML \in \mathfrak{a}$ is a diagonal matrix non-zero somewhere. By multiplying on the left by a basis matrix $e_{ij}$, and normalizing, we find that $\mathfrak{a}$ contains an element with a single non-zero entry on the diagonal. Multiplying by permutation matrices, we find that $\mathfrak{a}$ contains all matrices with a one in a certain position, and zeroes everywhere else. Finally, by summing up, we find $\mathfrak{a}$ contains all matrices in $M_n(\mathbf{C})$. Thus $M_n(\mathbf{C})$ is a {\it simple} algebra, it contains no non-trivial ideals. Consequently, we find every algebra homomorphism from $M_n(\mathbf{C})$ to another algebra must be injective, for the kernel is a proper ideal. Then, if $n > 1$, the character space of $M_n(\mathbf{C})$ is $\emptyset$; there is no injective homomorphism from $M_n(\mathbf{C})$ to $\mathbf{C}$.
\end{example}

There is a fairly natural topology on $\Phi_A$. We assign to each $M \in A$ the {\bf Gelfand transform} $\widehat{M}: \Phi_A \to \mathbf{C}$, which is defined by $\widehat{M}(\phi) = \phi(M)$. The {\bf Gelfand topology} is the weakest topology on $\Phi_A$ which makes each $\widehat{M}$ continuous. We shall denote the Gelfand transform itself as the map $\Gamma: A \to C(\Phi_A)$.

\begin{lemma}
    The map $M \mapsto \widehat{M}$ is a contractive algebra homomorphism into $C(\Phi_A)$, whose kernel is the Jacobson radical, the intersection of all maximal ideals. Thus the map is injective if and only if the algebra is semisimple. We have $\widehat{M}(\Phi_A) \subset \sigma(M)$, with equality in the commutative case.
\end{lemma}
\begin{proof}
    By direct calculation,
    %
    \[ \widehat{MN}(\phi) = \phi(MN) = \phi(M) \phi(N) = (\widehat{M} \widehat{N})(\phi) \]
    %
    If $\widehat{M} = 0$, then $\phi(M) = 0$ for all $\phi$. Thus $M$ is contained in every maximal ideal. The converse is equally trivial. That $\widehat{M}(\Phi_A) \subset \sigma(M)$ follows from Corollary \ref{spectralhomomorphism}. In the commutative case, if $\lambda \in \sigma(M)$, then $\lambda - M$ is contained in a unique maximal ideal in $A$, which corresponds to a unique homomorphism $\phi \in \Phi_A$ for which $\phi(\lambda - M) = 0$.
\end{proof}

It was Gelfand's idea to surplant the character spaces with a topological structure. This is the Gelfand topology of the character space. We shall now deduce its structure.

\begin{theorem}
    $\Phi_A$ into a compact, Hausdorff space.
\end{theorem}
\begin{proof}
    Let us first verify the Hausdorff condition. Given $\phi, \psi \in \Phi_A$, there is $M \in A$ such that $\phi(M) \neq \psi(M)$ (for otherwise they are equal as functions). If we pick a neighbourhood $U$ of $\phi(x)$ and a neighbourhood $V$ of $\psi(x)$ disjoint from each other, then $\widehat{M}^{-1}(U)$ and $\widehat{M}^{-1}(V)$ are disjoint neighbourhoods in $\Phi_A$ which separate $\phi$ and $\psi$.

    Compactness is a little trickier. We may view $\Phi_A$ as a subset of $A^*$, and $\Phi_A$ has a weaker topology than induced under the relative topology of $A^*$. For each $\phi \in \Phi_A$, $\| \phi \| \leq 1$, so the Gelfand space is contained in the unit ball of $A^*$, which is compact, by the Banach-Alaoglu theorem. Thus we need only verify that $\Phi_A$ is weak $*$ closed in $A^*$. Let $\Lambda$ be in the weak $*$ closure of $\Phi_A$. Fix $M,N \in A$, $\varepsilon > 0$. Consider
    %
    \[ W = \{ \psi \in A^* : | \psi - \Lambda | (z) < \varepsilon\ \text{for}\ z \in \{ 1, M, N, MN \} \} \]
    %
    Then $W$ is a weak $*$ neighbourhood of $\Lambda$, and contains some $\phi \in \Phi_A$. Thus
    %
    \[ |\phi(1) - \Lambda(1)| = | 1 - \Lambda(1) | < \varepsilon \]
    %
    so that $\Lambda(1) = 1$, since $\varepsilon$ was arbitrary. Furthermore,
    %
    \begin{align*}
        \Lambda(MN) -   \Lambda(M) \Lambda(N) &= [\Lambda(MN) - \phi(MN)] + [\phi(M)\phi(N) - \Lambda(M)\Lambda(N)]\\
        &= [\Lambda(MN) - \phi(MN)] + [\phi(N) - \Lambda(N)] \phi(M)\\
        &\ \ \ + [\phi(M) - \Lambda(M)] \Lambda(N)
    \end{align*}
    %
    Hence, since $\phi$ is a complex homomorphism,
    %
    \[ | \Lambda(MN) -   \Lambda(M) \Lambda(N) | < (1 + |\phi(M)| + |\Lambda(N)|) \varepsilon \leq (1 + \| M \| + \| \Lambda \| \| N \|) \varepsilon \]
    %
    Letting $\varepsilon \to 0$, we find $\Lambda(MN) = \Lambda(M) \Lambda(N)$, so $\Lambda \in \Phi_A$, and thus $\Phi_A$ is compact.
\end{proof}

Before we elaborate more on the Gelfand space, let us discuss an important property of semisimple algebras.

\begin{prop}
    Every homomorphism $T: A \to B$ into a semisimple commutative Banach algebra $B$ is continuous.
\end{prop}
\begin{proof}
    Suppose $M_i \to M$, $TM_i \to N$. By the closed graph theorem, we now need only verify that $TM = N$. If $\phi: B \to \mathbf{C}$ is a character of $B$, then $\phi \circ T$ is continuous, and so $\phi(TM) = \phi(N)$. But this implies that $TM - N$ is in the kernel of every character, implying (by the semisimplicity) that $TM = N$. This proves continuity.
\end{proof}

Thus every isomorphism between semisimple Banach algebras is a homeomorphism. The topology of a semisimple algebra is therefore completely determined by its algebraic structure.

\begin{theorem}
    If $A$ is a unital commutative Banach algebra generated by the identity and a single element $M$, then $\Phi_A$ is homeomorphic to $\sigma(M)$.
\end{theorem}
\begin{proof}
    If $\phi: A \to \mathbf{C}$ is a character, then $\phi(M) \in \sigma(M)$, which gives us a map from $\Phi_A$ to $\sigma(M)$. If $\phi(M) = \psi(M)$, then $\phi = \psi$, since thus must agree on all polynomials in $M$, and hence on $A$ by continuity. Thus the map from $\Phi_A$ to $\sigma(x)$ is bijective. The map is continuous, for it is just $\Gamma(M)$. It is then a homeomorphism since $\Phi_A$ is compact.
\end{proof}

Since $\Gamma(M)$ induces the `identity' map on the Gelfand space representation $\sigma(M)$, $\Gamma(\sum a_i M^i) = \sum a_i z^i$. This implies that $\Phi_A$ contains all complex polynomials, and thus $\Gamma(N)$ is the uniform limit of polynomials in $M$, and therefore holomorphic (if $\Phi_A$ is open). Similarily results can be obtained if $A$ is generated by $n$ elements $M_1, \dots, M_n$, in which case the Gelfand space can be represented as a subset of $\mathbf{C}^n$, and $\Gamma(N)$ is the uniform limit of the images of polynomials in the $M_i$.

\begin{example}
    Consider $\delta^1 \in L^1(\mathbf{Z})$. Then $\sigma(\delta^1) = \mathbf{T}$. We have $\sigma(\delta^1) \subset \mathbf{D}$ because $\| \delta^1 \| = 1$. $\delta^1$ has an inverse $\delta^{-1}$, and since $\| \delta^{-1} \| = 1$, $\sigma(\delta^1)^{-1} = \sigma(\delta^{-1}) \subset \mathbf{D}$ (That the inverse of the spectrum is the spectrum of the inverse can be proved easily as an algebraic exercise, but we will soon see it is a special case of a much more general method of mapping spectra), so we conclude $\sigma(\delta^1) \subset \mathbf{T}$. In fact, we have equality. To see this, let us try and invert $\lambda \delta - \delta^1$, for some $|\lambda| = 1$. If $c * (\lambda \delta - \delta^1) = \delta$, then
    %
    \[ \lambda c_n - c_{n-1} = \begin{cases} 1 & n = 0 \\ 0 & n \neq 0 \end{cases} \]
    %
    Solving these equations recursively, we find $c_n = \lambda^{-n} c_0$ for $n \geq 0$, and $c_{-n} = \lambda^{n-1} c_{-1}$. We must have $c \in L^1(\mathbf{Z})$, so that
    %
    \[ |c_0| \sum_{k = 0}^\infty 1 + |c_{-1}| \sum_{k = 1}^\infty 1 < \infty \]
    %
    Hence $c_0 = c_{-1} = 0$, which is impossible if we desire invertibility to hold. Since $\delta^1$ generates $L^1(\mathbf{Z})$, we find that the Gelfand space of $L^1(\mathbf{Z})$ is homeomorphic to $\mathbf{T}$.

    It is interesting to note that if $c \in L^1(\mathbf{Z})$, then $\Gamma(c) = \sum c_n z^n$, so the Gelfand transform is essentially a representation of the Fourier series transform between $\mathbf{Z}$ and $\mathbf{T}$. This follows because $\Gamma(\delta^1)$ induces the identity on $\mathbf{T}$, and the rest follows from linearity.
\end{example}

The key to Gelfand theory is noticing that characters of Banach algebras naturally reflect the structure of the Banach algebra in question. Understanding the character space leads to a natural understanding of the invertibility of the elements of the algebra.

\begin{example}
    Let
    %
    \[ f(z) = \sum a_n z^n \]
    %
    where $\sum |a_n| < \infty$. Then $f$ is the Gelfand transform of the sequence $a = (a_k)$ in $L^1(\mathbf{Z})$. If $f$ never vanishes, then $\sigma(a) = f(\mathbf{T})$ does not contain zero, so $a$ is invertible, and therefore there is a sequence $b \in L^1(\mathbf{Z})$ such that $a * b = \delta$. Then $\Gamma(b) = 1/f$, and so we have shown that
    %
    \[ (1/f)(z) \sim \sum b_n z^n \]
    %
    where $\sum |b_n| < \infty$. Before Gelfand theory, this was an incredibly difficult theorem of Fourier analysis. Similar results can be applied to non-continous functions by approximation, and to higher dimensional trigonometric series, in which case the Gelfand space are the higher dimensional torii $\mathbf{T}^n$.
\end{example}

Wiener had a long, technical and messy proof of the lemma long before Banach algebras were invented. The Gelfand theory reduces the theorem to a few statements. Here's another example from complex analysis.

\begin{example}
    Consider the commutative algebra $A(\mathbf{D})$. We also contend that $\Phi_{A(\mathbf{D})}$ are just the evaluations of functions at a point in the unit disk. The identity function $z$ has norm 1 in this space, implying that if $\phi$ is any homomorphism, then $\phi(z) = w \in \mathbf{D}$. Then
    %
    \[ \phi(\sum_{k = 0}^N a_k z^k) = \sum_{k = 0}^N a_k w^k \]
    %
    and these polynomials are dense in $A(\mathbf{D})$, so $\phi = \phi_w$.

    Now let $f_1, \dots, f_n \in A(\mathbf{D})$ be such that for each $z \in \mathbf{D}$, some $f_i(z)$ is non-zero. Then we claim there are $g_i \in A(\mathbf{D})$ for which $\sum f_i g_i = 1$. It is clear that this is equivalent to the ideal relation
    %
    \[ (f_1, \dots, f_n) = A(\mathbf{D}) \]
    %
    If this were not true, then $(f_1, \dots, f_n)$ would be contained in a closed, maximal ideal, and would therefore be annihilated by some $\phi \in \Phi_{A(\mathbf{D})}$. But $\phi$ corresponds to evaluation at some $w \in \mathbf{D}$, so $f_i(w) = 0$ for all $i$, a contradiction which shows $(f_1, \dots, f_n) = A(\mathbf{D})$. One can apply Runge's theorem to classify the character spaces of arbitrary $A(C)$, but we leave this to the reader.
\end{example}

\subsection{The Non-Unital Gelfand Space}

Let us also address the non-unital case of Gelfand theory. In the Gelfand theory, we have a little trouble defining ideals. We call an ideal $\mathfrak{a}$ (an additive subgroup closed under multiplication) for a non-unital algebra $A$ {\bf modular} if there is $N \in A$ such that $MN - M, NM - M \in \mathfrak{a}$ for all $M \in A$. If $\mathfrak{a}$ is modular, it follows that $A/\mathfrak{a}$ contains an identity. Maximal modular ideals exist with Zorn's help. We need to edit the proof which shows a maximal ideal is closed, which relies on an algebraic trick.

\begin{lemma}
    A maximal modular ideal in an algebra $A$ is closed.
\end{lemma}
\begin{proof}
    If a maximal ideal $\mathfrak{a}$ was not closed, then we would have $\overline{\mathfrak{a}} = A$. Let $N$ be a right modular identity for $\mathfrak{a}$. Then there is $M \in \mathfrak{a}$ with $\| M - N \| < 1$, so
    %
    \begin{align*}
        N &= (N - M) + M = \sum_{k = 1}^\infty (N - M)^k - \sum_{k = 2}^\infty (N - M)^k + M\\
        &= \sum_{k = 1}^\infty (N - M)^k - \left[ \sum_{k = 1}^\infty (N - M)^k \right] (N - M) + M\\
        &= \underbrace{\left[ \left( \sum_{k = 1}^\infty (N - M)^k \right) M + M \right]}_{\in \mathfrak{a}} - \underbrace{\left[ \left( \sum_{k = 1}^\infty (N - M)^k \right) N - \left( \sum_{k = 1}^\infty (N - M)^k \right) \right]}_{\in \mathfrak{a}}
    \end{align*}
    %
    Given any $L \in A$, $LM - M \in \mathfrak{a}$, implying $M \in \mathfrak{a}$. This implies $\mathfrak{a} = A$, an impossibility.
\end{proof}

\begin{lemma}
    The kernel of any {\it nonzero} algebra homomorphism from $A$ to $\mathbf{C}$ is a maximal modular ideal, and in a commutative non-unital algebra, any maximal modular ideal is the kernel of some non-zero algebra homomorphism.
\end{lemma}
\begin{proof}
    If $\phi: A \to \mathbf{C}$ is a non-zero algebra homomorphism, then $\phi$ is surjective, for if $\phi(M) \neq 0$, then $\phi(\mathbf{C} \cdot x) = \mathbf{C}$. We claim that if $\mathfrak{a} = \ker(\phi)$, and if $\phi(M) = 1$, then $M$ is a modular identify for $\mathfrak{a}$, because
    %
    \[ \phi(MN - N) = \phi(M)\phi(N) - \phi(N) = \phi(N) - \phi(N) = 0 \]
    %
    Implying $MN - N \in \mathfrak{a}$ for each $N$, so $\mathfrak{a}$ is a modular ideal. The first isomorphism theorem gives us an isomorphism from $A/\mathfrak{a}$ to $\mathbf{C}$, so $\mathfrak{a}$ is maximal, since $A/\mathfrak{a}$ is a field.

    If $A$ is commutative, and $\mathfrak{a}$ is a maximal modular ideal, then $A/\mathfrak{a}$ is an algebra with identity, whose only ideals consists of $(0)$ and $A/\mathfrak{a}$, since $\mathfrak{a}$ is maximal, so $A/\mathfrak{a}$ is a field, and is therefore isometric to $\mathbf{C}$, giving us an algebra homomorphism.
\end{proof}

We may therefore consider the Gelfand transform of a non-unital algebra, into the character space of all nonzero non-unital algebra homomorphisms.

\begin{lemma}
    If $A$ is an algebra without identity, and $\phi$ is a character, then there is a unique non-zero algebra functional $\tilde{\phi}$ defined on $A \ltimes \mathbf{C}$ which extends $\phi$.
\end{lemma}
\begin{proof}
    Define $\tilde{\phi}(M + \lambda) = \phi(M) + \lambda$. Then $\tilde{\phi}$ is linear, $\tilde{\phi}(1) = 1$, and
    %
    \begin{align*}
        \tilde{\phi}(M + \lambda) \tilde{\phi}(N + \gamma) &= [\phi(M) + \lambda][\phi(N) + \gamma]\\
        &= \phi(MN) + \phi(\lambda N) + \phi(\gamma M) + \lambda \gamma\\
        &= \tilde{\phi}((M + \lambda)(N + \gamma))
    \end{align*}
    %
    The uniqueness of the extension follows from the fact any maximal modular ideal of $A$ can be uniquely extended to a maximal ideal on $A \ltimes \mathbf{C}$, for the projection $M + \lambda \mapsto M$ maps ideals to ideals, and the only ideals of $\mathbf{C}$ are $(0)$ and $(1)$.
\end{proof}

\begin{corollary}
    The Gelfand space extension $\Phi_{A \ltimes \mathbf{C}} = \Phi_A \cup \{ \phi_\infty \}$, where
    %
    \[ \phi_\infty(M + \lambda) = \lambda \]
    %
    and $\sigma(M) \subset \widehat{M}(\Phi_A) \cup \{ 0 \}$, with equality in the commutative case.
\end{corollary}

\begin{example}
    The unitization of $C_0(X)$, where $X$ is locally compact, is equivalent to an embedding in $C(X^*)$. The maximal ideals of $C_0(X)$ correspond to the points in $X$, as well as the `infinity point', which is the one-point added in the compactification.
\end{example}

We may still apply a topology to $\Phi_A$, but it is no longer compact.

\begin{theorem}
    The Gelfand topology on $\Phi_A$ is locally compact when $A$ does not possess an identity, and the Gelfand transform maps $A$ into $C_0(\Phi_A)$.
\end{theorem}
\begin{proof}is the map
    Let $\psi \in \Phi_A$ be arbitrary, and fix $M \in A$ such that $\psi(M) = 1$. We shall verify that
    %
    \[ K = \{ \psi \in \Phi_A : |\psi(M)| \geq 1/2 \} \]
    %
    is compact. We need only verify if it is closed, for if $\psi \in \Phi_A$, and if $\tilde{\psi} \in \Phi_{A \ltimes \mathbf{C}}$ is the unique extension, then $\| \tilde{\psi} \| \leq 1$, so $\| \psi \| \leq 1$. Thus $K$ is contained in the closed unit ball of $A^*$. Since $K = \Lambda^{-1}([1/2, \infty))$, where $\Lambda: A^* \to \mathbf{C}$ is the map $\phi \mapsto |\phi(M)|$, then $\Lambda$ is weak $*$ continuous, so $K$ is closed, hence compact. $K$ contains an open neighbourhood
    %
    \[ U = \{ \psi \in \Phi_A : |\psi(M)| > 1/2 \} \]
    %
    so $\Phi_A$ is locally compact. If $M \in A$, $\widehat{M} \in C_0(\Phi_A)$, since
    %
    \[ \{ \phi \in \Phi_A : |\phi(M)| \geq \varepsilon \} \]
    %
    is compact, and outside of this set $\widehat{M} \leq \varepsilon$.
\end{proof}



\section{The Riesz Calculus}

\subsection{Vector-valued integration}

We would like to define the integral of arbitrary measurable functions $f: \Omega \to X$, where $(\Omega, \mu)$ is a measure space, and $X$ is some Banach space, to be some element $\int f d\mu \in X$. This is not so easy as in the real case, as we have no canonical order through which we can carry summations.

We would like our definition to have properties that ordinary integrals have, namely
%
\[ \Lambda \left( \int f d\mu \right) = \int (\Lambda f) d\mu \]
%
for every $\Lambda \in X^*$. This is certainly true of other measures, since it is true of finite sums, and the integral can be fort of as some limit of a net of sums.

We shall define a function $f: \Omega \to X$ to be {\bf measurable} if $\Lambda f$ is measurable for each $\Lambda \in X^*$, where $\Lambda f = \Lambda \circ f$. This is equivalent to the fact that $(\Lambda \circ f)^{-1}(U)$ is measurable for each measurable subset $U$ of the complex plane, and therefore that $f$ is Borel measurable when $X$ has the weak topology. We shall say that a measurable $f$ is {\bf Pettis Integrable} if there is $x \in X$ for which
%
\[ \Lambda x = \int (\Lambda f) d\mu \]
%
for all $\Lambda \in X^*$. We say $x$ is the integral of $f$, and denote $x$ by $\int f d\mu$. It is clear that $x$ is unique, for if $y$ also has the property, then $x - y$ is annihilated by all linear maps, hence $x - y = 0$.

\begin{theorem}
    Suppose $(\Omega, \mu)$ is a compact measure space with a positive Borel measure, $f: \Omega \to X$ is a continuous map into a Banach space, and $\overline{co}(f(\Omega))$ is compact in $X$, then $f$ is integrable, and $\int f d\mu \in \overline{co}(f(\Omega))$.
\end{theorem}
\begin{proof}
    Assume, without loss of generality, that $\mu$ is a probability measure. Let $H = \text{co}(f(Q))$. For a finite set $L$ is a finite set of functionals, let
    %
    \[ E_L = \left\{ x \in \overline{H} : \Lambda_i(x) = \int (\Lambda_i f) d\mu \right\} \]
    %
    Then $E_L \cap E_{L'} = E_{L \cup L'}$, and each $E_L$ is compact, and therefore has the finite intersection property. If we can show that $E_L$ is non-empty for each $L$, then $\bigcap E_L$ is non-empty, and $\int f d\mu$ exists. Given $L$, fix an ordering $(\Lambda_1, \dots, \Lambda_n)$, which defines a function from $X$ to $\mathbf{R}^n$, and consider $m = (m_1, \dots, m_n)$, where
    %
    \[ m_i = \int (\Lambda_i f) d\mu \]
    %
    We shall show that $m$ is in the convex hull of $(L \circ f)(\Omega)$. If $m$ were not in the hull, then, since $\{ m \}$ is a convex set, there is a linear functional which separates $m$ and the convex hull of $(L \circ f)(\Omega)$, and since we know the structure of linear functionals on $\mathbf{R}^n$, there would be $c_1, \dots, c_n \in \mathbf{R}$ for which
    %
    \[ \sum c_i v_i < \sum c_i m_i \]
    %
    whenever $v = (v_1, \dots, v_n)$ is in the hull. But then
    %
    \[ \sum c_i [(\Lambda_if) (\omega)] < \sum c_i m_i \]
    %
    and integrating,
    %
    \[ \sum c_i m_i = \sum c_i \int (\Lambda_i f) < \sum c_i m_i \]
    %
    Hence $m$ is in the convex hull. Since $L$ is linear, $m = Ly$ for some $y \in H$. At this $y$, we have
    %
    \[ \Lambda_i y = m_i = \int (\Lambda_i f) d\mu \]
    %
    so $y \in E_L$. Thus $\int f d\mu$ exists, and it lies in $\overline{\text{co}}(f(\Omega))$.
\end{proof}

This theorem may be generalized to arbitrary complex valued measures by Jordan decomposition.

\begin{theorem}
    If $X$ is a Banach space, $K \subset X$ is compact, and $\overline{H} = \overline{\text{co}}(K)$, then for each $x \in \overline{H}$ there is a regular Borel probability measure $\mu$ for which $x = \int \text{id}_K d\mu$.
\end{theorem}
\begin{proof}
    The Riesz representation theorem identifies $C(K)^*$ with the space $R$ of all Borel probability measures on $K$. Define $\phi: C(K)^* \to X$ by $\phi(\mu) = \int \text{id}_K d\mu$. The theorem says that $\phi(R) = \overline{H}$. Certainly $K \subset \phi(R)$, for if $\delta_x$ is the unit measure at $x \in K$, then $\phi(\delta_x) = x$, for $\Lambda x = \int \Lambda d \delta_x$. Since $R$ is convex, $\phi(R)$ is convex, and we know from the last theorem that it is a subset of $\overline{H}$. Thus all we need verify is that $\phi(R)$ is closed. This is a consequence of the fact that $R$ is weak $*$ compact in $C(Q)^*$, and $\phi$ is weak $*$ continuous into the weak topology on $X$, so $\phi(R)$ is weakly closed, and all weakly closed sets are strongly closed.

    To see that $R$ is weak $*$ compact, notice that each probability measure has unit variation, and thus lies within the unit ball of $C(K)^*$, which is weak $*$ compact. For $f \in C(K)^*$, let $E_f = \{ \mu : \int f d\mu \geq 0 \}$. Then each $E_f$ is weak $*$ closed, since the map $f \mapsto f(x)$ is weak $*$ continuous. Hence $\bigcap E_f$ is weak $*$ closed. Similarily, the set $E = \{ \mu : \int 1 d\mu = 1 \}$ is weak $*$ closed, and $E \cap \bigcap E_f = R$ is therefore weak $*$ closed.

    It is sufficient to show $\phi$ is weak $*$ continuous at the origin. $X$ has a weak subbasis of neighbourhoods at the origin of the form
    %
    \[ W = \{ x \in X : |\Lambda(x)| < r \} \]
    %
    for $r > 0$, $\Lambda \in X^*$. Thus we must verify that the set of $\mu$ for which $\left| \int \Lambda d\mu \right| < r$ is weak $*$ open. But $\Lambda|_K \in C(K)$, so this set is open almost by definition. Thus $\phi$ is weak $*$ continuous.
\end{proof}

\begin{theorem}
    If $\Omega$ is a compact Hausdorff space with a positive Borel measure $\mu$, $X$ is a Banach space, and $f: \Omega \to X$ is continuous, then
    %
    \[ \left\| \int f d\mu \right\| \leq \int \| f \| d\mu \]
\end{theorem}
\begin{proof}
    For each $\Lambda \in X^*$,
    %
    \[ \left| \Lambda\left(\int f d\mu\right) \right| = \left| \int \Lambda f d\mu \right| \leq \int | \Lambda f | d\mu \leq \| \Lambda \| \int | f | d\mu \]
    %
    If $\Lambda$ is chosen (by the Hahn Banach theorem) such that $\Lambda(\int f d\mu) = \| \int f d\mu \|$, and such that $\| \Lambda \| = 1$, then we obtain the needed inequality.
\end{proof}

If $A$ is a Banach algebra, then $\int f d\mu$ has an additional property, namely, for each $M \in A$,
%
\[ M \int f d\mu = \int Mf d\mu\ \ \ \ \ \ \ \ \ \ \left(\int f d\mu\right) M = \int fM d\mu \]
%
This follows since multiplication on the left and right by $x$ is a bounded linear function. If we let $L: N \mapsto MN$ be the left multiplication operator, and fix $\Lambda \in X^*$, then
%
\[ \Lambda \left( M \left( \int f d\mu \right) \right) = (\Lambda \circ L) \left( \int f d\mu \right) = \int (\Lambda \circ L) f d\mu = \int \Lambda(Mf) d\mu \]
%
the right case is similar.





\subsection{Holomorphic Functional Calculus}

Given a polynomial $P \in \mathbf{C}[X]$, the definition of $P(M)$, for $M \in A$, is obvious -- we just overload the operator. We don't need anything topological here. In this section, we use the topological structure of Banach algebra to evaluate holomorphic functions. For instance, in matrix theory it is useful to define the exponential of $M \in A$ by
%
\[ e^M = \sum_{k = 0}^\infty \frac{M^k}{k!} \]
%
We may define $\sin$ and $\cos$ similarily. More generally, for any power series $f = \sum_{k = 0}^\infty c_k (X - \alpha)^k$ with a radius of convergence $R$, we may define
%
\[ f(M) = \sum_{k = 0}^\infty c_k (M - \alpha)^k \]
%
And this function is defined for any $M$ satisfying $\|M - \alpha \| < R$. It is non-trivial to `overload' arbitrary analytic functions, but it is very useful.

We shall use the formulation of complex analysis in terms of chains, arbitrary finite `integer combinations' of differentiable curves $\gamma = \sum n_i \gamma_i$, where we let $- \gamma_i = \gamma_i^{-1}$ in the path homotopic sense (what we are really doing is considering the free abelian group $\mathbf{Z}\langle C^\infty(\mathbf{R}, \mathbf{C}) \rangle$, modulo the relation $- \gamma = \gamma^{-1}$). The trace of a curve will be its image,
%
\[ \text{tr}(\sum n_i \gamma_i) = \bigcup_{n_i \neq 0} \text{Im}(\gamma_i) \]
%
If $f$ is a continuous function defined on $\text{tr}(\gamma)$ valued in a Banach space, and $\gamma_i$ is defined on $[a_i, b_i]$ we define the integral
%
\[ \int_\gamma f(z) dz = \sum n_i \int_{\gamma_i} f(z) dz = \sum_i n_i \int_{a_i}^{b_i} (f \circ \gamma_i) \gamma_i' \]
%
where the integrals on the right side are Pettis integrals, if they exist.

If $\gamma$ is a chain in $\mathbf{C}$, and $w \in \mathbf{C}$ is given, then we define the {\bf index} of $\gamma$ with respect to $w$, denoted $\text{Ind}_w \gamma$, to be the value
%
\[ \frac{1}{2 \pi i} \int_\gamma \frac{dz}{z - w} \]
%
A chain $\gamma$ is {\bf positively oriented} if $\text{Ind}_w \gamma \in \{ 0, 1 \}$ for each $w \in \mathbf{C}$. We then let
%
\begin{align*}
    \text{Int}(\gamma) = \{ w \in \mathbf{C} : \text{Ind}_w \gamma = 1 \} \\
    \text{Ext}(\gamma) = \{ w \in \mathbf{C} : \text{Ind}_w \gamma = 0 \}
\end{align*}
%
Each is the union of connected components in $\mathbf{C} - \text{tr}(\gamma)$.

\begin{lemma} \label{banachcauchy}
    If $D$ is open, $f: D \to X$ is a holomorphic function into a Banach space, and if $\gamma$ satisfies $\text{Ind}_z \gamma = 0$ for all $z \not\in D$, then $\int_\gamma f(z) dz = 0$.
\end{lemma}
\begin{proof}
    Take $\Lambda \in X^*$. Then $\Lambda f: D \to \mathbf{C}$ is holomorphic, and so
    %
    \[ \int \Lambda f dz = 0 \]
    %
    But this implies that $\Lambda(\int f dz) = 0$ for all $\Lambda$, hence $\int f dz = 0$.
\end{proof}

\begin{lemma}
    Let $D$ be an open set, and $K$ a compact subset. There is a positively oriented chain whose interior contains $K$, and whose exterior contains $D^c$.
\end{lemma}
\begin{proof}
    Split $D$ into its connected components. It suffices to prove the theorem assuming $D$ is a component. Let $\varepsilon = d(K,D^c)$. Then $\varepsilon > 0$, since $K$ is disjoint from $\varepsilon$. Let $G$ be a set of positively oriented squares (a chain consisting of the sum of four clockwise lines), whose interior lies in $D$, and whose vertices are elements of the product $\mathbf{Z}/n \times \mathbf{Z}/n$, where $n > \varepsilon / 2$. If we let $\gamma = \sum_{g \in G} g$, then by construction, $\text{Ind}_z \gamma = 0$ for all $z \in D^c$, and since the interiors of the squares are disjoint, the winding number around each point in $K$ is 1. By construction, we have found a required chain.
\end{proof}

We shall temporarily say that such a chain has the $(K,D)$ separation property. We can now extend holomorphic functions to arbitrary algebras, by artificially introducing the cauchy integral theorem into the algebra. Let $A$ be a Banach algebra, consider an element $M \in A$, $D$ an open set containing all of $\sigma(M)$, and $f:D \to A$ a holomorphic function. Let $\gamma$ be a positively oriented curve satisfying the $(\sigma(M), D)$ separation property, with $\sigma(M) \subset \text{Int}(\gamma)$. Define
%
\[ f(M) = \int_\gamma f(z) (z - M)^{-1} dz \]
%
Then the extension of $f$ is a well-defined holomorphic function. Of course, we must verify the map is well-defined.

\begin{lemma}
    Let $D$ be an open set. If $f: D \to \mathbf{C}$ is holomorphic, and $\gamma, \lambda$ be two chains in $D$ satisfying the $(\sigma(M), D)$ separation property. When $f: D \to A$ is holomorphic, then
    %
    \[ \int_\gamma f(z) (z - M)^{-1} dz = \int_\lambda f(z) (z - M)^{-1} dz \]
\end{lemma}
\begin{proof}
    Then $\gamma - \lambda$ has winding number zero around any point in $\sigma(M)$ and $D^c$, and we have already verified this case in lemma \ref{banachcauchy}.
\end{proof}

We have laid the foundations for the holomorphic functional calculus.

\begin{theorem}
    The map $f \mapsto f(M)$ is an algebra homomorphism from $\mathcal{O}(D)$ into $A$, which is really just evaluation for polynomials, and is continuous in the locally uniform topology.
\end{theorem}
\begin{proof}
    It can easily be verified (plagiarized from the proof in the real case) that uniform convergence preserves convergence of integration. The evaluation map is obviously linear, so we need only show the polynomial evaluation property for monomials $P = X^n$. Select a $(\sigma(M), D)$ chain $\gamma$ far away enough from the origin that we may express inverses in Neumann series. Then
    %
    \begin{align*}
        P(M) &= \frac{1}{2 \pi i} \int_\gamma z^n (z - M)^{-1} dz = \frac{1}{2 \pi i} \int_\gamma z^{n-1} \sum_{k = 0}^\infty \frac{M^k}{z^k} dz\\
        &= \frac{1}{2 \pi i} \sum_{k = 0}^\infty M^k \int_\gamma z^{n - 1 - k} dz = M^n
    \end{align*}
    %
    since $\int_\gamma z^{n-1-k} dz \neq 0$ only when $n - 1 - k = -1$ ($n = k$).

    Now we need to show the multiplicity of the evaluation map. This is done by brute calculation. Let $f,g \in \mathcal{O}(D)$ be given. Pick $\gamma$ be as required, and consider another $\tilde{\gamma}$ satisfying $\text{Int}\ \tilde{\gamma} \supset \overline{\text{Int}\ \gamma}$, $\mathbf{C} - D \subset \text{ext}\ \tilde{\gamma}$. Then
    %
    \begin{align*}
        f(M) g(M) &= \frac{-1}{4 \pi^2} \left( \int_\gamma f(z) (z - M)^{-1} dz \right) \left( \int_{\tilde{\gamma}} g(w) (w - M)^{-1} \right)\\
        &= \frac{-1}{4 \pi^2} \int_\gamma \int_{\tilde{\gamma}} f(z) g(w) (z - M)^{-1} (w - M)^{-1} dw\ dz\\
        &= \frac{-1}{4 \pi^2} \int_\gamma \int_{\tilde{\gamma}} f(z) g(w) \left( \frac{1}{w - z} \left((z - M)^{-1} - (w - M)^{-1} \right) \right) dw\ dz\\
        &= \frac{-1}{4 \pi^2} \int_\gamma f(z) \left( \int_{\tilde{\gamma}} \frac{g(w)}{w - z} dw \right) (z - M)^{-1} dz\\
        &\ \ \ + \frac{1}{4 \pi^2} \int_{\tilde{\gamma}} g(w) \left( \int_\gamma \frac{f(z)}{w - z} dz \right) (w - M)^{-1} dw\\
        &= \frac{1}{2 \pi i} \int_\gamma f(z) \left( \frac{1}{2 \pi i} \int_{\tilde{\gamma}} \frac{g(w)}{w - z} dw \right) (z - M)^{-1} dz\\
        &= \frac{1}{2 \pi i} \int_\gamma f(z) g(z) (z - M)^{-1} dz\\
        &= (fg)(M)
    \end{align*}
    %
    and thus we have an algebra homomorphism.
\end{proof}

In fact, we will show that this homomorphism is the unique one satisfying the properties. This would be trivial if $D = \mathbf{C}$, since we could expand all functions as limits of polynomials via their power series. To prove the theorem in general, we need an advanced theorem in complex analysis, Runge's theorem, which states that any function holomorphic in an open set $D$ is locally uniformly approximated by rational functions whose poles lay outside of $D$. Now if $r_i$ is rational, and $\Lambda$ is a homomorphism such that $\Lambda P = P(M)$, then
%
\[ \Lambda (P/Q) = \Lambda(P) \Lambda(1/Q) = \Lambda(P) \Lambda(Q)^{-1} = P(M) Q(M)^{-1} = (P/Q)(M) \]
%
The homorphism is then uniquely defined, since the $r_i$ are dense in the set of all holomorphic functions.

\begin{example}
    Let $K$ be compact, and let $f \in C(K)$. Let $D$ be an open neighbourhood of $\sigma(f) = f(K)$. Consider the map from $\mathcal{O}(D) \to C(K)$ mapping $g$ to $g \circ f$. This map satisfies the properties of the holomorphic functional calculus, so it really is the holomorphic calculus in disguise.
\end{example}

\begin{theorem}
    Let $f:A \to B$ be a homomorphism between two Banach algebras, $M \in A$. Then $\sigma(f(M)) \subset \sigma(M)$, and if $g$ is holomorphic in a neighbourhood of $\sigma(M)$, then $(g \circ f)(M) = (f \circ g)(M)$.
\end{theorem}
\begin{proof}
    That $\sigma(f(M)) \subset \sigma(M)$ is trivial, for if $\alpha - M$ is invertible, then $f(\alpha - M) = \alpha - f(M)$ is invertible. If $g$ is a polynomial, then the theorem follows from the multiplicative property. But then the theorem is true for limits of polynomials, and hence for all $g$.
\end{proof}

\begin{corollary}
    If $D$ is a neighbourhood of $\sigma(M)$, and $f \in \mathcal{O}(D)$, then
    %
    \[ \widehat{f(M)} = f \circ \widehat{M} \]
\end{corollary}

The ultimatum of the holomorphic functional calculus is the proof of the spectral mapping theorem, that asserts that the `operation' of the spectrum commutes with holomorphic functions. To extend this theorem to non-commutative algebras, we need an algebraic trick. We introduce the center $Z(A)$ of an algebra, which is defined exactly how it is defined in all other algebraic structures. It is a closed subalgebra of $A$, for if $M_i \to M$, and each $M_i \in A$, then
%
\[ MN = \lim M_iN = \lim NM_i = NM \]
%
If $S$ is a subset of $A$, then we may consider
%
\[ Z(S) = \{ M \in A : (\forall N \in S: MN = NM) \} \]
%
which is a commutative subalgebra of $A$. For any subset $S$, the double centralizer $Z(Z(S))$ also contains $S$. $Z(Z(S))$ is commutative if $S \subset Z(S)$, for then $Z(Z(S)) \subset Z(S)$, and thus commutes with itself by definition.

Now consider the commutative subalgebra $B = Z(Z(\{ M \}))$ of an algbera $A$. If $\lambda - M$ is invertible in $A$, there is $N = (\lambda - M)^{-1}$. But then if $K \in Z(\{ M \})$, then $K$ commutes with $\lambda - M$, and
%
\[ KN = N(\lambda - M) K N = NK (\lambda - M) N = NK \]
%
which implies that $\lambda - M$ is invertible in $Z(Z(\{M\}))$, so $\sigma_A(M) = \sigma_B(M)$.

\begin{theorem}[Spectral Mapping Theorem]
    If $\sigma(M) \subset D$, with $D$ open, and $f \in \mathcal{O}(D)$, then $\sigma(f(M)) = f(\sigma(M))$.
\end{theorem}
\begin{proof}
    First, suppose $A$ is commutative. By Gelfand theory,
    %
    \[ \sigma(f(M)) = \widehat{f(M)}(\Phi_A) = f(\widehat{M}(\Phi_A)) = f(\sigma(M)) \]
    %
    If $A$ is not commutative, consider the commutative subalgebra $B = Z(Z(\{M\}))$. If $f$ is analytic in a neighbourhood of $\sigma_A(M)$, then $f(M) \in B$, for there is surely a homomorphism from analytic functions in a neighbourhood of $\sigma_A(M)$ into $B$, and the embedding of $B$ in $A$ produces the unique homomorphism required. Thus we have justified the computation
    %
    \[ \sigma_A(f(M)) = \sigma_B(f(M)) = f(\sigma_B(M)) = f(\sigma_A(M)) \]
    %
    and the spectral mapping theorem is proved in general.
\end{proof}

The holomorphic functional calculus is incredibly useful for it allows us to apply calculations on the complex plane to arbitary algebras. Here are some immediate uses.

\begin{theorem}
    Suppose $A$ is a Banach algebra, $M \in GL(A)$, and $\sigma(M)$ does not separate the origin from $\infty$, then for each $m$, there is $N$ such that $M = N^m$, and there is $N$ such that $M = e^N$. If $\sigma(M) \subset \mathbf{R}^+$, then we may choose the $N$ such that $\sigma(N) \subset \mathbf{R}^+$.
\end{theorem}
\begin{proof}
    Choose a branch of $\sqrt[m]{z}$ holomorphic on a neighbourhood of $\sigma(M)$. If $\sigma(M) \subset \mathbf{R}^+$, choose $\sqrt[m]{z}$ to be the principal branch. Then $\sqrt[m]{M}$ is well defined, and the function $z^m$ is holomorphic everywhere, so $(\sqrt[m]{M})^m$ is defined, and
    %
    \[ \left( {\sqrt[m]{M}}\ \right)^m = (z^m \circ \sqrt[m]{z})(M) = z(M) = M \]
    %
    The spectral mapping theorem implies $\sigma(\sqrt[m]{M}) = \sqrt[m]{\sigma(M)}$, so by the choice of the branch, if $\sigma(M) \subset \mathbf{R}^+$, $\sigma(\sqrt[m]{M}) \subset \mathbf{R}^+$. We may also define $\log(M)$, and $e^{\log(M)} = M$, by similar calculation.
\end{proof}

Even for finite dimensional matrix algebras, this theorem has nontrivial repercussions.

\begin{corollary}
    If $M \in GL_n(\mathbf{C})$, then for each $m$, there is $N \in GL_n(\mathbf{C})$ such that $M = N^m$, and there is $N$ such that $M = e^N$.
\end{corollary}

\begin{example}
    Now consider the exponential function $\exp: A \to A$. If $M$ and $N$ commute, it is easy to see from the power series representation that $\exp(M + N) = \exp(M) \exp(N)$. In particular, this implies that $\exp(M)$ is invertible, and $\exp(-M) = \exp(M)^{-1}$. Thus for each $M$, the image of the map
    %
    \[ t \mapsto \exp(tM) \]
    %
    is a one-parameter subgroup of $G(A)$, which gives us a path from the identity $\exp(0)$ to $\exp(M)$. The connected component of $G(A)$ containing the identity is the {\bf principal component}, denoted $G_0(A)$. Since $N \mapsto MNM^{-1}$ is a continuous map leaving the identity fixed, it maps $G_0(A)$ to itself, so $G_0(A)$ is a normal subgroup. The quotient group $G(A)/G_0(A)$ is sometimes called the {\bf index group}.
\end{example}

\begin{theorem}
    If $P(X) = (X - \alpha_1)^{m_1} \dots (X - \alpha_s)^{m_s}$, and $P(M) = 0$, then $\sigma(M) \subset \{ \alpha_1, \dots, \alpha_n \}$, and if $\Omega$ is an open subset of the plane containing $\alpha_1, \dots, \alpha_n$, then for any $f \in H(\Omega)$, there is a polynomial $Q$ of degree less than $m_1 + \dots + m_s$, such that $f - Q = gP$, for some $g \in H(\Omega)$, so $f(M) = Q(M)$.
\end{theorem}
\begin{proof}
    Applying the spectral theorem, we find $\{ 0 \} = \sigma(P(M)) = P(\sigma(M))$, from which the spectrum of $M$ is restricted to lie in the roots of $P$. The Laurent series about each $\alpha_i$ gives constants $c_{it}$ such that
    %
    \[ g(z) = f(z)/P(z) - \sum_{i = 1}^s \sum_{k = 1}^{m_i} \frac{c_{it}}{(z - \alpha_i)^k} \]
    %
    is holomorphic in $\Omega$, and
    %
    \[ gP = f - \sum_{i = 1}^s \sum_{k = 1}^{m_i} c_{it} P (z - \alpha_i)^{-k} \]
    %
    constructing $Q$ with degree less than $n$.
\end{proof}

In non-unital algebras, we cannot evaluate any polynomial of the form $\sum a_i z^i$, since $a_0 \in \mathbf{C}$ does not necessarily have an interpretation in the algebra. If $a_0 = 0$, we can evaluate $\sum a_i z^i$, and thus any holomorphic function $f$, provided $f(0) = 0$. To verify this, one need only use the simple trick of switching from a non-unital algebra $A$ to a unital algebra $\mathbf{C} \ltimes A$.






\chapter{Involutive Banach Algebras}

An important property of the complex numbers is the existence of the conjugation operation $z \mapsto \overline{z}$, which allows us to `reverse' the orientation of the space, and can therefore be viewed as a function which enables us to study non-holomorphic functions on the space. We shall find that Banach algebras that have a `conjugation operation' have a much richer theory, analogous to `infinite dimensional' complex theory.

Define an {\bf involution} on a Banach algebra $A$ to be an antilinear map $M \mapsto M^*$, which satisfies
%
\[ M^{**} = M\ \ \ \ \ (MN)^* = N^*M^* \]
%
A Banach algebra with a fixed involution possesses a canonical way to reverse the rotational structure provided by the complex structure on the algebra. A {\bf Banach $*$ Algebra} is a Banach algebra with a fixed involution. Note that an arbitrary involution could be highly discontinuous, so we normally require a least some structure on our $*$ algebras. A {\bf $\mathbf{C}^*$ algebra} is a Banach $*$ algebra $A$ which satisfies
%
\[ \| M^* M \| = \| M \|^2 \]
%
for all $M \in A$.

\begin{example}
    Let $H$ be a Hilbert space. Then the map $T \mapsto T^*$ gives an involution in $B(H)$, where $T^*$ is defined by the equation $\langle T^* x, y \rangle = \langle x, Ty \rangle$. Now if $\| x \|, \| y \| \leq 1$, then
    %
    \[ | \langle T^* T x, y \rangle | = | \langle Tx, Ty \rangle | \leq \| T \|^2 \]
    %
    and this is attained, for if there are $\| x_i \| \leq 1$ such that $\| Tx_i \| \to \| T \|$, then
    %
    \[ \langle T^* T x_i, x_i \rangle = \| Tx_i \|^2 \to \| T \|^2 \]
    %
    so $\| T^* T \| = \| T \|^2$. If we are working over $\mathbf{C}^n$, then each element of $B(\mathbf{C}^n)$ can be expressed as a matrix in $M_n(\mathbf{C})$ If $(a_{ij})$ is such a matrix, then one verifies that $(a_{ij})^* = (\overline{a_{ji}})$, so involution here is analogous to the complex transpose on sets of matrices.
\end{example}

\begin{example}
    The map $f \mapsto \overline{f}$ is an involution in $C_0(X)$, where $X$ is a locally compact Hausdorff space, and
    %
    \[ \| f \overline{f} \|_\infty = \| |f|^2 \|_\infty = \| f \|^2_\infty \]
    %
    The central result of commutative involution theory shows that all commutative $*$ algebras are isometric to a subspace of $C_0(X)$.
\end{example}

\begin{example}
    For $f \in A(\mathbf{D})$, define $f^*(z) = \overline{f(\overline{z})}$, then $*$ is an isometric involution. If $f(z) = e^{iz}$, then
    %
    \[ (ff^*)(z) = e^{iz} \overline{e^{i\overline{z}}} = e^{iz} e^{-iz} = 1 \]
    %
    Thus $\| f f^* \|_\infty = 1$, yet $f(1) = f^*(-1) = e > 1$, so
    %
    \[ \| f \|_\infty \| f^* \|_\infty > \| ff^* \|_\infty \]
    %
    which implies $A(\mathbf{D})$ is not a $C^*$ algebra.
\end{example}

\begin{example}
    $L^1(G)$ has the involution $\overline{f}(g) = \overline{f(g^{-1})}$. It also has the involution $\overline{f}(g) = \overline{f(g)}$, but it turns out the first is much more important to the theory of groups. Neither give $L^1(G)$ the structure of a $C^*$ algebra.
\end{example}

\begin{prop}
    In a $C^*$ algebra, $\| M \| = \| M^* \|$.
\end{prop}
\begin{proof}
    Note that
    %
    \[ \| M \|^2 = \| M M^* \| \leq \| M \| \| M^* \| \]
    %
    so $\| M \| \leq \| M^* \|$. But then, by symmetry,
    %
    \[ \| M^* \| \leq \| M^{**} \| = \| M \| \]
    %
    so we have equality.
\end{proof}

For many Banach algebras, involutions will always be continuous, but $C^*$ algebras are still much nicer objects to study.

\begin{prop}
    If a Banach algebra $A$ is commutative and semisimple, then every involution on $A$ is continuous.
\end{prop}
\begin{proof}
    If $T: A \to A$ is an involution, and $\phi$ is a character on $A$, then $\overline{\phi \circ T}$ is a character, and therefore continuous. This implies that if $a_i \to a$, and $Ta_i \to b$, then $\overline{\phi(Ta)} = \overline{\phi(b)}$, hence $\phi(Ta) = \phi(b)$. Since this is true for every character $\phi$, $Ta = b$.
\end{proof}

\begin{prop}
    For any $M$ in a $C^*$ algebra,
    %
    \[ \| M \| = \sup \{ \| MN \| : \| N \| \leq 1 \} = \sup \{ \| NM \| : \| N \| \leq 1 \} \]
\end{prop}
\begin{proof}
    If $N = M^* \|M\|^{-1}$, $\| N \| = 1$, and
    %
    \[ \| MN \| = \| MM^* \| \| M \|^{-1} = \| M \| \]
    %
    Thus the suprema on the right is greater than or equal to $\| M \|$. But if $\| N \| \leq 1$, then
    %
    \[ \| MN \| \leq \| M \| \| N \| \leq \| M \| \]
    %
    so the suprema is also less than or equal to $\| M \|$.
\end{proof}

It follows that we may isometrically embed any $C^*$-algebra $A$ into $B(A)$, by either considering the left or right multiplications. This is known as the {\bf left regular representation}. Thus if $A$ is a $C^*$ algebra without identity, then we may embed $A$ in $B(A)$, and then consider the smallest closed $C^*$-subalgebra of $B(A)$ which contains $A$ and the identity. The representation theory of $C^*$ algebras is quite rich, and we will soon learn about the Gelfand Naimark Segal theore, which gives rise to a universal representation of $A$ as a subalgebra of operators on a Hilbert space.

Given a (nonunital) $C^*$ algebra $A$, there is a natural $C^*$ structure on $A^\#$, given by
%
\[ (\lambda + M)^* = \overline{\lambda} + M^* \]
%
given that we define a new norm
%
\[ \| \lambda + M \| = \sup \{ \| MN + \lambda \| : N \in A, \| N \| \leq 1 \} \]
%
The above proposition verifies that the embedding of $A$ in $A^\#$ is an isometry.

$C^*$ algebras are very closely related to operator theory. In fact, one can show that all $C^*$ algebras are isometric to some $C^*$-subalgebra of $B(H)$, where $H$ is a Hilbert space. Thus it makes sense to extend terminology from Hilbert space theory to the $C^*$ domain. We say $M \in A$ is {\bf self-adjoint} or {\bf hermitian} if $M^* = M$. The set of all such elements is denoted $A_{\text{sa}}$, which forms a real subspace of $A$. Hermitian elements are analogous to the real numbers, which form a subset of $A$.

\begin{prop}
    In a Banach $*$ algebra, any $M \in A$ can be uniquely written as $M = T + iS$, where $T$ and $S$ are self-adjoint elements.
\end{prop}
\begin{proof}
    Write
    %
    \[ T = \frac{M + M^*}{2}\ \ \ \ \ S = \frac{M - M^*}{2i} \]
    %
    Then $T + iS = M$, and they are trivially verified to be self-adjoint. Now suppose $M = N + iL$, where $N$ and $L$ are self-adjoint. Then
    %
    \[ 0 = (N - T) + i(L - S) \]
    %
    is normal. Thus
    %
    \[ (N - T) + i(L - S) = (N - T) - i(L - S) \]
    %
    And $i(L - S) = -i(L - S)$, which can only occur when $L - S = 0$. But then $N - T = 0$ also.
\end{proof}

An element $M$ is {\bf normal} if
%
\[ M^* M = MM^* \]
%
These elements are important because the smallest $C^*$ subalgebra containing $M$ will be commutative, in which case we can apply Gelfand theory to its full extent. {\bf Unitary} elements satisfy
%
\[ MM^* = M^*M = 1 \]
%
the set of all such elements forming a closed subgroup $U(A)$ of $GL(A)$. A {\bf projection} is a hermitian element $M$ such that $M^2 = M$. Thus the terminology from operator theory carries across to arbitrary $*$ algebras, along with some of the interesting properties.

\begin{prop}
    If $A$ is a $C^*$ algebra, and $M \in A$ is normal, then $\| M \| = r(M)$.
\end{prop}
\begin{proof}
    A calculation reveals
    %
    \[ \| M^2 \|^2 = \| (M^2)^* M^2 \| = \| (M^* M) (M^* M)^* \| = \| M^* M \|^2 = \| M \|^4 \]
    %
    Thus $\| M^2 \|^{1/2} = \| M \|$, and more generally by induction,
    %
    \[ \| M^{2^k} \|^{1/2^k} = \left( \| (M^{2^{k-1}})^2 \|^{1/2} \right)^{1/2^{k-1}} = \| M^{2^{k-1}} \|^{1/2^{k-1}} = \| M \| \]
    %
    which implies
    %
    \[ r(M) = \lim_{k \to \infty} \| M^k \|^{1/k} = \lim_{k \to \infty} \| M^{2^k} \|^{1/2^k} = \| M \| \]
    %
    since $\| M^{2^k} \|^{1/2^k}$ is a subsequence of $\| M^k \|^{1/k}$.
\end{proof}

\begin{corollary}
    Every $C^*$ algebra $A$ has a unique norm making it a $C^*$ algebra.
\end{corollary}
\begin{proof}
    The spectral radius $r(M)$ is invariant of any particular $M \in A$. Thus if $\| \cdot \|$ is any $C^*$ norm, then
    %
    \[ \| M \|^2 = \| M^*M \| = r(M^*M) \]
    %
    and thus the norm is uniquely defined.
\end{proof}

In a commutative $C^*$ algebra, every element is trivially normal, so

\begin{corollary}
    The Gelfand transform from a commutative $C^*$ algebra $A$ to $C(\Phi_A)$ is an isometry.
\end{corollary}
\begin{proof}
    $\widehat{M}(\Phi_A) = \sigma(M)$, so $\|\widehat{M}\|_\infty = r(M) = \| M \|$.
\end{proof}

Gelfand's main result is that the isometry is actually surjective, which means all commutative $C^*$ algebras are just algebras of continuous functions in disguise. The surjectivity is the tricky part, which requires some work.

\begin{prop}
    In a $C^*$ algebra, if $M$ is unitary, then $\sigma(M) \subset \mathbf{T}$.
\end{prop}
\begin{proof}
    Since
    %
    \[ \| M \|^2 = \| M^* M \| = \| 1 \| = 1 \]
    %
    and because $M$ is normal, $r(M) = \| M \| = 1$. But $M^{-1} = M^*$ is also unitary, so $r(M^{-1}) = 1$, and by the spectral mapping theorem,
    %
    \[ \sigma(M^{-1}) = \sigma(M)^{-1} \]
    %
    which implies that both spectra lie on $\mathbf{T}$.
\end{proof}

\begin{prop}
    If $M$ is self-adjoint, then $\sigma(M) \subset \mathbf{R}$.
\end{prop}
\begin{proof}
    The operator $e^{iM}$ is unitary, for
    %
    \[ e^{-iM} = \sum_{k = 0}^\infty \frac{(-i)^k M^k}{k!} = \sum_{k = 0}^\infty \left( \frac{(iM)^k}{k!} \right)^* = (e^{iM})^* \]
    %
    The spectral mapping theorem implies that $\sigma(e^{iM}) = e^{i \sigma(M)} \subset \mathbf{T}$, which implies $\sigma(M) \subset \mathbf{R}$.
\end{proof}

\begin{prop}
    If $A$ is a $C^*$ algebra, and $\phi$ a character on $A$, then
    %
    \begin{enumerate}
        \item[(a)] $\phi(M) \in \mathbf{R}$ if $M$ is self-adjoint.
        \item[(b)] $\phi(M^*) = \phi(M)^*$.
        \item[(c)] $\phi(M^*M) \geq 0$ for all $M \in A$.
        \item[(d)] If $M$ is unitary, then $|\phi(M)| = 1$.
    \end{enumerate}
\end{prop}
\begin{proof}
    (a) and (d) follow from Gelfand theory, since we may unitize our algebra, adding at most one extra chracter which is uniformly zero on $A$. Now for any $M$, write $M = T + iS$, and then
    %
    \[ \phi(M^*) = \phi(T - iS) = \phi(T) - i \phi(S) = \overline{\phi(T) + i \phi(S)} = \overline{\phi(M)} \]
    %
    (c) follows immediately.
\end{proof}

The next result fully characterizes commutative $C^*$ algebras.

\begin{theorem}[Gelfand-Naimark]
    Let $A$ be a commutative $C^*$ algebra. Then the Gelfand transform is a $*$-isomorphism between $A$ and $C(\Phi_A)$.
\end{theorem}
\begin{proof}
    We shall show $\hat{A}$ is dense in $C(\Phi_A)$. It is surely a subalgebra. Given $M \in A$, write $M = N + iL$, with $N$ and $L$ self adjoint. Now
    %
    \[ \widehat{M^*}(\phi) = \overline{\widehat{M}(\phi)} \]
    %
    Thus $\widehat{A}$ is a subalgebra of $C(A)$ closed under conjugation, and which separates points. The complex Stone Weirstra{\ss} tells us $\widehat{A}$ is dense in $C(\Phi_A)$, and since it is closed, we obtain that $\widehat{A} = C(\Phi_A)$.
\end{proof}

The Gelfand Naimark theorem has two important applications. Recall that space $X$ is {\bf completely regular} if it is $T1$, and if for each closed $C$ and $x \in X - C$, there is $f \in C_b(X)$ for which $f|_F = 0$, and $f(x) = 1$. Urysohn's lemma tells us all normal spaces (for instance, those which are compact and Hausdorff, or metrizable) are completely regular.

\begin{theorem}
    Let $X$ be a completely regular space. Then there is a compact Hausdorff space $\beta X$ with an embedding $i: X \to \beta X$ onto a dense subset of $\beta X$ satisfying the following universal property; $j: X \to K$ is a continuous map into a compact Hausdorf space, then there is a unique $f: \beta X \to K$ such that
    %
    \begin{center}
    \begin{tikzcd}
        X \arrow{r}{i} \arrow{d}{j} & \beta X \arrow{ld}{f} \\
        K
    \end{tikzcd}
    \end{center}
    %
    The universal property means $\beta X$ is unique up to homeomorphism.
\end{theorem}
\begin{proof}
    The map $f \mapsto \overline{f}$ is an isometric involution on $C_b(X)$. Then $C_b(X)$ is a commutative $C^*$ algebra with identity, so $\Phi_{C_b(X)}$ is compact and Hausdorff. The map
    %
    \[ i: X \to \Phi_{C_b(X)}\ \ \ \ \ \ \ \ \ \ x \mapsto \phi_x \]
    %
    is continuous and injective. Fix $\phi \in \beta X - \overline{i(X)}$, if it exists. Pick $f \in C(\Phi_A)$ such that $f|_{\overline{i(X)}} = 0$, and $f(\phi) = 1$. But since $\widehat{\cdot}$ is an isomorphism, $f = \widehat{g}$ for some $g \in C_b(X)$, so $g(x) = 0$ for all $x \in X$, so $\widehat{g} = f = 0$, a contradiction, since $f \not \in i(X)$. Thus $i(X)$ is dense in $\beta X$

    Now we show $i$ is an embedding. Let $\{ x_\alpha \}$ be a net in $X$, and suppose $i(x_\alpha) \to i(x)$. Assume $x_\alpha$ does not tend to $x$. Then there is a neighbourhood $U$ of $x$ and a subnet $\{ x_\beta \}$ of $\{ x_\alpha \}$ such that $x_\beta \not \in U$ for all $\beta$. Since $X$ is completely regular, choose $f \in C_b(X)$ such that $f|_{X - U} = 0$, and $f(x) = 1$. Then $\widehat{f}(\phi_{x_\beta})$ does not tend to $\widehat{f}(\phi_x)$ even pointwise, and, consequently by the weak topology on $\Phi_{C_b(X)}$, $i(x_\beta)$ does not tend to $i(x)$, a contradiction.

    Finally, let $K$ be any compact Hausdorff space, and let $j: X \to K$ be continuous. Define
    %
    \[ j^*: C(K) \to C_b(X)\ \ \ \ \ f \mapsto f \circ j\ \ \ \ \ \ \ \ \ \ \widehat{j}: \Phi_{C_b(X)} \to \Phi_{C(K)} \cong K\ \ \ \ \ \phi \mapsto \phi \circ j^* \]
    %
    then $\widehat{j}$ is continuous, and for any $f \in C(K)$,
    %
    \[ [(\widehat{j} \circ i)(x)](f) = [\phi_x \circ j^*](f) = f(j(x)) \]
    %
    Thus if $\Gamma$ is the Gelfand transform from $\Phi_{C(K)}$ to $K$, then
    %
    \[ \Gamma^{-1}((\widehat{j} \circ i)(x)) = j(x) \]
    %
    so $g \circ \widehat{j} \circ i = j$, thus $g \circ \widehat{j}$ is a map making the diagram commute, and it must be unique, since $i(X)$ is dense in $\Phi_{C_b(X)}$.
\end{proof}

\begin{theorem}
    If  $B$ is a $C^*$-subalgebra of $A$, then $GL(A) = GL(B)$.
\end{theorem}
\begin{proof}
    Any involution satisfies $1^* = 1$, since the involution is linear, and satisfies, for any $w \in \mathbf{C}$,
    %
    \[ w^* = (w1)^* = w^*1^* \]
    %
    If $N \in GL(A)$, then $N^* \in GL(A)$, for
    %
    \[ N^*(N^{-1})^* = (N^{-1}N)^* = 1^* = 1\ \ \ \ \ \ \ \ (N^{-1})^* N^* = (NN^{-1})^* = 1^* = 1 \]
    %
    Thus $NN^*$ is an invertible, self adjoint operator, so $\sigma_A(NN^*) \subset \mathbf{R}$, implying $\sigma_A(NN^*) = \sigma_B(NN^*)$, which implies $NN^* \in GL(B)$. But then $N$ has a right inverse in $B$. Computing with $N^*N$ instead we find $N$ has a left inverse, so $N \in GL(B)$.
\end{proof}

\begin{corollary}
    If $A$ is a $C^*$ algebra, and $B$ is a $C^*$-subalgebra, then 
    %
    \[ \sigma_A(M) = \sigma_B(M) \]
    %
    so spectral theory over $C^*$-algebras is invariant of a particular $C^*$ algebra.
\end{corollary}

For normal operators, we can extend the holomorphic functional calculus to all continuous functions. Let $A$ be a $C^*$ algebra, and let $M \in A$. We let $C^*(M)$ be the smallest unital $C^*$ algebra containing $M$. It can be described as the smallest closed subalgebra which contains $1$, $M$, and $M^*$. Certainly, the latter is contained in the former. Consider a monomial, consider of products of $M$ and $M^*$. Then the involution of a monomial is a monomial, and by linearity, the span is closed under involution, hence the closed span is also closed under involution.

\begin{theorem}[Continuous Functional Calculus]
    Let $M$ be a normal element of a $C^*$ algebra $A$. Then there is a unique isometric $C^*$-isomorphism between $C(\sigma(M))$ and $C^*(M)$ such that
    %
    \[ 1 \mapsto 1\ \ \ \ \ \text{id}_{\sigma(M)} \mapsto A\ \ \ \ \ \overline{\text{id}_{\sigma(M)}} \mapsto A^* \]
\end{theorem}
\begin{proof}
    The uniqueness assumption is clear, since $C(\sigma(M))$ is generated by the identity function by the Stone Weirstra{\ss} theorem. Since $M$ is normal, $C^*(M)$ is a commutative $C^*$ algebra, so the Gelfand transform gives us an isometric isomorphism from $C^*(M)$ to $C(\Phi_{C^*(M)})$. But the map
    %
    \[ \widehat{M}: \Phi_{C^*(M)} \to \Sigma(M) \]
    %
    is a continuous bijection (for $C^*(M)$ is commutative), which therefore must be a homeomorphism. Then,
    %
    \[ f \mapsto \widehat{\cdot}\ ^{-1}(f \circ \widehat{M}) \]
    %
    is what we desire, where $\widehat{\cdot}\ ^{-1}$ is the inverse Gelfand transform.  We write the image of $f$ under this map as $f(M)$.
\end{proof}

\begin{corollary}
    If $A$ is a $C^*$ algebra, and $M \in A$ is normal, and let $f \in C(\sigma(M))$. Then
    %
    \[ f(\sigma(M)) = \sigma(f(M)) \]
\end{corollary}
\begin{proof}
    We just apply Gelfand theory.
    %
    \[ \sigma(f(M)) = \sigma(\Gamma^{-1}(f \circ \widehat{M})) = (f \circ \widehat{M})(\Phi_{C^*(M)}) = f(\sigma(M)) \]
    %
    Thus the spectrum of normal operators is nicely defined under spectral mappings.
\end{proof}

For Banach $*$ algebras, Gelfand theory is not nearly as successful. We need not even have $\phi(M^*) = \overline{\phi(M)}$, for instance, in $A(\mathbf{D})$ where $f^*(z) = \overline{f}(\overline{z})$, or in $L^1(\mathbf{Z})$, where $(a^*)_n = \overline{a}_n$. We call an algebra with this property a {\bf Symmetric Algebra}. If we define $(a^*)_n = \overline{a_{-n}}$, then $L^1(\mathbf{Z})$ becomes a symmetric algebra, because every character is of the form $\phi(a) = \sum a_k z^k$, for $z \in \mathbf{T}$, and
%
\[ \overline{\phi(a)} = \sum \overline{a_k} z^{-k} = \sum \overline{a_{-k}} z^k = \phi(a^*) \]
%
which is one of the reaons why this involution is so much more useful than pointwise conjugation.




\section{Positivity}

\subsection{Positive Elements of Algebras}

A self-adjoint element $M$ is {\bf positive} if $\sigma(M) \subset [0, \infty)$. In shorthand, we write $M \geq 0$. The set of all positive elements in an algebra $A$ is denoted $A_+$. We write $M \leq N$ if $N - M \geq 0$. The identification of positive elements gives us a useful ordering on $A_{\text{sa}}$, analogous to the ordering of the real numbers.

\begin{example}
    If $f \in C_b(X)$, then $f$ is normal if and only if $f$ is real-valued. Similarily, $f$ is positive if and only if $f \geq 0$, in the usual sense, for this occurs if and only if $\overline{f(X)} \subset [0,\infty)$.
\end{example}

\begin{prop}
    If $M \in A_+$, there is a unique $N \in A_+$ for which $M = N^2$.
\end{prop}
\begin{proof}
    Consider the map $f(z) = \sqrt{z}$. This is well defined, since $\sigma(M) \subset [0,\infty)$, and $f \in C(\sigma(M))$, so we may consider $f(M)$. Now
    %
    \[ f(M)^2 = (f^2)(M) = (\text{id}_{\sigma(M)})(M) = M \]
    %
    If $N^2 = M$, then
    %
    \[ NM = N^3 = (N^2)N = MN \]
    %
    similarily, $(N^*)^2 = (N^2)^* = M^*$, and since $N$ is normal,
    %
    \[ NM^* = NN^*N^* = N^*N^*N = M^*N \]
    %
    so, since $f(M)$ is a limit of certain sums of monomials of $M$ and $M^*$, under which $N$ commutes,
    %
    \[ Nf(M) = f(M)N \]
    %
    Consider $B = C^*(N, f(M))$. This is a commutative sub $C^*$-algebra of $A$. Applying the Gelfand transform, for any $\phi \in \Phi_B$, $\widehat{N}(\phi) \geq 0$, since $N$ is a positive operator. Similarily, $\widehat{f(A)}(\phi) \geq 0$. Since
    %
    \[ \widehat{N}(\phi)^2 = \widehat{M}(\phi) = \widehat{f(A)}(\phi)^2 \]
    %
    we must have $\widehat{N}(\phi) = \widehat{f(A)}(\phi)$, hence $\widehat{N} = \widehat{f(A)}$, so $N = f(A)$, since the Gelfand transform is injective for commutative algebras.
\end{proof}

We denote the unique positive square root of $M \in A_+$ by $\sqrt{M}$. If self-adjoint operators behave like real numbers, positive operators behave like positive real numbers.

\begin{prop}
    Let $A$ be a $C^*$ algebra, and let $M \in A_{\text{sa}}$. Then there are unique $M_+$, $M_- \in A_+$ for which $M = M_+ - M_-$, and $M_+ M_- = 0$.
\end{prop}
\begin{proof}
    Let $f_+(t) = \text{max}(t,0)$, and $f_-(t) = -\text{min}(t,0)$. Then $f_+$ and $f_-$ are continuous maps, and we may consider $f_+(M)$, $f_-(M)$. Since
    %
    \[ f_+ - f_- = \text{id}_{\sigma(M)} \]
    %
    $f_+(M) - f_-(M) = M$. By the spectral mapping theorem,
    %
    \[ \sigma(f_+(M)) = f_+(\sigma(M)) \subset [0,\infty)\ \ \ \ \ \sigma(f_-) = f_-(\sigma(M)) \subset [0,\infty) \]
    %
    so $f_+(M), f_-(M) \geq 0$. Since $f_+ f_- = 0$, $f_+(M) f_-(M) = 0$.

    To prove uniqueness, suppose $L - U = M$, and $LU = 0$. Let $B = C^*(L,U)$. Then $M \in B$, and thus $M_+$, $M_- \in B$. For any $\phi \in \Phi_B$, either $\widehat{L}(\phi) = 0$, or $\widehat{U}(\phi) = 0$. If $\widehat{M}(\phi) \geq 0$, then $\widehat{U}(\phi) = 0$, so $\widehat{L}(\phi) = \widehat{M}(\phi)$, and if $\widehat{M}(\phi) \leq 0$, then $\widehat{L}(\phi) = 0$, so $\widehat{U}(\phi) = -\widehat{M}(\phi)$. But then $\widehat{L} = \widehat{f_+(M)}$ and $\widehat{U} = \widehat{f_-(M)}$. Then equality is obtained, since $B$ is commutative, because $L$ and $U$ commute.
\end{proof}

\begin{example}
    Suppose $M$ is a positive element of $B(H)$. Then, for any $x \in H$,
    %
    \[ \langle Mx,x \rangle = \langle \sqrt{M}\sqrt{M}x,x \rangle = \langle \sqrt{M}x, \sqrt{M}x \rangle \geq 0 \]
    %
    Conversely, let $M \in B(H)$ be an operator such that, for any $x \in H$,
    %
    \[ \langle Mx, x \rangle \geq 0 \]
    %
    Write $M = N + iL$, with $N$ and $L$ self adjoint.Then
    %
    \[ \langle Mx, x \rangle = \langle Nx, x \rangle + i \langle Lx, x \rangle \geq 0 \]
    %
    implying $\langle Lx, x \rangle = 0$ for all $x$, so $L = 0$, and $M = N$ is self-adjoint. The last proposition implies that we may write $M = M_+ - M_-$. Then
    %
    \[ 0 \leq \langle MM_-x, M_-x \rangle = - \langle M_-^2 x, M_- x \rangle \leq 0 \]
    %
    Since $H = \overline{M_-H} \oplus \ker M_-$, we find that
    %
    \[ \langle M_-x, x \rangle = 0 \]
    %
    for all $x$, implying $M_- = 0$, so $M = M_+$ is positive. Thus positive operators have a nice characterization on Hilbert spaces.
\end{example}

\begin{lemma}
    If $M$ is self-adjoint, and $\| M - t \| \leq t$, then $M$ is positive. If $M$ is positive, and $\| M \| \leq t$, then $\| M - t \| \leq t$.
\end{lemma}
\begin{proof}
    Suppose $\| M - t \| \leq t$. Then $\sigma(M - t) \in [-t,t]$, so
    %
    \[ \sigma(M) = \sigma(M - t) + t \in [0, 2t] \]
    %
    Conversely, if $M$ is positive and $\| M \| \leq t$, then $\sigma(M) \subset [0,t]$, and
    %
    \[ \sigma(M - t) = \sigma(M) - t \subset [-t,0] \]
    %
    which means $\| M - t \| = r(M - t) \leq t$.
\end{proof}

\begin{prop}
    If $A$ is a $C^*$ algebra, then
    %
    \begin{enumerate}
        \item[(a)] $A_+$ is closed.
        \item[(b)] If $M,N \in A_+$, then $M + N \in A_+$.
        \item[(c)] If $M \in A_+$, and $t \geq 0$, then $tM \in A_+$.
        \item[(d)] $A_+ \cap -A_+ = (0)$.
    \end{enumerate}
\end{prop}
\begin{proof}
    Suppose that $N \in \overline{A_+}$, choose $K > \| N \|$. Fix $\varepsilon > 0$ and pick $M \in A_+$ with $\| N - M \| < \varepsilon$. Then if $\varepsilon$ is chosen small enough, then $\| M \| < K$, and
    %
    \[ \| N - K \| = \| N - M \| + \| M - K \| \leq \varepsilon + K \]
    %
    Letting $\varepsilon \to 0$, we find $N \in A_+$. To prove (b), we apply the inequality in the lemma to conclude
    %
    \[ \| M + N - (\| M \| + \| N \| ) \| \leq \| M - \| M \| \| + \| N - \| N \| \| \leq \| M \| + \| N \| \]
    %
    So $M + N \in A_+$. Since $\sigma(\lambda M) = \lambda \sigma(M)$, we obtain (c). To prove (d), suppose $M \in A_+ \cap -A_+$. Then $\sigma(M) = \{ 0 \}$, and since $M$ is normal, $\| M \| = r(M) = 0$.
\end{proof}

The next corollary says that `$| x | \leq 0$ if and only if $x = 0$', but in a $C^*$-algebra setting.

\begin{lemma}
    If $-M^*M \in A_+$, then $M = 0$.
\end{lemma}
\begin{proof}
    Since
    %
    \[ \sigma(MM^*) \cup \{ 0 \} = \sigma(M^*M) \cup \{ 0 \} \]
    %
    which can be proved by some algebraic tricks, $-MM^* \in A_+$ as well. If we write $M = T + iS$, where $T$ and $S$ are normal, then
    %
    \[ M^*M + MM^* = (T - iS)(T + iS) + (T + iS)(T - iS) = 2(T^2 + S^2) \in A_+ \]
    %
    Thus
    %
    \[ M^*M = 2(T^2 + S^2) - MM^* \in A_+ \]
    %
    which implies $M^*M = 0$.
\end{proof}

\begin{prop}
    The following are equivalent
    %
    \begin{enumerate}
        \item[(a)] $M \in A_+$.
        \item[(b)] There is $N \in A_+$ such that $M = N^2$.
        \item[(c)] There is $N \in A$ such that $N^*N = M$.
    \end{enumerate}
\end{prop}
\begin{proof}
    We have already show (a) implies (b). The proof of (c) from (b) is trivial. To prove (a) from (c), note that if $M = N^*N$, then $M$ is certainly self-adjoint, so we may write $M = M_+ - M_-$. Let $L = NM_-$. Then
    %
    \[ - L^*L = - M_-N^*NM_- = - M_- M M_- = M_-^3 \in A_+ \]
    %
    Implying $NM_- = 0$, hence
    %
    \[ 0 = N^*L = N^*NM_- = -M_-^2 \in A_+ \]
    %
    This implies $M_- = 0$, so $M = M_+ \in A_+$.
\end{proof}

Thus we are inclined to define $|M| = \sqrt{MM^*}$, for any $M \in A$, which is a positive element.

\begin{prop}
    If $M \leq N$, then
    %
    \begin{enumerate}
        \item[(a)] $M + L \leq N + L$.
        \item[(b)] $L^*ML \leq L^*NL$.
    \end{enumerate}
    %
    If in addition, $0 \leq M \leq N$, then
    %
    \begin{enumerate}
        \item[(c)] $\| M \| \leq \| N \|$.
        \item[(d)] $\sqrt{M} \leq \sqrt{N}$.
    \end{enumerate}
    %
    and if $M, N \in GL(A)$, then
    %
    \begin{enumerate}
        \item[(e)] $0 \leq N^{-1} \leq M^{-1}$
    \end{enumerate}
\end{prop}
\begin{proof}
    (a) is trivial. To prove (b), note that
    %
    \[ L^*(N - M)L = L^*\sqrt{N - M}\sqrt{N - M}L = (\sqrt{N - M} L)^* (\sqrt{N - M} L) \in A_+ \]
    %
    Let us prove (c). If $N$ is positive, then $N \leq \| N \|$, which follows by Gelfand theory. Thus if $M$ and $N$ are positive, then
    %
    \[ 0 \leq M \leq N \leq \| N \| \]
    %
    But then $\phi(M) \leq \| N \|$ for each $\phi$, so $\| M \| \leq \| N \|$.

    Fix $\varepsilon > 0$. Write
    %
    \[ (\varepsilon + \sqrt{N} + \sqrt{M})(\varepsilon + \sqrt{N} - \sqrt{M}) = T \]
    %
    where $T,U \in A_{\text{sa}}$. By calculation
    %
    \[ T = \varepsilon^2 + 2 \varepsilon \sqrt{N} + N - M \geq \varepsilon^2 \]
    %
    Thus $T$ is positive and invertible, so $\varepsilon + \sqrt{N} - \sqrt{M}$ is left invertible and therefore invertible. Thus $\varepsilon \not \in \sigma(\sqrt{M} - \sqrt{N})$, and thus
    %
    \[ \sigma(\sqrt{M} - \sqrt{N}) \subset (-\infty, 0) \]
    %
    But this implies $\sqrt{N} - \sqrt{M}$ is positive.

    If $M \geq \varepsilon$, then $M^{-1} \leq 1/\varepsilon$, which follows by Gelfand theory, since $\phi(M^{-1}) = \phi(M)^{-1}$. Since
    %
    \[ 1 = \sqrt{M}^{-1}M\sqrt{M}^{-1} \leq \sqrt{M}^{-1} N \sqrt{M}^{-1} \]
    %
    this yields
    %
    \[ \sqrt{M} N^{-1} \sqrt{M} \leq 1 \]
    %
    and by conjugation again,
    %
    \[ N^{-1} = \sqrt{M}^{-1} \sqrt{M} N^{-1} \sqrt{M} \sqrt{M}^{-1} \leq M^{-1} \]
\end{proof}

\subsection{Positive Approximate Units}

we show that all $C^*$ algebras have {\bf approximate units}, which are bounded approximate identities which are also increasing nets, in the sense of the positive ordering on the $C^*$ algebra.

\begin{lemma}
    If $0 \leq M \leq N$, then $M(1 + M)^{-1} \leq N(1 + N)^{-1}$.
\end{lemma}
\begin{proof}
    Note that
    %
    \[ M(1 + M)^{-1} = 1 - (1 + M)^{-1}\ \ \ \ \ N(1 + N)^{-1} = 1 - (1 + N)^{-1} \]
    %
    We know $1 + M \leq 1 + N$, so $(1 + N)^{-1} \leq (1 + M)^{-1}$, so
    %
    \[ N(1 + N)^{-1} = 1-(1 + N)^{-1} \leq 1-(1 + M)^{-1} = M(1 + M)^{-1} \]
    %
    which is exactly the inequality we needed.
\end{proof}

\begin{prop}
    For a $C^*$ algebra $A$, the net
    %
    \[ \{ M \in A_+ : \| M \| < 1 \} \]
    %
    ordered by the positivity of the algebra, is an approximate unit.
\end{prop}
\begin{proof}
    The set is a net, because any $M,N \in A_+$ are less than or equal to $\max(\|M\|,\|N\|)$. Fix $M \in A_{\text{sa}}$, and let $\varepsilon > 0$. Let $B$ by the (non-unital) $C^*$ algebra generated by $M$. Then $B$ is commutative, and $\widetilde{M} \in C_0(\Phi_B)$. Let $g \in C_0(\Phi_B)$ be a function such that $0 \leq g < 1$, such that $\| \widetilde{M} - g \widetilde{M} \| < \varepsilon^2$, and set $N$ to be the unique element of $B$ such that $\widetilde{N} = g$. Then $\| N \| \leq 1$, $N$ is positive, and $\| M - MN \| < \varepsilon^2$. Let $N \leq L < 1$. Then, since $\sigma(\sqrt{1 - L}) \in [0,1]$, $\| \sqrt{1 - L} \| < 1$, and
    %
    \begin{align*}
        \| M - LM \|^2 &= \| \sqrt{1 - L}^2 M \|^2 \leq \| \sqrt{1 - L}\ M \|^2 = \| M(1 - L)M \|\\
        &\leq \| M(1 - N)M \| \leq \| M \| \| (1 - N)M \| < \| M \| \varepsilon^2
    \end{align*}
    %
    Decreasing $\varepsilon$, we find $\lim LM = M$. Similar results show $\lim ML = M$.
\end{proof}

If $A$ is an abelian $C^*$ algebra, then $A = A_{\text{sa}}$ forms a real Riesz lattice, which can be easily seen from an isometric representation $A \cong C(X)$. Thus if $0 \leq M \leq N + L$, where $N,L \in A_+$, then there are $N',L'$ such that $N' \leq N$, $L' \leq L$ such that $M = N' + L'$. This theorem does not hold in general nonabelian $C^*$ algebras, but there is a weaker decomposition theorem which does hold.

\begin{prop}
    If $A$ is a $C^*$ algebra, and
    %
    \[ \sum_{i = 0}^m M_i^* M_i = \sum_{j = 0}^n N_i^* N_i \]
    %
    then there are $L_{i,j}$ such that
    %
    \[ M_i^* M_i = \sum_j L_{i,j}^* L_{i,j}\ \ \ \ \ \ N_j^* N_j = \sum_i L_{i,j}^* L_{i,j} \]
\end{prop}
\begin{proof}
    We may assume $A$ is unital. Let $X = \sum M_i^* M_i = \sum N_i^* N_i$. Write
    %
    \[ L_{i,j,t} = s \]
    %
    and so on and so forth FINISH THIS LATER.
\end{proof}







\section{Ideals and Quotients of C Star Algebras}

\subsection{C Star Operations on Quotients}

If we quotient a $C^*$ algebra by a closed ideal, we certainly get a Banach algebra back. But does this algebra still have an involution, or do we need to apply additional structure to our ideals? Such discussions naturally lead to the Gelfand Naimark construction, establishing that every $C^*$ algebra is isometric to some subalgebra of bounded operators on a Hilbert space. Thus such an endeavor is very fruitful to discuss.

\begin{prop}
    If $\mathfrak{a}$ is a closed left/right ideal in a $C^*$ algebra $A$ containing a self-adjoint element $M$, and if $f \in C(\sigma(M))$ satisfies $f(0) = 0$, then $f(M) \in \mathfrak{a}$.
\end{prop}
\begin{proof}
    Choose a sequence $P_i$ of polynomials which uniformly approximating $f$, with $P_i(0) = 0$ for all $i$. If we write
    %
    \[ P_i = \sum a_{ij} X^i \overline{X}^j \]
    %
    then $P_i(M) = \sum a_{ij} X^{i+j} \in \mathfrak{a}$, and since $\mathfrak{a}$ is closed, we find $f(M) \in \mathfrak{a}$.
\end{proof}

\begin{corollary}
    If $M \in A_{\text{sa}}$ is in $\mathfrak{a}$, then $M_+$, $M_-$, $|M|$, and $\sqrt{M}$ are all in $\mathfrak{a}$.
\end{corollary}

If an ideal is closed under involution, then it certainly must be a two sided ideal, for involution reverses multiplication. We shall show that all two-sided ideals are closed under involution.

\begin{prop}
    A closed ideal in a $C^*$-algebra is closed under involution.
\end{prop}
\begin{proof}
    Given a closed ideal $\mathfrak{a}$ of a $C^*$ algebra $A$, $\mathfrak{a}$ is a (non-unital) $C^*$ subalgebra of $A$. Let $\{ E_\alpha \}$ be a two-sided approximate identity for $\mathfrak{a}$. Then, if $M \in \mathfrak{a}$ is arbitrary,
    %
    \[ \lim M^*E_\alpha = \lim (E_\alpha M)^* = (\lim E_\alpha M)^* = M^* \]
    %
    so $M^*$ is the limit of elements of $\mathfrak{a}$, and is thus in $\mathfrak{a}$.
\end{proof}

Thus if $\mathfrak{a}$ is a closed ideal in a $C^*$-algebra $A$, then $A/\mathfrak{a}$ is verified to be a Banach $*$ algebra. It is a little bit more technical to verify that $A/\mathfrak{a}$ is a $C^*$ algebra, but we shall now carry out the details.

\begin{lemma}
    If $\mathfrak{a}$ is a closed ideal of $A$, and $\{ E_\alpha \}$ is the positive BAI for $\mathfrak{a}$, then $A/\mathfrak{a}$, then for any $M \in A$,
    %
    \[ \| M + \mathfrak{a} \| = \lim \| M - E_\alpha M \| = \lim \| M - M E_\alpha \| \]
\end{lemma}
\begin{proof}
    Fix $M \in A$, and let $\{ E_\alpha \}$ be the positive BAI for $\mathfrak{a}$. Choose $\varepsilon > 0$, and let $N \in \mathfrak{a}$ satisfy $\| M + N \| \leq \| M + \mathfrak{a} \| + \varepsilon$. Then
    %
    \begin{align*}
        \limsup_\alpha \| M - E_\alpha M \| &\leq \limsup_\alpha \| (1 - E_\alpha)(M + N) \| + \| E_\alpha N - N \|\\
        &\leq \limsup_\alpha \| (1 - E_\alpha)(M + N) \| + \limsup_\alpha \| N - E_\alpha N \|\\
        &= \limsup_\alpha \| (1 - E_\alpha) (M + N) \|\\
        &\leq \| M + N \| \leq \| M + \mathfrak{a} \| + \varepsilon
    \end{align*}
    %
    and $\varepsilon$ was arbitrary, so
    %
    \[ \limsup_\alpha \| M - E_\alpha M \| \leq \| M + \mathfrak{a} \| \]
    %
    But $E_\alpha M \in \mathfrak{a}$, so for each $\alpha$,
    %
    \[ \| M - E_\alpha M \| \geq \| M + \mathfrak{a} \| \]
    %
    and since we have bounded the $\limsup$ above and below by the same value, it is in fact a limit, and has value $\| M + \mathfrak{a} \|$. Similar results hold for $\| M - M E_\alpha \|$.
\end{proof}

The lemma makes sense, for $E_\alpha$ approximates elements of $\mathfrak{a}$ as best as possible, so subtracting $E_\alpha M$ subtracts the best approximation of $M$ in $\mathfrak{a}$, thus giving us the quotient norm.

\begin{theorem}
    If $\mathfrak{a}$ is a closed ideal of a $C^*$ algebra $A$, then $A/\mathfrak{a}$ is a $C^*$ algebra.
\end{theorem}
\begin{proof}
    Letting $\{ E_\alpha \}$ be the positive approximate identity for $\mathfrak{a}$, we find
    %
    \begin{align*}
        \| M + \mathfrak{a} \|^2 &= \lim_\alpha \| M (1 - E_\alpha) \|^2\\
        &= \lim_\alpha \| (1 - E_\alpha) M^*M (1 - E_\alpha) \|\\
        &\leq \lim_\alpha \| (1 - E_\alpha) M^*M \|\\
        &= \| M^*M + \mathfrak{a} \|\\
        &\leq \| M^* + \mathfrak{a} \| \| M + \mathfrak{a} \|
    \end{align*}
    %
    And it is easy to see that $\| M^* + \mathfrak{a} \| = \| M + \mathfrak{a} \|$, for $\mathfrak{a}$ is closed under involution, so we find
    %
    \[ \| M + \mathfrak{a} \|^2 = \| M^*M + \mathfrak{a} \| \]
    %
    and this is exactly the $C^*$ identity for the quotient algebra.
\end{proof}

It is often much easier to calculate the norm on $A/\mathfrak{a}$ by applying the theory we've created. If $\pi: A \to B$ is a $C^*$ homorphism with kernel $\mathfrak{a}$, then we obtain an induced diagram of $C^*$ morphisms
%
\begin{center}
\begin{tikzcd}
    A \arrow{r}{\pi} \arrow{d}{} & B\\
    A/\mathfrak{a} \arrow{ru}[below]{\tilde{\pi}}
\end{tikzcd}
\end{center}
%
and we know that $\tilde{\pi}$ is an isometry, since it is injective. Thus
%
\[ \| M + \mathfrak{a} \| = \| \pi(M) \| \]
%
and this is normally easier to calculate than either of the two equations we can use to calculate the quotient norm.

\begin{example}
    We know the closed ideals of $C(X)$ are of the form
    %
    \[ \mathfrak{a}_K = \{ f \in C(X) : (\forall x \in K: f(x) = 0) \} \]
    %
    for some closed set $K$. Then $C(X)/\mathfrak{a}_K$ is isometric to $C(K)$, because we have the $C^*$ homorphism $f \mapsto f|_K$, and $\mathfrak{a}_K$ is the kernel.
\end{example}

\begin{example}
    Let $H$ be a Hilbert space, and $\mathfrak{a}$ a closed ideal of $B(H)$. If $\mathfrak{a} \neq (0)$, we claim $K(H) \subset \mathfrak{a}$. For any $x,y \in H$, let $x \otimes y \in B(H)$ be the map
    %
    \[ z \mapsto \langle z, y \rangle x \]
    %
    It is clear that $K(H)$ is the closed linear hull of $\{ x \otimes y: x,y \in H \}$. If $T \in \mathfrak{a}$ and $T \neq 0$, then there are $v,w$ such that $\langle Tv, w \rangle = 1$. Then
    %
    \[ (x \otimes y) = (x \otimes w) \circ (Tv \otimes y) = (x \otimes w) \circ T \circ (v \otimes y) \in \mathfrak{a} \]
    %
    from which we obtain, since $x \otimes y$ was arbitrary, that $K(H) \subset \mathfrak{a}$.
\end{example}

\begin{example}
    In an infinite dimensional, separable Hilbert space $H$ the only closed ideals of $B(H)$ are $(0)$, $K(H)$, and $B(H)$. Let $\mathfrak{a}$ be an ideal properly containing $K(H)$. Let $T \in \mathfrak{a} - K(H)$, which is without loss of generality positive. FINISH THIS LATER WHEN BETTER AT SPECTRAL MEASURES.

    Thus in an infinite dimensional, separable Hilbert space $H$, the {\bf Calkin Algebra} $C(H) = B(H)/K(H)$ has no nontrivial closed ideals.
\end{example}

\subsection{* Homorphisms}

The natural morphism between $C^*$ algebras is a $*$ homorphism, an algebra homomorphism $T$ which satisfies $T(M^*) = T(M)^*$.

\begin{prop}
    If $T: A \to B$ is a $*$ morphism, then $\| T(M) \| \leq \| M \|$ for all $M$.
\end{prop}
\begin{proof}
    Assume first that $A$ is unital. For any $M$, $\sigma(T(M)) \subset \sigma(M)$. But then if $M$ is normal, then $T(M)$ is normal, and
    %
    \[ \| T(M) \| = r(T(M)) \leq r(M) = \| M \| \]
    %
    In general, for any $M$,
    %
    \[ \| T(M) \|^2 = \| T(M) T(M)^* \| = \| T(MM^*) \| \leq \| MM^* \| = \| M \|^2 \]
    %
    so $\| T \| \leq 1$. In the non-unital case, $T$ induces a $*$-extension between $A^\#$ and $B^\#$, in which we can apply the previous case.
\end{proof}

Thus every $*$-morphism is continuous. We shall find that injective $*$-morphisms are isometries. To prove this, we must first verify a corresponding theorem about continuous functions on a locally compact space, which says that the uniform norm on $C(X)$ is the sharpest algebra norm we can have.

\begin{lemma}
    If $\pi: A \to B$ is an injective $*$-morphism between abelian $C^*$ algebras, then the dual map $\pi^*$ is surjective.
\end{lemma}
\begin{proof}
    Let
    %
    \[ K = \{ \phi|_A : \phi \in \Phi_B \]
    %
    If $K \neq \Phi_A$, then using the continuous functional calculus, we may find non-zero $f$ and $g$ which vanish at $K$, for which $f = fg$. Then $1 \in \sigma(g)$, so there is $\phi \in \Phi_B$ such that $\phi(g) = 1$. But this is clearly impossible.
\end{proof}

\begin{lemma}
    Let $\vvvert \cdot \vvvert$ be a submultiplicative norm on $C(X)$. Then $\| \cdot \|_\infty \leq \vvvert \cdot \vvvert$.
\end{lemma}
\begin{proof}
    Let $A$ be the completion of $C(X)$ with respect to the $\vvvert \cdot \vvvert$ norm. Then $A$ is a Banach algebra, and we obtain
    %
    \[ \| f \|_\infty = r(f) \leq \vvvert f \vvvert \]
    %
    The fact that algebraic facts can hide within them topological facts is essential to the proof.
\end{proof}

Since arbitrary $C^*$ algebras contain abelian subalgebra on mass, we can almost effortlessly harvest facts about $*$ morphisms on arbitrary algebras from the above lemma.

\begin{prop}
    An injective, continuous morphism $\pi: A \to B$ from a $C^*$ algebra to a Banach algebra satisfies
    %
    \[ \| M \| \leq \| \pi \| \| \pi(M) \| \]
\end{prop}
\begin{proof}
    Given $M$, let $A_0 = C^*(M^*M)$. Define a norm on $A_0$ by letting
    %
    \[ \vvvert N \vvvert = \| \pi(N) \| \]
    %
    Then $\| \cdot \| \leq \vvvert \cdot \vvvert$, and
    %
    \[ \| M \|^2 = \| M^*M \| \leq \vvvert M^*M \vvvert = \| \pi(M^*M) \| \leq \| \pi \| \| M \| \| \pi(M) \| \]
    %
    which yields the claim.
\end{proof}

\begin{corollary}
    Let $\pi: A \to B$ be a $*$ homomorphism between $C^*$ algebras. Then $\pi(A)$ is a closed $C^*$ algebra, and $\pi$ is an isometry if it is injective.
\end{corollary}
\begin{proof}
    If $\pi$ is injective, it certainly has closed range by the last proposition. But in general, letting $\mathfrak{a} = \ker(\pi)$, $\pi$ induces an injective map
    %
    \begin{center}
    \begin{tikzcd}
        A \arrow{r}{\pi} \arrow{d}{} & B \\
        A/\mathfrak{a} \arrow{ru}{\tilde{\pi}}
    \end{tikzcd}
    \end{center}
    %
    and $\pi(A) = \tilde{\pi}(A)$ is closed. Thus if $\pi$ is injective we obtain an inverse $*$ homomorphism $\pi^{-1}: \pi(A) \to A$, and both must be contractible, hence isometries.
\end{proof}

\begin{prop}
    If $B$ is a $C^*$ subalgebra of $A$, and $\mathfrak{a}$ a closed ideal of $A$, then $B + \mathfrak{a}$ is a $C^*$ subalgebra of $A$.
\end{prop}
\begin{proof}
    Completeness is a three space property, and since $\mathfrak{a}$ is complete, we need only show that $(B + \mathfrak{a})/\mathfrak{a}$ is complete, and from this we will obtain completeness. Let $\pi: A \to A/\mathfrak{a}$ be the quotient map. But the composed $*$ morphism $\psi$ defined by
    %
    \[ B \to A \to A/\mathfrak{a} \]
    %
    tells us that the image of $B$ is closed, and
    %
    \[ \psi(B) = (B + \mathfrak{a})/\mathfrak{a} \]
    %
    so $(B + \mathfrak{a})/\mathfrak{a}$ is complete.
\end{proof}




\subsection{Positive Operators}

Every $C^*$ algebra is essentially a ring of bounded operators on a Hilbert space. The challenge is to construct a canonical Hilbert space from a $C^*$ algebra. The study of a certain subclass of operators will be essential. A {\bf positive} operator $\phi: A \to \mathbf{C}$ on a $C^*$ algebra maps $A_+$ into $[0,\infty)$. Certainly the set of positive maps form a real subspace of $A^*$.

\begin{example}
    If $\phi \in \Phi_A$, then $\phi$ is positive, for
    %
    \[ \phi(M^*M) = |\phi(M)|^2 \]
    %
    and every positive element can be written in the form $M^*M$.
\end{example}

\begin{example}
    If $X$ is locally compact and Hausdorff, and $\mu$ is a complex Borel measure on $X$, then the functional
    %
    \[ f \mapsto \int f d\mu \]
    %
    from $C_0(X)$ to $\mathbf{C}$ is positive if and only if $\mu$ is a positive measure.
\end{example}

\begin{example}
    Consider the trace map $\text{tr}: M_n(\mathbf{C}) \to \mathbf{C}$. Any positive matrix $M$ can be, by a change of basis, put into the form
    %
    \[ \begin{pmatrix} \lambda_1 & 0 & \dots & 0 \\ 0 & \lambda_2 & \dots & 0 \\ 0 & 0 & \ddots & 0 \\ 0 & 0 & \dots & \lambda_n \end{pmatrix} \]
    %
    where $\lambda_1, \dots, \lambda_n \geq 0$. The trace is invariant of a change in basis, and thus
    %
    \[ \text{tr}(M) = \sum \lambda_i \geq 0 \]
    %
    Thus the trace is positive.
\end{example}

\begin{example}
    For a fixed $x \in H$, the map $T \mapsto \langle Tx, x \rangle$ is positive.
\end{example}

One need not verify continuity for positive operators, for we obtain this automatically.

\begin{prop}
    A positive operator $\phi:A \to \mathbf{C}$ is continuous.
\end{prop}
\begin{proof}
    We claim the supremum
    %
    \[ K = \sup \{ \phi(M): M \in A_+, \| M \| \leq 1 \} \]
    %
    is finite. Otherwise, we may pick a sequence $M_1, M_2, \dots$ with $\| M_i \| \leq 1$ and $\phi(M_i) \geq 4^n$. Let
    %
    \[ M = \sum \frac{M_i}{2^i} \]
    %
    Then for each $n$, $2^n M \geq M_n$, so
    %
    \[ \phi(M) \geq \frac{\phi(M_n)}{2^n} \geq \frac{4^n}{2^n} = 2^n \]
    %
    which yields an immediate contradiction.

    Now suppose that $M$ is an arbitrary operator, choose $N$ and $L$ in $A_{\text{sa}}$ such that
    %
    \[ M = N + iL = N_+ - N_- + iL_+ - iL_- \]
    %
    Then if $\| M \| \leq 1$,
    %
    \[ |\phi(M)| = |\phi(N_+) - \phi(N_-) + i\phi(L_+) - i\phi(L_-)| \leq 4K \]
    %
    which shows that $\phi$ is continuous.
\end{proof}

\begin{theorem}
    The following are equivalent
    %
    \begin{enumerate}
        \item $\phi$ is positive operator.
        \item For every BAI $E_\alpha$ contained within the positive identity on $A$,
        %
        \[ \| \phi \| = \lim_\alpha \phi(E_\alpha) \]
        \item There is a BAI $E_\alpha$ in the positive identity on $A$ for which
        %
        \[ \| \phi \| = \lim_\alpha \phi(E_\alpha) \]
    \end{enumerate}
\end{theorem}
\begin{proof}
    Without loss of generality, assume $\| \phi \| = 1$. Suppose $\phi$ is positive. Fix an approximate identity $E_\alpha$. Then
    %
    \[ \limsup_\alpha \phi(E_\alpha) \leq 1 \]
    %
    On the other hand, when $\| M \| \leq 1$,
    %
    \[ |\phi(M)|^2 = \lim |\phi(E_\alpha M)|^2 \leq \liminf_\alpha \phi(E_\alpha^2) \phi(M^*M) \]
\end{proof}




\chapter{Von Neumann Algebras}





\section{Spectral Theorem for Normal Operators}

In finite dimensional theory, a normal linear transformation $T: \mathbf{C}^n \to \mathbf{C}^n$ can be written
%
\[ T = \sum \lambda_i P_i \]
%
where $P_i$ is projection onto the eigenspace corresponding to the eigenvalue $\lambda_i$. In this section we apply the theory of $C^*$ algebras to extend this theorem to arbitrary normal bounded transformations from a Hilbert space to itself. The trick is to `integrate' rather than sum up the subspaces.

A {\bf spectral measure} on a measure space $(\Omega, \mathcal{F})$ is a function $E: \mathcal{F} \to B(H)$, for which
%
\[ E(\Omega) = \text{id}_H \]
%
such that $E(S)$ is a projection for each $S$, and such that for a disjoint family of sets $\{ S_1, S_2, \dots \}$ in $\mathcal{F}$, then
%
\[ E \left( \bigcup S_i \right) = \sum E(S_i) \]
%
with convergence pointwise (the strong operator topology), rather than in the operator norm. It follows that $E(\emptyset) = 0$.

\begin{lemma}
    If $S \subset W$, $E(S) \leq E(W)$.
\end{lemma}
\begin{proof}
    Since $E(W - S)$ is a projection, $E(W - S) \geq 0$, so
    %
    \[ E(W) = E(W - S) + E(S) \geq E(S) \]
    %
    Hence a spectral measure is monotone on projections.
\end{proof}

\begin{lemma}
    If $S$ and $W$ are disjoint sets, then $E(S)E(W) = 0$.
\end{lemma}
\begin{proof}
    $E(S \cup W) = E(S) + E(W)$, so
    %
    \[ E(S \cup W) = E(S \cup W) E(S \cup W) = E(S) + E(W) E(S) + E(S)E(W) + E(W) \]
    %
    which implies
    %
    \[ E(S) + E(W) = E(S) + E(W) E(S) + E(S) E(W) + E(W) \]
    %
    hence $E(W) E(S) + E(S) E(W) = 0$.
\end{proof}

\begin{lemma}
    $E(S \cap W) = E(S) \circ E(W)$.
\end{lemma}
\begin{proof}
    We have
    %
    \[ E(S \cap W) + E(S - W) = E(S)\ \ \ \ \ \ \ \ E(S \cap W) + E(W - S) = E(W) \]
    %
    and
    %
    \[ E(S \cup W) = E(S \cap W) + E(S - W) + E(W - S) \]
    %
    From which it follows that
    %
    \begin{align*}
        E(S) + E(W) &= 2E(S \cap W) + E(S - W) + E(W - S)\\
        &= E(S \cap W) + E(S \cup W)
    \end{align*}
    %
    Now multiply both sides of the equation on the right by $E(W)$ gives
    %
    \[ E(S) E(W) + E(W)^2 = E(S \cap W) E(W) + E(S \cup W) E(W) \]
    %
    Since $S \subset S \cup W$, $E(S) \leq E(S \cup W)$, and
    %
    \[ E(S \cup W) E(S) = E(S) \]

    Now $E(S \cup W) E(S) = E(S)^2 + E(W - S) E(S) = E(S)$
\end{proof}

\begin{example}
    If $H = \mathbf{C}$, then $B(H) \cong \mathbf{C}$, and any $\{ 0, 1 \}$ valued measure $\mu$ is a spectral measure, provided $\mu(\Omega) = 1$. If $\mu(S_1) = \mu(S_2) = 1$, then
    %
    \[ 1 = \mu(S_1) = \mu(S_1 \cap S_2) + \mu(S_1 - S_2) \]
    %
    \[ 1 = \mu(S_2) = \mu(S_1 \cap S_2) + \mu(S_2 - S_1) \]
    %
    If $\mu(S_1 \cap S_2) = 0$, then $\mu(S_1 - S_2) = \mu(S_2 - S_1) = 1$, which implies
    %
    \[ \mu((S_1 - S_2) \cup (S_2 - S_1)) = \mu(S_1 - S_2) + \mu(S_2 - S_1) = 2 \]
    %
    an impossibility. Thus $\mu(S_1 \cap S_2) = 1$. If $\mu(S_1) = 0$, then $S_1 \cap S_2 \subset S_1$, so $\mu(S_1 \cap S_2) \leq \mu(S_1) = 0$. The same holds if $\mu(S_2) = 0$. Thus we have verified that $\mu(S_1 \cap S_2) = \mu(S_1) \circ \mu(S_2)$ on a case by case basis. The countable summation property holds by the property of the measure itself.
\end{example}

\begin{example}
    If $H$ is infinite dimensional, and $N \in K(H)$, then
    %
    \[ \sigma(N) = \{ \lambda_0, \lambda_1, \dots \} \]
    %
    where $\lambda_0 = 0$, and each $\lambda_i$ is an eigenvalue. For each $n > 0$, let $P_n$ be orthogonal projection onto $\text{ker}(\lambda_n - N)$, and let $P_0$ project onto the orthogonal complement of the closed linear space of the images of $P_1, P_2, \dots$. For $S \subset \sigma(N)$, let
    %
    \[ E(S) = \sum_{\lambda_n \in S} P_n \]
    %
    In the sense that the sum on the right is interpreted pointwise. Then $E$ is a spectral measure on $(\sigma(N), \mathcal{P}(\sigma(N)))$. Surely
    %
    \[ E(\sigma(N)) = \sum_{n = 0}^\infty P_n = \text{id}_H \]
    %
    And since $P_i \circ P_j = 0$ if $i \neq j$,
    %
    \begin{align*}
        E(S_1) \circ E(S_2) &= \left( \sum_{\lambda_n \in S_1} P_n \right) \circ  \left( \sum_{\lambda_m \in S_2} P_m \right)\\
        &= \sum_{\lambda_n \in S_1, \lambda_m \in S_2} P_n P_m\\
        &= \sum_{\lambda_n \in S_1 \cap S_2} P_n = E(S_1 \cap S_2)
    \end{align*}
    %
    And the countable additivity follows by the pointwise definition of the sum of operators.
\end{example}

\begin{example}
    If $\Omega$ is a $\sigma$-finite measure space with measure $\mu$, and $H = L^2(\Omega)$, define
    %
    \[ E(S) \xi = \chi_S \xi \]
    %
    Then $E$ is a spectral measure. Surely $E(\Omega) = \text{id}_H$, $E(\emptyset) = 0$, because $\chi_\Omega = 1$, $\chi_\emptyset = 0$. Now $E(S_1 \cap S_2) = E(S_1) \circ E(S_2)$ follows because $\chi_{S_1 \cap S_2} = \chi_{S_1} \chi_{S_2}$. Similarily, $\chi_{\bigcup S_i} = \sum \chi_{S_i}$ pointwise if the $S_i$ are disjoint.
\end{example}

\end{document}