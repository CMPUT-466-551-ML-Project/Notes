\documentclass{article}

\usepackage{amsthm}
\usepackage{amsmath}
\usepackage{multicol}
\usepackage{amssymb}
\usepackage{mathabx}
\usepackage{accents}
%\usepackage[margin=0.7in]{geometry}
\usepackage[english]{babel}
\usepackage{blindtext}

\theoremstyle{plain}
\newtheorem{lemma}{Lemma}
\newtheorem{prop}{Proposition}
\newtheorem*{example}{Example}
\newtheorem*{fact}{Fact}
\newtheorem*{corollary}{Corollary}

\usepackage{algorithm}
\usepackage[noend]{algpseudocode}

\theoremstyle{plain}
\newtheorem{theorem}{Theorem}
\newtheorem{proposition}[theorem]{Proposition}
\newtheorem*{remark}{Remark}

\def\changemargin#1#2{\list{}{\rightmargin#2\leftmargin#1}\item[]}
\let\endchangemargin=\endlist 

\DeclareMathOperator{\codim}{codim}

\title{Fractals Avoiding Fractal Sets}

\author{Jacob Denson\\ \and Malabika Pramanik\\ \and Josh Zahl}

\begin{document}

\maketitle

We now describe an avoidance technique applied at a single scale. The technique is repeated for an infinite sequence of scales to then get the general result. Given a dyadic length $L$, we let $\mathcal{B}(L,d)$ denote the partition of $\mathbf{R}^d$ into a family of half open cubes with corners lying on the lattice $(\mathbf{Z}/L)^d$. If the dimension is clear, we simplify denote $\mathcal{B}(L,d)$ as $\mathcal{B}(L)$.

\begin{lemma}
	Consider three dyadic scales $L \gg R \gg S$. If $I \subset \mathbf{R}^d$ is a union of $\mathcal{B}(L)$ cubes and $K \subset \mathbf{R}^{nd}$ a union of $\mathcal{B}(S)$ cubes, then we can find $J \subset I$ such that for distinct $\mathcal{B}(S)$ subcubes $J_1, \dots, J_n$ of $J$, $J_1 \times \dots \times J_n$ is disjoint from $K$, and $J$ contains a $\mathcal{B}(S)$ subcube from all but $|K|R^{-nd}$ of the $\mathcal{B}(R)$ subcubes of $I$. 
\end{lemma}
\begin{proof}
	Form a random set $U$ by selecting uniformly randomly, from each $\mathcal{B}(R)$ subcube of $I$, a single subcube in $\mathcal{B}(S)$. Thus the probability that any subcube is selected is $(S/R)^d$. Since any two $\mathcal{B}(S)$ subcubes of $U$ lie in distinct elements of $\mathcal{B}(R)$, the only chance that a $\mathcal{B}(S)$ subcube $I$ of $K$ with distinct sides intersects $U^n$ is if $I_1, \dots, I_n$ all lie in separate cubes in $\mathcal{B}(R)$. Then the chance that each occurs is independant of one another, and so
	%
	\begin{align*}
		\mathbf{P}(I \in U^n) &= \mathbf{P}(I_1 \in U) \dots \mathbf{P}(I_n \in U) = (S/R)^{nd}
	\end{align*}
	%
	If $E$ denotes the number of $\mathcal{B}(S)$ subcubes $I$ of $K$ contained in $U^n$,
	%
	\[ \mathbf{E}(E) = \sum_{\substack{I \subset K}} \mathbf{P}(I \in U^n) = [|K| S^{-nd}] [(S/R)^{nd}] = |K| R^{-nd} \]
	%
	If, for each $\mathcal{B}(S)$ subcube $I$ of $U^n$, we remove $I_i$ from $U$, for any index $i$, we obtain a set $J$ with $J_1 \times \dots \times J_n$ disjoint from $K$ for any distinct $\mathcal{B}(R)$ subcubes $J_i$ of $J$. The interval $J$ contains an interval from all but $E$ sidelength $R$ cubes. In particular, we can select some nonrandom choice of $U$ such that $E \leq |K| R^{-nd}$, which gives the required $J$.
\end{proof}

We will be most interested in applying this lemma when $L$ and $S$ are consecutive hyperdyadic numbers. We fix a small $\delta$, and define $H_N = 2^{-\lfloor (1 + \delta)^N \rfloor}$ to be the $N$'th hyperdyadic number. Hyperdyadic numbers are useful scales for discussing Hausdorff dimension, because a weak-type bound allows us understand coverings of the set. The idea to use such scales was inspired by Katz and Tao's 2001 paper on Falconer's distance problem. We say a sequence $Y_1, Y_2, \dots$ {\it strongly} covers $Y$ if $Y \subseteq \limsup Y_N$, so each $y \in Y$ is in infinitely many of the sets $Y_N$.

\begin{lemma}
	If $Y$ has Hausdorff dimension $\alpha$, then for each $\varepsilon > \delta$ we can find a strong cover of $Y$ by a sequence of sets $Y_N$, where $Y_N$ is a union of $O_\varepsilon(N/H_N^{\alpha + \varepsilon})$ hyperdyadic cubes with sidelength $H_N$.
\end{lemma}
\begin{proof}
	Since $Y$ is $\alpha$ dimensional, then for each $\varepsilon_0$ and $N$ we can find a covering of $Y$ by dyadic cubes $I_i$, where $I_i$ has sidelength $L_i \leq H_N$, and $\sum L_i^{\alpha + \varepsilon_0} \lesssim_{\varepsilon_0} 1$. But we can now apply a weak type bound to conclude that the number of $L_i$ between $H_{N+1}$ and $H_N$ is $O_{\varepsilon_0}(1/H_{N+1}^{\alpha + \varepsilon_0})$. The calculation
	%
	\[ 1/H_{N+1}^{\alpha + \varepsilon_0} = (H_N/H_{N+1})^{\alpha + \varepsilon_0} (1/H_N^{\alpha + \varepsilon_0}) \lesssim 1/H_N^{\alpha + \varepsilon_0 + \delta} \]
	%
	implies we have used $O_{\varepsilon_0}(1/H_N^{\alpha + \delta + \varepsilon_0})$ sidelength $H_N$ cubes to cover $Y$. If we now perform this process for each index $N$, then collect all cubes together from all the covers to obtain a strong cover, and set $\varepsilon = \delta + \varepsilon_0$, we obtain the required result.
\end{proof}

\begin{corollary}
	Suppose $R$ is the closest dyadic number to $S^\beta$, where $\beta < 1$. Furthermore, suppose $|K| \lesssim N S^{nd - \gamma}$. Then provided
	%
	\begin{align*}
		\beta &\leq \frac{nd - \gamma - \log_S(|I|N^{-1}) - O \left( \log(1/S)^{-1} \right)}{(n-1)d}%\\
%		&= \frac{nd - \gamma - \log_S(|I|N^{-1})}{(n-1)d} - O \left( \frac{1}{(1 + \delta)^{N+1}} \right)
	\end{align*}
	%
	% S^{nd - \gamma - \beta(n-1)d} \leq |I|/2AN
	% nd - \gamma - \log(2AN/|I|) / \log(1/S) \geq \beta(n-1)d
	% \beta \leq (nd - \gamma)/(n-1)d - \log(2AN/|I|) / \log(1/S) (n-1)d
	%
	the set $J$ obtained in the discrete avoidance lemma contains a portion of at least half of all the sidelength $L_1$ intervals contined in $I$.
\end{corollary}

\begin{proof}
	The inequality here is equivalent to
	%
	\begin{align*}
		nd - \gamma - \beta(n-1)d &\geq \log_S(|I|N^{-1}) + O(1/\log(1/S))
	\end{align*}
	%
	We therefore find
	%
	\begin{align*}
		&\frac{\# (\text{$\mathcal{B}(R)$ subcubes not selected from})}{\# (\text{all $\mathcal{B}(R)$ subcubes})} = \frac{|K| R^{-nd}}{|I|R^{-d}}\\
		&\ \ \ \ \ \lesssim \frac{[N S^{nd - \gamma}][S^{- \beta n d}]}{|I| S^{-\beta d}} = N |I|^{-1} S^{nd - \gamma - \beta (n-1)d}\\
		&\ \ \ \ \ \leq N |I|^{-1} S^{\log_S(|I|N^{-1}) + O(1/\log(1/S))} = S^{O(1/\log(1/S))}
	\end{align*}
	%
	By choosing the constant in the exponent appropriately, we can make the fraction we began with not exceed $1/2$, so less than half of the intervals are missed in the selection process.
\end{proof}

\begin{remark}
	If $\log(|I|N^{-1}) = o((1 + \delta)^{N+1})$, then $\log_S(|I|N^{-1}) = o(1)$, which enables us to take $\beta = (nd - \gamma)/d(n-1) - o(1)$. Now if $I$ is an interval obtained from repeatedly dissecting intervals at hyperdyadic scales according to the corollary above, throwing out at most half the intervals at each stage, we know
	%
	\begin{align*}
		|I| &\geq 2^{-N} (1/H_1^{\beta_1})^d(H_1/H_2^{\beta_2})^d \dots (H_{N-1}/H_N^{\beta_N})^d\\
		&= 2^{-N} H_1^{d(1 - \beta_1)} \dots H_{N-1}^{d(1 - \beta_N)} H_N^{-d \beta_N}
	\end{align*}
	%
	Thus
	%
	\begin{align*}
		|\log_2 |I|N^{-1}| &\approx \left| \beta (1 + \delta)^N - N - \log N - (1 - \beta) \sum_{k = 1}^{N-1} (1 + \delta)^k \right| \lesssim (1 + \delta)^N
	\end{align*}
	%
	This means we don't have $\log |I|N^{-1} = o((1 + \delta)^{N+1})$, only $|\log |I|N^{-1}| = O((1 + \delta)^{N+1})$. This prevents us from choosing our required $\beta$, but, only {\it barely}. It certainly means that we can lower bound $\beta$, so we can obtain some Hausdorff dimension output result given a Hausdorff dimension input. Nonetheless, it's not the right output dimension we would expect (CHANGE OUR HAUSDORFF DIMENSION TO MAKE THIS `JUST' WORK?).
\end{remark}

%\begin{theorem}[Katz-Tao, 2001]
%	If $Y \subset \mathbf{R}^{nd}$ is an $\alpha$ dimensional set, and a small $\varepsilon > 0$ is given, then we can find an efficient {\it strong cover} of $Y$ by a sequence of discretized sets $Y_N$. By a strong cover, we mean $Y \subseteq \limsup Y_N$. By discretized, we mean $Y_N$ is a union of cubes with sidelength's between $C_\varepsilon^{-1} H_N^{1 + C\varepsilon}$ and $C_\varepsilon H_N^{1 - C\varepsilon}$. By an efficient cover, we mean that $|Y_N| \leq C_\varepsilon H_N^{nd - \alpha - C\varepsilon}$.
%\end{theorem}

%If we replace $Y_N$ with a set obtained from uniform sidelength cubes, each with sidelength $C_\varepsilon H_N^{1 - C\varepsilon}$, then we obtain a new strong cover with total mass $O_\varepsilon(H_N^{nd - \alpha - C\varepsilon(2nd + 1)})$

%\begin{align*}
%	2 \sum C_\varepsilon^{nd} H_N^{nd(1 - C\varepsilon)} &= 2 C_\varepsilon^{2nd} H_N^{-2 C\varepsilon nd} \sum C_\varepsilon^{-nd} H_N^{nd(1 + C \varepsilon)}\\
%	&\leq 2 C_\varepsilon^{2nd} H_N^{-2C\varepsilon nd} C_\varepsilon H_N^{nd - \alpha - C\varepsilon}\\
%	&= 2C_\varepsilon^{2nd + 1} H_N^{nd - \alpha - C \varepsilon (2nd + 1)}
%\end{align*}

%If we replace each cube in $Y_N$ by an expanded cube with sidelength $C_\varepsilon H_N^{1 - C\varepsilon}$, we get a set constructed from cubes with a uniform sidelength, 

%$(C_\varepsilon^{-1} H_N^{1 + C\varepsilon})^{nd} \leq |B| \leq (C_\varepsilon H_N^{1 - C\varepsilon})^{nd}$

%If $L$ and $S$ are the closest dyadic numbers to $C_\varepsilon H_N^{1 - C\varepsilon}$ and $C_\varepsilon H_{N+1}^{1 - C\varepsilon}$ respectively, and $|K| \leq H_{N+1}^{(1 - C\varepsilon)(n - \alpha)}$

%We now describe a general method for finding solutions to fractal avoidance problems by breaking the problem down into a sequence of discrete configuration problems. To achive this, we rely on a hyperdyadic discretization of the set $Y$. If we fix a small constant $\delta$, we obtain a sequence of dyadic scales $\smash{L_N = 2^{- \lfloor (1 + \delta)^k \rfloor}}$. These lengths are known as {\it hyperdyadic}. Since $2^{-(1 + \delta)^k} \leq L_N \leq 2^{1- (1 + \delta)^k}$, the floor operation doesn't get in our way that much.

%For any length $L$, we let $\mathcal{B}(L,d)$ denote the family of all cubes with sidelength $L$, and with corners on the lattice $(\mathbf{Z}/L)^d$. If $d$ is clear, we abbreviate the notation as $\mathcal{B}(L)$. It is a result of Katz and Tao that if $Y \subset \mathbf{R}^{nd}$ is a set with Hausdorff dimension $0 < \alpha < nd$, and $\varepsilon$ is a small constant, then we can find an efficient {\it strong cover} of $Y$ by a union of sets $Y_N \subset \mathbf{R}^{nd}$, for $N \geq 1$, with $Y_N$ a union of cubes in $\mathcal{B}(C_\varepsilon L_N^{1 - \varepsilon})$. By a strong cover, we mean $Y \subseteq \limsup Y_N$, and by efficient, we mean that $Y_N$ is the union of at most $\smash{C_\varepsilon L_N^{-(\alpha + \varepsilon)}}$ cubes with sidelength $L_N$. For the purposes of our calculations, we fix a constant $A$ such that $Y_N$ contains at most $A/L_N^\alpha$ sidelength $L_N$ cubes\footnote{As Josh indicated, this bound might not be true, but can hopefully be adjusted as required for the true result which Katz and Tao prove}.

We now construct $X$ as a limit of discrete nested sets $X_N$, where $X_N$ is a union of cubes in $\mathcal{B}(L_N)$, and $X_N^n$ is disjoint from all {\it non-diagonal} cubes with length $L_N$ in $Y_N$. By a non-diagonal cube, we mean $I \subset \mathcal{B}(L_N)$ such that $I_i \neq I_j$ for $i \neq j$, where $I_i$ and $I_j$ are projections onto the $i$'th and $j$'th coordinate set of $\mathbf{R}^{nd} = (\mathbf{R}^d)^n$. In particular, this means that $X^n$ avoids all cubes at all lengths in the cover of $Y$, and in particular, avoids non-diagonal elements of $Y$.

\begin{lemma}
	If $X_N^n$ avoids non-diagonal cubes in $Y_N$, for large $N$, $X^n \cap Y \subset \Delta$.
\end{lemma}
\begin{proof}
	Let $y \in Y$ be a point not contained in $\Delta$. Because $Y_N$ forms a strong cover, we can find an infinite sequence of indices $N_k$ with $y \in Y_{N_k}$. For a suitably large choice of $K$, $\sqrt{n} \cdot L_{N_k} < d(y,\Delta)$ for $k \geq K$. But this means that the cube of $\mathcal{B}(L_{N_k})$ containing $y$ is disjoint from $\Delta$, and therefore cannot be a non-diagonal cube. This means that $X_{N_k}^n$ is disjoint from this cube, and therefore does not contain $y$. But then $X^n = \lim X_{N_k}^n$ cannot contain $y$. Taking contrapositives to our argument, we have shown every point in $X^n \cap Y$ must lie in $\Delta$.
\end{proof}

%We define $X_N$ recursively. As an initial set, for lack of an interesting choice, we let $X_0 = [0,1]^d$. We continue choosing $X_1 = [0,1]^d$, $X_2 = [0,1]^d$, and so on, until we get to an index $N$ which allows us to apply the discrete corollary with $I = X_N$, and $K = Y_N$, i.e.
%
%\[ \frac{\log N + 1}{(1 + \delta)^{N+1}} \lesssim 1 \]
%
%We then set $X_{N+1}$ equal to the resultant $J$. The fact that $J$ contains a portion of half of the coarse scale intervals implies that $|X_{N+1}| \geq 2^{-1} |X_N| H_{N+1}^{1 - \beta}$. 

\end{document}