\documentclass[12pt, dvipsnames]{report}

\usepackage{amsmath}
\usepackage{algorithm}
%\usepackage{algorithmic}
\usepackage[noend]{algpseudocode}

\usepackage{amsmath}
\usepackage{amssymb}
\usepackage{amsthm}
\usepackage{amsopn}

\usepackage{kpfonts}

\usepackage{graphicx}

% Probably don't need this on notes anymore
%\usepackage{kbordermatrix}

% Standard tool for drawing diagrams.
\usepackage{tikz}
\usepackage{tkz-berge}
\usepackage{tikz-cd}
\usepackage{tkz-graph}

\usepackage{comment}

%
\usepackage{multicol}

%
\usepackage{framed}

%
\usepackage{mathtools}

%
\usepackage{float}

%
\usepackage{subfig}

%
\usepackage{wrapfig}

%
\let\savewideparen\wideparen
\let\wideparen\relax
\usepackage{mathabx}
\let\wideparen\savewideparen

% Used for generating `enlightening quotes'
\usepackage{epigraph}

% Forget what this is used for :P
\usepackage[utf8]{inputenc}

% Used for generating quotes.
\usepackage{csquotes}

% Allows what to generate links inside
% generated pdf files
\usepackage{hyperref}

% Allows one to customize theorem
% environments in mathematical proofs.
\usepackage{thmtools}

% Gives access to a proof
\usepackage{lplfitch}

% I forget what this is for.
\usepackage{accents}

% A package for drawing simple trees,
% as a substitute for unnesacary TIKZ code
\usepackage{qtree}

% Enables sequent calculus proofs
\usepackage{ebproof}

% For braket notation
\usepackage{braket}

% To change line spacing when using mathematical notations which require some height!
\usepackage{setspace}

%\usepackage[dvipsnames]{xcolor}

\usepackage{float}

% For block commenting
\usepackage{comment}




\setlength\epigraphwidth{8cm}

\usetikzlibrary{arrows, petri, topaths, decorations.markings}

% So you can do calculations in coordinate specifications
\usetikzlibrary{calc}
\usetikzlibrary{angles}

\theoremstyle{plain}
\newtheorem{theorem}{Theorem}[chapter]
\newtheorem{axiom}{Axiom}
\newtheorem{lemma}[theorem]{Lemma}
\newtheorem{corollary}[theorem]{Corollary}
\newtheorem{prop}[theorem]{Proposition}
\newtheorem{exercise}{Exercise}[chapter]
\newtheorem{fact}{Fact}[chapter]

\newtheorem*{example}{Example}
\newtheorem*{proof*}{Proof}

\theoremstyle{remark}
\newtheorem*{exposition}{Exposition}
\newtheorem*{remark}{Remark}
\newtheorem*{remarks}{Remarks}

\theoremstyle{definition}
\newtheorem*{defi}{Definition}

\usepackage{hyperref}
\hypersetup{
    colorlinks = true,
    linkcolor = black,
}

\usepackage{textgreek}

\makeatletter
\renewcommand*\env@matrix[1][*\c@MaxMatrixCols c]{%
  \hskip -\arraycolsep
  \let\@ifnextchar\new@ifnextchar
  \array{#1}}
\makeatother

\renewcommand*\contentsname{\hfill Table Of Contents \hfill}

\newcommand{\optionalsection}[1]{\section[* #1]{(Important) #1}}
\newcommand{\deriv}[3]{\left. \frac{\partial #1}{\partial #2} \right|_{#3}} % partial derivative involving numerator and denominator.
\newcommand{\lcm}{\operatorname{lcm}}
\newcommand{\im}{\operatorname{im}}
\newcommand{\bint}{\mathbf{Z}}
\newcommand{\gen}[1]{\langle #1 \rangle}

\newcommand{\End}{\operatorname{End}}
\newcommand{\Mor}{\operatorname{Mor}}
\newcommand{\Id}{\operatorname{id}}
\newcommand{\visspace}{\text{\textvisiblespace}}
\newcommand{\Gal}{\text{Gal}}

\newcommand{\xor}{\oplus}
\newcommand{\ft}{\wedge}
\newcommand{\ift}{\vee}

\newcommand{\prob}{\mathbf{P}}
\newcommand{\expect}{\mathbf{E}}
\DeclareMathOperator{\Var}{\mathbf{V}}
\newcommand{\Ber}{\text{Ber}}
\newcommand{\Bin}{\text{Bin}}

%\newcommand{\widecheck}[1]{{#1}^{\ft}}

\DeclareMathOperator{\diam}{\text{diam}}

\DeclareMathOperator{\QQ}{\mathbf{Q}}
\DeclareMathOperator{\ZZ}{\mathbf{Z}}
\DeclareMathOperator{\RR}{\mathbf{R}}
\DeclareMathOperator{\HH}{\mathbf{H}}
\DeclareMathOperator{\CC}{\mathbf{C}}
\DeclareMathOperator{\AB}{\mathbf{A}}
\DeclareMathOperator{\PP}{\mathbf{P}}
\DeclareMathOperator{\MM}{\mathbf{M}}
\DeclareMathOperator{\VV}{\mathbf{V}}
\DeclareMathOperator{\TT}{\mathbf{T}}
\DeclareMathOperator{\LL}{\mathcal{L}}
\DeclareMathOperator{\EE}{\mathbf{E}}
\DeclareMathOperator{\NN}{\mathbf{N}}
\DeclareMathOperator{\DQ}{\mathcal{Q}}
\DeclareMathOperator{\IA}{\mathfrak{a}}
\DeclareMathOperator{\IB}{\mathfrak{b}}
\DeclareMathOperator{\IC}{\mathfrak{c}}
\DeclareMathOperator{\IP}{\mathfrak{p}}
\DeclareMathOperator{\IQ}{\mathfrak{q}}
\DeclareMathOperator{\IM}{\mathfrak{m}}
\DeclareMathOperator{\IN}{\mathfrak{n}}
\DeclareMathOperator{\IK}{\mathfrak{k}}
\DeclareMathOperator{\ord}{\text{ord}}
\DeclareMathOperator{\Ker}{\textsf{Ker}}
\DeclareMathOperator{\Coker}{\textsf{Coker}}
\DeclareMathOperator{\emphcoker}{\emph{coker}}
\DeclareMathOperator{\pp}{\partial}
\DeclareMathOperator{\tr}{\text{tr}}

\DeclareMathOperator{\supp}{\text{supp}}

\DeclareMathOperator{\codim}{\text{codim}}

\DeclareMathOperator{\minkdim}{\dim_{\mathbf{M}}}
\DeclareMathOperator{\hausdim}{\dim_{\mathbf{H}}}
\DeclareMathOperator{\lowminkdim}{\underline{\dim}_{\mathbf{M}}}
\DeclareMathOperator{\upminkdim}{\overline{\dim}_{\mathbf{M}}}
\DeclareMathOperator{\lhdim}{\underline{\dim}_{\mathbf{M}}}
\DeclareMathOperator{\lmbdim}{\underline{\dim}_{\mathbf{MB}}}
\DeclareMathOperator{\packdim}{\text{dim}_{\mathbf{P}}}
\DeclareMathOperator{\fordim}{\dim_{\mathbf{F}}}

\DeclareMathOperator*{\argmax}{arg\,max}
\DeclareMathOperator*{\argmin}{arg\,min}

\DeclareMathOperator{\ssm}{\smallsetminus}

\title{Putnum and Beyond Solution Manual}
\author{Jacob Denson}

\begin{document}

\pagenumbering{gobble}
\maketitle
\tableofcontents
\pagenumbering{arabic}

\chapter{Basic Concepts}

\section{Preliminaries}

\begin{exercise}
    Basic Vector Space Terminology.
    \begin{enumerate}
        \item[(a)] Show that if $A$ is an absorbing set or a nonempty balanced set, then $0 \in A$.
        \begin{proof}
            If $A$ is absorbing, there is $\lambda > 0$ for which $0 \in \lambda A$. But then
            %
            \[ 0 = \lambda^{-1} 0 \in \lambda^{-1} \lambda A = A \]
            %
            If $A$ is a non-empty balanced set, then $0 \in 0 A \subset A$, since $|0| < |1|$.
        \end{proof}

        \item[(b)] Show that if $A$ is balanced, then $\alpha A = A$ whenever $|\alpha| = 1$.
        \begin{proof}
            It is obvious that $\alpha A \subset A$. Conversely, since $|\alpha^{-1}| = 1$, $\alpha^{-1} A \subset A$. Given $a \in A$, $\alpha^{-1} a \in \alpha^{-1} A \subset A$, but then $a = \alpha(\alpha^{-1} a) \in \alpha A$.
        \end{proof}

        \item[(c)] Suppose that $\mathcal{B}$ is a collection of balanced subsets of $X$. Show that $\bigcup \{ S : S \in \mathcal{B} \}$ and $\bigcap \{ S : S \in \mathcal{B} \}$ are both balanced.
        \begin{proof}
            For any $|\alpha| \leq 1$, $B \in \mathcal{B}$, $\alpha B \subset B$, so that
            %
            \[ \alpha \bigcap_{B \in \mathcal{B}} B = \bigcap_{B \in \mathcal{B}} \alpha B \subset \bigcap_{B \in \mathcal{B}} B \]
            %
            \[ \alpha \bigcup_{B \in \mathcal{B}} B = \bigcup_{B \in \mathcal{B}} \alpha B \subset \bigcup_{B \in \mathcal{B}} B \]
            %
            and therefore the union and intersection of balanced sets is balanced.
        \end{proof}

        \item[(d)] Suppose that $\mathcal{C}$ is a collection of convex subsets of $X$. Show that $\bigcap \{ S : S \in \mathcal{C} \}$ is convex.
        \begin{proof}
            If $C \in \mathcal{C}$ $a, b \in C$, $\lambda \in [0,1]$, then $\lambda a + (1 - \lambda) B \in C$. By putting $\forall C \in \mathcal{C}$ in the front of these statements, we obtain the statement for the intersection.
        \end{proof}

        \item[(e)] Show that if $A$ is convex, then $x + A$ and $\alpha A$ are convex.
        \begin{proof}
            If $x + a, x + b \in x + A$, $\lambda \in [0,1]$, then
            %
            \[ \lambda (x + a) + (1 - \lambda) (x + b) = x + (\lambda a + (1 - \lambda) b) \in x + A \]
            %
            and therefore $x + A$ is convex.
        \end{proof}
    \end{enumerate}
\end{exercise}

\begin{exercise}
    \begin{enumerate}
        \item[(a)] Show that the ``addition'' and ``multiplication by scalars'' defined for sets obey the commutative and associative laws for vector spaces. That is, show that $A + B = B + A$, that $A + (B + C) = (A + B) + C$, and that $\alpha (\beta A) = (\alpha \beta) A$. Show also that $(x + A) + (y + B) = (x + y) + (A + B)$.
        \item[(b)] Show that $\alpha(A + B) = \alpha A + \alpha B$.
        \item[(c)] Show that $(\alpha + \beta)A \subset \alpha A + \beta A$, but that equality does not always hold.
        \begin{proof}
            The equations can be verified pointwise. If the equations is satisfied on the left side by a point, it holds on the right side, and vice versa. This is not true of the third question, since 
        \end{proof}
    \end{enumerate}
\end{exercise}

\begin{exercise}
    \begin{enumerate}
        \item[(a)] Prove that $A$ is convex if and only if $sA + tA = (s + t)A$ for all positive $s$ and $t$. (Consider the special case in which $s + t = 1$).
        \begin{proof}
            s
        \end{proof}
    \end{enumerate}
\end{exercise}

\section{Norms}

\section{First Properties of Norm Spaces}

\begin{exercise}
    Let $K$ be a compact Hausdorff space and let $X$ be a normed space. By Corollary 1.3.4, the collection of all continuous functions from $K$ into $X$ is a vector spacewhen functions are added and multiplied by scalars in the usual way. Define a norm on this vector space by the formula
    %
    \[ \| f \|_\infty = \begin{cases} \max \{ \|f(x)\| : x \in K \} & \text{if}\ K \neq \emptyset \\ 0 & K = \emptyset \end{cases} \]
    %
    The resulting normed space is denoted $C(K,X)$.
    %
    \begin{enumerate}
        \item[(a)] Show that $\| \cdot \|_\infty$ is in fact a norm on $C(K,X)$.
        \begin{proof}
            $\| f + g \| \leq \| f \| + \| g \|$, since $\sup (A + B) \leq \sup A + \sup B$ for any $A$ and $B$. $\| \alpha f \| = |\alpha| \| f \|$, since
            %
            \[ \max \{ \|\alpha f(x)\| : x \in K \} = |\alpha| \max \{ \|f(x)\| : x \in K \} \]
            %
            And if $\| f \| = 0$, then $\|f(x)\| = 0$ for all $x$, so that $f(x) = 0$ for all $x$.
        \end{proof}

        \item[(b)] Show that if $X$ is a Banach space, then so is $C(K,X)$.
        \begin{proof}
            Let $f_1, f_2, \dots$ be a Cauchy sequence in $C(K,X)$, so that $\| f_i - f_j \| \to 0$.
        \end{proof}
    \end{enumerate}
\end{exercise}

\end{document}