\documentclass{article}

\usepackage{kpfonts}
\usepackage{amsmath}
\usepackage{amssymb}
\usepackage{amsthm}
\usepackage{esint}
\usepackage{MnSymbol}

\theoremstyle{plain}
\newtheorem{theorem}{Theorem}
\newtheorem{lemma}[theorem]{Lemma}
\newtheorem{corollary}[theorem]{Corollary}
\newtheorem{prop}[theorem]{Proposition}
\newtheorem*{example}{Example}
\newtheorem*{remark}{Remark}

\title{Incidence Theorems in Arbitrary Characteristic}
\author{Jacob Denson}
\date{\today}

\DeclareMathOperator{\RR}{\mathbf{R}}
\DeclareMathOperator{\CC}{\mathbf{C}}
\DeclareMathOperator{\PP}{\mathbf{P}}
\DeclareMathOperator{\LL}{\mathcal{L}}
\DeclareMathOperator{\AAA}{\mathbf{A}}
\DeclareMathOperator{\ord}{\text{ord}}
\DeclareMathOperator{\coker}{\text{coker}}
\DeclareMathOperator{\emphcoker}{\emph{coker}}

\begin{document}

\maketitle

\begin{itemize}
    \item We explore incidence results from a non-topological perspective, i.e. bound
    %
    \[ I(P,L) = \sum_{p \in P} \ord_p(L)\ \ \ \ I(L) = \sum_p \ord_p(L) - 1 \]

    \item Szemeredi-Trotter: In real projective plane, $I(P,L) \lesssim |P|^{2/3} |L|^{2/3} + |P| + |L|$.

    \item If we try proving incidence results in $\PP^2$ over an arbitrary field $k$, we are forced to try and use non topological techniques.

    \item Finite field incidence results have many applications.

    \item Not just a heuristic: mathematical logic (the completeness of affine and projective geometry) shows a statement is provable in the real projective plane using only basic methods of affine or projective geometry if and only if it is true over any field $k$. If we add `synthetic' topology, like saying a point is `between' two other points, provable iff true over any ordered field.

    \item Using only Cauchy-Schwartz together with an $L^2$ incidence bound, we obtained $I(P,L) \leq |L|^{3/4} |M|^{3/4}$. This holds in any field, and is tight.

    \item Example: Let $k = \mathbf{F}_p$, with $P$ and $L$ consisting of all projective points and lines. Then $|P| = |L| = p^2 + p+1$, and we count
    %
    \[ I(P,L) = (p+1)(p^2+p+1) = \frac{p+1}{(p^2 + p + 1)^{1/2}} |L|^{3/4} |M|^{3/4} \]
    %
    As $p \to \infty$, $I(P,L) \sim |L|^{3/4} |M|^{3/4}$, so we obtain sharpness.

    \item Thus no purely projective proof of Szemeredi-Trotter theorem.

    \item Do we obtain Szemeredi-Trotter bound for `sparse' sets of points and lines?

    \item Goal: Discuss results of Koll\'{a}r for three dimensional incidences in three dimensions.

    \item Key idea: Arithmetic genus upper bounds incidences, is easily calculated, and works over any field.
\end{itemize}

\section{Main Result}

\begin{itemize}
    \item Many results in paper: We discuss the result which gives the clearest use of interesting principles.

    \item (Theorem 6): Let $L$ be a set of projective lines in a field of characteristic $0$, or of characteristic $p$ where $|L| \leq p^2$, and let $c > 0$ be a constant. If no plane or smooth quadric contains more than $c |L|^{1/2}$ lines, then $I(L) \leq (1000 + c) |L|^{3/2}$ (If you work hard enough, these results give $(29.1 + c/2) |L|^{3/2}$, and you need only assume $|L| < 11/6 p^2$).

    \item Also proved (Theorem 2): If $L$ is a set of lines, and $P$ a set of points in a field of characteristic zero, or $p > (6|P|)^{1/3}$, and no plane contains more than $c |L|^{1/2}$ lines, then
    %
    \[ I(L,P) \leq (10 + c^2)|L||P|^{1/3} + 10|P| \]

    \item (Theorem 3): Same assumptions, arbitrary field,
    %
    \[ I(L,P) \leq 10|P||L|^{2/5} + 10 |P|^{6/5} + c^2 |L||P|^{1/3} + 10 |P| \]

    \item Note the assymmetry here: Points are not dual to lines in $\PP^3$.
\end{itemize}

Theorem 6 is a variant of the Guth-Katz bound, but differs in three important ways:
%
\begin{itemize}
    \item Guth-Katz only holds when $K = \RR$.
    \item Their bound counts the cardinality of incidences rather than multiplicity.
    \item Explicit constants not obtained (in this field $10^{60}$ constants are not unknown).
\end{itemize}
%
Nonetheless, we obtain Theorem 6 by working very similarly to Guth-Katz, replacing polynomial partitioning.

%\begin{example}
%    NOTE TO DAN: THIS DOESN'T REALLY SHOW SHARPNESS IN THEOREM 6, RIGHT?

%    Consider the grid example, i.e. we let $L$ be the family of $3n^2$ axis parallel lines passing through the $n^3$ points in $\mathbf{Z}^3 \cap [1,n]^3$. For each point $p$ in this lattice, $\text{ord}_p(L) = 3$, and so $I(L) = 2n^3$. The only problem with this example is that two many lines lie in a single plane. But this is easily fixed. If we consider the Veronese-type embedding $(x,y,z) \mapsto (xyz,xy,xz,yz,x,y,z)$, then axis-parallel lines in $\RR^3$ remain straight in the embedding, whereas no hyperplane in $\RR^7$ contains more than two lines. Thus a generic projection from $\RR^7$ to $\RR^3$ gives a family of lines such that any plane contains no more than two lines. Thus the hypothesis of Theorem 1 is satisfied with $c = 1/n$, and so Theorem 1 shows that $I(L) \leq (29.1 + 0.5/n) 3^{3/2} n^3 = 151.2 n^3 + 2.59 n^2$
%\end{example}

\section{Proof of Theorem 6}

Fix $L$. Standard polynomial counting techniques enable us to find a surface $S$ containing all lines in $L$ with degree $d \leq (6|L|)^{1/2} - 2$. Decompose $S$ into irreducible components $S_1, \dots, S_k$, and let $L_i$ denote the set of lines that are contained in $S_i$, but not in $S_1, \dots, S_{i-1}$.

\begin{lemma}
    \[ I(L) = \underbrace{\sum_{i = 1}^K I(L_i)} + \underbrace{\sum_{i = 2}^k |L_i \cap (L_1 \cup \dots \cup L_{i-1}))}| \]
    %
    The first term is {\it Internal Incidences}, the second {\it External Incidences}.
\end{lemma}
\begin{proof}
    Note that $\{ L_i \}$ is a partition of $L$, and $\ord_p(L) = \sum \ord_p(L_i)$, so
    %
    \begin{align*}
        \sum_{i = 1}^k I(L_i) &= \sum_{i = 1}^k \sum_{p \in L_i} \ord_p(L_i) - 1\\
        &= \sum_{p \in L} \ord_p(L) - |\{ i: p \in L_i \}|\\
        &= I(L) - \sum_{i = 2}^k |L_i \cap (L_1 \cup \dots \cup L_{i-1})|. \qedhere
    \end{align*}
\end{proof}

\begin{itemize}
    \item Bounding external incidences is easy.
    \item Bounding internal incidences on ruled surfaces done analogously to Guth-Katz.
    \item Carrying out these calculations gives
%
\[ I(L) \leq c \cdot |L|^{3/2} + 10 \cdot |L|^{3/2} + \sum_{\text{non-ruled}\ S_i} I(L_i) \]
    \item So similar to Guth-Katz, so we omit the details. All calculations can be done over arbitrary fields.

    \item Novel feature of Koll\'{a}r: Switch polynomial partitioning with arithmetic genus calculation over non-ruled surfaces.
\end{itemize}

\section{Arithmetic Genus}

\begin{itemize}
    \item Incidence number if a projective invariant of a family of lines.

    \item Plausible that incidence number is related to other classical projective invariants.

    \item Kollar's observation: arithmetic genus is closely related to the incidence number, and furthermore, is not too difficult to control (linear algebra).
\end{itemize}

Recall the following notation:
%
\begin{itemize}
    \item Let $R = k[x_0, \dots, x_d]$.

    \item If $M$ is finitely generated over $R$, the {\it Hilbert function} is $H_M(t) = \dim_k(M_t)$.

    \item $H_M$ agrees with a degree $\leq d$ polynomial $P_M(t)$ for sufficiently large $t$.

%    \item Additivity: If $0 \to M \to N \to L \to 0$ is exact, then $P_N = P_M + P_L$.

    \item If $X$ is a projective variety/subscheme of $\PP^n$ corresponding to some homogenous ideal $I$, then $k[X] = R/I$ is the homogenous coordinate ring, and we let $H_X = H_{R/I}$, $P_X = P_{R/I}$.

    \item $P_X$ describes {\it all} additive invariants of $X$, i.e. invariants satisfying the rank nullity theorem. For instance,
    \begin{itemize}
        \item $\deg(P_X) = \dim(X)$.
        \item The leading coefficient is $\deg X / (\dim X)!$
        \item The constant coefficient is the Euler characteristic $\chi(X,\mathcal{O})$.
    \end{itemize}

    \item If $X$ is a projective curve, then one calls $g(X) = 1 - \chi(X,\mathcal{O})$ the {\it arithmetic genus} of $X$. This agrees with the topological genus if $X$ is a smooth curve over the complex numbers.
\end{itemize}

\begin{lemma}
    If $H = V(f)$ is a projective hypersurface of degree $a$, then
    %
    \[ P_{X \cap H}(t) = P_X(t) - P_X(t-a) \]
\end{lemma}
\begin{proof}
    If $f$ is homogenous of degree $a$, let $H$ be the projective variety corresponding to $f$. If $f$ is not a zero-divisor of $X$, the map $g \mapsto fg$ induces an exact sequence of vector spaces
    %
    \[ 0 \to k[X]_{t-a} \to k[X]_t \to k[X \cap H]_t \to 0 \]
    %
    Thus $P_X(t-a) + P_{X \cap H}(t) = P_X(t)$, for $t \geq a$.
\end{proof}

Consider projective hypersurfaces $H_1, \dots, H_{n-1}$, with degrees $a_1, \dots, a_{n-1}$. If $C = H_1 \cap \dots \cap H_{n-1}$ is one-dimensional, we say it is a {\it complete intersection curve}.

\begin{lemma}
    $g(C) = 1 + \frac{\sum a_i - (n + 1)}{2} \prod_{i = 1}^{n-1} a_i$.
\end{lemma}
\begin{proof}
     Since
    %
    \[ P_{\PP^n}(t) = {{t+n} \choose {n}} \]
    %
    Then using the last lemma recursively, we conclude
    %
    \[ P_C(t) = \left( \prod_{i = 1}^{n-1} a_i \right) \cdot t - \frac{\sum a_i - (n + 1)}{2} \prod_{i = 1}^{n-1} a_i. \]
    %
    Thus
    %
    \[ g(C) = 1 - \chi(C) = 1 + \frac{\sum a_i - (n + 1)}{2} \prod_{i = 1}^{n-1} a_i. \qedhere \]
\end{proof}

A few technicalities here: Intersections here are {\it scheme theoretical}, rather than {\it set theoretical}. And the genus can disagree for these two objects.

\begin{example}
    Let $I_1 = (xy-zt)$, and $I_2 = (x^2 + xy - zt)$ be homogenous ideals, generating hyperboloids $H_1 = V(I_1)$ and $H_2 = V(I_2)$ in $\PP^3$. Let $C = H_1 \cap H_2$ be the scheme theoretic intersection. Then $I_1 \oplus I_2 = (x^2, xy - zt)$, and an easy algebraic calculation shows $x \not \in I_1 \oplus I_2$, but $x^2 \in I_1 \oplus I_2$. Thus $R/(I_1 \oplus I_2)$ is not a reduced ring, and the reduction of this ring is $R/I(x,zt)$ with corresponding `reduced curve' $C'$. The last lemma shows $P_C(t) = 4t$, yet $P_{C'}(t) = 2(t+1) = 2t + 2$. Thus $g(C) = 1$, but $g(C') = -1$. Note that both of these differ from the topological genus of the set, which is zero.
\end{example}

 Note, that in this example $g(C') \leq g(C)$. Using standard techniques in scheme cohomology, we can show that for {\it any} complete intersection curve $C$, if $C'$ is a reduced version of $C$, then
 %
 \[ g(C') \leq g(C) = 1 + \frac{\sum a_i - (n+1)}{2} \prod_{i = 1}^{n-1} a_i \]
 %
 We leave the details to the paper, which involves some basic sheaf cohomology. According to the paper, this inequality is intuitive: For families of curves, the arithmetic genus tends to jump up near the singular curves in the family, i.e. for curves where the scheme theoretic intersection disagrees with the set theoretic intersection.

 So why should arithmetic genus tell us about the incidences of a set of lines? Recall that for a degree $d$ curve $C$, $H_C(t) = dt + 1 - g(C)$. Thus among all degree $d$ curves, those with small genus have the most degree $t$ homogenous functions defined on them. If $L$ is a {\it disjoint} union of projective lines $\{ L_i \}$, then a degree $t$ regular function on $L$ is precisely given by degree $t$ functions on each individual line. However, if the lines in $L$ are allowed to intersect, then the degree $t$ regular functions on each line must agree at the intersection points. Thus the dimension of the space of degree $t$ regular functions must decrease, hence the genus increases. Thus there should be a relation between the incidence number and the arithmetic genus.

Now suppose $L$ is formed from $k$ projective lines $L_1, \dots, L_k$. For each line $L_i$, the embedding $e_i: L_i \to L$ induces a graded restriction homomorphism $e_i^*: k[L] \to k[L_i]$ given by $e_i^*(f) = f|_{L_i}$. We can put these together to obtain a map $e^*: k[L] \to \bigoplus k[L_i]$. We let $e^*_t: k[L]_t \to \bigoplus k[L_i]_t$ denote the restriction of $e^*$ to $k[L]_t$.

\begin{lemma}
    For $t \gg 0$, $g(L) = \dim(\emphcoker(e^*_t)) + k-1$.
\end{lemma}
\begin{proof}
    Both $L$ and $L^* = \coprod L_i$ have the same degree, so $P_L$ differs from $P_{L^*}$ by a constant coefficient. If we consider
    %
    \[ 0 \to k[L]_t \to \bigoplus k[L_i]_t \to \coker(e^*_t) \to 0 \]
    %
    The rank nullity theorem then implies that
    %
    \[ \dim k[L]_t + \dim (\coker(e^*_t)) = \sum_{i = 1}^k \dim k[L_i]_t \]
    %
    Thus $\dim(\coker(e^*_t)) = H_L(t) - H_{L^*}(t)$. Note that both $L$ and $L^*$ have the same degree, so $H_L$ differs from $H_{L^*}$ by a constant for sufficiently large $t$. We calculate that $P_{L^*}(t) = k(t+1)$, so $g(L^*) = 1-k$. Thus $g(L) = \dim(\coker(e^*)) + k-1$.
\end{proof}

\begin{example}
    Suppose $L$ is a family of $k$ lines intersecting at the origin in $\mathbf{P}^3$. Since $e$ preserves elements at $\infty$, $e^*$ preserves the affine degree of any element of $k[L]$, i.e. the degree of $f(1,x_1,x_2,x_3)$ is the same as $(e^*_i f)(1,t_0,t_1)$ for each $i$. If we decompose
    %
    \[ k[L]_t = k[L]_{t1} \oplus \dots \oplus k[L]_{tt}\ \ \ k[L_i]_t = k[L_i]_t = k[L_i]_{t1} \oplus \dots \oplus k[L_i]_{tt} \]
    %
    where $k[L]_{ts}$ is the space of homogenous polynomials with projective degree $t$ and affine degree $s$, and $e_{ts}^*: k[L]_{ts} \to \bigoplus k[L_i]_{ts}$ then
    %
    \[ \dim \coker(e_t^*) = \bigoplus_{s = 1}^t \dim ( \coker(e_{ts}^*)) \]
    %
    We use the elementary estimate
    %
    \[ \dim ( \coker(e_{ts}^* )) \geq \sum_{i = 0}^k \dim k[L_i]_{ts} - \dim k[L]_{ts} \geq k -  {s+2 \choose 2} \geq k - (s+2)^2/2 \]
    %
    This estimate is only useful if $i \leq (2k)^{1/2} - 2$. For $s = 0$, we obtain the $\dim(\coker(e_t^*)) \geq k-1$. This will suffice for our purposes, but we can also calculate that
    %
    \begin{align*}
        \dim(\coker(e_t^*)) &\geq \sum_{i = 1}^{\lfloor (2k)^{1/2} - 2 \rfloor} k - (i+2)^2/2 \geq 0.7 \cdot (k-1)^{3/2}
    \end{align*}
    %
    In particular, $g(L) \geq 0.7 \cdot (k-1)^{3/2} + (k-1)$, which is more useful in the points incidence bounds we don't prove here (which use Cauchy-Schwarz like in Szemeredi-Trotter).
\end{example}

We can put this calculation together to get bounds on general $k$ element line incidences $L$. Given any point of incidence $p \in L$, let $L(p)$ denote all lines through $p$. We consider the restriction map $f^*: k[L] \to \bigoplus_p \bigoplus_{p \in L_i} k[L_i]$, and $e^*$ factors through $f^*$. If $t \geq (k-1)^2$, the map from one cokernel to the other is surjective, so
%
\[ \dim(\coker(e^*)) \geq \sum \dim(\coker(e^*_p)) \geq \sum_p (\ord_p(L) - 1) \]
%
%\[ \dim(\coker(e^*)) \geq \sum \dim(\coker(e^*_p)) \geq \sum_p 0.7 \cdot (r(p)-1)^{3/2}. \]
%
Thus $g(L) \geq \sum_p (\ord_p(L) - 1)$.

Now we put together this calculation with the theory of complete intersection curves. Let $L'$ be lines lying on the non-ruled components $S'$ of $S$. If $T$ denotes the surface of degree $\leq 11d - 24$ corresponding to the flecnode polynomial of $S$, then $S \cap T$ contains all the lines $L$. Monge's theorem implies that $T$ contains no component of $S$, so $S \cap T$ is one dimensional, and thus a complete intersection curve. Thus
%
\begin{align*}
    I(L') &= \sum \ord_p(L) - 1 \leq g(L) \leq g(S \cap T)\\
    &\leq 0.5 \cdot (d(11d - 24)(d + (11d - 24) - 2)) \leq 66d^3 \leq 66(6|L'|)^{3/2} \leq 970 |L'|^{3/2} \leq 970 |L|^{3/2}
\end{align*}
%
We therefore find that $I(L) \leq c |L|^{3/2} + 1000 |L|^{3/2}$

This completes the proof of Theorem 6, except for one technicality: Monge's theorem only applies over fields of characteristic zero; but a modern version actually holds over a field of characteristic $p$ provided $p \geq |L|^{1/2}$, so Theorem 6 remains true in this scenario.

%\begin{example}
%    Let $l_1$ and $l_2$ be two lines in $\PP^3$ which do not intersect. There is a unique line passing through any fixed point, and passing through $l_1$ and $l_2$. Over a finite field $\mathbf{F}_q$, this gives a set of $(p^3 + p^2+p+1)/(p+1) = p^2 + 1$ lines covering $\PP^3$. Given a generic sequence of $n$ pairs of such lines, there union is a set of $m$ lines
%\end{example}

\section{Fluff}

\begin{lemma}
    If the $S_i$ are appropriately ordered,
    %
    \[ \sum_{i = 2}^k |L_i \cap (L_1 \cup \dots \cup L_{i-1})| \leq 1.23 |L|^{3/2}. \]
\end{lemma}
\begin{proof}
    Suppose $l \in L_i$. Then $l$ is not contained in $S_1, \dots, S_{i-1}$. Bezout's theorem thus implies that $l \cap S_j$ contains at most $\deg(S_j)$ points if $j < i$. But $l \cap L_j \subset l \cap S_j$, so a union bound gives
    %
    \[ \sum_{i = 2}^k |L_i \cap (L_1 \cup \dots \cup L_{i-1})| \leq \sum_{j < i} |L_i| \deg(S_j) \leq \sum |L_i| \deg(S_j) = d|L| \leq 6^{1/2} |L|^{3/2} \]
    %
    We can do slightly better by reordering the $S_i$. If $|L_i|/\deg(S_i)$ is monotonically decreasing, then $|L_j|/\deg(S_j) \geq |L_i|/\deg(S_i)$
    %
    \[ \sum_{j < i} |L_i| \deg(S_j) \leq \sum_{j < i} \deg(S_i) |L_j| \]
    %
    Note that
    %
    \[ 2 \sum_{j < i} |L_i| \deg(S_j) \leq \sum_{j < i} |L_i| \deg(S_j) + \sum_{j > i} |L_i| \deg(S_j) \leq 6^{1/2} |L|^{3/2} \]
    %
    Thus $\sum_{j < i} |L_i| \deg(S_j) \leq (3/2)^{1/2} |L|^{3/2}$. This can be tight if $|L_i|\deg(S_j) = |L_j| \deg(S_i)$, and $|L_i|\deg(S_i)$ is small in comparison to the overall sum.
\end{proof}

We now bound internal incidences. Just as in the distinct distances problem, we must separate incidences on non-ruled surfaces from incidences on ruled surfaces, and further separate analysis on planes and quadrics from analysis on the other ruled surfaces.

\begin{lemma}
    If $S_i$ is a plane, $I(L_i) \leq 0.5c \cdot |L|^{1/2} |L_i|$.
\end{lemma}
\begin{proof}
    By assumption, $|L_i| \leq c |L|^{1/2}$. Since we are working in a projective context, any two lines in $L_i$ intersect one another. The number of incidences is maximized if any pair of lines intersects in a distinct location. Thus we find $I(L_i) \leq |L_i|^2/2$.
\end{proof}

\begin{lemma}
    If $S_i$ is a singular quadric (a cone), $I(L_i) = |L_i| - 1$.
\end{lemma}
\begin{proof}
    A singular quadric has a vertex, and any straight line on the surface passes through the vertex. This gives the result directly.
\end{proof}

\begin{lemma}
    If $S_i$ is a smooth quadric, $I(L_i) \leq 0.5c \cdot |L_i| |L|^{1/2}$.
\end{lemma}
\begin{proof}
    Since $S_i$ is a smooth quadric, the family of lines on $S$ splits into two rulings. Write $L_i = L_i^0 \cup L_i^1$, where $L_i^0$ is the first ruling, and $L_i^1$ the second ruling, and no lines intersect in the same family. Thus
    %
    \[ I(L_i) \leq |L_i^0| |L_i^1| \leq |L_i|^2/4 \leq 0.5c \cdot |L|^{1/2}. \qedhere \]
\end{proof}

We say a ruled surface is `non-exception' if it isn't a plane, cone, or quadric.

\begin{lemma}
    If $S_i$ is a `non-exceptional' ruled surface, then
    %
    \[ I(L_i) \leq 0.5 \cdot |L_i| \deg(S_i) + 2 |L_i|. \]
\end{lemma}
\begin{proof}
    If $S_i$ is a non-special ruled surface, then there are at most two disjoint lines on $S_i$, disjoint from each other, which intersect more than $\deg(S_i)$ lines, known as special. If we set $L_i = L_i^0 \cup L_i^1$, where $L_i^0$ are the special lines, then
    %
    \[ I(L_i) = I(L_i^0) + I(L_i^1) + |L_i^0 \cap L_i^1| \leq 0 + 0.5 \cdot \deg(S_i) \cdot |L_i^1| + 2|L_i^1|.\qedhere \]
%        &\leq 0.5 \cdot |L_i| \deg(S_i) + 2 |L_i|. \qedhere
\end{proof}

Thus
%
\begin{align*}
    \sum_{S_i\ \text{ruled}} I(L_i) &= \left( \sum_{\text{planes}} + \sum_{\text{smooth quadric}} \right) + \left( \sum_{\text{cones}} + \sum_{\text{non exceptional}} \right)\\
    &\leq \left( \sum 0.5c \cdot |L|^{1/2} |L_i| \right) + \left( 0.5 \cdot \sum |L_i| \deg(S_i) \right)\\
    &\leq \left( 0.5c \cdot |L|^{3/2} \right) + \left( 0.5 d |L| \right)\\
    &\leq 0.5c \cdot |L|^{3/2} + 1.23 \cdot |L|^{3/2}
\end{align*}
%
All these bounds are essentially as expected from the Guth-Katz result.

\end{document}