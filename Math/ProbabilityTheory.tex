\documentclass[12pt, dvipsnames]{report}

\usepackage{amsmath}
\usepackage{algorithm}
%\usepackage{algorithmic}
\usepackage[noend]{algpseudocode}

\usepackage{amsmath}
\usepackage{amssymb}
\usepackage{amsthm}
\usepackage{amsopn}

\usepackage{kpfonts}

\usepackage{graphicx}

% Probably don't need this on notes anymore
%\usepackage{kbordermatrix}

% Standard tool for drawing diagrams.
\usepackage{tikz}
\usepackage{tkz-berge}
\usepackage{tikz-cd}
\usepackage{tkz-graph}

\usepackage{comment}

%
\usepackage{multicol}

%
\usepackage{framed}

%
\usepackage{mathtools}

%
\usepackage{float}

%
\usepackage{subfig}

%
\usepackage{wrapfig}

%
\let\savewideparen\wideparen
\let\wideparen\relax
\usepackage{mathabx}
\let\wideparen\savewideparen

% Used for generating `enlightening quotes'
\usepackage{epigraph}

% Forget what this is used for :P
\usepackage[utf8]{inputenc}

% Used for generating quotes.
\usepackage{csquotes}

% Allows what to generate links inside
% generated pdf files
\usepackage{hyperref}

% Allows one to customize theorem
% environments in mathematical proofs.
\usepackage{thmtools}

% Gives access to a proof
\usepackage{lplfitch}

% I forget what this is for.
\usepackage{accents}

% A package for drawing simple trees,
% as a substitute for unnesacary TIKZ code
\usepackage{qtree}

% Enables sequent calculus proofs
\usepackage{ebproof}

% For braket notation
\usepackage{braket}

% To change line spacing when using mathematical notations which require some height!
\usepackage{setspace}

%\usepackage[dvipsnames]{xcolor}

\usepackage{float}

% For block commenting
\usepackage{comment}




\setlength\epigraphwidth{8cm}

\usetikzlibrary{arrows, petri, topaths, decorations.markings}

% So you can do calculations in coordinate specifications
\usetikzlibrary{calc}
\usetikzlibrary{angles}

\theoremstyle{plain}
\newtheorem{theorem}{Theorem}[chapter]
\newtheorem{axiom}{Axiom}
\newtheorem{lemma}[theorem]{Lemma}
\newtheorem{corollary}[theorem]{Corollary}
\newtheorem{prop}[theorem]{Proposition}
\newtheorem{exercise}{Exercise}[chapter]
\newtheorem{fact}{Fact}[chapter]

\newtheorem*{example}{Example}
\newtheorem*{proof*}{Proof}

\theoremstyle{remark}
\newtheorem*{exposition}{Exposition}
\newtheorem*{remark}{Remark}
\newtheorem*{remarks}{Remarks}

\theoremstyle{definition}
\newtheorem*{defi}{Definition}

\usepackage{hyperref}
\hypersetup{
    colorlinks = true,
    linkcolor = black,
}

\usepackage{textgreek}

\makeatletter
\renewcommand*\env@matrix[1][*\c@MaxMatrixCols c]{%
  \hskip -\arraycolsep
  \let\@ifnextchar\new@ifnextchar
  \array{#1}}
\makeatother

\renewcommand*\contentsname{\hfill Table Of Contents \hfill}

\newcommand{\optionalsection}[1]{\section[* #1]{(Important) #1}}
\newcommand{\deriv}[3]{\left. \frac{\partial #1}{\partial #2} \right|_{#3}} % partial derivative involving numerator and denominator.
\newcommand{\lcm}{\operatorname{lcm}}
\newcommand{\im}{\operatorname{im}}
\newcommand{\bint}{\mathbf{Z}}
\newcommand{\gen}[1]{\langle #1 \rangle}

\newcommand{\End}{\operatorname{End}}
\newcommand{\Mor}{\operatorname{Mor}}
\newcommand{\Id}{\operatorname{id}}
\newcommand{\visspace}{\text{\textvisiblespace}}
\newcommand{\Gal}{\text{Gal}}

\newcommand{\xor}{\oplus}
\newcommand{\ft}{\wedge}
\newcommand{\ift}{\vee}

\newcommand{\prob}{\mathbf{P}}
\newcommand{\expect}{\mathbf{E}}
\DeclareMathOperator{\Var}{\mathbf{V}}
\newcommand{\Ber}{\text{Ber}}
\newcommand{\Bin}{\text{Bin}}

%\newcommand{\widecheck}[1]{{#1}^{\ft}}

\DeclareMathOperator{\diam}{\text{diam}}

\DeclareMathOperator{\QQ}{\mathbf{Q}}
\DeclareMathOperator{\ZZ}{\mathbf{Z}}
\DeclareMathOperator{\RR}{\mathbf{R}}
\DeclareMathOperator{\HH}{\mathbf{H}}
\DeclareMathOperator{\CC}{\mathbf{C}}
\DeclareMathOperator{\AB}{\mathbf{A}}
\DeclareMathOperator{\PP}{\mathbf{P}}
\DeclareMathOperator{\MM}{\mathbf{M}}
\DeclareMathOperator{\VV}{\mathbf{V}}
\DeclareMathOperator{\TT}{\mathbf{T}}
\DeclareMathOperator{\LL}{\mathcal{L}}
\DeclareMathOperator{\EE}{\mathbf{E}}
\DeclareMathOperator{\NN}{\mathbf{N}}
\DeclareMathOperator{\DQ}{\mathcal{Q}}
\DeclareMathOperator{\IA}{\mathfrak{a}}
\DeclareMathOperator{\IB}{\mathfrak{b}}
\DeclareMathOperator{\IC}{\mathfrak{c}}
\DeclareMathOperator{\IP}{\mathfrak{p}}
\DeclareMathOperator{\IQ}{\mathfrak{q}}
\DeclareMathOperator{\IM}{\mathfrak{m}}
\DeclareMathOperator{\IN}{\mathfrak{n}}
\DeclareMathOperator{\IK}{\mathfrak{k}}
\DeclareMathOperator{\ord}{\text{ord}}
\DeclareMathOperator{\Ker}{\textsf{Ker}}
\DeclareMathOperator{\Coker}{\textsf{Coker}}
\DeclareMathOperator{\emphcoker}{\emph{coker}}
\DeclareMathOperator{\pp}{\partial}
\DeclareMathOperator{\tr}{\text{tr}}

\DeclareMathOperator{\supp}{\text{supp}}

\DeclareMathOperator{\codim}{\text{codim}}

\DeclareMathOperator{\minkdim}{\dim_{\mathbf{M}}}
\DeclareMathOperator{\hausdim}{\dim_{\mathbf{H}}}
\DeclareMathOperator{\lowminkdim}{\underline{\dim}_{\mathbf{M}}}
\DeclareMathOperator{\upminkdim}{\overline{\dim}_{\mathbf{M}}}
\DeclareMathOperator{\lhdim}{\underline{\dim}_{\mathbf{M}}}
\DeclareMathOperator{\lmbdim}{\underline{\dim}_{\mathbf{MB}}}
\DeclareMathOperator{\packdim}{\text{dim}_{\mathbf{P}}}
\DeclareMathOperator{\fordim}{\dim_{\mathbf{F}}}

\DeclareMathOperator*{\argmax}{arg\,max}
\DeclareMathOperator*{\argmin}{arg\,min}

\DeclareMathOperator{\ssm}{\smallsetminus}

\DeclareMathOperator{\Prob}{\mathbf{P}}
\DeclareMathOperator{\Ber}{\text{Ber}}
\DeclareMathOperator{\Bin}{\text{Bin}}
\DeclareMathOperator{\Geo}{\text{Geo}}
\DeclareMathOperator{\Poisson}{\text{Poisson}}
\DeclareMathOperator{\Uni}{\text{Uni}}
\DeclareMathOperator{\Multi}{\text{Multi}}
\DeclareMathOperator{\Expect}{\mathbf{E}}
\DeclareMathOperator{\Exp}{\text{Exp}}
\DeclareMathOperator{\Beta}{\text{Beta}}

\title{Probability Theory}
\author{Jacob Denson}

\begin{document}

\pagenumbering{gobble}
\maketitle
\tableofcontents
\pagenumbering{arabic}

\chapter{Foundations}

These notes outline the basics of probability theory, the framework which allows us to communicate that we are 80\% more likely to develop lung disease if you are a smoker rather than an average person, or that there is a 50/50 chance of rain today? But what is probability? What do we mean by these probabilistic statements? To a mathematicians, there need not be a rigorous interpretation -- probability theory is just a subfield of measure theory. To a natural scientist, probability theory is viewed in a different vein. In this chapter, we will explore the two major interpretations of probability theory in real life, each of which use the same underlying mathematical theory to make judgements about the world. After exploring these interpretations, we will make axiomatic definitions of probability (which hold regardless of which interpretation you have), and explore the basic consequences of the assumptions.

\section{Frequentist Probability}

Classical probability theory was developed according to the intuitions of what is now known as the frequentist school of probability theory, and is the simplest interpretation of probability to understand. It is easiest to understand from the point of view of the scientific experiment. Suppose you are repeatedly performing some event, in a manner which is controlled well enough that the outcome of the experiment should be the same. Even under rigorously controlled conditions however, the experiment will not always result in the same outcome -- we instead have a range of outcomes which we may observe from a single result in an experiment. Nonetheless, some outcomes will occur more frequently than others. Let us perform an experiment as often as desired, obtaining an infinite sequence of outcomes
%
\[ \omega_1, \omega_2, \omega_3, \dots \]
%
Let $D$ be a proposition decidable from the outcome of the experiment (for instance $D$ may represent whether a flipped coin lands heads up or heads down in the experiment of flipping a coin). Mathematically, a proposition is just a subset of the set of all outcomes in an experiment -- the outcomes for which the proposition is true. We may then define the relative frequency of this proposition being true in $n$ trials to be
%
\[ P_n(D) := \frac{\# \{ k \leq n : \omega_k \in D \}}{n} \]
%
The key assumption of the frequentist school is that, if our experiments are suitably controlled, then regardless of the specific sequence of measured outcomes, our relative frequencies will always converge to a well defined invariant ratio, which we define to be the probability of a certain event:
%
\[ \Prob(D) := \lim_{n \to \infty} P_n(D) \]
%
Let's explore some consequences of this doctrine. First, $0 \leq P_n(D) \leq 1$ is true for any $n$, so that $0 \leq \mathbf{P}(D) \leq 1$. If we let $\Omega$ denote the set of all possible outcomes to the experiment (a proposition true for all outcomes of the experiment), then
%
\[ P_n(D) = \frac{\# \{ k \leq n : \omega_k \in \Omega\}}{n} = \frac{\# \{ 1, 2, \dots, n \}}{n} = 1 \]
%
Thus $\Prob(\Omega) = 1$. If $A_1, A_2, \dots$ is a sequence of disjoint propositions (no more than one outcome is true for each outcome of the experiment), then
%
\[ P_n \left( \bigcup_i A_i \right) = \frac{\# \{ k \leq n : \omega_k \in \bigcup A_i \}}{n} = \frac{\sum_i \# \{ k \leq n : \omega_k \in A_i \}}{n} = \sum_i P_n(A_i) \]
%
Hence, $\Prob(\bigcup_i A_i) = \sum_i \Prob(A_i)$. This will be true for an arbitrary family of disjoint propositions, provided we interpret the sum of the propositions as the supremum of all finite sums. There is no real generality here, because only countably many disjoint propositions can be true in the sequence of experimental outcomes (for only one proposition can be true for each of the experiments), hence the probability of only countably many propositions is nonzero.

\section{Bayesian Probability}

The frequentist school is sufficient to use probability theory to model scientific experiments, but our own use of probability is much more general. If you turn on the news, it's common to hear that ``there is an 80\% chance of downpour this evening''. It is difficult to interpret this as a frequentist. Even if we see each night's temperament as an experimental trial, it is hard to convince yourself that these experiments are controlled enough to converge to a probabilistic result. The Bayesian school of probability redefines probability theory to be attuned to a person's individual belief.

One problem with the Bayesian interpretation of probability theory is that there is no way to go out into the world and `learn probabilities' -- it's all based on a person's individual interpretation. The only constraint we have on the choice of probabilities is that they are `consistant'. Consistancy can be formulated in various ways, but my favourite is the Dutch book method, developed by the Italian probabilist Bruno de Finetti; if you assign to $D$ a probability $\Prob(D)$, then you are willing to play the following game: If $D$ does not occur, you lose $\Prob(D)$ dollars, but if $D$ occurs, you win $1 - \Prob(D)$ dollars. You {\it must} also be willing to play the game where you lose $1-\Prob(D)$ dollars if $D$ occurs, and gain $\Prob(D)$ dollars if $D$ odes not occur, so that you think the bets are `fair' to both sides. A person's probability function is inconsistant if it possible to make a series of bets that will guarantee a profit regardless of the outcome: a Dutch book.

Here's an example of how the Dutch book method can be employed to obtain general rules of probability. We claim that for any $D$, $0 \leq \Prob(D) \leq 1$. If a person believed that $\Prob(D) < 0$, then I could make a bet that person that $D$ occured, and I would make money regardless of the outcome. Similar results occur from betting against $D$ if $\Prob(D) > 1$. It can be shown, via similar arguments, that the probability of the certain event is one, and if $\{ A_i \}$ is a countable collection of disjoint events, then $\Prob(\bigcup_i A_i) = \sum_i \Prob(A_i)$ (Definetti would have only accepted this statement for finite collections of events. Here, we allow one to make a countable number of bets at once, rather than only finitely many at any point - Allowing limit operations is too useful to ignore!)

What we have shown is that consistant degrees of belief in the Bayesian system have similar properties of experimental frequencies to a frequentist. Regardless of which philosophy you agree with, you will eventually have to agree on the same fundamental principles of probability theory. Neither of these systems rest of mathematical foundations, so we need to make a rigorous model, from which we can avoid the philosophical controversies that arise. Just as the game of chess does not have to be about knights and castles, the game of probability theory does not have to be about frequencies nor degrees of belief, but can be played from the basic assumptions which define the theory. Note, however, that the interpretations of probabilities have significant effects on how one performs statistical inference.

\section{Axioms of Probability}

Mathematically rigorous probability theory is defined under the banner of measure theory. The framework enables us to avoid some paradoxes which can be found if we aren't careful when analyzing experiments with infinitely many outcomes. Note, however, that probability theorists study {\it concepts} that are true in these measure spaces, and the framework provides the formality to understand these concepts. A {\bf probability space} is a measure space $X$ with a positive measure $\mu$ such that $\mu(X) = 1$. $X$ is known as the {\bf sample space}, and $\mu$ is known as the {\bf probability distribution} or {\bf probability measure}. We interpret $X$ as the space of outcomes to some random phenomena, and $\mu$ measuring the likelyhood of each outcome happening. An arbitrary probability measure $\mu$ is often denoted $\Prob$, since we often only need to talk about one distribution (or the distributions are defined such that we can always determine which measure we are talking about with the notation $\Prob$), but this is not the only notation!

If $X$ is countable, then a (complete) probability measure can be viewed a vector $v \geq 0$ in $\mathbf{R} \cdot X$ such that $\sum_{x \in X} v(x) = 1$ -- the $\sigma$ algebra of the measure space plays no real role in the theory.

\begin{example}
    Suppose we flip a coin. There is a certain chance of flipping a heads, or flipping a tails. Since the coin is essentially symmetric, we should expect that the chance of a heads is as equally likely as a chance of tails. We can encode the set of outcomes in the sample space $\{ H, T \}$, and then model the probability distribution as $\mathbf{P}(H) = \mathbf{P}(T) = 1/2$. More generally, if we have a finite sample space $S$, we can put a distribution on $S$ which considers all points equally known as the {\bf uniform distribution}, with distribution $\mathbf{P}(s) = 1/|S|$.
\end{example}

\begin{example}
    If $x \in X$ is fixed, the {\bf point mass distribution} $\delta_x$ at $x$ is the probability distribution defined by
    %
    \[ \delta_x(A) = \begin{cases} 1 & x \in A \\ 0 & x \not \in A \end{cases} \]
    %
    The distribution represents an event where an outcome is certain to occur.
\end{example}

The first immediately obvious fact from the axioms of a probability space $X$ is that $\Prob(A^c) = 1 - \Prob(A)$, since $A$ and $A^c$ are disjoint events whose union is the whole space $X$. A similar discussion shows that
%
\[ \Prob(A \cup B) = \Prob(A) + \Prob(B) - \Prob(AB) \]
%
because $A \cup B$ can be written as the union of the three disjoint events $AB$, $AB^c$, and $A^cB$, and
%
\[ \Prob(A) = \Prob(AB) + \Prob(AB^c)\ \ \ \ \ \Prob(B) = \Prob(AB) \cup \Prob(A^c B) \]
%
This process can be generalized to unions of finitely many events. We have
%
\[ \Prob(A \cup B \cup C) = \Prob(A) + \Prob(B) + \Prob(C) - \Prob(AB) - \Prob(AC) - \Prob(BC) + \Prob(ABC) \]
%
which can be reasoned by looking at how many times elements of $A \cup B \cup C$ are `counted' on the right hand side. In general, we have
%
\[ \Prob \left( \bigcup_{k = 1}^n A_k \right) = \sum_{S \subset \{ 1, \dots, k \}} (-1)^{|S|} A_S \]
%
where $A_S$ is the intersection of the $A_k$, with $k \in S$. This can be proven by a clumsy inductive calculation which we leave to the reader. Use of the formula is known as the inclusion-exclusion principle, because you are counting how many things occur in the inclusion of a set by counting certain exclusions.

But we aren't normally limited to taking finite unions. We often want to calculate the probability of an infinite union of sets $A_i$. Now the monotone convergence theorem combined with the inclusion-exclusion principle implies that
%
\[ \Prob \left(\bigcup_{i = 1}^\infty A_i \right) = \lim_{n \to \infty} \Prob \left( \bigcup_{i = 1}^n A_i \right) = \sum_{\substack{S \subset \mathbf{N}\\|S| < \infty}} (-1)^{|S|} \Prob(A_S) \]
%
where the sum on the right is taken as the limit of the partial sums where $S \subset \{ 1, \dots, n \}$ (the sum rarely convergences absolutely, so this is very important point). The inclusion-exclusion formula can be tricky to calculate in real examples, so we often rely on estimates. The trivial {\bf union bound}
%
\[ \Prob \left( \bigcup A_i \right) \leq \sum \Prob(A_i) \]
%
can be applied, and is tight provided that the $A_i$ are `nearly disjoint' (for instance, the bound is shockingly bad if all $A_i$ are equal).

Another useful fact is to notice that $\Prob(A_i) \to \Prob(A)$ if the sets $A_i$ `tend to' $A$ in some form of another. If the $A_i$ are an increasing sequence, whose union if $A$, then we can certainly conclude $\Prob(A_i) \to \Prob(A)$. Similarily, if $A_i$ is a decreasing sequence whose intersection is $A$, then $\Prob(A_i) \to \Prob(A)$. To obtain general results, we say that $A_i \to A$ if $\limsup A_i = \liminf A_i = A$, where
%
\[ \limsup A_i = \bigcap_{i = 1}^\infty \bigcup_{j \geq i} A_j\ \ \ \ \ \liminf A_i = \bigcup_{i = 1}^\infty \bigcap_{j \geq i} A_j \]
%
we then conclude that $\Prob(A_i) \to \Prob(A)$. To see this, note that the sets $\bigcup_{j \geq i} A_j$ form a decreasing subsequence, and the $\bigcap_{j \geq i} A_j$ an increasing subsequence. Thus
%
\[ \Prob(A) = \Prob(\limsup A_i) = \lim_{n \to \infty} \Prob \left(\bigcup_{i \geq n} A_i \right) \geq \lim_{n \to \infty} \Prob(A_i) \]
\[ \Prob(A) = \Prob(\liminf A_i) = \lim_{n \to \infty} \Prob \left( \bigcap_{i \geq n} A_i \right) \leq \lim_{n \to \infty} \Prob(A_i) \]
%
so the squeeze theorem applies.

\section{Conditional Probabilities}

In the Bayesian interpretation of probability theory, it is natural for probabilities to change over time as more information is gained about the system in question. That is, given that we know some proposition $B$ holds over the sample space, we obtain a new probability distribution over $X$, denoted $\Prob(D | B)$, which represents the ratio of winnings from the bet which is only played out if $B$ occurs. That is
%
\begin{itemize}
    \item You win $1-\Prob(D|B)$ dollars if $D$ occurs, and $B$ occurs.
    \item You lose $\Prob(D|B)$ dollars if $D$ does not occur, and $B$ occurs.
    \item You do not lose or win money if $B$ does not occur.
\end{itemize}
%
It then follows by a dutch bet argument that
%
\[ \Prob(B) \Prob(D | B) = \Prob(B \cap D) \]
%
Suppose instead  that $\Prob(B) \Prob(D|B) < \Prob(B \cap D)$. Bet on $B$ occuring, and also bet against $B \cap D$ occuring. Then
%
\begin{itemize}
    \item If $B$ does not occur, we gain $\Prob(B \cap D)$ dollars.
    \item If $B \cap D$ occurs, we lose $1 - \Prob(B \cap D)$ dollars, and gain $1 - \Prob(B) \Prob(D | B)$ dollars.
    \item If $B \cap D^c$ occurs, we gain $\Prob(B \cap D)$ dollars, and lose $\Prob(B) \Prob(D \ B)$ dollars.
\end{itemize}
%
The inequality guarantees that we always make a profit on these bets. Similarily results happen if we assume the opposite inequality, so we must have equality.

In the empirical interpretation, $\Prob(D | B)$ is the ratio of times that $D$ is true in experiments, where $B$ also occurs. That is, we define $\Prob(D | B)$ as the limit of the ratios
%
\[ \frac{\# \{ k \leq n : \omega_k \in B \}}{n} \frac{\# \{ k \leq n : \omega_k \in D, \omega_k \in B \}}{\# \{ k \leq n : \omega_k \in B \}} = \frac{\# \{ k \leq n : \omega_k \in D, \omega_k \in B \}}{n} \]
%
which gives us the formula $\Prob(B) \Prob(D | B) = \Prob(D \cap B)$. We must of course assume that $\Prob(B) \geq 0$, since overwise we are almost certain that $B$ will never occur, and we can therefore define $\Prob(D | B)$ arbitrarily (or not define it at all).

Thus we have motivation to define conditional probabilities by the formula $\Prob(B) \Prob(D | B) = \Prob(D \cap B)$. It enables us to model the information gained by restricting our knowledge to a particular subset of sample space. In particular, we can use the definition to identify sets which do not give us any information about a particular event. We say two events $D$ and $B$ are independant, denoted $D \amalg B$, if $\Prob(D | B) = \Prob(D)$, or equivalently, if $\Prob(D \cap B) = \Prob(D) \Prob(B)$; knowledge of $B$ gives us no foothold over knowledge of the likelihood of $D$.

\begin{example}
    The Monty Hall problem is an incredible example of how paradoxical probability theory can seem. We are on a gameshow. Suppose there are three doors in front of you. A (brand new) car is placed uniformly randomly behind one of the doors. After we pick a door, the gameshow host then randomly opens one of the other doors which you didn't pick, revealing the car isn't behind the door (his intention). What is the chance that the door you picked has the brand new car? It we pick door $i$, then certainly
    %
    \[ \mathbf{P}(i\ \text{has a car}) = 1/3 \]
    %
    Yet this implies, by symmetry, that for any $j \in \{ 2, 3 \}$,
    %
    \begin{align*}
        1/3 &= \sum_{j = 1}^3 \mathbf{P}(i\ \text{has a car} | \text{door $j$ is opened}) \mathbf{P}(\text{door $j$ is opened})\\
        &= 2(1/6 + 1/3) \mathbf{P}(i\ \text{has a car} | \text{door $j$ is opened})
    \end{align*}
    %
    Dividing out, the probability that our door is correct is $1/3$, so we should definitely switch to obtain our best chance of success.

    The argument above causes a great media uproar when it was published in 1990 in a popular magazine, because of how convincing the fallacious argument below is. The sample space of the problem can be described by the tuple $(j,k)$, where $j$ is where the car is, and $k$ is a door we open. We can explicitly enumerate the sample space as
    %
    \[ (1,2),(1,3),(2,3),(3,2) \]
    %
    and the car is seen to be in our door half of the time.
\end{example}

We end this chapter with a final probability rule which is important in statistical analysis. If $B$ is partitioned into a finite sequence of disjoint events $A_1, \dots, A_n$, then we have the formula $\Prob(B) = \sum_i \Prob(B | A_i) \Prob(A_i)$. This easily gives us Bayes rule
%
\[ \Prob(A_j | B) = \frac{\Prob(B | A_j)}{\sum_i \Prob(B | A_i) \Prob(A_i)} \]
%
If we view $A_j$ as a particular hypothesis from the set of all hypotheses, and $B$ as some obtained data, then Bayes rule enables us to compute the probability that $A_j$ is the true hypothesis from the probability that $B$ is the data generated given the hypothesis is true. This is incredibly important if you can interpret these probabilities correctly (if you are a Bayesian), but not so useful if you are an empiricist (in which case we assume there is a `true' result we are attempting to estimate from trials, so there is no probability distribution over the correctness hypothesis, other than perhaps a point mass, in which case Bayes rule gives us no information). We reiterate that Bayes rule is a theorem of probability theory, so is true in any interpretation, but can be used by Bayesians in a much more applicable way to their statistical analysis.

\section{Kolmogorov's Zero-One Law}

s

\chapter{Random Variables}

The formality of probability theory is ironic, because even though we require the theory of measures and real analysis to place the foundations of the theory, in the probabilistic way of thinking we try to eschew as much of this foundation as possible; studying properties of random variables which aren't `independent' of the sample space considered is avoided. As a rough approximation, if $T: X \to Y$ is a surjective measure preserving map between probability spaces ($X$ is an extension of the space $Y$, allowing more outcomes), then the random variable $Y \circ T$ is considered the `same' as the random variable $Y$, and the concepts studied in probability theory should be preserved under this extension. As we reach further and further into statistical theory, samples spaces will soon become a near distant memory.

The irony of introducing the sample space is unfortunate, because while the space is in the background, in the probabilistic way of thinking about problems we try and eschew the sample space as much as possible.

\section{Expectation}

\begin{theorem}
    For any $X \geq 0$,
    %
    \[ \mathbf{E}[X] = \int \mathbf{P}(X \geq x) dx \]
\end{theorem}
\begin{proof}
    Applying Fubini's theorem,
    %
    \begin{align*}
        \int_0^\infty \mathbf{P}(X \geq x) dx &= \int_0^\infty \int_x^\infty d \mathbf{P}_*(y)\ dx\\
        &= \int_0^\infty \int_0^y dx\ d\mathbf{P}_*(y)\\
        &= \int_0^\infty y d\mathbf{P}_*(y) = \mathbf{E}[X]
    \end{align*}
\end{proof}

\chapter{Inequalities}

It is often to calculate explicitly the probability values of a certain random variable, but it often suffices to bound the values, especially when discussing the convergence of certain variables.

The most important inequality bounds the chance that a probability will deviate from the mean. if $X$ has mean $\mu < \infty$, then the Lebesgue integral calculates
%
\[ \mu = \int X d\mathbf{P} \]
%
as the supremum of step functions. In particular, if we take the step function $x \mathbf{I}(X \geq x) \leq X$, then we find

\begin{theorem}[Markov's Inequality]
    If $X$ has finite mean $\mu$, then
    %
    \[ \Prob(X \geq x) \leq \frac{\mathbf{E}(X)}{x} \]
\end{theorem}

The bound is trivial, and is therefore very rough. Nonetheless, it suffices for many purposes. One can obtain better estimates by taking a more detailed step function bounded by $X$, but the payoff isn't normally that great. We obtain a somewhat sharper estimate if $X$ has a finite variance $\sigma$.

\begin{theorem}[Chebyshev's Inequality]
    If $X$ has mean $\mu$ and variance $\sigma^2$, then
    %
    \[ \Prob(|X - \mu| \geq x) \leq \frac{\sigma^2}{x^2} \]
    %
    If $Z = (X - \mu)/\sigma$, then
    %
    \[ \Prob(|Z| \geq x) \leq \frac{1}{x^2} \]
\end{theorem}
\begin{proof}
    Applying Markov's inequality, we find
    %
    \[ \Prob(|X - \mu| \geq x) = \Prob(|X - \mu|^2 \geq x^2) \leq \frac{\mathbf{E}|X - \mu|^2}{x^2} = \frac{\sigma^2}{x^2} \]
    %
    We obtain the inequality for $Z$ by carrying out coefficents and applying Chebyshev's inequality.
\end{proof}

We can continue this process. When $X$ has an $n$'th moment, then
%
\[ \Prob(|X - \mu| \geq x) \leq \frac{\mathbf{E}|X - \mu|^n}{x^n} \]
%
which shows that the existence of moments guarantees the decay of $X$. It is often difficult to calculate high degree moments, however, so this inequality does not occur as often.

\begin{example}
    Let $X_1, \dots, X_n \sim \Ber(p)$ by independant and identically distribution, where $p$ is an unknown value. A good way to estimate $p$ is via the random variable
    %
    \[ \widehat{p} = \frac{X_1 + \dots + X_n}{n} \]
    %
    which has a binomial distribution. We measure the utility of $\widehat{p}$ by minimizing the probability that $\widehat{p}$ deviates far from the mean. That is, $\Prob(|\widehat{p} - p| \geq x)$ is small for large values of $x$. We find $\widehat{p}$ has mean $p$ and variance $p(1-p)/n$, so we may apply Chebyshev's inequality to conclude
    %
    \[ \Prob(|\widehat{p} - p| \geq x) \leq \frac{p(1-p)}{nx^2} \leq \frac{1}{4nx^2} \]
    %
    so even for the worst possible choice of $p$, we still obtain inversely linear decay; not great, but still enough to guarantee that $\widehat{p}$ converges in distribution to the point mass measure at $p$ as $n \to \infty$. This is the weak law of large numbers for the Bernoulli distribution.
\end{example}

Measure theory gives us general bounds, which are just special results of more general inequalities. We have the Cauchy Schwarz inequality, which says $\Expect(XY) \leq \sqrt{\Expect(X^2) \Expect(Y^2)}$, and Jensen's inequality, which says that is $f$ is convex, then $f(\Expect(X)) \leq \Expect(f(X))$. If $f$ is concave, $f(\Expect(X)) \geq \Expect(f(X))$. In particular, Jensen's inequality shows
%
\[ \Expect(X^2) \geq [\Expect(X)]^2\ \ \ \ \ \Expect(1/X) \geq 1/\Expect(X)\ \ \ \ \ \Expect(\log x) \leq \log \Expect(X) \]
%
which is used in the more advanced theory to obtain deeper inequalities.

Hoeffding's inequality is similar to Markov's inequality, but is generally much sharper. It therefore has a more complicated formula.

\begin{theorem}[Hoeffding's Inequality]
    Let $X_1, \dots, X_n$ be centrally distributed i.i.d random variables, with $a_i \leq X_i \leq b_i$, then for any $t > 0$,
    %
    \[ \Prob \left( \sum X_i \geq x \right) \leq e^{-tx} \prod_{i = 1}^n e^{t^2 (b_i - a_i)^2/8} \]
\end{theorem}
\begin{proof}
    For any $t > 0$, Markov's inequality implies
    %
    \begin{align*}
        \Prob \left( \sum X_i \geq x \right) &= \Prob \left( t \sum X_i \geq tx \right)\\
        &= \Prob \left( e^{t \sum X_i} \geq e^{tx} \right)\\
        &\leq e^{-tx} \prod \mathbf{E}[e^{t X_i}]
    \end{align*}
    %
    We can write
    %
    \[ X_i = \Lambda a_i + (1 - \Lambda) b_i \]
    %
    for some function $0 \leq \Lambda \leq 1$, and by applying the convexity of the exponential function, we find
    %
    \[ e^{tX_i} \leq \Lambda e^{t a_i} + (1 - \Lambda) e^{t b_i} \]
    %
    Hence
    %
    \[ \Expect(e^{tX_i}) \leq \Expect(\Lambda) e^{t a_i} + (1 - \Expect(\Lambda)) e^{t b_i} \]
    %
    Now we may explicitly calculate $\Lambda = (X_i - a_i)/(b_i - a_i)$, so that
    %
    \[ \Expect(e^{tX_i}) \leq \frac{a_i}{a_i - b_i} e^{t a_i} + \frac{b_i}{b_i - a_i} e^{t b_i} = e^{F(t(b_i - a_i))} \]
    %
    Where $F(x) = - \lambda x + \log(1 - \lambda + \lambda e^x)$, where $\lambda = a_i/(a_i - b_i)$. Note that $F(0) = F'(0) = 0$, and $F''(x) \leq 1/4$ for $x > 0$, so that by Taylor's theorem, there is $y \in (0,x)$ such that
    %
    \[ F(x) = \frac{x^2}{2} g''(y) \leq \frac{x^2}{8} \]
    %
    Hence $\Expect(e^{tX_i}) \leq e^{t^2(b_i - a_i)^2/8}$, and this completes the proof.
\end{proof}

\begin{example}
    If $\widehat{p} \sim \Bin(n,p)$, and we take $X_i = (\widehat{p} - p)/n$, then $\mathbf{E}(X_i) = 0$, and $-p/n \leq X_i \leq 1/n-p/n$, and since $\sum X_i = \widehat{p} - p$, we find
    %
    \[ \Prob(\widehat{p} - p \geq x) \leq e^{t^2/8n - tx} \]
    %
    For $t = 4nx$, we find $\Prob(\widehat{p} - p \geq x) \leq e^{-2nx^2}$. By symmetry, we can calculate the absolute deviance as $\Prob(|\widehat{p} - p| \geq x) \leq 2e^{-2nx^2}$. This gives us a much sharper rate of convergence than our last result.
\end{example}

\section{Subgaussian Random Variables}

Hoeffding's inequality only applies to bounded random variables. In the general case, we can't apply the inequality (which relies on the bounded intervals to use convexity), and Chebyshev's inequality often does not suffice. We should still obtain fast tail decay in most circumstances, say, for instance a Gaussian distribution with variance $\sigma^2$. Calculating, we find
%
\begin{align*}
    \mathbf{P}(X - \mu \geq y) &= \int_{y}^\infty \frac{1}{\sqrt{2 \pi \sigma^2}} e^{-\frac{x^2}{2 \sigma^2}} dx\\
    &\leq \frac{1}{y \sqrt{2 \pi \sigma^2}} \int_{y}^\infty x e^{- \frac{x^2}{2\sigma^2}} dx\\
    &= \frac{\sigma e^{- \frac{y^2}{2 \sigma^2}}}{y \sqrt{2 \pi}}
\end{align*}
%
This quantity is almost always better than Chebyshev's inequality, since the ratio $x/y$, which measures the inaccuracy of our inequality, is nullified by the exponential function. We can find similar equalities for random variables which are `bounded' by normal distributions.

We shall say a random variable $X$ is $\sigma^2$-{\bf subgaussian} if for all $\lambda \in \mathbf{R}$,
%
\[ \mathbf{E}[e^{\lambda X}] \leq e^{\lambda^2 \sigma^2/2} \]
%
where we assume $e^{\lambda X}$ is integrable for all $\lambda$. Pointwise, we have
%
\[ e^{\lambda X} = \sum_{k = 0}^\infty \frac{\lambda^k X^k}{k!} \leq \sum_{k = 0}^\infty \frac{\lambda^{2k} \sigma^{2k}}{2^k k!} = e^{\lambda^2 \sigma^2/2} \]
%
since $e^{|\lambda| |X|}$ is integrable ($|X| = X^+ + X^-$, and the Cauchy Schwarz equality implies)
%
\[ \mathbf{E}[e^{|\lambda| X^+} e^{|\lambda| X^-}]^2 \leq \mathbf{E}[e^{2 |\lambda| X^+}] \mathbf{E}[e^{2 \lambda X^-}] < \infty \]
%
Thus we may apply the dominated convergence theorem to conclude
%
\[ \mathbf{E}[e^{\lambda X}] = \sum_{k = 0}^\infty \frac{\lambda^k}{k!} \mathbf{E}[X^k] \]
%
and for any $\lambda$,
%
\[ \sum_{k = 0}^\infty \frac{\mu^k}{k!} \mathbf{E}[X^k] \leq \sum_{k = 0}^\infty \frac{\mu^{2k} \sigma^{2k}}{2^k k!} \]
%
If $\mathbf{E}[X] > 0$, we may subtract by one and divide by $\mu \mathbf{E}[X]$ to conclude that for $\mu > 0$
%
\[ 1 + \mu \frac{\mathbf{E}[X^2]}{2 \mathbf{E}[X]} + \dots \leq \mu \frac{\sigma^2}{2} + \dots \]
%
and if we take $\mu \to 0$, we obtain $1 \leq 0$, a contradiction. If $\mathbf{E}[X] < 0$, the same equation holds for $\mu < 0$, so we must have $\mathbf{E}[X] = 0$. Similarily, the bound $\mathbf{V}[X] \leq \sigma^2$ is obtained by comparing coefficients.

\begin{example}
    If $X$ is a symmetric Bernoulli random variable with
    %
    \[ \mathbf{P}(X = -1) = \mathbf{P}(X = -1) = 1/2 \]
    %
    We have
    %
    \[ \mathbf{E}[e^{\lambda X}] = \frac{e^\lambda + e^{-\lambda}}{2} = \sum_{k = 0}^\infty \frac{\lambda^{2k}}{(2k)!} \leq \sum_{k = 0}^\infty \frac{(\lambda^2)^k}{2^k k!} = e^{\lambda^2/2} \]
    %
    so $X$ is a $1$ subgaussian random variable.
\end{example}

\begin{example}
    If $X$ is uniformly distributed on $[-n,n]$, then
    %
    \[ \mathbf{E}[X^k] = \int_{-n}^n \frac{x^k}{2n} dx = \frac{n^{k+1} - (-n)^{k+1}}{(k+1)2n} = \begin{cases} \frac{n^k}{k+1} & k\ \text{even} \\ 0 & k\ \text{odd} \end{cases} \]
    %
    So
    %
    \[ \mathbf{E}[e^{\lambda X}] = \sum_{k = 0}^\infty \frac{n^{2k} \lambda^{2k}}{(2k+1)(2k)!} \leq \sum_{k = 0}^\infty \frac{n^{2k} \lambda^{2k}}{2^k k!} = e^{n^2\lambda^2/2} \]
    %
    so $X$ is $n$-subgaussian.
\end{example}

\begin{example}
    In general, if a centrally distributed random variable $X$ satisfies $|X| \leq M$ almost surely, then $X$ is $M^2$ subgaussian. Assume without loss of generality that $M = 1$. Set $Y = X+1$, and
    %
    \[ f(t) = \frac{e^{2t} + 1}{2} - \mathbf{E}(e^{tY}) \]
    %
    Since $\mathbf{E}(Y) = 1$,
    %
    \[ f'(t) = \mathbf{E}(Y[e^{2t} - e^{tY}]) \]
    %
    since $Y \leq 2$ almost surely, $f'(t) \geq 0$, and so $f$ is increasing. In particular, $f(0) = 1 - 1 = 0$, so that for $t \geq 0$,
-    %
    \[ \mathbf{E}(e^{tX}) = e^{-t}\ \mathbf{E}(e^{tY}) \leq \frac{e^{t} + e^{-t}}{2} \leq e^{t^2/2} \]
    %
    Since we can perform the same argument for $-X$, we see that $X$ is $1$ subgaussian.
\end{example}

The set of subgaussian random variables form a vector space. If $X$ is a $\sigma^2$ subgaussian random variable, then $cX$ is $(c\sigma)^2$ subgaussian. If $Y$ is $\tau^2$ subgaussian, then, using the H\"{o}lder inequality, we find that for $p^{-1} + q^{-1} = 1$,
%
\[ \mathbf{E}[e^{\lambda(X+Y)}] = \mathbf{E}[e^{\lambda X} e^{\lambda Y}] \leq \mathbf{E}[e^{p \lambda X}]^{p^{-1}} \mathbf{E}[e^{q \lambda X}]^{q^{-1}} \leq e^{\frac{\lambda^2 \sigma^2}{2}} e^{\frac{\tau^2 q}{2}} = e^{\frac{\lambda^2}{2}(p \sigma^2 + q \tau^2)} \]
%
This value is minimized for $p = 1 + \tau/\sigma$, where
%
\[ p \sigma^2 + q \tau^2 = \sigma^2 + 2 \tau \sigma + \tau^2 = (\sigma + \tau)^2 \]
%
so $X + Y$ is $(\sigma + \tau)^2$ subgaussian. If $X$ and $Y$ are independant, then we actually have $X+Y$ a $\sigma^2 + \tau^2$ subgaussian variable. We can even make the set of subgaussian random variables into a Banach space, under the norm
%
\[ \sigma(X) = \inf \{ \sigma \geq 0 : X\ \text{is $\sigma^2$ subgaussian} \} \]
%
By continuity, $X$ is a $\sigma(X)$ subgaussian variable. The main reason for studying subgaussian random variables is that we obtain very good tail bounds for the distribution.

\begin{theorem}
    If $X$ is $\sigma^2$-subgaussian, then $\mathbf{P}(X \geq x) \leq e^{-x^2/2\sigma^2}$.
\end{theorem}
\begin{proof}
    Using Markov's inequality,
    %
    \begin{align*}
        \mathbf{P}(X \geq x) &= \mathbf{P}(e^{\lambda X} \geq e^{\lambda x})\\
        &\leq \mathbf{E}[e^{\lambda X}] e^{- \lambda x}\\
        &\leq e^{(\lambda^2 \sigma^2/2) - \lambda x}
    \end{align*}
    %
    The value of $\lambda$ which minizes this quantity $\lambda = x/\sigma^2$, which gives us the bound in question.
\end{proof}

The exponential decay of tails is exactly what specifies a subgaussian random variable. To prove this, note that if $\mathbf{P}(X \geq x) \leq e^{-x^2/2\sigma^2}$ holds, though we do not prove this.

\chapter{Existence Theorems}

In certain fields of probability theory, we wish to discuss collections of random variables defined over the same sample space. For instance, given a sequence $\mathbf{P}_1, \mathbf{P}_2, \dots, \mathbf{P}_n$ of probability distributions defined over a space $Y$, we may want to talk about a sequence of independent random variables $X_i: \Omega \to Y$, such that $\mathbf{P}(X_i \in U) = \mathbf{P}_i(U)$. The construction here is simple; we take $\Omega = Y^n$, let $X_i = \pi_i$ be the projection on the $i$'th variable, and let $\mathbf{P}$ be the probability measure induced by
%
\[ \mathbf{P}(U_1 \times U_2 \dots \times U_n) = \mathbf{P}_1(U_1) \mathbf{P}_2(U_2) \dots \mathbf{P}_n(U_n) \]
%
The construction here is simple because we have finitely many distributions, but the problem becomes much harder when we need to talk about an infinite family of distributions $\mathbf{P}_i$, or when we need to talk about non-independent random variables, with some specified relationships between the variables. The problem is to show there exists a sample space $\Omega$ `big enough' for the random variables to all be defined on the space.

\chapter{Entropy}

Let $\mu$ be a probability distribution. We would like to measure the expected `amount of information' contained in the distribution -- in essence, the average information entropy of $\mu$. It was Claude Shannon who found the correct formula to measure this.

Shannon considered the problem of efficient information transfer. Suppose there was a channel of communication between two friends $A$ and $B$. The friends have agreed on a standard dictionary $X$ of possible messages, along with a probability distribution $\mu$ over the dictionary, and we would like to encode these messages into bits, in such a way that the average length of the message is smallest. We then define this to be the information entropy of $\mu$. Shannon showed that if $\mu$ is discrete with probabilities $p_1, \dots, p_n$, then the entropy can be calculated as
%
\[ H(\mu) = \sum p_n \log_2 \left( \frac{1}{p_n} \right) \]
%
where the entropy is measured in bits, we can define the entropy in terms of the natural logarithm, in which case the entropy is said to be measured in nats. We assume that $p_i \log 1/p_i = 0$ for $p_i = 0$, which makes sense by the continuity of $x \log (1/x)$.

The entropy of a distribution also tells us 

Now suppose that we were attempting to optimize a message with respect to a discrete distribution $\mu$, and we instead encounter a distribution $\nu$. Then the policy we have used for messages will be less optimal than if we had known that $\nu$ was the distribution in the first place. We define the relative difference in information between $\mu$ and $\nu$ as the difference between the encoding of $\nu$ with respect to $\mu$, and the encoding of $\mu$ with respect to $\mu$. This is not a linearly ordered relation, $\nu$ does not possess more information than $\mu$, just different information. If $\mu$ takes probabilities $p_i$ and $\nu$ takes relative probabilities $q_i$, the difference in information is calculated to be
%
\[ D(\mu, \nu) = \sum p_i \log(1/q_i) - \sum p_i \log(1/p_i) = \sum p_i \log(p_i/q_i) \]
%
This is known as the {\bf Kullback Leibler distance} between $\mu$ and $\nu$.

Now suppose we are viewing independent samples $X_1, \dots, X_n$, but we do not know where the samples are drawn from $\mu$ or $\nu$. The larger $D(\mu, \nu)$ is, the less time we should take to make an accurate decision that the distribution is $\mu$ or $\nu$. Indeed, if $p_i > 0$ and $q_i = 0$, then $D(\mu, \nu) = \infty$, and we can conclude with certainty that the distribution is $\mu$ if we ever view the outcome corresponding to $p_i$.

It is necessary to define the `entropy' of an arbitrary distribution, but it is then not clear how to interpret the entropy, since an encoding of uncountably many values will always have an infinite expected number of bits. However, we can defined the relative entropy by performing a discretization; Let $\mu$ and $\nu$ be distributions on some sample space $X$. Consider function $f: X \to \{ 1, \dots, n \}$, and define
%
\[ D(\mu, \nu) = \sup_f D(f_* \mu, f_* \nu) \]
%
where $f_*$ pushes measures on $X$ onto discrete measures on $\{ 1, \dots, n \}$. For a fixed $f$, $D(\mu, \nu)$ upper bounds the difference in information we expect to see over a particular discretization. One can then calculate that
%
\[ D(\mu, \nu) = \begin{cases} \infty &: \mu \not \ll \nu \\ \int \log(\frac{d\mu}{d\nu}) d\mu &: \mu \ll \nu \end{cases} \]
%
The relative entropies of well known distributions are easy to compute. Normal distributions, for instance, have
%
\[ D(N(\mu_1, \sigma^2), N(\mu_2, \sigma^2)) = (\mu_1 - \mu_2)^2/2\sigma^2 \]
%
For Bernoulli distributions, we have
%
\[ D(B(p), B(q)) = p \log(p/q) + (1-p)\log \left( \frac{1-p}{1-q} \right) \]
%
Which is true except perhaps at boundary conditions.

The Kullback Leibler distance gives us certain bounds which are essential to information theoretic lower bounds. The bound is useful, for it relates the probabilities of distributions by the difference in information contained within.

\begin{theorem}[The High Probability Pinsker Bound]
    If $\mu$ and $\nu$ are probability measures on the same space $X$, and $U \subset X$ is measurable, then
    %
    \[ \mu(A) + \nu(A^c) \geq \frac{1}{2} e^{- D(\mu, \nu)} \]
\end{theorem}

Suppose we have a decision procedure which attempts to distinguish between events in probability distributions. If we choose an event $A$ upon which the decision procedure fails to make the correct decision on the measure $\mu$, and $A^c$ measures the decision to fail under the measure $\nu$, then the bound above shows the decision procedure cannot work reliably on both $\mu$ and $\nu$.

\begin{thebibliography}{10}
    \bibitem{intro} Larry Wasserman,
    \emph{All of Statistics}

    \bibitem{papa} Walter Rudin,
    \emph{Real and Complex Analysis}
\end{thebibliography}

\end{document}

\begin{example}
    The uniform distribution works well in modelling a fair coin, but what if we are flipping an unfair coin? The Bernoulli distribution assigns a probability of $p$ to getting a heads, and $1-p$ to getting a tails, for some parameter $p$. We write $\mu \sim \Ber(p)$ for this distribution. We normally rename the elements of the sample space by letting $\text{Heads} = 1$ and $\text{Tails} = 0$, so that the probability distribution counts the number of heads and tails from one coin flip. This permits an easy generalization to the Binomial distribution $\Bin(n,p)$, which counts the number of times we get a head in $n$ coin flips from a $\Ber(p)$ distribution. It is defined on $\mathbf{Z}$ by letting
    %
    \[ \Prob(m) = \binom{n}{m} p^m (1-p)^{n-m} \]
    %
    for $0 \leq m \leq n$, and $\Prob(m) = 0$ otherwise. If $\mu \sim \Bin(n,p)$, then we have a representation $\mu = \nu * \dots * \nu$ as an $n$-ary convolution, where $\nu \sim \Ber(p)$. In general, if $\mu$ and $\nu$ are probability measures on a group $G$, then $\mu * \nu$ is a probability measure, since
    %
    \[ (\mu * \nu)(X) = \int_{X^2} d\mu(x)\ d\nu(y) = \int_X d\mu(x) \ \int_X d\nu(Y) = 1 \]
    %
    which represents a certain `sum' over the two distributions, especially in the case $G = \mathbf{Z}$.
\end{example}

\begin{example}
    The Geometric distribution on $\mathbf{N}$ gives us the probability distribution of the number of flips required to get a heads from a weighted coin. If $\mu \sim \Geo(p)$, then
    %
    \[ \mu(m) = p (1-p)^m \]
    %
    The infinite sum over all $m \geq 0$ is easily evaluated to be one, so that the distribution truly is a probability measure.
\end{example}

\begin{example}
    The Poisson distribution on $\mathbf{N}$ is defined, for $\mu \sim \Poisson(\lambda)$, by letting
    %
    \[ \mu(n) = \lambda^n \frac{e^{-\lambda}}{n!} \]
    %
    This distribution is usually used to count rare, independant events, like for radioactive decay and traffic accidents. If $\mu \sim \Poisson(\lambda)$ and $\nu \sim \Poisson(\gamma)$, then $\mu * \nu \sim \Poisson(\lambda + \gamma)$.
\end{example}

To obtain measures on uncountable sample spaces, we rely on integration theory.

\begin{example}
    If $f \geq 0$ is a measurable function on a measure space $X$ with measure $\mu$ with $\| f \|_1 = 1$, then we may define a probability measure $\nu$ on $X$ by letting
    %
    \[ \nu(A) = \int_A f d\mu \]
    %
    The measure $\nu$ is often denoted $f d\mu$ because of the integral formula we use to obtain it.
\end{example}

The method in the last example gives us a number of canonical distributions, giving us a wide number of examples. A theorem of measure theory (the Radon-Nikodym theorem) tells us why almost all the important probability distributions can be given by integration (or summation, those measure theory tells us these are essentially the same processes).

\begin{example}
    We would like to have a probability distribution which models picking a point `uniformly' from an interval $[a,b]$. What we really mean is that the probability measure is translation invariant, in the sense that $\Prob(A + t) = \Prob(A)$, for all $A$ and $t$ for which this should be reasonably defined. What we are looking for is exactly the normalized Lebesgue measure restricted to $[0,1]$, which is rigorously defined in measure theory. We take the lebesgue measure $dx$, and then normalize to obtain a probability measure $dx/(b-a) \sim \Uni(a,b)$. It seem paradoxical that the probability that any particular point is chosen is zero, yet in an experiment we must choose a particular point. However, to obtain probabilities we take the limit of ratios over an infinite number of experiments, and it is entirely possible for the limit of the ratios to converge to zero, even if a point is chosen in some experiment.
\end{example}

\begin{example}
    Define a function $f:\mathbf{R} \to \mathbf{R}$ by the formula
    %
    \[ f(x) = \frac{1}{\sigma \sqrt{2\pi}} e^{-\frac{(x - \mu)^2}{2\sigma^2}} \]
    %
    the integral of this function over $\mathbf{R}$ is one, and is proved in multivariate calculus courses. The probability measure generated by $f$ is known as the {\bf normal distribution} with mean $\mu$ and standard deviation $\sigma$. We write $\Prob \sim N(\mu, \sigma^2)$ if $\Prob$ for such a measure.
\end{example}

\begin{example}
    The exponential distribution $\Exp(\beta)$ is defined on the positive real numbers by the measure induced by the lebesgue measure, and the function
    %
    \[ f(x) = \frac{e^{-x/\beta}}{\beta} \]
    %
    It measures the waiting times between rare events that occur at continuous times.
\end{example}

\begin{example}
    The Gamma distribution is defined with respect to the remarkable function known as the Gamma function
    %
    \[ \Gamma(x) = \int_0^\infty y^{x-1} e^{-y} dy \]
    %
    The Gamma distribution $\Gamma(\alpha, \beta)$ with positive parameters $\alpha$ and $\beta$ is induced from the Lebesgue measure by the function
    %
    \[ f(x) = \frac{x^{\alpha-1} e^{-x/\beta}}{\beta^\alpha \Gamma(\alpha)} \]
    %
    We have $\Exp(\beta) = \Gamma(1,\beta)$.
\end{example}

\begin{example}
    The Beta distribution $\Beta(\alpha, \beta)$ for positive $\alpha, \beta > 0$ is induced by the function
    %
    \[ f(x) = \frac{\Gamma(\alpha + \beta)}{\Gamma(\alpha) \Gamma(\beta)} x^{\alpha-1} (1 - x)^{\beta - 1} \]
    %
    It is a continuous version of the Binomial distribution.
\end{example}

\begin{example}
    The $t$ distribution with $\nu$ degrees of freedom, denoted $t_\nu$, if it is the measure taken by
    %
    \[ f(x) = \frac{\Gamma(\frac{\nu + 1}{2})}{\Gamma(\frac{\nu}{2})} \frac{1}{(1 + x^2/\nu)^{(\nu+1)/2}} \]
    %
    It is similar to the normal distribution, but has thicker tails. In fact, as $\nu \to \infty$, the distributions corresponding to $t_\nu$ look more and more like the normal distribution. The Cauchy distribution is the $t$ distribution with $\nu = 1$, defined by
    %
    \[ f(x) = \frac{1}{\pi(1 + x^2)} \]
\end{example}

\begin{example}
    The $\chi^2$ distribution with $p$ degrees of freedom, denoted $\chi_p^2$, has distribution induced on the positive real numbers by
    %
    \[ f(x) = \frac{x^{(p/2) - 1} e^{-x/2}}{\Gamma(p/2) 2^{p/2}} \]
    %
    if $\mu \sim N(0, 1)$, and $\nu$ is a measure defined on the positive real numbers by letting $\nu(A) = \mu(B)$, where $B = \{ x \in \mathbf{R}: x^2 \in A \}$, then the $p$-adic convolution $\nu * \dots * \nu \sim \chi_p^2$.
\end{example}