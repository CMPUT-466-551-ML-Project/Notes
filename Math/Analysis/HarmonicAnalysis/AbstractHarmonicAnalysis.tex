
\part{Abstract Harmonic Analysis}

The main property of spaces where Fourier analysis applies is symmetry -- for a function $\RR$, we can translate and negate. On $\RR^n$ we have not only translational symmetry but also rotational symmetry. It turns out that we can apply Fourier analysis to any `space with symmetry'. That is, functions on an Abelian group. We shall begin with the study of finite abelian groups, where convergence questions disappear, and with it much of the analytical questions involved in the theory. We then proceed to generalize to a study of infinite abelian groups with topological structure.







\chapter{Topological Groups}

In abstract harmonic analysis, the main subject matter is the {\bf topological group}, a group $G$ equipped with a topology which makes the operation of multiplication and inversion continuous. In the mid 20th century, it was realized that basic Fourier analysis could be generalized to a large class of groups. The nicest generalization occurs over the locally compact groups, which simplifies the theory considerably.

\begin{example}
    There are a few groups we should keep in mind for intuition in the general topological group.
    %
    \begin{itemize}
        \item The classical groups $\RR^n$ and $\mathbf{T}^n$, from which Fourier analysis originated.
        \item The group $\mu$ of roots of unity, rational numbers $\mathbf{Q}$, and cyclic groups $\mathbf{Z}_n$.
        \item The matrix subgroups of the general linear group $GL(n)$.
        \item The product $\mathbf{T}^\omega$ of Torii, occurring in the study of Dirichlet series.
        \item The product $\mathbf{Z}_2^\omega$, which occurs in probability theory, and other contexts.
        \item The field of $p$-adic numbers $\mathbf{Q}_p$, which are the completion of $\mathbf{Q}$ with respect to the absolute value $|p^{-m} q|_p = p^m$.
    \end{itemize}
\end{example}

\section{Basic Results}

The topological structure of a topological group naturally possesses large amounts of symmetry, simplifying the spatial structure. For any topological group, the maps
%
\[ x \mapsto gx\ \ \ \ \ \ \ \ \ \ x \mapsto xg\ \ \ \ \ \ \ \ \ \ x \mapsto x^{-1} \]
%
are homeomorphisms. Thus if $U$ is a neighbourhood of $x$, then $gU$ is a neighbourhood of $gx$, $Ug$ a neighbourhood of $xg$, and $U^{-1}$ a neighbourhood of $x^{-1}$, and as we vary $U$ through all neighbourhoods of $x$, we obtain all neighbourhoods of the other points. Understanding the topological structure at any point reduces to studying the neighbourhoods of the identity element of the group.

In topological group theory it is even more important than in basic group theory to discuss set multiplication. If $U$ and $V$ are subsets of a group, then we define
%
\[ U^{-1} = \{ x^{-1} : x \in U \}\ \ \ \ \ \ \ \ UV = \{ xy: x \in U, y \in V \} \]
%
We let $V^2 = VV$, $V^3 = VVV$, and so on.

\begin{theorem}
    Let $U$ and $V$ be subsets of a topological group.
    %
    \begin{enumerate}
        \item[(i)] If $U$ is open, then $UV$ is open.
        \item[(ii)] If $U$ is compact, and $V$ closed, then $UV$ is closed.
        \item[(iii)] If $U$ and $V$ are connected, $UV$ is connected.
        \item[(iv)] If $U$ and $V$ are compact, then $UV$ is compact.
    \end{enumerate}
\end{theorem}
\begin{proof}
    To see that (i) holds, we see that
    %
    \[ UV = \bigcup_{x \in V} Ux \]
    %
    and each $Ux$ is open. To see (ii), suppose $u_i v_i \to x$. Since $U$ is compact, there is a subnet $u_{i_k}$ converging to $y$. Then $y \in U$, and we find
    %
    \[ v_{i_k} = u_{i_k}^{-1} ( u_{i_k} v_{i_k} ) \to y^{-1} x \]
    %
    Thus $y^{-1} x \in V$, and so $x = y y^{-1} x \in UV$. (iii) follows immediately from the continuity of multiplication, and the fact that $U \times V$ is connected, and (iv) follows from similar reasoning.
\end{proof}

\begin{example}
    If $U$ is merely closed, then (ii) need not hold. For instance, in $\RR$, take $U = \alpha \mathbf{Z}$, and $V = \mathbf{Z}$, where $\alpha$ is an irrational number. Then $U + V = \alpha \mathbf{Z} + \mathbf{Z}$ is dense in $\RR$, and is hence not closed.
\end{example}

There are useful ways we can construct neighbourhoods under the group operations, which we list below.

\begin{lemma}
    Let $U$ be a neighbourhood of the identity. Then
    %
    \begin{itemize}
        \item[(1)] There is an open $V$ such that $V^2 \subset U$.
        \item[(2)] There is an open $V$ such that $V^{-1} \subset U$.
        \item[(3)] For any $x \in U$, there is an open $V$ such that $xV \subset U$.
        \item[(4)] For any $x$, there is an open $V$ such that $xVx^{-1} \subset U$.
    \end{itemize}
\end{lemma}
\begin{proof}
    (1) follows simply from the continuity of multiplication, and (2) from the continuity of inversion. (3) is verified because $x^{-1}U$ is a neighbourhood of the origin, so if $V = x^{-1}U$, then $xV = U \subset U$. Finally (4) follows in a manner analogously to (3) because $x^{-1}Ux$ contains the origin.
\end{proof}

If $\mathcal{U}$ is an open basis at the origin, then it is only a slight generalization to show that for any of the above situations, we can always select $V \in \mathcal{U}$. Conversely, suppose that $\mathcal{V}$ is a family of subsets of a (not yet topological) group $G$ containing $e$ such that (1), (2), (3), and (4) hold. Then the family $\mathcal{V}' = \{ xV : V \in \mathcal{V}, x \in G \}$ forms a subbasis for a topology on $G$ which forms a topological group. If $\mathcal{V}$ also has the base property, then $\mathcal{V}'$ is a basis.

\begin{theorem}
    If $K$ and $C$ are disjoint, $K$ is compact, and $C$ is closed, then there is a neighbourhood $V$ of the origin for which $KV$ and $CV$ is disjoint. If $G$ is locally compact, then we can select $V$ such that $KV$ is precompact.
\end{theorem}
\begin{proof}
    For each $x \in K$, $C^c$ is an open neighbourhood containing $x$, so by applying the last lemma recursively we find that there is a symmetric neighbourhood $V_x$ such that $x V_x^4 \subset C^c$. Since $K$ is compact, finitely many of the $xV_x$ cover $K$. If we then let $V$ be the open set obtained by intersecting the finite subfamily of the $V_x$, then $KV$ is disjoint from $CV$.
\end{proof}

Taking $K$ to be a point, we find that any open neighbourhood of a point contains a closed neighbourhood. Provided points are closed, we can set $C$ to be a point as well.

\begin{corollary}
    Every Kolmogorov topological group is Hausdorff.
\end{corollary}

\begin{theorem}
    For any set $A \subset G$,
    %
    \[ \overline{A} = \bigcap_V AV \]
    %
    Where $V$ ranges over the set of neighbourhoods of the origin.
\end{theorem}
\begin{proof}
    If $x \not \in \overline{A}$, then the last theorem guarantees that there is $V$ for which $\overline{A}V$ and $Ax$ are disjoint. We conclude $\bigcap AV \subset \overline{A}$. Conversely, any neighbourhood contains a closed neighbourhood, so that $\overline{A} \subset AV$ for a fixed $V$, and hence $\overline{A} \subset \bigcap AV$.
\end{proof}

\begin{theorem}
    Every open subgroup of $G$ is closed.
\end{theorem}
\begin{proof}
    Let $H$ be an open subgroup of $G$. Then $\overline{H} = \bigcap_V HV$. If $W$ is a neighbourhood of the origin contained in $H$, then we find $\overline{H} \subset HW \subset H$, so $H$ is closed.
\end{proof}

We see that open subgroups of a group therefore correspond to connected components of the group, so that connected groups have no proper open subgroups. This also tells us that a locally compact group is $\sigma$-compact on each of its components, for if $V$ is a pre-compact neighbourhood of the origin, then $V^2, V^3, \dots$ are all precompact, and $\bigcup_{k = 1}^\infty V^k$ is an open subgroup of $G$, which therefore contains the component of $e$, and is $\sigma$-compact. Since the topology of a topological group is homogenous, we can conclude that all components of the group are $\sigma$ compact.

\section{Quotient Groups}

If $G$ is a topological group, and $H$ is a subgroup, then $G/H$ can be given a topological structure in the obvious way. The quotient map is open, because $VH$ is open in $G$ for any open set $V$, and if $H$ is normal, $G/H$ is also a topological group, because multiplication is just induced from the quotient map of $G \times G$ to $G/H \times G/H$, and inversion from $G$ to $G/H$. We should think the quotient structure is pleasant, but if no conditions on $H$ are given, then $G/H$ can have pathological structure. One particular example is the quotient $\mathbf{T}/\mu_\infty$ of the torus modulo the roots of unity, where the quotient is lumpy.

\begin{theorem}
    If $H$ is closed, $G/H$ is Hausdorff.
\end{theorem}
\begin{proof}
    If $x \neq y \in G/H$, then $xHy^{-1}$ is a closed set in $G$, not containing $e$, so we may conclude there is a neighbourhood $V$ for which $V$ and $VxHy^{-1}$ are disjoint, so $VyH$ and $VxH$ are disjoint. This implies that the open sets $V(xH)$ and $V(yH)$ are disjoint in $G/H$.
\end{proof}

\begin{theorem}
    If $G$ is locally compact, $G/H$ is also.
\end{theorem}
\begin{proof}
    If $\{ U_i \}$ is a basis of precompact neighbourhoods at the origin, then $U_iH$ is a family of precompact neighbourhoods of the origin in $G/H$, and is in fact a basis, for if $V$ is any neighbourhood of the origin, there is $U_i \subset \pi^{-1}(V)$, and so $U_iH \subset V$.
\end{proof}

If $G$ is a non-Hausdorff group, then $\overline{\{e\}} \neq \{ e \}$, and $G/\overline{\{e\}}$ is Hausdorff. Thus we can get away with assuming all our topological groups are Hausdorff, because a slight modification in the algebraic structure of the topological group gives us this property.

\section{Uniform Continuity}

An advantage of the real line $\RR$ is that continuity can be explained in a {\it uniform sense}, because we can transport any topological questions about a certain point $x$ to questions about topological structure near the origin via the map $g \mapsto x^{-1}g$. We can then define a uniformly continuous function $f: \RR \to \RR$ to be a function possessing, for every $\varepsilon > 0$, a $\delta > 0$ such that if $|y| < \delta$, $|f(x+y) - f(x)|<\varepsilon$. Instead of having to specify a $\delta$ for every point on the domain, the $\delta$ works uniformly everywhere. The group structure is all we need to talk about these questions.

We say a function $f: G \to H$ between topological groups is (left) uniformly continuous if, for any open neighbourhood $U$ of the origin in $H$, there is a neighbourhood $V$ of the origin in $G$ such that for each $x$, $f(xV) \subset f(x) U$. Right continuity requires $f(Vx) \subset U f(x)$. The requirement of distinguishing between left and right uniformity is important when we study non-commutative groups, for there are certainly left uniform maps which are not right uniform in these groups. If $f: G \to \mathbf{C}$, then left uniform continuity is equivalent to the fact that $\| L_x f - f \|_\infty \to 0$ as $x \to 1$, where $(L_x f)(y) = f(xy)$. Right uniform continuity requires $\| R_x f - f \|_\infty \to 0$, where $(R_x f)(y) = f(yx)$. $R_x$ is a homomorphism, but $L_x$ is what is called an antihomomorphism.

\begin{example}
    Let $G$ be any Hausdorff non-commutative topological group, with sequences $x_i$ and $y_i$ for which $x_i y_i \to e$, $y_i x_i \to z \neq e$. Then the uniform structures on $G$ are not equivalent.
\end{example}

It is hopeless to express uniform continuity in terms of a new topology on $G$, because the topology only gives a local description of continuity, which prevents us from describing things uniformly across the whole group. However, we can express uniform continuity in terms of a new topology on $G \times G$. If $U \subset G$ is an open neighbourhood of the origin, let
%
\[ L_U = \{ (x,y): yx^{-1} \in U \}\ \ \ \ \ R_U = \{ (x,y): x^{-1}y \in U \} \]
%
The family of all $L_U$ (resp. $R_U$) is known as the left (right) uniform structure on $G$, denoted $LU(G)$ and $RU(G)$. Fix a map $f: G \to H$, and consider the map
%
\[ g(x,y) = (f(x), f(y)) \]
%
from $G^2$ to $H^2$. Then $f$ is left (right) uniformly continuous if and only if $g$ is continuous with respect to $LU(G)$ and $LU(H)$ ($RU(G)$ and $RU(H)$). $LU(G)$ and $RU(G)$ are weaker than the product topologies on $G$ and $H$, which reflects the fact that uniform continuity is a strong condition than normal continuity. We can also consider uniform maps with respect to $LU(G)$ and $RU(H)$, and so on and so forth. We can also consider uniform continuity on functions defined on an open subset of a group.

\begin{example}
    Here are a few examples of easily verified continuous maps.
    \begin{itemize}
        \item If the identity map on $G$ is left-right uniformly continuous, then $LU(G) = RU(G)$, and so uniform continuity is invariant of the uniform structure chosen.
        \item Translation maps $x \mapsto axb$, for $a,b \in G$, are left and right uniform.
        \item Inversion is uniformly continuous.
    \end{itemize}
\end{example}

\begin{theorem}
    All continuous maps on compact subsets of topological groups are uniformly continuous.
\end{theorem}
\begin{proof}
    Let $K$ be a compact subset of a group $G$, and let $f:K \to H$ be a continuous map into a topological group. We claim that $f$ is then uniformly continuous. Fix an open neighbourhood $V$ of the origin, and let $V'$ be a symmetric neighbourhood such that $V'^2 \subset V$. For any $x$, there is $U_x$ such that
    %
    \[ f(x)^{-1} f(xU_x) \subset V' \]
    %
    Choose $U'_x$ such that $U'^2_x \subset U_x$. The $xU'_x$ cover $K$, so there is a finite subcover corresponding to sets $U'_{x_1}, \dots, U'_{x_n}$. Let $U = U'_{x_1} \cap \dots \cap U'_{x_n}$. Fix $y \in G$, and suppose $y \in x_k U'_{x_k}$. Then
    %
    \begin{align*}
        f(y)^{-1} f(yU) &= f(y)^{-1} f(x_k) f(x_k)^{-1} f(yU)\\
        &\subset f(y)^{-1} f(x_k) f(x_k)^{-1} f(x_k Ux_k)\\
        &\subset f(y)^{-1} f(x_k) V'\\
        &\subset V'^2 \subset V
    \end{align*}
    %
    So that $f$ is left uniformly continuous. Right uniform continuity is proven in the exact same way.
\end{proof}

\begin{corollary}
    All maps with compact support are uniformly continuous.
\end{corollary}

\begin{corollary}
    Uniform continuity on compact groups is invariant of the uniform structure chosen.
\end{corollary}

\section{Ordered Groups}

In this section we describe a general class of groups which contain both interesting and pathological examples. Let $G$ be a group with an ordering $<$ preserved by the group operations, so that $a < b$ implies both $ag < bg$ and $ga < gb$. We now prove that the order topology gives $G$ the structure of a normal topological group (the normality follows because of general properties of order topologies).

First note, that $a < b$ implies $a^{-1} < b^{-1}$. This results from a simple algebraic trick, because
%
\[  a^{-1} = a^{-1} b b^{-1} > a^{-1} a b^{-1} = b^{-1} \]
%
This implies that the inverse image of an interval $(a,b)$ under inversion is $(b^{-1}, a^{-1})$, hence inversion is continuous.

Now let $e < b < a$. We claim that there is then $e < c$ such that $c^2 < a$. This follows because if $b^2 \geq a$, then $b \geq ab^{-1}$ and so
%
\[ (ab^{-1})^2 = ab^{-1}ab^{-1} \leq ab^{-1}b = a \]
%
Now suppose $a < e < b$. If $\inf \{ y : y > e \} = x > e$, then $(x^{-1}, x) = \{ e \}$, and the topology on $G$ is discrete, hence the continuity of operations is obvious. Otherwise, we may always find $c$ such that $c^2 < b$, $a < c^{-2}$, and then if $c^{-1} < g,h < c$, then
%
\[ a < c^{-2} < gh < c^2 < b \]
%
so multiplication is continuous at every pair $(x,x^{-1})$. In the general case, if $a < gh < b$, then $g^{-1}ah^{-1} < e < g^{-1}bh^{-1}$, so there is $c$ such that if $c^{-1} < g',h' < c$, then $g^{-1}ah^{-1} < g'h' < g^{-1}bh^{-1}$, so $a < gg'h'h < b$. The set of $gg'$, where $c^{-1} < g' < c$, is really just the set of $gc^{-1} < x < gc$, and the set of $h'h$ is really just the set of $c^{-1}h < x < ch$. Thus multiplication is continuous everywhere.

\begin{example}[Dieudonne]
    For any well ordered set $S$, the dictionary ordering on $\RR^S$ induces a linear ordering inducing a topological group structure on the set of maps from $S$ to $\RR$.
\end{example}

Let us study Dieudonne's topological group in more detail. If $S$ is a finite set, or more generally possesses a maximal element $w$, then the topology on $\RR^S$ can be defined such that $f_i \to f$ if eventually $f_i(s) = f(s)$ for all $s < w$ simultaneously, and $f_i(w) \to f(w)$. Thus $\RR^S$ is isomorphic (topologically) to a discrete union of a certain number of copies of $\RR$, one for each tuple in $S - \{ w \}$.

If $S$ has a countable cofinal subset $\{ s_i \}$, the topology is no longer so simple, but $\RR^S$ is still first countable, because the sets
%
\[ U_i = \{ f : (\forall w < s_i: f(w) = 0) \} \]
%
provide a countable neighbourhood basis of the origin.

The strangest properties of $\RR^S$ occur when $S$ has no countable cofinal set. Suppose that $f_i \to f$. We claim that it follows that $f_i = f$ eventually. To prove by contradiction, we assume without loss of generality (by thinning the sequence) that no $f_i$ is equal to $f$. For each $f_i$, find the largest $w_i \in S$ such that for $s < w_i$, $f_i(s) = f(s)$ (since $S$ is well ordered, the set of elements for which $f_i(s) \neq f(s)$ has a minimal element). Then the $w_i$ form a countable cofinal set, because if $v \in S$ is arbitrary, the $f_i$ eventually satisfy $f_i(s) = f(s)$ for $s < v$, hence the corresponding $w_i$ is greater than $v_i$. Hence, if $f_i \to f$ in $\RR^S$, where $S$ does not have a countable cofinal subset, then eventually $f_i = f$. We conclude all countable sets in $\RR^S$ are closed, and this proof easily generalises to show that if $S$ does not have a cofinal set of cardinality $\mathfrak{a}$, then every set of cardinality $\leq \mathfrak{a}$ is closed.

The simple corollary to this proof is that compact subsets are finite. Let $X = f_1, f_2, \dots$ be a denumerable, compact set. Since all subsets of $X$ are compact, we may assume $f_1 < f_2 < \dots$ (or $f_1 > f_2 > \dots$, which does not change the proof in any interesting way). There is certainly $g \in \RR^S$ such that $g < f_1$, and then the sets $(g,f_2), (f_1, f_3), (f_2,f_4), \dots$ form an open cover of $X$ with no finite subcover, hence $X$ cannot be compact. We conclude that the only compact subsets of $\RR^S$ are finite.

Furthermore, the class of open sets is closed under countable intersections. Consider a series of functions
%
\[ f_1 \leq f_2 \leq \dots < h < \dots \leq g_2 \leq g_1 \]
%
Suppose that $f_i \leq k < h < k' \leq g_j$. Then the intersection of the $(f_i, g_i)$ contains an interval $(k,k')$ around $h$, so that the intersection is open near $h$. The only other possiblity is that $f_i \to h$ or $g_i \to h$, which can only occur if $f_i = h$ or $g_i = h$ eventually, in which case we cannot have $f_i < h$, $h < g_i$. We conclude the intersection of countably many intervals is open, because we can always adjust any intersection to an intersection of this form without changing the resulting intersecting set (except if the set is empty, in which case the claim is trivial). The general case results from noting that any open set in an ordered group is a union of intervals.

\section{Topological Groups arising from Normal subgroups}

Let $G$ be a group, and $\mathcal{N}$ a family of normal subgroups closed under intersection. If we interpret $\mathcal{N}$ as a neighbourhood base at the origin, the resulting topology gives $G$ the structure of a totally disconnected topological group, which is Hausdorff if and only if $\bigcap \mathcal{N} = \{ e \}$. First note that $g_i \to g$ if $g_i$ is eventually in $gN$, for every $N \in \mathcal{N}$, which implies $g_i^{-1} \in Ng^{-1} = g^{-1}N$, hence inversion is continuous. Furthermore, if $h_i$ is eventually in $hN$, then $g_ih_i \in gNhN = ghN$, so multiplication is continuous. Finally note that $N^c = \bigcup_{g \neq e} gN$ is open, so that every open set is closed.

\begin{example}
    Consider $\mathcal{N} = \{ \mathbf{Z}, 2\mathbf{Z}, 3\mathbf{Z}, \dots \}$. Then $\mathcal{N}$ induces a Hausdorff topology on $\mathbf{Z}$, such that $g_i \to g$, if and only if $g_i$ is eventually in $g + n \mathbf{Z}$ for all $n$. In this topology, the series $1,2,3,\dots$ converges to zero!
\end{example}

This example gives us a novel proof, due to Furstenburg, that there are infinitely many primes. Suppose that there were only finitely many, $\{ p_1, p_2, \dots, p_n \}$. By the fundamental theorem of arithmetic,
%
\[ \{ -1, 1 \} = (\mathbf{Z} p_1)^c \cap \dots \cap (\mathbf{Z} p_n)^c \]
%
and is therefore an open set. But this is clearly not the case as open sets must contain infinite sequences.

\chapter{The Haar Measure}

One of the reasons that we isolate locally compact groups to study is that they possess an incredibly useful object allowing us to understand functions on the group, and thus the group itself. A {\bf left (right) Haar measure} for a group $G$ is a Radon measure $\mu$ for which $\mu(xE) = \mu(E)$ for any $x \in G$ and measurable $E$ ($\mu(Ex) = \mu(E)$ for all $x$ and $E$). For commutative groups, all left Haar measures are right Haar measures, but in non-commutative groups this need not hold. However, if $\mu$ is a right Haar measure, then $\nu(E) = \mu(E^{-1})$ is a left Haar measure, so there is no loss of generality in focusing our study on left Haar measures.

\begin{example}
    The example of a Haar measure that everyone knows is the Lebesgue measure on $\RR$ (or $\RR^n$). It commutes with translations because it is the measure induced by the linear functional corresponding to Riemann integration on $C_c^+(\RR^n)$. A similar theory of Darboux integration can be applied to linearly ordered groups, leading to the construction of a Haar measure on such a group.
\end{example}

\begin{example}
    If $G$ is a Lie group, consider a $2$-tensor $g_e \in T^2_e(G)$ inducing an inner product at the origin. Then the diffeomorphism $f: a \mapsto b^{-1}a$ allows us to consider $g_b = f^* \lambda \in T^2_b(G)$, and this is easily verified to be an inner product, hence we have a Riemannian metric. The associated Riemannian volume element can be integrated, producing a Haar measure on $G$.
\end{example}

\begin{example}
    If $G$ and $H$ have Haar measures $\mu$ and $\nu$, then $G \times H$ has a Haar measure $\mu \times \nu$, so that the class of topological groups with Haar measures is closed under the product operation. We can even allow infinite products, provided that the groups involved are compact, and the Haar measures are normalized to probability measures. This gives us measures on $F_2^\omega$ and $\mathbf{T}^\omega$, which models the probability of an infinite sequence of coin flips.
\end{example}

\begin{example}
    $dx/x$ is a Haar measure for the multiplicative group of positive real numbers, since
    %
    \[ \int_a^b \frac{1}{x} = \log(b) - \log(a) = \log(cb) - \log(ca) = \int_{ca}^{cb} \frac{1}{x} \]
    %
    If we take the multiplicative group of all non-negative real numbers, the Haar measure becomes $dx/|x|$.
\end{example}

\begin{example}
    $dx dy/(x^2 + y^2)$ is a Haar measure for the multiplicative group of complex numbers, since we have a basis of `arcs' around the origin, and by a change of variables to polar coordinates, we verify the integral is changed by multiplication. Another way to obtain this measure is by noticing that $\mathbf{C}^\times$ is topologically isomorphic to the product of the circle group and the multiplicative group of real numbers, and hence the measure obtained should be the product of these measures. Since
    %
    \[ \frac{dx dy}{x^2 + y^2} = \frac{dr d\theta}{r} \]
    %
    We see that this is just the product of the Haar measure on $\RR^+$, $dr/r$, and the Haar measure on $\mathbf{T}$, $d \theta$.
\end{example}

\begin{example}
    The space $M_n(\RR)$ of all $n$ by $n$ real matrices under addition has a Haar measure $dM$, which is essentially the Lebesgue measure on $\RR^{n^2}$. If we consider the measure on $GL_n(\RR)$, defined by
    %
    \[ \frac{dM}{\text{det}(M)^n} \]
    %
    To see this, note the determinant of the map $M \mapsto NM$ on $M_n(\RR)$ is $\text{det}(N)^n$, because we can view $M_n(\RR)$ as the product of $\RR^n$ $n$ times, multiplication operates on the space componentwise, and the volume of the image of the unit paralelliped in each $\RR^n$ is $\text{det}(N)$. Since the multiplicative group of complex numbers $z = x + iy$ can be identified with the group of matrices of the form
    %
    \[ \begin{pmatrix} x & -y \\ y & x \end{pmatrix} \]
    %
    and the measure on $\mathbf{C} - \{ 0 \}$ then takes the form $dM/\text{det}(M)$. More generally, if $G$ is an open subset of $\RR^n$, and left multiplication acts affinely, $xy = A(x)y + b(x)$, then $dx/|\text{det}(A(x))|$ is a left Haar measure on $G$, where $dx$ is Lebesgue measure.
\end{example}

It turns out that there is a Haar measure on any locally compact group, and what's more, it is unique up to scaling. The construction of the measure involves constructing a positive linear functional $\phi: C_c(G) \to \RR$ such that $\phi(L_x f) = \phi(f)$ for all $x$. The Riesz representation theorem then guarantees the existence of a Radon measure $\mu$ which represents this linear functional, and one then immediately verifies that this measure is a Haar measure.

\begin{theorem}
    Every locally compact group $G$ has a Haar measure.
\end{theorem}
\begin{proof}
    The idea of the proof is fairly simple. If $\mu$ was a Haar measure, $f \in C_c^+(G)$ was fixed, and $\phi \in C_c^+(G)$ was a function supported on a small set, and behaving like a step function, then we could approximate $f$ well by translates of $\phi$,
    %
    \[ f(x) \approx \sum c_i (L_{x_i} \phi) \]
    %
    Hence
    %
    \[ \int f(x) d \mu \approx \sum c_i \int L_{x_i} \phi = \sum c_i \int \phi \]
    %
    If $\int \phi = 1$, then we could approximate $\int f(x) d \mu$ as literal sums of coefficients $c_i$. Since $\mu$ is outer regular, and $\phi$ is supported on neighbourhoods, one can show $\int f(x) d\mu$ is the infinum of $\sum c_i$, over all choices of $c_i > 0$ and $\int \phi \geq 1$, for which $f \leq \sum c_i L_{x_i} \phi$. Without the integral, we cannot measure the size of the functions $\phi$, so we have to normalize by a different factor. We define $(f: \phi)$ to be the infinum of the sums $\sum c_i$, where $f \leq \sum c_i L_{x_i} \phi$ for some $x_i \in G$. We would then have
    %
    \[ \int f d \mu \leq (f: \phi) \int \phi d\mu \]
    %
    If $k$ is fixed with $\int k = 1$, then we would have
    %
    \[ \int f d\mu \leq (f: \phi) (\phi: k) \]
    %
    We cannot change $k$ if we wish to provide a limiting result in $\phi$, so we notice that $(f: g) (g: h) \leq (f:h)$, which allows us to write
    %
    \[ \int f d\mu \leq \frac{(f: \phi)}{(k : \phi)} \]
    %
    Taking the support of $\phi$ to be smaller and smaller, this value should approximate the integral perfectly accurately.

    Define the linear functional
    %
    \[ I_\phi(f) = \frac{(f: \phi)}{(k: \phi)} \]
    %
    Then $I_\phi$ is a sublinear, monotone, function with a functional bound
    %
    \[ (k: f)^{-1} \leq I_\phi(f) \leq (f: k) \]
    %
    Which effectively says that, regardless of how badly we choose $\phi$, the approximation factor $(f:\phi)$ is normalized by the approximation factor $(k:\phi)$ so that the integral is bounded. Now we need only prove that $I_\phi$ approximates a linear functional well enough that we can perform a limiting process to obtain a Haar integral. If $\varepsilon > 0$, and $g \in C_c^+(G)$ with $g = 1$ on $\text{supp}(f_1 + f_2)$, then the functions
    %
    \[ h = f_1 + f_2 + \varepsilon g \]
    %
    \[ h_1 = f_1/h \ \ \ \ \ h_2 = f_2/h \]
    %
    are in $C^+_0(G)$, if we define $h_i(x) = 0$ if $f_i(x) = 0$. This implies that there is a neighbourhood $V$ of $e$ such that if $x \in V$, and $y$ is arbitrary, then
    %
    \[ | h_1(xy) - h_1(y) | \leq \varepsilon\ \ \ \ \ | h_2(xy) - h_2(y) | < \varepsilon \]
    %
    If $\text{supp}(\phi) \subset V$, and $h \leq \sum c_i L_{x_i} \phi$, then
    %
    \[ f_j(x) = h(x) h_j(x) \leq \sum c_i \phi(x_i x) h_j(x) \leq \sum c_i \phi(x_i x) \left[ h_j(x_i^{-1}) + \varepsilon \right] \]
    %
    since we may assume that $x_i x \in \text{supp}(\phi) \subset V$. Then, because $h_1 + h_2 \leq 1$,
    %
    \[ (f_1: \phi) + (f_2 : \phi) \leq \sum c_j [h_1(x_j^{-1}) + \varepsilon] + \sum c_j [h_2(x_j^{-1}) + \varepsilon] \leq \sum c_j [1 + 2 \varepsilon] \]
    %
    Now we find, by taking infinums, that
    %
    \[ I_\phi(f_1) + I_\phi(f_2) \leq I_\phi(h) (1 + 2 \varepsilon) \leq [I_\phi(f_1 + f_2) + \varepsilon I_\phi(g)] [1 + 2 \varepsilon] \]
    %
    Since $g$ is fixed, and we have a bound $I_\phi(g) \leq (g: k)$, we may always find a neighbourhood $V$ (dependant on $f_1$, $f_2$) for any $\varepsilon > 0$ such that
    %
    \[ I_\phi(f_1) + I_\phi(f_2) \leq I_\phi(f_1 + f_2) + \varepsilon \]
    %
    if $\text{supp}(\phi) \subset V$.

    Now we have estimates on how well $I_\phi$ approximates a linear function, so we can apply a limiting process. Consider the product
    %
    \[ X = \prod_{f \in C^+_0(G)} [(k : f)^{-1}, (k: f_0)] \]
    %
    a compact space, by Tychonoff's theorem, consisting of $F: C_c^+(G) \to \RR$ such that $(k : f)^{-1} \leq F(f) \leq (f: k)$. For each neighbourhood $V$ of the identity, let $K(V)$ be the closure of the set of $I_\phi$ such that $\text{supp}(\phi) \subset V$. Then the set of all $K(V)$ has the finite intersection property, so we conclude there is some $I: C_c^+(G) \to \RR$ contained in $\bigcap K(V)$. This means that every neighbourhood of $I$ contains $I_\phi$ with $\text{supp}(\phi) \subset V$, for all $\phi$. This means that if $f_1, f_2 \in C_c^+(G)$, $\varepsilon > 0$, and $V$ is arbitrary, there is $\phi$ with $\text{supp}(\phi) \subset V$, and
    %
    \[ |I(f_1) - I_\phi(f_1)| < \varepsilon\ \ \ |I(f_2) - I_\phi(f_2)| < \varepsilon \]
    \[ |I(f_1 + f_2) - I_\phi(f_1 + f_2)| < \varepsilon \]
    %
    this implies that if $V$ is chosen small enough, then
    %
    \[ |I(f_1 + f_2) - (I(f_1) - I(f_2))| \leq 2 \varepsilon + |I_\phi(f_1 + f_2) - (I_\phi(f_1) + I_\phi(f_2))| < 3 \varepsilon \]
    %
    Taking $\varepsilon \to 0$, we conclude $I$ is linear. Similar limiting arguments show that $I$ is homogenous of degree 1, and commutes with all left translations. We conclude the extension of $I$ to a linear functional on $C_0(G)$ is well defined, and the Radon measure obtained by the Riesz representation theorem is a Haar measure.
\end{proof}

We shall prove that the Haar measure is unique, but first we show an incredibly useful regularity property.

\begin{prop}
    If $U$ is open, and $\mu$ is a Haar measure, then $\mu(U) > 0$. It follows that if $f$ is in $C_c^+(G)$, then $\int f d \mu > 0$.
\end{prop}
\begin{proof}
    If $\mu(U) = 0$, then for any $x_1, \dots, x_n \in G$,
    %
    \[ \mu \left( \bigcup_{i = 1}^n x_i U \right) \leq \sum_{i = 1}^n \mu(x_i U) = 0 \]
    %
    If $K$ is compact, then $K$ can be covered by finitely many translates of $U$, so $\mu(K) = 0$. But then $\mu = 0$ by regularity, a contradiction.
\end{proof}

\begin{theorem}
    Haar measures are unique up to a multiplicative constant.
\end{theorem}
\begin{proof}
    Let $\mu$ and $\nu$ be Haar measures. Fix a compact neighbourhood $V$ of the identity. If $f,g \in C_c^+(G)$, consider the compact sets
    %
    \[ A = \text{supp}(f) V \cup V \text{supp}(f)\ \ \ \ \ B = \text{supp}(g) V \cup V \text{supp}(g) \]
    %
    Then the functions $F_y(x) = f(xy) - f(yx)$ and $G_y(x) = g(xy) - g(yx)$ are supported on $A$ and $B$. There is a neighbourhood $W \subset V$ of the identity such that $\| F_y \|_\infty, \| G_y \|_\infty < \varepsilon$ if $y \in W$. Now find $h \in C_c^+(G)$ with $h(x) = h(x^{-1})$ and $\text{supp}(h) \subset W$ (take $h(x) = k(x) k(x^{-1})$ for some function $k \in C^+_c(G)$ with $\text{supp}(k) \subset W$, and $k = 1$ on a symmetric neighbourhood of the origin). Then
    %
    \begin{align*}
        \left( \int h d\mu \right) \left( \int f d\lambda \right) &= \int h(y) f(x) d\mu(y) d\lambda(x)\\
        &= \int h(y) f(yx) d\mu(y) d\lambda(x)
    \end{align*}
    %
    and
    %
    \begin{align*}
        \left( \int h d\lambda \right) \left( \int f d\mu \right) &= \int h(x) f(y) d\mu(y) d\lambda(x)\\
        &= \int h(y^{-1}x) f(y) d\mu(y) d\lambda(x)\\
        &= \int h(x^{-1}y) f(y) d\mu(y) d\lambda(x)\\
        &= \int h(y) f(xy) d\mu(y) d\lambda(x)
    \end{align*}
    %
    Hence, applying Fubini's theorem,
    %
    \begin{align*}
        \left| \int h d\mu \int f d\lambda - \int h d\lambda \int f d\mu \right| &\leq \int h(y) |F_y(x)| d\mu(y) d\lambda(x)\\
        &\leq \varepsilon \lambda(A) \int h d\mu
    \end{align*}
    %
    In the same way, we find this is also true when $f$ is swapped with $g$, and $A$ with $B$. Dividing this inequalities by $\int h d\mu \int f d\mu$, we find
    %
    \[  \left| \frac{\int f d\lambda}{\int f d\mu} - \frac{\int h d\lambda}{\int h d\mu} \right| \leq \frac{\varepsilon \lambda(A)}{\int f d\mu} \]
    %
    and this inequality holds with $f$ swapped out with $g$, $A$ with $B$. We then combine these inequalities to conclude
    %
    \[ \left| \frac{\int f d\lambda}{\int f d\mu} - \frac{\int g d\lambda}{\int g d\mu} \right| \leq \varepsilon \left[ \frac{\lambda(A)}{\int f d\mu} + \frac{\lambda(B)}{\int g d\mu} \right] \]
    %
    Taking $\varepsilon$ to zero, we find $\lambda(A), \lambda(B)$ remain bounded, and hence
    %
    \[ \frac{\int f d\lambda}{\int f d\mu} = \frac{\int g d\lambda}{\int g d\mu} \]
    %
    Thus there is a cosntant $c > 0$ such that $\int f d\lambda = c \int f d\mu$ for any function $f \in C_c^+(G)$, and we conclude that $\lambda = c \mu$.
\end{proof}

The theorem can also be proven by looking at the translation invariant properties of the derivative $f = d\mu/d\nu$, where $\nu = \mu + \lambda$ (We assume our group is $\sigma$ compact for now). Consider the function $g(x) = f(yx)$. Then
%
\[ \int_A g(x) d\nu = \int_{yA} f(x) d\nu = \mu(yA) = \mu(A) \]
%
so $g$ is derivative, and thus $f = g$ almost everywhere. Our interpretation is that for a fixed $y$, $f(yx) = f(x)$ almost everywhere with respect to $\nu$. Then (applying a discrete version of Fubini's theorem), we find that for almost all $x$ with respect to $\nu$, $f(yx) = f(x)$ holds for almost all $y$. But this implies that there exists an $x$ for which $f(yx) = f(x)$ holds almost everywhere. Thus for any measurable $A$,
%
\[ \mu(A) = \int_A f(y) d\nu(y) = f(x) \nu(A) = f(x) \mu(A) + f(x) \nu(A) \]
%
Now $(1 - f(x)) \mu(A) = f(x) \nu(A)$ for all $A$, implying (since $\mu, \nu \neq 0$), that $f(x) \neq 0,1$, and so
%
\[ \frac{1-f(x)}{f(x)} \mu(A) = \nu(A) \]
%
for all $A$. This shows the uniqueness property for all $\sigma$ compact groups. If $G$ is an arbitrary group with two measures $\mu$ and $\nu$, then there is $c$ such that $\mu = c \nu$ on every component of $G$, and thus on the union of countably many components. If $A$ intersects uncountably many components, then either $\mu(A) = \nu(A) = \infty$, or the intersection of $A$ on each set has positive measure on only countably many components, and in either case we have $\mu(A) = \nu(A)$.

\section{Fubini, Radon Nikodym, and Duality}

Before we continue, we briefly mention that integration theory is particularly nice over locally compact groups, even if we do not have $\sigma$ finiteness. This essentially follows because the component of the identity in $G$ is $\sigma$ compact (take a compact neighbourhood and its iterated multiples), hence all components in $G$ are $\sigma$ compact. The three theorems that break down outside of the $\sigma$ compact domain are Fubini's theorem, the Radon Nikodym theory, and the duality between $L^1(X)$ and $L^\infty(X)$. We show here that all three hold if $X$ is a locally compact topological group.

First, suppose that $f \in L^1(G \times G)$. Then the essential support of $f$ is contained within countably many components of $G \times G$ (which are simply products of components in $G$). Thus $f$ is supported on a $\sigma$ compact subset of $G \times G$ (as a locally compact topological group, each component of $G \times G$ is $\sigma$ compact), and we may apply Fubini's theorem on the countably many components (the countable union of $\sigma$ compact sets is $\sigma$ compact). The functions in $L^p(G)$, for $1 \leq p < \infty$, also vanish outside of a $\sigma$ compact subset (for if $f \in L^p(G)$, $|f|^p \in L^1(G)$ and thus vanishes outside of a $\sigma$ compact set). What's more, all finite sums and products of functions from these sets (in either variable) vanish outside of $\sigma$ compact subsets, so we almost never need to explicitly check the conditions for satisfying Fubini's theorem, and from now on we apply it wantonly.s

Now suppose $\mu$ and $\nu$ are both Radon measures, with $\nu \ll \mu$, and $\nu$ is $\sigma$-finite. By inner regularity, the support of $\nu$ is a $\sigma$ compact set $E$. By inner regularity, $\mu$ restricted to $E$ is $\sigma$ finite, and so we may find a Radon Nikodym derivative on $E$. This derivative can be extended to all of $G$ because $\nu$ vanishes on $G$.

Finally, we note that $L^\infty(X) = L^1(X)^*$ can be made to hold if $X$ is not $\sigma$ finite, but locally compact and Hausdorff, provided we are integrating with respect to a Radon measure $\mu$, and we modify $L^\infty(G)$ slightly. Call a set $E \subset X$ {\bf locally Borel} if $E \cap F$ is Borel whenever $F$ is Borel and $\mu(F) < \infty$. A locally Borel set is {\bf locally null} if $\mu(E \cap F) = 0$ whenever $\mu(F) < \infty$ and $F$ is Borel. We say a property holds {\bf locally almost everywhere} if it is true except on a locally null set. $f: X \to \mathbf{C}$ is {\bf locally measurable} if $f^{-1}(U)$ is locally Borel for every borel set $U \subset \mathbf{C}$. We now define $L^\infty(X)$ to be the space of all functions bounded except on a locally null set, modulo functions that are locally zero. That is, we define a norm
%
\[ \| f \|_\infty = \inf \{ c : |f(x)| \leq c\ \text{locally almost everywhere} \} \]
%
and then $L^\infty(X)$ consists of the functions that have finite norm. It then follows that if $f \in L^\infty(X)$ and $g \in L^1(X)$, then $g$ vanishes outside of a $\sigma$-finite set $Y$, so $fg \in L^1(X)$, and if we let $Y_1 \subset Y_2 \subset \dots \to Y$ be an increasing subsequence such that $\mu(Y_i) < \infty$, then $|f(x)| \leq \| f \|_\infty$ almost everywhere for $x \in Y_i$, and so by the monotone convergence theorem
%
\[ \int |fg| d\mu = \lim_{Y_i \to \infty} \int_{Y_i} |fg| d\mu \leq \| f \|_\infty \int_{Y_i} |g| d\mu \leq \| f \|_\infty \| g \|_1 \]
%
Thus the map $g \mapsto \int fg d\mu$ is a well defined, continuous linear functional with norm $\| f \|_\infty$. That $L^1(X)^* = L^\infty(X)$ follows from the decomposibility of the Carath\'{e}odory extension of $\mu$, a fact we leave to the general measure theorists.

\section{Unimodularity}

We have thus defined a left invariant measure, but make sure to note that such a function is not right invariant. We call a group who's left Haar measure is also right invariant {\bf unimodular}. Obviously all abelian groups are unimodular.

Given a fixed $y$, the measure $\mu_y(A) = \mu(Ay)$ is a new Haar measure on the space, hence there is a constant $\Delta(y) > 0$ depending only on $y$ such that $\mu(Ay) = \Delta(y) \mu(A)$ for all measurable $A$. Since $\mu(Axy) = \Delta(y) \mu(Ay) = \Delta(x) \Delta(y) \mu(A)$, we find that $\Delta(xy) = \Delta(x) \Delta(y)$, so $\Delta$ is a homomorphism from $G$ to the multiplicative group of real numbers. For any $f \in L^1(\mu)$, we have
%
\[ \int f(xy) d\mu(x) = \Delta(y^{-1}) \int f(x) d\mu(x)  \]
%
If $y_i \to e$, and $f \in C_c(G)$, then $\| R_{y_i} f - f \|_\infty \to 0$, so
%
\[ \Delta(y_i^{-1}) \int f(x) d\mu = \int f(xy_i) d\mu \to \int f(x) d\mu \]
%
Hence $\Delta(y_i^{-1}) \to 1$. This implies $\Delta$, known as the unimodular function, is a continuous homomorphism from $G$ to the real numbers. Note that $\Delta$ is trivial if and only if $G$ is unimodular.

\begin{theorem}
    Any compact group is unimodular.
\end{theorem}
\begin{proof}
    $\Delta: G \to \RR^*$ is a continuous homomorphism, hence $\Delta(G)$ is compact. But the only compact subgroup of $\RR$ is trivial, hence $\Delta$ is trivial.
\end{proof}

Let $G^c$ be the smallest closed subgroup of $G$ containing the commutators $[x,y] = xyx^{-1}y^{-1}$. It is verified to be a normal subgroup of $G$ by simple algebras.

\begin{theorem}
    If $G/G^c$ is compact, then $G$ is unimodular.
\end{theorem}
\begin{proof}
    $\Delta$ factors through $G/G^c$ since it is abelian. But if $\Delta$ is trivial on $G/G^c$, it must also be trivial on $G$.
\end{proof}

The modular function relates right multiplication to left multiplcation in the group. In particular, if $d \mu$ is a Left Haar measure, then $\Delta^{-1} d\mu$ is a right Haar measure. Hence any right Haar measure is a constant multiple of $\Delta^{-1} d\mu$. Hence the measure $\nu(A) = \mu(A^{-1})$ has a value $c$ such that for any function $f$,
%
\[ \int \frac{f(x)}{\Delta(x)} d\mu(x) = c \int f(x) d\nu(x) = c \int f(x^{-1}) d\mu \]
%
If $c \neq 1$, pick a symmetric neighbourhood $U$ such that for $x \in U$, $|\Delta(x) - 1| \leq \varepsilon |c - 1|$. Then if $f > 0$
%
\[ |c-1|\mu(U) = |c\mu(U^{-1}) - \mu(U)| = \left| \int_U [\Delta(x^{-1}) - 1] d\mu(x) \right| \leq \varepsilon \mu(U) |c-1| \]
%
A contradiction if $\varepsilon < 1$. Thus we have
%
\[ \int f(x^{-1}) d\mu(x) = \int \frac{f(x)}{\Delta(x)} d\mu(x) \]
%
A useful integration trick. When $\Delta$ is unbounded, then it follows that $L^p(\mu)$ and $L^p(\nu)$ do not consist of the same functions. There are two ways of mapping the sets isomorphically onto one another -- the map $f(x) \mapsto f(x^{-1})$, and the map $f(x) \mapsto \Delta(x)^{1/p} f(x)$.

From now on, we assume a left invariant Haar measure is fixed over an entire group. Since a Haar measure is uniquely determined up to a constant, this is no loss of generality, and we might as well denote our integration factors $d\mu(x)$ and $d\mu(y)$ as $dx$ and $dy$, where it is assumed that this integration is over the Lebesgue measure.

\section{Convolution}

If $G$ is a topological group, then $C(G)$ does not contain enough algebraic structure to identify $G$ -- for instance, if $G$ is a discrete group, then $C(G)$ is defined solely by the cardinality of $G$. The algebras we wish to study over $G$ is the space $M(G)$ of all complex valued Radon measures over $G$ and the space $L^1(G)$ of integrable functions with respect to the Haar measure, because here we can place a Banach algebra structure with an involution. We note that $L^1(G)$ can be isometrically identified as the space of all measures $\mu \in M(G)$ which are absolutely continuous with respect to the Haar measure. Given $\mu, \nu \in M(G)$, we define the convolution measure
%
\[ \int \phi d(\mu * \nu) = \int \phi(xy) d\mu(x) d\nu(y) \]
%
The measure is well defined, for if $\phi \in C_c^+(X)$ is supported on a compact set $K$, then
%
\begin{align*}
    \left| \int \phi(xy) d\mu(x) d\nu(y) \right| &\leq \int_G \int_G \phi(xy) d|\mu|(x) d|\nu|(y)\\
    &\leq \| \mu \| \| \nu \| \| \phi \|_\infty
\end{align*}
%
This defines an operation on $M(G)$ which is associative, since, by applying the associativity of $G$ and Fubini's theorem.
%
\begin{align*}
    \int \phi d((\mu * \nu) * \lambda) &= \int \int \phi(xz) d(\mu * \nu)(x) d\lambda(z)\\
    &= \int \int \int \phi((xy)z) d\mu(x) d\nu(y) d\lambda(z)\\
    &= \int \int \int \phi(x(yz)) d\mu(x) d\nu(y) d\lambda(z)\\
    &= \int \int \phi(xz) d\mu(x) d(\nu * \lambda)(z)\\
    &= \int \phi d(\mu * (\nu * \lambda))
\end{align*}
%
Thus we begin to see how the structure of $G$ gives us structure on $M(G)$. Another example is that convolution is commutative if and only if $G$ is commutative. We have the estimate $\| \mu * \nu \| \leq \| \mu \| \| \nu \|$, because of the bound we placed on the integrals above. $M(G)$ is therefore an involutive Banach algebra, which has a unit, the dirac delta measure at the identity.

As a remark, we note that involutive Banach algebras have nowhere as near a nice of a theory than that of $C^*$ algebras. $M(G)$ cannot be renormed to be a $C^*$ algebra, since every weakly convergent Cauchy sequence converges, which is impossible in a $C^*$ algebra, except in the finite dimensional case.

A {\bf discrete measure} on $G$ is a measure in $M(G)$ which vanishes outside a countable set of points, and the set of all such measures is denoted $M_d(G)$. A {\bf continuous measure} on $G$ is a measure $\mu$ such that $\mu(\{x\}) = 0$ for all $x \in G$. We then have a decomposition $M(G) = M_d(G) \oplus M_c(G)$, for if $\mu$ is any measure, then $\mu(\{x\}) \neq 0$ for at most countably many points $x$, for
%
\[ \| \mu \| \geq \sum_{x \in G} |\mu|(x) \]
%
This gives rise to a discrete measure $\nu$, and $\mu - \nu$ is continuous. If we had another decomposition, $\mu = \psi + \phi$, then $\mu(\{x\}) = \psi(\{x\}) = \nu(\{x\})$, so $\psi = \nu$ by discreteness, and we then conclude $\phi = \mu - \nu$. $M_c(G)$ is actually a closed subspace of $M(G)$, since if $\mu_i \to \mu$, and $\mu_i \in M_c(G)$, and $\| \mu_i - \mu \| < \varepsilon$, then for any $x \in G$,
%
\[ \varepsilon > \| \mu - \mu_i \| \geq |(\mu_i - \mu)(\{x\})| = |\mu(\{ x \})| \]
%
Letting $\varepsilon \to 0$ shows continuity.

The convolution on $M(G)$ gives rise to a convolution on $L^1(G)$, where
%
\[ (f*g)(x) = \int f(y) g(y^{-1}x) dy \]
%
which satisfies $\| f*g \|_1 \leq \| f \|_1 \| g \|_1$. This is induced by the identification of $f$ with $f(x) dx$, because then
%
\begin{align*}
    \int \phi (f(x) dx * g(x) dx) &= \int \int \phi(yx) f(y) g(x) dy dx\\
    &= \int \phi(y) \left( \int f(y) g(y^{-1}x) dx \right) dy
\end{align*}
%
Hence $f d\mu * g d\mu = (f * g) d\mu$. What's more,
%
\[ \| f \|_1 = \| f d\mu \| \]
%
If $\nu \in M(G)$, then we can still define $\nu * f \in L^1(G)$
%
\[ (\nu * f)(x) = \int f(y^{-1}x) d\mu(y) \]
%
which holds since
%
\[ \int \phi d(\nu * f \mu) = \int \phi(yx) f(x) d\nu(y) d\mu(x) = \int \phi(x) f(y^{-1}x) d\nu(y) d\mu(x) \]
%
If $G$ is unimodular, then we also find
%
\[ \int \phi d(f \mu * \nu) = \int \phi(yx) f(y) d\mu(y) d\nu(x) = \int \phi(x) f(y) d\mu(y) d\nu(y^{-1}x) \]
%
So we let $f * \mu(x) = \int f(y) d\mu(y^{-1}x)$.

\begin{theorem}
    $L^1(G)$ and $M_c(G)$ are closed ideals in $M(G)$, and $M_d(G)$ is a closed subalgebra.
\end{theorem}
\begin{proof}
    If $\mu_i \to \mu$, and each $\mu_i$ is discrete, the $\mu$ is discrete, because there is a countable set $K$ such that all $\mu_i$ are equal to zero outside of $K$, so $\mu$ must also vanish outside of $K$ (here we have used the fact that $M(G)$ is a Banach space, so that we need only consider sequences). Thus $M_d(G)$ is closed, and is easily verified to be subalgebra, essentially because $\delta_x * \delta_y = \delta_{xy}$. If $\mu_i \to \mu$, then $\mu_i(\{x\}) \to \mu(\{x\})$, so that $M_c(G)$ is closed in $M(G)$. If $\nu$ is an arbitrary measure, and $\mu$ is continuous, then
    %
    \[ (\mu * \nu)(\{ x \}) = \int_G \mu(\{ y \}) d\nu(y^{-1}x) = 0 \]
    \[ (\nu * \mu)(\{ x \}) = \int_G \mu(\{ y \}) d\nu(xy^{-1}) = 0 \]
    %
    so $M_c(G)$ is an ideal. Finally, we verify $L^1(G)$ is closed, because it is complete, and if $\nu \in M(G)$ is arbitrary, and if $U$ has null Haar measure, then
    %
    \[ (f dx * \nu)(U) = \int \chi_{U}(xy) f(x) dx\ d\nu(y) = \int_G \int_{y^{-1}U} f(x) dx d\nu(y) = 0 \]
    \[ (\nu * f dx)(U) = \int \chi_U(xy) d\nu(x) f(y) dy = \int_G \int_{Ux^{-1}} f(y) dy d\nu(x) = 0 \]
    %
    So $L^1(G)$ is a two-sided ideal.
\end{proof}

If we wish to integrate by right multiplication instead of left multiplication, we find by the substitution $y \mapsto xy$ that
%
\begin{align*}
    (f*g)(x) &= \int f(y) g(y^{-1}x) dy\\
    &= \int \int f(xy) g(y^{-1}) dy\\
    &= \int \int \frac{f(xy^{-1}) g(y)}{\Delta(y)} dy
\end{align*}
%
Observe that
%
\[ f*g = \int f(y) L_{y^{-1}} g\ dy \]
%
which can be interpreted as a vector valued integral, since for $\phi \in L^\infty(\mu)$,
%
\[ \int (f*g)(x) \phi(x) dx = \int f(y) g(y^{-1}x) \phi(x) dx dy \]
%
so we can see convolution as a generalized `averaging' of translate of $g$ with respect to the values of $f$. If $G$ is commutative, this is the same as the averaging of translates of $f$, but not in the noncommutative case. It then easily follows from operator computations $L_z (f*g) = (L_z f) * g$, and $R_z (f*g) = f * (R_zg)$, or from the fact that
%
\[ (f*g)(zx) = \int f(y) g(y^{-1}zx) dy = \int f(zy) g(y^{-1}x) dy = [(L_z f) * g](x) \]
\[ (f*g)(xz) = \int f(y) g(y^{-1}xz) dy = [f * (R_z g)](x) \]
%
Convolution can also be applied to the other $L^p$ spaces, but we have to be a bit more careful with our integration.

\begin{theorem}
    If $f \in L^1(G)$ and $g \in L^p(G)$, then $f*g$ is defined for almost all $x$, $f*g \in L^p(G)$, and $\| f*g\|_p \leq \|f \| \| g \|_p$. If $G$ is unimodular, then the same results hold for $g*f$, or if $G$ is not unimodular and $f$ has compact support.
\end{theorem}
\begin{proof}
    We use Minkowski's inequality to find
    %
    \begin{align*}
        \| f*g \|_p &= \left( \int \left| \int f(y) |g(y^{-1}x) dy \right|^{p} dx \right)^{1/p}\\
        &\leq \int |f(y)| \left( \int |g(y^{-1}x)|^p dx \right)^{1/p} dy\\
        &= \| f \|_1 \| g \|_p
    \end{align*}

    If $G$ is unimodular, then
    %
    \[ \| g*f \|_p = \left( \int \left| \int g(xy^{-1}) f(y) dy \right|^{p} dx \right)^{1/p} \]
    %
    and we may apply the same trick as used before.

    If $f$ has compact support $K$, then $1/\Delta$ is bounded above by $M > 0$ on $K$ and
    %
    \begin{align*}
        \| g * f \|_p &= \left( \int \left| \int \frac{ g(xy^{-1}) f(y)}{\Delta(y)} dy \right|^{p} dx \right)^{1/p}\\
        &\leq \int \left( \int \left| \frac{g(xy^{-1}) f(y)}{\Delta(y)} \right|^p dx \right)^{1/p} dy\\
        &= \| g \|_p \int_K \frac{|f(y)|}{\Delta(y)} d \mu(y)\\
        &\leq M \| g \|_p \| f \|_1
    \end{align*}
    %
    which shows that $g*f$ is defined almost everywhere.
\end{proof}

\begin{theorem}
    If $G$ is unimodular, $f \in L^p(G)$, $g \in L^q(G)$, and $p = q^*$, then $f*g \in C_0(G)$ and $\| f * g \|_\infty \leq \| f \|_p \| g \|_q$.
\end{theorem}
\begin{proof}
    First, note that
    %
    \begin{align*}
        |(f*g)(x)| &\leq \int |f(y)| |g(y^{-1}x)| dy\\
        &\leq \| f \|_p \left( \int |g(y^{-1}x)|^q dy \right)^{1/q}\\
        &= \| f \|_p \| g \|_q
    \end{align*}
    %
    For each $x$ and $y$, applying H\"{o}lder's inequality, we find
    %
    \begin{align*}
        |(f*g)(x) - (f*g)(y)| &\leq \int |f(z)| |g(z^{-1}x) - g(z^{-1}y)| dz\\
        &\leq \| f \|_p \left( \int |g(z^{-1}x) - g(z^{-1}y)|^q dz \right)^{1/q}\\
        &= \| f \|_p \left( \int |g(z) - g(zx^{-1}y)|^q dz \right)^{1/q}\\
        &= \| f \|_p \| g - R_{x^{-1}y} g \|_q
    \end{align*}
    %
    Thus to prove continuity (and in fact uniform continuity), we need only prove that $\| g - R_x g \|_q \to 0$ for $q \neq \infty$ as $x \to \infty$ or $x \to 0$. This is the content of the next lemma.
\end{proof}

We now show that the map $x \mapsto L_x$ is a continuous operation from $G$ to the weak $*$ topology on the $L_p$ spaces, for $p \neq \infty$. It is easily verified that translation is not continuous on $L_\infty$, by taking a suitable bumpy function.

\begin{theorem}
    If $p \neq \infty$, then $\| g - R_x g \|_p \to 0$ and $\| g - L_x g \|_p \to 0$ as $x \to 0$.
\end{theorem}
\begin{proof}
    If $g \in C_c(G)$, then one verifies the theorem by using left and right uniform continuity. In general, we let $g_i \in C_c(G)$ be a sequence of functions converging to $g$ in the $L_p$ norm, and we then find
    %
    \[ \| g - L_x g \|_p \leq \| g - g_i \|_p + \| g_i - L_x g_i \|_p + \| L_x (g_i - g) \|_p = 2 \| g - g_i \|_p + \| g_i - L_x g_i \|_p \]
    %
    Taking $i$ large enough, $x$ small enough, we find $\| g - L_x g \|_p \to 0$. The only problem for right translation is the appearance of the modular function
    %
    \begin{align*}
        \| R_x (g - g_i) \|_p = \frac{\| g - g_i \|_p}{\Delta(x)^{1/p}}
    \end{align*}
    %
    If we assume our $x$ values range only over a compact neighbourhood $K$ of the origin, we find that $\Delta(x)$ is bounded below, and hence $\| R_x (g - g_i) \|_p \to 0$, which effectively removes the problems in the proof.
\end{proof}

Since the map is linear, we have verified that the map $x \mapsto L_x f$ is uniformly continuous in $L^p$ for each $f \in L^p$. In the case where $p = \infty$, the same theorem cannot hold, but we have even better conditions that do not even require unimodularity.

\begin{theorem}
    If $f \in L^1(G)$ and $g \in L^\infty(G)$, then $f*g$ is left uniformly continuous, and $g*f$ is right uniformly continuous.
\end{theorem}
\begin{proof}
    We have
    %
    \[ \| L_z (f*g) - (f*g) \|_\infty = \| (L_z f - f) * g \|_\infty \leq \| L_z f - f \|_1 \| g \|_\infty \]
    %
    \[ \| R_z (g*f) - (g*f) \|_\infty = \| g * (R_z f - f) \|_\infty \leq \| g \|_\infty \| R_z f - f \|_1 \]
    %
    and both integrals converge to zero as $z \to 1$.
\end{proof}

The passage from $M(G)$ to $L^1(G)$ removes an identity from the Banach algebra in question (except if $G$ is discrete), but there is always a way to approximate an identity.

\begin{theorem}
    For each neighbourhood $U$ of the origin, pick a function $f_U \in (L^1)^+(G)$, with $\int \phi_U = 1$, $\text{supp}(f_U) \subset U$. Then if $g$ is any function in $L^p(G)$,
    %
    \[ \| f_U * g - g \|_p \to 0 \]
    %
    where we assume $g$ is left uniformly continuous if $p = \infty$, and if $f_U$ is viewed as a net with neighbourhoods ordered by inclusion. If in addition $f_U(x) = f_U(x^{-1})$, then $\| g * f_U - g \|_p \to 0$, where $g$ is right uniformly continuous for $p = \infty$.
\end{theorem}
\begin{proof}
    Let us first prove the theorem for $p \neq \infty$. If $g \in C_c(G)$ is supported on a compact $K$, and if $U$ is small enough that $|g(y^{-1}x) - g(x)| < \varepsilon$ for $y \in U$, then because $\int_U f_U(y) = 1$, and by applying Minkowski's inequality, we find
    %
    \begin{align*}
        \| f_U * g - g \|_p &= \left( \int \left| \int f_U(y) [g(y^{-1}x) - g(x)] dy \right|^p dx \right)^{1/p} \\
        &\leq \int f_U(y) \left( \int |g(y^{-1}x) - g(x)|^p dx \right)^{1/p} dy\\
        &\leq 2 \mu(K)\varepsilon \int f_U(y) dy \leq 2 \mu(K)\varepsilon
    \end{align*}
    %
    Results are then found for all of $L^p$ by taking limits. If $g$ is left uniformly continuous, then we may find $U$ such that $|g(y^{-1}x) - g(x)| < \varepsilon$ for $y \in U$ then
    %
    \[ |(f_U * g - g)(x)| = \left| \int f_U(y) [g(y^{-1}x) - g(x)] \right| \leq \varepsilon \]
    %
    For right convolution, we find that for $g \in C_c(G)$, where $|g(xy) - g(x)| < \varepsilon$ for $y \in U$, then
    %
    \begin{align*}
        \| g * f_U - g \|_p &= \left( \int \left| \int g(y) f_U(y^{-1}x) - g(x) dy \right|^p dx \right)^{1/p}\\
        &= \left( \int \left| \int [g(xy) - g(x)] f_U(y) dy \right|^p dx \right)^{1/p}\\
        &\leq \int \left( \int |g(xy) - g(x)|^p dx \right)^{1/p} f_U(y) dy\\
        &\leq \mu(K) \varepsilon \int f_U(y) (1 + \Delta(y)) dy\\
        &= \mu(K) \varepsilon + \mu(K) \varepsilon \int f_U(y) \Delta(y) dy
    \end{align*}
    %
    We may always choose $U$ small enough that $\Delta(y) < \varepsilon$ for $y \in U$, so we obtain a complete estimate $\mu(K) (\varepsilon + \varepsilon^2)$. If $g$ is right uniformly continuous, then choosing $U$ for which $|g(xy) - g(x)| < \varepsilon$, then
    %
    \[ |(g * f_U - g)(x)| = \left| \int [g(xy) - g(x)] f_U(y) dy \right| \leq \varepsilon \]
    %
    We will always assume from hereon out that the approximate identities in $L^1(G)$ are of this form.
\end{proof}

We have already obtained enough information to characterize the closed ideals of $L^1(G)$.

\begin{theorem}
    If $V$ is a closed subspace of $L^1(G)$, then $V$ is a left ideal if and only if it is closed under left translations, and a right ideal if and only if it is closed under right translations.
\end{theorem}
\begin{proof}
    If $V$ is a closed left ideal, and $f_U$ is an approximate identity at the origin, then for any $g$,
    %
    \[ \| (L_z f_U) * g - L_z g \|_1 = \| L_z (f_U * g - g) \|_1 = \| f_U * g - g \| \to 0 \]
    %
    so $L_z g \in V$. Conversely, if $V$ is closed under left translations, $g \in L^1(G)$, and $f \in V$, then
    %
    \[ g * f = \int g(y) L_{y^{-1}} f dy \]
    %
    which is in the closed linear space of the translates of $f$. Right translation is verified very similarily.
\end{proof}

\section{The Riesz Thorin Theorem}

We finalize our basic discussion by looking at convolutions of functions in $L^p * L^q$. Certainly $L^p * L^1 \subset L^p$, and $L^p * L^q \subset L^\infty$ for $q = p^*$. To prove general results, we require a foundational interpolation result.
%
\begin{theorem}
    For any $0 < \theta < 1$, and $0 < p,q \leq \infty$. If we define
    %
    \[ 1/r_\theta = (1-\theta)/p + \theta/q \]
    %
    to be the inverse interpolation of the two numbers. Then
    %
    \[ \| f \|_{r_\theta} \leq \| f \|_p^{1-\theta} \| f \|_q^\theta \]
\end{theorem}
\begin{proof}
    We apply H\"{o}lder's inequality to find
    %
    \[ \| f \|_{r_\theta} \leq \| f \|_{p/(1 - \theta)} \| f \|_{q/\theta} = \left( \int |f|^{p/(1 - \theta)} \right)^{(1-  \theta)/p} \left( \int |f|^{q/\theta} \right)^{\theta/q} \]
    %
    so it suffices to prove $\| f \|_{p/(1-\theta)} \leq \| f \|_p^{1-\theta}$, $\| f \|_{q/\theta} \leq \| f \|_q^\theta$.

    The map $x \mapsto x^p$ is concave for $0 < p < 1$, so we may apply Jensen's inequality in reverse to conclude
    %
    \[ \left( \int |f|^{p/(1 - \theta)} \right)^{(1-  \theta)/p} \leq \left( \int |f|^p \right)^{1/p} \]
\end{proof}

The Riesz Thorin interpolation theorem then implies $L^p * L^q \subset L^r$, for $p^{-1} + q^{-1} = 1 + r^{-1}$. However, these estimates only guarantee $L^1(G)$ is closed under convolution. If $G$ is compact, then $L_p(G)$ is closed under convolution for all $p$ (TODO). The $L_p$ conjecture says that this is true if and only if $G$ is compact. This was only resolved in 1990.

\section{Homogenous Spaces and Haar Measures}

The natural way for a locally compact topological group $G$ to act on a locally compact Hausdorff space $X$ is via a representation of $G$ in the homeomorphisms of $X$. We assume the action is transitive on $X$. The standard example are the action of $G$ on $G/H$, where $H$ is a closed subspace. These are effectively all examples, because if we fix $x \in X$, then the map $y \mapsto yx$ induces a continuous bijection from $G/H$ to $X$, where $H$ is the set of all $y$ for which $yx = x$. If $G$ is a $\sigma$ compact space, then this map is a homeomorphism.

\begin{theorem}
    If a $\sigma$ compact topological group $G$ has a transitive topological action on $X$, and $x \in X$, then the continuous bijection from $G/G_x$ to $X$ is a homeomorphism.
\end{theorem}
\begin{proof}
    It suffices to show that the map $\phi: G \to X$ is open, and we need only verify this for the neighbourhood basis of compact neighbourhoods $V$ of the origin by properties of the action. $G$ is covered by countably many translates $y_1V, y_2V, \dots$, and since each $\phi(y_kV) = y_k \phi(V)$ is closed (compactness), we conclude that $y_k \phi(V)$ has non-empty interior for some $y_k$, and hence $\phi(V)$ has a non-empty interior point $\phi(y_0)$. But then for any $y \in V$, $y$ is in the interior of $\phi(y V y_0^{-1}) \subset \phi(VV y_0^{-1})$, so if we fix a compact $U$, and find $V$ with $V^3 \subset U$, we have shown $\phi(U)$ is open in $X$.
\end{proof}

We shall say a space $X$ is homogenous if it is homeomorphic to $G/H$ for some group action of $G$ over $X$. The $H$ depends on our choice of basepoint $x$, but only up to conjugation, for if if we switch to a new basepoint $y$, and $c$ maps $x$ to $y$, then $ax = x$ holds if and only if $cac^{-1}y = y$. The question here is to determine whether we have a $G$-invariant measure on $X$. This is certainly not always possible. If we had a measure on $\RR$ invariant under the affine maps $ax + b$, then it would be equal to the Haar measure by uniqueness, but the Haar measure is not invariant under dilation $x \mapsto ax$.

Let $G$ and $H$ have left Haar measures $\mu$ and $\nu$ respectively, denote the projection of $G$ onto $G/H$ as $\pi: G \to G/H$, and let $\Delta_G$ and $\Delta_H$ be the respective modular functions. Define a map $P: C_c(G) \to C_c(G/H)$ by
%
\[ (Pf)(Hx) = \int_H f(xy) d\nu(y) = \int_H  \]
%
this is well defined by the invariance properties of $\nu$. $Pf$ is obviously continuous, and $\text{supp}(Pf) \subset \pi(\text{supp}(f))$. Moreover, if $\phi \in C(G/H)$ we have
%
\[ P((\phi \circ \pi) \cdot f)(Hx) = \phi(xH) \int_H f(xy) d\nu(y) \]
%
so $P((\phi \circ \pi) \cdot f) = \phi P(f)$.

\begin{lemma}
    If $E$ is a compact subset of $G/H$, there is a compact $K \subset G$ with $\pi(K) = E$.
\end{lemma}
\begin{proof}
    Let $V$ be a compact neighbourhood of the origin, and cover $E$ by finitely many translates of $\pi(V)$. We conclude that $\pi^{-1}(E)$ is covered by finitely many of the translates, and taking the intersections of these translates with $\pi^{-1}(E)$ gives us the desired $K$.
\end{proof}

\begin{lemma}
    A compact $F \subset G/H$ gives rise to a function $f \geq 0$ in $C_c(G)$ such that $Pf = 1$ on $E$.
\end{lemma}
\begin{proof}
    Let $E$ be a compact neighbourhood containing $F$, and if $\pi(K) = E$, there is a function $g \in C_c(G)$ with $g > 0$ on $K$, and $\phi \in C_c(G/H)$ is supported on $E$ and $\phi(x) = 1$ for $x \in F$, let
    %
    \[ f = \frac{\phi \circ \pi}{P g \circ \pi} g \]
    %
    Hence
    %
    \[ Pf = \frac{\phi}{Pg} Pg = \phi \]
\end{proof}

\begin{lemma}
    If $\phi \in C_c(G/H)$, there is $f \in C_c(G)$ with $Pf = \phi$, and $\pi(\text{supp} f) = \text{supp}(\phi)$, and also $f \geq 0$ if $\phi \geq 0$.
\end{lemma}
\begin{proof}
    There exists $g \geq 0$ in $C_c(G/H)$ with $Pg = 1$ on $\text{supp}(\phi)$, and then $f = (\phi \circ \pi) g$ satisfies the properties of the theorem.
\end{proof}

We can now provide conditions on the existence of a measure on $G/H$.

\begin{theorem}
    There is a $G$ invariant measure $\psi$ on $G/H$ if and only if $\Delta_G = \Delta_H$ when restricted to $H$. In this case, the measure is unique up to a common factor, and if the factor is chosen, we have
    %
    \[ \int_G f d\mu = \int_{G/H} Pf d\psi = \int_{G/H} \int_H f(xy) d\nu(y) d\psi(xH) \]
\end{theorem}
\begin{proof}
    Suppose $\psi$ existed. Then $f \mapsto \int Pf d \psi$ is a non-zero left invariant positive linear functional on $G/H$, so $\int Pf d\psi = c \int f d\mu$ for some $c > 0$. Since $P(C_c(G)) = C_c(G/H)$, we find that $\psi$ is determined up to a constant factor. We then compute, for $y \in H$,
    %
    \begin{align*}
        \Delta_G(y) \int f(x) d\mu(x) &= \int f(xy^{-1}) d\mu(x)\\
        &= \int_{G/H} \int_H f(xzy^{-1}) d\nu(z) d\psi(xH)\\
        &= \Delta_H(y) \int_{G/H} \int_H f(xz) d\nu(z) d\psi(xH)\\
        &= \Delta_H(y) \int f(x) d\mu(x)
    \end{align*}
    %
    Hence $\Delta_G = \Delta_H$. Conversely, suppose $\Delta_G = \Delta_H$. First, we claim if $f \in C_c(G)$ and $Pf = 0$, then $\int f d\mu = 0$. Indeed if $P\phi = 1$ on $\pi(\text{supp} f)$ then
    %
    \[ 0 = Pf(xH) = \int_H f(xy) d\nu(y) = \Delta_G(y^{-1}) \int_H f(xy^{-1}) d\nu(y) \]
    %
    so
    %
    \begin{align*}
        0 &= \int_G \int_H \Delta_G(y^{-1}) \phi(x) f(xy^{-1}) d\nu(y) d\mu(x)\\
        &= \int_H \int_G \phi(xy) f(x) d\mu(x) d\nu(y)\\
        &= \int_G P\phi(xH) f(x) d\mu(x)\\
        &= \int_G f(x) d\mu(x)
    \end{align*}
    %
    This implies that if $Pf = Pg$, then $\int_G f = \int_G g$. Thus the map $Pf \mapsto \int_G f$ is a well defined $G$ invariant positive linear functional on $C_c(G/H)$, and we obtain a Radon measure from the Riesz representation theorem.
\end{proof}

If $H$ is compact, then $\Delta_G$ and $\Delta_H$ are both continuous homomorphisms from $H$ to $\RR^+$, so $\Delta_G$ and $\Delta_H$ are both trivial, and we conclude a $G$ invariant measure exists on $G/H$.

\section{Function Spaces In Harmonic Analysis}

There are a couple other function spaces that are interesting in Harmonic analysis. We define $\text{AP}(G)$ to be the set of all almost periodic functions, functions $f \in L^\infty(G)$ such that $\{ L_x f : x \in G \}$ is relatively compact in $L^\infty(G)$. If this is true, then $\{ R_x f : x \in G \}$ is also relatively compact, a rather deep theorem. If we define $\text{WAP}(G)$ to be the space of weakly almost periodic functions (the translates are relatively compact in the weak topology). It is a deep fact that $\text{WAP}(G)$ contains $C_0(G)$, but $\text{AP}(G)$ can be quite small. The reason these function spaces are almost periodic is that in the real dimensional case, $\text{AP}(\RR)$ is just the closure of the set of all trigonometric polynomials.

\chapter{The Character Space}

Let $G$ be a locally compact group. A character on $G$ is a {\it continuous} homomorphism from $G$ to $\mathbf{T}$. The space of all characters of a group will be denoted $\Gamma(G)$.

\begin{example}
    Determining the characters of $\mathbf{T}$ involves much of classical Fourier analysis. Let $f: \mathbf{T} \to \mathbf{T}$ be an arbitrary continuous character. For each $w \in \mathbf{T}$, consider the function $g(z) = f(zw) = f(z)f(w)$. We know the Fourier series acts nicely under translation, telling us that
    %
    \[ \hat{g}(n) = w^n \hat{f}(n) \]
    %
    Conversely, since $g(z) = f(z)f(w)$,
    %
    \[ \hat{g}(n) = f(w) \hat{f}(n) \]
    %
    Thus $(w^n - f(w)) \hat{f}(n) = 0$ for all $w \in \mathbf{T}$, $n \in \mathbf{Z}$. Fixing $n$, we either have $f(w) = w^n$ for all $w$, or $\hat{f}(n) = 0$. This implies that if $f \neq 0$, then $f$ is just a power map for some $n \in \mathbf{Z}$.
\end{example}

\begin{example}
    The characters of $\RR$ are of the form $t \mapsto e(t\xi)$, for $\xi \in \RR$. To see this, let $e: \RR \to \mathbf{T}$ be an arbitrary character. Define
    %
    \[ F(x) = \int_0^x e(t) dt \]
    %
    Then $F'(x) = e(x)$. Since $e(0) = 1$, for suitably small $\delta$ we have
    %
    \[ F(\delta) = \int_0^\delta e(t) dt = c > 0 \]
    %
    and then it follows that
    %
    \[ F(x + \delta) - F(x) = \int_x^{x + \delta} e(t) dt = \int_0^\delta e(x + t) dt = c e(x) \]
    %
    As a function of $x$, $F$ is differentiable, and by the fundamental theorem of calculus,
    %
    \[ \frac{dF(x + \delta) - F(x)}{dt} = F'(x + \delta) - F'(x) = e(x + \delta) - e(x) \]
    %
    This implies the right side of the above equation is differentiable, and so
    %
    \[ ce'(x) = e(x + \delta) - e(x) = e(x) [e(\delta) - 1] \]
    %
    Implying $e'(x) = A e(x)$ for some $A \in \mathbf{C}$, so $e(x) = e^{Ax}$. We require that $e(x) \in \mathbf{T}$ for all $x$, so $A = \xi i$ for some $\xi \in \RR$.
\end{example}

\begin{example}
    Consider the group $\RR^+$ of positive real numbers under multiplication. The map $x \mapsto \log x$ is an isomorphism from $\RR^+$ and $\RR$, so that every character on $\RR^+$ is of the form $e(s \log x) = x^{is}$, for some $s \in \RR$. The character group is then $\RR$, since $x^{is} x^{is'} = x^{i(s + s')}$.
\end{example}

There is a connection between characters on $G$ and characters on $L^1(G)$ that is invaluable to the generalization of Fourier analysis to arbitrary groups.

\begin{theorem}
    For any character $\phi: G \to \mathbf{C}$, the map
    %
    \[ \varphi(f) = \int \frac{f(x)}{\phi(x)} dx \]
    %
    is a non-zero character on the convolution algebra $L^1(G)$, and all characters arise this way.
\end{theorem}
\begin{proof}
    The induced map is certainly linear, and
    %
    \begin{align*}
        \varphi(f * g) &= \int \int \frac{f(y) g(y^{-1}x)}{\phi(x)} dy dx\\
        &= \int \int \frac{f(y) g(x)}{\phi(y) \phi(x)} dy dx\\
        &= \int \frac{f(y)}{\phi(y)} dy \int \frac{g(x)}{\phi(x)} dx
    \end{align*}
    %
    Since $\phi$ is continuous, there is a compact subset $K$ of $G$ where $\phi > \varepsilon$ for some $\varepsilon > 0$, and we may then choose a positive $f$ supported on $K$ in such a way that $\varphi(f)$ is non-zero.

    The converse results from applying the duality theory of the $L^p$ spaces. Any character on $L^1(G)$ is a linear functional, hence is of the form
    %
    \[ f \mapsto \int f(x) \phi(x) dx \]
    %
    for some $\phi \in L^\infty(G)$. Now
    %
    \begin{align*}
        \int \int f(y) g(x) \phi(yx) dy dx &= \int \int f(y) g(y^{-1}x) \phi(x) dy dx\\
        &= \int f(x) \phi(x) dx \int g(y) \phi(y) dy\\
        &= \int f(x) g(y) \phi(x) \phi(y) dx dy
    \end{align*}
    %
    Since this holds for all functions $f$ and $g$ in $L^1(G)$, we must have $\phi(yx) = \phi(x) \phi(y)$ almost everywhere. Also
    %
    \begin{align*}
        \int \varphi(f) g(y) \phi(y) dy &= \varphi(f * g)\\
        &= \int \int g(y) f(y^{-1}x) \phi(x) dy dx\\
        &= \int \int (L_{y^{-1}} f)(x) g(y) \phi(x) dy dx\\
        &= \int \varphi(L_{y^{-1}} f) g(y) dy
    \end{align*}
    %
    which implies $\varphi(f) \phi(y) = \varphi(L_{y^{-1}} f)$ almost everywhere. Since the map $\varphi(L_{y^{-1}} f)/\varphi(f)$ is a uniformly continuous function of $y$, $\phi$ is continuous almost everywhere, and we might as well assume $\phi$ is continuous. We then conclude $\phi(xy) = \phi(x) \phi(y)$. Since $\| \phi \|_\infty = 1$ (this is the norm of any character operator on $L^1(G)$), we find $\phi$ maps into $\mathbf{T}$, for if $\| \phi(x) \| < 1$ for any particular $x$, $\| \phi(x^{-1}) \| > 1$.
\end{proof}

Thus there is a one-to-one correspondence with $\Gamma(G)$ and $\Gamma(L^1(G))$, which implies a connection with the Gelfand theory and the character theory of locally compact groups. This also gives us a locally compact topological structure on $\Gamma(G)$, induced by the Gelfand representation on $\Gamma(L^1(G))$. A sequence $\phi_i \to \phi$ if and only if
%
\[ \int \frac{f(x)}{\phi_i(x)} dx \to \int \frac{f(x)}{\phi(x)} dx \]
%
for all functions $f \in L^1(G)$. This actually makes the map
%
\[ (f,\phi) \mapsto \int \frac{f(x)}{\phi(x)} dx \]
%
a jointly continuous map, because as we verified in the proof above,
%
\[ \widehat{f}(\phi) \phi(y) = \widehat{L_y f}(\phi) \]
%
And the map $y \mapsto L_y f$ is a continuous map into $L^1(G)$. If $K \subset G$ and $C \subset \Gamma(G)$ are compact, this allows us to find open sets in $G$ and $\Gamma(G)$ of the form
%
\[ \{ \gamma : \| 1 - \gamma(x) \| < \varepsilon\ \text{for all}\ x \in K \}\ \ \ \ \ \{ x : \| 1 - \gamma(x) \| < \varepsilon\ \text{for all}\ \gamma \in C \} \]
%
And these sets actually form a base for the topology on $\Gamma(G)$.

\begin{theorem}
    If $G$ is discrete, $\Gamma(G)$ is compact, and if $G$ is compact, $\Gamma(G)$ is discrete.
\end{theorem}
\begin{proof}
    If $G$ is discrete, then $L^1(G)$ contains an identity, so $\Gamma(G) = \Gamma(L^1(G))$ is compact. Conversely, if $G$ is compact, then it contains the constant $1$ function, and
    %
    \[ \widehat{1}(\phi) = \int \frac{dx}{\phi(x)} \]
    %
    And
    %
    \[ \frac{1}{\phi(y)} \widehat{1}(\phi) = \int \frac{dx}{\phi(yx)} = \int \frac{dx}{\phi(x)} = \hat{1}(\phi) \]
    %
    So either $\phi(y) = 1$ for all $y$, and it is then verified by calculation that $\widehat{1}(\phi) = 1$, or $\widehat{1}(\phi) = 0$. Since $\widehat{1}$ is continuous, the trivial character must be an open set by itself, and hence $\Gamma(G)$ is discrete.
\end{proof}

Given a function $f \in L^1(G)$, we may take the Gelfand transform, obtaining a function on $C_0(\Gamma(L^1(G)))$. The identification then gives us a function on $C_0(\Gamma(G))$, if we give $\Gamma(G)$ the topology induced by the correspondence (which also makes $\Gamma(G)$ into a topological group). The formula is
%
\[ \widehat{f}(\phi) = \phi(f) = \int \frac{f(x)}{\phi(x)} \]
%
This gives us the classical correspondence between $L^1(\mathbf{T})$ and $C_0(\mathbf{Z})$, and $L^1(\RR)$ and $C_0(\RR)$, which is just the Fourier transform. Thus we see the Gelfand representation as a natural generalization of the Fourier transform. We shall also denote the Fourier transform by $\mathcal{F}$, especially when we try and understand it's properties as an operator. Gelfand's theory (and some basic computation) tells us instantly that

\begin{itemize}
    \item $\widehat{f * g} = \widehat{f} \widehat{g}$ (The transform is a homomorphism).
    \item $\mathcal{F}$ is norm decreasing and therefore continuous: $\| \widehat{f} \|_\infty \leq \| f \|_1$.
    \item If $G$ is unimodular, and $\gamma \in \Gamma(G)$, then $(f * \gamma)(x) = \gamma(x) \widehat{f}(\gamma)$.
\end{itemize}

Whenever we integrate a function with respect to the Haar measure, there is a natural generalization of the concept to the space of all measures on $G$. Thus, for $\mu \in M(G)$, we define
%
\[ \widehat{\mu}(\phi) = \int \frac{dx}{\phi(x)} \]
%
which we call the {\bf Fourier-Stieltjes transform} on $G$. It is essentially an extension of the Gelfand representation on $L^1(G)$ to $M(G)$. Each $\widehat{\mu}$ is a bounded, uniformly continuous function on $\Gamma(G)$, because the transform is still contracting, i.e.
%
\[ \left| \int \frac{d\mu(x)}{\phi(x)} dx \right| \leq \| \mu \| \]
%
It is uniformly continuous, because
%
\[ (L_{\nu} \widehat{\mu} - \widehat{\mu})(\phi) = \int \frac{1 - \nu(x)}{\nu(x) \phi(x)} d\mu(x)  \]
%
The regularity of $\mu$ implies that there is a compact set $K$ such that $|\mu|(K^c) < \varepsilon$. If $\nu_i \to 0$, then eventually we must have $|\nu_i(x) - 1| < \varepsilon$ for all $x \in K$, and then
%
\[ |(L_{\nu} \widehat{\mu} - \widehat{\mu})(\phi)| \leq 2|\mu|(K^c) + \varepsilon \| \mu \| \leq \varepsilon(2 + \|\mu\|) \]
%
Which implies uniform continuity.

Let us consider why it is natural to generalize operators on $L^1(G)$ to $M(G)$. The first reason is due to the intuition of physicists; most of classical Fourier analysis emerged from physical considerations, and it is in this field that $L^1(G)$ is often confused with $M(G)$. Take, for instance, the determination of the electric charge at a point in space. To determine this experimentally, we take the ratio of the charge over some region in space to the volume of the region, and then we limit the size of the region to zero. This is the historical way to obtain the density of a measure with respect to the Lebesgue measure, so that the function we obtain can be integrated to find the charge over a region. However, it is more natural to avoid taking limits, and to just think of charge as an element of $M(\RR^3)$. If we consider a finite number of discrete charges, then we obtain a discrete measure, whose density with respect to the Lebesgue measure does not exist. This doesn't prevent physicists from trying, so they think of the density obtained as shooting off to infinity at points. Essentially, we obtain the Dirac Delta function as a `generalized function'. This is fine for intuition, but things seem to get less intuitive when we consider the charge on a subsurface of $\RR^3$, where the `density' is `dirac'-esque near the function, where as measure theoretically we just obtain a density with respect to the two-dimensional Hausdorff measure on the surface. Thus, when physicists discuss quantities as functions, they are really thinking of measures, and trying to take densities, where really they may not exist.

There is a more austere explanation, which results from the fact that, with respect to integration, $L^1(G)$ is essentially equivalent to $M(G)$. Notice that if $\mu_i \to \mu$ in the weak-$*$ topology, then $\widehat{\mu_i} \to \widehat{\mu}$ pointwise, because
%
\[ \int \frac{d\mu_i(x)}{\phi(x)} \to \int \frac{d\mu(x)}{\phi(x)} \]
%
(This makes sense, because weak-$*$ convergence is essentially pointwise convergence in $M(G)$). Thus the Fourier-Stietjes transform is continuous with respect to these topologies. It is the unique continuous extension of the Fourier transform, because

\begin{theorem}
    $L^1(G)$ is weak-$*$ dense in $M(G)$.
\end{theorem}
\begin{proof}
    First, note that the Dirac delta function can be weak-$*$ approximated by elements of $L^1(G)$, since we have an approximate identity in the space.

    First, note that if $\mu_i \to \mu$, then $\mu_i * \nu \to \mu * \nu$, because
    %
    \[ \int f d(\mu_i * \nu) = \int \int f(xy) d\mu_i(x) d\nu(y) \]
    %
    The functions $y \mapsto \int f(xy) d\mu_i(x)$ converge pointwise to $\int f(xy) d\mu(y)$. Since
    %
    \[ \left| \int f(xy) d\mu_i(x) \right| \leq \| f \|_1 \| \mu_i \| \]

    If $i$ is taken large enough that
\end{proof}

If $\phi_\alpha \to \phi$, in the sense that $\phi_\alpha(x) \to \phi(x)$ for all $x \in G$, then, because $\| \phi_\alpha(x) \| = 1$ for all $x$, we can apply the dominated convergence theorem on any compact subset $K$ of $G$ to conclude
%
\[ \int_K \frac{d\mu(x)}{\phi_\alpha(x)} \to \int_K \frac{d\mu(x)}{\phi(x)} \]

It is immediately verified to be a map into $L^1(\Gamma(G))$, because
%
\[ \int \left| \int \frac{d\mu(x)}{\phi(x)} \right| d\phi \leq \int \int \| \mu \| \]

The formula above immediately suggests a generalization to a transform on $M(G)$. For $\nu \in M(G)$, we define
%
\[ \mathcal{F}(\nu)(\phi) = \int \frac{d \nu}{\phi} \]
%
If $\mathcal{G}: L^1(G) \to C_0(\Gamma(G))$ is the Gelfand transform, then the transform induces a map $\mathcal{G}^* : M(\Gamma(G)) \to L^\infty(G)$.

The duality in class-ical Fourier analysis is shown through the inversion formulas. That is, we have inversion functions
%-00
\[ \mathcal{F}^{-1}(\{ a_k \}) = \sum a_k e_k(t)\ \ \ \ \ \mathcal{F}^{-1}(f)(x) = \int f(t) e(x t) \]
%
which reverses the fourier transform on $\mathbf{T}$ and $\RR$ respectively, on a certain subclass of $L^1$. One of the challenges of Harmonic analysis is trying to find where this holds for the general class of measurable functions.

The first problem is to determine surjectivity. We denote by $A(G)$ the space of all continuous functions which can be represented as the fourier transform of some function in $L^1(G)$. It is to even determine $A(\mathbf{T})$, the most basic example. $A(G)$ always separates the points of $\Gamma(G)$, by Gelfand theory, and if $G$ is unimdoular, then it is closed under conjugation. If we let $g(x) = \overline{f(x^{-1})}$, we find
%
\[ \mathcal{F}(g)(\phi) = \int \frac{g(x)}{\phi(x)} dx = \overline{ \int \frac{f(x^{-1})}{\phi(x^{-1})} dx } = \int \frac{f(x)}{\phi(x)} dx = \overline{\mathcal{F}(f)(\phi)} \]
%
so that by the Stone Weirstrass theorem $A(G)$ is dense in $C_0(\Gamma(L^1(G)))$.

\chapter{Banach Algebra Techniques}

In the mid 20th century, it was realized that much of the analytic information about a topological group can be captured in various $C^*$ algebras related to the group. For instance, consider the Gelfand space of $L^1(\mathbf{Z})$ is $\mathbf{T}$, which represents the fact that one can represent functions over $\mathbf{T}$ as sequences of numbers. Similarily, we find the characters of $L^1(\RR)$ are the maps $f \mapsto \widehat{f}(x)$, so that the Gelfand space of $\RR$ is $\RR$, and the Gelfand transform is the Fourier transform on this space. For a general $G$, we may hope to find a generalized Fourier transform by understanding the Gelfand transform on $L^1(G)$. We can also generalize results by extending our understanding to the class $M(G)$ of regular, Borel measures on $G$.

\chapter{Vector Spaces}

If $\mathbf{K}$ is a closed, multiplicative subgroup of the complex numbers, then $\mathbf{K}$ is also a locally compact abelian group, and we can therefore understand $\mathbf{K}$ by looking at its dual group $\mathbf{K}^*$. The map $\langle x,y \rangle = xy$ is bilinear, in the set that it is a homomorphism in the variable $y$ for each fixed $x$, and a homomorphism in the variable $x$ for each $y$.

If $\mathbf{K}$ is a subfield of the complex numbers, then $\mathbf{K}$ is also an abelian group under addition, and we can consider the dual group $\mathbf{K}^*$. The inner product $\langle x, y \rangle = xy$ gives a continuous bilinear map $\mathbf{K} \times \mathbf{K} \to \mathbf{C}$, and therefore we can define $x^* \in \mathbf{K}^*$ by $x^*(y) = \langle x,y \rangle$. If $x^*(y) = xy = 0$ for all $y$, then in particular $x^*(1) = x$, so $x = 0$. This means that the homomorphism $\mathbf{K} \to \mathbf{K}^*$ is injective.

\chapter{Interpolation of Besov and Sobolev spaces}

An important class of operators arise as singular integrals, that is, they arise as convolution operators $T$ given by $T(f) = f * K$, where $K$ is an appropriate distribution. Taking Fourier transforms, these operators can also be defined by $\widehat{T(f)} = \widehat{f} \widehat{K}$. The function $\widehat{K}$ is known as a {\bf Fourier multiplier}, because it operates by multiplication on the frequencies of the function $f$. We say $\widehat{K}$ is a {\bf Fourier multiplier on $L^p(\RR^n)$} if $T$ is a bounded map from $S(\RR^n)$ to $L^p(\RR^n)$, under the $L^p$ norms. Such maps clearly extend uniquely to maps from $L^p(\RR^n)$ to $L^p(\RR^n)$, and so we can think of $T$ as operating by convolution on the space of $L^p$ functions. We will denote the space of all Fourier multipliers on $L^p$ by $M_p$. We define the $L^p$ norm on these distributions $K$, denoted $\| K \|_p$, to be the operator norm of the associated operator $T$.

\begin{example}
    Consider the space $M_\infty$. If $K$ is a distribution in $M_\infty$, then $\| K \|_\infty < \infty$, and since convolution commutes with translations, in the sense that $f_h * K = (f * K)_h$, then
    %
    \[ \| K \|_\infty = \sup_{f \in L^\infty(\RR^n)} \frac{|(f * K)(0)|}{\| f \|_\infty} \]
    %
    But then the map $f \mapsto (f * K)(0)$ is a bounded operator on the space of bounded continuous functions, and so the Riesz representation says there is a bounded Radon measure $\mu$ such that
    %
    \[ (f * K)(0) = \int f(-y)\; d\mu(y) \]
    %
    But now we know
    %
    \[ (f * K)(x) = (f_{-x} * K)(0) = \int f(x - y) d\mu(y) = (f * \mu)(x) \]
    %
    Thus $M_\infty$ is really just the space of all bounded Radon measures, and
    %
    \[ \| K \|_\infty = \sup_{f \in L^\infty(\RR^n)} \frac{\left| \int f(y)\; d\mu(y) \right|}{\| f \|_\infty} = \| \mu \|_1 \]
    %
    so $M_\infty$ even has the same norm as the space of all bounded Radon measures. Note that it becomes a Banach algebra under convolution of distributions, since the convolution of two bounded Radon measures is a bounded Radon measure.
\end{example}

\begin{theorem}
    For any $1 \leq p \leq \infty$, and $q = p^*$, then $M_p = M_q$.
\end{theorem}
\begin{proof}
    Let $f \in L^p$, and $g \in L^q$, then H\"{o}lder's inequality gives
    %
    \[ |(K * f * g)(0)| \leq \| K * f \|_p \| g \|_q \leq \| K \|_p \| f \|_p \| g \|_p \]
    %
    Thus $K * g \in L_q$, and that $K \in M_q$ with $\| K \|_q \leq \| K \|_p$. By symmetry, we find $\| K \|_p = \| K \|_q$.
\end{proof}

\begin{example}
    Consider $M_2$. If $K$ is a distribution with $\| f * K \|_2 \leq A \| f \|_2$, then Parsevel's inequality implies that
    %
    \[ \| \widehat{f} \widehat{K} \|_2 = \| f * K \|_2 \leq A \| f \|_2 = A \| \widehat{f} \|_2 \]
    %
    so for each $\widehat{f}$, TODO: PROVE THAT THIS IS REALLY JUST THE SPACE $L^\infty(\RR^n)$, with the supremum norm. Note that this is also a Banach algebra under pointwise multiplication.
\end{example}

Using the Riesz-Thorin interpolation theorem, we find that if $1/p = (1 - \theta)/p_0 + \theta/p_1$, then $\| K \|_p \leq \| K \|_{p_0}^{1 - \theta} \| K \|_{p_1}^\theta$, when $K$ lies in the three spaces. In particular, $\| K \|_p$ is a decreasing function of $p$ for $1 \leq p \leq 2$, so we find $M_1 \subset M_p \subset M_q \subset M_2$ for $1 \leq p < q \leq 2$. In particular, all Fourier multipliers can be viewed as Fourier multipliers with respect to bounded, measurable functions on $L^\infty$. Riesz interpolation shows that each $M_p$ is a Banach algebra under multiplication in the frequency domain, or convolution in the spatial domain.

\begin{theorem}
    Let $T: \RR^n \to \RR^m$ be a surjective affine transformation. Then the endomorphism $T^*$ on $M_p(\RR^n)$ defined by $(T^* f)(\xi) = f(T(\xi))$ is an isometry, and if $T$ is a bijection, so too is $T^*$.
\end{theorem}
\begin{proof}
    TODO
\end{proof}

The next theorem is the main tool to prove results about Sobolev and Besov space. Note that it assumes $1 < p < \infty$, and cannot be applied for $p = 1$ or $p = \infty$. The proof relies on two lemmas, the first of which is used frequently later, and the second is used universally in modern harmonic analysis.

\begin{lemma}
    There exists a Schwartz function $\varphi$ on $\RR^n$ which is supported on the annulus
    %
    \[ \{ \xi: 1/2 \leq |\xi| \leq 2 \} \]
    %
    is positive for $1/2 < |\xi| < 2$, and satisfies
    %
    \[ \sum_{k = -\infty}^\infty \varphi(2^{-k} \xi) = 1 \]
    %
    for all $\xi \neq 0$.
\end{lemma}

\begin{lemma}[Calderon-Zygmund Decomposition]
    Let $f \in L^1(\RR^n)$, and $\sigma > 0$. Then there are pairwise almost disjoint cubes $I_1, I_2, \dots$ with edges parallel to the coordinate axis and
    %
    \[ \sigma < \frac{1}{|I_n|} \int_{I_n} |f(x)|\; dx \leq 2^n \sigma \]
    %
    and with $|f(x)| \leq \sigma$ for almost all $x$ outside these cubes.
\end{lemma}

\begin{theorem}[The Mihlin Multiplier Theorem]
    Let $m$ be a bounded function on $\RR^n$ which is smooth except possibly at the origin, such that
    %
    \[ \sup_{\substack{\xi \in \RR^n\\|\alpha| \leq L}} |\xi|^{|\alpha|} |(D^\alpha m)(x)| < \infty \]
    %
    Then $m$ is an $L^p$ Fourier multiplier for $1 < p < \infty$.
\end{theorem}

\section{Besov Spaces}

Recall the Schwarz function $\varphi$ used to prove the Mihlin multiplier theorem. We now define functions $\varphi_k$ such that
%
\[ \widehat{\varphi_n}(\xi) = \varphi(2^{-n} \xi)\ \ \ \ \ \ \ \ \widehat{\psi}(\xi) = 1 - \sum_{n = 1}^\infty \varphi(2^{-n} \xi) \]
%
Thus $\varphi_n$ essentially covers the annulus $2^{n-1} \leq |\xi| \leq 2^{n+1}$, and the function $\psi$ covers the remaining low frequency parts covered in the frequency ball of radius 2. We have
%
\[ \varphi_n(\xi) = \widecheck{\varphi_{2^{-n}}}(\xi) = 2^{dn} \widecheck{\varphi}(2^n \xi) \]
%
Given $s \in \RR$, and $1 \leq p, q \leq \infty$, we write
%
\[ \| f \|_{pq}^s = \| \psi * f \|_p + \left( \sum_{n = 1}^\infty (2^{sn} \| \varphi_k * f \|_p)^q \right)^{1/q} \]
%
The convolution $\varphi_n * f$ essentially captures the portion of $f$ whose frequencies lie in the annulus $2^{n-1} \leq |\xi| \leq 2^{n+1}$

\section{Proof of The Projection Result}

As with Marstrand's projection theorem, we require an energy integral variant. Rather than considering the Riesz kernel on $\RR^n$, we consider the kernel on balls
%
\[ K_\alpha(x) = \frac{\chi_{B(0,R)}(x)}{|x|^\alpha} \]
%
where $R$ is a fixed radius. If $\alpha < \beta$, and $\mu$ is measure supported on a $\beta$ dimensional subset of $\RR^n$, then $\mu * K_\alpha \in L^\infty(\RR^d)$ because $\mu$ cancels out the singular part of $K_\alpha$. Assuming $\beta < d$, we conclude $\mu * K_\alpha \in L^1(\RR^d)$. Applying interpolation (TODO: Which interpolation), we conclude that $\nu * K_\rho$ 

\chapter{The Cap Set Problem}

The cap set problem comes out of additive combinatorics, whose goal is to understand additive structure in some abelian group, typically the integers. For instance, we can think of a set $A$  as being roughly closed under addition if $|A+A| = O(|A|)$. Over rings, we can study the interplay between additive and multiplicative structure. For instance, one conjecture of Erd\"{o}s and Szemer\'{e}di says that if $A$ is a finite subset of real numbers, then $\text{max}(|A+A|,|A \cdot A|) \gtrsim |A|^{1+c}$ for some positive $c \in (0,1)$. The best known $c$ so far is $c \sim 1/3$, though it is conjectured that we can take $c$ arbitrarily close to $1$. This can be seen as a discrete version of the results of Bourgain and Edgar-Miller on the Hausdorff dimensions of Borel subrings.

\begin{theorem}[Van Der Waerden - 1927]
    For any positive integes $r$ and $k$, there is $N$ such that if the integers in $[1,N]$ are given an $r$ coloring, then there is a monochromatic $k$ term arithmetic progression.
\end{theorem}

The coloring itself is not so important, more just the partitioning. We just pidgeonhole, using the density of $k$ term arithmetic progressions. This problem suggests the Ramsey type problem of determining the largest set $A$ of the integers $[1,N]$ which does not contain $k$ term arithmetic progressions. Behrend's theorem says we can choose $A$ to be on the order of $N\exp(-c \sqrt{\log N})$.

\begin{theorem}[Roth - 1956] If $A$ is a set of integers in $[1,N]$ which is free of three term arithmetic progressions, then $|A| = O(N/\log \log N)$.
\end{theorem}

Szemer\'{e}di proved that if $A$ is free of $k$ term arithmetic progressions, $|A| = o(N)$. If Erd\"{o}s Turan, if $\sum_{x \in X} 1/x$ diverges, then $X$ contains arbitrarily long arithmetic progressions. For now, we'll restrict our attention to three term arithmetic progressions. Heath and Brown showed that three term arithmetic progresisons are $O(N/(\log N)^c)$ for some constant $c$. In 2016, the best known bound was given by Bloom, given $O(N(\log \log N)^4/\log N)$.

One way we can simplify our problem is to note that avoiding three term arithmetic progressions is a local issue, so we can embed $[1,N]$ in $\mathbf{Z}/M\mathbf{Z}$ for suitably large $M$, and we lose none of the problems we had over the integers. A heuristic is that it is easier to solve these kind of problems in $\mathbf{F}_p^n$, where $p$ is small and $n$ is large, which should behave like $\{ 1, \dots, p^n \}$. This leads naturally to the cap set problem.

\begin{theorem}[Cap Set Problem]
    What is the largest subset of $\mathbf{F}_3^n$ containing no three term arithmetic progressions?
\end{theorem}

We look at $\mathbf{F}_3$ because it is the smallest case where three term arithmetic progressions become important.

\begin{theorem}[Meschulam - 1995]
    Let $A \subset \mathbf{F}_3^n$ be a cap set. Then $|A| = O(3^n/n)$. This is analogous to a $N/\log N$ case over the integers, giving evidence that the finite field case is easier.
\end{theorem}

In 2012, Bateman and Katz showed $|A| = O(3^n/n^{1 + \varepsilon})$ for some $c > 0$. This was a difficult proof. In 2016, there was a more significant breakthrough, which gave an easy proof using the polynomial method of an exponentially small bound of $c^n$, where $c < 4$, over $\mathbf{Z}/4\mathbf{Z}$, and a week later Ellenberg-Gijswijt used this argument in the $\mathbf{F}_3$ case to prove that if $A$ is a capset in $\mathbf{F}_3$,then $|A| = O(c^n)$, for $c = 2.7551\dots$.

The idea of the polynomial method is to take combinatorial information about some set, encode it as some algebraic structura information, and then apply the theory of polynomials to encode this algebraic information and use it to limit and enable certain properties to occur.

If $V$ is the space of polynomials of degree $d$ vanishing on a set $A$, then we know $\dim V \geq \dim \mathcal{P}_d - |A|$. This gives a lower bound on the size of $A$, whereas we want a lower bound. To get an upper bound, we take $|A|^c$ instead, which shows
%
\[ \dim V \geq \dim \mathcal{P}_d + |A| - 3^n \]
%
whichs gives $|A| \leq 3^n + \dim V - \dim \mathcal{P}_d$. Now using linear algebra, we can find a polynomial $P$ vanishing on $A^c$ with support of cardinality greater than or equal to $\dim V$, hence
%
\[ |A| \leq 3^n - \dim \mathcal{P}_d + \max |\text{supp}(P)| \]
%
It follows that $A$ is a cap set if and only if $x + y = 2z$, or $x + y + z = 0$ holds if and only if $x = y = z$. This is an algebraic property which says directly that $A$ has no nontrivial three term arithmetic progressions. Thus for any $a_1, \dots, a_m \in A$, $P(-a_i-a_j) = 0$ when $i \neq j$. Equivalently, this means $P(-a_i-a_j) \neq 0$ when $i = j$. This suggests we consider the $|A|$ by $|A|$ matrix $M$ with $M_{ij} = P(-a_i-a_j)$. This is a diagonal matrix, with $M_{ii} = P(a_i)$. Thus the rank of this matrix is the dimension of the support of $P$, so it suffices to upper bound the rank of $M$. The key observation, where we now explicitly employ the fact that $P$ is a polynomial, is that $P(-x-y)$ is a polynomial in $2n$ variables $x,y \in \mathbf{F}_3^n$,











