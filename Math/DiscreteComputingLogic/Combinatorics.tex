\documentclass[12pt, dvipsnames]{report}

\usepackage{amsmath}
\usepackage{algorithm}
%\usepackage{algorithmic}
\usepackage[noend]{algpseudocode}

\usepackage{amsmath}
\usepackage{amssymb}
\usepackage{amsthm}
\usepackage{amsopn}

\usepackage{kpfonts}

\usepackage{graphicx}

% Probably don't need this on notes anymore
%\usepackage{kbordermatrix}

% Standard tool for drawing diagrams.
\usepackage{tikz}
\usepackage{tkz-berge}
\usepackage{tikz-cd}
\usepackage{tkz-graph}

\usepackage{comment}

%
\usepackage{multicol}

%
\usepackage{framed}

%
\usepackage{mathtools}

%
\usepackage{float}

%
\usepackage{subfig}

%
\usepackage{wrapfig}

%
\let\savewideparen\wideparen
\let\wideparen\relax
\usepackage{mathabx}
\let\wideparen\savewideparen

% Used for generating `enlightening quotes'
\usepackage{epigraph}

% Forget what this is used for :P
\usepackage[utf8]{inputenc}

% Used for generating quotes.
\usepackage{csquotes}

% Allows what to generate links inside
% generated pdf files
\usepackage{hyperref}

% Allows one to customize theorem
% environments in mathematical proofs.
\usepackage{thmtools}

% Gives access to a proof
\usepackage{lplfitch}

% I forget what this is for.
\usepackage{accents}

% A package for drawing simple trees,
% as a substitute for unnesacary TIKZ code
\usepackage{qtree}

% Enables sequent calculus proofs
\usepackage{ebproof}

% For braket notation
\usepackage{braket}

% To change line spacing when using mathematical notations which require some height!
\usepackage{setspace}

%\usepackage[dvipsnames]{xcolor}

\usepackage{float}

% For block commenting
\usepackage{comment}




\setlength\epigraphwidth{8cm}

\usetikzlibrary{arrows, petri, topaths, decorations.markings}

% So you can do calculations in coordinate specifications
\usetikzlibrary{calc}
\usetikzlibrary{angles}

\theoremstyle{plain}
\newtheorem{theorem}{Theorem}[chapter]
\newtheorem{axiom}{Axiom}
\newtheorem{lemma}[theorem]{Lemma}
\newtheorem{corollary}[theorem]{Corollary}
\newtheorem{prop}[theorem]{Proposition}
\newtheorem{exercise}{Exercise}[chapter]
\newtheorem{fact}{Fact}[chapter]

\newtheorem*{example}{Example}
\newtheorem*{proof*}{Proof}

\theoremstyle{remark}
\newtheorem*{exposition}{Exposition}
\newtheorem*{remark}{Remark}
\newtheorem*{remarks}{Remarks}

\theoremstyle{definition}
\newtheorem*{defi}{Definition}

\usepackage{hyperref}
\hypersetup{
    colorlinks = true,
    linkcolor = black,
}

\usepackage{textgreek}

\makeatletter
\renewcommand*\env@matrix[1][*\c@MaxMatrixCols c]{%
  \hskip -\arraycolsep
  \let\@ifnextchar\new@ifnextchar
  \array{#1}}
\makeatother

\renewcommand*\contentsname{\hfill Table Of Contents \hfill}

\newcommand{\optionalsection}[1]{\section[* #1]{(Important) #1}}
\newcommand{\deriv}[3]{\left. \frac{\partial #1}{\partial #2} \right|_{#3}} % partial derivative involving numerator and denominator.
\newcommand{\lcm}{\operatorname{lcm}}
\newcommand{\im}{\operatorname{im}}
\newcommand{\bint}{\mathbf{Z}}
\newcommand{\gen}[1]{\langle #1 \rangle}

\newcommand{\End}{\operatorname{End}}
\newcommand{\Mor}{\operatorname{Mor}}
\newcommand{\Id}{\operatorname{id}}
\newcommand{\visspace}{\text{\textvisiblespace}}
\newcommand{\Gal}{\text{Gal}}

\newcommand{\xor}{\oplus}
\newcommand{\ft}{\wedge}
\newcommand{\ift}{\vee}

\newcommand{\prob}{\mathbf{P}}
\newcommand{\expect}{\mathbf{E}}
\DeclareMathOperator{\Var}{\mathbf{V}}
\newcommand{\Ber}{\text{Ber}}
\newcommand{\Bin}{\text{Bin}}

%\newcommand{\widecheck}[1]{{#1}^{\ft}}

\DeclareMathOperator{\diam}{\text{diam}}

\DeclareMathOperator{\QQ}{\mathbf{Q}}
\DeclareMathOperator{\ZZ}{\mathbf{Z}}
\DeclareMathOperator{\RR}{\mathbf{R}}
\DeclareMathOperator{\HH}{\mathbf{H}}
\DeclareMathOperator{\CC}{\mathbf{C}}
\DeclareMathOperator{\AB}{\mathbf{A}}
\DeclareMathOperator{\PP}{\mathbf{P}}
\DeclareMathOperator{\MM}{\mathbf{M}}
\DeclareMathOperator{\VV}{\mathbf{V}}
\DeclareMathOperator{\TT}{\mathbf{T}}
\DeclareMathOperator{\LL}{\mathcal{L}}
\DeclareMathOperator{\EE}{\mathbf{E}}
\DeclareMathOperator{\NN}{\mathbf{N}}
\DeclareMathOperator{\DQ}{\mathcal{Q}}
\DeclareMathOperator{\IA}{\mathfrak{a}}
\DeclareMathOperator{\IB}{\mathfrak{b}}
\DeclareMathOperator{\IC}{\mathfrak{c}}
\DeclareMathOperator{\IP}{\mathfrak{p}}
\DeclareMathOperator{\IQ}{\mathfrak{q}}
\DeclareMathOperator{\IM}{\mathfrak{m}}
\DeclareMathOperator{\IN}{\mathfrak{n}}
\DeclareMathOperator{\IK}{\mathfrak{k}}
\DeclareMathOperator{\ord}{\text{ord}}
\DeclareMathOperator{\Ker}{\textsf{Ker}}
\DeclareMathOperator{\Coker}{\textsf{Coker}}
\DeclareMathOperator{\emphcoker}{\emph{coker}}
\DeclareMathOperator{\pp}{\partial}
\DeclareMathOperator{\tr}{\text{tr}}

\DeclareMathOperator{\supp}{\text{supp}}

\DeclareMathOperator{\codim}{\text{codim}}

\DeclareMathOperator{\minkdim}{\dim_{\mathbf{M}}}
\DeclareMathOperator{\hausdim}{\dim_{\mathbf{H}}}
\DeclareMathOperator{\lowminkdim}{\underline{\dim}_{\mathbf{M}}}
\DeclareMathOperator{\upminkdim}{\overline{\dim}_{\mathbf{M}}}
\DeclareMathOperator{\lhdim}{\underline{\dim}_{\mathbf{M}}}
\DeclareMathOperator{\lmbdim}{\underline{\dim}_{\mathbf{MB}}}
\DeclareMathOperator{\packdim}{\text{dim}_{\mathbf{P}}}
\DeclareMathOperator{\fordim}{\dim_{\mathbf{F}}}

\DeclareMathOperator*{\argmax}{arg\,max}
\DeclareMathOperator*{\argmin}{arg\,min}

\DeclareMathOperator{\ssm}{\smallsetminus}

\title{Number Theory}
\author{Jacob Denson}

\begin{document}

\pagenumbering{gobble}

\maketitle

\tableofcontents

\pagenumbering{arabic}

\chapter{Generating Functions}

\begin{example}
    Suppose we are working in a country with only a one, a two, and a three penny coin. Given an integer $n$, let $r(n)$ denote the number of ways that a person can be paid $n$ pennies using these three coins. Since this is a question about the additivity of numbers, we can likely understand it using generating functions. Formally,
    %
    \[ r(n) = \# \{ (a,b,c): a + 2b + 3c = n \} \]
    %
    We note
    %
    \[ \left( \sum_{a = 0}^\infty z^a \right) \left( \sum_{b = 0}^\infty z^{2b} \right) \left( \sum_{c = 0}^\infty z^{3c} \right) = \sum_{a,b,c} z^{a + 2b + 3c} = \sum_{n = 0}^\infty r(n) z^n \]
    %
    Thus, for $|z| < 1$,
    %
    \[ \sum_{n = 0}^\infty r(n) z^n = \frac{1}{(1 - z)(1 - z^2)(1 - z^3)} \]
    %
    We can now perform a partial fraction decomposition, writing
    %
    \[ \frac{1}{(1-z)(1 - z^2)(1 - z^3)} = \frac{1}{(1-z)^3(1+z)(\omega - z)(\omega + z)} \]
    %
    where $\omega = e(1/3)$ is a primitive third root of unity. Some intense linear algebra shows this is equal to
    %
    \[ \frac{z + 2}{9(z^2 + z + 1)} + \frac{17 z^2 - 52 z + 47}{72 (1 - z)^3} + \frac{1}{8(1 + z)} \]
    %
    which can be further decomposed into
    %
    \begin{align*}
        - &\frac{\omega^2 + 3 \omega + 2}{9(1 - z/\omega)} + \frac{\omega^2 - \omega + 2}{9(1 - z/\omega^2)}\\
        &+ \frac{1}{6(1 - z)^3} + \frac{1}{4(1 - z)^2} + \frac{17}{72(1 - z)} + \frac{1}{8(1 + z)}
    \end{align*}
    %
    where $\omega = e(1/3)$. Taking power series and summing up, we find
    %
    \begin{align*}
        r(n) &= - \frac{\omega^2 + 3\omega + 2}{9 \omega^n} + \frac{\omega^2 - \omega + 2}{9 \omega^{2n}} + \frac{(n + 1)(n + 2)}{12} + \frac{n+1}{4} + \frac{17}{72} + \frac{(-1)^n}{8}\\
        &= \frac{6n^2 + 36n + 47 + 9(-1)^n}{72} + \begin{cases} 0 & n \equiv 0 (\text{mod}\ 3) \\ -2/9 & n \equiv 1\ \text{or}\ 2 (\text{mod}\ 3) \end{cases}\\
        &= \frac{(n + 3)^2}{12} + \frac{9 (-1)^n - 7}{72} + \begin{cases} 0 & n \equiv 0 (\text{mod}\ 3) \\ -16/72 & n \equiv 1\ \text{or}\ 2 (\text{mod}\ 3) \end{cases}    \end{align*}
        %
        We know $r(n)$ is an integer, and since
        %
        \[ \frac{9 + 7 + 16}{72} = \frac{32}{72} < \frac{1}{2} \]
        %
        So $r(n)$ is the closest integer to $(n + 3)^2/12$.
\end{example}

\end{document}