\documentclass[12pt, dvipsnames]{report}

\usepackage{amsmath}
\usepackage{algorithm}
%\usepackage{algorithmic}
\usepackage[noend]{algpseudocode}

\usepackage{amsmath}
\usepackage{amssymb}
\usepackage{amsthm}
\usepackage{amsopn}

\usepackage{kpfonts}

\usepackage{graphicx}

% Probably don't need this on notes anymore
%\usepackage{kbordermatrix}

% Standard tool for drawing diagrams.
\usepackage{tikz}
\usepackage{tkz-berge}
\usepackage{tikz-cd}
\usepackage{tkz-graph}

\usepackage{comment}

%
\usepackage{multicol}

%
\usepackage{framed}

%
\usepackage{mathtools}

%
\usepackage{float}

%
\usepackage{subfig}

%
\usepackage{wrapfig}

%
\let\savewideparen\wideparen
\let\wideparen\relax
\usepackage{mathabx}
\let\wideparen\savewideparen

% Used for generating `enlightening quotes'
\usepackage{epigraph}

% Forget what this is used for :P
\usepackage[utf8]{inputenc}

% Used for generating quotes.
\usepackage{csquotes}

% Allows what to generate links inside
% generated pdf files
\usepackage{hyperref}

% Allows one to customize theorem
% environments in mathematical proofs.
\usepackage{thmtools}

% Gives access to a proof
\usepackage{lplfitch}

% I forget what this is for.
\usepackage{accents}

% A package for drawing simple trees,
% as a substitute for unnesacary TIKZ code
\usepackage{qtree}

% Enables sequent calculus proofs
\usepackage{ebproof}

% For braket notation
\usepackage{braket}

% To change line spacing when using mathematical notations which require some height!
\usepackage{setspace}

%\usepackage[dvipsnames]{xcolor}

\usepackage{float}

% For block commenting
\usepackage{comment}




\setlength\epigraphwidth{8cm}

\usetikzlibrary{arrows, petri, topaths, decorations.markings}

% So you can do calculations in coordinate specifications
\usetikzlibrary{calc}
\usetikzlibrary{angles}

\theoremstyle{plain}
\newtheorem{theorem}{Theorem}[chapter]
\newtheorem{axiom}{Axiom}
\newtheorem{lemma}[theorem]{Lemma}
\newtheorem{corollary}[theorem]{Corollary}
\newtheorem{prop}[theorem]{Proposition}
\newtheorem{exercise}{Exercise}[chapter]
\newtheorem{fact}{Fact}[chapter]

\newtheorem*{example}{Example}
\newtheorem*{proof*}{Proof}

\theoremstyle{remark}
\newtheorem*{exposition}{Exposition}
\newtheorem*{remark}{Remark}
\newtheorem*{remarks}{Remarks}

\theoremstyle{definition}
\newtheorem*{defi}{Definition}

\usepackage{hyperref}
\hypersetup{
    colorlinks = true,
    linkcolor = black,
}

\usepackage{textgreek}

\makeatletter
\renewcommand*\env@matrix[1][*\c@MaxMatrixCols c]{%
  \hskip -\arraycolsep
  \let\@ifnextchar\new@ifnextchar
  \array{#1}}
\makeatother

\renewcommand*\contentsname{\hfill Table Of Contents \hfill}

\newcommand{\optionalsection}[1]{\section[* #1]{(Important) #1}}
\newcommand{\deriv}[3]{\left. \frac{\partial #1}{\partial #2} \right|_{#3}} % partial derivative involving numerator and denominator.
\newcommand{\lcm}{\operatorname{lcm}}
\newcommand{\im}{\operatorname{im}}
\newcommand{\bint}{\mathbf{Z}}
\newcommand{\gen}[1]{\langle #1 \rangle}

\newcommand{\End}{\operatorname{End}}
\newcommand{\Mor}{\operatorname{Mor}}
\newcommand{\Id}{\operatorname{id}}
\newcommand{\visspace}{\text{\textvisiblespace}}
\newcommand{\Gal}{\text{Gal}}

\newcommand{\xor}{\oplus}
\newcommand{\ft}{\wedge}
\newcommand{\ift}{\vee}

\newcommand{\prob}{\mathbf{P}}
\newcommand{\expect}{\mathbf{E}}
\DeclareMathOperator{\Var}{\mathbf{V}}
\newcommand{\Ber}{\text{Ber}}
\newcommand{\Bin}{\text{Bin}}

%\newcommand{\widecheck}[1]{{#1}^{\ft}}

\DeclareMathOperator{\diam}{\text{diam}}

\DeclareMathOperator{\QQ}{\mathbf{Q}}
\DeclareMathOperator{\ZZ}{\mathbf{Z}}
\DeclareMathOperator{\RR}{\mathbf{R}}
\DeclareMathOperator{\HH}{\mathbf{H}}
\DeclareMathOperator{\CC}{\mathbf{C}}
\DeclareMathOperator{\AB}{\mathbf{A}}
\DeclareMathOperator{\PP}{\mathbf{P}}
\DeclareMathOperator{\MM}{\mathbf{M}}
\DeclareMathOperator{\VV}{\mathbf{V}}
\DeclareMathOperator{\TT}{\mathbf{T}}
\DeclareMathOperator{\LL}{\mathcal{L}}
\DeclareMathOperator{\EE}{\mathbf{E}}
\DeclareMathOperator{\NN}{\mathbf{N}}
\DeclareMathOperator{\DQ}{\mathcal{Q}}
\DeclareMathOperator{\IA}{\mathfrak{a}}
\DeclareMathOperator{\IB}{\mathfrak{b}}
\DeclareMathOperator{\IC}{\mathfrak{c}}
\DeclareMathOperator{\IP}{\mathfrak{p}}
\DeclareMathOperator{\IQ}{\mathfrak{q}}
\DeclareMathOperator{\IM}{\mathfrak{m}}
\DeclareMathOperator{\IN}{\mathfrak{n}}
\DeclareMathOperator{\IK}{\mathfrak{k}}
\DeclareMathOperator{\ord}{\text{ord}}
\DeclareMathOperator{\Ker}{\textsf{Ker}}
\DeclareMathOperator{\Coker}{\textsf{Coker}}
\DeclareMathOperator{\emphcoker}{\emph{coker}}
\DeclareMathOperator{\pp}{\partial}
\DeclareMathOperator{\tr}{\text{tr}}

\DeclareMathOperator{\supp}{\text{supp}}

\DeclareMathOperator{\codim}{\text{codim}}

\DeclareMathOperator{\minkdim}{\dim_{\mathbf{M}}}
\DeclareMathOperator{\hausdim}{\dim_{\mathbf{H}}}
\DeclareMathOperator{\lowminkdim}{\underline{\dim}_{\mathbf{M}}}
\DeclareMathOperator{\upminkdim}{\overline{\dim}_{\mathbf{M}}}
\DeclareMathOperator{\lhdim}{\underline{\dim}_{\mathbf{M}}}
\DeclareMathOperator{\lmbdim}{\underline{\dim}_{\mathbf{MB}}}
\DeclareMathOperator{\packdim}{\text{dim}_{\mathbf{P}}}
\DeclareMathOperator{\fordim}{\dim_{\mathbf{F}}}

\DeclareMathOperator*{\argmax}{arg\,max}
\DeclareMathOperator*{\argmin}{arg\,min}

\DeclareMathOperator{\ssm}{\smallsetminus}

\title{Electromagnetics}
\author{Jacob Denson}

\begin{document}

\chapter{Electrostatics}

We begin with the most important observation of electromagnetics. Physical bodies possess a quantity called \emph{electrical charge}. This quantity can be positive, negative, or zero. It's important lies in the fact that this charge determines a force between objects, known as the \emph{electrical force}, which is proportional to the electrical charge that objects possess. This force is incredibly strong relative to gravity, namely, a \emph{billion billion billion billion} times stronger. But many objects, en masse, possess very small electrical charge, and so we do not notice this force in everyday life, until it does contain such a charge, and then we see very powerful effects, such as the ability of objects to \emph{float in the air} (like a maglev train, or your hair after rubbing a balloon). But despite this peculiarity, these forces are responsible for the behaviour of many physical objects to behave like they do intuitive, i.e. for a collective body of a huge number of particles to be held together so rigidly that we can picture them as behaving as if they were a single object.

The next law we observe is \emph{Coulomb's Law}. It states that the electrical force acts between two charged particles, pointing on the line between these two particles, and is proportional to the product of their charges, and inversely proportional to the square of the distance between the particles. The force is attractive if the charges have opposite sign, and repulsive if the charges have the same sign. Thus, given two point masses lying at points $x_1$ and $x_2$, and with charges $q_1$ and $q_2$, the force between them is proportional to
%
\[ \frac{q_1 q_2}{|x_1 - x_2|^2} \]
%
In the metric system, where charge is measure in Coulombs (with units $C$), the constant of proportionality here is denoted by $1/(4 \pi \varepsilon_0)$, where $\varepsilon_0$ is called the \emph{permittivity of free space}, and has value $8.854 \times 10^{-12} C^2 / N m^2$.

Often we specify electromagnetic force in terms of an electromagnetic field. We consider a distribution of charge fixed in space. This distribution induces a vector-valued function $E$ such that a new point mass added to the distribution, with charge $q$ will experience a force $q \cdot E(x)$ if it is placed at a point $x$. Thus $E$ has units of force per unit charge. Given a charge distribution $\rho$, the electromagnetic field will then be
%
\[ E(x) = \frac{1}{4 \pi \varepsilon_0} \int \rho(y) \cdot \frac{x - y}{|x - y|^3}\; dy = \frac{1}{\varepsilon_0} (\rho * \nabla \Phi). \]
%
The main advantage of the field is that all of classical electromagnetic theory can be codified in a series of equations about the electromagnetic field, called \emph{Maxwell's Equations}.

This equation, a summary of Columb's law given by integration, has a `differential form' called \emph{Gauss' law}. The equation above can be written as
%
\[ E(x) = \varepsilon_0^{-1} (\rho * \nabla \Phi) = \varepsilon_0^{-1} \nabla (\rho * \Phi), \]
%
where
%
\[ \Phi(x) = \frac{1}{4 \pi} \frac{1}{|x|} \]
%
is the fundamental solution of the Laplacian. But then
%
\[ \text{Div}(E) = \varepsilon_0^{-1} \Delta(\rho * \Phi) = \varepsilon_0^{-1} \rho. \]
%
Thus Coulomb's law can be expressed as a differential equation in terms of the electrical field. The divergence theorem gives an alternative integral formulation of the law, i.e. for any open, bounded region $\Omega$, we have
%
\[ \int_{\partial \Omega} E \cdot \widehat{n}\; dS = \frac{Q_\Omega}{\varepsilon_0}, \]
%
where $Q_\Omega$ is the total electrical charge contained in $\Omega$. If we assume that $\rho$ is concentrated at a single point $x$, with charge $q$, then we obtain the differential equation $\text{div}(E) = \varepsilon_0^{-1} \delta_x$. The only field $E$ solving this equation which is radially symmetric about $x$ is given by Coulomb's original formulation of his law. Thus Gauss' law gives as much information as Coulomb's law, \emph{provided we assume radial symmetry}. But we see that Gauss' law does not uniquely determine the field $E$ from the charge distribution without some other information. Thus we are motivated to introduce additional electrostatic laws.

Next, we discuss the \emph{circulation law}. Like gravitational force, electromagnetic force is observed to be \emph{conservative}. It follows that for any closed curve $C$,
%
\[ \int_C E \cdot dx = 0. \]
%
In differential form, by Stoke's theorem, we conclude that the \emph{curl} of $E$ is zero, i.e. $\nabla \times E = 0$.

If $E$ does not have singularities, then it follows from Poincar\'{e} Lemma that there exists a \emph{potential} $V$ such that $E = \nabla V$, and in general, $V$ is defined locally in a neighborhood of a point lying away from singularities of $E$. Gauss' law then becomes $\text{Div}(\nabla V) = \varepsilon_0^{-1} \rho$, or $\Delta V = \varepsilon_0^{-1} \rho$. This is \emph{Poisson's equation}. In a region not containing any electrical charge, Poisson's equation reads $\Delta V = 0$, which is \emph{Laplace's equation}.






% Divergence is the infinitisimal ratio of flux to the volume enclosed by the flux.
    % Thus divergence zero means 'incompressible', i.e. the overall density of a volume will be maintained if it follows an incompressible vector field.
% If we have a uniform continuous mass of points, and we let that volume travel along the velocity field induced by the vector field X, then the amount of mass in a small radius region of volume V around a point X will change at roughly a rate of V div(X), and more precisely, a rate measured by the divergence theorem, i.e. the integral of V div(X) over the interior of the region.

% Given a vector field X, Curl(X) is a vector such that, given a surface with area A and normal n, the line integral of X along the boundary of the region with orientation induced by the normal is proportional to A * (Curl(X) * n). Intuitively, if a wheel is placed at a point x with normal vector n, and if the points on the wheel travel along the vector field X, then the wheel will have angular velocity of Curl(X) * n in the orientation induced by the right hand rule.

% Divergence and Curl in spherical and cylindrical coordinates.

% Divergence to continuity equation

% Ampere's Circuital Law
% If a charge q travels in a direction X, then it has Current I = q X.
% Ampere's Circuital Law says that the magnetic field curls around this current at a rate proportional to a moving current. We might write this as Curl(B) = mu_0 J, where J is the density of current, and mu_0 = 1.257 * 10^{-6} N / Amp .

% Paradox:
% Maxwell says that Div(B) = 0. Thus B = Curl(A) for some vector field A.


% Maxwell's Equations show that the electromagnetic field and magnetic field propogate at a finite speed in the absense of an electromagnetic current, i.e. they satisfy the wave equation with speed c = 1/sqrt(epsilon_0 mu_0).


\end{document}