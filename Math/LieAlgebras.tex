\documentclass[12pt, dvipsnames]{report}

\usepackage{amsmath}
\usepackage{algorithm}
%\usepackage{algorithmic}
\usepackage[noend]{algpseudocode}

\usepackage{amsmath}
\usepackage{amssymb}
\usepackage{amsthm}
\usepackage{amsopn}

\usepackage{kpfonts}

\usepackage{graphicx}

% Probably don't need this on notes anymore
%\usepackage{kbordermatrix}

% Standard tool for drawing diagrams.
\usepackage{tikz}
\usepackage{tkz-berge}
\usepackage{tikz-cd}
\usepackage{tkz-graph}

\usepackage{comment}

%
\usepackage{multicol}

%
\usepackage{framed}

%
\usepackage{mathtools}

%
\usepackage{float}

%
\usepackage{subfig}

%
\usepackage{wrapfig}

%
\let\savewideparen\wideparen
\let\wideparen\relax
\usepackage{mathabx}
\let\wideparen\savewideparen

% Used for generating `enlightening quotes'
\usepackage{epigraph}

% Forget what this is used for :P
\usepackage[utf8]{inputenc}

% Used for generating quotes.
\usepackage{csquotes}

% Allows what to generate links inside
% generated pdf files
\usepackage{hyperref}

% Allows one to customize theorem
% environments in mathematical proofs.
\usepackage{thmtools}

% Gives access to a proof
\usepackage{lplfitch}

% I forget what this is for.
\usepackage{accents}

% A package for drawing simple trees,
% as a substitute for unnesacary TIKZ code
\usepackage{qtree}

% Enables sequent calculus proofs
\usepackage{ebproof}

% For braket notation
\usepackage{braket}

% To change line spacing when using mathematical notations which require some height!
\usepackage{setspace}

%\usepackage[dvipsnames]{xcolor}

\usepackage{float}

% For block commenting
\usepackage{comment}




\setlength\epigraphwidth{8cm}

\usetikzlibrary{arrows, petri, topaths, decorations.markings}

% So you can do calculations in coordinate specifications
\usetikzlibrary{calc}
\usetikzlibrary{angles}

\theoremstyle{plain}
\newtheorem{theorem}{Theorem}[chapter]
\newtheorem{axiom}{Axiom}
\newtheorem{lemma}[theorem]{Lemma}
\newtheorem{corollary}[theorem]{Corollary}
\newtheorem{prop}[theorem]{Proposition}
\newtheorem{exercise}{Exercise}[chapter]
\newtheorem{fact}{Fact}[chapter]

\newtheorem*{example}{Example}
\newtheorem*{proof*}{Proof}

\theoremstyle{remark}
\newtheorem*{exposition}{Exposition}
\newtheorem*{remark}{Remark}
\newtheorem*{remarks}{Remarks}

\theoremstyle{definition}
\newtheorem*{defi}{Definition}

\usepackage{hyperref}
\hypersetup{
    colorlinks = true,
    linkcolor = black,
}

\usepackage{textgreek}

\makeatletter
\renewcommand*\env@matrix[1][*\c@MaxMatrixCols c]{%
  \hskip -\arraycolsep
  \let\@ifnextchar\new@ifnextchar
  \array{#1}}
\makeatother

\renewcommand*\contentsname{\hfill Table Of Contents \hfill}

\newcommand{\optionalsection}[1]{\section[* #1]{(Important) #1}}
\newcommand{\deriv}[3]{\left. \frac{\partial #1}{\partial #2} \right|_{#3}} % partial derivative involving numerator and denominator.
\newcommand{\lcm}{\operatorname{lcm}}
\newcommand{\im}{\operatorname{im}}
\newcommand{\bint}{\mathbf{Z}}
\newcommand{\gen}[1]{\langle #1 \rangle}

\newcommand{\End}{\operatorname{End}}
\newcommand{\Mor}{\operatorname{Mor}}
\newcommand{\Id}{\operatorname{id}}
\newcommand{\visspace}{\text{\textvisiblespace}}
\newcommand{\Gal}{\text{Gal}}

\newcommand{\xor}{\oplus}
\newcommand{\ft}{\wedge}
\newcommand{\ift}{\vee}

\newcommand{\prob}{\mathbf{P}}
\newcommand{\expect}{\mathbf{E}}
\DeclareMathOperator{\Var}{\mathbf{V}}
\newcommand{\Ber}{\text{Ber}}
\newcommand{\Bin}{\text{Bin}}

%\newcommand{\widecheck}[1]{{#1}^{\ft}}

\DeclareMathOperator{\diam}{\text{diam}}

\DeclareMathOperator{\QQ}{\mathbf{Q}}
\DeclareMathOperator{\ZZ}{\mathbf{Z}}
\DeclareMathOperator{\RR}{\mathbf{R}}
\DeclareMathOperator{\HH}{\mathbf{H}}
\DeclareMathOperator{\CC}{\mathbf{C}}
\DeclareMathOperator{\AB}{\mathbf{A}}
\DeclareMathOperator{\PP}{\mathbf{P}}
\DeclareMathOperator{\MM}{\mathbf{M}}
\DeclareMathOperator{\VV}{\mathbf{V}}
\DeclareMathOperator{\TT}{\mathbf{T}}
\DeclareMathOperator{\LL}{\mathcal{L}}
\DeclareMathOperator{\EE}{\mathbf{E}}
\DeclareMathOperator{\NN}{\mathbf{N}}
\DeclareMathOperator{\DQ}{\mathcal{Q}}
\DeclareMathOperator{\IA}{\mathfrak{a}}
\DeclareMathOperator{\IB}{\mathfrak{b}}
\DeclareMathOperator{\IC}{\mathfrak{c}}
\DeclareMathOperator{\IP}{\mathfrak{p}}
\DeclareMathOperator{\IQ}{\mathfrak{q}}
\DeclareMathOperator{\IM}{\mathfrak{m}}
\DeclareMathOperator{\IN}{\mathfrak{n}}
\DeclareMathOperator{\IK}{\mathfrak{k}}
\DeclareMathOperator{\ord}{\text{ord}}
\DeclareMathOperator{\Ker}{\textsf{Ker}}
\DeclareMathOperator{\Coker}{\textsf{Coker}}
\DeclareMathOperator{\emphcoker}{\emph{coker}}
\DeclareMathOperator{\pp}{\partial}
\DeclareMathOperator{\tr}{\text{tr}}

\DeclareMathOperator{\supp}{\text{supp}}

\DeclareMathOperator{\codim}{\text{codim}}

\DeclareMathOperator{\minkdim}{\dim_{\mathbf{M}}}
\DeclareMathOperator{\hausdim}{\dim_{\mathbf{H}}}
\DeclareMathOperator{\lowminkdim}{\underline{\dim}_{\mathbf{M}}}
\DeclareMathOperator{\upminkdim}{\overline{\dim}_{\mathbf{M}}}
\DeclareMathOperator{\lhdim}{\underline{\dim}_{\mathbf{M}}}
\DeclareMathOperator{\lmbdim}{\underline{\dim}_{\mathbf{MB}}}
\DeclareMathOperator{\packdim}{\text{dim}_{\mathbf{P}}}
\DeclareMathOperator{\fordim}{\dim_{\mathbf{F}}}

\DeclareMathOperator*{\argmax}{arg\,max}
\DeclareMathOperator*{\argmin}{arg\,min}

\DeclareMathOperator{\ssm}{\smallsetminus}

\title{The Representation Theory of Lie Algebras}
\author{Jacob Denson}

\begin{document}

\pagenumbering{gobble}

\maketitle

\tableofcontents

\pagenumbering{arabic}

\chapter{Basic Definitions}

If $K$ is a field, then a Lie algebra over $K$ is a $K$ vector-space $\mathfrak{g}$ equipped with an alternating, bilinear form $[\cdot, \cdot]$, known as the {\bf Lie bracket}, satisfying the Jacobi identity
%
\[ [X,[Y,Z]] + [Y,[Z,X]] + [Z,[X,Y]] = 0 \]
%
for any $X,Y,Z \in \mathfrak{g}$. Note that the bracket operation need not be associative. That is, it is entirely possible for $[X,[Y,Z]] \neq [[X,Y],Z]$.

If $K$ is a complete field with respect to some absolute value, and an analytic group $G$ over $K$, the set $\mathfrak{g}$ of tangent vectors at the origin in $G$ has a natural Lie algebra structure. We normally denote this natural structure by the `frakturized' name of $G$. In geometry, and some parts of number theory, this is where the majority of the examples of Lie algebras occur.

\begin{example}
    The Lie algebra $\mathfrak{gl}_n(K)$ is the set of matrices in $M_n(K)$, under the Lie bracket $[X,Y] = XY - YX$. The Jacobi identity follows by a thorough calculation, and shows that the commutator operation forms a Lie bracket on any algebra over $K$. It corresponds to the Lie group $GL_n(K)$.
\end{example}

\begin{example}
    If $\mathfrak{g}$ is any vector space, then the trivial bracket $[X,Y] = 0$ gives $\mathfrak{g}$ a Lie algebra structure. $\mathfrak{g}$ is known as a commutative Lie algebra, since we then have $[X,Y] = [Y,X]$. This implies the property of being a Lie algebra is trivial, and we should not expect to learn much about the vector space structure of a Lie algebra from the properties of a Lie algebra.
\end{example}

\begin{example}
    On any field $K$, we may consider the differentiation of polynomials $f \in K[X]$ by the differential operator $\partial_X: f(X) \mapsto f'(X)$, and this extends to the formation of the ring of differential operators
    %
    \[ \sum_{k = 0}^N g_k(X) \partial_X^k \]
    %
    with coefficients $g_k \in K[X]$, where $\partial_X^k = \partial_X \circ \partial_X \circ \dots \partial_X$ is differentiation $k$ times. This operator is a ring under composition, because we have the identity
    %
    \[ \partial_X(Xf) = X \partial_X(f) + f \partial_X(X) = X \partial_X(f) + f \]
    %
    so that $\partial_X X = X \partial_X + 1$, and we may rearrange the differential operators to always occur to the rightmost side of any monomial in $\partial_X$ and $X$. This is the only relation between the differential operators, because if
    %
    \[ \sum_{k = 0}^N g_k(X) \partial_X^k = 0 \]
    %
    Then successively applying the operator to the monomials $X^m$ gives
    %
    \[ g_0(X) = g_1(X) = \dots = g_N(X) = 0 \]
    %
    Hence we have an isomorphism between the ring of operators with polynomial coeffients and $\mathbf{C}\langle X, \partial_X \rangle / (\partial_X X - X \partial_X - 1)$, known as the 1st Weyl algebra. In general, the ring of differential operators in $n$ variables is called the $n$'th Weyl algebra. It is obtained from $\mathbf{C}\langle X_1, \dots, X_n, \partial_{X_1}, \dots, \partial_{X_n} \rangle$ modulo the relations $\partial_{X_i} X_j - X_i \partial_{X_j} = \delta_{ij}$. The commutator on these rings gives a Lie algebra structure.
\end{example}

\begin{example}
    Given an algebra $A$, a derivation on $A$ is a map $d: A \to A$ satisfying $d(xy) = xd(y) + d(x)y$. Given two derivations $d$ and $d'$, $d \circ d'$ may not be a derivation, but the commutator $[d_1, d_2] = d_1 \circ d_2 - d_2 \circ d_1$ is always a derivation, because
    %
    \begin{align*}
        (d_1 \circ d_2 - d_2 \circ d_1)(fg) &= d_1(f d_2(g) + d_2(f) g) - d_2(d_1(f) g + f d_1(g))\\
        &= d_1(f) d_2(g) + f (d_1 \circ d_2)(g) + d_2(f) d_1(g) + (d_1 \circ d_2)(f) g\\
        &\ - [(d_2 \circ d_1)(f) g + d_1(f) d_2(g) + d_2(f) d_1(g) + f (d_2 \circ d_1)(g)]\\
        &= f(d_1 \circ d_2 - d_2 \circ d_1)(g) - (d_1 \circ d_2 - d_2 \circ d_1)(f) g
    \end{align*}
    %
    Thus the set of derivations on $A$ forms a Lie algebra.
\end{example}

\begin{example}
    The Lie algebra $\mathfrak{sl}_n(K)$ consists of the elements $X$ of $\mathfrak{gl}_n(K)$ such that $\text{tr}(X) = 0$. This is a Lie subalgebra of $\mathfrak{gl}_n(K)$, because of the formula
    %
    \[ \text{tr}(XY) = \text{tr}(YX) \]
    %
    because
    %
    \[ \text{tr}(XY) = \sum_{i = 1}^n (XY)_{ii} = \sum_{i = 1}^n \sum_{j = 1}^n X_{ij} Y_{ji} = \sum_{i = 1}^n \sum_{j = 1}^n X_{ji} Y_{ij} = \text{tr}(YX) \]
    %
    and so
    %
    \[ \text{tr}(XY - YX) = \text{tr}(XY) - \text{tr}(YX) = 0 \]
    %
    Thus we have the decomposition $\mathfrak{gl}_n(K) = \mathfrak{sl}_n(K) \oplus K^n$. The Lie group corresponding to $\mathfrak{sl}_n(K)$ is $SL_n(K)$.
\end{example}

\begin{example}
    The Special Orthogonal Lie Algebra $\mathfrak{so}_n(K)$ is the subalgebra of $\mathfrak{sl}_n(K)$ consisting of skew symmetric matrices, $X$ such that $X^t = -X$, because if $X,Y \in \mathfrak{so}_n(K)$, then
    %
    \[ (XY - YX)^t = Y^tX^t - X^tY^t = YX - XY = -(XY - YX) \]
    %
    so the Lie bracket is well defined. The Lie algebra corresponds to the Lie group $SO_n(K)$, and in fact also the Lie group $O_n(K)$.
\end{example}

\begin{example}
    If $n$ is even, $n = 2m$, define the matrix $J \in M_n(K)$ by
    %
    \[ J = \begin{pmatrix} 0 & I_m \\ -I_m & 0 \end{pmatrix} \]
    %
    And then define $\mathfrak{sp}_n(K)$ to be the matrices $X \in \mathfrak{sl}_n$ satisfying $X^tJ + JX = 0$. If we write
    %
    \[ X = \begin{pmatrix} A & B \\ C & D \end{pmatrix} \]
    %
    Then
    %
    \[ X^tJ = \begin{pmatrix} -B^t & A^t \\ -D^t & C^t \end{pmatrix}\ \ \ \ \ JX = \begin{pmatrix} C & D \\ -A & -B \end{pmatrix} \]
    %
    giving us a series of equations on the blocks of $X$ which define $\mathfrak{sp}_n(K)$. This Lie algebra corresponds to the Lie group $SP_n(K)$.
\end{example}

\begin{example}
    The Heisenberg group $H_n(K)$ is the Lie group of matrices of the form
    %
    \[ \begin{pmatrix} 1 & a & c \\ 0 & I_n & b \\ 0 & 0 & 1 \end{pmatrix} \]
    %
    where $a \in K^n$ is a row vector, $c \in K$, and $b \in K^n$ is a column vector. It's corresponding Lie algebra $\mathfrak{h}_n(K)$ consists of vectors of the form
    %
    \[ \begin{pmatrix} 0 & a & c \\ 0 & 0 & b \\ 0 & 0 & 0 \end{pmatrix} \]
    %
    which is flat, since the Heisenberg group is essentially flat.
\end{example}

A homomorphism of Lie algebras is, of course, one whish preserves the vector space structure and the Lie bracket operation. It is difficult to classify the Lie algebras, but we can classify the simple Lie algebras, which will occupy us through the course. It turns out that the only simple Lie algebras are
%
\[ \mathfrak{sl}_n, \mathfrak{so}_n, \mathfrak{sp}_n \]
%
and some `eccentric' algebras $E_6, E_7, E_8, F_4$, and $G_2$.

A good basis for $\mathfrak{gl}_n$ are the matrices $E_{ij}$, which are only non-zero on row $i$ and column $j$, where the matrix coefficient has value 1. Then
%
\[ [X, E_{ij}] = XE_{ij} - E_{ij}X = \sum_k X_{ki} E_{kj} - X_{jk} E_{ik} \]
%
If we define $H_k = E_{kk} - E_{k+1\ k+1}$, then the $H_k$ and $E_{ij}$ for $i \neq j$ span $\mathfrak{sl}_n$, and
%
\[ [H_k, E_{ij}] = (\delta_k^i - \delta_{k+1}^i - \delta_k^j E_{ij} + \delta_{k+1}^j) E_{ij} \]
%
So $E_{ij}$ is an eigenvector for the {\bf adjoint} map $\text{adj}(X)$, defined by $\text{adj}(X)(Y) = [X,Y]$, where $X = H_k$.

\end{document}