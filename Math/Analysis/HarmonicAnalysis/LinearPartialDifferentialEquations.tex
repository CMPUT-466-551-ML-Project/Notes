%% The following is a directive for TeXShop to indicate the main file
%%!TEX root = HarmonicAnalysis.tex
\part{Linear Partial Differential Equations}

Here we study the techniques of the \emph{constant coefficient} linear partial differential equations
%
\[ Lf(x) = \sum a_\alpha D^\alpha f(x) \]
%
and the \emph{non constant coefficient} partial differential equations
%
\[ Lf(x) = \sum a_\alpha(x) D^\alpha f(x) \]
%
The central problems of the theory are \emph{existence} and \emph{regularity}: Given $f$, does a solution $u$ to the equation $Lu = f$ exist, and if so, how does the regularity of $f$ relate to the regularity of $u$. We often phrase the regularity problems in terms of function spaces. Let $\{ X^s \}$ be a family of function spaces, indexed by a parameter $s \in \RR$, measuring the \emph{regularity} of functions in these spaces. Given a linear partial differential operator $L$, a regularity result might say that for some $s$ and $t$, if $Lu = f$, where $f \in X^s$, then $u$ in $X^t$. Such results often require one to use the special properties of the operator $L$; general theories are rarely available, except for certain very special families of partial differential equations, especially the \emph{elliptic partial differential equations}.

\chapter{Green's Functions}

Let $\Omega$ be a bounded, open set on $\RR^n$. Recall Green's second identity,
%
\[ \int_\Omega (u \Delta v - v \Delta u) = \int_{\partial \Omega} u \frac{\partial v}{\partial \eta} - v \frac{\partial u}{\partial \eta}. \]
%
We introduce the quantity
%
\[ u_x(y) = \frac{1}{|x - y|^{n-2}}, \]
%
which is Newton's gravitational potential, or Coulomb's electrostatic potential given a point mass at the point $x$. A simple computation shows that $\Delta u_x(y) = 0$ for $y \neq x$, perhaps using the fact that if $u(x)$ is a radial function, with $u(x) = f(r)$ for $r = |x|$, then
%
\[ \Delta u(x) = f''(r) + \frac{n - 1}{r} f'(r). \]
%
Plugging this function into Green's identity, if $v \in C^2(\Omega)$ is a Harmonic function on a domain $\Omega$, continuous on the closure of $\Omega$, then for any $x$, applying Green's identity on the domain $\Omega_{x,\varepsilon}$, obtained from $\Omega$ by removing the ball $B$ of radius $\varepsilon$ centered at $x$, where $\varepsilon$ is small enough so that the entire ball is contained in $\Omega$, we find that
%
\begin{align*}
	0 &= \int_{\partial \Omega_{x,\varepsilon}} \left( u_x \frac{\partial v}{\partial \eta} - v \frac{\partial u_x}{\partial \eta} \right)\\
	&= \int_{\partial \Omega} \left( u_x \frac{\partial v}{\partial \eta} - v \frac{\partial u_x}{\partial \eta} \right) - \int_{\partial B} u_x \frac{\partial v}{\partial \eta} - v \frac{\partial u_x}{\partial \eta}.
\end{align*}
%
Now for $y \in \partial B$, with $y = x + z$, $\eta(y) = \varepsilon^{-1} z$. And
%
\[ \nabla u_x(y) = -(n-2) \frac{z}{|z|^n}, \]
%
so
%
\[ \int_{\partial B} u_x \frac{\partial v}{\partial \eta} - v \frac{\partial u_x}{\partial \eta} = \int_{|z| = 1} \varepsilon \cdot \nabla v(x + \varepsilon z) \cdot z + (n-2) v(x + \varepsilon z). \]
%
Applying a Taylor series shows that as $\varepsilon \to 0$, this quantity converges to $(n-2) C_d v(x)$, where $C_d$ is the surface area of $S^{d-1}$. Thus we conclude that
%
\[ v(x) = \frac{-1}{C_d (n-2)} \int_{\partial \Omega} \left( u_x \frac{\partial v}{\partial \eta} - v \frac{\partial u_x}{\partial \eta} \right) \]
%
Thus $v$ is determined by it's boundary values, and boundary derivatives in the direction normal to $\Omega$. To get rid of these normal boundary derivatives, we suppose we can compute a function $h_x$, which is harmonic in $\Omega$, and equal to $u_x$ on $\partial \Omega$. Then we can apply Green's second identity again, which yields
%
\begin{align*}
	0 &= \int_{\partial \Omega} v \frac{\partial h_x}{\partial \eta} - \frac{\partial v}{\partial \eta} G\\
	&= \int_{\partial \Omega} v \frac{\partial h_x}{\partial \eta} - \frac{\partial v}{\partial \eta} u_x.
\end{align*}
%
Thus we conclude that
%
\[ v(x) = \frac{-1}{C_d (n-2)} \int_{\partial \Omega} v \left( \frac{\partial h_x}{\partial \eta} - \frac{\partial u_x}{\partial \eta} \right). \]
%
The function
%
\[ g_x(y) = \frac{-1}{C_d (n-2)} (h_x(y) - u_x(y)) \]
%
is called the \emph{Green's function} on the domain $\Omega$. It solves the equation $\Delta g_x = \delta_x$ on $\Omega$, while vanishing on the boundary. We have calculated this by virtue of the fact that for any harmonic function $v$ on $\Omega$,
%
\[ v(x) = \int_{\partial \Omega} \frac{\partial g_x}{\partial \eta} v\; dy. \]
%
Thus $v$ is uniquely determined by it's boundary values. What's more, we now have a natural operator which, given some sufficiently boundary conditions on $\partial \Omega$, allows us to find a unique solution to Laplace's equation on $\Omega$.




\chapter{Elliptic Equations}

Consider a partial differential operator $L f(x) = \sum a_\alpha(x) D^\alpha f(x)$, where $a_\alpha \in C^\infty(\RR^n)$. Associated with this operator is it's symbol $a(x,\xi) = \sum a_\alpha(x) \xi^\alpha$, and if $L$ has order $m$, it's principle symbol is $a_m(x,\xi) = \sum_{|\alpha| = m} a_\alpha(x) \xi^\alpha$. We say $L$ is \emph{elliptic} near $x_0$ if $a_m(x_0,\xi) = 0$ only when $\xi = 0$. This is equivalent to assuming that there exists $M > 0$, and two constants $c_1$ and $c_2$ and a small neighborhood $U$ of $x_0$ such that $c_1 |\xi|^m \leq |a(x,\xi)| \leq c_2 |\xi|^m$ for $|\xi| \geq M$.

Elliptic operators have a near complete theory, because they are \emph{invertible} modulo smoothing operators. More precisely




% To look at later: 2.2.13













\chapter{Dispersive Equations}

A partial differential equation is \emph{dispersive} if wave-like solutions to the partial differential equation with differing frequencies travel in different directions and speeds, so that linear combinations of such solutions tend to `spread out in space' over time. This phenomenon presents itself in various \emph{dispersive estimates} for such partial differential equations.

The simplest example where dispersive phenomena emerges is in the analysis of a constant-ceofficient hyperbolic first order system of the form
%
\[ \partial_t u = Lu \]
%
where $u: V \times \RR \to \RR$ for some finite dimensional Hilbert space $V$, and where $Lu = \sum c_\alpha D_x^\alpha u$ for some $c_\alpha \in \text{End}(V)$. In order to assume the differential equation is hyperbolic, and thus well-posed, we assume that $L$ is formally skew-adjont, i.e. that for all compactly supported, smooth functions $u: V \to \RR$ and $v: V \to \RR$,
%
\[ \langle Lu, v \rangle = - \langle u, - Lv \rangle. \]
%
If we write $L = 2 \pi i h(D_x)$ for some polynomial $h: \RR^d \to \text{End}(V)$, then this is equivalent to the coefficients of $h$ being self-adjoint. The polynomial $h$ is called the \emph{dispersion relation} of the equation.

Given such a PDE, we have \emph{planar wave solutions} of the form
%
\[ u(x,t) = e^{2 \pi i (\xi \cdot x + \omega t)} \]
% x = - (omega / \xi) t
precisely when $\omega = h(\xi)$. Thus the dispersion relation relates the \emph{wave number} $\xi$ of a planar wave with the \emph{angular frequency} $\omega$ of the planar waves involved. In particular, looking at the phase $\xi \cdot x + \omega t$ and trying to solve in the $x$-variable, we see the wave will look like it is travelling through space at a \emph{phase velocity}
%
\[ v_p(\xi) = - \frac{\xi}{|\xi|} \frac{h(\xi)}{|\xi|} \]
%
In particular, the phase speed is $|h(\xi)| / |\xi|$.

More generally, we can consider superposition solutions using the Fourier inversion formula, i.e. writing
%
\[ u(x,t) = \int \widehat{u_0}(\xi) e^{2 \pi i [\xi \cdot x + h(\xi) t]}\; d\xi \]
%
Taking $u_0(\xi) = e^{2 \pi i \xi \cdot x}$ gives the planar wave solutions above. A \emph{wave packet} solution is obtained by taking $u_0$ to be supported on a small region of phase-space, i.e. localized to a particular position in space, and a particular frequency. One example localized near $(x_0,\xi_0)$ would be of the form
%
\[ u_0(x) = e^{2 \pi i \xi_0 \cdot x} \phi \left( \frac{x - x_0}{\delta} \right), \]
%
where $\phi$ is a Schwartz function whose Fourier transform $\psi$ is smooth and compactly supported. Fourier inversion tells us that
%
\[ u(x,t) = \int \psi(\xi) e^{2 \pi i [ (\xi_0 + \xi) \cdot x + h(\xi_0 + \xi) t - \xi \cdot x_0 ]}\; d\xi. \]
%
The principle of nonstationary phase tells us that $u$ has spatial concentration near $x = x_0 - \nabla h(\xi_0) t + O(t)$. Thus the phase envelope of $u$ can be seen to be travelling starting at $x_0$, and travelling at a \emph{group velocity} $- \nabla h(\xi_0)$, though the slight differences in phase velocities for individual planar waves which give $u$ as a superposition can lead to the phase envelope spreading apart in time.

\begin{example}
	For $\omega_0 \in \RR$, we can consider the \emph{phase rotation equation} $\partial_t u = 2 \pi i \omega_0$, which can be explicitly solved by writing
	%
	\[ u(x,t) = e^{2 \pi i \omega_0 t} u_0(x). \]
	%
	Thus solutions have the envelope defined by $u_0$, and oscillate in time at the angular frequency $\omega_0$. In particular, we have planar-wave solutions of the form
	%
	\[ e^{2 \pi i (\xi \cdot x + \omega_0 t)}. \]
	%
	The phase velocity here is $- \omega_0 / \xi$, i.e. planar waves with a high spatial frequency will appear to travel much more slowly in time. The group velocity, however, is $0$, so a superposition of waves with similar spatial frequency will remain in the same position spatially.
\end{example}

\begin{example}
	Consider the \emph{transport equation}
	%
	\[ \partial_t u = - v_0 \cdot \nabla_x u \]
	%
	for some $v_0 \in \RR^d$. The solution to this equation is given by
	%
	\[ u(x,t) = u_0(x - v_0 t). \]
	%
	The planar-wave solutions are of the form
	%
	\[ e^{2 \pi i (\xi \cdot x - (\xi \cdot v_0) t))}. \]
	%
	The dispersion relation is $h(\xi) = - v_0 \cdot \xi$. The phase velocity is $v_p(\xi) = \xi (v_0 \cdot \xi) / |\xi|^2$; if $\xi$ is aligned in the direction $v_0$, then the phase velocity will be $v_0$, and if $\xi$ is orthogonal to $v_0$, the phase velocity will be zero, since the solution will be stationary in time. The group velocity is $v_0$, i.e. a wave packet will always roughly move in the direction $v_0$.
\end{example}

\begin{example}
	In the \emph{free Schr\"{o}dinger equation} $i \partial_t u = - (\hbar / 8 \pi^2 m) \Delta u$, the dispersion relation is $h(\xi) = - (\hbar / 2 m) |\xi|^2$, and the phase velocity is $v_p(\xi) = (\hbar / 2 m) \xi$. The group velocity here is $v_g(\xi) = (\hbar / m) \xi$. Thus we see that the phase of a wave packet travels at half it's group velocity. More interesting is the group velocity equation, which in physics is used as a derivation of \emph{De Broglie's law} $mv_g = \hbar \xi$, i.e. that the frequency of a wave packet is tied to it's velocity in time.
\end{example}

\begin{example}
	The \emph{one-dimensional Airy equation} is $\partial_t u = (2 \pi)^{-3} \partial_x^3 u$. It has dispersion relation $h(\xi) = |\xi|^3$. The phase velocity is $v_p(\xi) = \xi |\xi|$, and the group velocity is $v_g(\xi) = 3 \xi |\xi| = 3 v_p(\xi)$.
\end{example}

For higher order hyperbolic equations $P(\partial_t, D_x)$, say of order $m$, the dispersion relation $h$ is multi-valued, equal to the values $h(\xi)$ such that $P(i h(\xi),\xi) = 0$.

\begin{example}
	Consider the \emph{wave equation} $\Box u = 0$, where $\Box$ is the d'Alembertian operator $(2 \pi)^{-2} \Delta - \partial_t^2$. The resulting polynomial is $P(\tau,\xi) = -(\xi^2 + \tau^2)$. The dispersion relation here is thus given by the equation $(\xi^2 + (i h(\xi))^2) = 0$, i.e. we have $h(\xi) = \pm \xi$. The phase velocities are $v_p(\xi) = \pm \xi / |\xi|$, and the group velocities are also $v_g(\xi) = \pm \xi / |\xi|$.
\end{example}

\begin{example}
	The \emph{Klein-Gordan equation} $\Box u = u$ describes the movement of electrons where relativity is relevant. It is given by the polynomial $P(\tau,\xi) = - (|\xi|^2 + \tau^2 + 1)$, and has dispersion relation $h(\xi) = \pm \sqrt{1 + |\xi|^2}$. For large $\xi$, we thus see waves here behave like the wave equation, though for small $\xi$, waves behave like the transport equation.
\end{example}

Our goal is to study \emph{dispersive equations}, i.e. equations where wave packet solutions tend to spread apart over time. Since the velocity of a wave packet travelling at frequency $\xi_0$ is $\nabla h(\xi_0)$, wave packets near $\xi_0$ will travel at different frequencies if the \emph{Hessian} $H(\xi_0)$ of $h$ is invertible. If this is true, the equation is called \emph{fully dispersive}, like the Schr\"{o}dinger equation and Airy equation. In dimension one, for the wave equation, the Hessian is zero, so in this case the wave equation is very non-dispersive. But in $d$ dimensions, we see the Hessian has rank $d-1$, so we at least have `some dispersion', but not in the radial direction.

% f'(x) is continuous away from 0
% f'(0-) != f'(0+)
% Then f'(0) does not exist
% f(epsilon) = [f'(0+) + o(1)][epsilon + o(1)]







\section{Strichartz Estimates}

Our goal is to obtain \emph{space-time estimates} of the form $\| u \|_{L^q_t L^r_x}$ on solutions $u$ to a dispersive partial differential equation. Let's start with a dispersive equation of the form $\partial_t u = Lu$. let's look at pointwise estimates. Let $T_t = e^{2 \pi i t L}$.
%
\[ T_t f(x) = \int e^{2 \pi i [\xi \cdot x + h(\xi) t]} \widehat{f}(\xi)\; d\xi. \]
%
This is an operator with Fourier multiplier $m_t(\xi) = e^{2 \pi i h(\xi) t}$. To understand these operators, we apply Littlewood-Paley theory, writing
%
\[ T_t = T_{t,0} + T_{t,1} + \dots \]
%
where $T_{t,l}$ is a Fourier multiplier with symbol $m_{t,l}(\xi) = \psi_l(\xi) e^{2 \pi i h(\xi) t}$. Applying stationary phase, if $K_{t,l}$ is the inverse Fourier transform of $m_{t,l}$, then we have TODO

As $t \to \infty$, we can see Dispersion via the norm of the operators $T_t$.

\begin{theorem}
	Suppose that $T$ has a dispersion relation of the form $\| T_t \|_{L^q}$
\end{theorem}
\begin{proof}
	Let $T$ map functions on $\RR^d$ to functions on $\RR^d \times \RR$, given by
	%
	\[ Tf(x,t) = e^{2 \pi i t L} f(x) = T_t f(x). \]
	%
	The formal adjoint is $T^*$ mapping functions on $\RR^d \times \RR$ to functions on $\RR^d$ by
	%
	\[ T^*F(x) = \int e^{-2 \pi i t L} F(x,t)\; dt = \int T_{-t} F(x,t)\; dt. \]
	%
	Thus
	%
	\[ TT^*F(x,t) = \int e^{2 \pi i (t - s) L} F(x,s)\; ds = \int T_{t-s} F_s(x)\; ds. \]
	%
	We have dispersive estimate
	%
	\[ \| e^{2 \pi i (t - s)} f \|_{L^r_x} \lesssim |t|^{- (d/2 - d/r)} \| f \|_{L^{r^*}_x} \]
	%
	for $1/r + 1/r^* = 1$. Thus
	%
	\begin{align*}
		\| TT^* F \|_{L^q_t L^r_x} &\leq \left\| \int \left\| T_{t-s} F_s \right\|_{L^r_x}\; ds \right\|_{L^q_t}\\
		&\lesssim \left\| \int \frac{\| F_s \|_{L^{r^*}_x}}{|t - s|^{d/2 - d/r}} \; ds \right\|_{L^q_t}.
	\end{align*}
	%
	Now applying Hardy-Littlewood-Sobolev gives the required bound, i.e. that
	%
	\[ \| TT^* F \|_{L^q_t L^r_x} \lesssim \left\| \int \frac{\| F_s \|_{L^{r^*}_x}}{|t - s|^{d/2 - d/r}} \; ds \right\|_{L^q_t} \lesssim \| F_s \|_{L^q_s L^r_x}. \]
	%
	The $TT^*$ argument gives that
	%
	\[ \| Tf \|_{L^q_t L^r_x} \lesssim \| f \|_{L^2_x}. \qedhere \]
\end{proof}







