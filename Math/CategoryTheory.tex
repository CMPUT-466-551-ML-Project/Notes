\documentclass[12pt, dvipsnames]{report}

\usepackage{amsmath}
\usepackage{algorithm}
%\usepackage{algorithmic}
\usepackage[noend]{algpseudocode}

\usepackage{amsmath}
\usepackage{amssymb}
\usepackage{amsthm}
\usepackage{amsopn}

\usepackage{kpfonts}

\usepackage{graphicx}

% Probably don't need this on notes anymore
%\usepackage{kbordermatrix}

% Standard tool for drawing diagrams.
\usepackage{tikz}
\usepackage{tkz-berge}
\usepackage{tikz-cd}
\usepackage{tkz-graph}

\usepackage{comment}

%
\usepackage{multicol}

%
\usepackage{framed}

%
\usepackage{mathtools}

%
\usepackage{float}

%
\usepackage{subfig}

%
\usepackage{wrapfig}

%
\let\savewideparen\wideparen
\let\wideparen\relax
\usepackage{mathabx}
\let\wideparen\savewideparen

% Used for generating `enlightening quotes'
\usepackage{epigraph}

% Forget what this is used for :P
\usepackage[utf8]{inputenc}

% Used for generating quotes.
\usepackage{csquotes}

% Allows what to generate links inside
% generated pdf files
\usepackage{hyperref}

% Allows one to customize theorem
% environments in mathematical proofs.
\usepackage{thmtools}

% Gives access to a proof
\usepackage{lplfitch}

% I forget what this is for.
\usepackage{accents}

% A package for drawing simple trees,
% as a substitute for unnesacary TIKZ code
\usepackage{qtree}

% Enables sequent calculus proofs
\usepackage{ebproof}

% For braket notation
\usepackage{braket}

% To change line spacing when using mathematical notations which require some height!
\usepackage{setspace}

%\usepackage[dvipsnames]{xcolor}

\usepackage{float}

% For block commenting
\usepackage{comment}




\setlength\epigraphwidth{8cm}

\usetikzlibrary{arrows, petri, topaths, decorations.markings}

% So you can do calculations in coordinate specifications
\usetikzlibrary{calc}
\usetikzlibrary{angles}

\theoremstyle{plain}
\newtheorem{theorem}{Theorem}[chapter]
\newtheorem{axiom}{Axiom}
\newtheorem{lemma}[theorem]{Lemma}
\newtheorem{corollary}[theorem]{Corollary}
\newtheorem{prop}[theorem]{Proposition}
\newtheorem{exercise}{Exercise}[chapter]
\newtheorem{fact}{Fact}[chapter]

\newtheorem*{example}{Example}
\newtheorem*{proof*}{Proof}

\theoremstyle{remark}
\newtheorem*{exposition}{Exposition}
\newtheorem*{remark}{Remark}
\newtheorem*{remarks}{Remarks}

\theoremstyle{definition}
\newtheorem*{defi}{Definition}

\usepackage{hyperref}
\hypersetup{
    colorlinks = true,
    linkcolor = black,
}

\usepackage{textgreek}

\makeatletter
\renewcommand*\env@matrix[1][*\c@MaxMatrixCols c]{%
  \hskip -\arraycolsep
  \let\@ifnextchar\new@ifnextchar
  \array{#1}}
\makeatother

\renewcommand*\contentsname{\hfill Table Of Contents \hfill}

\newcommand{\optionalsection}[1]{\section[* #1]{(Important) #1}}
\newcommand{\deriv}[3]{\left. \frac{\partial #1}{\partial #2} \right|_{#3}} % partial derivative involving numerator and denominator.
\newcommand{\lcm}{\operatorname{lcm}}
\newcommand{\im}{\operatorname{im}}
\newcommand{\bint}{\mathbf{Z}}
\newcommand{\gen}[1]{\langle #1 \rangle}

\newcommand{\End}{\operatorname{End}}
\newcommand{\Mor}{\operatorname{Mor}}
\newcommand{\Id}{\operatorname{id}}
\newcommand{\visspace}{\text{\textvisiblespace}}
\newcommand{\Gal}{\text{Gal}}

\newcommand{\xor}{\oplus}
\newcommand{\ft}{\wedge}
\newcommand{\ift}{\vee}

\newcommand{\prob}{\mathbf{P}}
\newcommand{\expect}{\mathbf{E}}
\DeclareMathOperator{\Var}{\mathbf{V}}
\newcommand{\Ber}{\text{Ber}}
\newcommand{\Bin}{\text{Bin}}

%\newcommand{\widecheck}[1]{{#1}^{\ft}}

\DeclareMathOperator{\diam}{\text{diam}}

\DeclareMathOperator{\QQ}{\mathbf{Q}}
\DeclareMathOperator{\ZZ}{\mathbf{Z}}
\DeclareMathOperator{\RR}{\mathbf{R}}
\DeclareMathOperator{\HH}{\mathbf{H}}
\DeclareMathOperator{\CC}{\mathbf{C}}
\DeclareMathOperator{\AB}{\mathbf{A}}
\DeclareMathOperator{\PP}{\mathbf{P}}
\DeclareMathOperator{\MM}{\mathbf{M}}
\DeclareMathOperator{\VV}{\mathbf{V}}
\DeclareMathOperator{\TT}{\mathbf{T}}
\DeclareMathOperator{\LL}{\mathcal{L}}
\DeclareMathOperator{\EE}{\mathbf{E}}
\DeclareMathOperator{\NN}{\mathbf{N}}
\DeclareMathOperator{\DQ}{\mathcal{Q}}
\DeclareMathOperator{\IA}{\mathfrak{a}}
\DeclareMathOperator{\IB}{\mathfrak{b}}
\DeclareMathOperator{\IC}{\mathfrak{c}}
\DeclareMathOperator{\IP}{\mathfrak{p}}
\DeclareMathOperator{\IQ}{\mathfrak{q}}
\DeclareMathOperator{\IM}{\mathfrak{m}}
\DeclareMathOperator{\IN}{\mathfrak{n}}
\DeclareMathOperator{\IK}{\mathfrak{k}}
\DeclareMathOperator{\ord}{\text{ord}}
\DeclareMathOperator{\Ker}{\textsf{Ker}}
\DeclareMathOperator{\Coker}{\textsf{Coker}}
\DeclareMathOperator{\emphcoker}{\emph{coker}}
\DeclareMathOperator{\pp}{\partial}
\DeclareMathOperator{\tr}{\text{tr}}

\DeclareMathOperator{\supp}{\text{supp}}

\DeclareMathOperator{\codim}{\text{codim}}

\DeclareMathOperator{\minkdim}{\dim_{\mathbf{M}}}
\DeclareMathOperator{\hausdim}{\dim_{\mathbf{H}}}
\DeclareMathOperator{\lowminkdim}{\underline{\dim}_{\mathbf{M}}}
\DeclareMathOperator{\upminkdim}{\overline{\dim}_{\mathbf{M}}}
\DeclareMathOperator{\lhdim}{\underline{\dim}_{\mathbf{M}}}
\DeclareMathOperator{\lmbdim}{\underline{\dim}_{\mathbf{MB}}}
\DeclareMathOperator{\packdim}{\text{dim}_{\mathbf{P}}}
\DeclareMathOperator{\fordim}{\dim_{\mathbf{F}}}

\DeclareMathOperator*{\argmax}{arg\,max}
\DeclareMathOperator*{\argmin}{arg\,min}

\DeclareMathOperator{\ssm}{\smallsetminus}

\title{Category Theory}
\author{Jacob Denson}

\begin{document}

\pagenumbering{gobble}
\maketitle
\tableofcontents

\chapter{Basic Definitions}

\pagenumbering{arabic}

Category Theory is the language of transformations. A great many objects share some common formal behaviour with respect to the functions defined on them. Rather than looking at particular elements of particular groups, rings, and sets, we just look at the objects themselves, and describe the functions connecting them. The structure of certain constructions then become generalized to an incredibly varied set of mathematical situations. Thus when studying a new class of mathematical objects, we already have a set of intuitions as to how certain constructions `act'.

A {\bf category} $\mathcal{C}$ consists of objects $\text{Obj}(\mathcal{C})$, such that for each pair of objects $(A,B) \in \text{Obj}(\mathcal{C})$ we have a collection of morphisms $\text{Mor}(A,B)$, which are pairwise disjoint, together with a composition operation. We write $f \in \text{Mor}(A,B)$ as $f:A \to B$. For each triplet $A,B,C$ of objects, we have an associative composition map
%
\[ \circ: \text{Mor}(B,C) \times \text{Mor}(A,B) \to \text{Mor}(A,C) \]
%
For any object $A$, we have a morphism $\text{id}_A \in \text{Mor}(A,A)$, such that $\text{id} \circ f = f$ for any $f: B \to A$, and $g \circ \text{id}_A = g$ for any $g: A \to B$. Morphisms in $\text{Mor}(A,A)$ will be known as {\bf endomorphisms}, and the set of such objects will be abbreviated to $\text{End}(A)$. With the operation of composition, $\text{End}(A)$ becomes a monoid, and all monoids can be realized as endomorphisms over some object in a particular category. An {\bf operation} of a monoid $M$ on an object $A$ is a monoid homomorphism $f: M \to \text{End}(A)$.

\begin{example}
    Perhaps the most basic category is the category $\mathbf{Set}$ of sets, whose objects are sets, and whose morphisms are set-theoretic functions. Category theory can be seen as a generalization of this category, and most often categories will be seen as a subcategory of this category.
\end{example}

\begin{example}
    Algebra makes extensive use of category theory. The category $\mathbf{Grp}$ of groups has groups as objects, and whose morphisms are group homomorphisms. One similarily defines the categories $\mathbf{Rng}$ and $\mathbf{Vect}_K$ of rings and vector spaces over a fixed field, with ring homomorphisms and linear transformations as morphisms.
\end{example}

\begin{example}
    Often useful for constructing illuminating examples, every partially ordered set $X$ can be given the structure of a category such that, if $x \leq y$ in $X$, there is precisely one morphism from $x$ to $y$.
\end{example}

\begin{example}
    Category theory is also useful in analysis. Let $\mathbf{Top}$ be the category of topological spaces, whose morphisms are continuous maps. One may specialize to the category $\mathbf{Man}$ of manifolds, or even further to $\mathbf{Man}^\infty$, which consists of differentiable manifolds with differentiable maps as morphisms. Origianlly, category theory was invented in algebraic topology, with the natural category being $\mathbf{Toph}$, whose morphisms are homotopy classes of maps.
\end{example}

\begin{example}
    The category of partially ordered sets, with order preserving maps the morphisms. In particular, the full subcategory $\Delta$ consisting solely of the ordered sets $\{ 1, \dots, N \}$, for all $N \in \mathbf{N}$ known as the {\bf simplicial category}.
\end{example}

\begin{example}
    The category of graphs {\bf Graph}, whose objects consist of sets of vertices and edges between vertices, and whose morphisms map edges to edge, and vertices to vertices, such that the vertices of an edge are preserved. In some sense, this category is incredibly general, because every category can be seen as a graph with edges the morphisms between objects.
\end{example}

An {\bf isomorphism} in a category is a morphism $f:A \to B$ for which there is $g: B \to A$ such that $g \circ f = \text{id}_A$ and $f \circ g = \text{id}_B$. It follows trivially that $g$ is unique, for if $h$ is another inverse, then
%
\[ g = g \circ \text{id}_B = g \circ f \circ h = \text{id}_A \circ h = h \]
%
we denote $g$ by $f^{-1}$. Examples of isomorphism are algebraic isomorphisms, bijective maps, homeomorphisms, and diffeomorphisms, all in one concft. The set of isomorphisms from an object $A$ to itself will be denoted $\text{Aut}(A)$. It is a group, and all groups are isomorphic to automorphisms over some object in a category. By an {\bf operation} or {\bf representation} of a group $G$ on an object $A$ in a category we mean a group homomorphism $f: G \to \text{Aut}(A)$. Linear reputations, which are the representations of a group over the category of vector spaces, but also permutation representations, which are representations over the category of sets. Isomorphisms really are `the same object' in a categorical sense, because there are natural bijections between the morphisms of the object. Let $f: X \to Y$ be an isomorphism. The map $g \mapsto f \circ g \circ f^{-1}$ is then a bijection between $\text{Aut}(X)$ and $\text{Aut}(Y)$. Similarily, $g \mapsto g \circ f$ is a bijection between $\text{Mor}(Y,A)$ and $\text{Mor}(X,A)$. Other useful maps to describe are {\bf sections}, maps $f: X \to Y$ which are {\it left invertible}, such that there are maps $g: Y \to X$ such that $g \circ f = \text{id}_A$. The {\it right invertible} elements are called {\bf retractions}. Closely related to these objects are the {\bf monomorphisms} $f: X \to Y$, those maps such that if $f \circ g_0 = f \circ g_1$, then $g_0 = g_1$, and {\bf epimorphisms} $g: X \to Y$, those maps such that if $f_0 \circ g = f_1 \circ g$, then $f_0 = f_1$.

\begin{example}
    The monomorphisms in the category of sets are precisely the injective maps, and the epimorphisms are precisely the surjective maps. If $F$ is a faithful functor, and $F(f)$ is a monomorphism, then $f$ is a monomorphism, since if $g_0 \circ f = g_1 \circ f$, then $F(g_0) \circ F(f) = F(g_0 \circ f) = F(g_1 \circ f) = F(g_1) \circ F(f)$, so $F(g_0) = F(g_1)$, hence $g_0 = g_1$. Similarily, if $F(f)$ is an epimorphism, then $f$ is an epimorphism. In particular, if we are working in any category with a forgetful functor into the category of sets, then every `injective' morphism is a monomorphism, and every `surjective' morphism is an epimorphism. However, there are certainly monomorphisms that are not injective, and epimorphisms which aren't surjective. The morphism $f: \mathbf{Z} \to \mathbf{Q}$ is the inclusion of the integers in the rationals in the category of rings, then $f$ is not surjective, but is still an epimorphism. If $g_0 \circ f = g_1 \circ f$, then $g_0$ and $g_1$ agree on all integers. But then $g_0(a/b) = g_0(a)/g_0(b) = g_1(a)/g_1(b) = g_1(a/b)$, so $g_0$ agrees with $g_1$ on all rationals. Similarily, in the category of topological spaces, the inclusion of a dense subspace of another space in that space is an epimorphism which isn't surjective.
\end{example}

An isomorphism is certainly a monomorphism and an epimorphism.

\section{Universal Objects}

Let $\mathcal{C}$ be a category. An {\bf initial object} (or a universal repeller) $X$ in the category is an object such that for any other object $A$, there is a unique map $f: X \to A$. A {\bf final object} (or universal attractor) has unique maps $f: A \to X$ for any $A$. It is easy to see that any two initial and final objects in the same category are isomorphic.

\begin{example}
    The trivial group is both initial and final in the category of groups. Similarily, the trivial module is initial and final in the category of modules over a fixed ring. If we consider the category of rings not necessarily with identity, then the initial object is the zero ring, whereas the initial object in the category of rings with identity is the ring $\mathbf{Z}$ of integers.
\end{example}

\begin{example}
    The empty set is an initial object in the category of sets, and a singleton is a final object in this category. The same is true in the category of topological spaces and differentiable manifolds.
\end{example}

We shall describe other {\bf universal objects} in this section, which are objects satisfying some extremal functorial property. Most of the time, one can reduce the understanding of such objects as those which are initial or final in some category related to the original category we are studying. In the next example, we show that final objects are really initial objects in another category. Thus classifying properties of initial objects really classifies the properties of all universal objects.

\begin{example}
    Given a category $\mathcal{C}$, consider the category $\mathcal{C}^{\text{rev}}$, with the same objects, but if $f: A \to B$ is a morphism in $\mathcal{C}$, then $f^{\text{rev}}: B \to A$ is a morphism in $\mathcal{C}^{\text{rev}}$. An initial object in $\mathcal{C}^{\text{rev}}$ is simply a final object in $\mathcal{C}$.
\end{example}

We shall make common use of a certain type of construction. Given any category, we may form a new category whose objects consist of morphisms in the original category, and a morphism between two morphisms $f: A_0 \to B_0$ and $g: A_1 \to B_1$ is a pair of maps $\phi_A: A_0 \to A_1$ and $\phi_B: B_0 \to B_1$ such that
%
\begin{center}
\begin{tikzcd}
    A_0 \arrow{r}{f} \arrow{d}{\phi_A} & B_0 \arrow{d}{\phi_B}\\
    A_1 \arrow{r}{g} & B_1
\end{tikzcd}
\end{center}
%
commutes. In terms of the category of functors, this category is $\mathcal{C}^2$, where $2$ is the category consisting of two objects with a single arrow between them.

There are many variations to this category. For instance, we can fix an object $A$ in the category, and consider the category $A \downarrow \mathcal{C}$ whose objects are all morphisms with $A$ as a domain, and such that a morphism between $f: A \to B$ and $g: A \to C$ is a morphism $h: B \to C$ such that $h \circ f = g$. This is known as a {\bf comma category}. Similarily, we let $\mathcal{C} \downarrow A$ denote the category whose objects are morphisms with codomain $A$, and such that a morphism between $f: B \to A$ and $g: C \to A$ is $h: B \to C$ such that $f = h \circ g$, so that the diagram
%
\begin{center}
\begin{tikzcd}
    B \arrow{rr}{h} \arrow{rd}[below]{f} & & C \arrow{ld}{g}\\
    & A &
\end{tikzcd}
\end{center}
%
commutes. We use these types of constructions on categories to find a more varied class of universal objects involving morphisms between a fixed family of objects.

Given two objects $A$ and $B$, we will construct a {\bf product} object $A \times B$. Consider the category whose objects consist of triplets $(X,f,g)$, where $f: X \to A$, and $g: X \to B$ are morphisms. A morphism between $(X,f_0,f_1)$ and $(Y,g_0,g_1)$ is a morphism $\pi: X \to Y$, such that the diagram
%
\begin{center}
\begin{tikzcd}
    & X \arrow{d}{\pi} \arrow{ld}[above]{f_0} \arrow{rd}{f_1} & \\
    A & Y \arrow{l}{g_0} \arrow{r}{g_1} & B
\end{tikzcd}
\end{center}
%
commutes. A product for $A$ and $B$ is then a final object $(A \times B, \pi_A, \pi_B)$ in this category. It is clear that any products are not only isomorphic in the original category, but also isomorphic in the stronger sense that if $((A \times B)_0, \pi_A^0, \pi_B^0)$ and $((A \times B)_1, \pi_A^1, \pi_B^1)$ are two products, then there must exist a unique isomorphism $\phi: (A \times B)_0 \to (A \times B)_1$ such that $\pi^1_A \circ \phi = \pi^0_A$ and $\pi^1_B \circ \phi = \pi^0_B$. What we have argued is that if $f: X \to A$ and $g: X \to B$ are any two morphisms, then there exists a unique morphism $f \times g: X \to A \times B$, such that the diagram
%
\begin{center}
\begin{tikzcd}
    & X \arrow{d}{f \times g} \arrow{ld}[above]{f} \arrow{rd}{g} & \\
    A & A \times B \arrow{l}{\pi_A} \arrow{r}[below]{\pi_B} & B
\end{tikzcd}
\end{center}
%
commutes. One can define the products of an arbitrary families $A_\alpha$ of objects, and we denote this product as $\prod A_\alpha$.

\begin{example}
    The normal cartesian product $A \times B$ is the product of $A$ and $B$ in the category of sets, which is easy to verify.
\end{example}

\begin{example}
    Given two groups $G$ and $H$, one canonically defines the product $G \times H$ to be the set of all tuples $(g,h)$, with $g \in G$ and $h \in H$, and with multiplication structure $(g,h)(x,y) = (gx,hy)$. The same trick works for products of rings, modules, vector spaces, and sets, where the associated operations are adjusted accordingly.
\end{example}

\begin{example}
    Given two affine varieties $V$ and $W$ contained in $K^n$ and $K^m$, the set $V \times W$ viewed as a subset of $K^{n+m}$ is naturally the product in the category of affine varieties, with $\pi_V(v,w) = v$ and $\pi_W(v,w) = w$. This follows if $V$ is defined by the ideal $I \subset K[x]$, and $W$ by $J \subset K[y]$, then $V \times W$ is defined by $K[x,y] I \oplus K[x,y] J$. Since $\mathbf{P}^n \times \mathbf{P}^m$ embeds in $\mathbf{P}^{(n+1)(m+1) - 1}$ via the Segre embedding $(x,y) \mapsto [x y^T] = [x_i y_j]$. If $ab^T = \lambda xy^T$, and $x_i \neq 0$, then $(a_i/x_i) b_j = \lambda y_j$, so $[b] = [y]$. Similarily, if $y_j \neq 0$, we can divide by one side of the equation to conclude that $[x] = [y]$.
\end{example}

{\bf Coproducts} are obtained from the above definition by reversing the arrows. We consider the category of objects $(X,f,g)$, where $f: A \to X$ and $g: B \to X$ are morphisms. An initial object in this category is the coproduct of $A$ and $B$, denoted $A \amalg B$. Given $f: A \to X$ and $g: B \to X$, we have a unique map $f \amalg g: A \amalg B \to X$, such that
%
\begin{center}
\begin{tikzcd}
    & X & \\
    A \arrow{ru}{f} \arrow{r}[below]{i_A} & A \amalg B \arrow{u}{} & B \arrow{l}{i_B} \arrow{lu}[above]{g}
\end{tikzcd}
\end{center}
%
We may also take coproducts $\coprod A_\alpha$ of an arbitrary family of objects $\{ A_\alpha \}$.

\begin{example}
    If $A$ and $B$ are sets, then $A \amalg B$ can be constructed by taking $a_0 \in A \times \{ 0 \}$ and $b_1 \in B \times \{ 1 \}$, and considering $i_A(a) = a_0$ and $i_B(b) = b_1$. 
\end{example}

\begin{example}
    Given two groups, $G$ and $H$, the coproduct is the free product $G * H$, which is a quotient of the monoid of all finite words with elements in $G$ and $H$ (assumed disjoint) whose operation is concatenation. Consider the equivalence which identifies $(g,g')$ with $g * g'$, and $h * h'$ with $hh'$, and if $e$ is the identity in $G$, and $e'$ the identity in $H'$, then identify $e * h$ and $h * e$ with $h$, and $e' * g$ and $g * e'$ with $g$. Extend this to semigroup congruence. The monoid formed is the free product, and is a group, for $G * H$ is generated by $G \cup H$, and each $g \in G$ and $h \in H$ has an inverse in $G * H$. We have canonical embeddings $i_G: G \to G * H$ mapping $g$ to itself, and $i_H: H \to G * H$ mapping $h$ to itself.
\end{example}

\begin{example}
    Given two modules $M$ and $N$ over an abelian ring $R$, the coproduct is the direct sum $M \oplus N$, which is the set $M \times N$ (where $(m,n)$ is denoted $m \oplus n$) with operations $(m \oplus n) + (x \oplus y) = (m + x) \oplus (n + y)$.
\end{example}

Products and Coproducts are the most basic constructions in category theory, but we have some other occasionally useful objects. Fix an object $Z$ in a category. A product of morphisms in the category $\mathcal{C} \downarrow Z$ is called a {\bf fibre product}, and is the final object with respect to the diagram.
%
\begin{center}
\begin{tikzcd}
    & A \times_Z B \arrow{dd}{f \times_Z g} \arrow{ld}[above]{\pi_f} \arrow{rd}{\pi_g} & \\
    A \arrow{rd}[below]{f} & & B \arrow{ld}{g} \\
    & Z &
\end{tikzcd}
\end{center}
%
$\pi_f$ is known as the pullback of $f$ by $g$, and $\pi_g$ the pullback of $g$ by $f$. Similarily, we may consider {\bf fiber coproducts}, or {\bf pushouts}, the dual object, which is the coproduct of morphisms in the category $Z \downarrow \mathcal{C}$, satisfying the diagram
%
\begin{center}
\begin{tikzcd}
    & Z \arrow{ld}{f} \arrow{rd}{g} \arrow{dd}{f \coprod_Z g} & \\
    A \arrow{rd}{i_A} & & B \arrow{ld}{i_B}\\
    & A \coprod_Z B
\end{tikzcd}
\end{center}
%
More specifically, we have $f: Z \to A$ and $g: Z \to B$, and any family of maps $A \to C$ and $B \to C$ which causes the square of functions to commute factors through $A \coprod_Z B$.

\begin{example}
    Consider the category of sets. Given $f: A \to Z$ and $g: B \to Z$, the natural space is choose for $A \times_Z B$ is the subset of $A \times B$ consisting of $(a,b)$ such that $f(a) = g(b)$. Then $(f \times_Z g)(a,b) = f(a) = g(b)$. It is easy to see any such map into $A$ and $B$ which makes a required diagram commute factors uniquely through $A \times_Z B$. The normal product $A \times B$ does not satisfy the required property of the fibre product, since the factor might not be uniquely defined: the product space contains too much information. Conversely, a fibre coproduct in the category of sets, given $f: Z \to A$ and $g: Z \to B$, the fibre coproduct $A \coprod_Z B$ is the quotient of $A \coprod B$ obtained by identifying $f(z)$ with $g(z)$ for all $z \in Z$, so that the required inclusions $A \to A \coprod_Z B$ and $B \to B \coprod_Z B$ commute with $f$ and $g$. If $h_0: A \to C$ and $h_1: B \to C$ are such that $h_0 \circ f = h_1 \circ g$, then $h_0(f(z)) = h_1(g(z))$, so the maps from $A \coprod B$ to $C$ descend to a map from $A \coprod_Z B$ to $C$ uniquely. A special case is obtained when $Z$ consists of a single point, where $A \coprod_Z B$ is the wedge sum, obtained by identifying a point in $A$ and a point in $B$.
\end{example}

\begin{example}
    Fibre products exist in the category of groups. Let $f: G \to K$ and $g: H \to K$ be two maps. Let $G \times_K H = \{ (x,y) \in G \times H : f(x) = g(y) \}$, and let $\pi_f$ and $\pi_g$ be the standard projections, then define
    %
    \[ f \times_K g = f \circ \pi_f = g \circ \pi_g \]
    %
    Let $\pi: L \to G$, $\rho: L \to H$, and $\psi: L \to K$ be three maps such that
    %
    \[ f \circ \pi = g \circ \rho = \psi \]
    %
    Then we may consider $\pi \times \rho: L \to G \times H$, and the image of $\pi \times \rho$ is contained in $G \times_K H$, for $f(\pi(x)) = g(\rho(x))$, hence we may consider $\pi \times_K \rho: L \to G \times_K H$, obtained by restricting the domain. This map is unique, for the product map is unique. The normal product $G \times H$ does not satisfy the property of $G \times_K H$, because there may not be globally definable morphisms $\pi_f$ and $\pi_g$ making the diagram commute. Fibred coproducts exist, as a natural quotient of the free product.
\end{example}

\begin{example}
    If $Z$ is a final object in the category, then $X \times_Z Y$ is precisely $X \times Y$. This is because given $f_X: X \to Z$ and $f_Y: Y \to Z$, and $g_X: A \to X$, $g_Y: A \to Y$, the universal property of the product implies there is a map $g: A \to X \times Y$, and the projections commute as desired because the final object makes the final commuting parts of the diagram trivial.
\end{example}

\section{Functors and Natural Transformations}

The main reason to rigorously define groups is to define what a homomorphism is, so we can consider groups with similar structure. Categories were invented to define Functors and natural transformations. A {\bf (Covariant) Functor} $F$ between two categories $\mathcal{C}$ and $\mathcal{D}$ is an association of an object $X$ in $\mathcal{C}$ with an object $F(X)$ in $\mathcal{D}$, and associating a morphism $f: X \to Y$ with a morphism $F(f): F(X) \to F(Y)$, such that
%
\[ F(g \circ f) = F(g) \circ F(f) \]
%
whenever $g \circ f$ is defined. A {\bf Contravariant Functor} associative a morphism $f: X \to Y$ with a morphism $F(f): F(Y) \to F(X)$ such that
%
\[ F(g \circ f) = F(f) \circ F(g) \]
%
Often $F(f)$ is denoted $f_*$ in the case of a covariant functor, and $f^*$ for a contravariant functor. Functors are the natural `morphisms' of categories, and together form a category whose objects are themselves categories, denoted $\mathbf{Cat}$. A functor is {\bf faithful} if the map between morphisms is injective for each pair of objects, and {\bf full} if the map is surjective for each pair of objects. A subcategory of a category is called full if the inclusion functor is full.

\begin{example}
    The association of a set $S$ with the free abelian group $\mathbf{Z}\langle S \rangle$ is a covariant functor from $\mathbf{Set}$ to $\mathbf{Ab}$, since a map $f: S \to T$ induces a map
    %
    \[ f_*: \mathbf{Z} \langle S \rangle \to \mathbf{Z} \langle T \rangle \]
    %
    defined by $f_*(\sum_{s \in S} n_i s) = \sum_{s \in S} n_i f(s)$.
\end{example}

\begin{example}
    On the category of vector spaces, the operation of taking a dual space is a contravariant endofunctor on $\textbf{Vect}_K$, mapping a vector space $V$ to it's dual $V^*$, and mapping $f: V \to W$ to the map $f^*: W^* \to V^*$, where $f^*(\lambda) = \lambda \circ f$.
\end{example}

\begin{example}
    In almost every category, the objects are sets equipped with some additional structure. For instance, a group is a set equipped with an operation, a topological space a set equipped with a family of open sets. A morphism is then a function between sets with some additional structure. This leads to the notion of a {\it forgetful functor} into a category of sets. Given a category $\mathcal{C}$, a forgetful functor is a faithful functor $F: \mathcal{C} \to \text{Set}$. Thus an object $A$ in $\mathcal{C}$ corresponds to some set $F(A)$, and morphisms between two objects $A$ and $B$ are represented by functions between $F(A)$ and $F(B)$. It is known as a forgetful functor because it forgets information about the underlying category $\mathcal{C}$, giving us only the function representation of the morphisms in the category.
\end{example}

\begin{example}
    Functors most naturally occur in algebraic topology, which studies associations of topological space in $\mathbf{Top}$ with some algebraic structure, for instance, in the categories $\mathbf{Grp}$, $\mathbf{Ab}$, and $\mathbf{Rng}$. Functors were invented to discuss these associations. For instance, the homology is a covariant functor, associating each topological space $X$ a chain complex $H(X)$.
\end{example}

Given two categories $\mathcal{C}$ and $\mathcal{D}$, we can construct a category $\mathcal{C} \times \mathcal{D}$, whose objects consist of pairs $(A,B)$, where $A$ is an object in $\mathcal{C}$ and $B$ is an object in $\mathcal{D}$, and a morphism between $(A_0, B_0)$ and $(A_1,B_1)$ is a pair of morphisms from $A_0$ to $A_1$ and $B_0$ to $B_1$. One verifies that with this structure, $\mathcal{C} \times \mathcal{D}$ is a product for $\mathcal{C}$ and $\mathcal{D}$ in the category o fall categories. A functor with domain a product of $\mathcal{C}$ or $\mathcal{C}^{\text{opp}}$ with $\mathcal{D}$ or $\mathcal{D}^{\text{opp}}$ is known as a {\bf bifunctor}, covariant or contravariant in the various variables. One can verify that if for each $B \in \mathcal{D}$, we have a functor $R_B$ with domain $\mathcal{C}$, and for each $A \in \mathcal{C}$, we have a functor $L_A$ on $\mathcal{D}$ such that $L_A(B) = R_B(A)$ for all $A$ and $B$, then we can set $F(A,B) = L_A(B) = R_B(A)$. $F$ can extend the family of functors to a bifunctor if and only if the following property holds; for each $f: A_0 \to A_1$ and $g: B_0 \to B_1$, the only way we can compose them with the funcotrs in a way which relates the two maps is
%
\[ R_{B_0}(f): F(A_0,B_0) \to F(A_1,B_0)\ \ \ \ L_{A_1}(g): F(A_1,B_0) \to F(A_1,B_1) \]
\[ L_{A_0}(g): F(A_0,B_0) \to F(A_0,B_1)\ \ \ \ R_{B_1}(f): F(A_0,B_1) \to F(A_1,B_1) \]
%
We surely must have $L_{A_1}(g) \circ R_{B_0}(f) = R_{B_1}(f) \circ L_{A_0}(g)$, and this is sufficient to define a bifunctor.

\begin{example}
    Given a category $\mathcal{C}$, we have a bifunctor from $\mathcal{C} \times \mathcal{C}$ to {\bf Set} obtained by associating with each pair of objects $A$ and $B$ the set $\text{Mor}(A,B)$, which is covariant in $B$ and contravariant in $A$. Associated with each map $f: A_1 \to A_0$ we have a map $f^*_B: \text{Mor}(A_0,B) \to \text{Mor}(A_1,B)$ given by $f^*_B(g) = g \circ f$, and given $g: B_0 \to B_1$, we have $f_*^A: \text{Mor}(A,B_0) \to \text{Mor}(A,B_1)$ given by $f_*^A(g) = f \circ g$. One verifies that this defines functors $\text{Mor}(\cdot,B)$ and $\text{Mor}(A,\cdot)$. To see this extends to a bifunctor, it suffices to show that $g_*^{A_1} \circ f^*_{B_0} = f^*_{B_1} \circ g_*^{A_0}$ for any $A_0,A_1,B_0$, and $B_1$. But for any function $h: A_0 \to B_0$, we calculate both sides as $g \circ h \circ f$, so we really do have a bifunctor. The dual functor is a special case of this functor, associating a $K$ vector space $V$ with it's dual space $V^* = \text{Mor}(V,K)$: it is the functor $\text{Mor}(\cdot,K)$.
\end{example}

Natural transformations are the natural maps relating functors to each other. Given two functors $F$ and $G$ between two categories $\mathcal{C}$ and $\mathcal{D}$, a natural transformation is an association with each object $X \in \mathcal{C}$ a morphism $\eta_X: F(X) \to G(X)$, such that for each morphism $f: X \to Y$ in $\mathcal{C}$, the diagram
%
\begin{center}
\begin{tikzcd}
    F(X) \arrow{r}{F(f)} \arrow{d}[left]{\eta_X} & F(Y) \arrow{d}{\eta_Y}\\
    G(X) \arrow{r}{G(f)} & G(Y)
\end{tikzcd}
\end{center}
%
commutes. We may therefore consider isomorphisms of functors, known as {\bf natural equivalences}. We will often say a functor itself is natural if it is naturally equivalent to the identity functor on a category. An {\bf equivalence of categories} is a functor $F$ with an `inverse functor' $G$ such that $F \circ G$ and $G \circ F$ are both equivalence to the identity functor. Often, this is a better notion of saying two categories are `equal' then the two categories being isomorphic, which is too strong a condition.

\begin{example}
    The classic example of a natural transformation is that a finite dimensional vector space $V$ it `naturally isomorphic' to its double dual $V^{**}$. What we mean is that the endofunctor which associates $V$ with $V^{**}$, and associates $f: V \to W$ with $f^{**}: V^{**} \to W^{**}$ defined by
    %
    \[ f^{**}(\phi) = \phi \circ f^* \]
    %
    is naturally equivalent to the identity functor on the category of finite dimensional vector spaces. Given a vector space $V$, consider the `double dual' map $(\cdot)^{**}: V \to V^{**}$, which maps $v \in V$ to $v^{**}: V^* \to \mathbf{F}$, defined by $v^{**}(f) = f(v)$. We claim this is a natural transformation. Fix $f: V \to W$. Then, for each $v \in V$, and $\phi \in W^*$,
    %
    \[ f(v)^{**}(\phi) = \phi(f(v)) = (\phi \circ f)(v) = f^*(\phi)(v) = (v^{**} \circ f^*)(\phi) = f^{**}(v^{**})(\phi) \]
    %
    If we instead work over the category of all vector spaces, we can construct a natural transformation from the identity map to the dual functor, but not an inverse natural transformation. However, the operation is natural under other interpretations, such as in the category of all separable Hilbert spaces, where the dual space is interpreted as the space of all {\it continuous} linear functionals. On the same note, the association of $V$ with $V^*$ is `unnatural'; there is no `natural' morphism between $V$ and $V^*$ which does not depend on a choice of basis. However, if we restrict ourselves to the category of finite dimensional vector spaces equipped with a fixed nondegenerate bilinear form, and with morphisms the linear maps preserving this form, then $V$ is naturally isomorphic to $V^*$.
\end{example}

Consider a natural transformation $\eta$ between two bifunctors $F$ and $G$ from $\mathcal{C} \times \mathcal{D} \to \mathcal{E}$. Such a transformation associates with each pair of objects $A$ and $B$ a map $\eta_{A,B}: F(A,B) \to G(A,B)$. Given any such association, we say it is {\bf natural in $A$} if for each fixed $B$ the map $\eta_{\cdot,B}$ is a natural transformation from the functor $F(\cdot,B)$ to the functor $G(\cdot,B)$. Similarily, we can say the functor is natural in $B$. It is useful that $\eta$ is natural in both variables if and only if it is a natural transformation. To see this, given any pair of functions $f: A_0 \to A_1$ and $g: B_0 \to B_1$, we consider the commutative diagram
%
\begin{center}
\begin{tikzcd}
    F(A_0,B_0) \arrow{r}{F(f)} \arrow{d}{\eta_{A_0B_0}} & F(A_1,B_0) \arrow{d}{\eta_{A_1B_0}} \arrow{r}{F(g)} & F(A_1,B_1) \arrow{d}{\eta_{A_1B_1}}\\
    G(A_0,B_0) \arrow{r}{G(f)} & G(A_1,B_0) \arrow{r}{G(g)} & G(A_1,B_1)
\end{tikzcd}
\end{center}
%
and we know the smaller squares commute by naturality in each variable, hence the entire rectangle commutes. This comes up most importantly in the theory of adjoints, where we have a natural bijection between $\text{Mor}(FA,B)$ and $\text{Mor}(A,GB)$ which is natural in each variable $A$ and $B$.

\begin{example}
    If we let $\mathcal{C}$ be the category whose objects are the vector spaces
    %
    \[ K^0, K^1, K^2, \dots \]
    %
    and whose morphisms are the linear maps between $K^n$ and $K^m$, then $\mathcal{C}$ is equivalent to the category of all finite dimensional vector spaces. $\mathcal{C}$ is a subcategory of the category of all finite dimensional vector spaces, so the embedding functor $i$. If we fix, for each vector space $V$, an isomorphism $f_V: V \to K^{\dim(V)}$ once and for all, then one verifies that the functor $G(V) = K^{\dim(V)}$, such that if $g: V \to W$, $G(g) = f_W \circ g \circ f_V^{-1}$. Then the maps $f_V$ and $f_W$ are obviously the required natural equivalences.
\end{example}

The last example generalizes to the following, providing a little bit more intuition about what it means for two categories to be equivalent.

\begin{theorem}
    A functor $F: \mathcal{C} \to \mathcal{D}$ is an equivalence of categories if and only if it is fully faithful and every object $B \in \mathcal{D}$ is isomorphic to $F(A)$ for some $A \in \mathcal{C}$.
\end{theorem}
\begin{proof}
    For each object $B$, pick $A_B$ and an isomorphism $f_B: B \to F(A_B)$. Assume for simplicity that if $B$ is in the image of $F$, then $F(A_B) = B$. If we define $G(B) = A_B$, then for each $g: B \to B'$, we have a morphism $G(g): A_B \to A_{B'}$ given by the unique morphism with the property that $F(G(g)) = f_{B'} \circ g \circ f_B^{-1}$. The maps $f_B$ are then a natural equivalence between the identity and $F \circ G$, since the required square obviously commutes. But since $G$ is also fully faithful, and every object in $\mathcal{C}$ is isomorphic to $G(B)$ for some object $A \in \mathcal{C}$ (since $F(A)$ is isomorphic to $F(G(F(A))$, $A$ is isomorphic to $(G \circ F)(A)$), we can apply the previous case to conclude that $G \circ F$ is naturally equivalent to the identity. Conversely, given a functor $F$ with an inverse $G$ with a natural equivalence $\eta$ from $G \circ F$ to $\text{id}_{\mathcal{C}}$, and an equivalence $\nu$ from $F \circ G$ to $\text{id}_{\mathcal{D}}$, we claim that $F$ is fully faithful. If $F(f) = F(g)$, then by the naturality of $\eta$ and $\nu$ we have a diagram
    %
    \begin{center}
    \begin{tikzcd}
        A \arrow[bend left=90,swap]{rr}{\text{id}_A} \arrow[swap]{d}{f} & (G \circ F)(A) \arrow[swap]{l}{\eta_A} \arrow{r}{\eta_A} \arrow{d}{} & A \arrow{d}{g} \\
        B & (G \circ F)(B) \arrow{l}{\eta_B} \arrow[swap]{r}{\eta_B} & B \arrow[bend left=90, swap]{ll}{\text{id}_B}
    \end{tikzcd}
    \end{center}
    %
    which is commutative because the two squares are commutative, and the upper and lower semicircles are commutative. From it, we conclude that $f = g$. This shows that $F$ is faithful if it has a left equivalence inverse. To show that it is full, we note that since $G$ has a left equivalence inverse, it is also faithful. Thus, given $g: F(A) \to F(B)$, finding $f$ such that $F(f) = g$ is equivalent to finding $f$ such that $(G \circ F)(f) = G(g)$. Since the maps $\eta_A$ and $\eta_B$ are isomorphisms, there certainly exists a morphism $f$ such that the diagram below commutes
    %
    \begin{center}
    \begin{tikzcd}
        A \arrow[bend left=90,swap]{rr}{\text{id}_A} \arrow[swap]{d}{(G \circ F)(f)} & (G \circ F)(A) \arrow[swap]{l}{\eta_A} \arrow{r}{\eta_A} \arrow{d}{f} & A \arrow{d}{G(g)} \\
        B & (G \circ F)(B) \arrow{l}{\eta_B} \arrow[swap]{r}{\eta_B} & B \arrow[bend left=90, swap]{ll}{\text{id}_B}
    \end{tikzcd}
    \end{center}
    %
    from which it follows that $(G \circ F)(f) = G(g)$, and the fully faithfulness of $F$ is established. If $B$ is an object in the codomain, then it is isomorphic to $(G \circ F)(B)$ by the isomorphism $\eta_B$, completing the proof.
\end{proof}

The composition of two natural transformations is verified by checking the diagram to be a natural transformation, so for any two categories $\mathcal{C}$ to $\mathcal{D}$, the family $\mathcal{D}^{\mathcal{C}}$ of all functors from $\mathcal{C}$ to $\mathcal{D}$ forms a category, with morphisms the natural transformations. We let $\text{Nat}(F,G)$ be the family of all natural transformations between two functors $F$ and $G$. The isomorphisms in this category are precisely the natural equivalences between functors.

\begin{example}
    If $M$ is a monoid viewed as a category, the category $\textbf{Set}^M$ is the category of monoid actions on sets, with morphisms preserving the action of $M$. If $G$ is a group viewed as a category, and $A$ is a ring, then $\textbf{Mod}_A^G$ is the category of representations of $G$ over $A$ modules, with the morphisms the intwining operators.
\end{example}

\begin{example}
    The categor $\mathcal{C}^1$ is isomorphic to $\mathcal{C}$, where the objects of the category $\mathcal{C}^2$ are the arrows of the category $\mathcal{C}$, and the morphism those maps between the domain and codomain of arrows which cause the natural diagram to commute. If $X$ is a category with finitely many objects and no non-identity arrows, then $\mathcal{C}^X$ is just the product category of $\mathcal{C}$ $X$ times.
\end{example}

The family of natural transformations between functors has an additional `horizontal' way to compose two functors, aside from the normal composition of maps. Given two natural transformations $\eta: F_0 \to G_0$ and $\psi: F_1 \to G_1$ between functors $F_0$ and $G_0$ from $\mathcal{C}$ to $\mathcal{D}$ and $F_1$ and $G_1$ from $\mathcal{D}$ to $\mathcal{E}$, we can form the composition $F_1 \circ F_0$ and $G_1 \circ G_0$. For any object $A$, we have a commutative diagram
%
\begin{center}
\begin{tikzcd}
    (F_1 \circ F_0)(A) \arrow{r}{\psi_{F_0(A)}} \arrow{d}{F_1(\eta_A)} & (G_1 \circ F_0)(A) \arrow{d}{G_1(\eta_A)}\\
    (F_1 \circ G_0)(A) \arrow{r}{\psi_{G_0(A)}} & (G_1 \circ G_0)(A)
\end{tikzcd}
\end{center}
%
which commutes because $\psi$ is natural. The composition of either of the two directions is denoted by $(\eta \cdot \psi)_A: (F_1 \circ F_0)(A) \to (G_1 \circ G_0)(A)$. It is a natural transformation, because given any map $f: A \to B$, we can construct the diagram
%
\begin{center}
\begin{tikzcd}
    (F_1 \circ F_0)(A) \arrow[bend left=30,swap]{rr} \arrow{rd} \arrow{r} \arrow{d} & (G_1 \circ F_0)(A) \arrow{d} \arrow[bend left=30,swap]{rr}     & (F_1 \circ F_0)(B) \arrow{rd} \arrow{r} \arrow{d} & (G_1 \circ F_0)(B) \arrow{d}\\
    (F_1 \circ G_0)(A) \arrow{r} \arrow[bend right=30,swap]{rr} & (G_1 \circ G_0)(A) \arrow[bend right=30,swap]{rr} & (F_1 \circ G_0)(B) \arrow{r} & (G_1 \circ G_0)(B)\\
\end{tikzcd}
\end{center}
%
the two `circles' commute because $\eta$ is natural, the two squares commute because $\psi$ is natural, and the lower and upper curved lines commute also because $\psi$ is natural. This shows the entire diagram is commutative.

Thus we have two operations on natural transformations, vertical and horizontal composition. They are both easily verified to be associative, and the identity natural transformation acts as the identity under both operations. The most interesting relation is the `interchange law'
%
\[ (\psi_1 \cdot \psi_0) \circ (\eta_1 \cdot \eta_0) = (\psi_1 \circ \eta_1) \cdot (\psi_0 \circ \eta_0) \]
%
which asserts that `vertical composition' and `horizontal composition' commutes with one another. It is easily verified because both sides of the relation are the diagonal of the following diagram
%
\begin{center}
\begin{tikzcd}
    F_1(F_0(A)) \arrow{r} \arrow{d} \arrow{rd} & F_1(G_0(A)) \arrow{r} \arrow{d} & F_1(H_0(A)) \arrow{d}\\
    G_1(F_0(A)) \arrow{r} \arrow{d} & G_1(G_0(A)) \arrow{rd} \arrow{r} \arrow{d} & G_1(H_0(A)) \arrow{d}\\
    H_1(F_0(A)) \arrow{r} & H_1(G_0(A)) \arrow{r} & H_1(H_0(A))
\end{tikzcd}
\end{center}
%
Thus we have two products, defined on a certain subset of pairs of natural transformations, satisfying the interchange law when both sides of the law make sense. 



\chapter{Limits and Adjoints}


\section{The Yoneda Lemma}

The Yoneda lemma says that an object $A$ is fully understood by either the homomorphisms into it, or the homomorphisms out of it. More precisely, for any object $A$ in a category, we can consider a functor $h^A$, also denote $\text{Hom}(A,\cdot)$ from that category to {\bf Set}, with $h^A(X) = \text{Mor}(A,X)$, and such that if $f: X \to Y$, then $f_* = h^A(f): \text{Mor}(A,X) \to \text{Mor}(A,Y)$ is given by $f_*(g) = f \circ g$. We also recall the notion of a functor category $\mathcal{D}^{\mathcal{C}}$, whose object consist of functors from $\mathcal{C}$ to $\mathcal{D}$, and whose morphisms are natural transformations between two functors.

\begin{theorem}[Yoneda]
    For any functor $F$ from $\mathcal{C}$ to {\bf Set}, we have a one to one correspondence between $\text{Nat}(\text{Hom}(A,\cdot),F)$ and $F(A)$, which is natural equivalence when both sides are considered as bifunctors from $\mathcal{C} \times \mathbf{Set}^{\mathcal{C}}$ to {\bf Set}.
\end{theorem}
\begin{proof}
    Consider a natural transformation $\eta$ from $\text{Hom}(A,X)$ to $F$. Then we have for each object $X$ a map $\eta_X: \text{Mor}(A,X) \to F(X)$, and for each morphism $f: X \to Y$, and $g: A \to X$, $F(f)(\eta_X(g)) = \eta_Y(f \circ g)$. Define $x = \eta_A(\text{id}_A) \in F(A)$. Then we know that for any $f: A \to X$,
    %
    \[ F(f)(x) = F(f)(\eta_A(\text{id})) = \eta_X(f \circ \text{id}) = \eta_X(f) \]
    %
    \[ F(f)(x) = F(f)(\eta_A(\text{id})) = \eta_X(f) \]
    %
    Thus $\eta_A(\text{id})$ uniquely determines the natural transformation for all elements of $\text{Mor}(A,X)$. Given any $x \in F(A)$, the equation $\eta_X(f) = F(f)(x)$ is a natural transformation from $\text{Mor}(\cdot,A)$ to $F$, since then for any $f: X \to Y$, and $g: A \to X$,
    %
    \[ F(f)(\eta_X(g)) = (F(f) \circ F(g))(x) = F(f \circ g)(x) = \eta_Y(f \circ g) \]
    %
    For each functor $F$, and objects $A$ and $X$. Thus the correspondence really is a bijection. We now prove the association is natural. First, we prove naturality in $F$. Given a natural transformation $\psi$ between $F$ and some functor $G$, we obtain a square
    %
    \begin{center}
    \begin{tikzcd}
        \text{Nat}(\text{Hom}(A,\cdot),F) \arrow{d}{\psi_*} \arrow{r} & F(A) \arrow{d}{\psi_A} \\
        \text{Nat}(\text{Hom}(A,\cdot),G) \arrow{r} & G(A)
    \end{tikzcd}
    \end{center}
    %
    Let $\eta$ be a natural transformation between $\text{Hom}(A,\cdot)$ and $F$. Then we must show that $\psi_A(\eta_A(\text{id})) = (\psi_* \eta)_A(\text{id})$, which is precisely the definition of $\psi_*$. Thus the map is natural in $F$. Given a morphism $f: A \to B$, and a natural transformation $\eta$, between $\text{Hom}(A,\cdot)$ and $F$, which is a family of morphisms $\eta_X: \text{Hom}(A,X) \to F(X)$, we can define maps $f_*(\eta)_X: \text{Hom}(B,X) \to F(X)$ given by $f_*(\eta)_X(g) = \eta_X(g \circ f)$. We get a square
    %
    \begin{center}
    \begin{tikzcd}
        \text{Nat}(\text{Hom}(A,\cdot),F) \arrow{d}{f_*} \arrow{r} & F(A) \arrow{d}{F(f)} \\
        \text{Nat}(\text{Hom}(B,\cdot),F) \arrow{r} & F(B)
    \end{tikzcd}
    \end{center}
    %
    The fact that this diagram is natural follows precisely because $\eta$ is a natural transformation.
\end{proof}

\begin{example}
    Let $\mathcal{C}$ be a single object $X$ and such that every morphism is an isomorphism. Then $\mathcal{C}$ precisely describes the data of a group $G = \text{Mor}(X,X) = \text{Aut}(X,X)$, and a covariant functor from $\mathcal{C}$ to {\bf Set} is just an action of $G$ on a set $S$ in disguise. The map $h^X$ just maps $X$ to $\text{Aut}(X)$, and maps $x \in G$ to the action $x_*(y) = xy$. Thus the Yoneda lemma says that the $G$ morphisms from $h^X$ to $S$ is naturally in one to one correspondence with the elements of $S$. For each $s \in S$, we have the map $x \mapsto xs$, for $x \in \text{Aut}(x)$.
\end{example}

\section{Free Categories}

We now describe some constructions we can perform on categories which enable us to describe more advanced universal properties more naturally. Let $G$ be a directed graph. By the {\bf free category} on $G$ we mean the category $C(G)$ with objects the vertices of $G$, and whose morphisms between two vertices are the space of all paths in $G$. Given any category $\mathcal{C}$, we let $G(\mathcal{C})$ denote the graph defining $\mathcal{C}$. The free category has the universal property that any morphism $F: G \to G(\mathcal{C})$ of graphs extends to a unique functor from $C(G)$ to $\mathcal{C}$.

\begin{example}
    If $G$ is a graph with only a single vertex, and a single edge $e$, $C(G)$ consists of a single object, and the morphisms on $C(G)$ are $\text{id}$, $e$, $e^2$, and so on and so forth. More generally, given a set $X$ of edges on a single vertex, the free categories is just the construction of the free monoid on $X$.
\end{example}

We have a natural bijection from $\text{Mor}(C(G), \mathcal{C})$ to $\text{Mor}(G, G(\mathcal{C}))$, so that the free category on a graph is the left adjoint to the forgetful functor from the category of categories to the category of graphs.

\section{Comma Categories}

If $A$ is an object in a category $\mathcal{C}$, we can define a new category, denoted $A \downarrow \mathcal{C}$, whose objects are the arrows from $A$ to objects in $\mathcal{C}$, and a morphism between two arrows $f: A \to B$ and $g: A \to C$ is a morphism $h: B \to C$ such that $h \circ f = g$. We have already seen this category






\section{Adjoint Functors}

Universal properties characterize objects in a category. Adjoint functors characterize functors. Two functors $F: \mathcal{C} \to \mathcal{D}$ and $G: \mathcal{D} \to \mathcal{C}$ are known as an {\bf adjoint pair} if, for each $A \in \mathcal{C}$ and $B \in \mathcal{D}$ there is a natural bijection between $\text{Mor}(F(A), B)$ and $\text{Mor}(A,G(B))$. $F$ is known as a left adjoint to $G$, and $G$ a right adjoint to $F$. It is easy to check that being a right or left adjoint to a particular functor defines a functor up to natural isomorphism.

\begin{example}
    In the category of modules over a ring $A$, we have a bijection between $\text{Hom}(M \otimes N, P)$ and $\text{Hom}(M, \text{Hom}(N,P))$, where an element of $f: M \otimes N \to P$ on the left hand side, corresponds to $f'(x)(y) = f(x \otimes y)$ on the right hand side. We will show this bijection shows that the tensor product functor $(\cdot) \otimes N$ is a left adjoint to the homomorphism function $\text{Hom}(N,\cdot)$. For any $g: M_1 \to M_0$, we have a commutative diagram
    %
    \begin{center}
    \begin{tikzcd}
        \text{Hom}(M_0 \otimes N,P) \arrow{r}{g^*} \arrow{d} & \text{Hom}(M_1 \otimes N,P) \arrow{d}\\
        \text{Hom}(M_0,\text{Hom}(N,P)) \arrow{r}{g^*} & \text{Hom}(M_1,\text{Hom}(N,P))
    \end{tikzcd}
    \end{center}
    %
    because if $f: M_0 \otimes N \to P$, $x \in M_1$, and $y \in N$, then
    %
    \[ g^*(f')(x)(y) = (f' \circ g)(x)(y) = f'(g(x))(y) = f(g(x) \otimes y) \]
    \[ (g^* f)'(x)(y) = (g^* f)(x \otimes y) = f(g(x) \otimes y) \]
    %
    Similarily, given any $g: P_0 \to P_1$, we have a commutative diagram
    %
    \begin{center}
    \begin{tikzcd}
        \text{Hom}(M \otimes N,P_0) \arrow{r}{g_*} \arrow{d} & \text{Hom}(M \otimes N,P_1) \arrow{d}\\
        \text{Hom}(M,\text{Hom}(N,P_0)) \arrow{r}{g_*} & \text{Hom}(M,\text{Hom}(N,P_1))
    \end{tikzcd}
    \end{center}
    %
    where
    %
    \[ g_*(f')(x)(y) = g(f'(x)(y)) = g(f(x \otimes y)) \]
    \[ (g_* f)'(x)(y) = (g_* f)(x \otimes y) = g(f(x \otimes y)) \]
    %
    Thus the correspondence is natural.
\end{example}

\begin{example}
    If we have a morphism of rings $B \to A$, then every $A$ module can be considered as a $B$ module, so we get a `restriction of scalars' functor $F(M) = M_B$ from the category of $A$ modules to the category of $B$ modules. This functor is right adjoint to the `extension of scalars' functor $G(M) = M \otimes_B A$ from $B$ modules into $A$ modules. This means that there is a bijection between the homomorphisms $\text{Hom}_A(M \otimes_B A, N)$ and $\text{Hom}_B(M,N_B)$. If we have a $B$ morphism $f: M \to N_B$, then we obtain an $A$ linear morphism $f': M \otimes_B A \to N$ by $f'(x \otimes a) = af(x)$. The inverse map takes a map $g$ from $M \otimes_B A$ to $N$ and considers the induced map $x \mapsto x \otimes 1 \to N$ from $M$ to $N_B$.
\end{example}

\begin{example}
    If $S$ is an abelian semigroup, one can consider the groupification $G_S$ of $S$, obtained by the equivalence relation on $S \times S$ by setting $(a,b) \sim (c,d)$ if $a + d = c + b$. Then we embed $S$ in $\text{Grp}(S)$ by the map $s \mapsto [s,0]$. If $f: S \to H$ is a homomorphism, we can define a homomorphism $f: \text{Grp}(S) \to H$ by defining $f[a,b] = f(a) - f(b)$, and this is the unique homomorphism extending the map on $S$ to $\text{Grp}(S)$. The map $\text{Grp}$ is a functor from the category of abelian semigroups to the category of abelian groups. The functor $F$ associating each abelian group $H$ with itself as an {\it abelian semigroup} (a forgetful functor) is then a right-adjoint to the groupification functor. We have a bijection between $\text{Hom}(\text{Grp}(S),H)$ and $\text{Hom}(S,H)$, because every homomorphism $\text{Grp}(S) \to H$ restricts to a homomorphism $S \to H$, and every homomorphism $S \to H$ extends to $\text{Grp}(S) \to H$. The adjoint property follows automatically.
\end{example}

Both of the examples above can be considered in the same family of left and right adjoints. In both situations, we `forget' some structure to an object. The existence of right adjoints to these forgetful functors allows us to construct additional structure out of an existing structure in a way that doesn't really increase the number of morphisms we have.

\chapter{Abelian Categories}

In many categories, we can consider homological arguments. One naturally encounters the homology of vector spaces in linear algebra. In elementary algebraic topology, one obtained homological arguments using abelian groups. We now discuss the general categorical setting where we can consider these types of arguments. An {\bf additive category} is a category such that for any two objects $A$ and $B$, $\text{Mor}(A,B)$ is an abelian group, such that addition distributes over composition, the category has finite products, and the category has a {\bf zero object} $0$ (an object that is both initial and final). In such a scenario, $\text{Mor}(A,B)$ is often denoted $\text{Hom}(A,B)$. An {\bf additive functor} between additive categories is a functor preserving addition.

\begin{theorem}
    If $F: \mathcal{C} \to \mathcal{D}$ is a functor, then $F(0) = 0$
\end{theorem}

\end{document}