\documentclass{article}

\usepackage{amsmath}
\usepackage{amssymb}
\usepackage{amsthm}
\usepackage{esint}
\usepackage{MnSymbol}

\theoremstyle{plain}
\newtheorem{theorem}{Theorem}
\newtheorem{lemma}[theorem]{Lemma}
\newtheorem{corollary}[theorem]{Corollary}
\newtheorem{prop}[theorem]{Proposition}

\theoremstyle{remark}
\newtheorem*{example}{Example}
\newtheorem*{remark}{Remark}

\theoremstyle{definition}
\newtheorem*{defi}{Definition}
\newenvironment{definition}
    {\begin{samepage}\begin{framed}\begin{defi}}
    {\end{defi}\end{framed}\end{samepage}}

\title{Szemeredi Trotter in Three Dimensions}
\author{Jacob Denson}
\date{\today}

\DeclareMathOperator{\RR}{\mathbf{R}}
\DeclareMathOperator{\CC}{\mathbf{C}}
\DeclareMathOperator{\PP}{\mathbf{P}}
\DeclareMathOperator{\LL}{\mathcal{L}}
\DeclareMathOperator{\ord}{\text{ord}}

\begin{document}

\maketitle

In this lecture, we try and explore the Szemeredi-Trotter theorem from a non-topological perspective, i.e. we try and bound the quantities
%
\[ I(P,L) = \sum_{p \in P} \ord_p(L)\ \ \ \ I(L) = \sum_p \ord_p(L) - 1 \]
%
Recall that in the plane, using topological techniques, such as polynomial partitioning, one is able to conclude that for any set of lines $L$ and points $P$,
%
\[ |I(P,L)| \lesssim |P|^{2/3} |L|^{2/3} + |P| + |L| \]
%
One way to test the dependence of this theorem on topological techniques is to try and prove the conjecture over an arbitrary field $K$, rather than over $\RR$. Indeed, for an arbitrary field there are no precise topological properties one can exploit in a proof of an incidence bound. Using mathematical logic, we can even show this truly does test `non topological methods' on the real plane, e.g. a statement about $\RR^2$ is provable using only basic methods of affine/projective geometry if and only if that statement remains true over $K^2$ for any field. If we want to add `synthetic' topological methods, such as exploiting the fact that a point is `between' two other points, then any method using solely these techniques remains true when $K$ is any ordered field. Furthermore, incidence theorems on finite fields have independant interest and have many relations to many problems in number theory and combinatorics, so trying to prove incidence bounds over finite fields is a very important problem. Thus we work with incidence bounds in $\PP^2$, over as a general a field $K$ as possible.

To begin with, let's recall a method from the proof of the Szemeredi Trotter theorem in the plane. Using Cauchy-Schwartz together with an $L^2$ bound on incidences, we were able to prove the bound
%
\begin{equation} I(P,L) \leq |L|^{3/4} |M|^{3/4} \end{equation}
%
This proof was {\it purely} using incidence methods, and holds over any affine / projective space. We then applied polynomial partitioning to sharpen this bound, but we cannot use this technique over arbitrary fields. And indeed, one cannot sharpen (1).

\begin{example}
    Let $K = \mathbf{F}_p$, with $P$ and $L$ consisting of all projective points and lines. Then $|P| = |L| = p^2 + p+1$, and we count
    %
    \[ I(P,L) = (p+1)(p^2+p+1) = \frac{p+1}{(p^2 + p + 1)^{1/2}} |L|^{3/4} |M|^{3/4} \]
    %
    For large $p$, this gives sharpness.
\end{example}

Thus there is no purely affine proof that can obtain the full Szemeredi-Trotter bound in $\RR^2$. But it remains an open question whether the only counterexamples in the two dimensional finite field setting are {\it only} obtained by taking very large sets, i.e. is there a threshold at which sparse sets of points and lines behave like Euclidean space. Our goal in this talk is to discuss some results obtained by Koller of this kind for three dimensional incidences. A key idea here is to utilize the arithmetic genus to upper bound incidences, which is easily calculated using classical results of algebraic geometry.

We begin by discussing a variant of the Guth-Katz bound on $I(L)$, which works over a general class of fields. The proof given here is very close to what we discussed over distinct distances, but we must replace the parts of the proof which involve polynomial partitioning with alternative geometric results. We `almost' succeed to do this.

\begin{theorem}[Theorem 6]
    Let $L$ be a set of projective lines in a field of characteristic $0$, or of characteristic $p$ where $|L| \leq p^2$, and let $c > 0$ be a constant. If no plane or smooth quadric contains more than $c |L|^{1/2}$ lines, then $I(L) \leq (29.1 + 0.5c) |L|^{3/2}$.
\end{theorem}

This theorem has several differences from the Guth-Katz result:
%
\begin{itemize}
    \item Guth-Katz only holds when $K = \RR$.
    \item Their incidence bound counts cardinality rather than the multiplicity.
    \item Explicit constants are not obtained.
\end{itemize}
%
Nonetheless, we obtain Theorem 6 by working very similarly to Guth-Katz, except we must replace applications of polynomial partitioning with a more geometric method.

%\begin{example}
%    NOTE TO DAN: THIS DOESN'T REALLY SHOW SHARPNESS IN THEOREM 6, RIGHT?

%    Consider the grid example, i.e. we let $L$ be the family of $3n^2$ axis parallel lines passing through the $n^3$ points in $\mathbf{Z}^3 \cap [1,n]^3$. For each point $p$ in this lattice, $\text{ord}_p(L) = 3$, and so $I(L) = 2n^3$. The only problem with this example is that two many lines lie in a single plane. But this is easily fixed. If we consider the Veronese-type embedding $(x,y,z) \mapsto (xyz,xy,xz,yz,x,y,z)$, then axis-parallel lines in $\RR^3$ remain straight in the embedding, whereas no hyperplane in $\RR^7$ contains more than two lines. Thus a generic projection from $\RR^7$ to $\RR^3$ gives a family of lines such that any plane contains no more than two lines. Thus the hypothesis of Theorem 1 is satisfied with $c = 1/n$, and so Theorem 1 shows that $I(L) \leq (29.1 + 0.5/n) 3^{3/2} n^3 = 151.2 n^3 + 2.59 n^2$
%\end{example}

\section{Proof of Theorem 6}

Given $L$, standard polynomial counting techniques enable us to find a surface $S$ containing all lines with degree $d \leq (6|L|)^{1/2} - 2$. Decompose $S$ into irreducible components $S_1, \dots, S_k$, and let $L_i$ denote the set of lines that are contained in $S_i$, but not in $S_1, \dots, S_{i-1}$.

\begin{lemma}
    \[ I(L) = \underbrace{\sum_{i = 1}^K I(L_i)} + \underbrace{\sum_{i = 2}^k |L_i \cap (L_1 \cup \dots \cup L_{i-1}))}| \]
    %
    The first term is {\it Internal Incidences}, the second {\it External Incidences}.
\end{lemma}
\begin{proof}
    Note that $\{ L_i \}$ is a partition of $L$, and $\ord_p(L) = \sum \ord_p(L_i)$, so
    %
    \begin{align*}
        \sum_{i = 1}^k I(L_i) &= \sum_{i = 1}^k \sum_{p \in L_i} \ord_p(L_i) - 1\\
        &= \sum_{p \in L} \ord_p(L) - |\{ i: p \in L_i \}|\\
        &= I(L) - \sum_{i = 2}^k |L_i \cap (L_1 \cup \dots \cup L_{i-1})|. \qedhere
    \end{align*}
\end{proof}

External incidences are most easy to bound.

\begin{lemma}
    If the $S_i$ are appropriately ordered,
    %
    \[ \sum_{i = 2}^k |L_i \cap (L_1 \cup \dots \cup L_{i-1})| \leq 1.23 |L|^{3/2}. \]
\end{lemma}
\begin{proof}
    Suppose $l \in L_i$. Then $l$ is not contained in $S_1, \dots, S_{i-1}$. Bezout's theorem thus implies that $l \cap S_j$ contains at most $\deg(S_j)$ points if $j < i$. But $l \cap L_j \subset l \cap S_j$, so a union bound gives
    %
    \[ \sum_{i = 2}^k |L_i \cap (L_1 \cup \dots \cup L_{i-1})| \leq \sum_{j < i} |L_i| \deg(S_j) \leq \sum |L_i| \deg(S_j) = d|L| \leq 6^{1/2} |L|^{3/2} \]
    %
    We can do slightly better by reordering the $S_i$. If $|L_i|/\deg(S_i)$ is monotonically decreasing, then $|L_j|/\deg(S_j) \geq |L_i|/\deg(S_i)$
    %
    \[ \sum_{j < i} |L_i| \deg(S_j) \leq \sum_{j < i} \deg(S_i) |L_j| \]
    %
    Note that
    %
    \[ 2 \sum_{j < i} |L_i| \deg(S_j) \leq \sum_{j < i} |L_i| \deg(S_j) + \sum_{j > i} |L_i| \deg(S_j) \leq 6^{1/2} |L|^{3/2} \]
    %
    Thus $\sum_{j < i} |L_i| \deg(S_j) \leq (3/2)^{1/2} |L|^{3/2}$. This can be tight if $|L_i|\deg(S_j) = |L_j| \deg(S_i)$, and $|L_i|\deg(S_i)$ is small in comparison to the overall sum.
\end{proof}

We now bound internal incidences. Just as in the distinct distances problem, we must separate incidences on non-ruled surfaces from incidences on ruled surfaces, and further separate analysis on planes and quadrics from analysis on the other ruled surfaces.

\begin{lemma}
    If $S_i$ is a plane, $I(L_i) \leq 0.5c \cdot |L|^{1/2} |L_i|$.
\end{lemma}
\begin{proof}
    By assumption, $|L_i| \leq c |L|^{1/2}$. Since we are working in a projective context, any two lines in $L_i$ intersect one another. The number of incidences is maximized if any pair of lines intersects in a distinct location. Thus we find $I(L_i) \leq |L_i|^2/2$.
\end{proof}

\begin{lemma}
    If $S_i$ is a singular quadric (a cone), $I(L_i) = |L_i| - 1$.
\end{lemma}
\begin{proof}
    A singular quadric has a vertex, and any straight line on the surface passes through the vertex. This gives the result directly.
\end{proof}

\begin{lemma}
    If $S_i$ is a smooth quadric, $I(L_i) \leq 0.5c \cdot |L_i| |L|^{1/2}$.
\end{lemma}
\begin{proof}
    Since $S_i$ is a smooth quadric, the family of lines on $S$ splits into two rulings. Write $L_i = L_i^0 \cup L_i^1$, where $L_i^0$ is the first ruling, and $L_i^1$ the second ruling, and no lines intersect in the same family. Thus
    %
    \[ I(L_i) \leq |L_i^0| |L_i^1| \leq |L_i|^2/4 \leq 0.5c \cdot |L|^{1/2}. \qedhere \]
\end{proof}

We say a ruled surface is `non-exception' if it isn't a plane, cone, or quadric.

\begin{lemma}
    If $S_i$ is a `non-exceptional' ruled surface, then
    %
    \[ I(L_i) \leq 0.5 \cdot |L_i| \deg(S_i) + 2 |L_i|. \]
\end{lemma}
\begin{proof}
    If $S_i$ is a non-special ruled surface, then there are at most two disjoint lines on $S_i$, disjoint from each other, which intersect more than $\deg(S_i)$ lines, known as special. If we set $L_i = L_i^0 \cup L_i^1$, where $L_i^0$ are the special lines, then
    %
    \[ I(L_i) = I(L_i^0) + I(L_i^1) + |L_i^0 \cap L_i^1| \leq 0 + 0.5 \cdot \deg(S_i) \cdot |L_i^1| + 2|L_i^1|.\qedhere \]
%        &\leq 0.5 \cdot |L_i| \deg(S_i) + 2 |L_i|. \qedhere
\end{proof}

Thus
%
\begin{align*}
    \sum_{S_i\ \text{ruled}} I(L_i) &= \left( \sum_{\text{planes}} + \sum_{\text{smooth quadric}} \right) + \left( \sum_{\text{cones}} + \sum_{\text{non exceptional}} \right)\\
    &\leq \left( \sum 0.5c \cdot |L|^{1/2} |L_i| \right) + \left( 0.5 \cdot \sum |L_i| \deg(S_i) \right)\\
    &\leq \left( 0.5c \cdot |L|^{3/2} \right) + \left( 0.5 d |L| \right)\\
    &\leq 0.5c \cdot |L|^{3/2} + 1.23 \cdot |L|^{3/2}
\end{align*}
%
All these bounds are essentially as expected from the Guth-Katz result. It's the bounds we obtain for non-ruled surfaces where things can more funky. To work this out, we need to take an aside to discuss the arithmetic genus of a surface.

\section{Arithmetic Genus}

\begin{itemize}
    \item Let $R = k[x_0, \dots, x_d]$.

    \item For a f.g. module $M$ over $R$, the {\it Hilbert function} is $H_M(d) = \dim((R/I)_d)$.

    \item Agrees with degree $\leq d$ polynomial $P_M(d)$ for sufficiently large $d$.

    \item Addtivity: If $0 \to M \to N \to L \to 0$ is exact, then $P_N = P_M + P_L$.

    \item If $X$ is a projective variety/subscheme of $\PP^n$ corresponding to some homogenous ideal $I$, we let $P_X = P_{R/I}$.

    \item $P_X(d)$ describes {\it all} additive invariants of $X$, e.g.
    \begin{itemize}
        \item $\deg(P_X) = \dim(X)$.
        \item The leading coefficient is $\deg X / (\dim X)!$
        \item The constant coefficient is the Euler characteristic $\chi(X,\mathcal{O})$.
    \end{itemize}

    \item If $X$ is a projective curve, then one calls $g(X) = 1 - \chi(X,\mathcal{O})$ the {\it arithmetic genus} of $X$. This agrees with the topological genus if $X$ is a smooth curve over the complex numbers.
\end{itemize}

\begin{lemma}
    If $f$ is homogenous of degree $a$, and $H$ is the projective hypersurface corresponding to $a$, then for any $X$, $P_{X \cap H}(d) = P_X(d) - P_X(d-a)$.
\end{lemma}
\begin{proof}
    If $f$ is homogenous of degree $a$, let $H$ be the projective variety corresponding to $f$. If $f$ is not a zero-divisor of $X$, the map $g \mapsto fg$ induces an exact sequence of graded modules
    %
    \[ 0 \to [R/I(X)](a) \to R/I(X) \to R/I(X \cap H) \to 0 \]
    %
    Thus $P_X(d-a) + P_{X \cap H}(d) = P_X(d)$, for $d \geq a$.
\end{proof}

Let $H_i$ be a projective hypersurface with degree $a_i$, such that $B = H_1 \cap \dots \cap H_{n-1}$ is one-dimensional. $B$ is known as a {\it complete intersection curve}.

\begin{lemma}
    $g(B) = 1 + \frac{\sum a_i - (n + 1)}{2} \prod_{i = 1}^{n-1} a_i$.
\end{lemma}
\begin{proof}
     Since
    %
    \[ P_{\PP^n}(d) = {{d+n} \choose {n}} \]
    %
    Then using the last example recursively, we conclude
    %
    \[ P_B(d) = \left( \prod_{i = 1}^{n-1} a_i \right) \cdot d - \frac{\sum a_i - (n + 1)}{2} \prod_{i = 1}^{n-1} a_i. \]
    %
    Thus
    %
    \[ g(B) = 1 - \chi(B) = 1 + \frac{\sum a_i - (n + 1)}{2} \prod_{i = 1}^{n-1} a_i. \qedhere \]
\end{proof}

There are a few technicalities here, related to the fact that intersection here and in the last lemma are {\it scheme theoretical}, rather than {\it set theoretical}. And the genus can disagree for these two objects.

\begin{example}
    Let $I_1 = (xy-zt)$, and $I_2 = (x^2 + xy - zt)$ be homogenous ideals, generating hyperboloids $H_1 = V(I_1)$ and $H_2 = V(I_2)$ in $\PP^3$. Let $B = H_1 \cap H_2$ be the scheme theoretic intersection. Then $I_1 \oplus I_2 = (x^2, xy - zt)$, and an easy algebraic calculation shows $x \not \in I_1 \oplus I_2$, but $x^2 \in I_1 \oplus I_2$. Thus $R/(I_1 \oplus I_2)$ is not a reduced ring, and the reduction of this ring is $R/I(x,zt)$ with corresponding `reduced curve' $B'$. The last lemma shows $P_B(d) = 4d$, yet
    %
    \[ P_{B'}(d) = \sum_{i = 0}^d 2(d-i) = d^2 + d \]
\end{example}


\section{Salmon's Construction}

TODO: Define a ruled surface.

\begin{theorem}
    Let $f$ be a degree $d$ polynomial. Then there is a polynomial $\text{Flec}(f)$ of degree no more than $11d - 24$ which vanishes at every point $p \in V(f)$ for which there is a triple tangent for $f$ through $p$.
\end{theorem}

\begin{corollary}
    If $S$ is a surface of degree $d$, such that no irreducible component is ruled, then the surface $T$ with degree at most $11d - 24$ shares no irreducible components with $S$.
\end{corollary}

We say a surface $S$ is {\it ruled} if it contains a line through a generic point.

\begin{corollary}
    If $\text{Flec}(f)$, then 
\end{corollary}

If a line $l$ through $p$ is parameterized as $t \mapsto p + mt$, and $f$ is a polynomial 

In particular, $T$ contains every line that $S$ also contains.

\begin{theorem}[Monge-Salmon-Cayley]
    Let $S \subset \mathbf{CP}^3$ be a surface of degree $d$ without ruled irreducible components. Then there is a surface $T$ of degree at most $11d - 24$ such that $S$ and $T$ do not have common irreducible components and every line on $S$ is contained in $S \cap T$.
\end{theorem}
\begin{proof}
    Consider three homogenous polynomials in $\PP^2$:
    %
    \[ f_1(x) = \sum_{1 \leq i \leq 3} a_ix_i\ \ \ f_2(x) = \sum_{1 \leq i \leq j \leq 3} b_{ij} x_i x_j\ \ \ f_3(x) = \sum_{1 \leq i \leq j \leq k \leq 3} c_{ijk} x_i x_j x_k \]
    %
    %
    We want to understand when $V(f_1) \cap V(f_2) \cap V(f_3)$ is non-empty. If $x \in V(f_1)$, we know
    %
    \[ x_3 = -\frac{a_1x_1+a_2x_2}{a_3}. \]
    %
    Then substituting this into $f_2$ and $f_3$ induces forms of degree two and three in the variables $x_2$ and $x_3$, i.e.
    %
    \[ g_1(x_1,x_2) = f_1(x_1,x_2,(a_1x_1 + a_2x_2)/a_3) = \sum_{1 \leq i \leq j \leq 2} B_{ij} x_ix_j \]
    \[ g_2(x_1,x_2) = f_2(x_1,x_2,(a_1x_1 + a_2x_2)/a_3) = \sum_{1 \leq i \leq j \leq k \leq 2} C_{ijk} x_ix_jx_k \]
    %
    Now consider an irreducible surface $S$ given by an equation $f$. Fix $p \in S$, and 
\end{proof}

\section{Intersections of Surfaces}

Given the two surfaces $S$ and $T$, we now need to study it's intersection, which contains all lines in $S$. We rely on the higher dimensional version of Bezout's theorem: If $H_1, \dots, H_n$ are hypersurfaces with degrees $d_1, \dots, d_n$, then $H_1 \cap \dots \cap H_n$ contains an algebraic curve, or consists of at most $d_1 \dots d_n$ points.

\begin{theorem}
    Let $S,T \subset \mathbf{P}^3$ be two surfaces of degree $a$ and $b$ with no common irreducible components. Let $C = S \cap T$. Then
    %
    \begin{enumerate}
        \item $C$ has at most $ab$ irreducible components.
        \item $\sum_{p \in C} (r(p) - 1) \leq ab(a+b-2)/2$.
        \item $\sum_{p \in C} (r(p) - 1)^{3/2} \leq ab(a+b-2)/\sqrt{2}$.
        \item $\sum \left\{ r(p)(r(p) - 1) : p \in C\ \text{and} p\ \text{is smooth on $S$ or $T$} \right\}$
    \end{enumerate}
\end{theorem}
\begin{proof}
    If $H$ is a generic hyperplane, then $C \cap H = S \cap T \cap H$ consists of $ab$ points. Thus $C$ has at most $ab$ irreducible components (and if this number is achieved, $C$ is a union of lines).
\end{proof}

\end{document}