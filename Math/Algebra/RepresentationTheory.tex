\documentclass[12pt, dvipsnames]{report}

\usepackage{amsmath}
\usepackage{algorithm}
%\usepackage{algorithmic}
\usepackage[noend]{algpseudocode}

\usepackage{amsmath}
\usepackage{amssymb}
\usepackage{amsthm}
\usepackage{amsopn}

\usepackage{kpfonts}

\usepackage{graphicx}

% Probably don't need this on notes anymore
%\usepackage{kbordermatrix}

% Standard tool for drawing diagrams.
\usepackage{tikz}
\usepackage{tkz-berge}
\usepackage{tikz-cd}
\usepackage{tkz-graph}

\usepackage{comment}

%
\usepackage{multicol}

%
\usepackage{framed}

%
\usepackage{mathtools}

%
\usepackage{float}

%
\usepackage{subfig}

%
\usepackage{wrapfig}

%
\let\savewideparen\wideparen
\let\wideparen\relax
\usepackage{mathabx}
\let\wideparen\savewideparen

% Used for generating `enlightening quotes'
\usepackage{epigraph}

% Forget what this is used for :P
\usepackage[utf8]{inputenc}

% Used for generating quotes.
\usepackage{csquotes}

% Allows what to generate links inside
% generated pdf files
\usepackage{hyperref}

% Allows one to customize theorem
% environments in mathematical proofs.
\usepackage{thmtools}

% Gives access to a proof
\usepackage{lplfitch}

% I forget what this is for.
\usepackage{accents}

% A package for drawing simple trees,
% as a substitute for unnesacary TIKZ code
\usepackage{qtree}

% Enables sequent calculus proofs
\usepackage{ebproof}

% For braket notation
\usepackage{braket}

% To change line spacing when using mathematical notations which require some height!
\usepackage{setspace}

%\usepackage[dvipsnames]{xcolor}

\usepackage{float}

% For block commenting
\usepackage{comment}




\setlength\epigraphwidth{8cm}

\usetikzlibrary{arrows, petri, topaths, decorations.markings}

% So you can do calculations in coordinate specifications
\usetikzlibrary{calc}
\usetikzlibrary{angles}

\theoremstyle{plain}
\newtheorem{theorem}{Theorem}[chapter]
\newtheorem{axiom}{Axiom}
\newtheorem{lemma}[theorem]{Lemma}
\newtheorem{corollary}[theorem]{Corollary}
\newtheorem{prop}[theorem]{Proposition}
\newtheorem{exercise}{Exercise}[chapter]
\newtheorem{fact}{Fact}[chapter]

\newtheorem*{example}{Example}
\newtheorem*{proof*}{Proof}

\theoremstyle{remark}
\newtheorem*{exposition}{Exposition}
\newtheorem*{remark}{Remark}
\newtheorem*{remarks}{Remarks}

\theoremstyle{definition}
\newtheorem*{defi}{Definition}

\usepackage{hyperref}
\hypersetup{
    colorlinks = true,
    linkcolor = black,
}

\usepackage{textgreek}

\makeatletter
\renewcommand*\env@matrix[1][*\c@MaxMatrixCols c]{%
  \hskip -\arraycolsep
  \let\@ifnextchar\new@ifnextchar
  \array{#1}}
\makeatother

\renewcommand*\contentsname{\hfill Table Of Contents \hfill}

\newcommand{\optionalsection}[1]{\section[* #1]{(Important) #1}}
\newcommand{\deriv}[3]{\left. \frac{\partial #1}{\partial #2} \right|_{#3}} % partial derivative involving numerator and denominator.
\newcommand{\lcm}{\operatorname{lcm}}
\newcommand{\im}{\operatorname{im}}
\newcommand{\bint}{\mathbf{Z}}
\newcommand{\gen}[1]{\langle #1 \rangle}

\newcommand{\End}{\operatorname{End}}
\newcommand{\Mor}{\operatorname{Mor}}
\newcommand{\Id}{\operatorname{id}}
\newcommand{\visspace}{\text{\textvisiblespace}}
\newcommand{\Gal}{\text{Gal}}

\newcommand{\xor}{\oplus}
\newcommand{\ft}{\wedge}
\newcommand{\ift}{\vee}

\newcommand{\prob}{\mathbf{P}}
\newcommand{\expect}{\mathbf{E}}
\DeclareMathOperator{\Var}{\mathbf{V}}
\newcommand{\Ber}{\text{Ber}}
\newcommand{\Bin}{\text{Bin}}

%\newcommand{\widecheck}[1]{{#1}^{\ft}}

\DeclareMathOperator{\diam}{\text{diam}}

\DeclareMathOperator{\QQ}{\mathbf{Q}}
\DeclareMathOperator{\ZZ}{\mathbf{Z}}
\DeclareMathOperator{\RR}{\mathbf{R}}
\DeclareMathOperator{\HH}{\mathbf{H}}
\DeclareMathOperator{\CC}{\mathbf{C}}
\DeclareMathOperator{\AB}{\mathbf{A}}
\DeclareMathOperator{\PP}{\mathbf{P}}
\DeclareMathOperator{\MM}{\mathbf{M}}
\DeclareMathOperator{\VV}{\mathbf{V}}
\DeclareMathOperator{\TT}{\mathbf{T}}
\DeclareMathOperator{\LL}{\mathcal{L}}
\DeclareMathOperator{\EE}{\mathbf{E}}
\DeclareMathOperator{\NN}{\mathbf{N}}
\DeclareMathOperator{\DQ}{\mathcal{Q}}
\DeclareMathOperator{\IA}{\mathfrak{a}}
\DeclareMathOperator{\IB}{\mathfrak{b}}
\DeclareMathOperator{\IC}{\mathfrak{c}}
\DeclareMathOperator{\IP}{\mathfrak{p}}
\DeclareMathOperator{\IQ}{\mathfrak{q}}
\DeclareMathOperator{\IM}{\mathfrak{m}}
\DeclareMathOperator{\IN}{\mathfrak{n}}
\DeclareMathOperator{\IK}{\mathfrak{k}}
\DeclareMathOperator{\ord}{\text{ord}}
\DeclareMathOperator{\Ker}{\textsf{Ker}}
\DeclareMathOperator{\Coker}{\textsf{Coker}}
\DeclareMathOperator{\emphcoker}{\emph{coker}}
\DeclareMathOperator{\pp}{\partial}
\DeclareMathOperator{\tr}{\text{tr}}

\DeclareMathOperator{\supp}{\text{supp}}

\DeclareMathOperator{\codim}{\text{codim}}

\DeclareMathOperator{\minkdim}{\dim_{\mathbf{M}}}
\DeclareMathOperator{\hausdim}{\dim_{\mathbf{H}}}
\DeclareMathOperator{\lowminkdim}{\underline{\dim}_{\mathbf{M}}}
\DeclareMathOperator{\upminkdim}{\overline{\dim}_{\mathbf{M}}}
\DeclareMathOperator{\lhdim}{\underline{\dim}_{\mathbf{M}}}
\DeclareMathOperator{\lmbdim}{\underline{\dim}_{\mathbf{MB}}}
\DeclareMathOperator{\packdim}{\text{dim}_{\mathbf{P}}}
\DeclareMathOperator{\fordim}{\dim_{\mathbf{F}}}

\DeclareMathOperator*{\argmax}{arg\,max}
\DeclareMathOperator*{\argmin}{arg\,min}

\DeclareMathOperator{\ssm}{\smallsetminus}

\title{Representation Theory of Groups}
\author{Jacob Denson}

\begin{document}

\pagenumbering{gobble}

\maketitle

\tableofcontents

\pagenumbering{arabic}

\chapter{Basic Representation Theory}

The general linear group $GL(V)$ of invertible operators on a vector space $V$ is one of the easiest groups to understand. The techniques of finite dimensional linear algebra provide us with classifications of the elements under inner automorphisms, and spectral theory allows us to decompose the elements of $GL(V)$ into simpler suboperations on the space $V$. However, the group still has nontrivial structure, and we would hope that the group has enough structure that we can reduce problems about general groups to problems about the general linear group. This is the heart of representation theory, which tries to understand an arbitrary group $G$ by understanding its {\bf representations}, homomorphisms $\rho: G \to GL(V)$ over some vector space $V$. Representation theory is fundamental to the understanding of groups, so much that it is said that group theory was invented so that we could discuss representation theory. Indeed, most groups that were studied before abstract group theory came about were naturally representation as subgroups of some $GL(V)$.

Sometimes, it is convenient to eschew notation, and talk about a representation of a group on a vector space as an {\bf action}. That is, we think of a representation of a group $G$ over a vector space $V$ as as a bilinear map $G \times V \to V$, such that
%
\[ g(hv) = (gh)v\ \ \ \ \ g( \lambda v + \gamma w) = \lambda (gv) + \gamma (gw)\ \ \ \ \ ev = v \]
%
Given some homomorphism $\rho: G \to GL(V)$, we obtain an action by composing the homomorphism with the evaluation map $GL(V) \times V \to V$. Conversely, given an action $G \times V \to V$, we obtain a map $G \to GL(V)$ by mapping $g \in G$ to the operator $Tv = gv$. Thus we will interchangably switch between talking about a representation of $G$ on $GL(V)$ (by which we mean a homomorphism $G \to GL(V)$), and a representation of $G$ on $V$ (by which we mean an action of $G$ on $V$).

A {\bf subrepresentation} of a representation $V$ of a group $G$ is a subspace $W \subset V$ which is closed under the action of $G$. That is, if $\rho:G \to GL(V)$ is the representation, then $W$ is an invariant subspace of each $\rho(g)$. We may then restrict to a representation $\rho: G \to GL(W)$, so that $W$ is also a representation of $G$. If we have two {\it complementary subrepresentations} $V = W \oplus U$, then $V$ is called {\bf reducible}. If a nontrivial representation $V$ has no nonzero proper subrepresentation, then we say that $V$ is {\bf irreducible}. Given two representations $V$ and $W$, we can define a representation on the direct sum $V \oplus W$ of vector spaces by letting $g(v + w) = (gv) + (gw)$. The theory of subrepresentations allows us to break down representations into simpler forms, and to form more complicated representations into bigger structures. We shall find that, at least over the finite groups, all representations can be decomposed into irreducible representations, so representation theory breaks down into understanding the irreducible representations of some group $G$.

\begin{example}
    You may think that the study of representations is a subfield of the study of group actions on a set. Certainly every representation gives rise to a group action, but we can also represent general group actions as representations on a vector space. Given a group $G$, consider a group action of $G$ on some set $X$. The group action induces a representation of $G$ on the free vector space generated by $X$, given by
    %
    \[ g \left( \sum a_i x_i \right) = \sum a_i (gx_i) \]
    %
    This is known as a {\bf permutation representation} of $G$. More generally, an arbitrary representation of $G$ over a vector space $V$ is known as a permutation representation if there is a basis $\{ e_\alpha \}$ for $V$ over an index set $A$, and a homomorphism $\varphi: G \to \text{Sym}(A)$ such that $ge_\alpha = e_{g(\alpha)}$. This shows that representation theory can be used as a more general language for the study of permutation representations. Orbits of the group action give rise to subrepresentations, and fixed points correspond to one dimensional subrepresentations.
\end{example}

\begin{example}
    If $G$ is some finite group, and $K$ is a field, then we define the {\bf group algebra} of $G$, denoted $K[G]$, to be the set of all formal sums $\sum_{g \in G} a_g g$, with $a_g \in K$, where addition is componentwise on each group element, and the multiplication operation is
    %
    \[ \left( \sum_{g \in G} a_g g \right) \left( \sum_{h \in G} b_h h \right) = \sum_{g \in G} \sum_{h \in G} (a_g b_h) (gh) \]
    %
    We have a canonical embedding of $G$ in $K[G]$ which gives rise to a representation of $\lambda: G \to GL(K[G])$, letting
    %
    \[ \lambda(g) \left( \sum a_h h \right) = (g) \left( \sum a_h h \right) = \sum a_h gh \]
    %
    This is known as the {\bf left regular representation} of $G$. Another way to think of $K[G]$ is as the space $K^G$ of all $K$-valued functions on $G$, and given some function $f: G \to K$, we find that the group acts as
    %
    \[ (gf)(x) = f(g^{-1}x) \]
    %
    We could also define the right action of $G$ on $K^G$ by letting
    %
    \[ (fg)(x) = f(xg) \]
    %
    this is the {\bf right regular representation} $\rho: G \to GL(K^G)$, which satisfies
    %
    \[ \rho(gh) = \rho(h) \circ \rho(g) \]
    %
    so that $\rho$ is a homomorphism with respect to the reverse group structure on $G$.
\end{example}

As with the theory of group actions, the class of representations of a given group $G$ form a category, since we can consider morphisms between $\rho: G \to GL(V)$ and $\nu: G \to GL(W)$ to be a linear map $T: V \to W$ such that $T(\rho(g)(x)) = \nu(g)(Tx)$. That is, for each $g \in G$, the diagram
%
\begin{center}
\begin{tikzcd}
    V \arrow{d}[left]{\rho(g)} \arrow{r}{T} & W \arrow{d}{\nu(g)}\\
    V \arrow{r}{T} & W
\end{tikzcd}
\end{center}
%
commutes. We will call such a map {\bf $G$-linear}. The kernel of a $G$-linear map is a subrepresentation, and if $W$ is a subrepresentation of $V$, then $V/W$ has a natural representation structure given by the quotient of the representation on $V$. As with the category of vector spaces, we have all three standard isomorphism theorems. It is of interest to classify the representations of a group, because this determines all possible ways the group can act on a vector space, and this likely gives information about the structure of $G$ by itself. Most of representation theory can be seen as describing techniques for classifying the representations of a group.

\section{Complete Decomposability}

A representation is {\bf completely decomposable}, or {\bf semisimple}, if it can be written as the direct sum of irreducible representations. To understand the semisimple representations of a group, it suffices to understand the irreducible representations. For finite groups, every representation is semisimple, so we need only ever classify the irreducible representations of a group to understand every representation of the group.

To prove that all finite dimensional representations of a finite group decompose into irreducible representations, it suffices to prove that if $V$ is any representation with a nontrivial subrepresentation $W$, then there is another subrepresentation $U$ with $V = W \oplus U$. Since $V$ is finite dimensional, and $W$ and $U$ both have dimension less than $V$, we can recursively break down $W$ and $U$ into irreducible submodules.

We shall give two proofs, one which works only for representations over the complex numbers, and the second which works over more general fields. The complex proof is still of interest, despite its lack of generality, because it extends to give results in the representation theory of locally compact groups over a Hilbert space.

\begin{theorem}
    If $V$ is a finite representation of a finite group $G$ over the complex numbers with a subrepresentation $W$, then there is another subrepresentation $U$ with $V = W \oplus U$.
\end{theorem}
\begin{proof}
    On any complex vector space $V$, we may introduce a positive definite hermitian product $H_0: V \times V \to \mathbf{C}$. It is unlikely that $H_0$ is $G$-invariant, in the sense that $H_0(gv,gw) = H_0(v,w)$, but we can always use $H_0$ to construct a $G$-invariant inner product by defining
    %
    \[ H(v,w) = \sum_{g \in G} H_0(gv,gw) \]
    %
    If $W$ is a subrepresentation of $V$, then $U = W^\perp$ is a subrepresentation of $V$, because if $H(v,w) = 0$ for all $v \in W$, then $H(v,gw) = H(g^{-1}v, w) = 0$. Using the standard decomposition results for vector spaces, we conclude that $V = W \oplus U$, and since $W$ and $U$ are $G$-invariant, this is also a decomposition of representations.
\end{proof}

In the case of an irreducible representation, the hermitian product we constructed can be shown to be unique.

\begin{corollary}
    All representations of a finite group over complex vector spaces are completely reducible.
\end{corollary}

On an infinite group, we cannot use the averaging process to construct a $G$-invariant product from a standard inner product on the space, because there is no canonical way to interpret infinte sums. However, if the infinite group has a compact topological structure, then we find there is a finite left translation invariant measure $\mu$ on the group, called the Haar measure. Provided $V$ is an inner product space with inner product $H_0$, we can define a new inner product on $V$ by
%
\[ H(x,y) = \int H_0(gx, gy) d\mu \]
%
Given certain restrictions on the representation, we can use the lemma above to obtain that every finite dimensional representation is semisimple, and similar methods can obtain stronger results about infinite dimensional representations.

\begin{example}
    $\mathbf{R}$ is an abelian topological group, and has a representation over $\mathbf{R}^2$ under matrix the association
    %
    \[ t \mapsto \begin{pmatrix} 1 & t \\ 0 & 1 \end{pmatrix}\ \ \ \ t(x,y) = (x + ty,y) \]
    %
    This representation fixes the $x$-axis, but there is no complementary representation, because such a representation would have to contain some $(x,y)$ for $y \neq 0$, hence it would also contain $(x + y, y)$, and $(x + y, y)$ and $(x,y)$ are linearly independent, hence generate $\mathbf{R}^2$.
\end{example}

\begin{theorem}
    Over a field of characteristic $p$, where $p$ does not divide the order of a finite group $G$, all finite representations of $G$ are completely reducible.
\end{theorem}
\begin{proof}
    Given a representation $V$, let $W$ denote some subrepresentation, and let $\pi_0: V \to W$ be an arbitrary projection of $V$ onto $W$. We can then make $\pi_0$ into a $G$-linear projection by defining a new projection $\pi(v) = \sum_g g \pi_0(g^{-1}v)$. Then the kernel of $\pi$ is a subrepresentation $U$ of $V$, and $\pi$ acts as multiplication by $|G|$ on $W$ (since $|G|$ does not divide the characteristic of the field, this is nonzero), hence $\pi(V) = W$, and we can write $V = W \oplus U$.
\end{proof}

There are finite representations of finite groups over finite fields which are not completely reducible. This is what makes the study of {\bf modular representations}, representations over fields with positive characteristic, very difficult. In these notes we will almost always assume we are working over a field of characteristic zero, in particular the complex numbers.

The decomposition of a representation into irreducible subrepresentations is essentially unique. This is very similar to the theory of the eigenspaces of a linear operator on a finite dimensional vector space, in which case the {\it irreducible} invariant subspaces are uniquely characterized. To prove this, we require a theorem which is very useful in all aspects of representation theory.

\begin{theorem}[Schur]
    If $V$ and $W$ are irreducible representations, and $T: V \to W$ is a $G$-linear map, then $T$ is an isomorphism or $T = 0$.
\end{theorem}
\begin{proof}
    Since the kernel of $T$ is a subrepresentation of $V$, either the kernel is trivial, in which case $T$ is injective, or the kernel is everything, in which case $T = 0$. Similarily, the image of $T$ is a subrepresentation of $V$, hence the image is trivial, in which case $T = 0$, or $T$ is surjective.
\end{proof}

\begin{corollary}
    Every nonzero $G$-endomorphism $T: V \to V$ on an algebraically closed field is given by multiplication by a scalar.
\end{corollary}
\begin{proof}
    Since $V$ is over an algebraically closed field, $T$ has an eigenvector $x$. We know $T$ is an isomorphism, or $T = 0$, so if we assume $T \neq 0$ we know that $Tv = \lambda v$ for $\lambda \neq 0$, and $v \neq 0$. Then $T - \lambda$ is a $G$-endomorphism on $V$ with $(T - \lambda)(v) = 0$, hence $T - \lambda = 0$, so $T = \lambda$.
\end{proof}

As another consequence of Schur's lemma, we note that the $G$-invariant hermitian product we constructed in the proof of complete decomposibility is unique up to a positive scalar.

\begin{corollary}
    If $V$ is irreducible, there is a unique $G$-invariant hermitian product $H$, up to a scalar multiple.
\end{corollary}
\begin{proof}
    If $B: V \times V \to \mathbf{C}$ is a nondegenerate bilinear form, then the map $T_B: v \mapsto v^*_B$ from $V$ to $V^*$ defined by $v^*_B(w) = B(v,w)$ is a linear isomorphism. If $B$ is $G$-invariant, then $T_B$ is $G$-linear, because
    %
    \[ (gv^*)(w) = v^*(g^{-1}w) = B(v,g^{-1}w) = B(gv,w) = (gv)^*(w) \]
    %
    If $B'$ is any other nondegenerate $G$-invariant bilinear form, then $T_{B'}^{-1} \circ T_B$ is a $G$-linear endomorphism, and hence there is $\lambda \in \mathbf{C}$ such that $(T_{B'}^{-1} \circ T_B)(v) = \lambda v$, or $v^*_B = \lambda v^*_{B'}$, so that $B(v,w) = \lambda B'(v,w)$ for all $v,w$.
\end{proof}

Now suppose a representation $V$ has a decomposition into two families of $W_1, \dots, W_n$ and $U_1, \dots, U_n$ of irreducible representations. If, for each $U_i$, we consider the $G$-linear projections $\pi_i: V \to U_i$, and for each $W_j$ we consider the $G$-linear embeddings $i_j: W_j \to V$, then $\pi_i \circ i_j: W_j \to U_i$ is a $G$-linear map, so either $\pi_i \circ i_j = 0$, or $\pi_i \circ i_j$ is an isomorphism. By dimension counting, we know that for each $j$, there is one and only one $i$ such that $\pi_i \circ i_j$ is an isomorphism, and so elements $W_j$ and $U_i$ can be matched up, and we find $W_j = U_i$.

Thus given an arbitrary representation, we can understand it by finding the irreducible subrepresentations it contains, and provided we have a good understanding of the irreducible representations, we can put them back together to obtain results about the original representation. Thus our journery into the representations of an arbitrary group will always begin with classifying the irreducible representations.

\begin{example}
    Let $G$ be a finite {\it abelian} group. If $\rho: G \to GL(V)$ is a representation, then for each $g \in G$, $\rho(g): V \to V$ is actually a $G$-linear map, because
    %
    \[ \rho(g)(hv) = (\rho(g) \circ \rho(h))(v) = \rho(gh)(v) = \rho(hg)(v) = h(\rho(g)(v)) \]
    %
    Hence if $V$ is irreducible, $\rho(g)$ is always given by scalar multiplication. In particular, we see that any subspace of $V$ is a subrepresentation, hence the only finite representations of a finite abelian group are one-dimensional. If $K$ is a field, the representations are then given as {\bf characters} $\rho: G \to K^*$, and classifying the characters classifies the irreducible representations.
\end{example}

\begin{example}
    Consider the group $S_3$, which has 6 elements. The permutation representation gives a representation of $S_3$ on $K^3$, but it is not irreducible. Indeed, $K^3$ has an $S_3$ invariant subspace
    %
    \[ A = \{ (x,y,z): x = y = z \} \]
    %
    This invariant subspace is essentially trivial, but for a field of characteristic not equal to 3, the complementary subspace
    %
    \[ B = \{ (x,y,z) : x + y + z = 0 \} \]
    %
    is generated by $(+1,+0,-1)$ and $(+1,-1,+0)$, which are linearly independent. This subspace is a two dimensional irreducible subspace, because if $v \in B$ is any non-zero vector, then there is a permutation $\pi \in S_3$ such that $\pi(v)$ is independent of $v$, and hence the invariant subspace generated by $v$ is two dimensional, hence equal to $B$ by dimension counting. Thus the permutation representation breaks down into irreducible subspaces $A \oplus B$. $B$ is often called the {\bf standard representation} of $S_3$.
\end{example}

\begin{example}
    An example of a nontrivial character on $S_n$ on a field of characteristic not equal to 2 is the signature function $\pi \mapsto \text{sgn}(\pi)$, which maps all swaps $(i j)$ to $-1$, and in general measures the parity of the number of swaps required to generate a permutation.
\end{example}

\begin{example}
    If $\mu_1, \dots, \mu_n$ are the $n$th roots of unity of degree $n$ in $\mathbf{C}^*$, then all the characters of $\mathbf{Z}_n$ can be written as $f_i(k) = \mu_i^k$, for $1 \leq i \leq n$. To see this, note that if $f: \mathbf{Z}_n \to \mathbf{C}^*$ is any homomorphism, then for any $k \in \mathbf{Z}_n$,
    %
    \[ 1 = f(0) = f(nk) = f(k)^n \]
    %
    Thus $f$ is really a map into the $n$th roots of unity $\mu_n$. Since $\mathbf{Z}_n$ is generated by $1$, all homomorphisms $f$ are uniquely specified by $f(1)$, and the homomorphisms $f_i$ give all possible values for $f(1)$, since $f(1)$ must be an $n$th root of unity.
\end{example}

\begin{example}
    There is an embedding $i: \mathbf{Z}_3 \to S_3$ given by
    %
    \[ 0 \mapsto e\ \ \ \ \ 1 \mapsto (1 2 3)\ \ \ \ \ 2 \mapsto (1 3 2) \]
    %
    Given any complex representation of $S_3$ on a vector space $V$, we may use the embedding $i$ to obtain an induced representation of $\mathbf{Z}_3$ on $V$, and because we know the characters of $\mathbf{Z}_3$, we know there exists a basis $e_1, \dots, e_n$ for $V$, and 3rd roots of unity $\mu_1, \dots, \mu_n$ such that $(1 2 3)e_i = \mu_i e_i$. Since
    %
    \[ (123)(12)(123) = (12) \]
    %
    we find $\mu_i (123)[(12)e_i] = (1 2) e_i$, hence $(1 2) e_i$ is an eigenvector for $(1 23)$ with eigenvalue $\mu_i^2$. What's more, $(1 2)[(1 2) e_i] = e_i$, and so $\{ e_i, (1 2) e_i \}$ span a subrepresentation $W$ of $V$, since the vector space is invariant under the action of $(1 2)$ and $(1 2 3)$, and thus of all of $S_3$ since $(12)$ and $(123)$ generate the group. If $\mu_i \neq 1$, then $(1 2) e_i$ is linearly independent to $e_i$, and so this subrepresentation is two dimensional. All subrepresentations of this form are isomorphic to the standard representation of $S_3$, which has a basis $\{ (\mu,\mu^2,1), (1,\mu^2,\mu) \}$, if $\mu \neq 1$ is a 3rd root of unity. If $\mu_i = 1$, then $(1 2)e_i$ could be a scalar multiple of $e_i$, equal to $\lambda e_i$ for some $\lambda$, but then $e_i = (1 2)(1 2)e_i = \lambda^2e_i$, so $\lambda = \pm 1$, and these are just the trivial and alternating representations of $S_3$. If $\mu_i = 1$, but $(1 2)x_i$ is independent of $x_i$, then $(12)x_i + x_i$ and $(12)x_i - x_i$ are complementary irreducible subrepresentations of $W$, one corresponding to the trivial representation, the other to the alternating representation. Thus the only irreducible representations of $S_3$ are the trivial, alternating, and standard representations.
\end{example}

Thus we have classified all complex representations of $S_3$. What's more, we have classified these representations in such a way that we can break down an arbitrary representation of $S_3$ on $V$ into subrepresentations. We know that we can write
%
\[ V = A^{\oplus n} \oplus B^{\oplus m} \oplus C^{\oplus l} \]
%
where $A$ is the alternating representation, $B$ is the trivial representation, and $C$ is the standard representation. We see that the eigenspace for $(123)$ with eigenvalue $\mu$ and $\mu^2$ is $l$ dimensional, and the eigenspace with eigenvalue $1$ is $n + m$ dimensional. The eigenspace for $(12)$ with eigenvalue $1$ will be $m + l$ dimensional, and the eigenspace for $(12)$ with eigenvalue $-1$ will be $n + l$ dimensional. If we calculate the dimension of these eigenspaces, we can reconstruct $n,m$, and $l$ uniquely.

There are canonical ways to construct new representations out of previously constructed representations. We have seen that if $V$ and $W$ are representations, then so is $V \oplus W$. The tensor product $V \otimes W$ is also canonically a representation, if we define $g(v \otimes w) = (gv) \otimes (gw)$. The tensor product is associative, in the sense that $V \otimes (W \otimes U)$ is $G$-isomorphic to $(V \otimes W) \otimes U$, so it is customary to identify both spaces, and to denote an arbitrary monomial as $v \otimes w \otimes u$. On $V \otimes V$, the set of vectors spanned by elements of the form $v \otimes w + v \otimes w$, and the set spanned by $v \otimes w - w \otimes v$ form subrepresentations, and by taking the quotient, we obtain natural representations of $\text{Sym}^2(V)$ and $V \bigwedge V$. The dual space $V^*$ can also be given a canonical representation by $(gf)(x) = f(g^{-1}x)$. The justification for this definition is given in that it preserves the bilinear form $V \times V^* \to \mathbf{C}$ given by $\langle f, x \rangle = f(x)$. The canonical isomorphism between $V^* \otimes W$ and $\text{Hom}(V,W)$ then gives a representation of the vector space homomorphisms $\text{Hom}(V,W)$ given by $(gT)(x) = g[T(g^{-1}x)]$. The dual space $V^* = \text{Hom}(V,\mathbf{C})$ can be viewed as a special case of this fact. For now, we will use this constructions as tests of the complete reducibility theorem, providing nonirreducible representations from composing reducible representation together.

\begin{example}
    Consider the representation of $S_3$ on $V \otimes V$, where $V$ is the standard representation of $S_3$. If we take a basis $\{ x, y \}$ for $V$ with $(1 2 3)x = \mu x$, $(1 2 3)y = \mu^2 y$, $(1 2) x = y$, $(1 2) y = x$, then
    %
    \[ (1 2 3)(x \otimes y) = (x \otimes y)\ \ \ (1 2 3)(x \otimes x) = \mu^2 (x \otimes x) \]
    \[ (1 2 3)(y \otimes x) = (y \otimes x)\ \ \ (1 2 3)(y \otimes y) = \mu (y \otimes y) \]
    %
    Thus $l = 1$, and $n + m = 2$. Now we calculate
    %
    \[ (1 2)(x \otimes x) = (y \otimes y)\ \ \ (1 2)(x \otimes y) = (y \otimes x) \]
    \[ (1 2)(y \otimes x) = (x \otimes y)\ \ \ (1 2)(y \otimes y) = (x \otimes x) \]
    %
    So the linearly independent eigenvectors of $(1 2)$ are $x \otimes x + y \otimes y$, $x \otimes x - y \otimes y$, $y \otimes x + x \otimes y$, and $y \otimes x - x \otimes y$, and we find $m + l = n + l = 2$, hence $m = n = 1$, so $V \otimes V$ is $S_3$ isomorphic to $A \oplus B \oplus C$. Similarily, $\text{Sym}^2(V)$ has $l = 1$, $n + m = 1$, and $m + l = 1$, $n + l = 2$, so $\text{Sym}^2(V)$ is $S_3$ isomorphic to $B \oplus C$.
\end{example}

\begin{example}
    On $\text{Sym}^3(V)$,
    %
    \[ (1 2 3)x^3 = x^3\ \ \ (1 2 3)x^2y = \mu x^2y\ \ \ (1 2 3)xy^2 = \mu^2 xy^2\ \ \ (1 2 3)y^3 = y^3 \]
    %
    \[ (1 2)x^3 = y^3\ \ \ (1 2)x^2y = xy^2\ \ \ (1 2)xy^2 = x^2y\ \ \ (1 2)y^3 = x^3 \]
    %
    so $l = 1$, $m + n = 2$, and $(1 2)$ has eigenvectors $x^3 + y^3$, $x^3 - y^3$, $x^2y + xy^2$, and $xy^2 - x^2y$, so $m + l = n + l = 2$, hence $m = n = l = 1$, so $\text{Sym}^3(V)$ is isomorphic to $V \otimes V$ as an $S_3$ representation.
\end{example}

\section{Characters}

If $G$ is a commutative group, then all irreducible representations are one dimensional, and are classified by the characters $G \to \mathbf{C}^*$. This essentially follows because any commuting family of semisimple linear transformations on a vector space are simultaneously diagonalizable -- that is, there is a basis such that each element of $G$ is a diagonal matrix. Since $(g + h)x = g(hx)$, in the terms of this basis, the elements of the diagonal with respect to this basis give the characters involved in the decomposition. The relations between the elements of $G$ give constraints on the eigenvalues of the resulting family of transformations on the vector space. For instance, on $\mathbf{Z}_3$, this manifests in the eigenvalues of the elements of $\mathbf{Z}_3$ being 3rd roots of unity. In general, if $G$ is a noncommutative group, then a representation of $G$ need not be simultaneously diagonalizable, and this gives us the irreducible representations of $G$ with dimension greater than one. However, determining the eigenvalues of each element of the group with respect to this representation is still interesting, because it gives relations between the elements of $G$.

For finite groups of high enough order, determining the individual eigenvalues $\lambda_1, \dots, \lambda_n$ of each element of $G$ is an unwieldly task. However, if we know the sums $\sum \lambda_i$ of the eigenvalues for each element $g$, because then we know $\sum \lambda_i^k$, which is the sum of the eigenvalues for $g^k$, and these uniquely specify the coefficients of the characteristic polynomial of $g$, and hence the eigenvalues $\lambda_i$ up to permutation. Given an arbitary representation $\rho: G \to GL(V)$, we define the {\bf character} $\chi_V: G \to \mathbf{C}$ of the representation to be the map $\chi_V(g) = \text{tr}(\rho(g))$. For any $g,h \in G$, $\chi_V(hgh^{-1}) = \chi_V(g)$, so $\chi_V$ is defined on the conjugacy classes of elements of $G$.

\begin{example}
    If $V$ is a permutation representation of some group $G$ with respect to a homomorphism $\varphi: G \to S_n$, then $\chi_V(g)$ is the number of elements of $\{ 1, \dots, n \}$ fixed by $g$, because the matrix of $g$ with respect to the canonical basis on $V$ is a permutation matrix, and the number of $1$s on the diagonal correspond to the number of fixed points.
\end{example}

\begin{theorem}
    If $V$ and $W$ are representations, then
    %
    \[ \chi_{V \oplus W} = \chi_V + \chi_W\ \ \ \ \ \chi_{V \otimes W} = \chi_V \chi_W \]
    %
    \[ \chi_{V/W} = \chi_V - \chi_W \]
    %
    \[ \chi_{V \bigwedge V}(g) = \frac{\chi_V(g)^2 - \chi_V(g^2)}{2}\ \ \ \ \ \chi_{\text{Sym}^2(V)} = \frac{\chi_V(g)^2 + \chi_V(g^2)}{2} \]
    %
    If every element of $G$ has finite order, then $\chi_{V^*} = \overline{\chi_V}$.
\end{theorem}
\begin{proof}
    It is easy to calculate that
    %
    \[ \chi_{V \oplus W} = \chi_V + \chi_W\ \ \ \ \ \chi_{V \otimes W} = \chi_V \chi_W \]
    %
    The equation for $V \oplus W$ is trivial. If $V$ has a basis $\{ e_1, \dots, e_n \}$, and $W$ has a basis $\{ f_1, \dots, f_n \}$, such that $ge_i = \sum a_i^j e_j$, and $gf_i = \sum b_i^j f_j$, then we find $g(e_i \otimes f_j) = \sum a_i^k b_j^l (e_k \otimes f_l)$, and hence the trace with respect to $g$ is
    %
    \[ \sum a_i^i b_j^j = \chi_V(g) \chi_W(g) \]
    %
    For the case of the quotient, if we take some basis $\{ e_1, \dots, e_n \}$ for $W$, and extend it to a basis by some new elements $\{ f_1, \dots, f_m \}$ of $V$, then $f_i + W$ is a basis of $V/W$, and if we write
    %
    \[ ge_i = \sum a_i^j e_j + \sum b_i^j f_j \]
    \[ gf_i = \sum c_i^j e_j + \sum d_i^j f_j \]
    %
    Then $\chi_V(g) = \sum a_i^i + \sum d_i^i = \chi_W(g) + \sum d_i^i$, and since
    %
    \[ g(f_i + W) = \sum d_i^j (f_j + W) \]
    %
    We obtain that $\chi_{V/W}(g) = \sum d_i^i$, and so the formula holds. If every element of $G$ has finite order, then
    %
    \[ \chi_{V^*}(g) = \overline{\chi_V(g)} \]
    %
    because if $g$ has eigenvalues $\lambda_1, \dots, \lambda_n$, then $g^{-1}$ has eigenvalues $\lambda_1^{-1}, \dots, \lambda_n^{-1}$, and if each $\lambda_i \in \mathbf{T}$, then we see that $\sum \lambda_i^{-1} = \overline{\sum \lambda_i}$. We also find
    %
    \[ \chi_{V \wedge V}(g) = \chi_V(g)^2 - \chi_V(g^2) \]
    %
    because if we assume the basis $v_1, \dots, v_n$ puts $g$ into upper triangular form, with diagonal elements $\lambda_1, \dots, \lambda_n$, then $g^2$ is also upper triangular with diagonal elements $\lambda_1^2, \dots, \lambda_n^2$, and we find $g(v_i) = \sum_{i \leq j} a_i^j v_j$, hence
    %
    \[ g(v_i \wedge v_j) = \sum_{i \leq j} a_i^k a_j^l (v_k \wedge v_l) = (a_i^i a_j^j - a_i^j a_j^i) (v_i \wedge v_j) +\ \dots \]
    %
    Since $i \neq j$, then $a_i^j a_j^i = 0$, hence the trace of $g$ on $V \bigwedge V$ is
    %
    \[ \chi_{V \bigwedge V}(g) = \sum_{i < j} a_i^i a_j^j = \sum_{i < j} \lambda_i \lambda_j = \frac{\left( \sum \lambda_i \right)^2 - \sum (\lambda_i)^2}{2} = \frac{\chi_V(g)^2 - \chi_V(g^2)}{2} \]
    %
    Similarily, we find that
    %
    \[ \chi_{\text{Sym}^2(V)}(g) = \frac{\chi_V(g^2) + \chi_V(g)^2}{2} \]
    %
    because
    %
    \[ g(v_iv_j) = \sum_{k < l} (a_i^ka_j^l + a_j^ka_i^l) (v_kv_l) + \sum_k 2a_i^ka_j^k v_k^2 \]
    %
    hence the trace is
    %
    \[ \sum_{i < j} a_i^ia_j^j + 2 \sum_i (a_i^i)^2 = \frac{\left( \sum_i \lambda_i \right)^2 + \sum_i (\lambda_i^2)}{2} = \frac{\chi_V(g)^2 + \chi_V(g^2)}{2} \]
    %
    and this describes all required formulae.
\end{proof}

Given a group $G$, let $\text{Rep}_G$ denote the objects in the category of representations of $G$. We can really think of the character of representations as induced from a map $\text{Rep}_G \times G \to \mathbf{C}$ given by $(V,g) \mapsto \chi_V(g)$. The map is defined on isomorphism classes of representations, and conjugacy classes of the group. Provided the class of isomorphism classes of representations is finite, and the conjugacy classes of a group are finite, we can list these values in a table, and we call this the {\bf character table} of the group.

\begin{example}
    On $S_3$, the conjugacy classes of group elements are
    %
    \[ \alpha = \{ (123), (132) \}\ \ \ \beta = \{ (12), (23), (13) \}\ \ \ \kappa = \{ e \} \]
    %
    and the irreducible representations are the alternating representation $A$, the trivial representation $B$, and the standard representation $C$. The character table is
    %
    \begin{center}
    \begin{tabular}{ | c | c | c | c | }
    \hline
          & $\alpha$ & $\beta$ & $\kappa$\\
    \hline
        $A$ & 1     & -1 & 1\\
        $B$ & 1     &  1 & 1\\
        $C$ & -1     &  0 & 2\\
        \hline
    \end{tabular}
    \end{center}
    %
    If $V$ is an arbitrary representation of $S_3$, we can write $V = A^{\oplus n} \oplus B^{\oplus m} \oplus C^{\oplus l}$, and so $\chi_V = n \chi_A + m \chi_B + l \chi_C$. As elements of $\mathbf{C}^{\{ \alpha, \beta, \kappa \}}$, the characters of the irreducible representations are independent, so finding the decomposition of the character $\chi_V$ suffices to find the coefficients of the decomposition into irreducible representations. Since $\chi_{C \otimes C} = \chi_A^2 = \chi_A + \chi_B + \chi_C$, we see that $C \otimes C$ is isomorphic to $A \oplus B \oplus C$, which we already saw, but this gives a very easy method to verify this. But now we can very easily verify that
    %
    \[ \chi_{C^{\otimes n}} = \left( \frac{2^n + (-1)^n}{3} \right) \chi_C + \left( \frac{2^{n-1} + (-1)^{n+1}}{3} \right) \left( \chi_A + \chi_B \right) \]
    %
    so we have found the general decomposition of the tensor product.
\end{example}

Any group $G$ has a trivial one dimensional representation $V$, and this defines a character $\chi(g) = |G|$. Given an arbitrary representation, we might ask how many copies of the trivial representation it contains. It turns out that there is an explicit formula using characters to find this quantity. To start with, for any representation $V$, define $V^G = \{ v \in V : Gv = v \}$. The dimension of $V^G$ is exactly the number of trivial representations in the decomposition of $V$. Note that while the map $v \mapsto gv$ does not normally form a $G$-linear map, the average
%
\[ f: v \mapsto \frac{1}{|G|} \sum gv \]
%
is always a $G$-linear map, and is a projection of $V$ onto $V^G$. The trace of this map will be exactly the dimension of $V^G$, and therefore the number of trivial representations contained in the decomposition of $V$, and so
%
\[ \dim(V^G) = \text{tr}(f) = \frac{1}{|G|} \sum \text{tr}(g) = \frac{1}{|G|} \sum \chi_V(g) \]
%
In particular, if $V$ is nontrivial and irreducible, then $\sum \chi_V(g) = 0$.

This idea extends to give powerful results about irreducible representations of finite groups. Note that $\text{Hom}(V,W)^G$ is the set consisting of all $G$-linear maps from $V$ to $W$, so if $V$ is irreducible then $\dim \text{Hom}(V,W)^G$ gives the number of copies of $V$ in $W$, and in particular if $W$ is irreducible then $\dim \text{Hom}(V,W)^G = 1$ if $V$ is isomorphic to $W$, and $\dim \text{Hom}(V,W)^G = 0$ otherwise. But we now find that
%
\[ \chi_{\text{Hom}(V,W)} = \chi_{V^* \otimes W} = \overline{\chi_V} \chi_W \]
%
Using the character formula we derived in the last paragraph, we find
%
\[ \sum \overline{\chi_V}(g) \chi_W(g) = \delta_{VW} \]
%
so the family of characters of irreducible representations are orthonormal to one another with respect to the canonical inner product on $\mathbf{C}^G$, given by
%
\[ \langle f, h \rangle = \frac{1}{|G|} \sum f(g) h(g) \]
%
In particular this shows that the characters $\chi_V$ are linearly independent, and form a subspace of the set of class functions on $G$ (functions defined on conjugacy classes of $G$). Already, we derive an interesting result, that the number of irreducible representations of $G$ is bounded by the number of conjugacy classes of elements of $G$.

\begin{theorem}
    For any function $f: G \to \mathbf{C}$ on a finite group, and for any representation $V$, define $T_f = \sum f(g) \cdot g$ to be a linear function on $V$. If $f$ is a class function, then $T_f$ is $G$-linear, and if $T_f$ is $G$-linear for all representations $V$, then the converse holds.
\end{theorem}
\begin{proof}
    If $f$ is a class function, we calculate
    %
    \[ T_f(gv) = \sum f(h) hgv = \sum f(ghg^{-1}) ghg^{-1}gv = g \sum f(ghg^{-1}) hv = g T_f(v) \]
    %
    For the converse, consider the left regular representation of $G$, in which if $T_f$ is a $G$-linear map, we must have
    %
    \begin{align*}
        \sum f(k) a_h gkh &= g \left( T_f \left( \sum a_h h \right) \right) = T_f \left( g \sum a_h h \right)\\
        &= \sum f(k) a_h kgh = \sum f(gkg^{-1}) a_h gkh
    \end{align*}
    %
    hence $f(k) = f(gkg^{-1})$ for all $g, k \in G$.
\end{proof}

\begin{theorem}
    The set of $\chi_V$, where $V$ is an irreducible representation of $G$, form an orthonormal basis for the set of class functions on a finite group $G$.
\end{theorem}
\begin{proof}
    Suppose that $\langle \chi_V, f \rangle = 0$, for all irreducible representations $V$. Then $T_f: V \to V$ is given by multiplication by some scalar $\lambda$, and on $V$, we find
    %
    \[ \dim(V) \lambda = \text{tr}(T_f) = \sum f(g) \chi_V(g) = \langle f, \chi_V \rangle = 0 \]
    %
    hence $\lambda = 0$, so $\sum f(g) gv = 0$ for all $v$. It then follows that $T_f = 0$ for all representations of $V$ become the representation is completely decomposible. In particular, if we take $V$ to be the left regular representation of $G$, we conclude that $f(g) = 0$ for all $g$, hence $f = 0$.
\end{proof}

\begin{corollary}
    If $\chi_V = \chi_W$, then $V$ is isomorphic to $W$.
\end{corollary}
\begin{proof}
    If $V$ can be written as $\bigoplus V_i^{\oplus n_i}$, where $V_i$ are all irreducible representations, then $\chi_V = \sum n_i \chi_{V_i}$, and if $W$ is $\bigoplus V_i^{\oplus m_i}$, then $\chi_W = \sum m_i \chi_{V_i}$, hence if $\chi_V = \chi_W$, then $n_i = m_i$ for all $i$, and so $V$ is isomorphic to $W$.
\end{proof}

\begin{corollary}
    Given a representation $V$, the multiplicity of an irreducible representation $V_i$ in the decomposition of $V$ is $\langle \chi_V, \chi_{V_i} \rangle$. The number of irreducible representations of a finite group $G$ is the number of conjugacy classes. The representation $V$ is irreducible if and only if $\| \chi_V \|_2 = 1$.
\end{corollary}

\begin{example}
    Let $\mathbf{C}[G]$ be the regular representation of $G$. Then $\chi_{\mathbf{C}[G]}(g) = 0$ if $g \neq e$, and $\chi_{\mathbf{C}[G]}(e) = |G|$. We conclude that if $G \neq \{ e \}$, then $\mathbf{C}[G]$ is not irreducible, and if we write
    %
    \[ \mathbf{C}[G] = \bigoplus V_i^{\oplus n_i} \]
    %
    then
    %
    \[ n_i = \langle \chi_{V_i}, \chi_{\mathbf{C}[G]} \rangle = \text{dim}(V_i) \]
    %
    so every irreducible representation $V$ occurs in $\mathbf{C}[G]$ $\dim V$ times. In particular, we find that
    %
    \[ |G| = \dim(\mathbf{C}[G]) = \chi_{\mathbf{C}[G]}(e) = \sum (\dim V_i) \chi_{V_i}(e) = \sum (\dim V_i)^2 \]
    %
    and if $g \neq e$, then
    %
    \[ 0 = \chi_{\mathbf{C}[G]}(g) = \sum (\dim V_i) \chi_{V_i}(g) \]
    %
    and this gives a kind of inversion result for the characters.
    %
    \[ \sum \overline{\chi_{V_i}(g)} \chi_{V_i}(g) = \frac{|G|}{c(g)} \]
\end{example}

\begin{example}
    If $G$ is a finite abelian group, then there are $|G|$ conjugacy classes on $G$, and all complex valued functions on $G$ are class functions, hence we may write any $f: G \to \mathbf{C}$ as
    %
    \[ f(g) = \sum a_i \chi_i(g) \]
    %
    for some unique $a_g \in \mathbf{C}$, and where $\chi_i: G \to \mathbf{T}$ are the characters of $G$, which form an orthonormal basis with respect to summation. This constitutes exactly the theory of Fourier analysis on finite abelian groups.
\end{example}

\begin{example}
    On $S_n$, the conjugacy classes of elements in the group correspond to partitions of the integer $n$. That is, a partition of $n$ is a sum of the form $\sum n_k k$, where $n_k$ are some integers. This is because the conjugacy classes of $S_n$ can be classified by the types of cycles some permutation in $S_n$ contains, because an inner automorphisms $\tau \mapsto \sigma \tau \sigma^{-1}$ amounts to `renaming indices'. On $S_4$, we find 5 partitions
    %
    \[ (a)\ 4 = 4\ \ \ \ \ (b)\ 4 = 3 + 1\ \ \ \ \ (c)\ 4 = 2 + 2 \]
    \[ (d)\ 4 = 2 + 1 + 1\ \ \ \ \ (e)\ 4 = 1 + 1 + 1 + 1 \]
    %
    so there are 5 equivalence classes of elements in $S_4$, and thus 5 irreducible representations. The first few are easy to find -- we take the trivial representation $A$, the alternating representation $B$, and the standard representation $C$, which is the quotient of the action of $S_4$ on $\mathbf{C}^4$ by the subrepresentation consisting of elements fixed under $S_4$. These have characters
    %
    \[ \chi_A = (1,1,1,1,1)\ \ \ \ \ \chi_B = (-1,+1,+1,-1,+1) \]
    %
    and since the standard representation is a quotient,
    %
    \[ \chi_C = (0,1,0,2,4) - (1,1,1,1,1) = (-1,+0,-1,+1,+3) \]
    %
    Note that
    %
    \[ \| \chi_C \|_2^2 = \frac{6 \cdot (-1)^2 + 8 \cdot (0)^2 + 3 \cdot (-1)^2 + 6 \cdot (1)^2 + 1 \cdot (3)^2}{24} = 1 \]
    %
    so $C$ is irreducible. If $D$ and $E$ are the two remaining irreducible representations, then
    %
    \[ 1 + 1 + 3^2 + (\dim D)^2 + (\dim E)^2 = 24 \]
    %
    so $(\dim D)^2 + (\dim E)^2 = 13$, and the only integers satisfying this equation are $\dim D = 2$, and $\dim E = 3$. Now
    %
    \[ \chi_{B \otimes C} = (-1,+0,-1,-1,+3) \]
    %
    and we find $\| \chi_{B \otimes C} \|_2^2 = 1$, so $E = B \otimes C$ is an irreducible representation of dimension 3 independent of $A,B$, and $C$. We can determine the character of $D$ uniquely from the orthogonality representations as
    %
    \[ \chi_D = () \]
    %
    so $S_4$ has a character table
    %
    \begin{center}
    \begin{tabular}{|c|c|c|c|c|c|}
    \hline
        & $a$ (6) & $b$ (8) & $c$ (3) & $d$ (6) & $e$ (1) \\
    \hline
    $A$ & 1 & 1 & 1 & 1 & 1 \\
    $B$ & -1 & 1 & 1 & -1 & 1 \\
    $C$ & -1 & 0 & -1 & 1 & 3 \\
    $D$ & 0 & -2 & 4 & 0 & 4 \\
    $E$ & 1 & 0 & -1 & -1 & 3 \\
    \hline
    \end{tabular}
    \end{center}
\end{example}

\end{document}