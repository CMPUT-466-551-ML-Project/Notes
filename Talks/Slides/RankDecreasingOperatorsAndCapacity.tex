\documentclass[usenames,dvipsnames,12pt]{beamer}

\usepackage{tikz}
\usepackage{tkz-berge}
\usepackage{tkz-graph}
\usepackage{subcaption}
\usepackage{blkarray}
\usepackage{aligned-overset}
\usepackage{graphicx}
\usepackage{calc}

\setbeamertemplate{footline}[frame number]

\usetikzlibrary{patterns,arrows,decorations.pathreplacing}

\usepackage{xcolor}
\definecolor{dblue}{RGB}{20,66,129}
\definecolor{rose}{RGB}{255,101,122}
\definecolor{crimsonred}{RGB}{132,22,23}
\definecolor{darkblue}{RGB}{72,61,139}

\definecolor{deepblue}{RGB}{36,123,160}
\definecolor{deepred}{RGB}{255,22,84}
\definecolor{deeporange}{RGB}{240,111,62}

\definecolor{olive}{rgb}{0.3, 0.4, .1}
\definecolor{fore}{RGB}{249,242,215}
\definecolor{back}{RGB}{51,51,51}
\definecolor{title}{RGB}{255,0,90}
\definecolor{dgreen}{rgb}{0.,0.6,0.}
\definecolor{gold}{rgb}{1.,0.84,0.}
\definecolor{JungleGreen}{cmyk}{0.99,0,0.52,0}
\definecolor{BlueGreen}{cmyk}{0.85,0,0.33,0}
\definecolor{RawSienna}{cmyk}{0,0.72,1,0.45}
\definecolor{Magenta}{cmyk}{0,1,0,0}

\DeclareMathOperator{\QQ}{\mathbf{Q}}
\DeclareMathOperator{\ZZ}{\mathbf{Z}}
\DeclareMathOperator{\RR}{\mathbf{R}}
\DeclareMathOperator{\HH}{\mathbf{H}}
\DeclareMathOperator{\CC}{\mathbf{C}}
\DeclareMathOperator{\AB}{\mathbf{A}}
\DeclareMathOperator{\PP}{\mathbf{P}}
\DeclareMathOperator{\MM}{\mathbf{M}}
\DeclareMathOperator{\VV}{\mathbf{V}}
\DeclareMathOperator{\TT}{\mathbf{T}}
\DeclareMathOperator{\LL}{\mathcal{L}}
\DeclareMathOperator{\EE}{\mathbf{E}}
\DeclareMathOperator{\NN}{\mathbf{N}}
\DeclareMathOperator{\DQ}{\mathcal{Q}}
\DeclareMathOperator{\IA}{\mathfrak{a}}
\DeclareMathOperator{\IB}{\mathfrak{b}}
\DeclareMathOperator{\IC}{\mathfrak{c}}
\DeclareMathOperator{\IP}{\mathfrak{p}}
\DeclareMathOperator{\IQ}{\mathfrak{q}}
\DeclareMathOperator{\IM}{\mathfrak{m}}
\DeclareMathOperator{\IN}{\mathfrak{n}}
\DeclareMathOperator{\IK}{\mathfrak{k}}
\DeclareMathOperator{\ord}{\text{ord}}
\DeclareMathOperator{\Ker}{\textsf{Ker}}
\DeclareMathOperator{\Coker}{\textsf{Coker}}
\DeclareMathOperator{\emphcoker}{\emph{coker}}
\DeclareMathOperator{\pp}{\partial}
\DeclareMathOperator{\tr}{\text{tr}}



\title{Algorithmic Aspects of the Brascamp Lieb Inequality}
\author{Jacob Denson}

\institute{University of Wisconsin Madison}

\begin{document}

\maketitle

\begin{frame}

\begin{itemize}
    \item \emph{Classical Complexity and Quantum Entanglement}

    Gurvits, 2004.
    \item \emph{Algorithmic and Optimization Aspects of Brascamp-Lieb Inequalities, Via Operator Scaling}

    Garg, Gurvits, Oliveira, Wigderson, 2016.
    \item \emph{A Deterministic Polynomial Time Algorithm For Non-Commutative Rational Identity Testing}

    Garg, Gurvits, Oliveira, Wigderson, 2016.
\end{itemize}

\end{frame}

\begin{frame}
    \frametitle{Lieb's Theorem}

    \vspace{-0.5em}
    \[ \int_{\RR^n} \prod_{i = 1}^m |f_i(B_i x)|^{p_i}\; dx \leq \text{BL}(B,p) \cdot \prod_{i=1}^m \| f_i \|_{L^1(\RR^{n_i})}^{p_i}.  \]

    \begin{itemize}
        \pause
        % Related to Aswin and Valentina's Talk
        \item (Lieb, 1990) To calculate BL(B,p), plug in Gaussians.

        \pause
        \item If $f_i(x) = e^{- \pi (A_i x) \cdot x}$ for $A_i \succ 0$, then
        \pause
%        \vspace{-1em}
%        \small
        \[ \int_{\RR^n} \prod_{i = 1}^m |f_i(B_i x)|^{p_i}\; dx = \det(\sum p_i B_i^* A_i B_i)^{-1/2} \]
%        \normalsize
        \pause
%        \vspace{-0.8em}
%        \small
        \[ \prod_{i = 1}^m \| f_i \|_{L^1(\RR^{n_i})}^{p_i} = (\prod_{i = 1}^m \det(A_i)^{p_i})^{-1/2}. \]
%        \normalsize

        \pause
        Thus
        %
        \[ \det(\sum p_i B_i^* A_i B_i)^{-1/2} \leq \text{BL}(B,p) \cdot (\prod_{i = 1}^m \det(A_i)^{p_i})^{-1/2} \]

    \end{itemize}
\end{frame}

\begin{frame}
    \frametitle{Lieb's Theorem}

    \begin{itemize}
        % Shows up in Victor's Talk
        \item Rearranging gives
        %
        \[ \text{BL}(B,p) = \left( \sup_{A_1,\dots,A_m \succ 0} \frac{\prod_i \det(A_i)^{p_i}}{\det(\sum p_i B_i^* A_i B_i)} \right)^{1/2}. \]

        \pause
        \item Non convex objective, so tricky to optimize (NP hard).

        \pause
        \item Checking finiteness efficiently also non-obvious.

        \begin{itemize}
            \pause
            \item BCCT gives a family of conditions to check finiteness.
            %
            \[ \dim(V) \leq \sum p_i \dim(B_i V). \]
            
            \pause
            Since $0 \leq \dim(V) \leq n$ and $0 \leq \dim(B_i V) \leq n_i$, only finitely many linear conditions on the exponents $p_i$.
    
            \pause
            \item Can be exponentially many constraints, so inefficient.
        \end{itemize}
    \end{itemize}
\end{frame}

\begin{frame}
    \frametitle{Geometric Brascamp Lieb}

    \vspace{-1.5em}
    \[ \int_{\RR^n} \prod_{i = 1}^m |f_i(B_i x)|^{p_i}\; dx \leq \text{BL}(B,p) \cdot \prod_{i = 1}^m \| f_i \|_{L^1(\RR^{n_i})}^{p_i} \]

    \begin{itemize}
        \pause
        \item A Brascamp-Lieb inequality is \emph{geometric} if
        \begin{itemize}
            \item (Projection Property): $B_i B_i^* = I$.

            \item (Isotropy Property): $\sum_i p_i B_i^* B_i = I$.
        \end{itemize}

        \pause
        \item (Barthe, 1998), generalizing (Ball, 1989), showed that if $(B,p)$ is geometric, $\text{BL}(B,p) = 1$.
    \end{itemize}
\end{frame}

\begin{frame}
    \frametitle{Operator Rescaling}

    \begin{itemize}
        \item What happens to $\text{BL}(B,p)$ if we `change coordinates'.

        \pause
        \item Fix invertible matrices $M$ and $M_1, \dots, M_m$, and consider the Brascamp-Lieb inequality with matrices $B_i' = M_i B_i M$,
        %
        \[ \int_{\RR^n} \prod_{i = 1}^m |f_i(M_i B_i M x)|^{p_i}\; dx \leq \text{BL}(B',p) \cdot \prod_{i = 1}^m \| f_i \|_{L^1(\RR^{n_i})}^{p_i}. \]
        %
        \pause
        %
        \begin{align*}
            \text{BL}(B',p) &= \left( \sup_{A_1,\dots,A_m \succ 0} \frac{\prod_i \det(A_i)^{p_i}}{\det(\sum_i p_i M^* B_i^* M_i^* A_i M_i B_i M)} \right)^{1/2}\\
%            &= \sup_{A_1,\dots,A_m \succ 0} \frac{\prod_i \det((M_i^{-1})^* A_i M_i^{-1})^{p_i}}{\det(M^* (\sum_i p_i B_i^* A_i B_i) M)}\\
            &= \det(M)^{-1} \prod_i \det(M_i)^{-p_i} \cdot \text{BL}(B,p). 
        \end{align*}
    \end{itemize}
\end{frame}

\begin{frame}
    \frametitle{Operator Rescaling}

    \[ \text{BL}(B',p) = \det(M)^{-1} \prod\nolimits_i \det(M_i)^{-p_i} \cdot \text{BL}(B,p) \]

    \begin{itemize}
        \pause
        \item If $(B',p)$ is geometric, $\text{BL}(B',p) = 1$, so
        %
        \[ \text{BL}(B,p) = \det(M) \prod\nolimits_i \det(M_i)^{p_i}. \]

        \pause
        \item (BCCT) Geometric rescaling possible iff extremizers exist.
    \end{itemize}
\end{frame}

\begin{frame}
    \frametitle{Main Result}

    \begin{itemize}
        \item (Garg, Gurvits, Oliveira, Wigderson, 2016) If $\text{BL}(B,p) < \infty$, we can rescale to $(B',p)$ with $\text{BL}(B',p) \leq 1 + \varepsilon$, for any $\varepsilon > 0$.
        \begin{itemize}
            \pause
            \item For $\varepsilon \ll 1$, $\text{BL}(B,p) \approx \det(M)^2 \prod\nolimits_i \det(M_i)^{2p_i}$.

            \pause
            \item We can do this algorithmically, i.e. if $p_i$ are rationals, with common denominator $d$, a computer can compute a $\varepsilon$-approximate geometric rescaling in $\text{Poly}(\text{Bits}(B,p),d,1/\varepsilon)$ computations. 

            \pause
            \item
            Conversely, we can determine if $\text{BL}(B,p) = \infty$, and find a violating subspace $V$ to the condition $\dim(V) \leq \sum p_i \dim(B_i V)$ in this case, in $\text{Poly}(\text{Bits}(B,p),d)$ computations.

            \pause
            \item Open Problem: Can we can improve this to $\text{Poly}(\text{Bits}(B,p),d,\log(1/\varepsilon))$ computations?
        \end{itemize}
    \end{itemize}
\end{frame}

\begin{frame}
    \frametitle{Analytic Consequences}

    \begin{itemize}
        \pause
        \item Typing a single digit is a single unit of computation.

        \pause
        \begin{itemize}
            \item Thus given $(B,p)$, either $\text{BL}(B,p) < \infty$, or
            %
           \[ \text{BL}(B,p) \leq 2^{\text{Poly}(\text{Bits}(B,p),d,\log(1/\varepsilon))}. \]
           %
           and
           %
           \[ \text{BL}(B,p) \geq 2^{-\text{Poly}(\text{Bits}(B,p),d,\log(1/\varepsilon))} \]
        \end{itemize}

        \pause
        \item Calculation is also stable.

        \pause
        \begin{itemize}
            \item Thus we conclude a \emph{continuity bound}:
            \pause

            \item For each $(B,p)$ with $\text{BL}(B,p) < \infty$, and each $\varepsilon > 0$, \pause if
            %
            \[ \| B_1 - B \| \leq \exp \left( - \frac{\text{Poly}(\text{Bits}(B,p), d)}{\varepsilon^3} \right), \]
            %
            \pause
            then
            %
            \[ (1 - \varepsilon) \text{BL}(B,p) \leq \text{BL}(B_1,p) \leq (1 + \varepsilon) \text{BL}(B,p). \]
        \end{itemize}
    \end{itemize}
\end{frame}

%\begin{frame}
%    \frametitle{The Avoidance Problem}

%        \begin{itemize}

%            \item Our method naturally considers an equivalent setup.

%            \item {\bf Fractal Avoidance Problem}: Given $Y \subset \mathbf{R}^{nd}$, find $X \subset \mathbf{R}^d$ such that $X^n \cap Y \subset \Delta$, where $\Delta = \{ x: x_i = x_j\ \text{for some $i,j$} \}$.

%            \item Equivalent by setting $Y = f^{-1}(0)$, or $f = \mathbf{I}_{Y^c}$.

%            \item Our method only uses the structure of the zero set, not $f$.
%        \end{itemize}
%\end{frame}

\begin{frame}
    \frametitle{Computing Permanents (An Analogous Problem)}

    \begin{itemize}
        \pause
        \item Given an $n \times n$ matrix $A$ with non-negative entries, define
        %
        \[ \text{Perm}(A) = \sum\nolimits_{\sigma \in S_n} \prod\nolimits_i A_{i \sigma(i)}. \]
        
        \pause
        \item This is (surprisingly) NP hard to compute.

        \begin{itemize}
            \pause
            \item If
            %
            \footnotesize
            \[ R = \begin{pmatrix} \lambda_1 & 0 & 0 \\ 0 & \ddots & 0 \\ 0 & 0 & \lambda_n \end{pmatrix} \quad\text{and}\quad C = \begin{pmatrix} \gamma_1 & 0 & 0 \\ 0 & \ddots & 0 \\ 0 & 0 & \gamma_n \end{pmatrix}, \]
            \normalsize
            %
            then
            %
            \begin{align*}
                \text{Perm}(RAC) &= \text{Perm}(\lambda_i A_{ij} \gamma_j)\\
                &= (\lambda_1 \dots \lambda_n) (\gamma_1 \dots \gamma_n) \text{Perm}(A)\\
                &= \det(R) \det(C) \text{Perm}(A).
            \end{align*}
        \end{itemize}
    \end{itemize}
\end{frame}






\begin{frame}
    \frametitle{Computing Permanents (An Analogous Problem)}

    \[ \text{Perm}(RAC) = \det(R) \det(C) \text{Perm}(A). \]

    \begin{itemize}
        \pause
        \item (Egorychev, 1981), (Falikman, 1981) If $A$ is \emph{doubly stochastic}, $e^{-n} \leq \text{Perm}(A) \leq 1$.

        \pause
        \item If $RAC$ is doubly stochastic, then
        %
        \[ \text{Perm}(A) \leq \det(R)^{-1} \det(C)^{-1} \leq e^n \cdot \text{Perm}(A), \]
        %
        so $\text{Perm}(A) \approx \det(R)^{-1} \det(C)^{-1}$.
    \end{itemize}
\end{frame}





\begin{frame}
    \frametitle{The Algorithm}
\end{frame}




\begin{frame}
    \begin{itemize}
        \item Given a non-negative matrix $A_0$, alternatively apply row and column normalization, obtaining a sequence
        %
        \[ A_0 \to A_1 \to A_2 \to \dots. \]

        \pause
        \item Claim: If $\text{Per}(A_0) > 0$, $d(A_i, \mathbf{S}) \to 0$, where $\mathbf{S}$ is the family of doubly stochastic matrices.

        \pause
        \item For even (odd) $i$, let $\gamma_{ij}$ be the $j$th row (column) sum of $A_i$, so that $\text{Per}(A_{i+1}) = (\gamma_{i1} \dots \gamma_{in})^{-1} \cdot \text{Per}(A_i)$.

        \pause
        \item {\bf Two Key Facts Ensuring Convergence}:
        \begin{itemize}
            \pause
            \item[(1)] $\text{Per}(A) \leq 1$ if $A$ is partially normalized.
            \pause
            \item[(2)] If $\Delta_i = \sum_j (\gamma_{ij} - 1)^2$, $\text{Per}(A_{i+1}) \geq (1 + C \Delta_i) \cdot \text{Per}(A_i)$.
        \end{itemize}

        \pause
        \item Thus $\text{Per}(A_i)$ is bounded, monotonic, converges to $P \leq 1$.

        \pause
        \item If $\text{Per}(A_i) \geq P - \varepsilon$ for $\varepsilon \ll 1$, then
        %
        \[ P \geq \text{Per}(A_{i+1}) \geq (1 + C \cdot \Delta_i) \cdot \text{Per}(A_i) \geq (1 + C \cdot \Delta_i)(P - \varepsilon). \]
        % C (\varepsilon / P) \geq C_2 \delta_i
        Thus $\Delta_i \lesssim \varepsilon$. Taking $\varepsilon \to 0$ shows $\Delta_i \to 0$.
    \end{itemize}
\end{frame}


\begin{frame}
    \frametitle{Sinkhorn Iteration}

    \begin{itemize}
        \item {\bf Proof that $\text{Per}(A_i) \leq 1$:}
        \begin{itemize}
            \pause
            \item Simple inductive argument: the hypothesis is that if each row of a matrix $B$ all sum up to less than one, then $\text{Per}(B) \leq 1$.
        \end{itemize}

        \pause
        \item {\bf Proof that $\text{Per}(A_{i+1}) \geq (1 + C \Delta_i) \cdot \text{Per}(A_i)$:}
        \begin{itemize}
            \pause
            \item Really just more robust form of AGM inequality.

            \pause
            \item If $\gamma_{ij}$ are the row sums, then because the column sums are all one,
            %
            \[ \frac{1}{n} \sum_j \gamma_{ij} = 1. \]

            \pause
            \item AGM implies $\gamma_{i1} \dots \gamma_{in} \geq 1$, and monotonicity follows from
            %
            \[ \text{Per}(A_{i+1}) = (\gamma_{i1} \dots \gamma_{in})^{-1} \text{Per}(A_i). \]

%            \pause
%            \item Write $\gamma_{ij} = 1 + \delta_j$.

%            \begin{itemize}
%                \pause
%                \item Then $\sum \delta_j^2 = \Delta_j$, and $\sum \delta_j = 0$.
%            \end{itemize}

%            \pause
%            \item Since $1 + t \leq \exp(t - t^2/2 + t^3/3)$,

%            \vspace{-2em}
%            \begin{align*}
%                \text{Per}(A_i) / \text{Per}(A_{i+1}) &= \gamma_1 \dots \gamma_n \\
%                &= (1 + \delta_1) \dots (1 + \delta_n)\\
%                &\leq \exp \left(\sum \delta_j - \sum \delta_j^2/2 + \sum \delta_j^3/3 \right)\\
%                &\leq \exp(0 - \Delta_j/2 + \Delta_j^{3/2}/3)\\
%                &= 1 - \Delta_j / 2 + O(\Delta_j^{3/2}).
%            \end{align*}
        \end{itemize}
    \end{itemize}
\end{frame}


\begin{frame}

And now, back to our regularly scheduled programming

\[ \text{BL}(B,p) = \sqrt{ \sup_{A_1,\dots,A_m \succ 0} \frac{\prod_i \det(A_i)^{p_i}}{\det(\sum p_i \cdot B_i^* A_i B_i)}}. \]

\begin{itemize}
    \pause
    \item Goal: Rescale our inputs so that
    %
    \begin{itemize}
        \pause
        \item (Isotropy) $\sum p_i B_i^* B_i = I$.
        
        \pause
        \item (Projection) $B_i B_i^* = I$ for each $i$.
    \end{itemize}

\end{itemize}

\end{frame}


\begin{frame}
    \frametitle{Iteration of Operator Rescaling}

    \begin{itemize}
    \item Sinkhorn: Alternately apply the following two procedures:
    \begin{itemize}
        \pause
        \item (Isotropy Normalization)
        \begin{itemize}
            \pause
            \item Let $M = \sum_i p_i B_i^* B_i$.
            \pause
            \item Replace $B_i$ with $B_i' = B_i M^{-1/2}$.
            \pause
            \item Then $\sum p_i (B_i')^* B_i' = 1$, i.e. isotropy holds.
        \end{itemize}

        \pause
        \item (Projection Normalization)
        \begin{itemize}
            \pause
            \item Let $M_i = B_i B_i^*$.
            
            \pause
            \item Replace $B_i$ with $B_i' = M_i^{-1/2} B_i$.
            
            \pause
            \item Then $(B_i')^* B_i' = I$ for each $i$.
        \end{itemize}

        \pause
        \item We obtain a sequence $B \to B_1 \to B_2 \to \dots$.
    \end{itemize}
\end{itemize}

\end{frame}




\begin{frame}
    \frametitle{Iteration of Operator Rescaling}

    \begin{itemize}
        \item {\bf Two Key Facts Ensuring Convergence:}
        \begin{itemize}
            \pause
            \item[(1)] $\text{BL}(B,p) \geq 1$ if $(B,p)$ is partially normalized.
            \pause
            \item[(2)] For some $r \leq \text{Poly}(\text{Bits}(B),d,n)$, if
            %
            \[ \text{BL}(B_i,p) \geq 1 + \varepsilon, \]
            %
            then
            %
            \[ \text{BL}(B_{i+1},p) \leq (1 - C \varepsilon^r) \text{BL}(B_i,p). \]
        \end{itemize}

        \pause
        \item Thus convergence occurs as with Sinkhorn iteration provided that $\text{BL}(B,p) < \infty$.

        \pause
        \item (1) and (2) follow from techniques in the study of \emph{positive operators}.
    \end{itemize}

\end{frame}


\begin{frame}
    \frametitle{Positive Operators}

    \begin{itemize}
        \item A linear map $T: M_n \to M_n$ is \emph{completely positive} if there are $n \times n$ matrices $B_1,\dots,B_K$ and $p_i > 0$ such that
        %
        \[ T(A) = \sum p_i B_i A B_i^*. \]

        \pause
        \item $T: M_n \to M_n$ is \emph{positive} if $A \succeq 0$ implies $T(A) \succeq 0$.

        \pause
        \item Can reduce the study of non-negative matrices to positive operators: For a non-negative matrix $S$, $T(A)$ is the diagonal matrix whose entries are precisely the vector $Sa$, where $a$ is the vector formed from the diagonal entries of $A$.

        \pause
        \item Given $T$, we have $T^*(A) = \sum p_i B_i^* A B_i$.
    \end{itemize}
\end{frame}




\begin{frame}
    \frametitle{Further Connections}

    \begin{itemize}
        \item For simplicity, look at Brascamp Lieb where all spaces have the same dimension (all $B_i$ are square matrices). 

        \pause
        \item $\text{BL}(B,p) < \infty$ can only hold if $\sum p_i = 1$.

        \pause
        \item Consider optimizing the quantity
        %
        \[ \inf_{A \succ 0} \frac{\det(\sum p_i B_i^* A B_i)}{\det(A)}  \]
        %
        analogous to
        %
        \[ \text{BL}(B,p) = \sup_{A_1,\dots,A_m \succ 0} \sqrt{ \frac{\prod_i \det(A_i)^{p_i}}{\det(\sum p_i \cdot B_i^* A_i B_i)}}, \]
        %
        if all $A_i$ are equal.
    \end{itemize}
\end{frame}



\begin{frame}
    \frametitle{Capacity of Operators}

    \[ \inf_{A \succ 0} \frac{\det(\sum p_i B_i^* A B_i)}{\det(A)}  \]

    \begin{itemize}
        \pause
        \item (Gurvits, 2004) The \emph{capacity} of $T: M_n \to M_n$ is
        %
        \[ \text{Cap}(T) = \inf_{A \succ 0} \frac{\det(TA)}{\det(A)}. \]

        \pause
        \item For any Brascamp-Lieb data $(B,p)$, there exists a positive $T: M_n \to M_n$ and $k$ such that $\text{Cap}(T) = 1/\text{BL}(B,p)^{2k}$.

        \pause
        \item Positive operators are well studied in the quantum information theory literature, so reduction of BL to  this theory is useful.
    \end{itemize}
\end{frame}

\begin{frame}
    \frametitle{Doubly Stochastic Positive Operators}

    \begin{itemize}
        \item (Isotropy) Let $T(A) = \sum p_i B_i^* A B_i$.
        \begin{itemize}
            \pause
            \item $\sum p_i B_i^* B_i = I$ holds iff $T(I) = I$.
        \end{itemize}

        \pause
        \item (Projection) Let $T(A) = B_i^* A B_i$.
        \begin{itemize}
            \pause
            \item $B_i B_i^* = I$ if and only if $T^*(I) = I$.
        \end{itemize}

        \pause
        \item If $(B,p)$ is a Brascamp-Lieb datum with associated operator $T: M_n \to M_n$, then $(B,p)$ is geometric if and only if $T$ is \emph{doubly stochastic}, i.e. $T(I) = I$ and $T^*(I) = I$.
    \end{itemize}
\end{frame}

\begin{frame}
    \frametitle{Operator Rescaling for Positive Operators}

    \begin{itemize}
        \item If $T$ is doubly stochastic, $\text{Cap}(T) = 1$.

        \pause
        \item We can rescale. If
        %
        \[ T_{M_1M_2}(A) = M_2^* T(M_1^* A M_1) M_2, \]
        %
        then $\text{Cap}(T_{M_1,M_2}) = \det(M_1)^2 \det(M_2)^2 \cdot \text{Cap}(T)$.

        \pause
        \item Sinkhorn says to iterate
        %
        \[ T \mapsto T_{I,T(I)^{-1/2}} \quad\text{and}\quad T \mapsto T_{T^*(I)^{-1/2}, I}. \]

        \pause
        \item If $\text{Cap}(T) > 0$, iteration yields a rescaling arbirarily close to a doubly stochastic operator, in $\text{Poly}(\text{Bits}(B), 1/\varepsilon)$ time.
    \end{itemize}
\end{frame}




\begin{frame}

\frametitle{Upper Bounds For Capacity}

\begin{itemize}
    \pause
    \item To guarantee efficiency, we need to show that for partially normalized $T$, $\text{Cap}(T) \geq 1/e^{\text{Poly}(\text{Bits}(B))}$.

    \pause
    \item (Gurvits, 2004) If $T(A) = \sum B_i A B_i^*$, and $\det(\sum B_i) \neq 0$, then $\text{Cap}(T) \gtrsim (\text{Bits}(B) \cdot n)^{- O(n)}$.

    \pause
    \item If $\text{Cap}(T) > 0$, there is $d > 0$ and $d \times d$ matrices $C_i$ s.t.
    %
    \[ \det(\sum C_i \otimes B_i) \neq 0 \quad\text{and}\quad \text{Bits}(C) \leq \text{Poly}(d,\text{Bits}(B)). \]

    \pause
    \item 'Fairly simple' to show that if $S(A) = \sum C_i A C_i^*$, and $(S \otimes T)(A) = \sum (C_i \otimes B_i) A (C_i \otimes B_i)^*$, then
    %
    \[ \text{Cap}(S \otimes T) \leq \text{Cap}(S)^n \text{Cap}(T)^d. \]

    \pause
    \item Since $\text{Cap}(L) \leq 1$, it follows that
    %
    \[ \text{Cap}(T) \geq \text{Cap}(S \otimes T)^{1/d} \gtrsim (\text{Poly}(d, Bits(B)) n)^{-O(n)}. \]

    \pause
    \item Invariant theory shows we can choose $d \leq n^4 [(n+1)!]^2$.
\end{itemize}

\end{frame}








\begin{frame}

\frametitle{The Invariant Theory}

\begin{itemize}
    \pause
    \item We have a group action of $\text{SL}_n \times \text{SL}_n$ on tuples $B = (B_1,\dots,B_m)$, such that
    %
    \[ (M,N) \circ B = (MB_1N, \dots, MB_mN). \]

    \pause
    \item Invariant Theory: Find the ring $R$ of all `invariant polynomials' $f(B)$ such that
    %
    \[ f((M,N) \circ B) = f(B) \]
    %
    for all $(M,N) \in \text{SL}_n \times \text{SL}_n$ and all tuples $B$.

    \pause
    \item Note: $f_C(B) = \det(\sum C_i \otimes B_i)$ is an invariant homogeneous polynomial under this action for any $C_i$.
\end{itemize}

\end{frame}






\begin{frame}

\frametitle{The Invariant Theory}

\begin{itemize}
    \item (Nayak and Subrahmanyam, 2010) $R$ is a ring generated by the homogeneous polynomials $f_C(A) = \det(\sum C_i \circ A_i)$, for $d \times d$ matrices $C_i$, for all $d > 0$.
    \begin{itemize}
        \pause
        \item (1890) Hilbert showed that for a fairly general family of group actions, $R$ is finitely generated, so we should expect there is some $d_0$ such that $R$ is generated by the polynomials above for $d \leq d_0$.

        \pause
        \item (Ivanyos, Qiao, Subrahmanyam, 2015) $d_0 \lesssim n^4 [(n+1)!]^2$.

        \pause
        \item Suppose there are $d \times d$ matrices $C_i$ (with $d$ minimal) such that $f_C(B) \neq 0$.

        \pause
        \item Find families of matrices $C(1), \dots, C(n)$ of dimension at most $d_0$ such that
        %
        \[ f_C(B) = \sum c_\alpha f_{C(1)}(B)^{\alpha_1} \dots f_{C(n)}(B)^{\alpha_n}.  \]

        \pause
        \item Since $f_C(B) \neq 0$, there must exist $i$ with $f_{C(i)}(B) \neq 0$.

        \pause
        \item Thus $d \leq d_0$.
    \end{itemize}
\end{itemize}

\end{frame}










\begin{frame}
    \frametitle{Rank Decreasing Operators}

    \[ \int_{\RR^n} \prod_{i = 1}^m |f_i(B_i x)|^{p_i}\; dx \leq \text{BL}(B,p) \cdot \prod_{i=1}^m \| f_i \|_{L^1(\RR^{n_i})}^{p_i}.  \]

    \begin{itemize}
        \pause
        \item (Bennett et al, 2008) implies that $\text{BL}(B,p) < \infty$ if and only if $\sum p_i n_i = n$, and for any subspace $V \subset \RR^n$,
        %
        \[ \dim(V) \leq \sum p_i \dim(B_i V). \]

        \pause
        \item An operator $T: M_n \to M_n$ is \emph{rank non-decreasing} if for any $A \succeq 0$, $\text{Rank}(TA) \geq \text{Rank}(A)$.

        \pause
        \item (Gurvits, 2004) $T: M_n \to M_n$ is rank non-decreasing if and only if $\text{Cap}(T) > 0$.
    \end{itemize}
\end{frame}


\begin{frame}
    \frametitle{Proof Idea}

    (Gurvits, 2004) $T: M_n \to M_n$ is rank non-decreasing if and only if $\text{Cap}(T) > 0$.

    \begin{itemize}
        \pause
        \item Results from (Gurvits and Samorodnitsky, 2002) show the result is true if $T(X) = \sum X_{ii} A_i$, where $A_i \succeq 0$. The general case can be reduced to this case.

        \pause
        \item Given an orthonormal basis $U = \{ u_1, \dots, u_N \}$, we define the \emph{decoherence operator} $D_U(A) = \sum \langle Au_i, u_i \rangle \cdot u_i u_i^*$.

        \pause
        \item Define $T_U = D_U \circ T$.

        \pause
        \item Result follows from the following two facts:
        \begin{itemize}
            \pause
            \item[(1)] $T$ is rank non-decreasing if and only if $T_U$ is rank non-decreasing for all $U$.
            
            \pause
            \item[(2)] $\text{Cap}(T) = \inf_U \text{Cap}(T_U)$.
        \end{itemize}

        \pause
        \item To prove (1) and (2), use a simple trick: Given $A \succeq 0$, find $U$ diagonalizing $T(A)$. Then $T(A) = T_U(A)$.
    \end{itemize}
\end{frame}


\begin{frame}
    \frametitle{Thanks For Listening!}
\end{frame}

\end{document}