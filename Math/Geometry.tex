\documentclass[12pt, dvipsnames]{report}

\usepackage{amsmath}
\usepackage{algorithm}
%\usepackage{algorithmic}
\usepackage[noend]{algpseudocode}

\usepackage{amsmath}
\usepackage{amssymb}
\usepackage{amsthm}
\usepackage{amsopn}

\usepackage{kpfonts}

\usepackage{graphicx}

% Probably don't need this on notes anymore
%\usepackage{kbordermatrix}

% Standard tool for drawing diagrams.
\usepackage{tikz}
\usepackage{tkz-berge}
\usepackage{tikz-cd}
\usepackage{tkz-graph}

\usepackage{comment}

%
\usepackage{multicol}

%
\usepackage{framed}

%
\usepackage{mathtools}

%
\usepackage{float}

%
\usepackage{subfig}

%
\usepackage{wrapfig}

%
\let\savewideparen\wideparen
\let\wideparen\relax
\usepackage{mathabx}
\let\wideparen\savewideparen

% Used for generating `enlightening quotes'
\usepackage{epigraph}

% Forget what this is used for :P
\usepackage[utf8]{inputenc}

% Used for generating quotes.
\usepackage{csquotes}

% Allows what to generate links inside
% generated pdf files
\usepackage{hyperref}

% Allows one to customize theorem
% environments in mathematical proofs.
\usepackage{thmtools}

% Gives access to a proof
\usepackage{lplfitch}

% I forget what this is for.
\usepackage{accents}

% A package for drawing simple trees,
% as a substitute for unnesacary TIKZ code
\usepackage{qtree}

% Enables sequent calculus proofs
\usepackage{ebproof}

% For braket notation
\usepackage{braket}

% To change line spacing when using mathematical notations which require some height!
\usepackage{setspace}

%\usepackage[dvipsnames]{xcolor}

\usepackage{float}

% For block commenting
\usepackage{comment}




\setlength\epigraphwidth{8cm}

\usetikzlibrary{arrows, petri, topaths, decorations.markings}

% So you can do calculations in coordinate specifications
\usetikzlibrary{calc}
\usetikzlibrary{angles}

\theoremstyle{plain}
\newtheorem{theorem}{Theorem}[chapter]
\newtheorem{axiom}{Axiom}
\newtheorem{lemma}[theorem]{Lemma}
\newtheorem{corollary}[theorem]{Corollary}
\newtheorem{prop}[theorem]{Proposition}
\newtheorem{exercise}{Exercise}[chapter]
\newtheorem{fact}{Fact}[chapter]

\newtheorem*{example}{Example}
\newtheorem*{proof*}{Proof}

\theoremstyle{remark}
\newtheorem*{exposition}{Exposition}
\newtheorem*{remark}{Remark}
\newtheorem*{remarks}{Remarks}

\theoremstyle{definition}
\newtheorem*{defi}{Definition}

\usepackage{hyperref}
\hypersetup{
    colorlinks = true,
    linkcolor = black,
}

\usepackage{textgreek}

\makeatletter
\renewcommand*\env@matrix[1][*\c@MaxMatrixCols c]{%
  \hskip -\arraycolsep
  \let\@ifnextchar\new@ifnextchar
  \array{#1}}
\makeatother

\renewcommand*\contentsname{\hfill Table Of Contents \hfill}

\newcommand{\optionalsection}[1]{\section[* #1]{(Important) #1}}
\newcommand{\deriv}[3]{\left. \frac{\partial #1}{\partial #2} \right|_{#3}} % partial derivative involving numerator and denominator.
\newcommand{\lcm}{\operatorname{lcm}}
\newcommand{\im}{\operatorname{im}}
\newcommand{\bint}{\mathbf{Z}}
\newcommand{\gen}[1]{\langle #1 \rangle}

\newcommand{\End}{\operatorname{End}}
\newcommand{\Mor}{\operatorname{Mor}}
\newcommand{\Id}{\operatorname{id}}
\newcommand{\visspace}{\text{\textvisiblespace}}
\newcommand{\Gal}{\text{Gal}}

\newcommand{\xor}{\oplus}
\newcommand{\ft}{\wedge}
\newcommand{\ift}{\vee}

\newcommand{\prob}{\mathbf{P}}
\newcommand{\expect}{\mathbf{E}}
\DeclareMathOperator{\Var}{\mathbf{V}}
\newcommand{\Ber}{\text{Ber}}
\newcommand{\Bin}{\text{Bin}}

%\newcommand{\widecheck}[1]{{#1}^{\ft}}

\DeclareMathOperator{\diam}{\text{diam}}

\DeclareMathOperator{\QQ}{\mathbf{Q}}
\DeclareMathOperator{\ZZ}{\mathbf{Z}}
\DeclareMathOperator{\RR}{\mathbf{R}}
\DeclareMathOperator{\HH}{\mathbf{H}}
\DeclareMathOperator{\CC}{\mathbf{C}}
\DeclareMathOperator{\AB}{\mathbf{A}}
\DeclareMathOperator{\PP}{\mathbf{P}}
\DeclareMathOperator{\MM}{\mathbf{M}}
\DeclareMathOperator{\VV}{\mathbf{V}}
\DeclareMathOperator{\TT}{\mathbf{T}}
\DeclareMathOperator{\LL}{\mathcal{L}}
\DeclareMathOperator{\EE}{\mathbf{E}}
\DeclareMathOperator{\NN}{\mathbf{N}}
\DeclareMathOperator{\DQ}{\mathcal{Q}}
\DeclareMathOperator{\IA}{\mathfrak{a}}
\DeclareMathOperator{\IB}{\mathfrak{b}}
\DeclareMathOperator{\IC}{\mathfrak{c}}
\DeclareMathOperator{\IP}{\mathfrak{p}}
\DeclareMathOperator{\IQ}{\mathfrak{q}}
\DeclareMathOperator{\IM}{\mathfrak{m}}
\DeclareMathOperator{\IN}{\mathfrak{n}}
\DeclareMathOperator{\IK}{\mathfrak{k}}
\DeclareMathOperator{\ord}{\text{ord}}
\DeclareMathOperator{\Ker}{\textsf{Ker}}
\DeclareMathOperator{\Coker}{\textsf{Coker}}
\DeclareMathOperator{\emphcoker}{\emph{coker}}
\DeclareMathOperator{\pp}{\partial}
\DeclareMathOperator{\tr}{\text{tr}}

\DeclareMathOperator{\supp}{\text{supp}}

\DeclareMathOperator{\codim}{\text{codim}}

\DeclareMathOperator{\minkdim}{\dim_{\mathbf{M}}}
\DeclareMathOperator{\hausdim}{\dim_{\mathbf{H}}}
\DeclareMathOperator{\lowminkdim}{\underline{\dim}_{\mathbf{M}}}
\DeclareMathOperator{\upminkdim}{\overline{\dim}_{\mathbf{M}}}
\DeclareMathOperator{\lhdim}{\underline{\dim}_{\mathbf{M}}}
\DeclareMathOperator{\lmbdim}{\underline{\dim}_{\mathbf{MB}}}
\DeclareMathOperator{\packdim}{\text{dim}_{\mathbf{P}}}
\DeclareMathOperator{\fordim}{\dim_{\mathbf{F}}}

\DeclareMathOperator*{\argmax}{arg\,max}
\DeclareMathOperator*{\argmin}{arg\,min}

\DeclareMathOperator{\ssm}{\smallsetminus}

\title{Geometry}
\author{Jacob Denson}

\begin{document}

\pagenumbering{gobble}
\maketitle
\tableofcontents
\pagenumbering{arabic}

\part{Euclid}

I'm writing these notes so that I can understand Euclidean geometry better. We'll build up the axioms from the ground up, so I can understand Euclid's work from the ground up. Thus these notes probably won't be useful for someone trying to understand Euclid themselves, because it's just my ramblings about the subject.

\chapter{Book I}

Basic Euclidean geometry consists of three objects: Points, Lines (both finite lines with endpoints, an infinite lines with no extremities), and Circles (defined by a point and a radius). Classically, these objects were seen as distinct, but with the power of set theory, it is easier to model lines and circles as sets of points. This has the advantage of making things notationally simple. These is no real logical difference between switching to this notation -- any theorem provable in one system is provable in the other. However, we'll avoid from using set theory too much, to avoid making the exposition too austere.

Euclid was the first to pioneer the axiomatic method in mathematics. However, the philosophy behind his proofs was different to ours. At the end of the day, his arguments attack a particular model of the planar geometry find in our world, and he proves things like a physicist, adopted some methods of proof not explicitly stated in his assumptions. This causes problems for us when we try and look at his proofs from a modern day perspective. We will eventually look at other logical systems for geometry, but for now a naive approach will be most useful.

Most of Euclid's proofs concern constructions of certain figures in the planes. Rather than a proof of existence, Euclid literally builds these figures from the ground up. In the early parts of the text these figures will all be defined by a simple curve consisting of straight lines, so that we may describe such a figure by the sequence of points which define the figure. If $X_1, \dots, X_n$ are points, then $X_1 \dots X_n$ will denote the figure obtained by drawing the line $X_1 X_2$, then $X_2 X_3$, and so on, finishing off by drawing $X_n X_1$. Two such figures will be considered equal if we may obtain the points of one from the points of the other by performing a cycle permutation of the points. For instance, a {\bf triangle} is just a sequence of distinct points $ABC$, and $ABC = BCA = CAB$, and we can abuse the notation, denoting a line between two points $A$ and $B$ as $AB = BA$. The question of whether this is a unique description of such a line is settled by the first axiom of geometry.

\begin{axiom}
    There is a unique straight line between any pair of points, having those points as endpoints.
\end{axiom}

Euclid does not assume that the straight line which exists between the points is unique, but later he uses the fact that a finite line is defined by its endpoints, so we can only assume that he really wants this fact to hold. In order to discuss the lengths of lines, we shall be required to discuss circles at points, and so we introduce the second axiom.

\begin{axiom}
    A circle may be described with any centre and radius.
\end{axiom}

A circle is {\it defined} by its centre and radius, so the circle which exists by this axiom is unique. Note that circles with a different radii and the same centre may still be equal. Indeed, this happens exactly when the two radii have the same length, a concept we will very shortly discuss.

Euclid defines an {\bf equilateral triangle} as `a triangle whose three sides are equal', which he really means as saying the {\it magnitude}, or length, of the sides are equal. In Euclid's synthetic geometry, there do not exist real numbers to assign length to, and as is well known most Greek's did not even believe in irrational numbers. But we shall find that we can get away with much of the theory of magnitude without ever mentioning the concept of a number, which gives a certain sense of satisfaction.

Right now, we only need equality in the length of lines, and we shall discuss a very aggreeable manner in checking equality. If we have two lines $AB$ and $AC$ with a common point, we can check if they have equal length by checking if the circles constructed with centre $A$ and radii $AB$ and $AC$ are equal. This gives us an equivalence relation on the set of lines extending out from $A$. We shall require that this equivalence relation describes exactly the set of circles with centre $A$, so that a point $C$ lies on the circle with centre $A$ and radius $AB$ if and only if the length of $AC$ is equal to the length of $AB$.

\begin{axiom}
    If $C$ lies on the circle with radius $A$ and radii $AB$, then $AC$ has the same length as $AB$.
\end{axiom}

In order to generalize equality of length of arbitrary lines, we just make the relation transitive. The relation is already reflexive and symmetric, so this generates an equivalence relation on the set of all lines in the plane. Thus we see that the only basic way to check if two lines $AB$ and $CD$ are equal is to form a sequence of lines beginning at $B$, and ending at $C$, which are all equal to one another as lines extending from the same basepoint.

\begin{theorem}
    Any finite line lies on an equilateral triangle.
\end{theorem}
\begin{proof}
    To prove the existence of an equilateral triangle at a line $AB$, Euclid constructs the circle with radius $AB$ and centre $A$, and the circle with centre $B$ and radius $AB$, and considers their point of intersection $C$. Since $C$ lies on the first circle, $AB$ has the same length as $AC$, and since $C$ lies on the second circle, $CB$ has the same length as $AB$. But then the lines $AB, BC$, and $CA$ describe an equilateral triangle, and so $ABC$ is the triangle required.
\end{proof}

There is only one problem remaining in this proof. There is nothing saying that the two circles given will have a common point of intersection. We could describe an axiom which supplies us with such a point, but this axiom would probably be more general than the theorem itself. Indeed, the existence of a point on the intersection of two circles with the same radius but different centres is equivalent to the theorem we set out to prove. Thus we shall have to settle on the fact that theorem one must be treated as an assumption from our current viewpoint.

\begin{theorem}
    Given a point $A$ and line $BC$, to construct a line extending from $A$ with the same length as $BC$.
\end{theorem}
\begin{proof}
    Construct an equilateral triangle $ABD$ on the line $AB$. Then construct the circle with centre $B$ and radius $BC$. Find an intersection point $E$ on the circle which either lies on the line $BD$, or extends the line, and then construct the circle with centre $B$ and radius $BE$. Extend the line $DA$ from the extremity $A$ to an intersection point $F$ on the circle. We claim $AF$ has the same length as $BC$. Indeed, the length of $DF$ is the sum of the length of $DA$ and $AF$, and the length of $DE$ is the sum of $DB$ and $BE$. Since the length of $DF$ is equal to $DE$, since they both lie on the same circle extending from $D$, and the length of $DA$ is equal to the length of $DB$, we may subtract to conclude that the length of $AF$ is the same as the length of $BE$. But $BE$ has the same length of $BC$, which is all that is required to show $AF$ has the same length as $BC$.
\end{proof}

\chapter{Analytic Geometry}

It is sometimes interesting to weaken the axioms of geometry to see that a more general class of models results. In this chapter, we study geometries satisfying three axioms
%
\begin{itemize}
    \item Through any two distinct points there is a unique line between them. Given two points $X$ and $Y$, we shall let $XY$ denote the unique line through $X$ and $Y$.
    \item Given a line and some point, there is a unique line through the point parallel to the line (A line is parallel to another line if they don't intersect, or if they are equal to one another).
    \item There exist three non colinear points in the geometry.
\end{itemize}
%
Euclidean geometry is an extension of this axiom set, so any model of Euclidean geometry is automatically a model of this weaker axiom set, and we call the study of models of this axiom set {\bf affine geometry}.

\begin{example}
    Let $K$ be a field, and consider the geometry whose points are elements of $K^2$, and whose lines are the affine span $v + Kw = \{ v + xw : x \in K \}$, for a nonzero $w \in K^2$. It is easy to see that every line through a vector $v$ can be written as $v + Kw$, and for two lines $v + Kw_0$ and $v + Kw_1$, either the two lines intersect only at $v$, or the two lines are equal and $w_0$ is a scalar multiple of $w_1$. This tells us that lines intersecting at two or more common points are equal, and therefore there is a unique line $v + K(w-v)$ between any two points $v$ and $w$. A line $v_0 + Kw_0$ is parallel to a line $v_1 + Kw_1$ if and only if $w_0$ is a scalar multiple of $w_1$, and for any point $v_1$ not on a line $v_0 + Kw$, the line $v_1 + Kw$ is a line containing $v_1$ and parallel to $v_0 + Kw$, and this is the unique such line. Finally, the points $(0,1)$, $(1,0)$, and $(0,0)$ are non colinear, so $K^2$ is a model of affine geometry.
\end{example}

Our main result will be that these geometries describe all the possible models of affine geometry. That is, every model of affine geometry can be associated with a copy of $K^2$, where lines in one space map to lines in the other. This shows that over the class of affine geometries, `Cartesian analytic geometry' suffices to verify any result. To begin with this construct, consider any affine geometry. We will give a field structure to any line $OI$ in the geometry, with which we may identify the geometry as the vector space $(OI)^2$. We shall define an algebraic structure in which $O$ is the additive identity, and $I$ is the multiplicative identity.

\begin{center}
\begin{tikzpicture}
    \coordinate[label=left:A] (A) at (0,0);
    \coordinate[label=right:B] (B) at (3,0);
    \coordinate[label=left:C] (C) at (1,2);
    \coordinate[label=right:D] (D) at (4,2);

    \draw (A)--(C);
    \draw (A)--(B);
    \draw (B)--(D);
    \draw (C)--(D);

    \foreach \p in {A,B,C,D} \fill[fill=white,draw=black,thick] (\p) circle (2pt);
\end{tikzpicture}
\end{center}

\begin{lemma}
    If $AB$ is a line, then for any $C \not \in AB$, there is a unique $D$ such that $CD$ is parallel to $AB$, and $BD$ is parallel to $AC$.
\end{lemma}
\begin{proof}
    Since $C \not \in AB$, there is a unique line through $C$ parallel to $AB$, and a unique line through $B$ parallel to $AC$. Since $AC$ is not parallel to $AB$, the lines we have formed intersect at a unique point, which is the point $D$ required.
\end{proof}

\begin{corollary}
    In the context of the previous theorem, there is a unique bijection $f: AB \to CD$ such that $Xf(X)$ is parallel to $AC$.
\end{corollary}
\begin{proof}
    Using the previous theorem, for each $X$ there is a point $Y$ such that $CY$ is parallel to $AB$, and $XY$ is parallel to $AC$. This implies that $CY = CD$, so $Y \in CD$ and we can {\it define} a map $f: AB \to CD$ satisfying the property of the lemma, and we have certainly justified it is unique. Using the symmetry of the problem, there is a unique map $g: CD \to AB$ such that $Xg(X)$ is parallel to $AC$ for each $X$. Then $(g \circ f)(X) = X$, because $f(X)X$ is parallel to $AC$, and by symmetry, we also find $(f \circ g)(X) = X$, so $f$ really is a bijection.
\end{proof}

Now given $O$, fix a line $L$ through $O$. For each $A,B \in L$, fix $M \not \in L$, and let $L_M$ be the line through $M$ parallel to $L$. Then there is a unique point $C$ such that $AC$ is parallel to $OM$. Similarily, there is a unique point $D$ on $L$ such that $BM$ is parallel to $CD$. We define $D = A + B$. We require a certain lemma to verify this is well defined.

\begin{lemma}
    If $A_0B_0$ is parallel to $A_1B_1$, $B_0C_0$ is parallel to $B_1C_1$, and $A_0A_1$, $B_0B_1$, and $C_0C_1$ are all parallel, then $A_0C_0$ is parallel to $A_1C_1$.
\end{lemma}
\begin{proof}
    Suppose that $A_0C_0$ and $A_1C_1$ meet at a point $X$. If $C_0 \neq C_1$, then there is a point $Y \neq X$ such that $XY$ is parallel to $C_0C_1$
\end{proof}

To verify this is well define, if $N \not \in L$ is any other point, let $L_N$ be the line through $N$ parallel to $L$. Then there is a unique point $C'$ such that $AC'$ is parallel to $ON$. Then $CC'$ is parallel to $MN$


Let $f_0: L \to L_M$ be the map such that $Xf_0(X)$ is parallel to $OM$. Similarily, define $f_1: L_M \to L$ be such that $Xf_1(X)$ is parallel to $Bf_0(X)$, and we define $A + B = f_1(f_0(A))$.

Now consider the line $OI$ we fixed. Then for any $A,B \in OI$,

for a fixed $M \not \in OI$ there is a unique point $C$ lying on the line through $M$ parallel to $OI$ such that $AC$ is parallel to $OM$. Now we can use the lemma above in reverse to find a point $D \in OI$ such that $CD$ is parallel to $BM$, and we define $A + B = D$.

To verify that addition is well defined, fix some $N \not \in OI$. Let $L_M$ be the line through $M$ parallel to $OI$, and $L_N$ the line through $N$ parallel to $OI$. Then let $f: OI \to L_M$, $g: OI \to L_N$ be the unique bijections constructed by the corollary above. Then $D = f(A)$


Then $g \circ f^{-1}: L_M \to L_N$ is such that $X(g \circ f^{-1})(X)$ is parallel to $MN$

To verify that $D$ is uniquely defined, fix some $N \not \in OI$. Then there is a unique point $C'$ lying on the line through $M$ parallel to $OI$ such that $AC'$ is parallel to $ON$. Since the line through $M$ parallel to $OI$ is parallel to the line through $N$ parallel to $OI$, $C'$ is the unique point on the line through $N$ parallel to $OI$ such $CC'$ is parallel to $MN$.

First, note that for any point $M \not \in OI$, there is a unique line through $M$ parallel to $OI$. Given $X \in OI$, there is a unique line through $X$ parallel to $OM$. Since $OM$ is not parallel to $OI$, there is a unique point $N$ lying on the intersection of these lines, thus $MN$ is parallel to $OI$, and $XN$ is parallel to $OM$. For any point on $OM$, and for a fixed $X$, we can always perform this construction, yielding a map $f: OM \to XN$, with $f(O) = X$, and $f(M) = N$, and more generally, $f(Y)$ is the unique point such that $Xf(Y)$ is parallel to $OM$, and $Yf(Y)$ is parallel to $OI$. It is injective, because if $XZ$ is parallel to $OM$, and $Y_0Z$ and $Y_1Z$ are both parallel to $OI$, then by uniqueness of parallel lines we find $Y_0Z = Y_1Z$, and since distinct lines intersect in a unique position, we must have $Y_0 = Y_1$. The surjectivity follows because we may always swap $O$ and $X$ in the theorem to conclude that for any $Z \in XN$, there is a unique point $Y \in OM$ such that $YZ$ is parallel to $OI$, and then $f(Y) = Z$.

Now given $OI$, consider $X,Y \in OI$, and fix $M \not \in OI$. Consider the unique line through $M$ parallel to $OI$, and consider the bijection $f$ above with $f(x)$

\end{document}