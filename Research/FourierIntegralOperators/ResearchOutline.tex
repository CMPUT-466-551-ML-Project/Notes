\documentclass[12pt]{article}

\usepackage{amsmath}
\usepackage{amssymb}
\usepackage{amsthm}
\usepackage{esint}

\usepackage{mathptmx}
\usepackage[margin=0.75in]{geometry}

\DeclareMathOperator{\RR}{\mathbf{R}}

\theoremstyle{plain}
\newtheorem{theorem}{Theorem}
\newtheorem{lemma}[theorem]{Lemma}
\newtheorem{corollary}[theorem]{Corollary}
\newtheorem{prop}[theorem]{Proposition}

\theoremstyle{remark}
\newtheorem*{example}{Example}
\newtheorem*{remark}{Remark}

\theoremstyle{definition}
\newtheorem*{defi}{Definition}
\newenvironment{definition}
    {\begin{samepage}\begin{framed}\begin{defi}}
    {\end{defi}\end{framed}\end{samepage}}

\title{Outline of Proposed Research}
\author{Jacob Denson}
\date{\today}

\begin{document}

\maketitle

The $L^p$ boundedness of translation-invariant operators on $\RR^n$ has proved central to the development of modern harmonic analysis. 
Indeed, such questions underpin any subtle understanding of the Fourier transform, since we can associate with essentially any such operator $T$ a function $m(\xi)$, known as the \emph{symbol} of $T$, such that $\widehat{Tf} = m \widehat{f}$ for any such function $f$; thus translation invariant operators are also called \emph{Fourier multiplier operators}. Questions about such operators emerged from classical questions concerning the convergence properties of Fourier series, and in the study of the classical equations of physics, like the heat and wave equation. Such questions often have rotational symmetry, so it is natural to restrict our attention to translation-invariant operators which are also rotation-invariant. These operators are called \emph{radial Fourier multipliers}, since the associated symbol will always be a radial function. This research project proposes the study of necessary and sufficient conditions to guarantee the boundedness of radial Fourier multipliers, stimulated by recent developments which indicate lines of attack for three related problems in the field.

The general study of conditions guaranteeing the boundedness of Fourier multipliers was intiated in the 1960s, most explicitly in the work of Mikhlin and H\"{o}rmander. It was quickly realized that the most fundamental estimates for a translation-invariant operator $T$ were $L^p$ estimates of the form $\| Tf \|_{L^p(\RR^n)} \lesssim \| f \|_{L^p(\RR^n)}$ for $1 \leq p \leq 2$. For $p = 1$ and $p = 2$ mathematicians found simple necessary and sufficient conditions to ensure such a bound held \cite{Hormander1}. For $1 < p < 2$, sufficient conditions were found ensuring boundedness \cite{Mikhlin}, but the next half century proved that the problem of finding a simple characterization of $L^p$ boundedness was nigh impenetrable. Indeed, many problems which were formulated in this period, such as the Bochner-Riesz conjecture (CITE?), and Radon-like averaging problems, concern $L^p$ boundedness for \emph{particular} Fourier multiplier operators, and still remain largely unsolved today.

Thus it was surprising when simple necessary conditions \emph{were} found when the multiplier operator was assumed to be \emph{radial}. First came the result of \cite{GarrigosandSeeger}, which gave a simple sufficient and necessary condition for bounds of the form $\| Tf \|_{L^p(\RR^n)} \lesssim \| f \|_{L^p(\RR^n)}$ to hold for a particular radial multiplier operator $T$ uniformly over \emph{radially symmetric} functions $f$ precisely when $1 < p < 2 - 2/(n+1)$. It is a natural conjecture that the same criterion, applied in the same range of $p$, gives the bound $\| Tf \|_{L^p(\RR^n)} \lesssim \| f \|_{L^p(\RR^n)}$ for \emph{general} functions $f$. We now know \cite{HeoandNazarovandSeeger} that this conjecture is true when $n \geq 4$ and $1 < p < 2 - 4/(n+1)$. We also know \cite{Cladek} the criterion is sufficient when $n = 3$ and $1 < p < 2 - 3.66/(n+1)$, and when $n = 4$ to $1 < p < 2 - 3.79/(n+1)$, provided one restricts to the study of \emph{compactly supported} radial Fourier multiplier operators. But the conjecture implied by \cite{GarrigosandSeeger} has not yet been completed resolved in any dimension $n$.

If fully proved, the conjecture of \cite{GarrigosandSeeger} would imply the Bochner-Riesz conjecture, and thus the Kakeya and restriction conjectures as a result. A complete resolution of the problem is thus far beyond the scope of current research techniques. On the other hand, the Bochner Riesz conjecture is completely resolved when $n = 2$, while in contrast, no results related to the conjecture of \cite{GarrigosandSeeger} are known in this dimension at all. And in any dimension $n > 2$, the range under which the Bochner-Riesz multiplier is known to hold (see \cite{GuoandOhandWangandWuandZhang} for the most state of the art results) is also strictly larger than the range under which the conjecture of \cite{GarrigosandSeeger} is known to hold, even given the result of \cite{Cladek}, restricted to radial Fourier multipliers. Thus it still seems within hope that the techniques recently applied to improve results for Bochner-Riesz problem, such as broad-narrow analysis \cite{BourgainandGuth}, polynomial partitioning \cite{GuthTextbook}, and the polynomial Wolff axioms \cite{KatzandRogers}, can also be applied to give improvements to current results in the scheme we have described.

Our hopes are further emboldened when we consult the proofs in \cite{HeoandNazarovandSeeger} and \cite{Cladek}, which reduce the problem to $L^p$ bounds of the form $\| \sum_{(y,r) \in \mathcal{E}} F_{Y,r} \|_{L^p(\RR^n)}$, where $\mathcal{E} \subset (0,\infty) \times \RR^n$ is a finite set of pairs, and $F_{y,r}$ is, very roughly speaking, comparable to an indicator function supported on a neighborhood of a sphere in $\RR^n$ of radius $r$ centered at a point $y$. The $L^p$ norm of this sum is closely related to how the spheres involved in the sum intersect tangentially. The study of tangential intersections of spheres has very recently been studied in more combinatorial settings using polynomial partitioning methods \cite{Zahl}, which further indicates that the methods above might yield further estimates.

In addition to improving the range of $p$ values on which we can bound the $L^p$ sum in the last paragraph, it is also of independant interest to obtain estimates for the ranges of $p$ considered in \cite{Cladek}, when the sums do \emph{not} range over a set $\mathcal{E}$ which is a Cartesian product. This would not improve results on the conjecture of \cite{GarrigosandSeeger}. But it would imply new results for the `endpoint' local smoothing conjecture, which concerns the regularity of solutions to the wave equation over closed time intervals. The methods of incidence geometry have been recently applied to yield results on the `non-endpoint' local smoothing conjecture \cite{GuthandWangandZhang}, which suggests these might be pushed to yield better results in the range of values of $p$ that \cite{Cladek} considers.

A third line of questioning follows by studying `radial Fourier multipliers' on more general domains than $\RR^n$. For a Riemannian manifold $M$, the natural analogue of a radial Fourier multiplier operator is an operator of the form $m(-\Delta)$, where $m: [0,\infty) \to \mathbf{C}$, and $\Delta$ is the Laplace-Beltrami operator on $M$. Just like multiplier operators on $\RR^n$ are crucial to an understanding of the Fourier transform, multiplier operators on $M$ are crucial to understand the behaviour of eigenfunctions of the Laplace-Beltrami operator on $M$.

In the case where $M$ is compact, the direct analogue of the boundedness problem for Fourier multipliers is trivial, since any compactly supported function $m$ will induce an operator $m(-\Delta)$ satisfying estimates of the form $\| m(-\Delta) f \|_{L^p(M)} \lesssim \| f \|_{L^p(M)}$ for all $1 \leq p \leq 2$. To make the problem at least as hard as the original problem on $\RR^n$, we apply dilation, considering bounds of the form $\sup_{R > 0} \| m(-\Delta/R) f \|_{L^p(M)} \lesssim \| f \|_{L^p(M)}$. In $\RR^n$, this equation is equivalent to the bound $\| m(-\Delta) f \|_{L^p(M)} \lesssim \| f \|_{L^p(M)}$, but not on compact manifolds. In particular, a result of Mityagin implies that if the rescaled bound holds for a multiplier $m(-\Delta)$ holds, then a bound of the form $\| Tf \|_{L^p(\RR^n)} \lesssim \| f \|_{L^p(\RR^n)}$ holds where $T$ is the Fourier multiplier operator associated with the symbol $m(|\xi|)$. The simple condition of \cite{GarrigosandSeeger} has a natural extension to this setting, and so we can now ask whether, in the appropriate range of exponents, this condition is necessary and sufficient for the bound to hold for any compact manifold $M$. TODO: TALK TO ANDREAS ABOUT WHAT THIS IMPLIES ABOUT EIGENFUNCTIONS.  There are many results in the literature generalizing results on $\RR^n$ to result on the sphere, like the classical result of \cite{Sogge} who generalized the Carleson-Sj\"{o}lin theorem to the study of a family of Fourier multiplier operators on the sphere. It therefore seems that, at least in the ranges established in \cite{HeoandNazarovandSeeger} or even \cite{Cladek}, a resolution of this problem seems within reach, at least if we restrict ourselves to simple manifolds like the sphere.

In conclusion, the results of \cite{HeoandNazarovandSeeger} and \cite{Cladek} indicate three lines of questioning about radial Fourier multiplier operators, which current research techniques seem to allow us to now answer. The first question is whether we can extend the range of exponents upon which the conjecture of \cite{GarrigosandSeeger} is true, at least in the case $n = 2$ where Bochner-Riesz has been solved. The second is whether we can use more sophisticated arguments to prove the $L^p$ sum bounds obtained in \cite{Cladek} when the sums are no longer cartesian products, thus obtaining new results about the endpoint local smoothing conjecture. The third question is whether we can generalize these bounds obtained in these two papers to radial Fourier multipliers on compact Riemannian manifolds, at least on the sphere. LAST SENTENCE WHAT TO SAY?



\newpage

TODO: WHY DOESN'T CLADEK'S PROOF WORK FOR NON COMPACTLY SUPPORTED MULTIPLIERS.

Many fundamental questions remain unsolved about multiplier operators, even if we restrict our study to radial Fourier multipliers. Indeed, the Bochner-Riesz conjecture, a major conjecture in the field which despite decades of research still seems quite far from being resolved, concerns the boundedness properties of a particular compactly supported radial Fourier multiplier. This research project proposes the study of a family of related problems in the study of radial multipliers, which, stimulated by recent developments in the theory of planar radial Fourier multipliers, seem 

outside of the class of operators associated with a compactly supported, smooth symbol. Indeed, 

However, even restricted to the class of radial Fourier multipliers, many fundamental questiosn remain unsolved. Indeed, 

Since many equations of physics are rotationally symmetric, it is natural to restrict our attention to translation-invariant operators which are also \emph{rotation-invariant}, 


But much still remains unknown about such operators today, even for simple, particular examples, such as that given by a compactly supported radially symmetric bump function $m(\xi)$

 where $m$ is given by a simple 



 they are given , has proved to be a central question throughout the development of harmonic analysis. Such questions naturally arise from classical questions, such as the convergence properties of Fourier series, and in the study of the classical equations of physics, such as the heat or wave equation. But much still remains unknown about Fourier multiplier operators, even if one restricts to the natural subclass of \emph{radial} Fourier multipliers, i.e. translation-invariant operators which are also invariant under rotations. For instance, the Bochner-Riesz conjecture asks to determine the boundedness properties of a single radial Fourier multiplier operator, namely, 


 whether for classical questions such as the convergence properties of Fourier series, or more modern questions such as averaging problems over low dimensional curves and surfaces.

For the past half century, determining the boundedness of

A Fourier multiplier operator takes in as input a function formed from the superposition of planar waves travelling in various directions and at various frequencies, and transforms this function by rescaling the amplitudes of each of these waves independently. Simple examples of such operators include the Hilbert transform and any constant coefficient linear differential operator. Radial Fourier multiplier operators form an important subclass, which are multipliers with the property that the associated rescaling factor for each wave depends solely on the frequency of the wave, and not the direction it is travelling in. If instead of considering planar waves, one instead considers Fourier multipliers for `surface', or `ground' waves, travelling on other geometric spaces, such as spheres, torii, or other higher dimensional manifolds, then one obtains the study of 'variable coefficient' Fourier multipliers. This research project studies conditions guaranteeing the stability of `variable-coefficient' radial Fourier multipliers, stimulated by recent developments in the theory of planar radial Fourier multipliers which indicate lines of attack to three related problems.

Fourier multiplier operators are crucial to the understanding of many areas of pure mathematics and the sciences. They occur in complex analysis, via the `Hilbert transform' and it's variants, and Riemannian geometry, where the `Riesz transform' allows one to decompose vector fields into incompressible and conservative vector fields. Outside of mathematics, they are a key tool in signals processing (where Fourier multiplier operators are called 'filters'), forming key components of physical systems such audio encoding software, and medical technology like MRI and CAT scanners. Many solutions to partial differential equations can be expressed in terms of Fourier multiplier operators, whether classical equations like the heat equation, Schr\"{o}dinger equation, and wave equation (ADD CITATION), or more specialized equations, such as the bidomain Allen-Kahn equation, used to model a human heart's electrical activity for various applications, including the analysis of the effects of cardiac pacemakers on the function of a heart (ADD YOICHIRO MORI CITATION). Thus Fourier multiplier operators are crucial to the understanding of physical phenomena. Radial Fourier multipliers naturally occur in these settings, since most physical phenomena obeys some form of rotational symmetry.

% MAYBE INCLUDE THIS FOR AN APPLICATION TO WAVES ON SPHERES: https://reader.elsevier.com/reader/sd/pii/S0167278916306352?token=D6888E33CEC29E51A0DD831AB7647DC6CEA0D43F9E5AE1811A7B7FFC86FDA8E772AC5ED83B68421DBDC803A713432BCA&originRegion=us-east-1&originCreation=20211008055535

In order to apply a Fourier multiplier operator in a physical scenario, it is crucial to understand the stability properties of the operator. A filter cannot be used to process a signal unless there is a guarantee that the filter will behave well under the effects of noisy input values. And instabilities of the operators involved in the study of cardiac pacemakers, via the Allen-Kahn equation, may prove fatal, leading to a lethal cardiac arrythmia. In pure mathematics, understanding the stability of Fourier multiplier operators is also important, giving us insights such as the theory of local smoothing for the wave equation (INCLUDE CITATION), and (INCLUDE ANOTHER EXAMPLE HERE). The study of the stability of Fourier multiplier operators is thus a central question in harmonic analysis, and also has many important consequences outside of mathematics.

In harmonic analysis, we frame the stability problem for a Fourier multiplier operator in terms of analyzing the boundedness of the multiplier, viewed as an operator between two Banach spaces, often from an $L^p$ space to itself for some $1 \leq p \leq 2$. Showing an operator is bounded between two Banach spaces thus guarantees the stability of the operator when the input is perturbed by a quantity with small norm. The classical study of the stability of Multiplier operators was initiated in the 1960s by H\"{o}rmander (INCLUDE 1960s PAPER CITATION), who obtained a simple necessary and sufficient criterion for a Fourier multiplier operator to be bounded from $L^1(\RR^n)$ to $L^1(\RR^n)$ and from $L^2(\RR^n)$ to $L^2(\RR^n)$. Over the next half century, various sufficient criteria were found to guarantee the boundedness of operators for intermediate values of $p$, such as the H\"{o}rmander Mikhlin multiplier theorem, a result still being iterated upon today (Grafakos, Slavikona CITATION 2017). But no simple necessary and \emph{sufficient} conditions have been found in the past 50 years which guarantee a Fourier multiplier is bounded from $L^p(\RR^n)$ to $L^p(\RR^n)$ for any other value of $p \not \in \{ 1, 2, \infty \}$, with many speculating a simple condition may not even exist for other values of $p$.

Thus it was surprising when (CITATION) showed that for \emph{radial} Fourier multipliers, such a simple necessary and sufficient criterion to ensure boundedness exists from $L^p(\RR^n)$ to $L^p(\RR^n)$ \emph{does} exist for $n \geq 4$ and $1 < p < 2(n-1)/(n+1)$. This same criterion was later shown (CITATION) to be necessary and sufficient for boundedness when $n = 3$ and $1 < p < 13/12$ or improved to the range $1 < p < 36/29$ when $n = 4$, under the additional assumption that the Fourier multiplier operator is compactly supported. It remains an incredibly important and difficult open question to determine whether the criterion developed in (CITATION) continues to hold in the larger range $1 < p < 2d/(d+1)$, which is know to be the maximal interval under which the simple criterion developed by (CITATION) could hold. A positive proof of this result would answer several major conjectures in harmonic analysis, including the Bochner-Riesz conjecture, the Kakeya conjecture, and the restriction conjecture, and so further analysis of the techniques developed by results such as (CITATION) and (CITATION) is integral to the development of the field of harmonic analysis.

One way we can better understand the techniques developed in (CITATION) and (CITATION) is to see whether the criterion developed in these papers continues to hold in the variable coefficient setting, where the curvature of the underlying spaces upon which the waves are travelling complicates the analysis such that one cannot simply trivially reduce the analysis to the planar case. On the other hand, there is evidence to suggest that such an analysis is both within research of a research project, and is nontrivial enough to have fruitful consequences. Variable coefficient generalizations of results in harmonic analysis have already been in other settings, such as the result of (SOGGE CITATION), who found a way to generalize the Tomas-Stein restriction theorem to a variable coefficient setting, with the consequence that the result implied much better control on the $L^p$ norms of eigenfunctions for the Laplace-Beltrami operator on general Riemannian manifolds. (WHAT CONSEQUENCES WOULD OUR VARIABLE COEFFICIENT FORMULATION HAVE?)

Given the importance of the study of variable coefficient generalizations of the results of (CITATION) and (CITATION) to both harmonic analysis and it's applications in the general sciences, and the evidence to suggest that such a result is within reach, we feel that the analysis of such results should be pursued as a research project.


 

\begin{thebibliography}{9}

\bibitem{Hormander1}
    Lars H\"{o}rmander,
    \emph{Estimates for translation invariant operators in $L^p$ spaces},
    Acta Math.,
    1960.

\bibitem{Mikhlin}
    Soloman G. Mikhlin,
    \emph{On the multipliers of {Fourier} integrals},
    Dokl. Akad. Nauk,
    1956.

\bibitem{GarrigosandSeeger}
    G. Garrig\'{o}s and Andreas Seeger,
    \emph{Characterizations of {Hankel} multipliers},
    Math. Ann.,
    2008.

\bibitem{HeoandNazarovandSeeger}
    Yaryong Heo and F\"{e}dor Nazarov and Andreas Seeger,
    \emph{Radial {Fourier} multipliers in high dimensions},
    Acta. Math.,
    2011.    

\bibitem{Cladek}
    Laura Cladek,
    \emph{Radial {Fourier} Multipliers in $\RR^3$ and $\RR^4$},
    Anal. PDE,
    2018.

\bibitem{GuoandOhandWangandWuandZhang}
    Shaoming Guo and Changkeun Oh and Hong Wang and Shukun Wu and Ruixiang Zhang,
    \emph{The {Bochner}-{Riesz} problem: An Old Approach Revisited},
    2021.

\bibitem{KatzandRogers}
    Nets Hawk Katz and Keith M. Rogers,
    \emph{On the polynomial {Wolff} axioms},
    Geom. Funct. Anal.,
    2018.

\bibitem{BourgainandGuth}
    Jean Bourgain and Larry Guth,
    \emph{Bounds on oscillatory integral operators based on multilinear estimates},
    Geom. Funct. Anal.,
    2011.

\bibitem{GuthTextbook}
    Larry Guth,
    \emph{Polynomial Methods in Combinatorics},
    AMS University Lecture Series,
    2016.

\bibitem{Zahl}
    Joshua Zahl,
    \emph{Sphere tangencies, line incidences, and {Lie's} line-sphere correspondence},
    2021.

\bibitem{GuthandWangandZhang}
    Larry Guth and Hong Wang and Ruixiang Zhang,
    \emph{A Sharp Square Function Estimates for the Cone in $\RR^3$},
    2020.

\bibitem{Sogge}
    Christopher Sogge,
    \emph{Oscillatory Integrals and Spherical Harmonics},
    1985.

\end{thebibliography}

\end{document}