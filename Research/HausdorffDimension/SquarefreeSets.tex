\documentclass{report}

\usepackage{amsthm}
\usepackage{amsmath}
\usepackage{multicol}
\usepackage{amssymb}
\usepackage{mathabx}

\theoremstyle{plain}
\newtheorem{lemma}{Lemma}
\newtheorem{prop}{Proposition}
\newtheorem*{example}{Example}
\newtheorem*{fact}{Fact}
\newtheorem*{corollary}{Corollary}

\usepackage{algorithm}
\usepackage[noend]{algpseudocode}

\theoremstyle{plain}
\newtheorem{theorem}{Theorem}
\newtheorem{proposition}[theorem]{Proposition}
\newtheorem*{remark}{Remark}

\title{Squarefree Sets}
\author{Jacob Denson}

\begin{document}

\maketitle

\tableofcontents

\newpage

\chapter{Background}

\section{Rusza: Difference Sets Without Squares}

In this section, we describe the work of Ruzsa on the discrete squarefree difference problem, which provides inspiration for our speculated results for the squarefree subset problem in the continuous setting. If $X$ and $Y$ are subsets of integers, we shall let
%
\[ X \pm Y = \{ x \pm y: x \in X, y \in Y, x \pm y > 0 \} \]
%
denote the sums and difference of the set. The {\it differences} of a set $X$ are elements of $X - X$, and so the squarefree difference set problem asks to consider how large a subset of the integers can be, whose differences do not contain the square of any positive integer. We let $D(N)$ denote the maximum number of integers which can be selected from $[1,N]$ whose differences do not contain a square.

\begin{example}
    The set $X = \{ 1, 3, 6, 8 \}$ is squarefree, because $X - X = \{ 2, 3, 5, 7 \}$, and none of these elements are perfect squares. On the other hand, $\{ 1, 3, 5 \}$ is not a squarefree subset, because $5 - 1 = 4$ is a perfect square.
\end{example}

There are a few tricks to constructing large subsets of integers avoiding squares. If $p$ is prime, then $p \mathbf{Z} \cap [1,p^2)$ avoids squares, because the difference of two numbers must be divisible by $p$, but not by $p^2$. If $n = p^2$, this gives a set with $n^{1/2}$ elements. However, we can do just as well without using any properties of the set of squares except for their sparsity, by greedily applying a sieve. We start by writing out a large list of integers $1,2,3,4,\dots,N$. Then, while we still have numbers to pick, we greedily select the smallest number $x_*$ we haven't crossed out of the list, add it to our set $X$ of squarefree numbers, and then cross out all integers $y$ such that $y - x_*$ is a positive square. Thus we cross out $x_*$, $x_* + 1$, $x_* + 4$, and so on, all the way up to $x_* + m^2$, where $m$ is the largest integer with $x_* + m^2 \leq N$. This implies $m \leq \sqrt{N - x_*} \leq \sqrt{N-1}$, hence we cross out at most $\sqrt{N-1} + 1$ integers whenever with add a new element $x_*$ to $X$. When the algorithm terminates, all integers must be crossed out, and if the algorithm runs $n$ iterations, a union bound gives that we cross out at most $n[\sqrt{N-1} + 1]$ integers, hence $n[\sqrt{N-1} + 1] \geq N$. It follows that the set $X$ we end up with satisfies
%
\[ |X| \geq \frac{N}{\sqrt{N-1} + 1} = \Omega \left( \sqrt{N} \right) \]
%
What's more, this algorithm generates an increasing family of squarefree subsets of the integers as $n$ increases, so we may take the union of these subsets over all $N$ to find an infinite squarefree subset $X$ with $|X \cap [1,N]| = \Omega(\sqrt{N})$ for all $N$.

In 1978, S\'{a}rk\"{o}zy proved an upper bound on the size of squarefree subsets of the integers, showing $D(N) = O(N (\log N)^{-1/3 + \varepsilon})$ for every $\varepsilon > 0$. In particular, this proves a conjecture of Lov\'{a}sz that every infinite squarefree subset has density zero, because if $X$ is any infinite squarefree subset, then
%
\[ \frac{|X \cap [1,N]|}{N} \leq \frac{D(N)}{N} = O(\log(N)^{-1/3 + \varepsilon}) = o(1) \]
%
S\'{a}rk\"{o}zy even conjectured that $D(N) = O(N^{1/2 + \varepsilon})$ for all $\varepsilon > 0$. Thus the sieve technique is essentially optimal, an incredibly pessimistic point of view, since the Sieve method doesn't depend on any properties of the set of perfect squares. Ruzsa's results shows we should be more optimistic, taking advantage of the digit expansion of numbers to obtain infinite squarefree subsets $X$ with $|X \cap [1,N]| = \Omega(N^{0.73})$. The method reduces the problem to a finitary problem of maximizing squarefree subsets modulo a squarefree integer $m$.

\begin{theorem}
    If $m$ is a squarefree integer, then
    %
    \[ D(N) \geq \frac{n^{\gamma_m}}{m} = \Omega_m(n^{\gamma_m}) \]
    %
    where
    %
    \[ \gamma_m = \frac{1}{2} + \frac{\log_m |R^*|}{m} \]
    %
    and $R^*$ denotes the maximal subset of $[1,m]$ whose differences contain no squares modulo $m$. Setting $m = 65$ gives
    %
    \[ \gamma_m = \frac{1}{2} \left( 1 + \frac{\log 7}{\log 65} \right) = 0.733077 \dots \]
    %
    and therefore $D(N) = \Omega(n^{0.7})$. For $m = 2$, we find $D(N) \geq \sqrt{N}/2$, which is only slightly worse than the sieve result.
\end{theorem}

\begin{remark}
    Let us look at the analysis of the sieve method backwards. Rather than fixing $N$ and trying to find optimal solutions of $[1,N]$, let's fix a particular strategy (to start with, the sieve strategy), and think of varying $N$ and seeing how the size of the solution given by the strategy on $[1,N]$ increases over time. In our analysis, the size of a solution is directly related to the number of iterations the stategy can produce before it runs out of integers to add to a solution set. Because we apply a union bound in our analysis, the cost of each particular new iteration is the same as the cost of the other iterations. If the cost of each iteration was independant of $N$, we could increase the solution size by increasing $N$ by a fixed constant, leading to family of solutions which increases on the order of $N$. However, as we increase $N$, the cost of each iteration increases on the order of $\sqrt{N}$, leading to us only being able to perform $N/\sqrt{N} = \sqrt{N}$ iterations for a fixed $N$. Rusza's method applies the properties of the perfect squares to perform a similar method of expansion. At an exponential cost, Rusza's method increases the solution size exponentially. The advantage of exponentials is that, since Rusza's is based on a particular parameter, a squarefree integer $m$, we can vary $m$ to improve the iteration numbers more naturally.
\end{remark}

The idea of Rusza's construction is to break the problem into exponentially large intervals, upon which we can solve the problem modulo an integer. More enerally, Rusza constructs a set whose differences are free of $d$'th powers.

\begin{theorem}
    Let $R \subset [1,m]$ be a subset of integers such that no difference is a power of $d$ modulo $m$, where $m$ is a {\it squarefree integer}. Construct the set
    %
    \[ A = \left\{ \sum_{k = 0}^n r_k m^k : 0 \leq n < \infty, r_k \in \left. \begin{cases} R & d\ \text{divides}\ N\\ [1,m] & \text{otherwise} \end{cases} \right\} \right\} \]
    %
    Then $A$ is squarefree.
\end{theorem}
\begin{proof}
    Suppose that we can write $\sum (r_k - r_k') m^k = N^d$. Let $s$ to be the smallest index with $r_s \neq r_s'$. Then $(r_s - r_s') m^s + M m^{s+1} = N^d$ where $M$ is some positive integer. If $s = ds_0$, then $(N/m^{s_0})^d = (r_s - r_s') + M m$, and this contradicts the fact that $r_s - r_s'$ cannot be a $d$'th power modulo $m$. On the other hand, we know $m^s$ divides $N^d$, but $m^{s+1}$ does not. This is impossible if $s$ is not divisible by $d$, because primes in $N^d$ occur in multiples of $d$, and $m$ is squarefree.
\end{proof}

For any $n$, we find
%
\[ A \cap [1,m^n - 1] = \left\{ \sum_{k = 0}^{n-1} r_km^k : r_k \in [1,m], r_k \in R\ \text{when $d$ divides $k$} \right\} \]
%
which therefore has cardinality
%
\begin{align*}
    |R|^{1 + \lfloor \frac{n-1}{d} \rfloor} m^{n-1- \lfloor \frac{n-1}{d} \rfloor} = m^n \left( \frac{|R|}{m} \right)^{1 + \lfloor \frac{n-1}{d} \rfloor} \geq m^n \left( \frac{|R|}{m} \right)^{n/d} = m^{n \gamma(m,d)}
\end{align*}
%
where $\gamma(m,d) = 1 - 1/d + \log_m |R|/d$. Therefore, for $m^{n+1} - 1 \geq k \geq m^n - 1$
%
\[ A \cap [1,k] \geq A \cap [1,m^n] \geq m^{n \gamma(m,d)} = \frac{m^{(n+1) \gamma(m,d)}}{m} \geq \frac{k^{\gamma(m,d)}}{m} \]
%
This completes Rusza's construction. Thus we have proved a more general result than was required.

\begin{theorem}
    For every $d$ and squarefree integer $m$, we can construct a set $X$ whose differences contain no $d$th powers and
    %
    \[ |X \cap [1,n]| \geq \frac{n^{\gamma(d,m)}}{m} = \Omega(n^{\gamma(d,m)}) \]
    %
    where $\gamma(d,m) = 1 - 1/d + \log_m |R^*|/d$, and $R^*$ is the largest subset of $[1,m]$ containing no $d$'th powers modulo $m$.
\end{theorem}

For $m = 65$, the group $\mathbf{Z}_{65}^* \cong \mathbf{Z}_{5}^* \times \mathbf{Z}_{13}^*$ has a set of squarefree residues of the form $\{ (0,0), (0,2), (1,8), (2,1), (2,3), (3,9), (4,7) \}$, which gives the required value for $\gamma_{65}$. In 2016, Mikhail Gabdullin proved that if $m$ is squarefree, then in $\mathbf{Z}_m$, any set $R$ such that $R - R$ is squarefree has $|R| \leq me^{-c \log m / \log \log m}$, where $n$ denotes the number of odd prime divisors of $m$, so that
%
\[ \gamma(d,m) \leq 1 - 1/d + \frac{\log(me^{-c \log m / \log \log m})}{m} \]

Rusza believes that we cannot choose $m$ to construct squarefree subsets of the integers growing better than $\Omega(n^{3/4})$, and he claims to have proved this assuming $m$ is squarefree and consists only of primes congruent to 1 modulo 4. Looking at some sophisticated papers in number theory (Though I forgot to write down the particular references), it seems that using modern estimates this is quite easy to prove. Thus expanding on Rusza's result in the discrete case requires a new strategy, or perhaps Rusza's result is the best possible.

Let $D(N,d)$ denote the largest subset of $[1,N]$ containing no $d$th powers of positive integer. The last part of Rusza's paper is devoted to lower bounding the polynomial growth of $D(N,d)$ over asymptotically with respect to $N$. Rusza proves

\begin{theorem}
    If $p$ is the least prime congruent to one modulo $2d$, then
    %
    \[ \limsup_{N \to \infty} \frac{\log D(N,d)}{\log N} \geq 1 - \frac{1}{d} + \frac{\log_p d}{d} \]
\end{theorem}
\begin{proof}
    The set $X$ we constructed in the last theorem shows that for any $m$,
    %
    \[ \frac{\log D(N,d)}{\log n} \geq \gamma(d,m) - \frac{\log m}{\log n} = 1 - \frac{1}{d} + \frac{\log_m |R^*|}{d} - \frac{\log m}{\log n} \]
    %
    Hence
    %
    \[ \limsup_{N \to \infty} \frac{\log D(N,d)}{\log n} \geq 1 - \frac{1}{d} + \frac{\log_m |R^*|}{d} \]
    %
    The claim is then completed by the following lemma.
\end{proof}

\begin{lemma}
    If $p$ is a prime congruent to $1$ modulo $2d$, then we can construct a set $R \subset [1,p]$ whose differences do not contain a $d$th power modulo $p$ with $|R| \geq d$.
\end{lemma}
\begin{proof}
    Let $Q \subset [1,p]$ be the set of powers $1^k, 2^k, \dots, p^k$ modulo $p$. We have
    %
    \[ |Q| = \frac{p-1}{k} + 1 \]
    %
    This follows because the nonzero elements of $Q$ are the images of the group homomorphism $x \mapsto x^k$ from $\mathbf{Z}_p^*$ to itself. Since $\mathbf{Z}_p^*$ is cyclic, the equation $x^k = 1$ has the same number of solutions as the equation $kx = 0$ modulo $p-1$, and since $p \equiv 1$ modulo $2k$, there are exactly $k$ solutions to this equation. The sieve method yields a $k$th power modulo $p$ free subset of size greater than or equal to
    %
    \[ p/q = \frac{p}{1 + \frac{p-1}{k}} = \frac{pk}{p + k - 1} \to k \]
    %
    as $p \to \infty$, which is greater than $k-1$ for large enough $p$ (this shows the theorem is essentially trivial for large enough primes, because we don't need to use any particularly interesting properties of the squares to prove the theorem). However, for smaller primes a more robust analysis is required. We shall construct a sequence $b_1, \dots, b_k \in \mathbf{Z}_p$ such that $b_i - b_j \not \in Q$ for any $i,j$ and
    %
    \[ |B_j + Q| \leq 1 + j(q-1) \]
    %
    Given $b_1, \dots, b_j$, let $b_{j+1}$ be any element of
    %
    \[ (B_j + Q + Q) - (B_j + Q) \]
    %
    Since $b_{j+1} \not \in B_j + Q$, $b_{j+1} - b_i \not \in Q$ for any $i$. Since $b_{j+1} \in B_j + Q + Q$, the sets $B_j + Q$ and $b_{j+1} + Q$ are not disjoint (note $Q = -Q$ because $p \equiv 1$ mod $2k$), and so
    %
    \begin{align*}
        |B_{j+1} + Q| &= |(B_j + Q) \cup (b_{j+1} + Q)|\\
        &\leq |B_j + Q| + |b_{j+1} + Q| - 1\\
        &\leq 1 + j(q-1) + q - 1\\
        &= 1 + (j+1)(q-1)
    \end{align*}
    %
    This procedure ends when $B_j + Q + Q = B_j + Q$, and this can only happen if $B_j + Q = \mathbf{Z}_p$, because we can obtain all integers by adding elements of $Q$ recursively, so $1 + j(q-1) \geq p$, and thus $j \geq k$.
\end{proof}

\begin{corollary}
    In the special case of avoiding squarefree numbers, we find 
    \[ \limsup \frac{\log D(N)}{\log N} \geq \frac{1}{2} + \frac{\log_5 2}{2} = 0.71533\dots \]
    %
    which is only slightly worse than the bound we obtain with $m = 65$.
\end{corollary}

Rusza's leaves the ultimate question of whether one can calculate
%
\[ \alpha = \lim_{N \to \infty} \log D(N) / \log N \]
%
or even whether it exists at all. The consequence of this would essentially solve the squarefree integers problem, since it would give the exact growth of $D(N) \sim N^\alpha$ in terms of a monomial. Because of how conclusive this problem is, we should not expect to find a nontrivial way to calculate this constant.











\section{Keleti: Translate Avoiding Sets}

Keleti's two page paper constructs a full dimensional subset $X$ of $[0,1]$ such that $X$ intersects $t + X$ in at most one place for each $t \in \mathbf{R}$. Malabika has adapted this technique to construct high dimensional subsets avoiding nontrivial solutions to differentiable functions, and she thinks we can further exploit these ideas to obtain dimension one squarefree sets. The basic, but fundamental idea of Keleti is to introduce memory into Cantor set type constructions so the sets avoid progressions, and have dimension one. At each point in the process, he constructs a nested family of sets $X_0 \supset X_1 \supset \dots$, each set $X_n$ a union of disjoint intervals of the same length $l_n$. It will be the set $X = \lim X_n$ which avoids translates. Initially, $X_1 = [0,1]$ is the unit interval. To aid in this process, Keletti considers a queue $Q$ with a history of all intervals created from the start to the end of the process, so $Q$ initially just contains $[0,1]$. We then perform the following procedure repeatedly:
%
\begin{itemize}
    \item Take an interval $I$ off the front of the queue $Q$.

    \item Let $X_{n+1}$ be formed by taking $N_{n+1}$ intervals of length $l_{n+1}$ from each interval $J$ in $X_n$, with the startpoints of each interval separated by $\varepsilon_{n+1}$. The intervals start directly at the beginning of the interval $J$, except in the case where $J \subset I$, where we shift the intervals to begin at a distance $\Delta_{n+1}$ from the startpoint.
\end{itemize}
%
In order for this process to be well defined, we must have $l_1 = N_1 = 1$, $\varepsilon_n \geq l_n$, and $\Delta_{n+1} + N_{n+1} \varepsilon_{n+1} \leq l_n$. Of course, we must also have $l_n \to 0$, for otherwise we have no fractal structure in our sets. We claim that, with the appropriate choice of parameters, $X$ is a set avoiding translates. In this section, and in the sequel, we shall find it is most convenient to avoid certain configurations by expressing them as a particular equation, whose properties we can then exploit.

\begin{lemma}
    A set $X$ avoids translates if and only if there do not exists values $x_1 < x_2 \leq x_3 < x_4$ in $X$ with $x_2 - x_1 = x_4 - x_3$.
\end{lemma}
\begin{proof}

    Suppose $t + X \cap X$ contains two points $a < b$. Without loss of generality, we may assume that $t > 0$. If $a \leq b - t$, then the equation
    %
    \[ a - (a - t) = t = b - (b - t) \]
    %
    satisfies the constraints, since then $a - t < a \leq b - t < b$ are all elements of $X$. We also have
    %
    \[ (b - t) - (a - t) = b - a \]
    %
    which satisfies the constraints if $a - t < b - t \leq a < b$. This covers all possible cases. Conversely, if there are $x_1 < x_2 \leq x_3 < x_4$ in $X$ with $x_2 - x_1 = t = x_4 - x_3$, then $X + t$ contains $x_2 = x_1 + (x_2 - x_1)$ and $x_4 = x_3 + (x_4 - x_3)$.
\end{proof}

We now claim that, if the constants $l_n, N_n, \varepsilon_n$, and $\Delta_n$ possess suitable properties, then the set $X$ is translation avoiding. We suppose the existence of $x_2 - x_1 = x_4 - x_3$ as in the lemma, with $x_1 < x_2 \leq x_3 < x_4$. Since $l_n \to 0$, we know that eventually $x_1$ is contained in an interval $J$ that $x_2,x_3$, and $x_4$ are not contained in. If we follow the procedure to a suitable depth $N$, when the interval $J$ is taken off the front of the queue $Q$, the intervals containing $x_1$ are shifted, whereas the intervals for $x_2,x_3$, and $x_4$ are not shifted. To understand what this means, we find the startpoints $x_1^\circ, x_2^\circ$, $x_3^\circ$, and $x_4^\circ$ to the length $l_N$ intervals containing $x_1$, $x_2$, $x_3$, and $x_4$. Thus $0 \leq x_n - x_n^\circ \leq l_N$. If we choose our parameters such that the startpoints $x_1^\circ, \dots, x_4^\circ$ all lie at integer multiples of $\varepsilon_N$, and $4l_N < \varepsilon_N$, then the equation $x_2 - x_1 = x_4 - x_3$ forces the discrete equation $x_2^\circ - x_1^\circ = x_4^\circ - x_3^\circ$. Now if we also force $x_2^\circ, x_3^\circ, x_4^\circ$ to lie at even multiples of $\Delta_N$, and $x_1^\circ$ to lie at an odd multiple of $\Delta_N$, which is possible by the shifting in the construction, then it is impossible for this equation to hold, so by contradiction, $X$ must be translation invariant.

If we want to determine what makes the startpoints of the intervals in $X_n$ lie at the appropriate multiples of $\Delta_n$, we may assume by induction that the startpoints of $X_n$ lie at integer multiples of $\Delta_n$, and we now try and make the startpoints of $X_{n+1}$ lie at integer multiples of $\Delta_{n+1}$. If $I \subset J$ are two intervals, with $J \subset X_n$ and $I \subset X_{n+1}$, then $J$ starts at some point $A \Delta_n$, and either $I$ is unshifted, and starts at a point $A \Delta_n + B \varepsilon_{n+1}$, or it shifted, and is of the form $A \Delta_n + B \varepsilon_{n+1} + \Delta_{n+1}$. It is clear that the required constraints hold if $A \Delta_n + B \varepsilon_{n+1}$ is an even multiple of $\Delta_{n+1}$ for all $A$ and $B$, and so we obtain the required results if $2 \Delta_{n+1} \divides \Delta_n, \varepsilon_{n+1}$. 

Thus to summarize, $X$ is well defined and translation invariant provided
%
\[ \Delta_{n+1} + N_{n+1}\varepsilon_{n+1} \leq l_n\ \ \ 4l_n < \varepsilon_n\ \ \ 2\Delta_{n+1} \divides \Delta_n, \varepsilon_{n+1} \]
%
Playing around with these inequalities shows the most elegant choices of parameters for a particular sequence $N_n$ is
%
\[ \Delta_n = l_n\ \ \ \varepsilon_n = 6l_n\ \ \ l_n = \frac{l_{n-1}}{8N_n} \]
%
At the $N$'th scale of the algorithm, we have $\prod N_n$ intervals, each of length $(8^n \prod N_n)^{-1}$. The fact that $\prod N_n / (8^n \prod N_n) = o(1)$ implies that our set will have Lebesgue measure zero. However, if the product $\prod N_n$ grows suitably fast, then the $8^n$ will be overwhelmed, so that for each $\varepsilon$, $(8^n \prod N_n)^{1-\varepsilon} \lesssim_\varepsilon \prod N_n$, i.e. $8^n \lesssim_\varepsilon (\prod N_n)^\varepsilon$ for each $\varepsilon$. This hints at the fact that $H^{1-\varepsilon}(X) = \infty$ for all $\varepsilon > 0$, so $X$ has Hausdorff dimension one. This means setting $N_n = M$ for some constant $M$ is {\it not} sufficient to obtain a Hausdorff dimension one set. Choosing a fast growing sequence, such as $N_n = n$ or $N_n = 2^{n-1}$, suffices to obtain a Hausdorff dimension one set.

\begin{lemma}
    The set $X = \lim X_n$ has Hausdorff dimension one.
\end{lemma}
\begin{proof}
%Recall Frostman's lemma, which says that the $s$ dimensional Hausdorff measure $H_s(X)$ of a Euclidean set $X$ is positive if and only if there is a finite positive Borel measure $\mu$ supported on $X$ with $\mu(B_r(x)) \lesssim r^s$, for a universal constant depending only on $\mu$. If such a measure can be constructed on a set $X$, it therefore follows that $\dim_{\mathbf{H}}(X) \geq s$. Thus to prove $X$ has dimension one, it suffices to construct a probability measure $\mu$ on $X$ with $\mu(B_r(x)) \lesssim_s r^s$, for each $s < 1$. We can construct such a measure using what is often called the {\it mass distribution principle}; we construct a probability measure $\mu_n$ supported on $X_n$ in such a way that a weak limit $\mu = \lim \mu_n$ exists, in which case $\mu$ is supported on $X$. To do this, we let $\mu_1$ be the uniform probability measure on $[0,1]$. Then, to construct $\mu_{n+1}$ from $\mu_n$, we divide the mass of each interval $J$ in $X_n$ uniformly over the intervals in $X_{n+1}$ contained in $J$. The distribution functions of these measures converge uniformly, and therefore the $\mu_n$ converge weakly to a measure $\mu$ supported on $X$.

We use the mass distribution principle, as used in our note on calculating Hausdorff dimensions. If $X = \lim X_N$ is our set, then each $X_N$ consists of $\prod_{n \leq N} N_n$ intervals, and so any interval $I$ at this scale has
%
\[ \mu(I) = \prod_{m = 1}^n \frac{1}{N_m} \]
%
We then calculate
%
\[ \prod_{m = 1}^n \frac{1}{N_m} = \frac{8^{(n-1)(1-\varepsilon)}}{\left( \prod_{m = 1}^n N_m \right)^\varepsilon} l_n^{1-\varepsilon} \]
%
%If $J$ is any interval of length $l_n$, then $\mu(J)$ can intersect at most two intervals of length $l_n$ in $X_n$, and so we obtain the general bound
%
%\[ \mu(J) \leq 2 \prod_{m = 1}^n \frac{1}{N_m} = \frac{2}{s} l_n^{1-\varepsilon} \lesssim l_n^{1-\varepsilon} \]
%
A bound $\mu(I) \lesssim_\varepsilon l_n^{1-\varepsilon}$ is thereby obtained if for every $\varepsilon > 0$,
%
\[ 8^{(n-1)(1-\varepsilon)} \lesssim_\varepsilon \left( \prod_{n = 1}^N N_n \right)^\varepsilon \]
%
This is true, for instance, if $N_n = n$. Since $l_n/l_{n+1} = 8N_{n+1}$, we can apply the note in the appendix to conclude $\mu(I) \lesssim_\varepsilon |I|^{1-\varepsilon}$ for all intervals $I$ assuming that $N_{N+1} \lesssim_\varepsilon \prod_{n \leq N} N_n^\varepsilon$ for every $\varepsilon > 0$. This is true, in particular, if $N_n = n$, which applies because $n \lesssim_\varepsilon (n!)^\varepsilon$, and if $N_n = 2^{n-1}$, in which case we can use the fact that $\smash{2^n \lesssim_\varepsilon 2^{\varepsilon n^2}}$.

%
%We now hope for a bound
%
%\[ \mu(I) \lesssim_\varepsilon |I|^{1-\varepsilon} \]
%If we are to hope for a bound $\mu(I) \lesssim_\varepsilon |I|^{1-\varepsilon}$ for every $\varepsilon$, then there exists a constant $A_\varepsilon$ such that
%
%\[ \prod_{m = 1}^n \frac{1}{N_m} \leq A_\varepsilon l_n^{1-\varepsilon} = \]
%
%In order to satisfy this inequality by induction, we would then have
%
%\[ \prod_{m = 1}^{n+1} \frac{1}{N_m} \leq \frac{A_\varepsilon l_n^{1-\varepsilon}}{N_{n+1}} \leq A_\varepsilon l_{n+1}^{1-\varepsilon} \]
%
%so in order to obtain this inequality, we need $(l_n/l_{n+1})^{1-\varepsilon} \leq N_{n+1}$ for suitably large $n$. Thus the lengths of the intervals cannot become too small compared to the number $N_{n+1}$ intervals we maintain. However, in order to interpolate this bound, and to obtain a bound $\mu(I) \lesssim_\varepsilon |I|^{1-\varepsilon}$ for all intervals $I$, and not just the particular length $l_n$ scales, we shall need a tighter for the measures on intervals at the length $l_n$ scale. Using the constants $N_n = n$, and $l_n = 1/8^{n-1}n!$, which we chose above, we find
%
%\[ \mu(J) \leq \frac{2}{n!} = \frac{2}{n! l_n^{1-\varepsilon}} l_n^{1-\varepsilon} = \frac{2 \cdot 8^{(n-1)(1-\varepsilon)}}{(n!)^\varepsilon} l_n^{1-\varepsilon} \lesssim_\varepsilon \frac{l_n^{1-\varepsilon}}{(n!)^{\varepsilon/2}} \]
%
%To obtain the bound for general intervals $I$, with $l_{n+1} \leq |I| \leq l_n$, we cover $I$ by as few as $|I|l_{n+1}^{-1}$ intervals of length $l_{n+1}$, from which we obtain
%
%\[ \mu(I) \lesssim_\varepsilon |I|l_{n+1}^{-1} \frac{l_{n+1}^{1-\varepsilon}}{(n!)^{\varepsilon/2}} = \left( \frac{|I|}{l_{n+1} (n!)^{1/2}} \right)^\varepsilon |I|^{1-\varepsilon} \leq \left( \frac{l_n/l_{n+1}}{(n!)^{1/2}} \right)^\varepsilon |I|^{1-\varepsilon} \]
%
%and since $l_n/l_{n+1} = O(n) = O((n!)^{1/2})$, we obtain the required bound. Thus we obtain $\dim_{\mathbf{H}}(X) \geq 1 - \varepsilon$, and as $\varepsilon \to 0$, we conclude $\dim_{\mathbf{H}}(X) = 1$.
\end{proof}

%\begin{remark}
%    Here's why we need the tighter bound $\mu(I) \lesssim_\varepsilon |I|^{1-\varepsilon}/(n!)^{\varepsilon/2}$ at the discrete scales to successively interpolate our bounds to all interval scales, rather than just the simpler bound $\mu(I) \lesssim_\varepsilon |I|^{1-\varepsilon}$. If $l_{N+1} \leq |I| \leq l_N$, and we cover $I$ by $|I|l_{N+1}^{-1}$ length $l_{N+1}$ intervals, then we obtain that
    %
%    \[ \mu(I) \lesssim_\varepsilon |I|l_{N+1}^{-1} l_{N+1}^{1-\varepsilon} = |I| l_{N+1}^{-\varepsilon} = \left( \frac{|I|}{l_{N+1}} \right)^\varepsilon |I|^{1-\varepsilon} \]
    %
%    Similarily, if we cover $I$ by a single length $l_N$ interval, then
    %
%    \[ \mu(I) \lesssim_\varepsilon l_N^{1-\varepsilon} = \left( \frac{l_N}{|I|} \right)^{1-\varepsilon} |I|^{1-\varepsilon} \]
    %
%    If we are to hope that these bounds give us a $\lesssim_\varepsilon |I|^{1-\varepsilon}$ bound for all $\varepsilon$, then we must have
    %
%    \[ \max_{l_{N+1} \leq |I| \leq l_N} \min \left( \left( \frac{|I|}{l_{N+1}} \right)^\varepsilon, \left( \frac{l_N}{|I|} \right)^{1-\varepsilon} \right) \lesssim_\varepsilon 1 \]
    %
%    The minimization is maximized when
    %
%    \[ \left( \frac{|I|}{l_{N+1}} \right)^\varepsilon = \left( \frac{l_N}{|I|} \right)^{1-\varepsilon} \]
    %
%    or when $|I| = l_N^{1-\varepsilon} l_{N+1}^\varepsilon$. Inputting this into the formula, we obtain that
    %
%    \[ \max_{l_{N+1} \leq |I| \leq l_N} \min \left( \left( \frac{|I|}{l_{N+1}} \right)^\varepsilon, \left( \frac{l_N}{|I|} \right)^{1-\varepsilon} \right) = \left( \frac{l_N}{l_{N+1}} \right)^{\varepsilon (1 - \varepsilon)} \]
    %
%    With the choice of parameters given, we have $l_N/l_{N+1} = 8(n+1)$, and we do not have $(8(n+1))^{\varepsilon(1-\varepsilon)} \lesssim_\varepsilon 1$. Thus, with the bounds we have used, there is no way to obtain a constant coefficient bound for all scales lying inbetween the discrete scales if we use the $\mu(I) \lesssim_\varepsilon |I|^{1-\varepsilon}$ bound for the discrete scales. However, the tighter bound $\mu(I) \lesssim_\varepsilon |I|^{1-\varepsilon}/(n!)^{\varepsilon/2}$ causes the $O(n)$ term for $l_N/l_{N+1}$ to be annihilated, which results in a constant term bound at the continuous range of scales.
%\end{remark}

\begin{remark}
    In Keleti's paper, he also remarks that by replacing the 8 in the choice of $l_n$ with a slowly increasing set of even numbers, one can obtain a Hausdorff dimension one set which is linearly independant over the rational numbers. To see why this works, the condition of linear independence would fail if $\smash{a_1 x_1 + \dots + a_M x_M = 0}$, where $\smash{x_1 < x_2 < \dots < x_M}$, and $\smash{a_1, \dots, a_M}$ are integers with no common factor. Just as before, we find $x_n^\circ$ with $0 \leq x_n - x_n^\circ \leq l_N$. Provided that $(a_1 + \dots + a_M) l_N < \varepsilon_N$, and the $x_n^\circ$ lie at integer multiples of $\varepsilon_N$, we conclude that $a_1 x_1^\circ + \dots + a_M x_M^\circ = 0$. If $K$ is an integer not dividing $a_1$, then for suitably large $N$ we assume that each $x_2^\circ, \dots, x_M^\circ$ lies at multiples of $K\Delta_N$. Shifting $x_1^\circ$ by a single multiple of $\Delta_N$ then breaks the equation from ever occuring in the first place. In order to guarantee this, we must first set $\varepsilon_n = A_n! l_n$ where $A_n$ is an increasing sequence with $A_n \to \infty$. We also guarantee that $x_2^\circ, \dots, x_M^\circ$ lies at multiples of $A_n! \Delta_n$. This can be guaranteed by induction if $A_{n+1}! \Delta_{n+1} \divides \Delta_n, \varepsilon_{n+1}$. Thus the parameters
    %
    \[ \Delta_n = l_n\ \ \ \varepsilon_n = A_n! l_n\ \ \ l_{n+1} = \frac{l_n}{2N_{n+1}A_{n+1}!} \]
    %
    give a linearly independant set. Assuming the $A_n$ grow incredibly slowly relative to the $N_n$, i.e.
    %
    \[ N_n = n\ \ A_n = \log \log n + O(1)\ \ \ \ N_n = 2^n\ \ A_n = \log n + O(1)\ \ \ \ N_n = 2^{n^2}\ \ A_n = n \]
    %
    then we obtain a set with Hausdorff dimension one.
%Assuming the $A_n$ grow incredibly slowly, we can still hope for this set to have Hausdorff dimension one. fI we construct the probability measure $\mu$ as before, we find that for any length $l_n$ interval $J$
    %
%    \[ \mu(J) \leq \frac{2}{n!} = \frac{2}{n!l_n^{1-\varepsilon}} l_n^{1-\varepsilon} = \left( \frac{2}{(n!)^\varepsilon} \left( \prod A_m! \right)^{1-\varepsilon} \right) l_n^{1-\varepsilon} \]
    %
%    We can choose the $A_m$ to grow slowly enough that for any $\varepsilon > 0$,
    %
%    \[ \left( \prod A_m! \right)^{1-\varepsilon} \lesssim_\varepsilon (n!)^{\varepsilon/2} \]
    %
%    Testing this inequality leads to the fact that $A_{n+1}! \leq (n+1)^{\varepsilon/2(1-\varepsilon)}$ must eventually hold for $n$ large enough, so taking $A_n \to \infty$ but growing slower than any polynomial in $n$ satisfies the inequality, i.e. if $A_n!$ is the largest factorial smaller than $\log n$. Thus
    %
%    \[ \mu(J) \lesssim_\varepsilon \frac{l_n^{1-\varepsilon}}{(n!)^{\varepsilon/2}} \]
    %
%    and the interpolation bound as in the previous problem then guarantee $\mu(I) \lesssim_\varepsilon |I|^{1-\varepsilon}$ for all $I$, since $l_n/l_{n+1} = O(nA_n!) = O((n!)^{1/2})$.
\end{remark}

\begin{remark}
    We attempted to obtain a squarefree subset of $[0,1]$ by combining Ruzsa's squarefree discrete strategy with Keleti's decomposition approach to find a high dimensional continuous squarefree set. However, using these techniques we were only able to obtain a dimension 1/2 set, which is only slightly better than a dimension 1/3 set which exists from the general results given by Math\'{e}'s result, or Pramanik and Fraser's result, and is much less than the dimension 1 set that Malabika expects.
\end{remark}







\section{Fraser/Pramanik: Extending Keleti Translation to Smooth Configurations}

Inspired by Keleti's result, Pramanik and Fraser obtained a generalization of the queue method which allows one to find sets avoiding solutions to {\it any} smooth function satisfying suitably mild regularity conditions. To do this, rather than making a linear shift in one of the intervals we avoid as in Keleti's approach, one must use the smoothness properties of the function to find large segments of an interval avoiding solutions to another interval.

\begin{theorem}
    Suppose that $f: \mathbf{R}^{d+1} \to \mathbf{R}$ is a $C^1$ function, and there are sets $T_0, \dots, T_d \subset [0,1]$, with each $T_n$ a union of almost disjoint closed intervals of length $1/M$ such that $A \leq |\partial_0 f|$ and $|\nabla f| \leq B$ on $T_0 \times \dots \times T_d$. There there exists a rational constant $C$ and arbitrarily large integers $N \in M \mathbf{Z}$ for which there exist subsets $S_n \subset T_n$ such that
    %
    \begin{itemize}
        \item[(i)] $f(x) \neq 0$ for $x \in S_0 \times \dots \times S_d$.

        \item[(ii)] For $n \neq 0$, if we divide each interval $T_n$ into length $1/N$ intervals, then $S_n$ contains an interval of length $C/N^d$ of each of these intervals.

        \item[(iii)] If $T_0$ is split into length $1/N$ intervals, then for a fraction $1 - 1/M$ of such intervals, $S_0$ is a union of length $C/N^d$ intervals with total length $C/N$.
    \end{itemize}
\end{theorem}
\begin{proof}
    We begin by dividing the sets $T_1, \dots, T_d$ into length $1/N$ intervals, and let $S_n$ be defined by including a length $C_0/N^d$ segment, for some constant $C_0$ to be chosen later. Then once we fix $C_0$, the $S_n$ will satisfy property (ii) of the theorem. We define
    %
    \[ \mathbf{A} = \{ a \in \mathbf{R}^{d-1} : a_n\ \text{is a startpoint of a length $1/N$ interval in}\ T_n \} \]
    %
    Then $|\mathbf{A}| \leq N^d$, since each interval $T_n$ is contained in $[0,1]$, and therefore can only contain at most $N$ almost disjoint intervals of length $1/N$. Hence if we define the set of `bad points' in $T_0$ as
    %
    \[ \mathbf{B} = \{ x \in T_0: \text{there is}\ a \in \mathbf{A}\ \text{such that}\ f(x,a) = 0 \} \]
    %
    Then $|\mathbf{B}| \leq MN^d$. This is because for each fixed $a$, the function $x \mapsto f(x,a)$ is either strictly increasing or decreasing over each interval in the decomposition of $T_0$, or which there are at most $M$ because $T_0 \subset [0,1]$. If we split $T_0$ into length $1/N$ intervals, and choose a subcollection of such intervals $I$ such that $|I \cap \mathbf{B}| \leq M^3N^{d-1}$, then we throw away at most $MN^d/M^3N^{d-1} = N/M^2$ intervals, and so we keep $(N/M)(1 - 1/M)$ intervals, which is $1 - 1/M$ of the total number of intervals in the decomposition of $T_0$. The lemma we prove after this theorem implies that there exists a constant $C_1$ such that if $x \in S_n$, and $f(y,x) = 0$, then $d(y,\mathbf{B}) \leq C_0C_1/N^d$. If we split each interval $I$ with $|I \cap \mathbf{B}| \leq M^3N^{d-1}$ into $4M^3N^{d-1}$ length $1/4M^3N^d$ intervals, and we choose $C_0$ such that $C_0C_1 < 1/4M^3$, then the set $S_0$ obtained by discarding each interval that contains or is adjacent to an interval containing an element of $\mathbf{B}$ satisfies $d(S_0,\mathbf{B}) > C_0C_1/N^d$, and therefore there does not exist any $x_n \in S_n$ and $y \in S_0$ such that $f(y,x) = 0$. $S_0$ satisfies property (iii) of the theorem since for the interval $I$ we are considering, we keep at least $M^3N^{d-1}$ length $1/4M^3N^d$ intervals, which in total has length at least $1/4N$.
\end{proof}

\begin{remark}
    Suppose our equation is $f(x_0, Tx_1) = 0$, where $T$ is a rational matrix of rank $k$, and $x_0$ is a scalar. Then we can actually obtain that $|\mathbf{B}| \lesssim |\mathbf{A}| \lesssim N^k$, since the points of $\mathbf{A}$ are contained in a rank $k$ lattice. Then the same lemma implies that we can actually build intervals of size $1/N^k$, leading to a dimension $1/k$ set rather than a dimension $1/(d-1)$ set (these dimensions agree if $T$ has full rank). If our equation is $f(x_0, g_1(x_1), \dots, g_k(x_1)) = 0$, where each $g_n$ is a rational polynomial with rank at most $m$, the same techniques should show that we get a dimension $1/mk$ set.

    Is this true for equations of the form $f(g_1(x), \dots, g_k(x))$ as well? If we define $\mathbf{B}$ as the set of bad points, and if $f(g_1(x_1,x_2), \dots, g_k(x_1,x_2)) = 0$. This can only be true if the zero set of $f$ has some special structure with respect to the polynomials $g_k$.
\end{remark}

\begin{remark}
    The length $1/N$ portion of each interval guaranteed by (iii) is unneccesary to the Hausdorff dimension bound, since the slightly better bounds obtained on scales where an interval is dissected as a $1/N$ are decimated when we eventually divide the further subintervals into $1/N^{d-1}$ intervals. The importance of (iii) is that it implies that the set we will construct has full {\it Minkowski dimension}. The reason for this is that Minkowski dimension lacks the ability to look at varying dissection depths at once, and since, at any particular depth, there exists a length $1/N$ dissection, the process appears to Minkowski to be full dimensional, even though at later scales this $1/N$ dissection is dissected into $1/N^{d-1}$ intervals.
\end{remark}

\begin{lemma}
    Given the $f$, $T_0, \dots, T_d$, there exists a constant $C_1$ depending on these quantities, such that for any $C_0$, and $x \in S_1 \times \dots \times S_{d-1}$, if $f(y,x) = 0$, then $d(y, \mathbf{B}) \leq C_0C_1/N^{d-1}$.
\end{lemma}
\begin{proof}
    Since $T_0 \times \dots \times T_d$ breaks into finitely many cubes with sidelengths $1/M$, it suffices to prove the theorem for a particular cube $J$ in this decomposition, where we assume the zeroset of $f$ intersects $J$. If $J = I \times J'$, where $I$ is an interval, we let $U$ be the set of all $x \in J'$ for which there is $y$ in the interior of $I$ such that $f(y,x) = 0$. Then $U$ is open. The implicit function theorem implies that there exists a $C^1$ function $g: U \to I$ such that $f(x,y) = 0$ if and only if $y = g(x)$. Then the function $h(x) = f(x,g(x))$ vanishes uniformly, so
    %
    \[ 0 = \partial_n h(x) = (\partial_n f) (g(x),x) + (\partial_0 f) (g(x),x) \partial_n g(x) \]
    %
    Hence for $x \in U$,
    %
    \[ |(\nabla g)(x)| = \frac{|(\nabla f)(x)|}{|(\partial_d f)(x,g(x))|} \leq \frac{B}{A} \]
    %
    If $N$ is chosen large enough, then for every $x \in U \cap (S_1 \times \dots \times S_d)$ there is $a \in \mathbf{A} \cap U$ in the same connected component of $U$ as $x$ with $|x - a| \lesssim C_0/N^{d-1}$, and this means that
    %
    \[ |g(x) - g(a)| \leq \| \nabla g \|_\infty |x - a| \lesssim \frac{BC_0}{A N^{d-1}} \]
    %
    and $g(a) \in \mathbf{B}$, completing the proof.
\end{proof}

How do we use this lemma to construct a set avoiding solutions to $f$? We form an infinite queue which will eventually filter out all the possible zeroes of the equation. Divide the interval $[0,1]$ into $d$ intervals, and consider all orderings of $d - 1$ subsets of these intervals, and add them to the queue. Now on each iteration $N$ of the algorithm, we have a set $X_N \subset [0,1]$. We take a particular sequence of intervals $T_1, \dots, T_d$ from the queue, and then use the lemma above to dissect the $X_N \cap T_n$, which are unions of intervals, into sets avoiding solutions to the equation, and describe the remaining points as $X_{N+1}$. We then add all possible orderings of $d$ intervals created into the end of the queue, and rinse and repeat. The set $X = \lim X_n$ then avoids all solutions to the equation with distinct inputs.

What remains is to bound the Hausdorff dimension of $X$ by constructing a probability measure supported on $X$ with suitable decay. To construct our probability measure, we begin with a uniform measure on the interval, and then, whenever our interval is refined, we uniformly distribute the volume on that particular interval uniformly over the new refinement. Let $\mu$ denote the weak limit of this sequence of probability distributions. At each step $n$ of the process, we let $1/M_n$ denote the size of the intervals at the beginning of the $n$'th subdivision, $1/N_n$ denote the size of the split intervals in the lemma, and $C_n$ the $n$'th constant. We have the relation $1/M_{n+1} = C_n/N_n^{d-1}$. If $K$ is a length $1/M_{N+1}$ interval, $J$ a length $1/N_N$ interval, and $I$ a length $1/M_N$ interval with $K \subset J \subset I$ and all recieving some mass in $\mu$. To calculate a bound on their mass, we consider the decompositions considered in the algorithm:
%
\begin{itemize}
        \item If $J$ is subdivided in the non-specialized manner, then every length $1/N_N$ interval recieves the same mass, which is allocated to a single length $1/M_{N+1}$ interval it contains. Thus $\mu(K) = \mu(J) \leq (M_N/N_N) \mu(I)$.
        \item In the second case, at least a fraction $1 - 1/M_N$ of the length $1/N_N$ intervals are assigned mass, so $\mu(J) = (M_N/N_N)(1 - 1/M_N)^{-1} \leq (2M_N/N_N) \mu(I)$, and more than $C_N/N_N$ of each length $1/N_N$ interval is maintained, so
        %
        \[ \mu(K) = \frac{N_N}{C_NM_{N+1}} \mu(J) \leq \frac{2M_N}{C_NM_{N+1}} \mu(I) \]
\end{itemize}
%
Thus in both cases, we have $\mu(J) \lesssim (N_N/M_N) \mu(I)$, $\mu(K) \lesssim_N |K|$, and $N_N = M_{N+1}^{1/(d-1)}/C_N \lesssim_N M_{N+1}^{1/(d-1)}$. From this, we conclude using the results of the appendix that there exists a family of rapidly decaying parameters which gives a $1/(d-1)$ dimensional set.

\begin{remark}
    The set constructed is precisely a $1/(d-1)$ dimensional set. TODO: INCLUDE PROOF OF THIS FROM NOTEBOOK.
\end{remark}

%Thus, in both cases, we have $\mu(J) \lesssim (N_N/M_N) \mu(I)$, which means we can apply the second method of appendix to calculate Hausdorff dimension with rapidly growing constants, where $l_N = 1/M_N$ and $r_N = 1/N_N$. We have $\mu(K) \lesssim_N $ and $N_N = M_{N+1}^{1/(d-1)}/C_N$ and


%
%Thus, in both cases, we have $\mu(J) \lesssim_N 1/M_{N+1}$. If $J \subset I$ is any length $1/N_N$ interval considered in the algorithm, then either $\mu(J) = (M_N/N_N) \mu(I)$, as in the first case of the subdivision, or we can apply the second case of the subdivision, giving $\mu(J) = (M_N/N_N)(1 - 1/M_N)^{-1} \mu(I) \leq (2M_N/N_N) \mu(I)$. This means we can apply the second method in the appendix. The fact that 

%by induction, if $I$ is a length $1/M_N$ interval considered in the process, then
%
%\begin{align*}
%    \mu(I) \leq \prod_{n < N} \frac{M_n}{(C_n M_{n+1})^{\frac{1}{d-1}}} = \left( \prod_{n < N} \frac{M_{n+1}^{1-\frac{1}{d-1} }}{C_n^{\frac{1}{d-1}}} \right) \frac{1}{M_N} = \frac{A_N}{M_N}
%\end{align*}
%
%If $J \subset I$ is any length $1/N_N$ interval considered in the algorithm, then either $\mu(J) = (M_N/N_N) \mu(I)$, as in the first case, or in the second case, $\mu(J) = (M_N/N_N(1 - 1/M_N)) \leq 2M_N/N_N \mu(I)$, so in general $\mu(J) \leq 2A_N/N_N$. This means we can apply the second method in the appendix for bounding Hausdorff dimension, with $l_N = 1/M_N$ and $r_N = 1/N_N$. To obtain 

%Now if $1/N_N \leq |I| \leq 1/M_N$, then $I$ can be covered by $|I|N_N$ intervals of length $1/N_N$, and so
%
%\begin{align*}
%    \mu(I) &\leq 2|I|N_N \frac{A_N}{N_N} = 2A_N|I| = \frac{2A_{N-1} M_N^{1 - \frac{1}{d-1}}}{C_{N-1}^{\frac{1}{d-1}}} |I| \lesssim_\varepsilon |I|M_N^{1 - \frac{1}{d-1} - \varepsilon} \leq |I|^{\frac{1}{d-1} - \varepsilon}
%\end{align*}
%
%Provided that we can choose $M_N$ such that $A_N/C_N \lesssim_\varepsilon M_{N+1}^\varepsilon$ for all $\varepsilon$ (this is why it is incredibly important that the values in the lemma are independent of $N$ in the proof above). On the other hand, if $1/M_{N+1} \leq |I| \leq 1/N_N$, then $I$ can be covered by a single length $1/N_N$ interval, hence
%
%\[ \mu(I) \leq \frac{2A_N}{N_N} = \frac{2A_N}{N_N} = \frac{2A_N}{(C_NM_{N+1})^{\frac{1}{d-1}}} \lesssim_\varepsilon \frac{1}{M_{N+1}^{\frac{1}{d-1} - \varepsilon}} \leq |I|^{\frac{1}{d-1} - \varepsilon} \]
%
%Thus we obtain the theorem if $M_{N+1} = \exp(A_N/C_N)$, for instance.

\section{A Set Avoiding All Functions With A Common Derivative}

In the latter part's of their paper, Pramanik and Fraser apply an iterative technique to construct, for each $\alpha$ with $\sum \alpha_n = 0$ and $K > 0$, a set $E$ of positive Hausdorff dimension avoiding solutions to any function $f: \mathbf{R}^d \to \mathbf{R}$ satisfying wth $(\partial_n f)(0) = \alpha_n$,
%
\[ \left| f(x) - \sum \alpha_n x_n \right| \leq K \sum_{n \neq 1} (x_n - x_1)^2 \]
%
The set of such $f$ is an uncountable family, which makes this situation interesting. The technique to create such a set relies on another iterative procedure.

\begin{lemma}
    Let $I \subsetneq [1,d]$ be a strict subset of indices, and $\delta_0 > 0$. Then there exists $\varepsilon > 0$ such that for any $\lambda > 0$ and two disjoint intervals $J_1$ and $J_2$, with $J_1$ occuring before $J_2$, and if we set
    %
    \[ [a_n,b_n] = \begin{cases} J_1 & n \in I \\ J_2 & n \not \in I \end{cases} \]
    %
    then for $\delta < \delta_0$, either for all $x_n \in [a_n,a_n+\varepsilon \lambda]$ or for all $x_n \in [b_n - \varepsilon \lambda, b_n]$,
    %
    \[ \left| \sum \alpha_n x_n \right| \geq \delta \lambda \]
\end{lemma}
\begin{proof}
    If $C^* = \sum |\alpha_n|$, then for $|x_n - a_n| \leq \varepsilon \lambda$,
    %
    \[ |\sum \alpha_n (x_n - a_n)| \leq C^* \varepsilon \lambda \]
    %
    Thus if $|\sum \alpha_n a_n| > (\delta + \varepsilon C^*)\lambda$, then $|\sum \alpha_n x_n| \geq \delta \lambda$. If this does not occur
\end{proof}









\section{Equidistribution Results}

The classical Weyl equidistribution theorem says that if $\alpha$ is an irrational number, then the decimal parts of the numbers $n \alpha$ are equidistributed in $\mathbf{T} = \mathbf{R}/\mathbf{Z}$, in the sense that for any continuous function $f: \mathbf{T} \to \mathbf{R}$,
%
\[ \frac{1}{N} \sum_{n \leq N} f(n \alpha) \to \int_{\mathbf{T}} f(x)\; dx \]
%
This is equivalent to prove that for any interval $[a,b] \in [0,1)$,
%
\[ \frac{\# \{ 1 \leq n \leq N : x_n \in [a,b] \}}{N} \to b - a \]
%
as $N \to \infty$. By approximating a continuous function $f$ by a Fourier series, to prove this is true for a particular sequence, it suffices to prove it for $f(x) = e(nx) = e^{2 \pi i n x}$, for each nonzero integer $n$. Certain techniques we are developing in the theory of cantor decompositions require a higher dimensional variant of such a result, so this section details some information which might help us in the future. We will encounter sequences that are not equidistributed over the entire space, so if $G$ is any closed subgroup of $\mathbf{T}$, we say a sequence $x_n$ is equidistributed over $G$ if $x_n \in G$ for all $n$, and for any continuous function $f: G \to \mathbf{R}$,
%
\[ \frac{1}{N} \sum_{n \leq N} f(n \alpha) \to \int_G f(x)\; dx \]
%
or alternatively, if for any closed set $K$ in $G$,
%
\[ \frac{\# \{ 1 \leq n \leq N: x_n \in K \}}{N} \to |K| \]
%
where $|K|$ is taken with respect to the Haar probability measure on $G$.

\begin{theorem}
    A sequence $x_1, x_2, \dots$ is equidistributed in $\mathbf{T}^d$ if and only if for every nonzero $\xi \in \mathbf{Z}^n$,
    %
    \[ \frac{1}{N} \sum_{n \leq N} e(\xi \cdot x_n) \to 0 \]
\end{theorem}
\begin{proof}
    We prove that the exponential sum condition implies the general result, noting that the other direction is clear. Clearly the exponential sum condition implies the result for all functions $f$ which are trigonometric polynomials. But then, by the Stone Weirstrass theorem and basic Abelian Harmonic analysis, the multivariate trigonometric polynomials are dense in $C(\mathbf{T}^d)$, and we may apply a standard limiting argument.
\end{proof}

\begin{example}
    If $x_n = n \alpha + \beta$, then
    %
    \begin{align*}
        \frac{1}{N} \sum_{n \leq N} e(\xi \cdot x_n) &= \frac{e(\xi \cdot \beta)}{N} \sum_{n \leq N} e(n \xi \cdot \alpha) = \begin{cases} \frac{1}{N} \frac{e(\xi \cdot \beta)(e((n+1) \xi \cdot \alpha) - 1)}{e(\xi \cdot \alpha) - 1} & : \xi \cdot \alpha \not \in \mathbf{Z} \\ e(\xi \cdot \beta) & : \xi \cdot \alpha \in \mathbf{Z} \end{cases}
    \end{align*}
    %
    Thus we see that we can apply Weyl's exponential sum theorem for a particular additive sequence provided that $\xi \cdot \alpha \not \in \mathbf{Z}$ for all $\xi \in \mathbf{Z}^n$, . What's more, $x_n$ is still equidistributed for any subsequence whose indices form an arithmetic progression.
\end{example}

Thus in one dimension, an arithmetic sequence is either equidistributed over the entire torus, or over a discrete set of points forming a {\it discrete subgroup} of the torus. We can think of this discrete subgroup as a zero dimension torus, which leads us to suspect that in higher dimensions, an arithmetic sequence is always equidistributed, but not necessarily over the whole torus, but instead over a lower dimensional subtorus.

\begin{theorem}[Ratner]
    If $x_n = n \alpha + \beta$, then we can write $\alpha = \alpha_0 + \alpha_1$, where the sequence $n \alpha_0 + \beta$ is periodic in $\mathbf{T}^d$, and $n \alpha_1 + \beta$ is equidistributed over a subtorus of $\mathbf{T}^d$. In particular, if $\beta = 0$, then Ratner's theorem says that $x_n$ has an evenly spaced subsequence equidistributed over a subtorus of the space containing the origin.
\end{theorem}
\begin{proof}
    We induct on the dimension. For $d = 1$, the theorem is obvious, since either $\alpha$ is rational, and therefore $n \alpha + \beta$ is periodic, or $\alpha$ is irrational, and $n \alpha + \beta$ is equidistributed on $\mathbf{T}$. So now let us consider a sequence on the torus $\mathbf{T}^{d+1}$. If $\alpha$ is irrational, then $x_n$ is equidistributed, and the theorem is obvious. Otherwise, there exists $\xi \in \mathbf{Z}^n$ such that $\xi \cdot \alpha \in \mathbf{Z}$. We may write $\alpha = \alpha_0 + \alpha_1$, where $\alpha_0 \in \mathbf{Q}^d$, and $\xi \cdot \alpha_1 = 0$. The sequence $n\alpha_0 + \beta$ is periodic, whereas $n \alpha_1$ takes values in the subtorus $T$ of points $x$ with $\xi \cdot x = 0$. Since $\xi \neq 0$, $T$ is a $d$ dimensional compact subgroup of $\mathbf{T}^{d+1}$ isomorphic to $\mathbf{T}^d$. To see this, we assume for simplicity that $\xi_d \neq 0$. Then $\xi^\perp$ is generated by the basis of $d$ vectors $v_n = \xi_d e_n - \xi_n e_d$, for $1 \leq n \leq d$. Thus we get a homomorphism between $\mathbf{R}^d$ and $T$ given by the map $x \mapsto \sum x_n v_n$. If $\sum x_n v_n \in \mathbf{Z}$, so that $\sum x_n v_n = 0$ on $T$, then this means that $x_n \in \mathbf{Z}/\xi_d$ for each $n$, implying that the kernel of this homomorphism is discrete. Lattice theory implies that we can write the kernel as $\bigoplus \mathbf{Z} \langle w_n \rangle$, for $d$ generating vectors $w_1, \dots, w_d$. But then the map $f(x) = \sum x_n w_n$ gives an isomorphism between $\mathbf{T}^d$ and $T$. If we set $\beta = f^{-1}(\alpha_0)$, then $f^{-1}(n\alpha_0) = n \beta$, and so by induction, we can write $\beta = \beta_0 + \beta_1$, where $n\beta_0$ is periodic, and $n \beta_1$ is equidistributed over a subtorus of $\mathbf{T}^d$. But then $n f(\beta_0) = f(n \beta_0)$ is periodic on $T$, and $n f(\beta_1) = f(n \beta_1)$ is equidistributed over a subtorus of $T$. Since the sum of two periodic sequences is periodic, $n \alpha_0 + n f(\beta_0)$ is periodic, and $n f(\beta_1)$ is equidistributed over a subtorus. We have $\alpha_0 + f(\beta_0) + f(\beta_1) = \alpha_0 + f(\beta) = \alpha_0 + \alpha_1$, completing the proof.
\end{proof}

In general, ergodic theory results do not give rates on how long it takes for a sequence to equidistribute over a set. This is not a problem in the constructions we perform, since the rates that our intervals shrink can be arbitrarily fast. However, it is important to note that the convergence rates are uniform across all intervals.

\begin{theorem}
    If $x_n$ is equidistributed over a torus $\mathbf{T}^d$, and
    %
    \[ A_N = \sup_I \left| \frac{\# \{ 1 \leq n \leq N : x_n \in I \}}{N} - |I| \right| \]
    %
    where $I$ ranges over all boxes in $\mathbf{T}^d$, then $A_N \to 0$ as $N \to \infty$.
\end{theorem}
\begin{proof}
    For notational simplicity, we let
    %
    \[ \# (I,N) = \# \{ 1 \leq n \leq N: x_n \in I \} \]
    %
    For each $n$, we can partition $\mathbf{T}^d$ into finitely many disjoint cubes $\{ I_n \}$ with sidelengths $1/n$. Since there are only finitely many such cubes, there is $N_n$ such that for $M \geq N_n$, $| \#(I_n,M)/M - |I_n|| \leq 1/n$. Now given any box $J$, we can find sets $J_1$ and $J_2$, each unions of the cubes $I_n$, with $J_1 \subset J \subset J_2$ and $|J - J_1|, |J_2 - J| \lesssim_d 1/n$. Thus
    %
    \[ |J| - \frac{2}{n} \leq |J_1| - \frac{1}{n} \leq \frac{\#(J_1,M)}{M} \leq \frac{\#(J,M)}{M} \leq \frac{\#(J_2,M)}{M} \leq |J_2| + \frac{1}{n} \leq |J| + \frac{2}{n} \]
    %
    which completes the proof, since $N_n$ is independent of $J$.
\end{proof}







\chapter{Ideas on Squarefree Subsets}

\section{Ideas For New Work}

A continuous formulation of the squarefree difference problem is not so clear to formulate, because every positive real number has a square root. Instead, we consider a problem which introduces a similar structure to avoid in the continuous domain rather than the discrete. Unfortunately, there is no direct continuous anology to the squarefree subset problem on the interval $[0,1]$, because there is no canonical subset of $[0,1]$ which can be identified as `perfect squares', unlike in $\mathbf{Z}$. If we only restrict ourselves to perfect squares of a countable set, like perfect squares of rational numbers, a result of Keleti gives us a set of full Hausdorff dimension avoiding this set. Thus, instead, we say a set $X \subset [0,1]$ is (continuously) {\bf squarefree} if there are no nontrivial solutions to the equation $x - y = (u - v)^2$, in the sense that there are no $x,y,u,v \in X$ satisfying the equation for $x \neq y, u \neq v$. In this section we consider some blue sky ideas that might give us what we need.

How do we adopt Rusza's power series method to this continuous formulation of the problem? We want to scale up the problem exponentially in a way we can vary to give a better control of the exponentials. Note that for a fixed $m$, every elements $x \in [0,1]$ has an essentially unique $m$-ary expansion
%
\[ x = \sum_{n = 1}^\infty \frac{x_n}{m^n} \]
%
and the pullback to the Haar measure on $\mathbf{F}_m^\infty$ is measure preserving (with respect to the natural Haar measure on $\mathbf{F}_m^\infty$), so perhaps there is a way to reformulate the problem natural as finding nice subsets of $\mathbf{F}_m^\infty$ avoiding squares. In terms of this expansion, the equation $x - y = (u - v)^2$ can be rewritten as
%
\[ \sum_{n = 1}^\infty \frac{x_n - y_n}{m^n} = \left( \sum_{k = 1}^\infty \frac{u_n - v_n}{m^n} \right)^2 = \sum_{n = 1}^\infty \left( \sum_{k = 1}^{n-1} (u_k - v_k)(u_{n-k} - v_{n-k}) \right) \frac{1}{m^n} \]
%
One problem with this expansion is that the sums of the differences of each element do not remain in $\{ 0, \dots, m-1 \}$, so the sum on the right cannot be considered an equivalent formal expansion to the expansion on the left. Perhaps $\mathbf{F}_m^\infty$ might be a simpler domain to explore the properties of squarefree subsets, in relation to Ruzsa's discrete strategy. What if we now consider the problem of finding the largest subset $X$ of $\mathbf{F}_m^\infty$ such that there do not exist $x,y,u,v \in \mathbf{F}_m^\infty$ such that if $x,y,u,v \in X$, $x \neq y$, $u \neq v$, then for any $n$
%
\[ x_n - y_n \neq \sum_{k = 1}^{n-1} (u_k - v_k)(u_{n-k} - v_{n-k}) \]

What if we consider the problem modulo $m$, so that the convolution is considered modulo $m$, and we want to avoid such differences modulo $m$. So in particular, we do not find any solutions to the equation
%
\begin{align*}
    x_2 - y_2 &= (u_1 - v_1)^2\\
    x_3 - y_3 &= 2 (u_1 - v_1)(u_2 - v_2)\\
    x_4 - y_4 &= (u_1 - v_1)(u_3 - v_3) + (u_2 - v_2)^2\\
    &\ \ \vdots
\end{align*}
%
which are considered modulo $m$. The topology of the $p$-adic numbers induces a power series relationship which `goes up' and might be useful to our analysis, if the measure theory of the $p$-adic numbers agrees with the measure theory of normal numbers in some way, or as an alternate domain to analyze the squarefree problem as with $\mathbf{F}_m^\infty$.

The problem with the squarefree subset problem is that we are trying to optimize over two quantities. We want to choose a set $X$ such that the number of distinct differences $x - y$ as small as possible, while keeping the set as large as possible. This double optimization is distinctly different from the problem of finding squarefree difference subsets of the integers. Perhaps a more natural analogy is to fix a set $V$, and to find the largest subset $X$ of $[0,1]$ such that $x - y = (u - v)^2$, where $x \neq y \in X$, and $u \neq v \in V$. Then we are just avoiding subsets of $[0,1]$ which avoid a particular set of differences, and I imagine this subset has a large theory. But now we can solve the general subset problem by finding large subsets $X$ such that $(X - X)^2 \subset V$ and $X$ containing no differences in $V$. Does Rusza's method utilize the fact that the problem is a single optimization? Can we adapt Rusza's method work to give better results about finding subsets $X$ of the integers such that $X - X$ is disjoint from $(X - X)^2$?

\section{Squarefree Sets Using Modulus Techniques}

We now try to adapt Ruzsa's idea of applying congruences modulo $m$ to avoid squarefree differences on the integers to finding high dimensional subsets of $[0,1]$ which satisfy a continuous analogy of the integer constraint. One problem with the squarefree problem is that solutions are non-scalable, in the sense that if $X \subset [N]$ is squarefree, $\alpha X$ may not be squarefree. This makes sense, since avoiding solutions to $\alpha (x - y) = \alpha^2 (u - v)^2$ is clearly not equivalent to the equation $x - y = (u - v)^2$. As an example, $X = \{ 0, 1/2 \}$ is squarefree, but $2X = \{ 0, 1 \}$ isn't. On the other hand, if $X$ avoids squarefree differences modulo $N$, it {\it is} scalable by a number congruent to 1 modulo $N$. More generally, if $\alpha$ is a rational number of the form $p/q$, then $\alpha X$ will avoid nontrivial solutions to $q (x - y) = p (u - v)^2$, and if $p$ and $q$ are both congruent to 1 modulo $N$, then $X$ is squarefree, so modulo arithmetic enables us to scale down. Since the set of rational numbers with numerator and denominator congruent to 1 is dense in $\mathbf{R}$, {\it essentially} all scales of $X$ are continuously squarefree. Since $X$ is discrete, it has Hausdorff dimension zero, but we can `fatten' the scales of $X$ to obtain a high dimension continuously squarefree set. To initially simplify the situation, we now choose to avoid nontrivial solutions to $y - x = (z - x)^2$, removing a single degree of freedom from the domain of the equation.

So we now fix a subset $X$ of $\{ 0, \dots, m-1 \}$ avoiding squares modulo $m$. We now ask how large can we make $\varepsilon$ such that nontrivial solutions to $x - y = (x - z)^2$ in the set
%
\[ E = \bigcup_{x \in X} [\alpha x, \alpha x + \varepsilon) \]
%
occur in a common interval, if $\alpha$ is just short of $1/m^n$. This will allow us to recursively place a scaled, `fattened' version of $X$ in every interval, and then consider a limiting process to obtain a high dimensional continuously squarefree set. If we have a nontrivial solution triple, we can write it as $\alpha x + \delta_1, \alpha y + \delta_2$, and $\alpha z + \delta_3$, with $\delta_1, \delta_2, \delta_3 < \varepsilon$. Expanding the solution leads to
%
\[ \alpha (x - y) + (\delta_1 - \delta_2) = \alpha^2 (x - z)^2 + 2\alpha(x - z)(\delta_1 - \delta_3) + (\delta_1 - \delta_3)^2 \]
%
If $x$, $y$, and $z$ are all distinct, then, as we have discussed, we cannot have $\alpha (x - y) = \alpha^2 (x - z)^2$. if $\alpha$ is chosen close enough to $1/m^n$, then we obtain an approximate inequality
%
\[ |\alpha (x - y) - \alpha^2 (x - z)^2| \geq \alpha^2 \]
%
(we require $\alpha$ to be close enough to $1/n$ for some $n$ to guarantee this). Thus we can guarantee at least two of $x$, $y$, and $z$ are equal to one another if
%
\[ |2\alpha(x - z)(\delta_1 - \delta_3) + (\delta_1 - \delta_3)^2 - (\delta_1 - \delta_2)| < \frac{1}{m^{2n}} \]
%
We calculate that
%
\[ 2\alpha(x - z)(\delta_1 - \delta_3) + (\delta_1 - \delta_3)^2 - (\delta_1 - \delta_2) < 2\alpha(m-1)\varepsilon + \varepsilon^2 + \varepsilon \]
\[ (\delta_1 - \delta_2) - 2\alpha(x - z)(\delta_1 - \delta_3) - (\delta_1 - \delta_3)^2 \leq \varepsilon + 2\alpha(m-1)\varepsilon \]
%
So it suffices to choose $\varepsilon$ such that
%
\[ \varepsilon^2 + [2\alpha(m-1) + 1]\varepsilon \leq \alpha^2 \]
%
This is equivalent to picking
%
\[ \varepsilon \leq \sqrt{\left( \frac{2 \alpha(m - 1) + 1}{2} \right)^2 + \alpha^2} - \frac{2\alpha(m-1) + 1}{2} \approx \frac{\alpha^2}{2\alpha(m-1) + 1} \]

We split the remaining discussion of the bound we must place on $\varepsilon$ into the three cases where two of $x$, $y$, and $z$ are equal, but one is distinct, to determine how small $\varepsilon$ must be to prevent this from happening. Now
%
\begin{itemize}
    \item If $y = z$, but $x$ is distinct, then because we know $\alpha(x - y) = \alpha^2(x - y)^2$ has no solution in $X$, we obtain that (provided $\alpha$ is close enough to $1/m^n$),
    %
    \[ |\alpha(x - y) - \alpha^2(x-y)^2| \geq \alpha^2 \]
    %
    and the same inequality that worked for the case where the three equations are distinct now applies for this case.

    \item If $x = y$, but $z$ is distinct, we are left with the equation
    %
    \[ \delta_1 - \delta_2 = \alpha^2(x - z)^2 + 2\alpha(x - z)(\delta_1 - \delta_3) + (\delta_1 - \delta_3)^2 \]
    %
    Now $\alpha^2(x - z)^2 \geq \alpha^2$, and
    %
    \[ \delta_1 - \delta_2 - 2\alpha(x-z)(\delta_1 - \delta_3) - (\delta_1 - \delta_3)^2 < \varepsilon + 2\alpha(m-1)\varepsilon \]
    %
    so we need the additional constraint $\varepsilon + 2\alpha(m-1)\varepsilon \leq \alpha^2$, which is equivalent to saying
    %
    \[ \varepsilon \leq \frac{\alpha^2}{1 + 2\alpha(m-1)} \]

    \item If $x = z$, but $y$ is distinct, we are left with the equation
    %
    \[ \alpha(x - y) + (\delta_1 - \delta_2) = (\delta_1 - \delta_3)^2 \]
    %
    Now $|\alpha(x-y)| \geq \alpha$, and
    %
    \[ (\delta_1 - \delta_3)^2 - (\delta_1 - \delta_2) < \varepsilon^2 + \varepsilon \]
    \[ (\delta_1 - \delta_2) - (\delta_1 - \delta_3)^2 < \varepsilon \]
    %
    so to avoid this case, we need $\varepsilon^2 + \varepsilon \leq \alpha$, or
    %
    \[ \varepsilon \leq \frac{\sqrt{1 + 4\alpha} - 1}{2} \approx \alpha \]
\end{itemize}
%
Provided $\varepsilon$ is chosen as above, all solutions in $E$ must occur in a common interval. Thus, if we now replace the intervals with a recursive fattened scaling of $X$, all solutions must occur in smaller and smaller intervals. If we choose the size of these scalings to go to zero, these solutions are required to lie in a common interval of length zero, and thus the three values must be equal to one another. Rigorously, we set $\varepsilon \approx 1/m^2$, and $\alpha \approx 1/m$, we can define a recursive construction by setting
%
\[ E_1 = \bigcup_{x \in X} [\alpha x, \alpha x + \varepsilon_1) \]
%
and if we then set $X_n$ to be the set of startpoints of the intervals in $E_n$, then
%
\[ E_{n+1} = \bigcup_{x \in X_n} (x + \alpha^2 E_n) \]
%
Then $\bigcap E_n$ is a continuously squarefree subset. But what is it's dimension?

\section{Idea; Delaying Swaps}

By delaying the removing in the pattern removal queue, we may assume in our dissection methods that we are working with sets with certain properties, i.e. we can swap an interval with a dimension one set avoiding translates.

\section{Squarefree Subsets Using Interval Dissection Methods}

The main idea of Keleti's proof was that, for a function $f$, given a method that takes a sequence of disjoint unions of sets $J_1, \dots, J_N$, each a union of almost disjoint closed intervals of the same length, and gives large subsets $J_n' \subset J_n$, each a union of almost disjoint intervals of a much smaller length, such that $f(x_1, \dots, x_n) \neq 0$ for $x_n \in J_n'$. Then one can find high dimensional subsets $K$ of the real line such that $f(x_1, \dots, x_n) \neq 0$ for a sequence of distinct $x_1, \dots, x_n \in K$. The larger the subsets $J_n'$ are compared to $J_n$, the higher the Hausdorff dimension of $K$. We now try and apply this method to construct large subsets avoiding solutions to the equation $f(x,y,z) = (x - y) - (x - z)^2$. In this case, since solutions to the equation above satisfy $y = x - (x-z)^2$, given $J_1, J_2, J_3$, finding $J_1', J_2', J_3'$ as in the method above is the same as choosing $J_1'$ and $J_3'$ such that the image of $J_1' \times J_3'$ under the map $g(x,z) = x - (x-z)^2$ is small in $J_2$. We begin by discretizing the problem, splitting $J_1$ and $J_3$ into unions of smaller intervals, and then choosing large subsets of these intervals, and finding large intervals of $J_2$ avoiding the images of the startpoints to these intervals.

So suppose that $J_1,J_2$, and $J_3$ are unions of intervals of length $1/M$, for which we may find subsets $A,B \subset [M]$ of the integers such that
%
\[ J_1 = \bigcup_{a \in A} \left[\frac{a}{M}, \frac{a + 1}{M} \right]\ \ \ \ \ J_3 = \bigcup_{b \in B} \left[ \frac{b}{M} , \frac{b + 1}{M} \right] \]
%
If we split $J_1$ and $J_3$ into intervals of length $1/NM$, for some $N \gg M$ to be specified later (though we will assume it is a perfect square), then
%
\[ J_1 = \bigcup_{\substack{a \in A\\0 \leq k < N}} \left[ \frac{Na + k}{NM}, \frac{Na + k}{NM} + \frac{1}{NM} \right]\ \ \ \ \ J_3 = \bigcup_{\substack{b \in A\\0 \leq l < N}} \left[ \frac{Nb + l}{NM}, \frac{Nb + l}{NM} + \frac{1}{NM} \right] \]
%
We now calculate $g$ over the startpoints of these intervals, writing
%
\begin{align*}
    g \left( \frac{Na + k}{NM}, \frac{Nb + l}{NM} \right) &= \frac{Na + k}{NM} - \left( \frac{N(a - b) + (k-l)}{NM} \right)^2\\
    &= \frac{a}{M} - \frac{(a-b)^2}{M^2} + \frac{k}{NM} - \frac{2(a-b)(k-l)}{NM^2} + \frac{(k-l)^2}{(NM)^2}
\end{align*}
%
which splits the terms into their various scales. If we write $m = k - l$, then $m$ can range on the integers in $(-N,N)$, and so, ignoring the first scale of the equation, we are motivated to consider the distribution of the set of points of the form
%
\[ \frac{k}{NM} - \frac{2(a-b)m}{NM^2} + \frac{m^2}{(NM)^2} \]
%
where $k$ is an integer in $[0,N)$, and $m$ an integer in $(-N,N)$. To do this, fix $\varepsilon > 0$. Suppose that we find some value $\alpha \in [0,1]$ such that $S$ intersects
%
\[ \left[ \alpha , \alpha + \frac{1}{N^{1 + \varepsilon}} \right] \]
%
Then there is $k$ and $m$ such that
%
\[ 0 \leq \frac{kNM - 2N(a-b)m + m^2}{(NM)^2} - \alpha \leq \frac{1}{N^{1 + \varepsilon}} \]
%
Write $m = q \sqrt{N} + r$ (remember that we chose $N$ so it's square root is an integer), with $0 \leq r < \sqrt{N}$. Then $m^2 = qN + 2qr \sqrt{N} + r^2$, and if $2qr = Q\sqrt{N} + R$, where $0 \leq R < \sqrt{N}$, then we find
%
\[ -\frac{R}{M^2 N^{3/2}} - \frac{r^2}{(NM)^2} \leq \frac{kM - 2(a-b)m + q + Q}{NM^2} - \alpha \leq \frac{1}{N^{1 + \varepsilon}} - \frac{R}{M^2 N^{3/2}} - \frac{r^2}{(NM)^2} \]
%
Thus
%
\[ d(\alpha, \mathbf{Z}/NM^2) \leq \max \left( \frac{1}{N^{1+\varepsilon}} - \frac{R}{\sqrt{N}} - \frac{r^2}{N}, \frac{R}{M^2 N^{3/2}} + \frac{r^2}{(NM)^2} \right) \]
%
If we now restrict our attention to the set $S$ consisting of the expressions we are studying where $R \leq (\delta_0/2) \sqrt{N}$, $r \leq \sqrt{\delta_0 N/2}$, then if the interval corresponding to $\alpha$ intersects $S$, then
%
\[ d(\alpha, \mathbf{Z}/NM^2) \leq \max \left( \frac{1}{N^{1+\varepsilon}} , \frac{\delta_0}{NM^2} \right) \]
%
If $N^\varepsilon \geq M^2/\delta_0$, then we can force $d(\alpha, \mathbf{Z}/NM^2) \leq \delta_0/NM^2$ for all $\alpha$ intersecting $S$. Thus, if we split $J_2$ into intervals starting at points of the form
%
\[ \frac{k + 1/2}{NM^2} \]
%
each of length $1/N^{1+\varepsilon}$, then provided $\delta_0 < 1/2$, we conclude that these intervals do not contain any points in $S$, since
%
\[ d \left( \frac{k + 1/2}{NM^2}, \mathbf{Z}/NM^2 \right) = \frac{1}{2NM^2} > \frac{\delta_0}{NM^2} \]
%
So we're well on our way to using Pramanik and Fraser's recursive result, since this argument shows that, provided points in $J_1$ and $J_3$ are chosen carefully, we can keep $O_M(1/N^{1 + \varepsilon})$ of each interval in $J_2$, which should lead to a dimension bound arbitrarily close to one.

\section{Finding Many Startpoints of Small Modulus}

To ensure a high dimension corresponding to the recursive construction, it now suffices to show $J_1$ and $J_3$ contain many startpoints corresponding to points in $S$, so that the refinements can be chosen to obtain $O_M(1/N)$ of each of the original intervals. Define $T$ to be the set of all integers $m \in (-N,N)$ with $m = q \sqrt{N} + r$ and $r \leq \sqrt{\delta_0 N/2}$ and $2qr = Q\sqrt{N} + R$ with $R \leq (\delta_0/2) \sqrt{N}$. Because of the uniqueness of the division decomposition, we find $T$ is in one to one correspondence with the set $T'$ of all pairs of integers $(q,r)$, with $q \in (-\sqrt{N},\sqrt{N})$ and $r \in [0,\sqrt{N})$, with $r \leq \sqrt{\delta_0 N/2}$, $2qr = Q \sqrt{N} + R$, and $R \leq (\delta_0/2) \sqrt{N}$. Thus we require some more refined techniques to better upper bound the size of this set.

Let's simplify notation, generalizing the situation. Given a fixed $\varepsilon$, We want to find a large number of integers $n \in (-N,N)$ with a decomposition $n = qr$, where $r \leq \varepsilon \sqrt{N}$, and $q \leq \sqrt{N}$. The following result reduces our problem to understanding the distribution of the smooth integers.

\begin{lemma}
    Fix constants $A,B$, and let $n \leq AN$ be an integer. If all prime factors of $n$ are $\leq BN^{1-\delta}$, then $n$ can be decomposed as $qr$ with $r \leq \varepsilon \sqrt{N}$ and $q \leq \sqrt{N}$.
\end{lemma}
\begin{proof}
    Order the prime factors of $n$ in increasing order as $p_1 \leq p_2 \leq \dots \leq p_K$. Let $r = p_1 \dots p_m$ denote the largest product of the first prime factors such that $r \leq \varepsilon \sqrt{N}$. If $r = n$, we can set $q = 1$, and we're finished. Otherwise, we know $r p_{m+1} > \varepsilon \sqrt{N}$, hence
    %
    \[ r > \frac{\varepsilon \sqrt{N}}{p_{m+1}} \geq \frac{\varepsilon \sqrt{N}}{B N^{1-\delta}} = \frac{\varepsilon}{B} N^{\delta - 1/2} \]
    %
    And if we set $q = n/r$, the inequality above implies
    %
    \[ q < \frac{nB}{\varepsilon} N^{1/2 - \delta} \leq \frac{AB}{\varepsilon} N^{3/2-\delta} \]
    %
    But now we run into a problem, because the only way we can set $q < \sqrt{N}$ while keeping $A$, $B$, and $\varepsilon$ fixed constants is to set $\delta = 1$, and $AB/\varepsilon \leq 1$.
\end{proof}

\begin{remark}
    Should we expect this method to work? Unless there's a particular reason why values of $(q,r)$ should accumulate near $Q = 0$, we should expect to lose all but $N^{-1/2}$ of the $N$ values we started with, so how can we expect to get $\Omega(N)$ values in our analysis. On the other hand, if a number $n$ is suitably smooth, in a linear amount of cases we should be able to divide up primes into two numbers $q$ and $r$ such that $r$ is small and $q$ fits into a suitable value of $Q$, so maybe this method will still work.
\end{remark}

Regardless of whether the lemma above actually holds through, we describe an asymptotic formula for perfect numbers which might come in handy. If $\Psi(N,M)$ denotes the number of integers $n \leq N$ with no prime factor exceeding $M$, then Karl Dickman showed
%
\[ \Psi(N,N^{1/u}) = N \rho(u) + O \left( \frac{u N}{\log N} \right) \]
%
This is essentially linear for a fixed $u$, which could show the set of $(q,r)$ is $\Omega_\varepsilon(N)$, which is what we want. Additional information can be obtained from Hildebrand and Tenenbaum's survey paper ``Integers Without Large Prime Factors''.

\section{A Better Approach}

Remember that we can write a general value in our set as
%
\[ x = \frac{-2(a-b)m}{NM^2} + \frac{q}{NM^2} + \frac{Q}{NM^2} + \frac{R}{N^{3/2} M^2} + \frac{r^2}{N^2M^2} \]
%
with the hope of guaranteeing the existence of many points, rather than forcing $R$ to be small, we now force $R$ to be close to some scaled value of $\sqrt{N}$, 
%
\[ |R - n \varepsilon \sqrt{N}| = \delta \sqrt{N} \leq \varepsilon \sqrt{N} \]
%
Then
%
\[ x = \frac{-2(a-b)m + q + Q + n\varepsilon + \delta}{NM^2} + \frac{r^2}{N^2M^2} \]
%
So
%
\[ d \left( x, \mathbf{Z}/NM^2 + n\varepsilon / NM^2 \right) \leq \frac{\varepsilon}{NM^2} + \frac{1}{4NM^2} = \frac{\varepsilon + 1/4}{NM^2} \]
%
By the pidgeonhole principle, since $R < \sqrt{N}$, there are $1/\varepsilon$ choices for $n$, whereas there are


The choice has the benefit of automatically possessing a lot of points by the pidgeonhole principle,

\chapter{Low Rank Coordinate Changes}

\section{Boosting the Dimension of Pattern Avoiding Sets by Low Rank Coordinate Changes}

We now consider finding subsets of $[0,1]$ avoiding solutions to the equation $y = f(Tx)$, where $T$ is a rank $k$ linear transformation with integer coefficients with respect to standard coordinates, and $f$ is real-valued and Lipschitz continuous. Fix a constant $A$ bounding the operator norm of $T$, in the sense that $|Tx| \leq A|x|$ for all $x \in \mathbf{R}^n$, and a constant $B$ such that $|f(x+y) - f(x)| \leq B|y|$ for all $x$ and $y$ for which the equation makes sense (if $f$ is $C^1$, this is equivalent to a bound $\| \nabla f \|_\infty \leq B$). Consider sets $J_0, J_1, \dots, J_n, \subset [0,1]$, which are unions of intervals of length $1/M$, with startpoints lying on integer multiples of $1/M$. The next theorem works as a `building block lemma' used in our algorithm for constructing a set avoiding solutions to the equation with Hausdorff dimension $k$ and full Minkowski dimension.

\begin{theorem}
    For infinitely many integers $N$, there exists $S_i \subset J_i$ avoiding solutions to $y = f(Tx)$ with $y \in S_0$ and $x_n \in S_n$, such that
    %
    \begin{itemize}
        \item For $n \neq 0$, if we decompose each $J_i$ into length $1/N$ consecutive intervals, $S_i$ contains an initial portion $\Omega(1/N^k)$ of each length $1/N$ interval. This part of the decomposition gives the Hausdorff dimension $1/k$ bound for the set we will construct.

        \item If we decompose $J_i$ into length $1/N$ intervals, and then subdivide these intervals into length $\Omega(1/N^k)$ intervals, then $S_0$ contains a subcollection of these $1/N^k$ intervals which contains a total length $\Omega(1/N)$ of a fraction $1 - 1/M$ of the length $1/N$ intervals. This property gives that our resultant set will have full Minkowski dimension.
    \end{itemize}
    %
    The implicit constants in these bounds depend only on $A$, $B$, $n$, and $k$.
\end{theorem}
\begin{proof}
Split each interval of $J_a$ into length $1/N$ intervals, and then set
%
\[ \mathbf{A} = \{ x : x_a\ \text{is a startpoint of a $1/N$ interval in $J_a$} \} \]
%
Since the startpoints of the intervals are integer multiples of $1/N$, $T(\mathbf{A})$ is contained with a rank $k$ sublattice of $(\mathbf{Z}/N)^m$. The operator norm also guarantees $T(\mathbf{A})$ is contained within the ball $B_A$ of radius $A$ in $\mathbf{R}^m$. Because of the lattice structure of the image, $| x - y | \gtrsim_n 1/N$ for each distinct pair $x,y \in T(\mathbf{A})$. For any $R$, we can cover $\Sigma \cap B_A$ by $O_{n,k}((A/R)^k)$ balls of radius $R$. If $R \gtrsim_n 1/N$, then each ball can contain only a single element of $T(\mathbf{A})$, so we conclude that $|T(\mathbf{A})| \lesssim_{n,k} (AN)^k$. If we define the set of `bad points' to be
%
\[ \mathbf{B} = \{ y \in [0,1] : \text{there is $x \in \mathbf{A}$ such that $y = f(T(x))$} \} \]
%
Then
%
\[ |\mathbf{B}| = |f(T(\mathbf{A}))| \leq |T(\mathbf{A})| = O_{A,n,k}(N^k) \]
%
For simplicity, we now introduce an integer constant $C_0 = C_0(A,n,k,M)$ such that $|\mathbf{B}| \leq (C_0/M^2) N^k$. We now split each length $1/M$ interval in $J_0$ into length $1/N$ intervals, and filter out those intervals containing more than $C_0N^{k-1}$ elements of $\mathbf{B}$. Because of the cardinality bound we have on $\mathbf{B}$ there can be at most $N/M^2$ such intervals, so we discard at most a fraction $1/M$ of any particular length $1/M$ interval in $J_0$. If we now dissect the remaining intervals into $4C_0N^{k-1}$ intervals of length $1/4C_0N^k$, and discard any intervals containing an element of $\mathbf{B}$, or adjacent to such an interval, then the remaining such intervals $I$ satisfy $d(I,\mathbf{B}) \geq 1/4C_0N^k \gtrsim_{A,n,M,k}(1/N^k)$, and because of our bound on the number of elements of $\mathbf{B}$ in these intervals, there are at least $C_0N^{k-1}$ intervals remaining, with total length exceeding $C_0N^{k-1}/4C_0N^k = \Omega(1/N)$. If $f$ is $C^1$ with $\| \nabla f \|_\infty \leq B$, or more generally, if $f$ is Lipschitz continuous of magnitude $B$, then
%
\[ | f(Tx) - f(Tx')| \leq AB |x - x'| \]
%
and so we may choose $S_i \subset J_i$ by thickening each startpoint $x \in J_i$ to a length $O(1/N^k)$ interval while still avoiding solutions to the equation $y = f(T(x))$.
\end{proof}

\begin{remark}
    If $T$ is a rank $k$ linear transformation with rational coefficients, then there is some number $a$ such that $aT$ has integer coefficients, and then the equation $y = f(Tx)$ is the same as the equation $y = f_0((aT)(x))$, where $f_0(x) = f(x)/a$. Since $f_0$ is also Lipschitz continuous, we conclude that we still get the dimension $1/k$ bound if $T$ has rational rather than integral coefficients. More generally, this trick shows the result applies unperturbed if all coefficients of $T$ are integer multiples of some fixed real number. More generally, by varying the lengths of our length $1/N$ decomposition by a constant amount, we can further generalize this to the case where each column of $T$ are integers multiples of some fixed real number.
\end{remark}

\begin{remark}
    To form $\mathbf{A}$, we take startpoints lying at equal spaced $1/N$ points. However, by instead taking startpoints at varying points in the length $1/N$ intervals, we might be able to make points cluster more than in the original algorithm. Maybe the probabalistic method would be able to guarantee the existence of a choice of startpoints whose images are tightly clustered together.
\end{remark}

\begin{remark}
    Since the condition $y = f(Tx)$ automatically assumes a kind of `non-vanishing derivative' condition on our solutions, we do not need to assume the regularity of $f$, and so the theorem extends naturally to a more general class of functions than Rob's result, i.e. the Lipschitz continuous functions.
\end{remark}

Using essentially the same approach as the last argument shows that we can avoid solutions to $y = f(Tx)$, where $y$ and $x$ are now vectors in some $\mathbf{R}^m$, and $T$ has rank $k$. If we consider unions of $1/M$ cubes $J_0, \dots, J_n$. If we fix startpoints of each $x_k$ forming lattice spaced apart by $\Omega(1/N)$, and consider the space $\mathbf{A}$ of products, then there are $O(N^k)$ points in $T(\mathbf{A})$, and so there are $O(N^k)$ elements in $\mathbf{B}$. We now split each $1/M$ cube in $J_0$ into length $1/N$ cubes, and discard those cubes which contain more than $O(N^{k-m})$ bad points, then we discard at most $1 - 1/M$ of all such cubes. We can dissect the remaining length $1/N$ cubes into $O(N^{k-m})$ length $\Omega(1/N^{k/m})$ cubes, and as in the previous argument, the cubes not containing elements of $\mathbf{B}$ nor adjacent to an element have total volume $\Omega(1/N^m)$, which we keep. The startpoints in the other intervals $T_i$ may then be thickened to a length $\Omega(1/N^{m/k})$ portion while still avoiding solutions. This gives a set with full Minkowski dimension and Hausdorff dimension $m/k$ avoiding solutions to $y = f(Tx)$. (I don't yet understand Minkowski dimension enough to understand this, but the techniques of the appendix make proving the Hausdorff dimension $1/k$ bound easy)

\section{Extension to Well Approximable Numbers}

If the coefficients of the linear transformation $T$ in the equation $y = f(Tx)$ are non-rational, then the images of startpoints under the action of $T$ do not form a lattice, and so points may not overlap so easily when avoiding solutions to the equations $y = f(Tx)$. However, if $T$ is `very close' to a family of rational coefficient linear transformations, then we can show the images of the startpoints are `very close' to a lattice, which will still enable us to find points avoiding solutions by replacing the direct combinatorial approach in the argument for integer matrices with a covering argument.

Suppose that $T$ is a real-coefficient linear transformation with the property that for each coefficient $x$ there are infinitely many rational numbers $p/q$ with $|x - p/q| \leq 1/q^\alpha$, for some fixed $\alpha$. For infinitely many $K$, we can therefore find a linear transformation $S$ with coefficients in $\mathbf{Z}/K$ with each coefficient of $T$ differing from the corresponding coefficient in $S$ by at most $1/K^\alpha$. Then for each $x$, we find
%
\[ \| (T - S)(x) \|_\infty \leq (n/K^\alpha) \| x \|_\infty \]
%
If we now consider $T_0, \dots, T_n$, splitting $T_1, \dots, T_n$ into length $1/N$ intervals, and considering $\mathbf{A}$ as in the last section, then $S(\mathbf{A})$ lie in a $k$ dimensional sublattice of $(\mathbf{Z}/KN)^m$, hence containing at most $(2A)^k (KN)^k = O_{T,n}((KN)^k)$ points. By our error term calculation of $T-S$, the elements of $T(\mathbf{A})$ are contained in cubes centered at these lattice points with side-lengths $2n/K^\alpha$, or balls centered at these points with radius $n^{3/2}/K^\alpha$. If $\| \nabla f \| \leq B$, then the images of the radius $n^{3/2}/K^\alpha$ balls under the action of $f$ are contained in length $Bn^{3/2}/K^\alpha$ intervals. Thus the total length of the image of all these balls under $f$ is $(Bn^{3/2}/K^\alpha)(2A)^k(KN)^k = (2A)^k Bn^{3/2} K^{k-\alpha} N^k$. If $k < \alpha$, then we can take $K$ arbitrarily large, so that there exists intervals with $\text{dist}(I,\mathbf{B}) = \Omega_{A,k,M}(1/N^k)$. But I believe that, after adding the explicit constants in, we cannot let $k = \alpha$.

\begin{remark}
    One problem is that, if $T$ has rank $k$, we might not be able to choose $S$ to be rank $k$ as well. Is this a problem? If $T$ has full rank, then the set of all such matrices is open so if $T$ and $S$ are close enough, $S$ also has rank $k$, but this need not be true if $T$ does not have full rank.
\end{remark}

\begin{example}
    If $T$ has rank 1, then Dirichlet's theorem says that every irrational number $x$ can be approximated by infinitely many $p/q$ with $|x - p/q| < 1/q^2$, so every real-valued rank 1 linear transformation can be avoided with a dimension one bound.
\end{example}

\section{Equidistribution and Real Valued Matrices}

If $T$ is a non-invertible matrix containing irrational coefficients, then the values $Tx$, for $x \in \mathbf{Z}^n$, do not form a lattice, and therefore we cannot use the direct combinatorial arguments of the past section to obtain the decomposition lemma. However, without loss of generality, we can write $T(x) = S(x_1) + U(x_2)$, where $x = (x_1,x_2)$, $x_1 \in \mathbf{R}^k$, $x_2 \in \mathbf{R}^{n-k}$, and $S$ has full rank $k$. Then $S$ is an embedding of $\mathbf{R}^k$ into $\mathbf{R}^n$, so $\Gamma = S((\mathbf{Z}/N)^k)$ forms a lattice with points spaced apart by a distance on the order of $\Omega(1/N)$. Since $T$ has rank $k$, the image of $U$ is contained within the image of $S$. We let $\mathbf{T}$ denote the torus obtained by quotienting the $k$ dimensional subplane forming the image of $T$ by $\Gamma$. Then the image of $S$ in $\mathbf{T}$ is contained within an $\alpha$ dimensional subtorus of $\mathbf{T}$. Note that $\alpha = 0$ precisely when $T$ still has rank $k$ over the rational numbers, so that in a suitable basis $T$ is an integer valued matrix. If $\alpha = 1$, then by an appropriate scaling in the values $x_2$ we can still make the values of $S$ lie at lattice points, which should give a Hausdorff dimension one set. When $\alpha = 2$, we run into problems.

If $\pi: \mathbf{R}^k \to T$ is the homomorphism obtained by composing the quotient map onto the torus with the linear map $T$, then $\pi(x_1,0) = 0$ for all $x_1 \in (\mathbf{Z}/N)^k$. On the other hand, Ratner's theorem implies that for each $x_2 \in \mathbf{R}^{n-k}$, there is some $M$ such that the sequence $\pi(0, nMx_2) = Mn \pi(0, x_2)$ is equidistributed on a subtorus of $T$. Equidistribution may be useful in extending the rational matrix result to all real matrices with some dimension loss, since a matrix is rational if and only if $n M \pi(0, x_2)$ is equidistributed on a zero dimensional lattice -- it may ensure that points are closely clustered to lattice points.

Lets consider the simplest case, where $n = k+1$, so $x_2 \in \mathbf{R}$. Consider our setup, with intervals $J_0, \dots, J_n$, and an equation $y = f(Tx)$, where $f$ is Lipschitz with Lipschitz norm bounded by $B$. Now if $T = S + U$, then $S(\mathbf{Z}^k)$ forms a rank $k$ lattice, and $U(\mathbf{Z})$ equidistributes over an $\alpha$ dimensional subtorus of the torus generated over the lattice. In particular, since the set $\mathbf{Z} \cap NJ_n$ contains $\Omega(N |J_n|) = \Omega(N)$ consecutive points, for any $\varepsilon$ and suitably large $N$, $\mathbf{Z} \cap NJ_n$ contains $\Omega(Nr^\alpha)$ points $x$ such that $S(x)$ is within a distance $r$ from a lattice point, for any $r$. Dividing by $N$ tells us that we have $O(Nr^\alpha)$ points $x$ in $\mathbf{Z}/N \cap J_n$ such that $U(x)$ is at a distance $r/N$ from a lattice point in $S((\mathbf{Z}/N)^k)$. If $r = 1/N^\beta$, then  we have $O(N^{1-\alpha\beta})$ points at a distance $1/N^{\beta + 1}$ from a lattice point. There are $O(N^k)$ points in the lattice, and so provided that $k < 1 + \beta$, we can find a large subset avoiding the images of these startpoints, and we should be able to thicken the startpoints to length $\Omega(1/N^k)$ intervals, hence we should expect the set we construct to have Hausdorff dimension $1/k(k-1)$ if this process is repeated to construct our solution avoiding set.

\section{Applications of Low Rank Coordinate Changes}

\begin{example}
Our initial exploration of low rank coordinate changes was inspired by trying to find solutions to the equation
%
\[ y - x = (u - w)^2 \]
%
Our algorithm gives a Hausdorff dimension $1/2$ set avoiding solutions to this equation. This equals Math\'{e}'s result. But this dimension for us now depends on the shifts involved in the equation, not on the exponent, so we can actually avoid solutions to the equation
%
\[ y - x = (u - w)^n \]
%
for any $n$, in a set of Hausdorff dimension $1/2$. More generally, if $X$ is a set, then given a smooth function $f$ of $n$ variables, we can find a set $X$ of Hausdorff dimension $1/n$ such that there is no $x \in X$, and $y_1, \dots, y_n \in X - X$ such that $x = f(y_1, \dots, y_n)$. This is better than the $1/2n$ bound that is obtained by Malabika and Fraser's result.
\end{example}

\begin{example}
For any fixed $m$, we can find a set $X \subset \mathbf{R}^n$ of full Hausdorff dimension which contains no solutions to
%
\[ a_1x_1 + \dots + a_nx_n = 0 \]
%
for {\it any} rational numbers $a_n$ which are not all zero. Since Malabika/Fraser's technique's solutions are bounded by the number of variables, they cannot let $n \to \infty$ to obtain a linearly independant set over the rational numbers. But since the Hausdorff dimension of our sets now only depends on the rank of $T$, rather than the total number of variables in $T$, we can let $n \to \infty$ to obtain full sets linearly independant over the rationals. More generally, for any Lipschitz continuous function $f: \mathbf{R} \to \mathbf{R}$, we can find a full Hausdorff dimensional set such that there are no solutions
%
\[ f(a_1x_1 + \dots + a_nx_n, y) \]
%
for any $n$, and for any rational numbers $a_n$ that are not all zero.
\end{example}

\begin{example}
The easiest applications of the low rank coordinate change method are probably involving configuration problems involving pairwise distances between $m$ points in $\mathbf{R}^n$, where $m \ll n$, since this can best take advantage of our rank condition. Perhaps one way to encompass this is to avoid $m$ vertex polyhedra in $n$ dimensional space, where $m \ll n$. In order to distinguish this problem from something that can be solved from Math\'{e}'s approach, we can probably find a high dimensional set avoiding $m$ vertex polyhedra on a parameterized $n$ dimensional manifold, where $m \ll n$. There is a result in projective geometry which says that every projectively invariant property of $m$ points in $\mathbf{RP}^d$ is expressible as a function in the ${m \choose d}$ bracket polynomials with respect to these $m$ points. In particular, our result says that we can avoid a countable collection of such invariants in a dimension $1/{m \choose d}$ set. This is a better choice of coordinates than Euclidean coordinates if ${m \choose d} \leq m$. Update: I don't think this is ever the case.
\end{example}

\begin{remark}
    Because of how we construct our set $X$, we can find a dimension $1/k$ set avoiding solutions to $y = f(Tx)$ for {\it all} rank $k$ rational matrices $T$, without losing any Hausdorff dimension. Maybe this will help us avoid solutions to more general problems?
\end{remark}

\begin{example}
Given a smooth curve $\Gamma$ in $\mathbf{R}^n$, can we find a subset $E$ with high Hausdorff dimension avoiding isoceles triangles. That is, if the curve is parameterized by $\gamma: [0,1] \to \mathbf{R}^n$, can we find $E \subset [0,1]$ such that for any $t_1, t_2, t_3$, $\gamma(t_1)$, $\gamma(t_2)$, and $\gamma(t_3)$ do not form the vertices of an isoceles triangle. This is, in a sense, a non-linear generalization of sets avoiding arithmetic progressions, since if $\Gamma$ is a line, an isoceles triangle is given by arithmetic progressions. Assuming our curve is simple, we must avoid zeroes of the function
%
\[ |\gamma(t_1) - \gamma(t_2)|^2 = |\gamma(t_2) - \gamma(t_3)|^2 \]
%
If we take a sufficiently small segment of this curve, and we assume the curve has non-zero curvature on this curve, we can assume that $t_1 < t_2 < t_3$ in our dissection method.

If the coordinates of $\gamma$ are given by polynomials with maximum degree $d$, then the equation
%
\[ |\gamma(t_1) - \gamma(t_2)|^2 - |\gamma(t_2) - \gamma(t_3)|^2 \]
%
is a polynomial of degree $2d$, and so Math\'{e}'s result gives a set of dimension $1/2d$ avoiding isoceles triangles. In the case where $\Gamma$ is a line, then the function $f(t_1,t_2,t_3) = \gamma(t_1) + \gamma(t_3) - 2\gamma(t_2)$ avoids arithmetic progresions, and Math\'{e}'s result gives a dimension one set avoiding such progressions. Rob and Malabika's algorithm easily gives a set with dimension $1/2$ for any curve $\Gamma$. Our algorithm doesn't seem to be able to do much better here.
\end{example}

\begin{example}
    What is the largest dimension of a set in Euclidean space such that for any value $\lambda$, there is at most one pair of points $x,y$ in the set such that $|x - y| = \lambda$.
\end{example}

\begin{example}
    What is the largest dimension of a set which avoids certain angles, i.e. for which a triplet $x,y,z$ avoids certain planar configurations.
\end{example}

\begin{example}
    A set of points $x_0, \dots, x_d \in \mathbf{R}^d$ lie in a hyperplane if and only if the determinant formed by the vectors $x_n - x_0$, for $n \in \{ 1, \dots, d \}$, is zero. This is a degree $d$ polynomial, hence Math\'{e}'s result gives a dimension one set with no set of $d+1$ points lying in a hyperplane. On the other hand, a theorem of Mattila shows that every analytic set $E$ with dimension exceeding one contains $d + 1$ points in a hyperplane. Can we generalize this to a more general example avoiding points on a rotational, translation invariant family of manifolds using our results?
\end{example}

\begin{example}
    Given a set $F$ not containing the origin, what is the largest Hausdorff dimension of a set $E$ such that for any for any distinct rational $a_1, \dots, a_N$, the sum $a_1E + \dots + a_NE$ does not contain any elements of $F$. Thus the vector space over the rationals generated by $E$ does not contain any elements of $F$. We can also take the non-linear values $f(a_1E + \dots + a_NE)$ avoiding elements of $F$. $F$ must have non-empty interior for the problem to be interesting. Then can we find a smooth function $f$ with non-nanishing derivative which vanishes over $F$, or a family of smooth functions with non-vanishing derivative around $F$.
\end{example}

\section{Idea: Generalizing This Problem to low rank smooth functions}

Suppose we are able to find dimension $1/k$ sets avoiding configurations $y = g(f(x))$, where $f$ is a smooth function from $\mathbf{R}^n \to \mathbf{R}^m$ of rank $k$. Then given any function $g(f(x))$, where $f$ has rank $k$, if $g(f(x)) = 0$, then the implicit function theorem guarantees that there is a cover $U_\alpha$ and functions $h_\alpha: U^k \to \mathbf{R}^{n-k}$ such that for each if $g(f(x)) = 0$, for $x \in U_\alpha$, then there is a subset of $k$ indices $I$ such that $x_{I^c} = h_\alpha(x_I)$.

Then given any function $f(x)$ with rank $k$, we can use the implicit function theorem to find sets $U_\alpha$, indices $n_\alpha$, and functions $g_\alpha$ such that if $f(x) = 0$, for $x \in U_\alpha$, then $x_{n_\alpha} = g_\alpha(x_1, \dots, \widehat{x_{n_\alpha}}, \dots, x_n)$. Thus we need only avoid this type of configuration to avoid configurations of a general low rank function. If we don't believe that we are able to get dimension $1/(k-1)$ sets for rank $k$ configurations, then we shouldn't be able to find sets of Hausdorff dimension $1/k$ avoiding configurations of the form $y = f(x)$, where $f$ has rank $k$. 

\begin{remark}
    If the functions $g_\alpha$ are only partially defined, this makes the problem easier than if the functions were globally defined, because the constraint condition is now smaller than the original constraint.
\end{remark}

\begin{remark}
    This would solve our problem of avoiding $y - x = (u - v)^2$, since if $f(x,y,u,v) = y - x - (u - v)^2$, then
    %
    \[ \nabla f \]
\end{remark}

\section{Idea: Algebraic Number Fields}

If $\mathbf{Q}(\omega)$ is a quadratic extension of the rational numbers, then the ring of integers in this field form a lattice. Perhaps we can use this to generalize our approach to avoiding configurations $y = f(Tx)$, where all coefficients of the matrix $T$ lie in some common quadratic extension of the rational numbers.

\section{A Scheme for Avoiding Configurations}

Math\'{e}'s result can be reconfigured in terms of a building block strategy for implementation in our algorithm.

\begin{theorem}
    Let $f$ be a polynomial of degree $m$, and consider unions of length $1/M$ intervals $T_0, \dots, T_d \subset [0,1]$, with rational start-points. If $\partial_0 f$ is non-vanishing on $T_0 \times \dots \times T_d$, then there exists arbitrarily large integers $N$ and a constant $C$ not depending on $N$ and sets $S_n \subset T_n$ such that
    %
    \begin{itemize}
        \item $f(x) \neq 0$ for $x \in S_0 \times \dots \times S_d$.
        \item If $T_0, \dots, T_d$ are split into length $1/N$ intervals, then $S_n$ contains a length $C/N^d$ region of each interval.
    \end{itemize}
\end{theorem}
\begin{proof}
    Without loss of generality (by subdividing the initial intervals), let $M$ be the greatest common divisor of all of the startpoints of the intervals in $T_n$. Divide each interval $T_n$ into length $1/N$ intervals, and let $\mathbf{A} \subset (\mathbf{Z}/N)^d$ be the cartesian product of all startpoints of these length $1/N$ intervals. Since $f$ has degree $m$, $f(\mathbf{A}) \subset \mathbf{Z}/N^m$. If $A_0 \leq |\partial_0 f| \leq A_1$ on $T_0 \times \dots \times T_d$, then for any $a \in \mathbf{A}$, and $\delta_0$, there exists $\delta_1$ between $0$ and $\delta_0$ for which
    %
    \[ |f(a + \delta_0 e_0)| - f(a)| = \delta_0 |(\partial_0 f)(a + \delta_1)| \]
    %
    If $K$ is fixed such that $A_1 \leq (K-1)A_0$, so that we can choose
    %
    \[ \frac{1/K}{A_0N^m} \leq \delta_0 \leq \frac{\left( 1 - 1/K \right)}{A_1N^m} \]
    %
    Then
    %
    \[ \frac{1/K}{N^m} \leq |f(a + \delta_0 e_0) - f(a)| \leq \frac{1 - 1/K}{N^m} \]
    %
    Thus $d(f(\mathbf{A} + \delta_0 e_0), \mathbf{Z}/N^m) \geq 1/KN^m$. Thus if we thicken the coordinates of $\mathbf{A} + \delta_0$ to intervals of length $O(1/N^m)$, then we obtain sets $S_0, \dots, S_n$ avoiding solutions.
\end{proof}

\section{Square Free Sets}

We now look at avoiding solutions to the equation $x - y = (u - v)^2$. We consider two sets $I$ and $J$. Suppose that we can select a subset $\mathbf{S}$ from $\mathbf{Z} \cap N^2I$ such that if $x,y \in \mathbf{S}$ are distinct, $x - y$ is not a perfect square, and $|\mathbf{S}| \gtrsim |\mathbf{Z} \cap N^2I|^\alpha$. Then for any distinct $u,v \in \mathbf{Z} \cap NJ$, $(u - v)^2 \not \in \mathbf{S} - \mathbf{S}$. But this means that if we thicken the points in $\mathbf{S}/N^2$ to length $O(1/N^2)$ intervals, and the points in $\mathbf{Z}/N \cap J$ into length $O(1/N)$ intervals, then the resultant set will avoid solutions to $x - y = (u - v)^2$. This should give a dimension $\alpha$ set.






\chapter{Appendix: Calculating Dimensions of Cantor Like Sets}

This section considers an inconvenience in the calculation of the Hausdorff dimension of Cantor like sets, and an attempt to resolve it in the most likely of scenarios. That is, we consider the Hausdorff dimension of sets $X$ constructed as the limit of a nested family of sets $X_0 \supset X_1 \supset \dots$, where $X_0 = [0,1]$, and each $X_{n+1}$ is a union of intervals of some length $l_{n+1}$ obtained by dividing and removing some segments of the intervals $X_n$. To prove such sets are $\alpha$ dimensional, we rely on Frostman's lemma, which says that $H^\alpha(X) > 0$ if and only if there is a Borel probability measure $\mu$ supported on $X$ with $\mu(I) \lesssim |I|^\alpha$ for each interval $I$. Since the construction is obtained as a limit of intervals, it is often to construct $\mu$ by the {\it mass distribution principle}. That is, we let $\mu$ denote the weak limit of the probability masses $\mu_n$, where $\mu_0$ is a uniform distribution over $\mu_0$, and $\mu_{n+1}$ is obtained from $\mu_n$ by distributing the mass $\mu_n(I)$ of each length $l_n$ interval $I$ contained in $X_n$ over the portion of $I$ that remains in $X_{n+1}$. The cumulative distribution functions of the $\mu_n$ uniformly converge, hence the $\mu_n$ converge weakly to some $\mu$, which satisfy $\mu(I) = \mu_n(I)$ for each interval $I$ as above. Because of this discreteness, it is most easy to establish a bound $\mu(I) \lesssim l_n^\alpha$ when $I \subset X_n$ is a length $l_n$ interval. Since any interval $I$ of length $l_n$ is contained within at least two such intervals (or is contained in other length $l_n$ intervals that $\mu$ assigns no mass to), we have the general bound $\mu(I) \lesssim l_n^\alpha$ for all intervals $I$ of length $l_n$. Hausdorff dimension is a local property of a set\footnote{If we define $\dim_{\mathbf{H}}(x) = \lim_{r \downarrow 0} \dim_{\mathbf{H}}(B_r(x) \cap X)$ then $\dim_{\mathbf{H}}(X) = \sup_{x \in X} \dim_{\mathbf{H}}(x)$.}, so it is natural to expect that we can obtain a general bound $\mu(I) \lesssim_\alpha|I|^\alpha$ given that one has established precisely the same estimate, but restricted to intervals $I$ with $|I| = l_N$. This section concerns itself with ways that we can establish this general bound, and thus prove that $\dim_{\mathbf{H}}(X) \geq \alpha$.

\begin{remark}
    If this particular probability measure fails to satisfy the required bounds, does it follow that the set itself doesn't have the required Hausdorff dimension? Is there a counterexample to this statement? It seems this probability measure is strongly related to the structure of the Cantor set so it seems that if this measure fails it is doubtful any other measure will fail, at least locally.
\end{remark}

The main way of extending the results at particular scales to general results is by means of a covering argument. This is done in two ways for a given interval $I$ with $l_{N+1} \leq |I| \leq l_N$.
%
\begin{itemize}
    \item We can cover $I$ by a single length $l_N$ interval, so
    %
    \[ \mu(I) \lesssim l_N^\alpha = \left( \frac{l_N}{|I|} \right)^\alpha |I|^\alpha \leq \left( \frac{l_N}{l_{N+1}} \right)^\alpha |I|^\alpha \]
    \item We can cover $I$ by $|I|/l_{N+1}$ intervals of length $l_{N+1}$, so
    %
    \[ \mu(I) \lesssim \frac{|I|}{l_{N+1}} l_{N+1}^\alpha = \left( \frac{|I|}{l_{N+1}} \right)^{1 - \alpha} |I|^\alpha \leq \left( \frac{l_N}{l_{N+1}} \right)^{1 - \alpha} |I|^\alpha \]
\end{itemize}
%
The first bound is better for $\alpha \leq 1/2$, the second for $\alpha \geq 1/2$, but only up to the exponent in the polynomial. By considering both covering techniques simultaneously, and then choosing the best such bound for any particular interval length $|I|$, we can actually obtain the tighter bound
%
\[ \mu(I) \lesssim \left( \frac{l_N}{l_{N+1}} \right)^{(1-\alpha)\alpha} |I|^\alpha \leq \left( \frac{l_N}{l_{N+1}} \right)^{1/4} |I|^\alpha \]
%
Ultimately, however, we shall find that in general, without additional information about the measure $\mu$, any of the three techniques suffice, because the polynomial power doesn't really help make a required bound more or less difficult to prove. For instance, here we find that using any of the three bounds we have obtained that our argument is only completed if $l_N/l_{N+1}$ is upper bounded independently of $N$. In most cases this is very difficult to obtain, since it means that we can only restrict $l_N$ to have exponential decay.

We can obtain slightly better results by giving ourselves an epsilon of room. Rewording the two bounds, we have
%
\[ \mu(I) \leq \left( \frac{l_N}{|I|^{1 - \varepsilon/\alpha}} \right)^\alpha |I|^{\alpha - \varepsilon}\ \ \ \ \ \mu(I) \leq \left( \frac{|I|^{1 + \varepsilon/(1 - \alpha)}}{l_{N+1}} \right)^{1 - \alpha} |I|^{\alpha - \varepsilon} \]
%
Taking the minimum of these two bounds and optimizing gives
%
\[ \mu(I) \leq \left( \frac{l_N}{l_{N+1}} \right)^{\alpha(1-\alpha)} \left( l_N^{\alpha}l_{N+1}^{1-\alpha} \right)^\varepsilon |I|^{\alpha - \varepsilon} \]
%
Thus we can easily obtain that $\dim_{\mathbf{H}}(X) \geq \alpha - \varepsilon$ for each $\varepsilon$, if $l_N/l_{N+1} \lesssim_\varepsilon l_{N+1}^\varepsilon$. This can be expected in a great many scenarios, since the lengths $l_N/l_{N+1}$ are obtained at a single scale, whereas the lengths $l_{N+1}$ are obtained from compounding lengths over a great many scales. If we write $l_N = e^{-r_N}$, then $l_N/l_{N+1} = e^{r_{N+1} - r_N}$, so $l_{N+1}^{-\varepsilon} = e^{\varepsilon r_{N+1}}$, so it suffices to show $r_{N+1} - r_N - \varepsilon r_{N+1} \to -\infty$ for each $\varepsilon$. If $r_N$ is differentiable and accelerating, this means precisely that $r_N' - \varepsilon r_N \to -\infty$, so in particular, if $\smash{l_N = e^{-x^M}}$ for any $M$ we can apply this principle.

\begin{example}
    Kaleti constructs a full dimensional fractal set avoiding translates of itself using the Cantor set method. He has $l_N = 1/8^{N-1}N!$, in which case $l_N/l_{N+1} = 8N$, which is eventually bounded by $l_N^{-\varepsilon} = (8^{N-1}N!)^\varepsilon$ for any $\varepsilon > 0$.
\end{example}

It is, of course, easy to construct examples of $l_N$ for which we cannot use this technique. For instance, if $\smash{l_N = e^{-N!}}$, or $\smash{l_N = e^{-e^N}}$, the latter of which can be seen because $e^N$ is it's own derivative. Our second method for interpolating requires extra knowledge of the dissection process, but enables us to choose the $l_N$ decaying much more rapidly, when the filtration of the mass of $\mu$ only occurs at very sparse lengths. In particular, in Fraser and Pramanik's paper, the mass is essentially uniformly distributed over the length $1/N$ dissections, and then filtered to very particular length $1/N^k$ sections. Since the length $1/N$ and length $1/N^k$ differ in length only be a polynomial scale, the actual choice of $N$ relative to $M$ isn't really important, so we can let $N$ become as large as we desire relative to the choice of $M$. In particular, we can choose the $M_N$ to grow as fast as we want, which Fraser and Pramanik take full advantage of, choosing $M_N$ even bigger than $2 \uparrow \uparrow N$, which enables us to overwhelm the arbitrary constants in the construction of the configuration avoiding sets they create. We consider, for each $\beta = \{ l_1 \geq r_1 \geq l_2 \geq r_2 \geq \dots \to 0 \}$, a Cantor set $X^\beta$, and a corresponding probability measure $\mu_\beta$ supported on $X^\beta$. Our goal is to find a particular index set $\beta$ such that $X^\beta$ has Hausdorff dimension exceeding $\alpha$. The reason for introducing this notation is so that we can choose bounds on the measures $\mu_\beta$ as functions of the $l_1$ and $r_1$, which often occurs in recursive constructions as a consequence of the influence of the parameters on the construction. In particular, we will let $\lesssim_N$ stand for an inequality with a hidden constant depending on the parameters $\{ l_1, r_1, l_2, r_2, \dots, l_N \}$.

\begin{theorem}
    If $\mu_\beta(I) \lesssim_{N-1} l_N^\alpha$ for $|I| = l_N$, $r_N \lesssim_N l_{N+1}^\alpha$, and we have $\mu_\beta(J) \lesssim_{N-1} (r_N/l_N) \mu_\beta(I)$ for $J \subset I$ with $|J| = r_N$ and $|I| = l_N$, where $I$ and $J$ range over a disjoint family of intervals covering the support of $\mu_\beta$, then there exists a sequence $\beta$ such that $X^\beta$ is $\alpha$ dimensional.
\end{theorem}
\begin{proof}
    The main technique in this theorem is that since we have bounds enabling us to skip the analysis of large portions of intervals, we can take $r_N$ arbitrarily large relative to $l_N$, rendering any previous constants that occur in the algorithm to become immediately negligible. The first advantage of this is that for $r_N \leq |I| \leq l_N$, we can cover $I$ by length $r_N$ intervals to obtain that if the $l_N$ decay rapidly enough with respect to $A_{N-1}$,
%
\[ \mu(I) \lesssim_{N-1} (|I|/r_N) (r_N/l_N) l_N^\alpha = \frac{|I|}{l_N^{1 - \alpha}} \leq l_N^\varepsilon |I|^{\alpha - \varepsilon} \]
%
Note that since the hidden constants only depend on parameters up to index $N-1$, we can let $l_N$ decrease rapidly enough that we obtain $\mu(I) \lesssim_\varepsilon |I|^{\alpha - \varepsilon}$ independently of $N$. Given the existence of an inequality $\mu_\beta(I) \lesssim_{N-1} l_N$ for $|I| = l_N$, we automatically get the bound $\mu_\beta(I) \lesssim_N r_N$ for $|I| = r_N$. If $r_N \lesssim_N l_{N+1}^\alpha$, then for any $l_{N+1} \leq |I| \leq r_N$, we can cover $I$ by a single length $r_N$ interval to conclude that
%
\[ \mu_\beta(I) \lesssim_N r_N \lesssim_N l_{N+1}^{(1 - \varepsilon) \alpha} \]
%
Since $r_N$ is chosen after the parameters determine the hidden constants in this bound are chose, we can let $r_N$ decay rapidly enough that we obtain a bound $\mu_\beta(I) \lesssim_\varepsilon l_{N+1}^{\alpha - \varepsilon}$ independently of $N$. This guarantees the existence of a sequence $\beta$ with $X^\beta$ an $\alpha$ dimensional set.
\end{proof}

\begin{remark}
    The condition $\mu_\beta(J) \lesssim_{N-1} (r_N/l_N) \mu_\beta(I)$ essentially means that the probability mass on a length $l_N$ interval $I$ is uniformly distributed over the length $r_N$ intervals it contains. This is what enables us to remove the discussion of the growth of the sequence $\beta$ over time from discussion.
\end{remark}

\begin{example}
    In Fraser and Pramanik's paper, for $\alpha = 1/(d-1)$, they choose $r_N = A_N l_{N+1}^\alpha$ for some constant $A_N$, and choose $l_N$ suitably rapidly decaying to establish the inequalities $\mu_\beta(I) \lesssim |I|^\alpha$ for $|I| = l_N$. The decomposition algorithm naturally leads to the inequality $\mu_\beta(J) \lesssim (r_N/l_N) \mu_\beta(I)$, which shows that we can choose $l_N$ giving an $\alpha$ dimensional set.
\end{example}

\begin{thebibliography}{9}

\bibitem{RuzsaSetsWithoutSquares}
I. Z. Ruzsa
\textit{Difference Sets Without Squares}

\bibitem{KeletiDimOneSet}
Tam\'{a}s Keleti
\textit{A 1-Dimensional Subset of the Reals that Intersects Each of its Translates in at Most a Single Point}

\bibitem{MalabikaRob}
Robert Fraser, Malabika Pramanik
\textit{Large Sets Avoiding Patterns}

\bibitem{DickmanK}
Karl Dickman
\textit{On the Frequency of Numbers Containing Prime Factors of a Certain Relative Magnitude}

\end{thebibliography}

\end{document}