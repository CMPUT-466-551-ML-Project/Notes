\documentclass[12pt, dvipsnames]{report}

\usepackage{amsmath}
\usepackage{algorithm}
%\usepackage{algorithmic}
\usepackage[noend]{algpseudocode}

\usepackage{amsmath}
\usepackage{amssymb}
\usepackage{amsthm}
\usepackage{amsopn}

\usepackage{kpfonts}

\usepackage{graphicx}

% Probably don't need this on notes anymore
%\usepackage{kbordermatrix}

% Standard tool for drawing diagrams.
\usepackage{tikz}
\usepackage{tkz-berge}
\usepackage{tikz-cd}
\usepackage{tkz-graph}

\usepackage{comment}

%
\usepackage{multicol}

%
\usepackage{framed}

%
\usepackage{mathtools}

%
\usepackage{float}

%
\usepackage{subfig}

%
\usepackage{wrapfig}

%
\let\savewideparen\wideparen
\let\wideparen\relax
\usepackage{mathabx}
\let\wideparen\savewideparen

% Used for generating `enlightening quotes'
\usepackage{epigraph}

% Forget what this is used for :P
\usepackage[utf8]{inputenc}

% Used for generating quotes.
\usepackage{csquotes}

% Allows what to generate links inside
% generated pdf files
\usepackage{hyperref}

% Allows one to customize theorem
% environments in mathematical proofs.
\usepackage{thmtools}

% Gives access to a proof
\usepackage{lplfitch}

% I forget what this is for.
\usepackage{accents}

% A package for drawing simple trees,
% as a substitute for unnesacary TIKZ code
\usepackage{qtree}

% Enables sequent calculus proofs
\usepackage{ebproof}

% For braket notation
\usepackage{braket}

% To change line spacing when using mathematical notations which require some height!
\usepackage{setspace}

%\usepackage[dvipsnames]{xcolor}

\usepackage{float}

% For block commenting
\usepackage{comment}




\setlength\epigraphwidth{8cm}

\usetikzlibrary{arrows, petri, topaths, decorations.markings}

% So you can do calculations in coordinate specifications
\usetikzlibrary{calc}
\usetikzlibrary{angles}

\theoremstyle{plain}
\newtheorem{theorem}{Theorem}[chapter]
\newtheorem{axiom}{Axiom}
\newtheorem{lemma}[theorem]{Lemma}
\newtheorem{corollary}[theorem]{Corollary}
\newtheorem{prop}[theorem]{Proposition}
\newtheorem{exercise}{Exercise}[chapter]
\newtheorem{fact}{Fact}[chapter]

\newtheorem*{example}{Example}
\newtheorem*{proof*}{Proof}

\theoremstyle{remark}
\newtheorem*{exposition}{Exposition}
\newtheorem*{remark}{Remark}
\newtheorem*{remarks}{Remarks}

\theoremstyle{definition}
\newtheorem*{defi}{Definition}

\usepackage{hyperref}
\hypersetup{
    colorlinks = true,
    linkcolor = black,
}

\usepackage{textgreek}

\makeatletter
\renewcommand*\env@matrix[1][*\c@MaxMatrixCols c]{%
  \hskip -\arraycolsep
  \let\@ifnextchar\new@ifnextchar
  \array{#1}}
\makeatother

\renewcommand*\contentsname{\hfill Table Of Contents \hfill}

\newcommand{\optionalsection}[1]{\section[* #1]{(Important) #1}}
\newcommand{\deriv}[3]{\left. \frac{\partial #1}{\partial #2} \right|_{#3}} % partial derivative involving numerator and denominator.
\newcommand{\lcm}{\operatorname{lcm}}
\newcommand{\im}{\operatorname{im}}
\newcommand{\bint}{\mathbf{Z}}
\newcommand{\gen}[1]{\langle #1 \rangle}

\newcommand{\End}{\operatorname{End}}
\newcommand{\Mor}{\operatorname{Mor}}
\newcommand{\Id}{\operatorname{id}}
\newcommand{\visspace}{\text{\textvisiblespace}}
\newcommand{\Gal}{\text{Gal}}

\newcommand{\xor}{\oplus}
\newcommand{\ft}{\wedge}
\newcommand{\ift}{\vee}

\newcommand{\prob}{\mathbf{P}}
\newcommand{\expect}{\mathbf{E}}
\DeclareMathOperator{\Var}{\mathbf{V}}
\newcommand{\Ber}{\text{Ber}}
\newcommand{\Bin}{\text{Bin}}

%\newcommand{\widecheck}[1]{{#1}^{\ft}}

\DeclareMathOperator{\diam}{\text{diam}}

\DeclareMathOperator{\QQ}{\mathbf{Q}}
\DeclareMathOperator{\ZZ}{\mathbf{Z}}
\DeclareMathOperator{\RR}{\mathbf{R}}
\DeclareMathOperator{\HH}{\mathbf{H}}
\DeclareMathOperator{\CC}{\mathbf{C}}
\DeclareMathOperator{\AB}{\mathbf{A}}
\DeclareMathOperator{\PP}{\mathbf{P}}
\DeclareMathOperator{\MM}{\mathbf{M}}
\DeclareMathOperator{\VV}{\mathbf{V}}
\DeclareMathOperator{\TT}{\mathbf{T}}
\DeclareMathOperator{\LL}{\mathcal{L}}
\DeclareMathOperator{\EE}{\mathbf{E}}
\DeclareMathOperator{\NN}{\mathbf{N}}
\DeclareMathOperator{\DQ}{\mathcal{Q}}
\DeclareMathOperator{\IA}{\mathfrak{a}}
\DeclareMathOperator{\IB}{\mathfrak{b}}
\DeclareMathOperator{\IC}{\mathfrak{c}}
\DeclareMathOperator{\IP}{\mathfrak{p}}
\DeclareMathOperator{\IQ}{\mathfrak{q}}
\DeclareMathOperator{\IM}{\mathfrak{m}}
\DeclareMathOperator{\IN}{\mathfrak{n}}
\DeclareMathOperator{\IK}{\mathfrak{k}}
\DeclareMathOperator{\ord}{\text{ord}}
\DeclareMathOperator{\Ker}{\textsf{Ker}}
\DeclareMathOperator{\Coker}{\textsf{Coker}}
\DeclareMathOperator{\emphcoker}{\emph{coker}}
\DeclareMathOperator{\pp}{\partial}
\DeclareMathOperator{\tr}{\text{tr}}

\DeclareMathOperator{\supp}{\text{supp}}

\DeclareMathOperator{\codim}{\text{codim}}

\DeclareMathOperator{\minkdim}{\dim_{\mathbf{M}}}
\DeclareMathOperator{\hausdim}{\dim_{\mathbf{H}}}
\DeclareMathOperator{\lowminkdim}{\underline{\dim}_{\mathbf{M}}}
\DeclareMathOperator{\upminkdim}{\overline{\dim}_{\mathbf{M}}}
\DeclareMathOperator{\lhdim}{\underline{\dim}_{\mathbf{M}}}
\DeclareMathOperator{\lmbdim}{\underline{\dim}_{\mathbf{MB}}}
\DeclareMathOperator{\packdim}{\text{dim}_{\mathbf{P}}}
\DeclareMathOperator{\fordim}{\dim_{\mathbf{F}}}

\DeclareMathOperator*{\argmax}{arg\,max}
\DeclareMathOperator*{\argmin}{arg\,min}

\DeclareMathOperator{\ssm}{\smallsetminus}

\title{Algebraic Geometry}
\author{Jacob Denson}

\begin{document}

\pagenumbering{gobble}
\maketitle
\tableofcontents
\pagenumbering{arabic}

\chapter{Affine Algebraic Sets}

In Euclidean geometry, we find that most of the interesting geometric objects can be described as sets of points which can be described as solutions to some multivariate polynomial. Almost all of the elegant mathematical formulas occur as relations between polynomials.
%
\begin{itemize}
    \item The unit circle in the plane can be described as the set of points satisfying the equation $X^2 + Y^2 = 1$.
    \item A parabola can be described as those points satisfying $Y = X^2$.
    \item A hyperbola the solution set to $X^2 - Y^2 = 1$.
\end{itemize}
%
These are the basic conic sections which occur in high school geometry. Even if something we analyze cannot be described by a polynomial relationship, we can always approximate by polynomials. The field of algebraic geometry is the study of solution sets to polynomial equations. It has surprising applications to modern physics, combinatorics, and even some parts of analysis.

From a Euclidean perspective, our study starts with an $n$ dimensional affine space $\mathbf{A}^n$ over a field $K$, which after an identification of coordinates can be identified with $K^n$. $\mathbf{A}^1$ is often referred to coloquially as the affine line, and $\mathbf{A}^2$ as the affine plane. Given a polynomial $f \in K[X_1, \dots, X_n]$, we can consider the zero set $V(f) = \{ p \in \mathbf{A}^n : f(p) = 0 \}$, which is the {\bf hypersurface} defined by $f$. For instance, the unit circle is the hypersurface $V(X^2 + Y^2 - 1)$ in $\mathbf{A}^2$, and the parabola is $V(X^2 - Y)$. More generally, given a family $S$ of polynomials, we can consider the set $V(S)$, which is the set of points which are the common zeroes of all polynomials in $S$, and we call these objects {\bf affine varieties}, and they are the main object of study in algebraic geometry. 

\begin{example}
    The unit circle is the variety $V(X^2 + Y^2 - 1)$.
\end{example}

\begin{example}
    The set $\{ (t,t^2,t^3): t \in K \}$ is the zero set of $V(Y^2 - X, Z^3 - X)$, because it is an affine variety in $\mathbf{A}^3$.
\end{example}

\begin{example}
    The set of points whose polar coordinates $(r,\theta)$ satisfy $r = \sin(\theta)$ form a variety. Because $r \sin(\theta) = Y$, and $r^2 = X^2 + Y^2$, we may multiply the equation by $r$ to obtain the equation $X^2 + Y^2 = Y$. We find that the equation can be rearranged to $X^2 + (Y-1/2)^2 = 1/4$, so the solution set is just the circle of radius $1/2$ centered at $(0,1/2)$.
\end{example}

The class of varieties is interesting from a Euclidean perspective because it is invariant under affine transformations. Every affine transformation $T$ on $\mathbf{A}^n$ has an affine inverse $U = T^{-1}$, and $T$ maps $V(\mathfrak{a})$ onto $V(U^* \mathfrak{a})$, where $U^*: K[X_1, \dots, X_n] \to K[X_1, \dots, X_n]$ is the algebra isomorphism $U^*f = f \circ U$. The map $U^*$ preserves the degree of the polynomial, because if $U(X_i) = \smash{\sum a_{ij} X_j} + b_i$, then $U^*$ maps a monomial $X_1^{m_1} \dots X_n^{m_n}$ to $(\sum a_{1j} X_j)^{m_1} \dots (\sum a_{nj} X_j)^{m_n}$, whose degree cannot be greater than $m_1 + \dots + m_n$, so $\deg(Uf) \leq \deg f$. However, the algebraic identity $(T \circ U)^* = U^* \circ T^*$ implies that $(U^{-1})^* \circ U^* = (U \circ U^{-1})^* = \text{id}^* = \text{id}$, hence
%
\[ \deg f = \deg (U^{-1})^*(U^* f) \leq \deg(U^* f) \leq \deg f \]
%
Hence $\deg(f) = \deg(U^* f)$. This can be applied in classical geometry to obtain interesting results. First, define an {\bf affine plane curve} to be a hypersurface in $\mathbf{A}^2$.

\begin{theorem}
    An affine plane curve specified by a polynomial of degree $n$ intersects a line in at most $n$ places.
\end{theorem}
\begin{proof}
    Consider first the easy case where the line is just the $X$ axis. In this case, the zeroes of a polynomial $f(X,Y) = \sum a_{ij} X^iY^j$ which lie on the $X$ axis are in one to one correspondence with the set of solutions to the univariate polynomial $f(X,0) = \sum a_{i0} X^i$, which can have at most $\deg f(X,0) \leq \deg f = n$ separate points. Now in general, we can apply an affine transformation $T$ to any line to map it to the $X$ axis, and the points of $V(f)$ which lie on the line are in one to one correspondence with the points of $V(U^*f)$ which lie on the $X$ axis, where $U$ is the inverse of $T$. The theorem is then proved because $U^* f$ has the same degree as $f$.
\end{proof}

This theorem can be used to proved that certain planar curves are {\it not} affine varieties. We shall find that affine varieties are a very rigid class of objects, and we can prove many deep theorems about this class of objects.

\begin{example}
    The set of points $(x,y)$ in the real affine plane satisfying $y = \sin(x)$ cannot form a planar curve, because the curve intersects the $X$ axis infinitely often. If the points did form an algebraic variety $V(\mathfrak{a})$, where $\mathfrak{a}$ contains some nonzero polynomial $f$, then $V(f) \supset V(\mathfrak{a})$ would intersect the $X$ axis infinitely often, which is clearly impossible. This argument works for verifying that general varieties in the plane cannot intersect lines infinitely often, except in the trivial case where the variety is $\mathbf{A}^2$.
\end{example}

\begin{example}
    The complex sphere is the set of points $(z,w)$ in the complex affine plane satisfying $|z|^2 + |w|^2 = 1$, because the intersection of the complex sphere with the $z$ axis form a circle, which has infinitely many points. This justifies that the set cannot form a planar curve, and the same technique as in the last example shows the sphere cannot be an affine variety in general.
\end{example}

\begin{example}
    The set of points $\{ (\cos t, \sin t, t): t \in \mathbf{A}^3 \}$ over real affine space is not a variety, because it contains infinitely many points of the form $(1,0,\pi n)$, which lie on the same line.
\end{example}

There are some elementary observations we can make of this construction, which open the floodworks to the ring theory of $K[X_1, \dots, X_n]$.
%
\begin{itemize}
    \item If $\mathfrak{a}$ is the smallest ideal containing $S$, then $V(S) = V(\mathfrak{a})$, so every affine variety can be described as the common zeroes of some ideal. This follows because for any set $X \subset \mathbf{A}^n$, the set $I(X)$ of polynomials which vanish over $X$ forms an ideal, and it is clear that in the case where $X = V(S)$, the ideal contains all elements of $S$, hence all elements of $\mathfrak{a}$.

    \item If we have a family $\{ \mathfrak{a}_\alpha \}$ of ideals, then $V(\bigcup \mathfrak{a}_\alpha) = \bigcap V(\mathfrak{a}_\alpha)$, so the intersection of an arbitrary family of varieties forms a variety.

    \item If $\mathfrak{a} \subset \mathfrak{b}$, then $V(\mathfrak{b}) \subset V(\mathfrak{a})$.

    \item For any two polynomials $f$ and $g$, $V(fg) = V(f) \cup V(g)$. More generally, if $\mathfrak{a}$ and $\mathfrak{b}$ are ideals, then $V(\mathfrak{a}\mathfrak{b}) = V(\mathfrak{a}) \cup V(\mathfrak{b})$, so finite unions of varieties are varieties.

    \item $V(0) = \mathbf{A}^n$, $V(1) = \emptyset$, and for any $a \in K^n$, $V(X_1-a_1,\dots,X_n - a_n)$ is just the singleton set $\{ a \}$. It follows from the last point that finite point sets are varieties.

    \item If $\mathfrak{a} \subset K[X_1, \dots, X_n]$ and $\mathfrak{b} \subset K[Y_1, \dots, Y_m]$, then these ideals generate ideals $\mathfrak{a}'$ and $\mathfrak{b}'$ in $K[X_1, \dots, X_n, Y_1, \dots, Y_m]$. The intersection $V(\mathfrak{a}') \cap V(\mathfrak{b}')$ corresponds to the set of points $(x,y)$ with $x \in V(\mathfrak{a})$ and $y \in V(\mathfrak{b})$, and we call this the {\bf product variety} $V(\mathfrak{a}) \times V(\mathfrak{b})$.
\end{itemize}
%
Because of these properties, we begin to see that the analysis of $K[X_1, \dots, X_n]$ and its ideals is key to the study of algebraic geometry.

\begin{example}
    The varieties of $\mathbf{A}^1$ are exactly the finite point sets (other than the trivial variety $V(0) = \mathbf{A}^1$. First, note that since $K[X]$ is a principal ideal domain, we may assume we are considering the varieties of the form $V(f)$ for some particular polynomial $f(X) = \sum a_i X^i$. We know that $f(a) = 0$ if and only if $X - a$ is one of the prime factors of $f$. Since $f$ decomposes into finitely many prime factors, it follows that $V(f)$ can consist of at most $\text{deg}(f)$ points, a finite quantity. Thus the theory of one dimensional algebraic geometry is essentially trivial. This example shows that the countable union of affine varieties need not be a variety, because the countable union of finite sets need not be finite.
\end{example}

\begin{example}
    If $K$ is a finite field, then all subsets of $\mathbf{A}^n$ are varieties, because all subsets of $\mathbf{A}^n$ are finite subsets.
\end{example}

\begin{prop}
    If $f$ is a non constant polynomial over an algebraically complete field, $\mathbf{A}^n - V(f)$ contains infinitely many points for $n \geq 1$, and $V(f)$ contains infinitely many points for $n \geq 2$.
\end{prop}
\begin{proof}
    First, recall that every algebraically complete field $K$ must have infinitely many points, because if $K$ only contains $a_1, \dots, a_n$, we are unable to factor $(X - a_1) \dots (X - a_n) + 1$ into linear factors. It follows that $\mathbf{A}^1 - V(f)$ is infinite for any polynomial $f \in K[X]$, because $V(f)$ is finite. Given any polynomial $f \in K[X_1, \dots, X_n]$, there is a line in $\mathbf{A}^n$ upon which $f$ is not identically zero (for otherwise $f$ is equal to zero everywhere), and reducing our argument to the one dimensional case, we see that infinitely many points on this line cannot be zeroes of $f$. Arguing similarily, given any $f \in K[X_1, \dots, X_n]$, there is a plane upon which $f \neq 0$, and so we must show that any nonconstant $f(X,Y) = \sum a_{ij} X^i Y^j$ in the plane has infinitely many zeroes. For any line $[-a:1]$ through the origin, the polynomial takes the form $f(X,aX) = \sum a_{ij} a^j X^{i+j}$, and we know this polynomial has a zero unless it is constant, and if this occurs, then for any $1 \leq m < \infty$,
    %
    \[ \sum a_{(m-k)k} a^k = 0 \]
    %
    Each of these are polynomials in $K[a]$, and at least one of these polynomials is nonzero, so we conclude there can only be finitely many values $a$ such that $[a:1]$ does not have an intersection with $V(f)$, and it follows that $V(f)$ is infinite because $K$ is infinite.
\end{proof}

The generation of an ideal $I(X)$ from a set $X$ is dual to the notion of generating a set $V(\mathfrak{a})$ from an ideal $\mathfrak{a}$. We make a few elementary observations about this operator.
%
\begin{itemize}
    \item It is clear that if $X \subset Y$, then $I(Y) \subset I(X)$.
    \item $I(\emptyset) = K[X_1, \dots, X_n]$, and $I(\mathbf{A}^n) = (0)$.
    \item $S \subset I(V(S))$ for any subset $S$ of polynomials, and $X \subset V(I(X))$.
    \item Combining the last two points, it follows that $V(I(V(S))) = V(S)$, because $V(S) \subset V(I(V(S)))$ follows from the second point of the last bullet, and $V(I(V(S))$ follows because $S \subset I(V(S))$, hence $V(S) \supset V(I(V(S))$. Similarily, we can argue that $I(V(I(X))) = I(X)$. Thus if $X$ is an algebraic set, then $V(I(X)) = X$, and if $\mathfrak{a}$ is an ideal equal to $I(X)$ for some set $X$, then $I(V(\mathfrak{a})) = \mathfrak{a}$.
    \item If $f^n \in I(X)$, then $f \in I(X)$ because $K$ is an integral domain, so $f^n(x) = 0$ if and only if $f(x) = 0$. This means exactly that $I(X)$ is a {\it radical ideal}.
\end{itemize}

\begin{prop}
    For any two algebraic sets $V$ and $W$, $I(V) = I(W)$ if and only if $V = W$.
\end{prop}
\begin{proof}
    This is certainly true in one dimension, because then the algebraic sets form finite point sets. In general, if $I(V) = I(W)$, then for any line $l$, we can consider the surjective ring homomorphism from $K[X_1, \dots,X_n]$ to $K[Y]$ obtained by taking the embedding $T$ of a line $l$ in $\mathbf{A}^n$, and then considering $T^*: K[X_1, \dots, X_n] \to K[Y]$ after fixing a coordinate system on $l$. Not $f \in K[X_1, \dots, X_n]$ satisfies $f(x) = 0$ for all $x \in V$ if and only if $f(x) = 0$ for all $x \in W$.
\end{proof}

\end{document}